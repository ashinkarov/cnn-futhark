\section{Related Work\label{sec:relatedwork}}

% - "Verified Tensor-Program Optimization Via High-level Scheduling Rewrites"
% - "Efficient Differentiable Programming in a Functional Array-Processing Language"
% - "Verified tensor-program optimization via high-level scheduling rewrites"
% - "Indexed Streams: A Formal Intermediate Representation for Fused Contraction Programs"
% - "You only linearize once"
% - Dependent ML
% - ATS
% - Jax
% - Remora

Automatic differentiation has been around for many decades~\cite{early-ad1, early-ad2},
so it is well-understood at a conceptual level.  However,
a number of questions related to bringing AD into the context of
programming languages remain open.  Recent successes in machine learning
have spurred further interest in AD which has led to several new developments.
For the context of this paper, we focus on recent work that contributes to 
the perspective of balancing correctness guarantees and performance.
Our selection here is by no means exhaustive, for
a broader scope we refer the reader to~\cite{autodiff-survey}.

There has been a number of programming-language-oriented approaches that explain
how to add AD to a programming language of choice. Examples of these include
Futhark~\cite{futhark/sc22ad}, Haskell~\cite{ad-haskell}, and
Jax~\cite{ad-jax}. Furthermore, a number of machine learning
frameworks that incorporate AD have been proposed in recent years: TensorFlow~\cite{ad-tf},
PyTorch~\cite{ad-pytorch}, MXNet~\cite{ad-mxnet} and many more.
While in particular the dedicated frameworks tend to find widespread 
acceptance by practitioners, both, correctness and performance leave
two open questions: (i) how is it possible to
ensure that the AD algorithm is implemented correctly? (ii) if the
language or the framework do not perform as expected, what are the
options to solve this?  Unfortunately, for many cases the answer to
both questions is unsatisfying.  Most of the languages/frameworks do not
come with formal correctness guarantees, so one has to trust the
implementers of these tools.  One can run tests as well to gain trust 
in the implementation but that is far from a 
formal guarantee.  If one relies
on the AD provided by a chosen language/framework, and the generated code does not
perform well, one has to modify the language/framework, as these solutions
are tightly integrated with the tools. The problem here is that most of of these tools
have very large and sophisticated implementations typically comprising
of hundreds of thousands of lines of code.

Another line of work studies high-level principles of AD using
category theory~\cite{ad-theor1, ad-theor2, ad-theor3}.
While this indeed comes with great correctness guarantees due to
some naturality principles, it is not always clear how to implement
this in a way that leads to efficiently executable specifications.  Also, the
entire treatment of index-safe tensors is unclear.

In~\cite{ad-elliott} the author proposes to view AD problem using
the language of cartesian categories.  It has been shown that
this approach can be used in practice by implementing the proposed
technique in Haskell.  Type classes are a vehicle to restrict expressions
that are differentiable.  There is a Haskell plugin that translates
expressions that are instances of the mentioned type classes into
categorical primitives, AD is performed on these and the code is reflected
back to Haskell.  This is a nice approach that makes it easy
to verify the correctness of the algorithm.  However, the treatment
of tensors and general extractability remains a little unclear.
While it is briefly mentioned that representable functors
are supported, it is unclear whether this is sufficient to
represent rank-polymorphic arrays with strict bound checks.
Also, correctness guarantees are inevitably restricted by the
Haskell type system, so we are likely to find invariants that
are inexpressible in that setup.

Verified high-performance computing is an area of active research, and
one line of research is based on the idea of separating the high level
\emph{specification} of a problem from the \emph{schedule} that
describe how it is to be executed. Verifying that such schedules are
semantics-preserving is the topic of several recent
publications~\cite{10.1145/3527328,10.1145/3498717}. Our approach
differs by not using a scheduling language, and instead focuses on
verifying the specification itself, and also by having a particularly
expressive specification language.
