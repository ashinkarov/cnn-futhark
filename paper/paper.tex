\documentclass[acmsmall,screen,anonymous,review]{acmart}
\usepackage{listings}
\usepackage{pgfplots}
\usepackage{pgfplotstable}
\usepackage{microtype}
\usepackage{mathtools} % needed for \doublecolon symbol
\usepackage{bbold} % For extended blackboard bold symbols
\usepackage{fancyvrb}

\usepackage{todonotes}

% Add definitions for abbreviations like e.g.; i.e; etc. with
% correct spacing depending on the parameters.  In this case
% `all' exposes all the definitions of the package, and `british'
% makes sure that there is no comma after e.g. or i.e.
\usepackage[all,british]{foreign}

\lstdefinelanguage{SaC}[]{}{%
    language=C,
    morekeywords={inline,return,for,shape,dim,with,fold,modarray,genarray},
    otherkeywords={?,:,->},
    comment=[l][\color{green!50!brown}]{//},
    morecomment=[s][\color{blue!70!gray}]{int[}{]},
    morecomment=[s][\color{blue!70!gray}]{double[}{]}
}
\lstset{%
    basicstyle=\fontsize{8}{8}\selectfont\ttfamily,
    language=SaC,
    keywordstyle=\color{blue!70!gray},
    commentstyle=\color{green!50!brown},
    stringstyle=\color{violet!40!magenta},
    %showstringspaces=false,
}

% https://tex.stackexchange.com/questions/318142/typeseeting-a-multiset-with-double-curly-braces
\newcommand*{\xlbrace}{[\mskip-5mu[}
\newcommand*{\xrbrace}{]\mskip-5mu]}

\newcommand*{\ldblbrace}{\{\mskip-5mu\{}
\newcommand*{\rdblbrace}{\}\mskip-5mu\}}


\usepackage{bm}
\usepackage{agda}
\usepackage{newunicodechar}

\usepackage{mathpartir}
\usepackage{varwidth}
% Do we need this?
\usepackage{microtype} 

\newcommand\codeblock[1]{%
  %{\fbox{\begin{varwidth}{0.9\textwidth}#1\end{varwidth}}}
  {\begin{varwidth}{0.9\textwidth}#1\end{varwidth}}
} 


\newunicodechar{₀}{\ensuremath{_0}}
\newunicodechar{₁}{\ensuremath{_1}}
\newunicodechar{₂}{\ensuremath{_2}}
\newunicodechar{₃}{\ensuremath{_3}}
\newunicodechar{₄}{\ensuremath{_4}}
\newunicodechar{₅}{\ensuremath{_5}}
\newunicodechar{₆}{\ensuremath{_6}}
\newunicodechar{₇}{\ensuremath{_7}}
\newunicodechar{₈}{\ensuremath{_8}}
\newunicodechar{₉}{\ensuremath{_9}}
\newunicodechar{ι}{\ensuremath{\iota}}
\newunicodechar{ₗ}{\ensuremath{_l}}
\newunicodechar{ₐ}{\ensuremath{_a}}
\newunicodechar{ₖ}{\ensuremath{_k}}
\newunicodechar{ᵢ}{\ensuremath{_i}}
\newunicodechar{ⱼ}{\ensuremath{_j}}
\newunicodechar{ᵥ}{\ensuremath{_v}}
\newunicodechar{ₕ}{\ensuremath{_h}}
\newunicodechar{ₚ}{\ensuremath{_p}}
\newunicodechar{ₛ}{\ensuremath{_s}}
\newunicodechar{ₒ}{\ensuremath{_o}}
\newunicodechar{ₙ}{\ensuremath{_n}}


\newunicodechar{ᵣ}{\ensuremath{_r}}
\newunicodechar{ʳ}{\ensuremath{^r}}
\newunicodechar{ˡ}{\ensuremath{^l}}
\newunicodechar{ᵛ}{\ensuremath{^v}}
\newunicodechar{ᵉ}{\ensuremath{^e}}
\newunicodechar{ℕ}{\ensuremath{\mathbb{N}}}
\newunicodechar{ℝ}{\ensuremath{\mathbb{R}}}
\newunicodechar{𝟘}{\ensuremath{\mathbb{0}}}

\newunicodechar{∀}{\ensuremath{\forall}}
\newunicodechar{∃}{\ensuremath{\exists}}
\newunicodechar{≡}{\ensuremath{\equiv}}
\newunicodechar{≈}{\ensuremath{\approx}}
\newunicodechar{≟}{\ensuremath{\stackrel{?}{=}}}
\newunicodechar{∎}{\ensuremath{\blacksquare}}
\newunicodechar{⊛}{\ensuremath{\circledast}}
\newunicodechar{⊗}{\ensuremath{\otimes}}
\newunicodechar{⊕}{\ensuremath{\oplus}}
\newunicodechar{⊝}{\ensuremath{\ominus}}
\newunicodechar{≤}{\ensuremath{\le}}
\newunicodechar{φ}{\ensuremath{\phi}}
\newunicodechar{ψ}{\ensuremath{\psi}}
\newunicodechar{ε}{\ensuremath{\epsilon}}
\newunicodechar{δ}{\ensuremath{\delta}}
\newunicodechar{λ}{\ensuremath{\lambda}}
\newunicodechar{σ}{\ensuremath{\sigma}}
\newunicodechar{ρ}{\ensuremath{\rho}}
\newunicodechar{ν}{\ensuremath{\rho}}
\newunicodechar{∂}{\ensuremath{\partial}}

\newunicodechar{Γ}{\ensuremath{\Gamma}}
\newunicodechar{Δ}{\ensuremath{\Delta}}
\newunicodechar{Ψ}{\ensuremath{\Phi}}
\newunicodechar{Ξ}{\ensuremath{\Xi}}

\newunicodechar{′}{{'}}
\newunicodechar{∷}{\ensuremath{\dblcolon}}
\newunicodechar{↔}{\ensuremath{\leftrightarrow}}
\newunicodechar{↦}{\ensuremath{\mapsto}}
\newunicodechar{∘}{\ensuremath{\circ}}
\newunicodechar{⁻}{\ensuremath{^{-}}}
\newunicodechar{∙}{\ensuremath{\boldsymbol{\cdot}}}
\newunicodechar{▹}{\ensuremath{\triangleright}}
\newunicodechar{⋯}{\ensuremath{\cdots}}
\newunicodechar{∇}{\ensuremath{\nabla}}
\newunicodechar{Σ}{\ensuremath{\Sigma}}
\newunicodechar{∈}{\ensuremath{\in}}
\newunicodechar{⟦}{\ensuremath{\llbracket}}
\newunicodechar{⟧}{\ensuremath{\rrbracket}}
\newunicodechar{⦃}{\ensuremath{\ldblbrace}}
\newunicodechar{⦄}{\ensuremath{\rdblbrace}}
\newunicodechar{⌊}{\ensuremath{\lfloor}}
\newunicodechar{⌋}{\ensuremath{\rfloor}}
  

\newunicodechar{⇒}{\ensuremath{\Rightarrow}}
\newunicodechar{⟪}{\ensuremath{\xlbrace}}
\newunicodechar{⟫}{\ensuremath{\xrbrace}}
\newunicodechar{⊠}{\ensuremath{\boxtimes}}
\newunicodechar{⊞}{\ensuremath{\boxplus}}
\newunicodechar{∋}{\ensuremath{\ni}}
\newunicodechar{∶}{\ensuremath{\bm{:}}}
\newunicodechar{↦}{\ensuremath{\mapsto}}
\newunicodechar{⊤}{\ensuremath{\top}}
\newunicodechar{⊆}{\ensuremath{\subseteq}}




% Some shortcut commands for agda symbols
\newcommand{\AD}[1]{\AgdaDatatype{#1}}
\newcommand{\AC}[1]{\AgdaInductiveConstructor{#1}}
\newcommand{\AF}[1]{\AgdaFunction{#1}}
\newcommand{\AB}[1]{\AgdaBound{#1}}
\newcommand{\AK}[1]{\AgdaKeyword{#1}}
\newcommand{\AR}[1]{\AgdaField{#1}}
\newcommand{\AM}[1]{\AgdaModule{#1}}
\newcommand{\AN}[1]{\AgdaNumber{#1}}
\newcommand{\AS}[1]{\AgdaString{#1}}

\renewcommand{\AgdaCommentFontStyle}[1]{\textrm{#1}}
\renewcommand{\AgdaFontStyle}[1]{\textrm{#1}}
\renewcommand{\AgdaKeywordFontStyle}[1]{\textrm{#1}}

\pgfplotstableset{col sep=comma}

\title{Correctness is Demanding, Performance is Frustrating}

\author{Artjoms {\v{S}}inkarovs}
\email{A.Sinkarovs@hw.ac.uk}
\orcid{0000-0003-3292-2985}
\affiliation{%
  \institution{Heriot-Watt University}
  \streetaddress{Heriot-Watt University, Edinburgh Campus}
  \city{Edinburgh}
  \country{Scotland}
  \postcode{EH14 4AS}
}
\author{Thomas Koopman}
\email{Thomas.Koopman@ru.nl}
\orcid{0009-0001-1031-7226}
\affiliation{%
  \institution{Radboud University}
  \streetaddress{Houtlaan 4}
  \city{Nijmegen}
  \country{Netherlands}
  \postcode{6525 XZ}
}
\author{Sven-Bodo Scholz}
\email{SvenBodo.Scholz@ru.nl}
\orcid{0000-0002-8663-1043}
\affiliation{%
  \institution{Radboud University}
  \streetaddress{Houtlaan 4}
  \city{Nijmegen}
  \country{Netherlands}
  \postcode{6525 XZ}
}

%\renewcommand{\shortauthors}%
%  {A.~{\v{S}}inkarovs, T.~Koopman, and S.~Scholz}


\pgfkeys{/pgf/number format/.cd,fixed,precision=0}


\keywords{Dependent Types, Agda, Array Programming, Automatic Differentiation, SaC}

% TODO: Insert CCSXML and CCSdesc later.
%\begin{CCSXML}
%\end{CCSXML}
%
% \ccsdesc[500]{Theory of computation~Data structures design and analysis}
% \ccsdesc[500]{Theory of computation~Algorithm design techniques}
% \ccsdesc[500]{Software and its engineering~Source code generation}
% \ccsdesc[300]{Software and its engineering~Functional languages}
% \ccsdesc[500]{Software and its engineering~Data types and structures}


\begin{document}

\begin{abstract}
In this paper we demonstrate a technique for developing high performance applications
with strong correctness guarantees.  We use a theorem prover to derive a high-level
specification of the application that includes correctness invariants of our choice.
After that, within the same theorem prover, we implement an extraction of the
specified application into a high-performance language of our choice.  Concretely,
we are using Agda to specify a framework for automatic differentiation (reverse mode)
that is focused on index-safe tensors.  This framework comes
with an optimiser for tensor expressions and the ability to translate these
expressions into SaC and C.  We specify a canonical convolutional neural network
within the proposed framework, compute the derivatives needed for the training
phase and then demonstrate that the generated code matches the performance of hand-written
code when running on a multi-core machine.
\end{abstract}

\maketitle

\section{Introduction\label{sec:intro}}

The year is 2025, and programmers still face a trade-off between correctness and
performance. Low-level languages like C or Fortran allow careful control over
hardware, but offer few guarantees about safety or program structure.
Dependently-typed languages like Agda or Lean, on the other hand, make it
possible to encode precise invariants, but typically lack the infrastructure for
high-performance computation.

Rather than chasing a mythical language that promises both complete correctness
and high performance, we explore a practical compromise: leveraging a
dependently-typed proof assistant for the parts of a scientific programming
pipeline where full verification is most crucial, and integrating it with a
high-performance backend for execution. Specifically, we investigate how Agda
can be used not only for specification and reasoning, but also for
transformation, optimisation, and code generation in a realistic machine
learning workload.

To explore this idea, we investigate the concrete problem of automatic
differentiation (AD) which is often found in machine learning applications. This
is a convenient case study as it comes with the following challenges. From the
correctness perspective, it is crucially important to track the shapes and ranks
of the tensors, guaranteeing the absence of out-of-bound indexing. This is a
very common source of errors that can be difficult to find. Secondly, we have to
compute derivatives of the given tensor expressions, preserve safe indexing
guarantees while we do so, and we have to be able to translate the computed
expressions into some high-performance language. As machine learning
applications are known to be numerically intensive problems, our performance
challenge lies in running the program as fast as we can on the chosen hardware
architecture.

In an ideal world, all parts would of course come with full formal
specifications and accompanying proof, but in practice some parts remain an
entirely separate and nontrivial research effort to verify for functional
correctness, such as AD. Other parts---such as code generation--can be handled
using known techniques, but are somewhat time consuming. Despite all this, we
show that even in those cases where resources preclude full verification, Agda
can still be used as a practical tool, with full verification employed where it
is deemed desirable and practical.

We follow~\cite{cnn-array} which demonstrates that it is possible to implement
one of the canonical convolutional neural network (CNN) in the array language
SaC~\cite{sac1, sac2}, obtaining good sequential and parallel performance that
is competitive with TensorFlow~\cite{ad-tf} and PyTorch~\cite{ad-pytorch}.
Focusing on correctness, we propose a theory of rank-polymorphic
arrays~\cite{rank-poly} in Agda~\cite{agda-2-6-3}. Within this framework, we
encode the CNN from~\cite{cnn-array} and lift it into an embedded DSL. We
implement AD (reverse mode) and domain-specific optimisations for expressions in
that DSL. Finally, we implement an extraction into Futhark (a functional array
language), apply the CNN to the MNIST digit recognition problem, and compare
performance with TensorFlow.

As a result, we demonstrate an approach where the entire specification,
optimiser, AD and code generation are available to us within a proof assistant
of our choice. We can prove facts about all the stages of the pipeline and
easily adjust them to our liking. We also demonstrate that some components can
be left only partially verified, where the effort of verification is judged
disproportionate to the benefit. We argue that such a liberating approach is
feasible in practice, at least for the times of dialectic of correctness and
performance. The overall goals is to demonstrate how to formulate a
computationally intensive problem in a proof assistant, transform it with
verification of key safety properties, and then pass it on to the other system
that can run the algorithm efficiently.

The contributions of this paper are as follows:
\begin{enumerate}
  \item a rank-polymorphic array theory and implementation of
        the CNN from~\cite{cnn-array} in Agda;
  \item an embedded DSL in Agda which supports AD (reverse mode);
  \item an extraction mechanism for generating Futhark code from the DSL; and
  \item an experimental evaluation of the generated code.
\end{enumerate}

This paper is written in literate Agda, which guarantees that all the code
snippets have been type-checked.



\section{Background\label{sec:background}}

Automatic differentiation has been around for many decades~\cite{early-ad1, early-ad2},
so it is well-understood at a conceptual level.  However,
a number of questions related to bringing AD into the context of
programming languages remain open.  Recent successes in machine learning
have spurred further interest in AD which has led to several new developments.
For the context of this paper, we focus on recent work that contributes to 
the perspective of balancing correctness guarantees and performance.
Our selection here is by no means exhaustive, for
a broader scope we refer the reader to~\cite{autodiff-survey}.

There has been a number of programming-language-oriented approaches that explain
how to add AD to a programming language of choice. Examples of these include
Futhark~\cite{futhark/sc22ad}, Haskell~\cite{ad-haskell}, and
Jax~\cite{ad-jax}. Furthermore, a number of machine learning
frameworks that incorporate AD have been proposed in recent years: TensorFlow~\cite{ad-tf},
PyTorch~\cite{ad-pytorch}, MXNet~\cite{ad-mxnet} and many more.
While in particular the dedicated frameworks tend to find widespread 
acceptance by practitioners, both, correctness and performance leave
two open questions: (i) how is it possible to
ensure that the AD algorithm is implemented correctly? (ii) if the
language or the framework do not perform as expected, what are the
options to solve this?  Unfortunately, for many cases the answer to
both questions is unsatisfying.  Most of the languages/frameworks do not
come with formal correctness guarantees, so one has to trust the
implementers of these tools.  One can run tests as well to gain trust 
in the implementation but that is far from a 
formal guarantee.  If one relies
on the AD provided by a chosen language/framework, and the generated code does not
perform well, one has to modify the language/framework, as these solutions
are tightly integrated with the tools. The problem here is that most of of these tools
have very large and sophisticated implementations typically comprising
of hundreds of thousands of lines of code.  Furthermore, these systems
often rely on sophisticated 
combinations of sub-systems that need to be fine-tuned to the executing hardware.

Another line of work studies high-level principles of AD using
category theory~\cite{ad-theor1, ad-theor2, ad-theor3}.
While this indeed comes with great correctness guarantees due to
some naturality principles, it is not always clear how to implement
this in a way that leads to efficiently executable specifications.  Also, the
entire treatment of index-safe tensors is unclear.

In~\cite{ad-elliott} the author proposes to view AD problem using
the language of cartesian categories.  It has been shown that
this approach can be used in practice by implementing the proposed
technique in Haskell.  Type classes are a vehicle to restrict expressions
that are differentiable.  There is a Haskell plugin that translates
expressions that are instances of the mentioned type classes into
categorical primitives, AD is performed on these and the code is reflected
back to Haskell.  This is a nice approach that makes it easy
to verify the correctness of the algorithm.  However, the treatment
of tensors and general extractability remains a little unclear.
While it is briefly mentioned that representable functors
are supported, it is unclear whether this is sufficient to
represent rank-polymorphic arrays with strict bound checks.
Also, correctness guarantees are inevitably restricted by the
Haskell type system, so we are likely to find invariants that
are inexpressible in that setup.

\section{Array Theory\label{sec:array-theory}}

\begin{code}[hide]%
\>[0]\AgdaKeyword{open}\AgdaSpace{}%
\AgdaKeyword{import}\AgdaSpace{}%
\AgdaModule{Relation.Binary.PropositionalEquality}\<%
\\
\>[0]\AgdaKeyword{open}\AgdaSpace{}%
\AgdaKeyword{import}\AgdaSpace{}%
\AgdaModule{Relation.Nullary}\<%
\\
\>[0]\AgdaKeyword{open}\AgdaSpace{}%
\AgdaKeyword{import}\AgdaSpace{}%
\AgdaModule{Data.List}\AgdaSpace{}%
\AgdaKeyword{using}\AgdaSpace{}%
\AgdaSymbol{(}\AgdaDatatype{List}\AgdaSymbol{;}\AgdaSpace{}%
\AgdaInductiveConstructor{[]}\AgdaSymbol{;}\AgdaSpace{}%
\AgdaOperator{\AgdaInductiveConstructor{\AgdaUnderscore{}∷\AgdaUnderscore{}}}\AgdaSymbol{)}\<%
\\
\>[0]\AgdaKeyword{open}\AgdaSpace{}%
\AgdaKeyword{import}\AgdaSpace{}%
\AgdaModule{Data.Empty}\<%
\\
\>[0]\AgdaKeyword{open}\AgdaSpace{}%
\AgdaKeyword{import}\AgdaSpace{}%
\AgdaModule{Function}\<%
\\
%
\\[\AgdaEmptyExtraSkip]%
\>[0]\AgdaKeyword{module}\AgdaSpace{}%
\AgdaModule{\AgdaUnderscore{}}\AgdaSpace{}%
\AgdaKeyword{where}\<%
\\
\>[0]\AgdaKeyword{module}\AgdaSpace{}%
\AgdaModule{Array}\AgdaSpace{}%
\AgdaKeyword{where}\<%
\\
\>[0][@{}l@{\AgdaIndent{0}}]%
\>[2]\AgdaKeyword{open}\AgdaSpace{}%
\AgdaKeyword{import}\AgdaSpace{}%
\AgdaModule{Data.Nat}\AgdaSpace{}%
\AgdaKeyword{using}\AgdaSpace{}%
\AgdaSymbol{(}\AgdaInductiveConstructor{zero}\AgdaSymbol{;}\AgdaSpace{}%
\AgdaInductiveConstructor{suc}\AgdaSymbol{;}\AgdaSpace{}%
\AgdaDatatype{ℕ}\AgdaSymbol{;}\AgdaSpace{}%
\AgdaOperator{\AgdaPrimitive{\AgdaUnderscore{}+\AgdaUnderscore{}}}\AgdaSymbol{;}\AgdaSpace{}%
\AgdaOperator{\AgdaPrimitive{\AgdaUnderscore{}*\AgdaUnderscore{}}}\AgdaSymbol{;}\AgdaSpace{}%
\AgdaOperator{\AgdaDatatype{\AgdaUnderscore{}≤\AgdaUnderscore{}}}\AgdaSymbol{;}\AgdaSpace{}%
\AgdaInductiveConstructor{s≤s}\AgdaSymbol{;}\AgdaSpace{}%
\AgdaInductiveConstructor{z≤n}\AgdaSymbol{;}\AgdaSpace{}%
\AgdaOperator{\AgdaFunction{\AgdaUnderscore{}<\AgdaUnderscore{}}}\AgdaSymbol{)}\<%
\\
%
\>[2]\AgdaKeyword{open}\AgdaSpace{}%
\AgdaKeyword{import}\AgdaSpace{}%
\AgdaModule{Data.Nat.Properties}\AgdaSpace{}%
\AgdaKeyword{using}\AgdaSpace{}%
\AgdaSymbol{(}\AgdaFunction{+-mono-≤}\AgdaSymbol{;}\AgdaSpace{}%
\AgdaFunction{≤-step}\AgdaSymbol{;}\AgdaSpace{}%
\AgdaFunction{≤-pred}\AgdaSymbol{;}\AgdaSpace{}%
\AgdaOperator{\AgdaFunction{\AgdaUnderscore{}≟\AgdaUnderscore{}}}\AgdaSymbol{;}\AgdaSpace{}%
\AgdaFunction{+-comm}\AgdaSymbol{;}\AgdaSpace{}%
\AgdaFunction{+-suc}\AgdaSymbol{)}\<%
\\
%
\>[2]\AgdaKeyword{open}\AgdaSpace{}%
\AgdaKeyword{import}\AgdaSpace{}%
\AgdaModule{Data.Fin}\AgdaSpace{}%
\AgdaSymbol{as}\AgdaSpace{}%
\AgdaModule{F}\AgdaSpace{}%
\AgdaKeyword{using}\AgdaSpace{}%
\AgdaSymbol{(}\AgdaInductiveConstructor{zero}\AgdaSymbol{;}\AgdaSpace{}%
\AgdaInductiveConstructor{suc}\AgdaSymbol{;}\AgdaSpace{}%
\AgdaDatatype{Fin}\AgdaSymbol{;}\AgdaSpace{}%
\AgdaFunction{combine}\AgdaSymbol{;}\AgdaSpace{}%
\AgdaFunction{remQuot}\AgdaSymbol{;}\AgdaSpace{}%
\AgdaFunction{fromℕ<}\AgdaSymbol{;}\AgdaSpace{}%
\AgdaFunction{inject+}\AgdaSymbol{;}\AgdaSpace{}%
\AgdaFunction{splitAt}\AgdaSymbol{)}\<%
\\
%
\>[2]\AgdaKeyword{open}\AgdaSpace{}%
\AgdaKeyword{import}\AgdaSpace{}%
\AgdaModule{Data.Fin.Properties}\AgdaSpace{}%
\AgdaKeyword{using}\AgdaSpace{}%
\AgdaSymbol{(}\AgdaFunction{suc-injective}\AgdaSymbol{;}\AgdaSpace{}%
\AgdaFunction{toℕ<n}\AgdaSymbol{;}\AgdaSpace{}%
\AgdaFunction{splitAt-inject+}\AgdaSymbol{)}\<%
\\
%
\>[2]\AgdaComment{--open\ import\ Fin2\ using\ (Fin;\ \#\AgdaUnderscore{};\ combine;\ remQuot;\ zerof;\ sucf;\ \AgdaUnderscore{}⊕\AgdaUnderscore{};\ \AgdaUnderscore{}⊝\AgdaUnderscore{})}\<%
\\
%
\>[2]\AgdaKeyword{open}\AgdaSpace{}%
\AgdaKeyword{import}\AgdaSpace{}%
\AgdaModule{Data.Sum}\AgdaSpace{}%
\AgdaKeyword{using}\AgdaSpace{}%
\AgdaSymbol{(}\AgdaOperator{\AgdaDatatype{\AgdaUnderscore{}⊎\AgdaUnderscore{}}}\AgdaSymbol{;}\AgdaSpace{}%
\AgdaInductiveConstructor{inj₁}\AgdaSymbol{;}\AgdaSpace{}%
\AgdaInductiveConstructor{inj₂}\AgdaSymbol{)}\<%
\\
%
\>[2]\AgdaKeyword{open}\AgdaSpace{}%
\AgdaKeyword{import}\AgdaSpace{}%
\AgdaModule{Data.Product}\AgdaSpace{}%
\AgdaSymbol{as}\AgdaSpace{}%
\AgdaModule{Prod}\AgdaSpace{}%
\AgdaKeyword{using}\AgdaSpace{}%
\AgdaSymbol{(}\AgdaFunction{∃}\AgdaSymbol{;}\AgdaSpace{}%
\AgdaOperator{\AgdaInductiveConstructor{\AgdaUnderscore{},\AgdaUnderscore{}}}\AgdaSymbol{;}\AgdaSpace{}%
\AgdaOperator{\AgdaFunction{\AgdaUnderscore{}×\AgdaUnderscore{}}}\AgdaSymbol{;}\AgdaSpace{}%
\AgdaFunction{uncurry}\AgdaSymbol{)}\<%
\end{code}

The central data structure of our case study is a multi-dimensional array (ML
uses the term \emph{tensor}).  This section presents a minimalist array theory in Agda
which is well-suited for specifying numerical applications such as CNNs.

The work in the rest of the paper is presented in Agda, with which we assume some
familiarity.
For gentle introductions to Agda we refer to one of the tutorials that are freely available
online.\footnote{See \url{https://agda.readthedocs.io/en/v2.7.0.1/getting-started/tutorial-list.html}.}

The conciseness of the CNN specification
in~\cite{cnn-array} relies on rank-polymorphism, which is the ability to operate
on arrays of arbitrary ranks.  Our array theory is rank polymorphic
which distinguishes it from most existing approaches.
The central consideration when working with dependent types is how to represent data.
Some encodings are better suited for reasoning, others are more efficient
at runtime.  Due to our two-language setup, our choice of representation is
driven by proof considerations only.
This is why we represent arrays as functions from indices to values.

Absence of out-of-bound errors means that all array indices fall within
the shapes of the arrays that they are selecting from.
The shape of array describes the extent of each of its axes.  We represent
shapes as lists of natural numbers using the data type \AD{S}.
The \AC{[]} shape describes an array of rank zero that contains exactly one
element (arrays of such shape are often called \emph{scalars} and we use this
terminology in the rest of the paper).
The cons operation \AC{\_∷\_} prepends a new axis to the left of the shape.
Note on the notation: underscores in \AC{\_∷\_} specify positions where
arguments go, turning \AC{∷} into an infix binary operation.

Array positions (indices) are given by the dependent type \AD{P} which
is indexed by shapes \AD{S}.  A position within an array of shape \AB{s}
is a list of natural numbers of the same length as $s$ where all elements
are less than the corresponding elements of $s$.

Arrays are given by the type \AF{Ar} \AB{s} \AB{X} where $s$ is a shape of the
array and $X$ is the type of array elements. We allow shapes to be empty, in
which case the array represents a scalar. Formal definitions of \AF{S}, \AF{P}
and \AF{Ar} are as follows:

\begin{mathpar}
\codeblock{\begin{code}%
%
\>[2]\AgdaKeyword{data}\AgdaSpace{}%
\AgdaDatatype{S}\AgdaSpace{}%
\AgdaSymbol{:}\AgdaSpace{}%
\AgdaPrimitive{Set}\AgdaSpace{}%
\AgdaKeyword{where}\<%
\\
\>[2][@{}l@{\AgdaIndent{0}}]%
\>[4]\AgdaInductiveConstructor{[]}%
\>[9]\AgdaSymbol{:}\AgdaSpace{}%
\AgdaDatatype{S}\<%
\\
%
\>[4]\AgdaOperator{\AgdaInductiveConstructor{\AgdaUnderscore{}∷\AgdaUnderscore{}}}%
\>[9]\AgdaSymbol{:}\AgdaSpace{}%
\AgdaDatatype{ℕ}\AgdaSpace{}%
\AgdaSymbol{→}\AgdaSpace{}%
\AgdaDatatype{S}\AgdaSpace{}%
\AgdaSymbol{→}\AgdaSpace{}%
\AgdaDatatype{S}\<%
\end{code}
\begin{code}[hide]%
%
\>[2]\AgdaKeyword{variable}\<%
\\
\>[2][@{}l@{\AgdaIndent{0}}]%
\>[4]\AgdaGeneralizable{m}\AgdaSpace{}%
\AgdaGeneralizable{n}\AgdaSpace{}%
\AgdaGeneralizable{k}\AgdaSpace{}%
\AgdaSymbol{:}\AgdaSpace{}%
\AgdaDatatype{ℕ}\<%
\\
%
\>[4]\AgdaGeneralizable{s}\AgdaSpace{}%
\AgdaGeneralizable{p}\AgdaSpace{}%
\AgdaGeneralizable{q}\AgdaSpace{}%
\AgdaGeneralizable{r}\AgdaSpace{}%
\AgdaGeneralizable{u}\AgdaSpace{}%
\AgdaGeneralizable{w}\AgdaSpace{}%
\AgdaSymbol{:}\AgdaSpace{}%
\AgdaDatatype{S}\<%
\\
%
\>[4]\AgdaGeneralizable{X}\AgdaSpace{}%
\AgdaGeneralizable{Y}\AgdaSpace{}%
\AgdaGeneralizable{Z}\AgdaSpace{}%
\AgdaSymbol{:}\AgdaSpace{}%
\AgdaPrimitive{Set}\<%
\end{code}}
\and
\codeblock{\begin{code}%
%
\>[2]\AgdaKeyword{data}\AgdaSpace{}%
\AgdaDatatype{P}\AgdaSpace{}%
\AgdaSymbol{:}\AgdaSpace{}%
\AgdaDatatype{S}\AgdaSpace{}%
\AgdaSymbol{→}\AgdaSpace{}%
\AgdaPrimitive{Set}\AgdaSpace{}%
\AgdaKeyword{where}\<%
\\
\>[2][@{}l@{\AgdaIndent{0}}]%
\>[4]\AgdaInductiveConstructor{[]}%
\>[9]\AgdaSymbol{:}\AgdaSpace{}%
\AgdaDatatype{P}\AgdaSpace{}%
\AgdaInductiveConstructor{[]}\<%
\\
%
\>[4]\AgdaOperator{\AgdaInductiveConstructor{\AgdaUnderscore{}∷\AgdaUnderscore{}}}%
\>[9]\AgdaSymbol{:}\AgdaSpace{}%
\AgdaDatatype{Fin}\AgdaSpace{}%
\AgdaGeneralizable{n}\AgdaSpace{}%
\AgdaSymbol{→}\AgdaSpace{}%
\AgdaDatatype{P}\AgdaSpace{}%
\AgdaGeneralizable{s}\AgdaSpace{}%
\AgdaSymbol{→}\AgdaSpace{}%
\AgdaDatatype{P}\AgdaSpace{}%
\AgdaSymbol{(}\AgdaGeneralizable{n}\AgdaSpace{}%
\AgdaOperator{\AgdaInductiveConstructor{∷}}\AgdaSpace{}%
\AgdaGeneralizable{s}\AgdaSymbol{)}\<%
\end{code}}
\and
\codeblock{\begin{code}%
%
\>[2]\AgdaFunction{Ar}\AgdaSpace{}%
\AgdaSymbol{:}\AgdaSpace{}%
\AgdaDatatype{S}\AgdaSpace{}%
\AgdaSymbol{→}\AgdaSpace{}%
\AgdaPrimitive{Set}\AgdaSpace{}%
\AgdaSymbol{→}\AgdaSpace{}%
\AgdaPrimitive{Set}\<%
\\
%
\>[2]\AgdaFunction{Ar}\AgdaSpace{}%
\AgdaBound{s}\AgdaSpace{}%
\AgdaBound{X}\AgdaSpace{}%
\AgdaSymbol{=}\AgdaSpace{}%
\AgdaDatatype{P}\AgdaSpace{}%
\AgdaBound{s}\AgdaSpace{}%
\AgdaSymbol{→}\AgdaSpace{}%
\AgdaBound{X}\<%
\end{code}}
\end{mathpar}
The type \AF{Fin} $n$ represents natural numbers bounded by $n$.
As arrays are functions, selections are function applications and
the array constructor is a function definition (\eg{} via $\lambda$-abstraction).

\paragraph{Array Combinators} It is helpful to invest a little time
in defining array combinators.  First, we can observe that \AD{Ar} of
a fixed shape is an applicative functor~\cite{applicative}, so we can trivially derive:
\AF{K}\ \AB{x} to produce a constant array; \AF{map}\ \AB{f}\ \AB{a}
to apply \AB{f} to all the elements of \AB{a}; and \AF{zipWith}\ \AB{f}
\ \AB{a}\ \AB{b} to point-wise apply the binary operation 
\AB{f} to \AB{a} and \AB{b}.
\begin{mathpar}
\codeblock{\begin{code}%
%
\>[2]\AgdaFunction{K}\AgdaSpace{}%
\AgdaSymbol{:}\AgdaSpace{}%
\AgdaGeneralizable{X}\AgdaSpace{}%
\AgdaSymbol{→}\AgdaSpace{}%
\AgdaFunction{Ar}\AgdaSpace{}%
\AgdaGeneralizable{s}\AgdaSpace{}%
\AgdaGeneralizable{X}\<%
\\
%
\>[2]\AgdaFunction{K}\AgdaSpace{}%
\AgdaBound{x}\AgdaSpace{}%
\AgdaBound{i}\AgdaSpace{}%
\AgdaSymbol{=}\AgdaSpace{}%
\AgdaBound{x}\<%
\end{code}}
\and
\codeblock{\begin{code}%
%
\>[2]\AgdaFunction{map}\AgdaSpace{}%
\AgdaSymbol{:}\AgdaSpace{}%
\AgdaSymbol{(}\AgdaGeneralizable{X}\AgdaSpace{}%
\AgdaSymbol{→}\AgdaSpace{}%
\AgdaGeneralizable{Y}\AgdaSymbol{)}\AgdaSpace{}%
\AgdaSymbol{→}\AgdaSpace{}%
\AgdaFunction{Ar}\AgdaSpace{}%
\AgdaGeneralizable{s}\AgdaSpace{}%
\AgdaGeneralizable{X}\AgdaSpace{}%
\AgdaSymbol{→}\AgdaSpace{}%
\AgdaFunction{Ar}\AgdaSpace{}%
\AgdaGeneralizable{s}\AgdaSpace{}%
\AgdaGeneralizable{Y}\<%
\\
%
\>[2]\AgdaFunction{map}\AgdaSpace{}%
\AgdaBound{f}\AgdaSpace{}%
\AgdaBound{a}\AgdaSpace{}%
\AgdaBound{i}\AgdaSpace{}%
\AgdaSymbol{=}\AgdaSpace{}%
\AgdaBound{f}\AgdaSpace{}%
\AgdaSymbol{(}\AgdaBound{a}\AgdaSpace{}%
\AgdaBound{i}\AgdaSymbol{)}\<%
\end{code}}
\and
\codeblock{\begin{code}%
%
\>[2]\AgdaFunction{zipWith}\AgdaSpace{}%
\AgdaSymbol{:}\AgdaSpace{}%
\AgdaSymbol{(}\AgdaGeneralizable{X}\AgdaSpace{}%
\AgdaSymbol{→}\AgdaSpace{}%
\AgdaGeneralizable{Y}\AgdaSpace{}%
\AgdaSymbol{→}\AgdaSpace{}%
\AgdaGeneralizable{Z}\AgdaSymbol{)}\AgdaSpace{}%
\AgdaSymbol{→}\AgdaSpace{}%
\AgdaFunction{Ar}\AgdaSpace{}%
\AgdaGeneralizable{s}\AgdaSpace{}%
\AgdaGeneralizable{X}\AgdaSpace{}%
\AgdaSymbol{→}\AgdaSpace{}%
\AgdaFunction{Ar}\AgdaSpace{}%
\AgdaGeneralizable{s}\AgdaSpace{}%
\AgdaGeneralizable{Y}\AgdaSpace{}%
\AgdaSymbol{→}\AgdaSpace{}%
\AgdaFunction{Ar}\AgdaSpace{}%
\AgdaGeneralizable{s}\AgdaSpace{}%
\AgdaGeneralizable{Z}\<%
\\
%
\>[2]\AgdaFunction{zipWith}\AgdaSpace{}%
\AgdaBound{f}\AgdaSpace{}%
\AgdaBound{a}\AgdaSpace{}%
\AgdaBound{b}\AgdaSpace{}%
\AgdaBound{i}\AgdaSpace{}%
\AgdaSymbol{=}\AgdaSpace{}%
\AgdaBound{f}\AgdaSpace{}%
\AgdaSymbol{(}\AgdaBound{a}\AgdaSpace{}%
\AgdaBound{i}\AgdaSymbol{)}\AgdaSpace{}%
\AgdaSymbol{(}\AgdaBound{b}\AgdaSpace{}%
\AgdaBound{i}\AgdaSymbol{)}\<%
\end{code}}
\end{mathpar}

Array shapes can be concatenated as lists.  We call this operation
\emph{shape product} and we denote it with \AF{\_⊗\_} (because this
corresponds to the shape of tensor product).  Positions of sub-shapes
can be joined into a position of a product shape using the \AF{\_⊗ₚ\_}
operation.  Dually, positions of a product shape can be split into
positions of the corresponding subshapes using \AF{split}.  The types
of these three operations are as follows.
\begin{mathpar}
\codeblock{\begin{code}%
%
\>[2]\AgdaOperator{\AgdaFunction{\AgdaUnderscore{}⊗\AgdaUnderscore{}}}\AgdaSpace{}%
\AgdaSymbol{:}\AgdaSpace{}%
\AgdaDatatype{S}\AgdaSpace{}%
\AgdaSymbol{→}\AgdaSpace{}%
\AgdaDatatype{S}\AgdaSpace{}%
\AgdaSymbol{→}\AgdaSpace{}%
\AgdaDatatype{S}\<%
\end{code}}
\and
\codeblock{\begin{code}%
%
\>[2]\AgdaOperator{\AgdaFunction{\AgdaUnderscore{}⊗ₚ\AgdaUnderscore{}}}\AgdaSpace{}%
\AgdaSymbol{:}\AgdaSpace{}%
\AgdaDatatype{P}\AgdaSpace{}%
\AgdaGeneralizable{s}\AgdaSpace{}%
\AgdaSymbol{→}\AgdaSpace{}%
\AgdaDatatype{P}\AgdaSpace{}%
\AgdaGeneralizable{p}\AgdaSpace{}%
\AgdaSymbol{→}\AgdaSpace{}%
\AgdaDatatype{P}\AgdaSpace{}%
\AgdaSymbol{(}\AgdaGeneralizable{s}\AgdaSpace{}%
\AgdaOperator{\AgdaFunction{⊗}}\AgdaSpace{}%
\AgdaGeneralizable{p}\AgdaSymbol{)}\<%
\end{code}}
\and
\codeblock{\begin{code}%
%
\>[2]\AgdaFunction{split}\AgdaSpace{}%
\AgdaSymbol{:}\AgdaSpace{}%
\AgdaDatatype{P}\AgdaSpace{}%
\AgdaSymbol{(}\AgdaGeneralizable{s}\AgdaSpace{}%
\AgdaOperator{\AgdaFunction{⊗}}\AgdaSpace{}%
\AgdaGeneralizable{p}\AgdaSymbol{)}\AgdaSpace{}%
\AgdaSymbol{→}\AgdaSpace{}%
\AgdaDatatype{P}\AgdaSpace{}%
\AgdaGeneralizable{s}\AgdaSpace{}%
\AgdaOperator{\AgdaFunction{×}}\AgdaSpace{}%
\AgdaDatatype{P}\AgdaSpace{}%
\AgdaGeneralizable{p}\<%
\end{code}}
\end{mathpar}
\begin{code}[hide]%
%
\>[2]\AgdaInductiveConstructor{[]}\AgdaSpace{}%
\AgdaOperator{\AgdaFunction{⊗}}\AgdaSpace{}%
\AgdaBound{p}\AgdaSpace{}%
\AgdaSymbol{=}\AgdaSpace{}%
\AgdaBound{p}\<%
\\
%
\>[2]\AgdaSymbol{(}\AgdaBound{n}\AgdaSpace{}%
\AgdaOperator{\AgdaInductiveConstructor{∷}}\AgdaSpace{}%
\AgdaBound{s}\AgdaSymbol{)}\AgdaSpace{}%
\AgdaOperator{\AgdaFunction{⊗}}\AgdaSpace{}%
\AgdaBound{p}\AgdaSpace{}%
\AgdaSymbol{=}\AgdaSpace{}%
\AgdaBound{n}\AgdaSpace{}%
\AgdaOperator{\AgdaInductiveConstructor{∷}}\AgdaSpace{}%
\AgdaSymbol{(}\AgdaBound{s}\AgdaSpace{}%
\AgdaOperator{\AgdaFunction{⊗}}\AgdaSpace{}%
\AgdaBound{p}\AgdaSymbol{)}\<%
\\
%
\\[\AgdaEmptyExtraSkip]%
%
\>[2]\AgdaInductiveConstructor{[]}\AgdaSpace{}%
\AgdaOperator{\AgdaFunction{⊗ₚ}}\AgdaSpace{}%
\AgdaBound{jv}\AgdaSpace{}%
\AgdaSymbol{=}\AgdaSpace{}%
\AgdaBound{jv}\<%
\\
%
\>[2]\AgdaSymbol{(}\AgdaBound{i}\AgdaSpace{}%
\AgdaOperator{\AgdaInductiveConstructor{∷}}\AgdaSpace{}%
\AgdaBound{iv}\AgdaSymbol{)}\AgdaSpace{}%
\AgdaOperator{\AgdaFunction{⊗ₚ}}\AgdaSpace{}%
\AgdaBound{jv}\AgdaSpace{}%
\AgdaSymbol{=}\AgdaSpace{}%
\AgdaBound{i}\AgdaSpace{}%
\AgdaOperator{\AgdaInductiveConstructor{∷}}\AgdaSpace{}%
\AgdaSymbol{(}\AgdaBound{iv}\AgdaSpace{}%
\AgdaOperator{\AgdaFunction{⊗ₚ}}\AgdaSpace{}%
\AgdaBound{jv}\AgdaSymbol{)}\<%
\\
%
\\[\AgdaEmptyExtraSkip]%
%
\>[2]\AgdaFunction{split}\AgdaSpace{}%
\AgdaSymbol{\{}\AgdaArgument{s}\AgdaSpace{}%
\AgdaSymbol{=}\AgdaSpace{}%
\AgdaInductiveConstructor{[]}\AgdaSymbol{\}}%
\>[20]\AgdaBound{is}\AgdaSpace{}%
\AgdaSymbol{=}\AgdaSpace{}%
\AgdaInductiveConstructor{[]}\AgdaSpace{}%
\AgdaOperator{\AgdaInductiveConstructor{,}}\AgdaSpace{}%
\AgdaBound{is}\<%
\\
%
\>[2]\AgdaFunction{split}\AgdaSpace{}%
\AgdaSymbol{\{}\AgdaArgument{s}\AgdaSpace{}%
\AgdaSymbol{=}\AgdaSpace{}%
\AgdaBound{x}\AgdaSpace{}%
\AgdaOperator{\AgdaInductiveConstructor{∷}}\AgdaSpace{}%
\AgdaBound{s}\AgdaSymbol{\}}\AgdaSpace{}%
\AgdaSymbol{(}\AgdaBound{i}\AgdaSpace{}%
\AgdaOperator{\AgdaInductiveConstructor{∷}}\AgdaSpace{}%
\AgdaBound{is}\AgdaSymbol{)}\AgdaSpace{}%
\AgdaSymbol{=}\AgdaSpace{}%
\AgdaFunction{Prod.map₁}\AgdaSpace{}%
\AgdaSymbol{(}\AgdaBound{i}\AgdaSpace{}%
\AgdaOperator{\AgdaInductiveConstructor{∷\AgdaUnderscore{}}}\AgdaSymbol{)}\AgdaSpace{}%
\AgdaSymbol{(}\AgdaFunction{split}\AgdaSpace{}%
\AgdaBound{is}\AgdaSymbol{)}\<%
\\
%
\\[\AgdaEmptyExtraSkip]%
%
\>[2]\AgdaOperator{\AgdaFunction{\AgdaUnderscore{}≟ₚ\AgdaUnderscore{}}}\AgdaSpace{}%
\AgdaSymbol{:}\AgdaSpace{}%
\AgdaSymbol{(}\AgdaBound{i}\AgdaSpace{}%
\AgdaBound{j}\AgdaSpace{}%
\AgdaSymbol{:}\AgdaSpace{}%
\AgdaDatatype{P}\AgdaSpace{}%
\AgdaGeneralizable{s}\AgdaSymbol{)}\AgdaSpace{}%
\AgdaSymbol{→}\AgdaSpace{}%
\AgdaRecord{Dec}\AgdaSpace{}%
\AgdaSymbol{(}\AgdaBound{i}\AgdaSpace{}%
\AgdaOperator{\AgdaDatatype{≡}}\AgdaSpace{}%
\AgdaBound{j}\AgdaSymbol{)}\<%
\\
%
\>[2]\AgdaOperator{\AgdaFunction{\AgdaUnderscore{}≟ₚ\AgdaUnderscore{}}}\AgdaSpace{}%
\AgdaSymbol{\{}\AgdaInductiveConstructor{[]}\AgdaSymbol{\}}\AgdaSpace{}%
\AgdaInductiveConstructor{[]}\AgdaSpace{}%
\AgdaInductiveConstructor{[]}\AgdaSpace{}%
\AgdaSymbol{=}\AgdaSpace{}%
\AgdaInductiveConstructor{yes}\AgdaSpace{}%
\AgdaInductiveConstructor{refl}\<%
\\
%
\>[2]\AgdaOperator{\AgdaFunction{\AgdaUnderscore{}≟ₚ\AgdaUnderscore{}}}\AgdaSpace{}%
\AgdaSymbol{\{}\AgdaBound{x}\AgdaSpace{}%
\AgdaOperator{\AgdaInductiveConstructor{∷}}\AgdaSpace{}%
\AgdaBound{s}\AgdaSymbol{\}}\AgdaSpace{}%
\AgdaSymbol{(}\AgdaBound{i}\AgdaSpace{}%
\AgdaOperator{\AgdaInductiveConstructor{∷}}\AgdaSpace{}%
\AgdaBound{is}\AgdaSymbol{)}\AgdaSpace{}%
\AgdaSymbol{(}\AgdaBound{j}\AgdaSpace{}%
\AgdaOperator{\AgdaInductiveConstructor{∷}}\AgdaSpace{}%
\AgdaBound{js}\AgdaSymbol{)}\AgdaSpace{}%
\AgdaKeyword{with}\AgdaSpace{}%
\AgdaBound{i}\AgdaSpace{}%
\AgdaOperator{\AgdaFunction{F.≟}}\AgdaSpace{}%
\AgdaBound{j}\<%
\\
%
\>[2]\AgdaSymbol{...}\AgdaSpace{}%
\AgdaSymbol{|}\AgdaSpace{}%
\AgdaInductiveConstructor{no}\AgdaSpace{}%
\AgdaBound{¬p}\AgdaSpace{}%
\AgdaSymbol{=}\AgdaSpace{}%
\AgdaInductiveConstructor{no}\AgdaSpace{}%
\AgdaSymbol{λ}\AgdaSpace{}%
\AgdaSymbol{\{}\AgdaSpace{}%
\AgdaInductiveConstructor{refl}\AgdaSpace{}%
\AgdaSymbol{→}\AgdaSpace{}%
\AgdaBound{¬p}\AgdaSpace{}%
\AgdaInductiveConstructor{refl}\AgdaSpace{}%
\AgdaSymbol{\}}\<%
\\
%
\>[2]\AgdaSymbol{...}\AgdaSpace{}%
\AgdaSymbol{|}\AgdaSpace{}%
\AgdaInductiveConstructor{yes}\AgdaSpace{}%
\AgdaInductiveConstructor{refl}\AgdaSpace{}%
\AgdaKeyword{with}\AgdaSpace{}%
\AgdaBound{is}\AgdaSpace{}%
\AgdaOperator{\AgdaFunction{≟ₚ}}\AgdaSpace{}%
\AgdaBound{js}\<%
\\
%
\>[2]\AgdaSymbol{...}\AgdaSpace{}%
\AgdaSymbol{|}\AgdaSpace{}%
\AgdaInductiveConstructor{no}\AgdaSpace{}%
\AgdaBound{¬q}\AgdaSpace{}%
\AgdaSymbol{=}\AgdaSpace{}%
\AgdaInductiveConstructor{no}\AgdaSpace{}%
\AgdaSymbol{λ}\AgdaSpace{}%
\AgdaSymbol{\{}\AgdaSpace{}%
\AgdaInductiveConstructor{refl}\AgdaSpace{}%
\AgdaSymbol{→}\AgdaSpace{}%
\AgdaBound{¬q}\AgdaSpace{}%
\AgdaInductiveConstructor{refl}\AgdaSpace{}%
\AgdaSymbol{\}}\<%
\\
%
\>[2]\AgdaSymbol{...}\AgdaSpace{}%
\AgdaSymbol{|}\AgdaSpace{}%
\AgdaInductiveConstructor{yes}\AgdaSpace{}%
\AgdaInductiveConstructor{refl}\AgdaSpace{}%
\AgdaSymbol{=}\AgdaSpace{}%
\AgdaInductiveConstructor{yes}\AgdaSpace{}%
\AgdaInductiveConstructor{refl}\<%
\end{code}

Arrays are homogeneously nested, \ie{} the shapes of all the sub-arrays
have to be the same.  Therefore, we can switch between the array of a product
shape and the nested array (array of arrays).  This operation is very similar
to currying except it happens at the level of shapes.  The combinators that
achieve this are named \AF{nest} and \AF{unnest} and their definitions are:
\begin{mathpar}
\codeblock{\begin{code}%
%
\>[2]\AgdaFunction{nest}\AgdaSpace{}%
\AgdaSymbol{:}\AgdaSpace{}%
\AgdaFunction{Ar}\AgdaSpace{}%
\AgdaSymbol{(}\AgdaGeneralizable{s}\AgdaSpace{}%
\AgdaOperator{\AgdaFunction{⊗}}\AgdaSpace{}%
\AgdaGeneralizable{p}\AgdaSymbol{)}\AgdaSpace{}%
\AgdaGeneralizable{X}\AgdaSpace{}%
\AgdaSymbol{→}\AgdaSpace{}%
\AgdaFunction{Ar}\AgdaSpace{}%
\AgdaGeneralizable{s}\AgdaSpace{}%
\AgdaSymbol{(}\AgdaFunction{Ar}\AgdaSpace{}%
\AgdaGeneralizable{p}\AgdaSpace{}%
\AgdaGeneralizable{X}\AgdaSymbol{)}\<%
\\
%
\>[2]\AgdaFunction{nest}\AgdaSpace{}%
\AgdaBound{a}\AgdaSpace{}%
\AgdaBound{i}\AgdaSpace{}%
\AgdaBound{j}\AgdaSpace{}%
\AgdaSymbol{=}\AgdaSpace{}%
\AgdaBound{a}\AgdaSpace{}%
\AgdaSymbol{(}\AgdaBound{i}\AgdaSpace{}%
\AgdaOperator{\AgdaFunction{⊗ₚ}}\AgdaSpace{}%
\AgdaBound{j}\AgdaSymbol{)}\<%
\end{code}}
\and
\codeblock{\begin{code} %
%
\>[2]\AgdaFunction{unnest}\AgdaSpace{}%
\AgdaSymbol{:}\AgdaSpace{}%
\AgdaFunction{Ar}\AgdaSpace{}%
\AgdaGeneralizable{s}\AgdaSpace{}%
\AgdaSymbol{(}\AgdaFunction{Ar}\AgdaSpace{}%
\AgdaGeneralizable{p}\AgdaSpace{}%
\AgdaGeneralizable{X}\AgdaSymbol{)}\AgdaSpace{}%
\AgdaSymbol{→}\AgdaSpace{}%
\AgdaFunction{Ar}\AgdaSpace{}%
\AgdaSymbol{(}\AgdaGeneralizable{s}\AgdaSpace{}%
\AgdaOperator{\AgdaFunction{⊗}}\AgdaSpace{}%
\AgdaGeneralizable{p}\AgdaSymbol{)}\AgdaSpace{}%
\AgdaGeneralizable{X}\<%
\\
%
\>[2]\AgdaFunction{unnest}\AgdaSpace{}%
\AgdaBound{a}\AgdaSpace{}%
\AgdaBound{i}\AgdaSpace{}%
\AgdaSymbol{=}\AgdaSpace{}%
\AgdaFunction{uncurry}\AgdaSpace{}%
\AgdaBound{a}\AgdaSpace{}%
\AgdaSymbol{(}\AgdaFunction{split}\AgdaSpace{}%
\AgdaBound{i}\AgdaSymbol{)}\<%
\end{code}}
\end{mathpar}


\paragraph{Reduction} We implement reduction of the binary operations
over arrays in two steps.  Firstly, we define 1-d reductions  that
we call \AD{sum₁} which is similar to right fold on lists.
Arrays of higher ranks iterate \AF{sum₁} bottom-up.  The definition
of the primitives are as follows:
\begin{mathpar}
\codeblock{\begin{code}%
%
\>[2]\AgdaKeyword{pattern}\AgdaSpace{}%
\AgdaInductiveConstructor{ι}\AgdaSpace{}%
\AgdaBound{n}\AgdaSpace{}%
\AgdaSymbol{=}\AgdaSpace{}%
\AgdaBound{n}\AgdaSpace{}%
\AgdaOperator{\AgdaInductiveConstructor{∷}}\AgdaSpace{}%
\AgdaInductiveConstructor{[]}\<%
\\
%
\\[\AgdaEmptyExtraSkip]%
%
\>[2]\AgdaFunction{ιsuc}\AgdaSpace{}%
\AgdaSymbol{:}\AgdaSpace{}%
\AgdaDatatype{P}\AgdaSpace{}%
\AgdaSymbol{(}\AgdaInductiveConstructor{ι}\AgdaSpace{}%
\AgdaGeneralizable{n}\AgdaSymbol{)}\AgdaSpace{}%
\AgdaSymbol{→}\AgdaSpace{}%
\AgdaDatatype{P}\AgdaSpace{}%
\AgdaSymbol{(}\AgdaInductiveConstructor{ι}\AgdaSpace{}%
\AgdaSymbol{(}\AgdaInductiveConstructor{suc}\AgdaSpace{}%
\AgdaGeneralizable{n}\AgdaSymbol{))}\<%
\\
%
\>[2]\AgdaFunction{ιsuc}\AgdaSpace{}%
\AgdaSymbol{(}\AgdaInductiveConstructor{ι}\AgdaSpace{}%
\AgdaBound{i}\AgdaSymbol{)}\AgdaSpace{}%
\AgdaSymbol{=}\AgdaSpace{}%
\AgdaInductiveConstructor{ι}\AgdaSpace{}%
\AgdaSymbol{(}\AgdaInductiveConstructor{suc}\AgdaSpace{}%
\AgdaBound{i}\AgdaSymbol{)}\<%
\end{code}}
\and
\codeblock{\begin{code}%
%
\>[2]\AgdaFunction{sum₁}\AgdaSpace{}%
\AgdaSymbol{:}\AgdaSpace{}%
\AgdaSymbol{(}\AgdaGeneralizable{X}\AgdaSpace{}%
\AgdaSymbol{→}\AgdaSpace{}%
\AgdaGeneralizable{X}\AgdaSpace{}%
\AgdaSymbol{→}\AgdaSpace{}%
\AgdaGeneralizable{X}\AgdaSymbol{)}\AgdaSpace{}%
\AgdaSymbol{→}\AgdaSpace{}%
\AgdaGeneralizable{X}\AgdaSpace{}%
\AgdaSymbol{→}\AgdaSpace{}%
\AgdaFunction{Ar}\AgdaSpace{}%
\AgdaSymbol{(}\AgdaInductiveConstructor{ι}\AgdaSpace{}%
\AgdaGeneralizable{n}\AgdaSymbol{)}\AgdaSpace{}%
\AgdaGeneralizable{X}\AgdaSpace{}%
\AgdaSymbol{→}\AgdaSpace{}%
\AgdaGeneralizable{X}\<%
\\
%
\>[2]\AgdaFunction{sum₁}\AgdaSpace{}%
\AgdaSymbol{\{}\AgdaArgument{n}\AgdaSpace{}%
\AgdaSymbol{=}\AgdaSpace{}%
\AgdaInductiveConstructor{zero}\AgdaSymbol{\}}%
\>[20]\AgdaBound{f}\AgdaSpace{}%
\AgdaBound{ε}\AgdaSpace{}%
\AgdaBound{a}\AgdaSpace{}%
\AgdaSymbol{=}\AgdaSpace{}%
\AgdaBound{ε}\<%
\\
%
\>[2]\AgdaFunction{sum₁}\AgdaSpace{}%
\AgdaSymbol{\{}\AgdaArgument{n}\AgdaSpace{}%
\AgdaSymbol{=}\AgdaSpace{}%
\AgdaInductiveConstructor{suc}\AgdaSpace{}%
\AgdaBound{n}\AgdaSymbol{\}}%
\>[20]\AgdaBound{f}\AgdaSpace{}%
\AgdaBound{ε}\AgdaSpace{}%
\AgdaBound{a}\AgdaSpace{}%
\AgdaSymbol{=}\AgdaSpace{}%
\AgdaBound{f}\AgdaSpace{}%
\AgdaSymbol{(}\AgdaBound{a}\AgdaSpace{}%
\AgdaSymbol{(}\AgdaInductiveConstructor{ι}\AgdaSpace{}%
\AgdaInductiveConstructor{zero}\AgdaSymbol{))}\AgdaSpace{}%
\AgdaSymbol{(}\AgdaFunction{sum₁}\AgdaSpace{}%
\AgdaBound{f}\AgdaSpace{}%
\AgdaBound{ε}\AgdaSpace{}%
\AgdaSymbol{(}\AgdaBound{a}\AgdaSpace{}%
\AgdaOperator{\AgdaFunction{∘}}\AgdaSpace{}%
\AgdaFunction{ιsuc}\AgdaSymbol{))}\<%
\end{code}}
\and
\codeblock{\begin{code}%
%
\>[2]\AgdaFunction{sum}\AgdaSpace{}%
\AgdaSymbol{:}\AgdaSpace{}%
\AgdaSymbol{(}\AgdaGeneralizable{X}\AgdaSpace{}%
\AgdaSymbol{→}\AgdaSpace{}%
\AgdaGeneralizable{X}\AgdaSpace{}%
\AgdaSymbol{→}\AgdaSpace{}%
\AgdaGeneralizable{X}\AgdaSymbol{)}\AgdaSpace{}%
\AgdaSymbol{→}\AgdaSpace{}%
\AgdaGeneralizable{X}\AgdaSpace{}%
\AgdaSymbol{→}\AgdaSpace{}%
\AgdaFunction{Ar}\AgdaSpace{}%
\AgdaGeneralizable{s}\AgdaSpace{}%
\AgdaGeneralizable{X}\AgdaSpace{}%
\AgdaSymbol{→}\AgdaSpace{}%
\AgdaGeneralizable{X}\<%
\\
%
\>[2]\AgdaFunction{sum}\AgdaSpace{}%
\AgdaSymbol{\{}\AgdaArgument{s}\AgdaSpace{}%
\AgdaSymbol{=}\AgdaSpace{}%
\AgdaInductiveConstructor{[]}\AgdaSymbol{\}}%
\>[19]\AgdaBound{f}\AgdaSpace{}%
\AgdaBound{ε}\AgdaSpace{}%
\AgdaBound{a}\AgdaSpace{}%
\AgdaSymbol{=}\AgdaSpace{}%
\AgdaBound{a}\AgdaSpace{}%
\AgdaInductiveConstructor{[]}\<%
\\
%
\>[2]\AgdaFunction{sum}\AgdaSpace{}%
\AgdaSymbol{\{}\AgdaArgument{s}\AgdaSpace{}%
\AgdaSymbol{=}\AgdaSpace{}%
\AgdaBound{x}\AgdaSpace{}%
\AgdaOperator{\AgdaInductiveConstructor{∷}}\AgdaSpace{}%
\AgdaBound{s}\AgdaSymbol{\}}%
\>[19]\AgdaBound{f}\AgdaSpace{}%
\AgdaBound{ε}\AgdaSpace{}%
\AgdaBound{a}\AgdaSpace{}%
\AgdaSymbol{=}\AgdaSpace{}%
\AgdaFunction{sum₁}\AgdaSpace{}%
\AgdaBound{f}\AgdaSpace{}%
\AgdaBound{ε}\AgdaSpace{}%
\AgdaOperator{\AgdaFunction{\$}}\AgdaSpace{}%
\AgdaFunction{map}\AgdaSpace{}%
\AgdaSymbol{(}\AgdaFunction{sum}\AgdaSpace{}%
\AgdaBound{f}\AgdaSpace{}%
\AgdaBound{ε}\AgdaSymbol{)}\AgdaSpace{}%
\AgdaSymbol{(}\AgdaFunction{nest}\AgdaSpace{}%
\AgdaBound{a}\AgdaSymbol{)}\<%
\end{code}}
\end{mathpar}

While \AF{sum} resembles \texttt{foldr}, its behaviour differs from that of a
conventional \texttt{foldr} over a free monad. Intuitively, rather than
selecting an order for the elements and reducing them, \AF{sum} reduces
lower-dimensional sub-arrays (conceptually in parallel) and subsequently
reduces the result. For instance, if we fix the binary operation \AB{f} and the
neutral element \AB{ε} (for example, \AF{σ} = \AF{sum} \AB{f} \AB{ε}), we can
demonstrate that \AF{σ} (\AF{map} \AF{σ} \AB{a}) \AF{≡} \AF{σ} (\AF{unnest}
\AB{a}) for all arrays \(a\). This property simplifies some of the proofs;
however, this subtlety becomes irrelevant when we operate within a monoid where
\AB{f} is the binary operation and \AB{e} is the neutral element.

It is also important to note that \AF{sum} enforces the types of the arguments
of the binary operation to be identical, which distinguishes it from the
conventional definitions of \texttt{foldr}. Although this generality is not
necessary for the purposes of this paper, it is noteworthy that the standard
behaviour can be recovered\footnote{We recover the regular fold behaviour by
applying \AD{sum} over function composition:
\begin{code}%
%
\>[2]\AgdaFunction{sum′}\AgdaSpace{}%
\AgdaSymbol{:}\AgdaSpace{}%
\AgdaSymbol{(}\AgdaGeneralizable{X}\AgdaSpace{}%
\AgdaSymbol{→}\AgdaSpace{}%
\AgdaGeneralizable{Y}\AgdaSpace{}%
\AgdaSymbol{→}\AgdaSpace{}%
\AgdaGeneralizable{Y}\AgdaSymbol{)}\AgdaSpace{}%
\AgdaSymbol{→}\AgdaSpace{}%
\AgdaGeneralizable{Y}\AgdaSpace{}%
\AgdaSymbol{→}\AgdaSpace{}%
\AgdaFunction{Ar}\AgdaSpace{}%
\AgdaGeneralizable{s}\AgdaSpace{}%
\AgdaGeneralizable{X}\AgdaSpace{}%
\AgdaSymbol{→}\AgdaSpace{}%
\AgdaGeneralizable{Y}\<%
\\
%
\>[2]\AgdaFunction{sum′}\AgdaSpace{}%
\AgdaBound{f}\AgdaSpace{}%
\AgdaBound{ε}\AgdaSpace{}%
\AgdaBound{a}\AgdaSpace{}%
\AgdaSymbol{=}\AgdaSpace{}%
\AgdaFunction{sum}\AgdaSpace{}%
\AgdaOperator{\AgdaFunction{\AgdaUnderscore{}∘′\AgdaUnderscore{}}}\AgdaSpace{}%
\AgdaFunction{id}\AgdaSpace{}%
\AgdaSymbol{(}\AgdaFunction{map}\AgdaSpace{}%
\AgdaBound{f}\AgdaSpace{}%
\AgdaBound{a}\AgdaSymbol{)}\AgdaSpace{}%
\AgdaBound{ε}\<%
\end{code}
} through reduction of function composition.

% \paragraph{Reshaping}
% One common operation on arrays is element-preserving change of shape.  We call
% such an operation \AF{reshape}.  It is clear that array elements can be preserved only in
% cases when the number of elements in the original array and the reshaped one
% is the same.  We define an inductive relation \AF{Reshape} that relates
% only those shapes that preserve the number of array elements.  
% \begin{code}[hide]
%   infixr 5 _∙_
%   --infixl 10 _,_
% \end{code}
% \begin{mathpar}
% \codeblock{\begin{code}
%   data Reshape : S → S → Set where
%     eq      : Reshape s s
%     _∙_     : Reshape p q → Reshape s p → Reshape s q
%     _,_     : Reshape s p → Reshape q r → Reshape (s ⊗ q) (p ⊗ r)
%     split   : Reshape (ι (m * n)) (ι m ⊗ ι n)
%     flat    : Reshape (ι m ⊗ ι n) (ι (m * n))
%     swap    : Reshape (s ⊗ p) (p ⊗ s)
%     assocl  : Reshape (s ⊗ (p ⊗ q)) ((s ⊗ p) ⊗ q)
%     assocr  : Reshape ((s ⊗ p) ⊗ q) (s ⊗ (p ⊗ q))
% \end{code}}
% \end{mathpar}
% Any expression $r$ of
% the type (\AF{Reshape} \AB{s} \AB{p}) comes with a straight-forward action on
% indices that we denote \AF{\_⟨\_⟩} (its definition is omitted).
% Such a (contravariant) action translates
% the index within the shape \AB{p} into the index within the shape \AB{s}.
% Given this translation, we can easily define \AF{reshape} as shown below.
% \AF{Reshape} is constructed such that if $s$ and $p$ are related, then 
% $p$ and $s$ are related too.  This fact is given by the function \AF{rev}
% (its definition is omitted) and it immediately implies that all the
% actions on indices as well as array \AF{reshape}s are invertible.
% 
% Note that two shapes can be related by \AF{Reshape} in more than
% one way, which results in different array reshapes.  
% For example, consider \AF{Reshape} (\AC{ι} 5 \AC{⊗} \AC{ι} 4) (\AC{ι} 5 \AC{⊗} \AC{ι} 4)
% given by \AC{swap} or through (\AC{split} \AC{∙} \AC{flat}).  While the former transposes 
% the array elements, the latter does not.
% \begin{mathpar}
% \codeblock{\begin{code}
%   _⟨_⟩ : P p → Reshape s p → P s
% \end{code}}
% \and
% \codeblock{\begin{code}
%   reshape : Reshape s p → Ar s X → Ar p X
%   reshape r a = λ ix → a (ix ⟨ r ⟩)
% \end{code}}
% \and
% \codeblock{\begin{code}
%   rev : Reshape s p → Reshape p s
% \end{code}}
% \end{mathpar}
% From the perspective of category theory, if \AF{S} is an object then \AF{Reshape}
% is a Hom set, where \AC{eq} is identity and \AC{\_∙\_} is a composition with
% the expected properties.  In the language of containers~\cite{containers}, \AF{Ar} is
% a container and \AF{Reshape} is an inductive subset of cartesian container morphisms.
% 


% \begin{code}[hide]
%   i ⟨ eq ⟩ = i
%   (i ⊗ j) ⟨ r , r₁ ⟩ = (i ⟨ r ⟩) ⊗ (j ⟨ r₁ ⟩)
%   i ⟨ r ∙ r₁ ⟩ = i ⟨ r ⟩ ⟨ r₁ ⟩
%   (ι i ⊗ ι j) ⟨ split ⟩ = ι (combine i j)
%   ι i ⟨ flat ⟩ = let a , b = remQuot _ i in ι a ⊗ ι b
%   (i ⊗ j) ⟨ swap ⟩ = j ⊗ i
%   ((i ⊗ j) ⊗ k) ⟨ assocl ⟩ = i ⊗ (j ⊗ k)
%   (i ⊗ (j ⊗ k)) ⟨ assocr ⟩ = (i ⊗ j) ⊗ k
%   
%   
%   rev eq = eq
%   rev (r , r₁) = rev r , rev r₁
%   rev (r ∙ r₁) = rev r₁ ∙ rev r
%   rev split = flat
%   rev flat = split
%   rev swap = swap
%   rev assocl = assocr
%   rev assocr = assocl
% \end{code}


\section{CNN Building Blocks\label{sec:cnn}}

Using the array theory and combinators from the previous section we
define the primitives that are needed for the CNN.

\subsection{One-dimensional convolution}
We start with plus and minus operations for 1-d indices which are
prerequisites for defining convolution:
\begin{code}[hide]%
%
\>[2]\AgdaFunction{inject-left}\AgdaSpace{}%
\AgdaSymbol{:}\AgdaSpace{}%
\AgdaDatatype{Fin}\AgdaSpace{}%
\AgdaSymbol{(}\AgdaInductiveConstructor{suc}\AgdaSpace{}%
\AgdaGeneralizable{m}\AgdaSymbol{)}\AgdaSpace{}%
\AgdaSymbol{→}\AgdaSpace{}%
\AgdaDatatype{Fin}\AgdaSpace{}%
\AgdaSymbol{(}\AgdaInductiveConstructor{suc}\AgdaSpace{}%
\AgdaSymbol{(}\AgdaGeneralizable{n}\AgdaSpace{}%
\AgdaOperator{\AgdaPrimitive{+}}\AgdaSpace{}%
\AgdaGeneralizable{m}\AgdaSymbol{))}\<%
\\
%
\>[2]\AgdaFunction{inject-left}\AgdaSpace{}%
\AgdaSymbol{\{}\AgdaBound{m}\AgdaSymbol{\}}\AgdaSpace{}%
\AgdaSymbol{\{}\AgdaBound{n}\AgdaSymbol{\}}\AgdaSpace{}%
\AgdaBound{i}\AgdaSpace{}%
\AgdaKeyword{rewrite}\AgdaSpace{}%
\AgdaFunction{+-comm}\AgdaSpace{}%
\AgdaBound{n}\AgdaSpace{}%
\AgdaBound{m}%
\>[44]\AgdaSymbol{=}\AgdaSpace{}%
\AgdaFunction{inject+}\AgdaSpace{}%
\AgdaSymbol{\AgdaUnderscore{}}\AgdaSpace{}%
\AgdaBound{i}\<%
\\
\>[0]\<%
\\
%
\>[2]\AgdaFunction{split-inj₁}\AgdaSpace{}%
\AgdaSymbol{:}\AgdaSpace{}%
\AgdaSymbol{(}\AgdaBound{i}\AgdaSpace{}%
\AgdaSymbol{:}\AgdaSpace{}%
\AgdaDatatype{Fin}\AgdaSpace{}%
\AgdaSymbol{(}\AgdaGeneralizable{m}\AgdaSpace{}%
\AgdaOperator{\AgdaPrimitive{+}}\AgdaSpace{}%
\AgdaGeneralizable{n}\AgdaSymbol{))}\AgdaSpace{}%
\AgdaSymbol{(}\AgdaBound{k}\AgdaSpace{}%
\AgdaSymbol{:}\AgdaSpace{}%
\AgdaDatatype{Fin}\AgdaSpace{}%
\AgdaGeneralizable{m}\AgdaSymbol{)}\AgdaSpace{}%
\AgdaSymbol{→}\AgdaSpace{}%
\AgdaFunction{splitAt}\AgdaSpace{}%
\AgdaGeneralizable{m}\AgdaSpace{}%
\AgdaBound{i}\AgdaSpace{}%
\AgdaOperator{\AgdaDatatype{≡}}\AgdaSpace{}%
\AgdaInductiveConstructor{inj₁}\AgdaSpace{}%
\AgdaBound{k}\AgdaSpace{}%
\AgdaSymbol{→}\AgdaSpace{}%
\AgdaFunction{inject+}\AgdaSpace{}%
\AgdaSymbol{\AgdaUnderscore{}}\AgdaSpace{}%
\AgdaBound{k}\AgdaSpace{}%
\AgdaOperator{\AgdaDatatype{≡}}\AgdaSpace{}%
\AgdaBound{i}\<%
\\
%
\>[2]\AgdaFunction{split-inj₁}\AgdaSpace{}%
\AgdaSymbol{\{}\AgdaInductiveConstructor{suc}\AgdaSpace{}%
\AgdaBound{m}\AgdaSymbol{\}}\AgdaSpace{}%
\AgdaInductiveConstructor{zero}\AgdaSpace{}%
\AgdaDottedPattern{\AgdaSymbol{.}}\AgdaDottedPattern{\AgdaInductiveConstructor{zero}}\AgdaSpace{}%
\AgdaInductiveConstructor{refl}\AgdaSpace{}%
\AgdaSymbol{=}\AgdaSpace{}%
\AgdaInductiveConstructor{refl}\<%
\\
%
\>[2]\AgdaFunction{split-inj₁}\AgdaSpace{}%
\AgdaSymbol{\{}\AgdaInductiveConstructor{suc}\AgdaSpace{}%
\AgdaBound{m}\AgdaSymbol{\}}\AgdaSpace{}%
\AgdaSymbol{(}\AgdaInductiveConstructor{suc}\AgdaSpace{}%
\AgdaBound{i}\AgdaSymbol{)}\AgdaSpace{}%
\AgdaInductiveConstructor{zero}\AgdaSpace{}%
\AgdaBound{p}\AgdaSpace{}%
\AgdaKeyword{with}\AgdaSpace{}%
\AgdaFunction{splitAt}\AgdaSpace{}%
\AgdaBound{m}\AgdaSpace{}%
\AgdaBound{i}\AgdaSpace{}%
\AgdaSymbol{|}\AgdaSpace{}%
\AgdaFunction{inspect}\AgdaSpace{}%
\AgdaSymbol{(}\AgdaFunction{splitAt}\AgdaSpace{}%
\AgdaBound{m}\AgdaSymbol{)}\AgdaSpace{}%
\AgdaBound{i}\<%
\\
%
\>[2]\AgdaFunction{split-inj₁}\AgdaSpace{}%
\AgdaSymbol{\{}\AgdaInductiveConstructor{suc}\AgdaSpace{}%
\AgdaBound{m}\AgdaSymbol{\}}\AgdaSpace{}%
\AgdaSymbol{(}\AgdaInductiveConstructor{suc}\AgdaSpace{}%
\AgdaBound{i}\AgdaSymbol{)}\AgdaSpace{}%
\AgdaInductiveConstructor{zero}\AgdaSpace{}%
\AgdaSymbol{()}\AgdaSpace{}%
\AgdaSymbol{|}\AgdaSpace{}%
\AgdaInductiveConstructor{inj₁}\AgdaSpace{}%
\AgdaBound{x}\AgdaSpace{}%
\AgdaSymbol{|}\AgdaSpace{}%
\AgdaOperator{\AgdaInductiveConstructor{[}}\AgdaSpace{}%
\AgdaBound{r}\AgdaSpace{}%
\AgdaOperator{\AgdaInductiveConstructor{]}}\<%
\\
%
\>[2]\AgdaFunction{split-inj₁}\AgdaSpace{}%
\AgdaSymbol{\{}\AgdaInductiveConstructor{suc}\AgdaSpace{}%
\AgdaBound{m}\AgdaSymbol{\}}\AgdaSpace{}%
\AgdaSymbol{(}\AgdaInductiveConstructor{suc}\AgdaSpace{}%
\AgdaBound{i}\AgdaSymbol{)}\AgdaSpace{}%
\AgdaInductiveConstructor{zero}\AgdaSpace{}%
\AgdaSymbol{()}\AgdaSpace{}%
\AgdaSymbol{|}\AgdaSpace{}%
\AgdaInductiveConstructor{inj₂}\AgdaSpace{}%
\AgdaBound{y}\AgdaSpace{}%
\AgdaSymbol{|}\AgdaSpace{}%
\AgdaOperator{\AgdaInductiveConstructor{[}}\AgdaSpace{}%
\AgdaBound{r}\AgdaSpace{}%
\AgdaOperator{\AgdaInductiveConstructor{]}}\<%
\\
%
\>[2]\AgdaFunction{split-inj₁}\AgdaSpace{}%
\AgdaSymbol{\{}\AgdaInductiveConstructor{suc}\AgdaSpace{}%
\AgdaBound{m}\AgdaSymbol{\}}\AgdaSpace{}%
\AgdaSymbol{(}\AgdaInductiveConstructor{suc}\AgdaSpace{}%
\AgdaBound{i}\AgdaSymbol{)}\AgdaSpace{}%
\AgdaSymbol{(}\AgdaInductiveConstructor{suc}\AgdaSpace{}%
\AgdaBound{k}\AgdaSymbol{)}\AgdaSpace{}%
\AgdaBound{p}\AgdaSpace{}%
\AgdaKeyword{with}\AgdaSpace{}%
\AgdaFunction{splitAt}\AgdaSpace{}%
\AgdaBound{m}\AgdaSpace{}%
\AgdaBound{i}\AgdaSpace{}%
\AgdaSymbol{|}\AgdaSpace{}%
\AgdaFunction{inspect}\AgdaSpace{}%
\AgdaSymbol{(}\AgdaFunction{splitAt}\AgdaSpace{}%
\AgdaBound{m}\AgdaSymbol{)}\AgdaSpace{}%
\AgdaBound{i}\<%
\\
%
\>[2]\AgdaFunction{split-inj₁}\AgdaSpace{}%
\AgdaSymbol{\{}\AgdaInductiveConstructor{suc}\AgdaSpace{}%
\AgdaBound{m}\AgdaSymbol{\}}\AgdaSpace{}%
\AgdaSymbol{(}\AgdaInductiveConstructor{suc}\AgdaSpace{}%
\AgdaBound{i}\AgdaSymbol{)}\AgdaSpace{}%
\AgdaSymbol{(}\AgdaInductiveConstructor{suc}\AgdaSpace{}%
\AgdaDottedPattern{\AgdaSymbol{.}}\AgdaDottedPattern{\AgdaBound{x}}\AgdaSymbol{)}\AgdaSpace{}%
\AgdaInductiveConstructor{refl}\AgdaSpace{}%
\AgdaSymbol{|}\AgdaSpace{}%
\AgdaInductiveConstructor{inj₁}\AgdaSpace{}%
\AgdaBound{x}\AgdaSpace{}%
\AgdaSymbol{|}\AgdaSpace{}%
\AgdaOperator{\AgdaInductiveConstructor{[}}\AgdaSpace{}%
\AgdaBound{r}\AgdaSpace{}%
\AgdaOperator{\AgdaInductiveConstructor{]}}\AgdaSpace{}%
\AgdaSymbol{=}\AgdaSpace{}%
\AgdaFunction{cong}\AgdaSpace{}%
\AgdaInductiveConstructor{suc}\AgdaSpace{}%
\AgdaSymbol{(}\AgdaFunction{split-inj₁}\AgdaSpace{}%
\AgdaBound{i}\AgdaSpace{}%
\AgdaBound{x}\AgdaSpace{}%
\AgdaBound{r}\AgdaSymbol{)}\<%
\\
\>[0]\<%
\\
%
\>[2]\AgdaFunction{inj₁₂}\AgdaSpace{}%
\AgdaSymbol{:}\AgdaSpace{}%
\AgdaSymbol{\{}\AgdaBound{A}\AgdaSpace{}%
\AgdaBound{B}\AgdaSpace{}%
\AgdaSymbol{:}\AgdaSpace{}%
\AgdaPrimitive{Set}\AgdaSymbol{\}\{}\AgdaBound{x}\AgdaSpace{}%
\AgdaSymbol{:}\AgdaSpace{}%
\AgdaBound{A}\AgdaSymbol{\}\{}\AgdaBound{y}\AgdaSpace{}%
\AgdaSymbol{:}\AgdaSpace{}%
\AgdaBound{B}\AgdaSymbol{\}}\AgdaSpace{}%
\AgdaSymbol{→}\AgdaSpace{}%
\AgdaInductiveConstructor{inj₁}\AgdaSpace{}%
\AgdaBound{x}\AgdaSpace{}%
\AgdaOperator{\AgdaDatatype{≡}}\AgdaSpace{}%
\AgdaInductiveConstructor{inj₂}\AgdaSpace{}%
\AgdaBound{y}\AgdaSpace{}%
\AgdaSymbol{→}\AgdaSpace{}%
\AgdaFunction{⊥}\<%
\\
%
\>[2]\AgdaFunction{inj₁₂}\AgdaSpace{}%
\AgdaSymbol{()}\<%
\end{code}
\begin{mathpar}
\codeblock{\begin{code}%
%
\>[2]\AgdaOperator{\AgdaFunction{\AgdaUnderscore{}⊕\AgdaUnderscore{}}}\AgdaSpace{}%
\AgdaSymbol{:}\AgdaSpace{}%
\AgdaDatatype{Fin}\AgdaSpace{}%
\AgdaGeneralizable{m}\AgdaSpace{}%
\AgdaSymbol{→}\AgdaSpace{}%
\AgdaDatatype{Fin}\AgdaSpace{}%
\AgdaSymbol{(}\AgdaNumber{1}\AgdaSpace{}%
\AgdaOperator{\AgdaPrimitive{+}}\AgdaSpace{}%
\AgdaGeneralizable{n}\AgdaSymbol{)}\AgdaSpace{}%
\AgdaSymbol{→}\AgdaSpace{}%
\AgdaDatatype{Fin}\AgdaSpace{}%
\AgdaSymbol{(}\AgdaGeneralizable{m}\AgdaSpace{}%
\AgdaOperator{\AgdaPrimitive{+}}\AgdaSpace{}%
\AgdaGeneralizable{n}\AgdaSymbol{)}\<%
\\
%
\>[2]\AgdaInductiveConstructor{zero}%
\>[9]\AgdaOperator{\AgdaFunction{⊕}}\AgdaSpace{}%
\AgdaBound{j}\AgdaSpace{}%
\AgdaSymbol{=}\AgdaSpace{}%
\AgdaFunction{inject-left}\AgdaSpace{}%
\AgdaBound{j}\<%
\\
%
\>[2]\AgdaInductiveConstructor{suc}\AgdaSpace{}%
\AgdaBound{i}%
\>[9]\AgdaOperator{\AgdaFunction{⊕}}\AgdaSpace{}%
\AgdaBound{j}\AgdaSpace{}%
\AgdaSymbol{=}\AgdaSpace{}%
\AgdaInductiveConstructor{suc}\AgdaSpace{}%
\AgdaSymbol{(}\AgdaBound{i}\AgdaSpace{}%
\AgdaOperator{\AgdaFunction{⊕}}\AgdaSpace{}%
\AgdaBound{j}\AgdaSymbol{)}\<%
\end{code}}
\and
\codeblock{\begin{code}%
%
\>[2]\AgdaOperator{\AgdaFunction{\AgdaUnderscore{}⊝\AgdaUnderscore{}}}%
\>[665I]\AgdaSymbol{:}\AgdaSpace{}%
\AgdaSymbol{(}\AgdaBound{i}\AgdaSpace{}%
\AgdaSymbol{:}\AgdaSpace{}%
\AgdaDatatype{Fin}\AgdaSpace{}%
\AgdaSymbol{(}\AgdaGeneralizable{m}\AgdaSpace{}%
\AgdaOperator{\AgdaPrimitive{+}}\AgdaSpace{}%
\AgdaGeneralizable{n}\AgdaSymbol{))}\AgdaSpace{}%
\AgdaSymbol{(}\AgdaBound{j}\AgdaSpace{}%
\AgdaSymbol{:}\AgdaSpace{}%
\AgdaDatatype{Fin}\AgdaSpace{}%
\AgdaGeneralizable{m}\AgdaSymbol{)}\<%
\\
\>[.][@{}l@{}]\<[665I]%
\>[6]\AgdaSymbol{→}\AgdaSpace{}%
\AgdaRecord{Dec}\AgdaSpace{}%
\AgdaSymbol{(}\AgdaFunction{∃}\AgdaSpace{}%
\AgdaSymbol{λ}\AgdaSpace{}%
\AgdaBound{k}\AgdaSpace{}%
\AgdaSymbol{→}\AgdaSpace{}%
\AgdaBound{j}\AgdaSpace{}%
\AgdaOperator{\AgdaFunction{⊕}}\AgdaSpace{}%
\AgdaBound{k}\AgdaSpace{}%
\AgdaOperator{\AgdaDatatype{≡}}\AgdaSpace{}%
\AgdaBound{i}\AgdaSymbol{)}\<%
\end{code}}
\end{mathpar}
\begin{code}[hide]%
%
\>[2]\AgdaOperator{\AgdaFunction{\AgdaUnderscore{}⊝\AgdaUnderscore{}}}\AgdaSpace{}%
\AgdaSymbol{\{}\AgdaInductiveConstructor{suc}\AgdaSpace{}%
\AgdaBound{m}\AgdaSymbol{\}}\AgdaSpace{}%
\AgdaSymbol{\{}\AgdaBound{n}\AgdaSymbol{\}}\AgdaSpace{}%
\AgdaBound{i}\AgdaSpace{}%
\AgdaInductiveConstructor{zero}\AgdaSpace{}%
\AgdaKeyword{rewrite}\AgdaSpace{}%
\AgdaFunction{+-comm}\AgdaSpace{}%
\AgdaBound{m}\AgdaSpace{}%
\AgdaBound{n}\AgdaSpace{}%
\AgdaKeyword{with}\AgdaSpace{}%
\AgdaFunction{splitAt}\AgdaSpace{}%
\AgdaSymbol{(}\AgdaInductiveConstructor{suc}\AgdaSpace{}%
\AgdaBound{n}\AgdaSymbol{)}\AgdaSpace{}%
\AgdaBound{i}\AgdaSpace{}%
\AgdaSymbol{|}\AgdaSpace{}%
\AgdaFunction{inspect}\AgdaSpace{}%
\AgdaSymbol{(}\AgdaFunction{splitAt}\AgdaSpace{}%
\AgdaSymbol{(}\AgdaInductiveConstructor{suc}\AgdaSpace{}%
\AgdaBound{n}\AgdaSymbol{))}\AgdaSpace{}%
\AgdaBound{i}\<%
\\
%
\>[2]\AgdaSymbol{...}\AgdaSpace{}%
\AgdaSymbol{|}\AgdaSpace{}%
\AgdaInductiveConstructor{inj₁}\AgdaSpace{}%
\AgdaBound{k}\AgdaSpace{}%
\AgdaSymbol{|}\AgdaSpace{}%
\AgdaOperator{\AgdaInductiveConstructor{[}}\AgdaSpace{}%
\AgdaBound{r}\AgdaSpace{}%
\AgdaOperator{\AgdaInductiveConstructor{]}}\AgdaSpace{}%
\AgdaSymbol{=}\AgdaSpace{}%
\AgdaInductiveConstructor{yes}\AgdaSpace{}%
\AgdaSymbol{(}\AgdaBound{k}\AgdaSpace{}%
\AgdaOperator{\AgdaInductiveConstructor{,}}\AgdaSpace{}%
\AgdaFunction{split-inj₁}\AgdaSpace{}%
\AgdaBound{i}\AgdaSpace{}%
\AgdaBound{k}\AgdaSpace{}%
\AgdaBound{r}\AgdaSymbol{)}\<%
\\
%
\>[2]\AgdaSymbol{...}\AgdaSpace{}%
\AgdaSymbol{|}\AgdaSpace{}%
\AgdaInductiveConstructor{inj₂}\AgdaSpace{}%
\AgdaBound{k}\AgdaSpace{}%
\AgdaSymbol{|}\AgdaSpace{}%
\AgdaOperator{\AgdaInductiveConstructor{[}}\AgdaSpace{}%
\AgdaBound{r}\AgdaSpace{}%
\AgdaOperator{\AgdaInductiveConstructor{]}}\AgdaSpace{}%
\AgdaSymbol{=}\AgdaSpace{}%
\AgdaInductiveConstructor{no}\AgdaSpace{}%
\AgdaFunction{reason}\<%
\\
\>[2][@{}l@{\AgdaIndent{0}}]%
\>[4]\AgdaKeyword{where}\<%
\\
\>[4][@{}l@{\AgdaIndent{0}}]%
\>[6]\AgdaFunction{reason}\AgdaSpace{}%
\AgdaSymbol{:}\AgdaSpace{}%
\AgdaSymbol{\AgdaUnderscore{}}\<%
\\
%
\>[6]\AgdaFunction{reason}\AgdaSpace{}%
\AgdaSymbol{(}\AgdaBound{k}\AgdaSpace{}%
\AgdaOperator{\AgdaInductiveConstructor{,}}\AgdaSpace{}%
\AgdaInductiveConstructor{refl}\AgdaSymbol{)}\AgdaSpace{}%
\AgdaKeyword{rewrite}\AgdaSpace{}%
\AgdaFunction{splitAt-inject+}\AgdaSpace{}%
\AgdaSymbol{(}\AgdaInductiveConstructor{suc}\AgdaSpace{}%
\AgdaBound{n}\AgdaSymbol{)}\AgdaSpace{}%
\AgdaBound{m}\AgdaSpace{}%
\AgdaBound{k}\AgdaSpace{}%
\AgdaSymbol{=}\AgdaSpace{}%
\AgdaFunction{inj₁₂}\AgdaSpace{}%
\AgdaBound{r}\<%
\\
%
\>[2]\AgdaInductiveConstructor{zero}\AgdaSpace{}%
\AgdaOperator{\AgdaFunction{⊝}}\AgdaSpace{}%
\AgdaInductiveConstructor{suc}\AgdaSpace{}%
\AgdaBound{j}\AgdaSpace{}%
\AgdaSymbol{=}\AgdaSpace{}%
\AgdaInductiveConstructor{no}\AgdaSpace{}%
\AgdaSymbol{λ}\AgdaSpace{}%
\AgdaSymbol{\{}\AgdaSpace{}%
\AgdaSymbol{(}\AgdaBound{k}\AgdaSpace{}%
\AgdaOperator{\AgdaInductiveConstructor{,}}\AgdaSpace{}%
\AgdaSymbol{())}\AgdaSpace{}%
\AgdaSymbol{\}}\<%
\\
%
\>[2]\AgdaInductiveConstructor{suc}\AgdaSpace{}%
\AgdaBound{i}\AgdaSpace{}%
\AgdaOperator{\AgdaFunction{⊝}}\AgdaSpace{}%
\AgdaInductiveConstructor{suc}\AgdaSpace{}%
\AgdaBound{j}\AgdaSpace{}%
\AgdaKeyword{with}\AgdaSpace{}%
\AgdaBound{i}\AgdaSpace{}%
\AgdaOperator{\AgdaFunction{⊝}}\AgdaSpace{}%
\AgdaBound{j}\<%
\\
%
\>[2]\AgdaSymbol{...}\AgdaSpace{}%
\AgdaSymbol{|}\AgdaSpace{}%
\AgdaInductiveConstructor{yes}\AgdaSpace{}%
\AgdaSymbol{(}\AgdaBound{k}\AgdaSpace{}%
\AgdaOperator{\AgdaInductiveConstructor{,}}\AgdaSpace{}%
\AgdaBound{p}\AgdaSymbol{)}\AgdaSpace{}%
\AgdaSymbol{=}\AgdaSpace{}%
\AgdaInductiveConstructor{yes}\AgdaSpace{}%
\AgdaSymbol{(}\AgdaBound{k}\AgdaSpace{}%
\AgdaOperator{\AgdaInductiveConstructor{,}}\AgdaSpace{}%
\AgdaFunction{cong}\AgdaSpace{}%
\AgdaInductiveConstructor{suc}\AgdaSpace{}%
\AgdaBound{p}\AgdaSymbol{)}\<%
\\
%
\>[2]\AgdaSymbol{...}\AgdaSpace{}%
\AgdaSymbol{|}\AgdaSpace{}%
\AgdaInductiveConstructor{no}\AgdaSpace{}%
\AgdaBound{¬p}\AgdaSpace{}%
\AgdaSymbol{=}\AgdaSpace{}%
\AgdaInductiveConstructor{no}\AgdaSpace{}%
\AgdaSymbol{λ}\AgdaSpace{}%
\AgdaSymbol{\{}\AgdaSpace{}%
\AgdaSymbol{(}\AgdaBound{k}\AgdaSpace{}%
\AgdaOperator{\AgdaInductiveConstructor{,}}\AgdaSpace{}%
\AgdaBound{p}\AgdaSymbol{)}\AgdaSpace{}%
\AgdaSymbol{→}\AgdaSpace{}%
\AgdaBound{¬p}\AgdaSpace{}%
\AgdaSymbol{(}\AgdaBound{k}\AgdaSpace{}%
\AgdaOperator{\AgdaInductiveConstructor{,}}\AgdaSpace{}%
\AgdaFunction{suc-injective}\AgdaSpace{}%
\AgdaBound{p}\AgdaSymbol{)}\AgdaSpace{}%
\AgdaSymbol{\}}\<%
\\
%
\\[\AgdaEmptyExtraSkip]%
%
\>[2]\AgdaFunction{inject-left-zero}\AgdaSpace{}%
\AgdaSymbol{:}\AgdaSpace{}%
\AgdaFunction{inject-left}\AgdaSpace{}%
\AgdaSymbol{\{}\AgdaGeneralizable{m}\AgdaSymbol{\}}\AgdaSpace{}%
\AgdaSymbol{\{}\AgdaGeneralizable{n}\AgdaSymbol{\}}\AgdaSpace{}%
\AgdaInductiveConstructor{zero}\AgdaSpace{}%
\AgdaOperator{\AgdaDatatype{≡}}\AgdaSpace{}%
\AgdaInductiveConstructor{zero}\<%
\\
%
\>[2]\AgdaFunction{inject-left-zero}\AgdaSpace{}%
\AgdaSymbol{\{}\AgdaBound{m}\AgdaSymbol{\}}\AgdaSpace{}%
\AgdaSymbol{\{}\AgdaBound{n}\AgdaSymbol{\}}\AgdaSpace{}%
\AgdaKeyword{rewrite}\AgdaSpace{}%
\AgdaFunction{+-comm}\AgdaSpace{}%
\AgdaBound{n}\AgdaSpace{}%
\AgdaBound{m}\AgdaSpace{}%
\AgdaSymbol{=}\AgdaSpace{}%
\AgdaInductiveConstructor{refl}\<%
\\
%
\\[\AgdaEmptyExtraSkip]%
%
\>[2]\AgdaFunction{suc-not-zero}\AgdaSpace{}%
\AgdaSymbol{:}\AgdaSpace{}%
\AgdaSymbol{\{}\AgdaBound{i}\AgdaSpace{}%
\AgdaSymbol{:}\AgdaSpace{}%
\AgdaDatatype{Fin}\AgdaSpace{}%
\AgdaGeneralizable{m}\AgdaSymbol{\}}\AgdaSpace{}%
\AgdaSymbol{→}\AgdaSpace{}%
\AgdaOperator{\AgdaDatatype{\AgdaUnderscore{}≡\AgdaUnderscore{}}}\AgdaSpace{}%
\AgdaSymbol{\{}\AgdaArgument{A}\AgdaSpace{}%
\AgdaSymbol{=}\AgdaSpace{}%
\AgdaDatatype{Fin}\AgdaSpace{}%
\AgdaSymbol{(}\AgdaInductiveConstructor{suc}\AgdaSpace{}%
\AgdaGeneralizable{m}\AgdaSymbol{)\}}\AgdaSpace{}%
\AgdaSymbol{(}\AgdaInductiveConstructor{suc}\AgdaSpace{}%
\AgdaBound{i}\AgdaSymbol{)}\AgdaSpace{}%
\AgdaInductiveConstructor{zero}\AgdaSpace{}%
\AgdaSymbol{→}\AgdaSpace{}%
\AgdaFunction{⊥}\<%
\\
%
\>[2]\AgdaFunction{suc-not-zero}\AgdaSpace{}%
\AgdaSymbol{()}\<%
\\
%
\\[\AgdaEmptyExtraSkip]%
%
\>[2]\AgdaFunction{inject-left-suc}\AgdaSpace{}%
\AgdaSymbol{:}\AgdaSpace{}%
\AgdaSymbol{∀}\AgdaSpace{}%
\AgdaSymbol{(}\AgdaBound{i}\AgdaSpace{}%
\AgdaSymbol{:}\AgdaSpace{}%
\AgdaDatatype{Fin}\AgdaSpace{}%
\AgdaGeneralizable{m}\AgdaSymbol{)}\AgdaSpace{}%
\AgdaSymbol{→}\AgdaSpace{}%
\AgdaFunction{inject-left}\AgdaSpace{}%
\AgdaSymbol{\{}\AgdaGeneralizable{m}\AgdaSymbol{\}}\AgdaSpace{}%
\AgdaSymbol{\{}\AgdaGeneralizable{n}\AgdaSymbol{\}}\AgdaSpace{}%
\AgdaSymbol{(}\AgdaInductiveConstructor{suc}\AgdaSpace{}%
\AgdaBound{i}\AgdaSymbol{)}\AgdaSpace{}%
\AgdaOperator{\AgdaDatatype{≡}}\AgdaSpace{}%
\AgdaInductiveConstructor{zero}\AgdaSpace{}%
\AgdaSymbol{→}\AgdaSpace{}%
\AgdaFunction{⊥}\<%
\\
%
\>[2]\AgdaFunction{inject-left-suc}\AgdaSpace{}%
\AgdaSymbol{\{}\AgdaBound{m}\AgdaSymbol{\}}\AgdaSpace{}%
\AgdaSymbol{\{}\AgdaBound{n}\AgdaSymbol{\}}\AgdaSpace{}%
\AgdaBound{i}\AgdaSpace{}%
\AgdaBound{p}\AgdaSpace{}%
\AgdaKeyword{rewrite}\AgdaSpace{}%
\AgdaFunction{+-comm}\AgdaSpace{}%
\AgdaBound{n}\AgdaSpace{}%
\AgdaBound{m}\AgdaSpace{}%
\AgdaSymbol{=}\AgdaSpace{}%
\AgdaFunction{suc-not-zero}\AgdaSpace{}%
\AgdaBound{p}\<%
\\
%
\\[\AgdaEmptyExtraSkip]%
%
\>[2]\AgdaFunction{zero-suc-⊥}\AgdaSpace{}%
\AgdaSymbol{:}\AgdaSpace{}%
\AgdaSymbol{∀}\AgdaSpace{}%
\AgdaSymbol{\{}\AgdaBound{i}\AgdaSpace{}%
\AgdaSymbol{:}\AgdaSpace{}%
\AgdaDatatype{Fin}\AgdaSpace{}%
\AgdaGeneralizable{n}\AgdaSymbol{\}}\AgdaSpace{}%
\AgdaSymbol{→}\AgdaSpace{}%
\AgdaOperator{\AgdaDatatype{\AgdaUnderscore{}≡\AgdaUnderscore{}}}\AgdaSpace{}%
\AgdaSymbol{\{}\AgdaArgument{A}\AgdaSpace{}%
\AgdaSymbol{=}\AgdaSpace{}%
\AgdaDatatype{Fin}\AgdaSpace{}%
\AgdaSymbol{(}\AgdaInductiveConstructor{suc}\AgdaSpace{}%
\AgdaGeneralizable{n}\AgdaSymbol{)\}}\AgdaSpace{}%
\AgdaInductiveConstructor{zero}\AgdaSpace{}%
\AgdaSymbol{(}\AgdaInductiveConstructor{suc}\AgdaSpace{}%
\AgdaBound{i}\AgdaSymbol{)}\AgdaSpace{}%
\AgdaSymbol{→}\AgdaSpace{}%
\AgdaFunction{⊥}\<%
\\
%
\>[2]\AgdaFunction{zero-suc-⊥}\AgdaSpace{}%
\AgdaSymbol{()}\<%
\\
%
\\[\AgdaEmptyExtraSkip]%
%
\>[2]\AgdaComment{--\ TODO:\ this\ is\ annoying\ to\ do\ inductively\ on\ Fin,\ it\ is\ easier\ to}\<%
\\
%
\>[2]\AgdaComment{--\ \ \ \ \ \ \ implement\ this\ via\ Fin\ n\ =\ Σ\ ℕ\ (\AgdaUnderscore{}<\ n)\ representation}\<%
\\
%
\>[2]\AgdaComment{--\ minusx\ :\ (i\ :\ Fin\ (m\ +\ n))\ →\ (j\ :\ Fin\ (suc\ n))\ →\ Dec\ (∃\ λ\ k\ →\ k\ ⊕\ j\ ≡\ i)}\<%
\\
%
\>[2]\AgdaComment{--\ minusx\ \{zero\}\ i\ zero\ =\ no\ λ\ \{\ (()\ ,\ \AgdaUnderscore{})\ \}}\<%
\\
%
\>[2]\AgdaComment{--\ minusx\ \{suc\ m\}\ \{n\}\ zero\ zero\ =\ yes\ (zero\ ,\ inject-left-zero\ \{n\}\ \{m\})}\<%
\\
%
\>[2]\AgdaComment{--\ minusx\ \{suc\ m\}\ \{n\}\ (suc\ i)\ zero\ with\ minusx\ \{m\}\ i\ zero}\<%
\\
%
\>[2]\AgdaComment{--\ ...\ |\ yes\ (j\ ,\ p)\ =\ yes\ (suc\ j\ ,\ cong\ suc\ p)}\<%
\\
%
\>[2]\AgdaComment{--\ ...\ |\ no\ ¬p\ =\ no\ λ\ \{\ (zero\ ,\ p)\ →\ let\ rr\ =\ trans\ (sym\ \$\ inject-left-zero\ \{n\}\ \{m\})\ p\ }\<%
\\
%
\>[2]\AgdaComment{--\ \ \ \ \ \ \ \ \ \ \ \ \ \ \ \ \ \ \ \ \ \ \ \ \ \ \ \ \ \ \ \ \ \ \ in\ zero-suc-⊥\ rr}\<%
\\
%
\>[2]\AgdaComment{--\ \ \ \ \ \ \ \ \ \ \ \ \ \ \ \ \ \ \ \ ;\ (suc\ j\ ,\ p)\ →\ ¬p\ (j\ ,\ suc-injective\ p)\ \}}\<%
\\
%
\\[\AgdaEmptyExtraSkip]%
%
\>[2]\AgdaComment{--\ minusx\ \{zero\}\ i\ (suc\ j)\ =\ no\ λ\ \{\ (()\ ,\ p)\ \}}\<%
\\
%
\>[2]\AgdaComment{--\ minusx\ \{suc\ m\}\ zero\ (suc\ j)\ =\ no\ λ\ \{\ (zero\ ,\ p)\ →\ inject-left-suc\ j\ p}\<%
\\
%
\>[2]\AgdaComment{--\ \ \ \ \ \ \ \ \ \ \ \ \ \ \ \ \ \ \ \ \ \ \ \ \ \ \ \ \ \ \ \ \ \ \ \ ;\ (suc\ k\ ,\ ())\ \}}\<%
\\
%
\>[2]\AgdaComment{--\ minusx\ \{suc\ m\}\ \{suc\ n\}\ (suc\ i)\ (suc\ j)\ =\ ?\ }\<%
\end{code}
Recall that the type \AF{Fin} $n$ is a type for natural numbers $i$ that
are bounded by $n$ (\ie{} $i < n$).  Plus adds two bounded indices $i$ and $j$
where $i < m$ and $j < 1 + n$ (both $i$ and $j$ are non-negative as any
element of \AF{Fin}).
The indices $i$ and $j$ are added as natural numbers, so there is
no easy way to apply type isomorphisms such as \AD{Fin} $(m + n)$ $\cong$
\AD{Fin} $m$ $⊎$ \AD{Fin} n.  Minus is a partial inverse of plus described below.

While both definitions look innocent, their types carry non-trivial
information about the bounds.  Consider the bounds in the \AF{\_⊕\_} operation:
\begin{mathpar}
  \inferrule*
    {i < m \and j < 1 + n}
    {i+j < m + n}
\end{mathpar}
This looks a little surprising, but this indeed holds for natural numbers.
Readers may convince themselves by considering the maximum value that $i$ and $j$
can possibly take.  The \AF{\_⊕\_} operation have partial inverses making it possible
to define left and right subtraction.  We consider left subtraction \AF{\_⊝\_}.
Its type says that there exists a decision procedure for finding $k$ of type
\AF{Fin} (1 + \AB{n}) (\eg{} $k < 1 + n$) together with the proof that $k$ is
an inverse of \AF{⊕}.
In some sense \AF{Dec} is similar to \AF{Maybe} type, except it forces one
to prove why the value does not exist as opposed to just returning \AC{nothing}.
For example, if we were to evaluate $i ⊝ j$ where $i = 1 < 3 + 5$ and $j = 2 < 3$,
we will get a proof that there is no natural number $k < 1 + 5$ such that $2 ⊕ k ≡ 1$.
Here dependent types come very useful, as we eliminate the possibility of
introducing off-by-one errors in the definition of \AF{⊝}.


Now we are ready to define a 1-dimensional convolution.
A side note for mathematically inclined readers: we use the term
\emph{convolution} in the way it is used in machine learning.  Technically,
we compute a cross-correlation, because the array of weights is not flipped.
However, in practice this is not a problem, as we assume that weights are
stored flipped in memory.

We define type synonyms \AF{Vec} and \AF{Ix} which are 1-dimensional versions
of \AF{Ar} and \AF{P}.
\begin{mathpar}
\codeblock{\begin{code}%
%
\>[2]\AgdaFunction{Vec}\AgdaSpace{}%
\AgdaSymbol{:}\AgdaSpace{}%
\AgdaDatatype{ℕ}\AgdaSpace{}%
\AgdaSymbol{→}\AgdaSpace{}%
\AgdaPrimitive{Set}\AgdaSpace{}%
\AgdaSymbol{→}\AgdaSpace{}%
\AgdaPrimitive{Set}\<%
\\
%
\>[2]\AgdaFunction{Vec}\AgdaSpace{}%
\AgdaBound{m}\AgdaSpace{}%
\AgdaBound{X}\AgdaSpace{}%
\AgdaSymbol{=}\AgdaSpace{}%
\AgdaFunction{Ar}\AgdaSpace{}%
\AgdaSymbol{(}\AgdaInductiveConstructor{ι}\AgdaSpace{}%
\AgdaBound{m}\AgdaSymbol{)}\AgdaSpace{}%
\AgdaBound{X}\<%
\end{code}}
\and
\codeblock{\begin{code}%
%
\>[2]\AgdaFunction{Ix}\AgdaSpace{}%
\AgdaSymbol{:}\AgdaSpace{}%
\AgdaDatatype{ℕ}\AgdaSpace{}%
\AgdaSymbol{→}\AgdaSpace{}%
\AgdaPrimitive{Set}\<%
\\
%
\>[2]\AgdaFunction{Ix}\AgdaSpace{}%
\AgdaBound{m}\AgdaSpace{}%
\AgdaSymbol{=}\AgdaSpace{}%
\AgdaDatatype{P}\AgdaSpace{}%
\AgdaSymbol{(}\AgdaInductiveConstructor{ι}\AgdaSpace{}%
\AgdaBound{m}\AgdaSymbol{)}\<%
\end{code}}
\end{mathpar}
We introduce the \AF{slide₁} primitive that selects a $(1+n)$-element vector
from the $(m+n)$-element vector starting at the offset $i$.  Then,
following~\cite{cnn-array}, we compute $m$-element array of slides
and then sum it up.
\begin{mathpar}
\codeblock{\begin{code}%
%
\>[2]\AgdaFunction{slide₁}\AgdaSpace{}%
\AgdaSymbol{:}\AgdaSpace{}%
\AgdaFunction{Ix}\AgdaSpace{}%
\AgdaGeneralizable{m}\AgdaSpace{}%
\AgdaSymbol{→}\AgdaSpace{}%
\AgdaFunction{Vec}\AgdaSpace{}%
\AgdaSymbol{(}\AgdaGeneralizable{m}\AgdaSpace{}%
\AgdaOperator{\AgdaPrimitive{+}}\AgdaSpace{}%
\AgdaGeneralizable{n}\AgdaSymbol{)}\AgdaSpace{}%
\AgdaGeneralizable{X}\AgdaSpace{}%
\AgdaSymbol{→}\AgdaSpace{}%
\AgdaFunction{Vec}\AgdaSpace{}%
\AgdaSymbol{(}\AgdaNumber{1}\AgdaSpace{}%
\AgdaOperator{\AgdaPrimitive{+}}\AgdaSpace{}%
\AgdaGeneralizable{n}\AgdaSymbol{)}\AgdaSpace{}%
\AgdaGeneralizable{X}\<%
\\
%
\>[2]\AgdaFunction{slide₁}\AgdaSpace{}%
\AgdaSymbol{(}\AgdaInductiveConstructor{ι}\AgdaSpace{}%
\AgdaBound{i}\AgdaSymbol{)}\AgdaSpace{}%
\AgdaBound{v}\AgdaSpace{}%
\AgdaSymbol{(}\AgdaInductiveConstructor{ι}\AgdaSpace{}%
\AgdaBound{j}\AgdaSymbol{)}\AgdaSpace{}%
\AgdaSymbol{=}\AgdaSpace{}%
\AgdaBound{v}\AgdaSpace{}%
\AgdaSymbol{(}\AgdaInductiveConstructor{ι}\AgdaSpace{}%
\AgdaSymbol{(}\AgdaBound{i}\AgdaSpace{}%
\AgdaOperator{\AgdaFunction{⊕}}\AgdaSpace{}%
\AgdaBound{j}\AgdaSymbol{))}\<%
\\
%
\\[\AgdaEmptyExtraSkip]%
%
\>[2]\AgdaFunction{conv₁}\AgdaSpace{}%
\AgdaSymbol{:}\AgdaSpace{}%
\AgdaFunction{Vec}\AgdaSpace{}%
\AgdaSymbol{(}\AgdaGeneralizable{m}\AgdaSpace{}%
\AgdaOperator{\AgdaPrimitive{+}}\AgdaSpace{}%
\AgdaGeneralizable{n}\AgdaSymbol{)}\AgdaSpace{}%
\AgdaDatatype{ℕ}\AgdaSpace{}%
\AgdaSymbol{→}\AgdaSpace{}%
\AgdaFunction{Vec}\AgdaSpace{}%
\AgdaGeneralizable{m}\AgdaSpace{}%
\AgdaDatatype{ℕ}\AgdaSpace{}%
\AgdaSymbol{→}\AgdaSpace{}%
\AgdaFunction{Vec}\AgdaSpace{}%
\AgdaSymbol{(}\AgdaNumber{1}\AgdaSpace{}%
\AgdaOperator{\AgdaPrimitive{+}}\AgdaSpace{}%
\AgdaGeneralizable{n}\AgdaSymbol{)}\AgdaSpace{}%
\AgdaDatatype{ℕ}\<%
\\
%
\>[2]\AgdaFunction{conv₁}\AgdaSpace{}%
\AgdaBound{a}\AgdaSpace{}%
\AgdaBound{w}\AgdaSpace{}%
\AgdaSymbol{=}\AgdaSpace{}%
\AgdaFunction{sum}\AgdaSpace{}%
\AgdaSymbol{(}\AgdaFunction{zipWith}\AgdaSpace{}%
\AgdaOperator{\AgdaPrimitive{\AgdaUnderscore{}+\AgdaUnderscore{}}}\AgdaSymbol{)}\AgdaSpace{}%
\AgdaSymbol{(}\AgdaFunction{K}\AgdaSpace{}%
\AgdaNumber{0}\AgdaSymbol{)}\AgdaSpace{}%
\AgdaSymbol{(λ}\AgdaSpace{}%
\AgdaBound{i}\AgdaSpace{}%
\AgdaSymbol{→}\AgdaSpace{}%
\AgdaFunction{map}\AgdaSpace{}%
\AgdaSymbol{(}\AgdaBound{w}\AgdaSpace{}%
\AgdaBound{i}\AgdaSpace{}%
\AgdaOperator{\AgdaPrimitive{*\AgdaUnderscore{}}}\AgdaSymbol{)}\AgdaSpace{}%
\AgdaSymbol{(}\AgdaFunction{slide₁}\AgdaSpace{}%
\AgdaBound{i}\AgdaSpace{}%
\AgdaBound{a}\AgdaSymbol{))}\<%
\end{code}}
\end{mathpar}
Note that in the definition of \AF{conv₁} we use a standard array language
trick --- we pull summation to the outside.  For example, for $m = 3$, $n = 2$,
a straight-forward way to compute (\AF{conv₁} $[a_1, a_2, a_3, a_4, a_5]$
$[w_1, w_2, w_3]$) would be $[a_1w_1 + a_2w_2 + a_3w_3, a_2w_1 + a_3w_2 +
a_4w_3,\dots]$.  However, the above definition proceeds as $w_1[a_1,a_2,a_3] +
w_2[a_2,a_3,a_4] + w_3[a_3,a_4,a_5]$ which computes the same result.  Such
definition makes it easy to replace the implementation of slide, obtaining
other versions of convolution such as the one with constant or cyclic
boundaries.  As we demonstrate in the next section, this pattern generalises
nicely to higher ranks.



\subsection{Generalisation\label{sec:general-ix-ops}}
Now we generalise 1-dimensional slide for arrays of higher ranks.
This requires generalising vector shapes $m + n$ and $1 + n$ for the cases
when $m$ and $n$ for arbitrary shapes.  In case of addition, we need a witness
that both shapes
have the same length.  If they do, their components are added point-wise.
We define a three-way relation \AF{\_+\_≈\_} that combines the witness and
the action.  That is, the type \AB{p} \AF{+} \AB{q} \AF{≈} \AB{r} says that
$p$ and $q$ have the same length and that $r$ is a point-wise addition
of $p$ and $q$.  A similar relation \AF{suc\_≈\_} is introduced for $1 + n$
case, and \AF{\_*\_≈\_} witnesses point-wise
multiplication that will be needed for blocking.  We define these relations
in two steps.  Firstly, we give a generalised pointwise relations for binary
and ternary relations on natural numbers:
\begin{mathpar}
\codeblock{\begin{code}%
%
\>[2]\AgdaKeyword{data}%
\>[954I]\AgdaDatatype{Pw₂}\AgdaSpace{}%
\AgdaSymbol{(}\AgdaBound{R}\AgdaSpace{}%
\AgdaSymbol{:}\AgdaSpace{}%
\AgdaSymbol{(}\AgdaBound{a}\AgdaSpace{}%
\AgdaBound{b}\AgdaSpace{}%
\AgdaSymbol{:}\AgdaSpace{}%
\AgdaDatatype{ℕ}\AgdaSymbol{)}\AgdaSpace{}%
\AgdaSymbol{→}\AgdaSpace{}%
\AgdaPrimitive{Set}\AgdaSymbol{)}\<%
\\
\>[.][@{}l@{}]\<[954I]%
\>[7]\AgdaSymbol{:}\AgdaSpace{}%
\AgdaSymbol{(}\AgdaBound{a}\AgdaSpace{}%
\AgdaBound{b}\AgdaSpace{}%
\AgdaSymbol{:}\AgdaSpace{}%
\AgdaDatatype{S}\AgdaSymbol{)}\AgdaSpace{}%
\AgdaSymbol{→}\AgdaSpace{}%
\AgdaPrimitive{Set}\AgdaSpace{}%
\AgdaKeyword{where}\AgdaSpace{}%
\AgdaKeyword{instance}\<%
\\
\>[2][@{}l@{\AgdaIndent{0}}]%
\>[6]\AgdaInductiveConstructor{[]}%
\>[12]\AgdaSymbol{:}\AgdaSpace{}%
\AgdaDatatype{Pw₂}\AgdaSpace{}%
\AgdaBound{R}\AgdaSpace{}%
\AgdaInductiveConstructor{[]}\AgdaSpace{}%
\AgdaInductiveConstructor{[]}\<%
\\
%
\>[6]\AgdaInductiveConstructor{cons}%
\>[12]\AgdaSymbol{:}\AgdaSpace{}%
\AgdaSymbol{⦃}\AgdaSpace{}%
\AgdaBound{R}\AgdaSpace{}%
\AgdaGeneralizable{m}\AgdaSpace{}%
\AgdaGeneralizable{n}\AgdaSpace{}%
\AgdaSymbol{⦄}\AgdaSpace{}%
\AgdaSymbol{→}\AgdaSpace{}%
\AgdaSymbol{⦃}\AgdaSpace{}%
\AgdaDatatype{Pw₂}\AgdaSpace{}%
\AgdaBound{R}\AgdaSpace{}%
\AgdaGeneralizable{s}\AgdaSpace{}%
\AgdaGeneralizable{p}\AgdaSpace{}%
\AgdaSymbol{⦄}\<%
\\
%
\>[12]\AgdaSymbol{→}\AgdaSpace{}%
\AgdaDatatype{Pw₂}\AgdaSpace{}%
\AgdaBound{R}\AgdaSpace{}%
\AgdaSymbol{(}\AgdaGeneralizable{m}\AgdaSpace{}%
\AgdaOperator{\AgdaInductiveConstructor{∷}}\AgdaSpace{}%
\AgdaGeneralizable{s}\AgdaSymbol{)}\AgdaSpace{}%
\AgdaSymbol{(}\AgdaGeneralizable{n}\AgdaSpace{}%
\AgdaOperator{\AgdaInductiveConstructor{∷}}\AgdaSpace{}%
\AgdaGeneralizable{p}\AgdaSymbol{)}\<%
\end{code}}
\and
\codeblock{\begin{code}%
%
\>[2]\AgdaKeyword{data}%
\>[995I]\AgdaDatatype{Pw₃}\AgdaSpace{}%
\AgdaSymbol{(}\AgdaBound{R}\AgdaSpace{}%
\AgdaSymbol{:}\AgdaSpace{}%
\AgdaSymbol{(}\AgdaBound{a}\AgdaSpace{}%
\AgdaBound{b}\AgdaSpace{}%
\AgdaBound{c}\AgdaSpace{}%
\AgdaSymbol{:}\AgdaSpace{}%
\AgdaDatatype{ℕ}\AgdaSymbol{)}\AgdaSpace{}%
\AgdaSymbol{→}\AgdaSpace{}%
\AgdaPrimitive{Set}\AgdaSymbol{)}\<%
\\
\>[.][@{}l@{}]\<[995I]%
\>[7]\AgdaSymbol{:}\AgdaSpace{}%
\AgdaSymbol{(}\AgdaBound{a}\AgdaSpace{}%
\AgdaBound{b}\AgdaSpace{}%
\AgdaBound{c}\AgdaSpace{}%
\AgdaSymbol{:}\AgdaSpace{}%
\AgdaDatatype{S}\AgdaSymbol{)}\AgdaSpace{}%
\AgdaSymbol{→}\AgdaSpace{}%
\AgdaPrimitive{Set}\AgdaSpace{}%
\AgdaKeyword{where}\AgdaSpace{}%
\AgdaKeyword{instance}\<%
\\
\>[2][@{}l@{\AgdaIndent{0}}]%
\>[6]\AgdaInductiveConstructor{[]}%
\>[12]\AgdaSymbol{:}\AgdaSpace{}%
\AgdaDatatype{Pw₃}\AgdaSpace{}%
\AgdaBound{R}\AgdaSpace{}%
\AgdaInductiveConstructor{[]}\AgdaSpace{}%
\AgdaInductiveConstructor{[]}\AgdaSpace{}%
\AgdaInductiveConstructor{[]}\<%
\\
%
\>[6]\AgdaInductiveConstructor{cons}%
\>[12]\AgdaSymbol{:}\AgdaSpace{}%
\AgdaSymbol{⦃}\AgdaSpace{}%
\AgdaBound{R}\AgdaSpace{}%
\AgdaGeneralizable{m}\AgdaSpace{}%
\AgdaGeneralizable{n}\AgdaSpace{}%
\AgdaGeneralizable{k}\AgdaSpace{}%
\AgdaSymbol{⦄}\AgdaSpace{}%
\AgdaSymbol{→}\AgdaSpace{}%
\AgdaSymbol{⦃}\AgdaSpace{}%
\AgdaDatatype{Pw₃}\AgdaSpace{}%
\AgdaBound{R}\AgdaSpace{}%
\AgdaGeneralizable{s}\AgdaSpace{}%
\AgdaGeneralizable{p}\AgdaSpace{}%
\AgdaGeneralizable{q}\AgdaSpace{}%
\AgdaSymbol{⦄}\<%
\\
%
\>[12]\AgdaSymbol{→}\AgdaSpace{}%
\AgdaDatatype{Pw₃}\AgdaSpace{}%
\AgdaBound{R}\AgdaSpace{}%
\AgdaSymbol{(}\AgdaGeneralizable{m}\AgdaSpace{}%
\AgdaOperator{\AgdaInductiveConstructor{∷}}\AgdaSpace{}%
\AgdaGeneralizable{s}\AgdaSymbol{)}\AgdaSpace{}%
\AgdaSymbol{(}\AgdaGeneralizable{n}\AgdaSpace{}%
\AgdaOperator{\AgdaInductiveConstructor{∷}}\AgdaSpace{}%
\AgdaGeneralizable{p}\AgdaSymbol{)}\AgdaSpace{}%
\AgdaSymbol{(}\AgdaGeneralizable{k}\AgdaSpace{}%
\AgdaOperator{\AgdaInductiveConstructor{∷}}\AgdaSpace{}%
\AgdaGeneralizable{q}\AgdaSymbol{)}\<%
\end{code}}
\end{mathpar}
While the definition is straight-forward, note that we mark constructors
with the keyword \AK{instance} and we turn the arguments of \AC{cons}
into instance arguments\footnote{See \url{https://agda.readthedocs.io/en/v2.7.0.1/language/instance-arguments.html} for more details.}.  These arguments
behave like the hidden arguments, except Agda will apply an instance
search when solving them.  This allows us to omit these proofs in
a larger number of cases than if we were to use hidden arguments.

\begin{code}[hide]%
%
\>[2]\AgdaKeyword{infix}\AgdaSpace{}%
\AgdaNumber{5}\AgdaSpace{}%
\AgdaOperator{\AgdaFunction{\AgdaUnderscore{}+\AgdaUnderscore{}≈\AgdaUnderscore{}}}\<%
\\
%
\>[2]\AgdaKeyword{infix}\AgdaSpace{}%
\AgdaNumber{5}\AgdaSpace{}%
\AgdaOperator{\AgdaFunction{suc\AgdaUnderscore{}≈\AgdaUnderscore{}}}\<%
\\
%
\>[2]\AgdaKeyword{infix}\AgdaSpace{}%
\AgdaNumber{5}\AgdaSpace{}%
\AgdaOperator{\AgdaFunction{\AgdaUnderscore{}*\AgdaUnderscore{}≈\AgdaUnderscore{}}}\<%
\\
%
\>[2]\AgdaKeyword{infixl}\AgdaSpace{}%
\AgdaNumber{8}\AgdaSpace{}%
\AgdaOperator{\AgdaFunction{\AgdaUnderscore{}⊝ₚ\AgdaUnderscore{}}}\<%
\end{code}

The second step is to define the actual relations.  With the help of composition
combinators ($f$ \AF{∘} $g$ = λ x → $f$ ($g$ x)) and ($f$ \AF{∘₂} $g$ = λ x y → $f$ ($g$ x y))
the definitions are as follows.
\begin{mathpar}
\codeblock{\begin{code}%
%
\>[2]\AgdaOperator{\AgdaFunction{\AgdaUnderscore{}+\AgdaUnderscore{}≈\AgdaUnderscore{}}}\AgdaSpace{}%
\AgdaSymbol{:}\AgdaSpace{}%
\AgdaSymbol{(}\AgdaBound{s}\AgdaSpace{}%
\AgdaBound{p}\AgdaSpace{}%
\AgdaBound{q}\AgdaSpace{}%
\AgdaSymbol{:}\AgdaSpace{}%
\AgdaDatatype{S}\AgdaSymbol{)}\AgdaSpace{}%
\AgdaSymbol{→}\AgdaSpace{}%
\AgdaPrimitive{Set}\<%
\\
%
\>[2]\AgdaOperator{\AgdaFunction{\AgdaUnderscore{}+\AgdaUnderscore{}≈\AgdaUnderscore{}}}\AgdaSpace{}%
\AgdaSymbol{=}\AgdaSpace{}%
\AgdaDatatype{Pw₃}\AgdaSpace{}%
\AgdaSymbol{(}\AgdaOperator{\AgdaDatatype{\AgdaUnderscore{}≡\AgdaUnderscore{}}}\AgdaSpace{}%
\AgdaOperator{\AgdaFunction{∘₂}}\AgdaSpace{}%
\AgdaOperator{\AgdaPrimitive{\AgdaUnderscore{}+\AgdaUnderscore{}}}\AgdaSymbol{)}\<%
\end{code}}
\and
\codeblock{\begin{code}%
%
\>[2]\AgdaOperator{\AgdaFunction{\AgdaUnderscore{}*\AgdaUnderscore{}≈\AgdaUnderscore{}}}\AgdaSpace{}%
\AgdaSymbol{:}\AgdaSpace{}%
\AgdaSymbol{(}\AgdaBound{s}\AgdaSpace{}%
\AgdaBound{p}\AgdaSpace{}%
\AgdaBound{q}\AgdaSpace{}%
\AgdaSymbol{:}\AgdaSpace{}%
\AgdaDatatype{S}\AgdaSymbol{)}\AgdaSpace{}%
\AgdaSymbol{→}\AgdaSpace{}%
\AgdaPrimitive{Set}\<%
\\
%
\>[2]\AgdaOperator{\AgdaFunction{\AgdaUnderscore{}*\AgdaUnderscore{}≈\AgdaUnderscore{}}}\AgdaSpace{}%
\AgdaSymbol{=}\AgdaSpace{}%
\AgdaDatatype{Pw₃}\AgdaSpace{}%
\AgdaSymbol{(}\AgdaOperator{\AgdaDatatype{\AgdaUnderscore{}≡\AgdaUnderscore{}}}\AgdaSpace{}%
\AgdaOperator{\AgdaFunction{∘₂}}\AgdaSpace{}%
\AgdaOperator{\AgdaPrimitive{\AgdaUnderscore{}*\AgdaUnderscore{}}}\AgdaSymbol{)}\<%
\end{code}}
\and
\codeblock{\begin{code}%
%
\>[2]\AgdaOperator{\AgdaFunction{suc\AgdaUnderscore{}≈\AgdaUnderscore{}}}\AgdaSpace{}%
\AgdaSymbol{:}\AgdaSpace{}%
\AgdaSymbol{(}\AgdaBound{s}\AgdaSpace{}%
\AgdaBound{p}\AgdaSpace{}%
\AgdaSymbol{:}\AgdaSpace{}%
\AgdaDatatype{S}\AgdaSymbol{)}\AgdaSpace{}%
\AgdaSymbol{→}\AgdaSpace{}%
\AgdaPrimitive{Set}\<%
\\
%
\>[2]\AgdaOperator{\AgdaFunction{suc\AgdaUnderscore{}≈\AgdaUnderscore{}}}\AgdaSpace{}%
\AgdaSymbol{=}\AgdaSpace{}%
\AgdaDatatype{Pw₂}\AgdaSpace{}%
\AgdaSymbol{(}\AgdaOperator{\AgdaDatatype{\AgdaUnderscore{}≡\AgdaUnderscore{}}}\AgdaSpace{}%
\AgdaOperator{\AgdaFunction{∘}}\AgdaSpace{}%
\AgdaInductiveConstructor{suc}\AgdaSymbol{)}\<%
\end{code}}
\end{mathpar}

With these relations in place, we could define generalised convolution
similarly to \AF{sum} where we recurse over the shape, performing one
operation at a time.  However, there is a good point made
in~\cite{cnn-array} about shifting the shape recursion into index computation.
% Talk about mental model of runtime where arrays are flat and indices are offsets
Therefore we define \AF{\_⊕ₚ\_} and \AF{\_⊝ₚ\_} which generalise \AF{\_⊕\_} and
\AF{\_⊝\_} for higher ranks.  Once again, \AD{Dec} type forces \AF{⊝ₚ} to justify
the cases when the inverse does not exist.
\begin{mathpar}
\codeblock{\begin{code}%
%
\>[2]\AgdaOperator{\AgdaFunction{\AgdaUnderscore{}⊕ₚ\AgdaUnderscore{}}}\AgdaSpace{}%
\AgdaSymbol{:}\AgdaSpace{}%
\AgdaDatatype{P}\AgdaSpace{}%
\AgdaGeneralizable{s}\AgdaSpace{}%
\AgdaSymbol{→}\AgdaSpace{}%
\AgdaDatatype{P}\AgdaSpace{}%
\AgdaGeneralizable{u}\AgdaSpace{}%
\AgdaSymbol{→}\AgdaSpace{}%
\AgdaOperator{\AgdaFunction{suc}}\AgdaSpace{}%
\AgdaGeneralizable{p}\AgdaSpace{}%
\AgdaOperator{\AgdaFunction{≈}}\AgdaSpace{}%
\AgdaGeneralizable{u}\AgdaSpace{}%
\AgdaSymbol{→}\AgdaSpace{}%
\AgdaGeneralizable{s}\AgdaSpace{}%
\AgdaOperator{\AgdaFunction{+}}\AgdaSpace{}%
\AgdaGeneralizable{p}\AgdaSpace{}%
\AgdaOperator{\AgdaFunction{≈}}\AgdaSpace{}%
\AgdaGeneralizable{r}\AgdaSpace{}%
\AgdaSymbol{→}\AgdaSpace{}%
\AgdaDatatype{P}\AgdaSpace{}%
\AgdaGeneralizable{r}\<%
\\
%
\>[2]\AgdaOperator{\AgdaFunction{\AgdaUnderscore{}⊝ₚ\AgdaUnderscore{}}}\AgdaSpace{}%
\AgdaSymbol{:}\AgdaSpace{}%
\AgdaSymbol{(}\AgdaBound{i}\AgdaSpace{}%
\AgdaSymbol{:}\AgdaSpace{}%
\AgdaDatatype{P}\AgdaSpace{}%
\AgdaGeneralizable{r}\AgdaSymbol{)}\AgdaSpace{}%
\AgdaSymbol{(}\AgdaBound{j}\AgdaSpace{}%
\AgdaSymbol{:}\AgdaSpace{}%
\AgdaDatatype{P}\AgdaSpace{}%
\AgdaGeneralizable{s}\AgdaSymbol{)}\AgdaSpace{}%
\AgdaSymbol{(}\AgdaBound{su}\AgdaSpace{}%
\AgdaSymbol{:}\AgdaSpace{}%
\AgdaOperator{\AgdaFunction{suc}}\AgdaSpace{}%
\AgdaGeneralizable{p}\AgdaSpace{}%
\AgdaOperator{\AgdaFunction{≈}}\AgdaSpace{}%
\AgdaGeneralizable{u}\AgdaSymbol{)}\AgdaSpace{}%
\AgdaSymbol{(}\AgdaBound{sp}\AgdaSpace{}%
\AgdaSymbol{:}\AgdaSpace{}%
\AgdaGeneralizable{s}\AgdaSpace{}%
\AgdaOperator{\AgdaFunction{+}}\AgdaSpace{}%
\AgdaGeneralizable{p}\AgdaSpace{}%
\AgdaOperator{\AgdaFunction{≈}}\AgdaSpace{}%
\AgdaGeneralizable{r}\AgdaSymbol{)}\AgdaSpace{}%
\AgdaSymbol{→}\AgdaSpace{}%
\AgdaRecord{Dec}\AgdaSpace{}%
\AgdaSymbol{(}\AgdaFunction{∃}\AgdaSpace{}%
\AgdaSymbol{λ}\AgdaSpace{}%
\AgdaBound{k}\AgdaSpace{}%
\AgdaSymbol{→}\AgdaSpace{}%
\AgdaSymbol{(}\AgdaBound{j}\AgdaSpace{}%
\AgdaOperator{\AgdaFunction{⊕ₚ}}\AgdaSpace{}%
\AgdaBound{k}\AgdaSymbol{)}\AgdaSpace{}%
\AgdaBound{su}\AgdaSpace{}%
\AgdaBound{sp}\AgdaSpace{}%
\AgdaOperator{\AgdaDatatype{≡}}\AgdaSpace{}%
\AgdaBound{i}\AgdaSymbol{)}\<%
\end{code}}
\end{mathpar}
The implementations of \AF{⊕ₚ} and \AF{⊝ₚ} simply apply \AF{⊕} and \AF{⊝}.
In the \AF{⊝} case a little plumbing is required when constructing the
proof of (non-)existence of the inverse.
\begin{code}[hide]%
%
\>[2]\AgdaSymbol{(}\AgdaBound{i}\AgdaSpace{}%
\AgdaOperator{\AgdaFunction{⊕ₚ}}\AgdaSpace{}%
\AgdaBound{j}\AgdaSymbol{)}\AgdaSpace{}%
\AgdaInductiveConstructor{[]}\AgdaSpace{}%
\AgdaInductiveConstructor{[]}\AgdaSpace{}%
\AgdaSymbol{=}\AgdaSpace{}%
\AgdaBound{j}\<%
\\
%
\>[2]\AgdaSymbol{((}\AgdaBound{i}\AgdaSpace{}%
\AgdaOperator{\AgdaInductiveConstructor{∷}}\AgdaSpace{}%
\AgdaBound{is}\AgdaSymbol{)}\AgdaSpace{}%
\AgdaOperator{\AgdaFunction{⊕ₚ}}\AgdaSpace{}%
\AgdaSymbol{(}\AgdaBound{j}\AgdaSpace{}%
\AgdaOperator{\AgdaInductiveConstructor{∷}}\AgdaSpace{}%
\AgdaBound{js}\AgdaSymbol{))}\AgdaSpace{}%
\AgdaSymbol{(}\AgdaInductiveConstructor{cons}\AgdaSpace{}%
\AgdaSymbol{⦃}\AgdaSpace{}%
\AgdaInductiveConstructor{refl}\AgdaSpace{}%
\AgdaSymbol{⦄}\AgdaSpace{}%
\AgdaSymbol{⦃}\AgdaSpace{}%
\AgdaBound{sp}\AgdaSpace{}%
\AgdaSymbol{⦄)}\AgdaSpace{}%
\AgdaSymbol{(}\AgdaInductiveConstructor{cons}\AgdaSpace{}%
\AgdaSymbol{⦃}\AgdaSpace{}%
\AgdaInductiveConstructor{refl}\AgdaSpace{}%
\AgdaSymbol{⦄}\AgdaSpace{}%
\AgdaSymbol{⦃}\AgdaSpace{}%
\AgdaBound{s+p}\AgdaSpace{}%
\AgdaSymbol{⦄)}\<%
\\
\>[2][@{}l@{\AgdaIndent{0}}]%
\>[4]\AgdaSymbol{=}\AgdaSpace{}%
\AgdaSymbol{(}\AgdaBound{i}\AgdaSpace{}%
\AgdaOperator{\AgdaFunction{⊕}}\AgdaSpace{}%
\AgdaBound{j}\AgdaSymbol{)}\AgdaSpace{}%
\AgdaOperator{\AgdaInductiveConstructor{∷}}\AgdaSpace{}%
\AgdaSymbol{(}\AgdaBound{is}\AgdaSpace{}%
\AgdaOperator{\AgdaFunction{⊕ₚ}}\AgdaSpace{}%
\AgdaBound{js}\AgdaSymbol{)}\AgdaSpace{}%
\AgdaBound{sp}\AgdaSpace{}%
\AgdaBound{s+p}\<%
\\
%
\\[\AgdaEmptyExtraSkip]%
%
\>[2]\AgdaSymbol{(}\AgdaInductiveConstructor{[]}\AgdaSpace{}%
\AgdaOperator{\AgdaFunction{⊝ₚ}}\AgdaSpace{}%
\AgdaBound{j}\AgdaSymbol{)}\AgdaSpace{}%
\AgdaInductiveConstructor{[]}\AgdaSpace{}%
\AgdaInductiveConstructor{[]}\AgdaSpace{}%
\AgdaSymbol{=}\AgdaSpace{}%
\AgdaInductiveConstructor{yes}\AgdaSpace{}%
\AgdaSymbol{(}\AgdaInductiveConstructor{[]}\AgdaSpace{}%
\AgdaOperator{\AgdaInductiveConstructor{,}}\AgdaSpace{}%
\AgdaInductiveConstructor{refl}\AgdaSymbol{)}\<%
\\
%
\>[2]\AgdaSymbol{((}\AgdaBound{i}\AgdaSpace{}%
\AgdaOperator{\AgdaInductiveConstructor{∷}}%
\>[1190I]\AgdaBound{is}\AgdaSymbol{)}\AgdaSpace{}%
\AgdaOperator{\AgdaFunction{⊝ₚ}}\AgdaSpace{}%
\AgdaSymbol{(}\AgdaBound{j}\AgdaSpace{}%
\AgdaOperator{\AgdaInductiveConstructor{∷}}\AgdaSpace{}%
\AgdaBound{js}\AgdaSymbol{))}\AgdaSpace{}%
\AgdaSymbol{(}\AgdaInductiveConstructor{cons}\AgdaSpace{}%
\AgdaSymbol{⦃}\AgdaSpace{}%
\AgdaInductiveConstructor{refl}\AgdaSpace{}%
\AgdaSymbol{⦄}\AgdaSpace{}%
\AgdaSymbol{⦃}\AgdaSpace{}%
\AgdaBound{sp}\AgdaSpace{}%
\AgdaSymbol{⦄)}\AgdaSpace{}%
\AgdaSymbol{(}\AgdaInductiveConstructor{cons}\AgdaSpace{}%
\AgdaSymbol{⦃}\AgdaSpace{}%
\AgdaInductiveConstructor{refl}\AgdaSpace{}%
\AgdaSymbol{⦄}\AgdaSpace{}%
\AgdaSymbol{⦃}\AgdaSpace{}%
\AgdaBound{s+p}\AgdaSpace{}%
\AgdaSymbol{⦄)}\<%
\\
\>[.][@{}l@{}]\<[1190I]%
\>[8]\AgdaKeyword{with}\AgdaSpace{}%
\AgdaBound{i}\AgdaSpace{}%
\AgdaOperator{\AgdaFunction{⊝}}\AgdaSpace{}%
\AgdaBound{j}\<%
\\
%
\>[2]\AgdaSymbol{...}\AgdaSpace{}%
\AgdaSymbol{|}\AgdaSpace{}%
\AgdaInductiveConstructor{no}\AgdaSpace{}%
\AgdaBound{¬p}\AgdaSpace{}%
\AgdaSymbol{=}\AgdaSpace{}%
\AgdaInductiveConstructor{no}\AgdaSpace{}%
\AgdaSymbol{λ}\AgdaSpace{}%
\AgdaSymbol{\{}\AgdaSpace{}%
\AgdaSymbol{((}\AgdaBound{k}\AgdaSpace{}%
\AgdaOperator{\AgdaInductiveConstructor{∷}}\AgdaSpace{}%
\AgdaSymbol{\AgdaUnderscore{})}\AgdaSpace{}%
\AgdaOperator{\AgdaInductiveConstructor{,}}\AgdaSpace{}%
\AgdaInductiveConstructor{refl}\AgdaSymbol{)}\AgdaSpace{}%
\AgdaSymbol{→}\AgdaSpace{}%
\AgdaBound{¬p}\AgdaSpace{}%
\AgdaSymbol{(}\AgdaBound{k}\AgdaSpace{}%
\AgdaOperator{\AgdaInductiveConstructor{,}}\AgdaSpace{}%
\AgdaInductiveConstructor{refl}\AgdaSymbol{)}\AgdaSpace{}%
\AgdaSymbol{\}}\<%
\\
%
\>[2]\AgdaSymbol{...}\AgdaSpace{}%
\AgdaSymbol{|}\AgdaSpace{}%
\AgdaInductiveConstructor{yes}\AgdaSpace{}%
\AgdaSymbol{(}\AgdaBound{k}\AgdaSpace{}%
\AgdaOperator{\AgdaInductiveConstructor{,}}\AgdaSpace{}%
\AgdaBound{p}\AgdaSymbol{)}\AgdaSpace{}%
\AgdaKeyword{with}\AgdaSpace{}%
\AgdaSymbol{(}\AgdaBound{is}\AgdaSpace{}%
\AgdaOperator{\AgdaFunction{⊝ₚ}}\AgdaSpace{}%
\AgdaBound{js}\AgdaSymbol{)}\AgdaSpace{}%
\AgdaBound{sp}\AgdaSpace{}%
\AgdaBound{s+p}\<%
\\
%
\>[2]\AgdaSymbol{...}\AgdaSpace{}%
\AgdaSymbol{|}\AgdaSpace{}%
\AgdaInductiveConstructor{no}\AgdaSpace{}%
\AgdaBound{¬q}\AgdaSpace{}%
\AgdaSymbol{=}\AgdaSpace{}%
\AgdaInductiveConstructor{no}\AgdaSpace{}%
\AgdaSymbol{λ}\AgdaSpace{}%
\AgdaSymbol{\{}\AgdaSpace{}%
\AgdaSymbol{((\AgdaUnderscore{}}\AgdaSpace{}%
\AgdaOperator{\AgdaInductiveConstructor{∷}}\AgdaSpace{}%
\AgdaBound{xs}\AgdaSymbol{)}\AgdaSpace{}%
\AgdaOperator{\AgdaInductiveConstructor{,}}\AgdaSpace{}%
\AgdaInductiveConstructor{refl}\AgdaSymbol{)}\AgdaSpace{}%
\AgdaSymbol{→}\AgdaSpace{}%
\AgdaBound{¬q}\AgdaSpace{}%
\AgdaSymbol{(}\AgdaBound{xs}\AgdaSpace{}%
\AgdaOperator{\AgdaInductiveConstructor{,}}\AgdaSpace{}%
\AgdaInductiveConstructor{refl}\AgdaSymbol{)}\AgdaSpace{}%
\AgdaSymbol{\}}\<%
\\
%
\>[2]\AgdaSymbol{...}\AgdaSpace{}%
\AgdaSymbol{|}\AgdaSpace{}%
\AgdaInductiveConstructor{yes}\AgdaSpace{}%
\AgdaSymbol{(}\AgdaBound{ks}\AgdaSpace{}%
\AgdaOperator{\AgdaInductiveConstructor{,}}\AgdaSpace{}%
\AgdaBound{q}\AgdaSymbol{)}\AgdaSpace{}%
\AgdaSymbol{=}\AgdaSpace{}%
\AgdaInductiveConstructor{yes}\AgdaSpace{}%
\AgdaSymbol{(}\AgdaBound{k}\AgdaSpace{}%
\AgdaOperator{\AgdaInductiveConstructor{∷}}\AgdaSpace{}%
\AgdaBound{ks}\AgdaSpace{}%
\AgdaOperator{\AgdaInductiveConstructor{,}}\AgdaSpace{}%
\AgdaFunction{cong₂}\AgdaSpace{}%
\AgdaOperator{\AgdaInductiveConstructor{\AgdaUnderscore{}∷\AgdaUnderscore{}}}\AgdaSpace{}%
\AgdaBound{p}\AgdaSpace{}%
\AgdaBound{q}\AgdaSymbol{)}\<%
\end{code}

Generalised \AF{slide} looks very similar to its 1-dimensional
counterpart, except that \AF{⊕} is replaced with \AF{⊕ₚ}
We also introduce a section of \AF{slide} that we call \AF{backslide}.
It embeds a $(1+p)$-dimensional array into a $(s+p)$-dimensional
one at the offset $i$ using \AB{def} to fill the outer region.
\begin{mathpar}
\codeblock{\begin{code}%
%
\>[2]\AgdaFunction{slide}\AgdaSpace{}%
\AgdaSymbol{:}\AgdaSpace{}%
\AgdaDatatype{P}\AgdaSpace{}%
\AgdaGeneralizable{s}\AgdaSpace{}%
\AgdaSymbol{→}\AgdaSpace{}%
\AgdaGeneralizable{s}\AgdaSpace{}%
\AgdaOperator{\AgdaFunction{+}}\AgdaSpace{}%
\AgdaGeneralizable{p}\AgdaSpace{}%
\AgdaOperator{\AgdaFunction{≈}}\AgdaSpace{}%
\AgdaGeneralizable{r}\AgdaSpace{}%
\AgdaSymbol{→}\AgdaSpace{}%
\AgdaFunction{Ar}\AgdaSpace{}%
\AgdaGeneralizable{r}\AgdaSpace{}%
\AgdaGeneralizable{X}\AgdaSpace{}%
\AgdaSymbol{→}\AgdaSpace{}%
\AgdaOperator{\AgdaFunction{suc}}\AgdaSpace{}%
\AgdaGeneralizable{p}\AgdaSpace{}%
\AgdaOperator{\AgdaFunction{≈}}\AgdaSpace{}%
\AgdaGeneralizable{u}\AgdaSpace{}%
\AgdaSymbol{→}\AgdaSpace{}%
\AgdaFunction{Ar}\AgdaSpace{}%
\AgdaGeneralizable{u}\AgdaSpace{}%
\AgdaGeneralizable{X}\<%
\\
%
\>[2]\AgdaFunction{slide}\AgdaSpace{}%
\AgdaBound{i}\AgdaSpace{}%
\AgdaBound{pl}\AgdaSpace{}%
\AgdaBound{a}\AgdaSpace{}%
\AgdaBound{su}\AgdaSpace{}%
\AgdaBound{j}\AgdaSpace{}%
\AgdaSymbol{=}\AgdaSpace{}%
\AgdaBound{a}\AgdaSpace{}%
\AgdaSymbol{((}\AgdaBound{i}\AgdaSpace{}%
\AgdaOperator{\AgdaFunction{⊕ₚ}}\AgdaSpace{}%
\AgdaBound{j}\AgdaSymbol{)}\AgdaSpace{}%
\AgdaBound{su}\AgdaSpace{}%
\AgdaBound{pl}\AgdaSymbol{)}\<%
\\
%
\\[\AgdaEmptyExtraSkip]%
%
\>[2]\AgdaFunction{backslide}\AgdaSpace{}%
\AgdaSymbol{:}\AgdaSpace{}%
\AgdaDatatype{P}\AgdaSpace{}%
\AgdaGeneralizable{s}\AgdaSpace{}%
\AgdaSymbol{→}\AgdaSpace{}%
\AgdaFunction{Ar}\AgdaSpace{}%
\AgdaGeneralizable{u}\AgdaSpace{}%
\AgdaGeneralizable{X}\AgdaSpace{}%
\AgdaSymbol{→}\AgdaSpace{}%
\AgdaOperator{\AgdaFunction{suc}}\AgdaSpace{}%
\AgdaGeneralizable{p}\AgdaSpace{}%
\AgdaOperator{\AgdaFunction{≈}}\AgdaSpace{}%
\AgdaGeneralizable{u}\AgdaSpace{}%
\AgdaSymbol{→}\AgdaSpace{}%
\AgdaSymbol{(}\AgdaBound{def}\AgdaSpace{}%
\AgdaSymbol{:}\AgdaSpace{}%
\AgdaGeneralizable{X}\AgdaSymbol{)}\AgdaSpace{}%
\AgdaSymbol{→}\AgdaSpace{}%
\AgdaGeneralizable{s}\AgdaSpace{}%
\AgdaOperator{\AgdaFunction{+}}\AgdaSpace{}%
\AgdaGeneralizable{p}\AgdaSpace{}%
\AgdaOperator{\AgdaFunction{≈}}\AgdaSpace{}%
\AgdaGeneralizable{r}\AgdaSpace{}%
\AgdaSymbol{→}\AgdaSpace{}%
\AgdaFunction{Ar}\AgdaSpace{}%
\AgdaGeneralizable{r}\AgdaSpace{}%
\AgdaGeneralizable{X}\<%
\\
%
\>[2]\AgdaFunction{backslide}\AgdaSpace{}%
\AgdaBound{i}\AgdaSpace{}%
\AgdaBound{a}\AgdaSpace{}%
\AgdaBound{su}\AgdaSpace{}%
\AgdaBound{def}\AgdaSpace{}%
\AgdaBound{pl}\AgdaSpace{}%
\AgdaBound{j}\AgdaSpace{}%
\AgdaKeyword{with}\AgdaSpace{}%
\AgdaSymbol{((}\AgdaBound{j}\AgdaSpace{}%
\AgdaOperator{\AgdaFunction{⊝ₚ}}\AgdaSpace{}%
\AgdaBound{i}\AgdaSymbol{)}\AgdaSpace{}%
\AgdaBound{su}\AgdaSpace{}%
\AgdaBound{pl}\AgdaSymbol{)}\<%
\\
%
\>[2]\AgdaSymbol{...}\AgdaSpace{}%
\AgdaSymbol{|}\AgdaSpace{}%
\AgdaInductiveConstructor{yes}\AgdaSpace{}%
\AgdaSymbol{(}\AgdaBound{k}\AgdaSpace{}%
\AgdaOperator{\AgdaInductiveConstructor{,}}\AgdaSpace{}%
\AgdaSymbol{\AgdaUnderscore{})}%
\>[21]\AgdaSymbol{=}\AgdaSpace{}%
\AgdaBound{a}\AgdaSpace{}%
\AgdaBound{k}\<%
\\
%
\>[2]\AgdaCatchallClause{\AgdaSymbol{...}}\AgdaSpace{}%
\AgdaCatchallClause{\AgdaSymbol{|}}\AgdaSpace{}%
\AgdaCatchallClause{\AgdaSymbol{\AgdaUnderscore{}}}%
\>[21]\AgdaSymbol{=}\AgdaSpace{}%
\AgdaBound{def}\<%
\end{code}}
\end{mathpar}

\paragraph{Remark on indexing} We would like to address a general remark that
is often made by functional programmers that index-oriented definitions such as
\AF{slide} and \AF{backslide} should be replaced by some construction that use
algebraic data types.  While this is of course a matter of taste, here are
important points that justify our choice. Firstly, array computations that use
explicit indices are easier to compile into efficient code. At runtime, arrays
will be represented as flat regions of memory without cons cells or other
pointer-connected structures. Index computations will be turned into offset
computations that are efficient on most architectures.  Secondly, many
rank-polymorphic operations on arrays are easier to express via index
manipulation (our indices have non-trivial structure) rather than via
traversals of algebraic data structures.  For example, consider a data
structure for a rank-polymorphic array similar to \AD{Ar}.  One needs something
like a free monad over a \AD{Vec} type, which can be easily defined.  Now,
consider defining a generalised transpose on such representation.  Transpose of
an \AD{Ar} array is simply a selection on a reversed index: λ ix → a
(\AF{reverse} ix). In case of free monads, this is a significantly more
complicated recursive expression.  Finally, when arrays are
functions, fusion equalities (\eg{} map f ∘ map g $\cong$ map (f ∘ g))
come for free through normalisation, which makes formal reasoning easier.





\subsection{CNN primitives\label{sec:ar-cnn-prim}}
Here we implement CNN-specific primitives that are needed for our running example.
All these primitives operate on arrays of reals.  We use builtin Agda floats in
the rest of the section that we refer to as \AD{ℝ}.  The only reason for this
is the ability to evaluate our specification with concrete values.
Later we are going to abstract over concrete implementation of \AD{ℝ}.

Generalised convolution is given by \AF{conv}, and it is almost identical to its
1-dimensional counterpart (except it uses \AF{slide} instead of \AF{slide₁}).
The \AF{mconv} runs $u$ \AF{conv}s (conceptually in parallel) and then it adds a
corresponding bias from the array $b$ (of shape $u$) to each convolution.
\begin{code}[hide]%
\>[0]\AgdaKeyword{module}\AgdaSpace{}%
\AgdaModule{CNN}\AgdaSpace{}%
\AgdaKeyword{where}\<%
\\
\>[0][@{}l@{\AgdaIndent{0}}]%
\>[2]\AgdaKeyword{open}\AgdaSpace{}%
\AgdaKeyword{import}\AgdaSpace{}%
\AgdaModule{Data.Nat}\AgdaSpace{}%
\AgdaSymbol{as}\AgdaSpace{}%
\AgdaModule{ℕ}\AgdaSpace{}%
\AgdaKeyword{using}\AgdaSpace{}%
\AgdaSymbol{(}\AgdaDatatype{ℕ}\AgdaSymbol{)}\<%
\\
%
\>[2]\AgdaKeyword{open}\AgdaSpace{}%
\AgdaKeyword{import}\AgdaSpace{}%
\AgdaModule{Data.Float}\AgdaSpace{}%
\AgdaSymbol{as}\AgdaSpace{}%
\AgdaModule{F}\AgdaSpace{}%
\AgdaKeyword{using}\AgdaSpace{}%
\AgdaSymbol{(}\AgdaPrimitive{\AgdaUnderscore{}+\AgdaUnderscore{}}\AgdaSymbol{;}\AgdaSpace{}%
\AgdaPrimitive{\AgdaUnderscore{}*\AgdaUnderscore{}}\AgdaSymbol{;}\AgdaSpace{}%
\AgdaPrimitive{\AgdaUnderscore{}÷\AgdaUnderscore{}}\AgdaSymbol{;}\AgdaSpace{}%
\AgdaPrimitive{e\textasciicircum{}\AgdaUnderscore{}}\AgdaSymbol{;}\AgdaSpace{}%
\AgdaPrimitive{-\AgdaUnderscore{}}\AgdaSymbol{;}\AgdaSpace{}%
\AgdaPrimitive{fromℕ}\AgdaSymbol{)}\AgdaSpace{}%
\AgdaKeyword{renaming}\AgdaSpace{}%
\AgdaSymbol{(}\AgdaPostulate{Float}\AgdaSpace{}%
\AgdaSymbol{to}\AgdaSpace{}%
\AgdaPostulate{ℝ}\AgdaSymbol{)}\<%
\\
%
\>[2]\AgdaKeyword{open}\AgdaSpace{}%
\AgdaKeyword{import}\AgdaSpace{}%
\AgdaModule{Data.Product}\AgdaSpace{}%
\AgdaSymbol{as}\AgdaSpace{}%
\AgdaModule{Prod}\AgdaSpace{}%
\AgdaKeyword{using}\AgdaSpace{}%
\AgdaSymbol{()}\<%
\\
%
\>[2]\AgdaKeyword{open}\AgdaSpace{}%
\AgdaKeyword{import}\AgdaSpace{}%
\AgdaModule{Data.Fin}\AgdaSpace{}%
\AgdaSymbol{as}\AgdaSpace{}%
\AgdaModule{F}\AgdaSpace{}%
\AgdaKeyword{using}\AgdaSpace{}%
\AgdaSymbol{(}\AgdaInductiveConstructor{zero}\AgdaSymbol{;}\AgdaSpace{}%
\AgdaInductiveConstructor{suc}\AgdaSymbol{;}\AgdaSpace{}%
\AgdaDatatype{Fin}\AgdaSymbol{;}\AgdaSpace{}%
\AgdaFunction{combine}\AgdaSymbol{;}\AgdaSpace{}%
\AgdaFunction{remQuot}\AgdaSymbol{;}\AgdaSpace{}%
\AgdaFunction{fromℕ<}\AgdaSymbol{;}\AgdaSpace{}%
\AgdaFunction{inject+}\AgdaSymbol{;}\AgdaSpace{}%
\AgdaFunction{splitAt}\AgdaSymbol{)}\<%
\\
%
\>[2]\AgdaKeyword{open}\AgdaSpace{}%
\AgdaModule{Array}\<%
\end{code}

\begin{code}%
%
\>[2]\AgdaFunction{conv}\AgdaSpace{}%
\AgdaSymbol{:}\AgdaSpace{}%
\AgdaGeneralizable{s}\AgdaSpace{}%
\AgdaOperator{\AgdaFunction{+}}\AgdaSpace{}%
\AgdaGeneralizable{p}\AgdaSpace{}%
\AgdaOperator{\AgdaFunction{≈}}\AgdaSpace{}%
\AgdaGeneralizable{r}\AgdaSpace{}%
\AgdaSymbol{→}\AgdaSpace{}%
\AgdaFunction{Ar}\AgdaSpace{}%
\AgdaGeneralizable{r}\AgdaSpace{}%
\AgdaPostulate{ℝ}\AgdaSpace{}%
\AgdaSymbol{→}\AgdaSpace{}%
\AgdaFunction{Ar}\AgdaSpace{}%
\AgdaGeneralizable{s}\AgdaSpace{}%
\AgdaPostulate{ℝ}\AgdaSpace{}%
\AgdaSymbol{→}\AgdaSpace{}%
\AgdaOperator{\AgdaFunction{suc}}\AgdaSpace{}%
\AgdaGeneralizable{p}\AgdaSpace{}%
\AgdaOperator{\AgdaFunction{≈}}\AgdaSpace{}%
\AgdaGeneralizable{u}\AgdaSpace{}%
\AgdaSymbol{→}\AgdaSpace{}%
\AgdaFunction{Ar}\AgdaSpace{}%
\AgdaGeneralizable{u}\AgdaSpace{}%
\AgdaPostulate{ℝ}\<%
\\
%
\>[2]\AgdaFunction{conv}\AgdaSpace{}%
\AgdaBound{sp}\AgdaSpace{}%
\AgdaBound{a}\AgdaSpace{}%
\AgdaBound{w}\AgdaSpace{}%
\AgdaBound{su}\AgdaSpace{}%
\AgdaSymbol{=}\AgdaSpace{}%
\AgdaFunction{sum}\AgdaSpace{}%
\AgdaSymbol{(}\AgdaFunction{zipWith}\AgdaSpace{}%
\AgdaOperator{\AgdaPrimitive{\AgdaUnderscore{}+\AgdaUnderscore{}}}\AgdaSymbol{)}\AgdaSpace{}%
\AgdaSymbol{(}\AgdaFunction{K}\AgdaSpace{}%
\AgdaNumber{0.0}\AgdaSymbol{)}\AgdaSpace{}%
\AgdaSymbol{λ}\AgdaSpace{}%
\AgdaBound{i}\AgdaSpace{}%
\AgdaSymbol{→}\AgdaSpace{}%
\AgdaFunction{map}\AgdaSpace{}%
\AgdaSymbol{(}\AgdaBound{w}\AgdaSpace{}%
\AgdaBound{i}\AgdaSpace{}%
\AgdaOperator{\AgdaPrimitive{*\AgdaUnderscore{}}}\AgdaSymbol{)}\AgdaSpace{}%
\AgdaSymbol{(}\AgdaFunction{slide}\AgdaSpace{}%
\AgdaBound{i}\AgdaSpace{}%
\AgdaBound{sp}\AgdaSpace{}%
\AgdaBound{a}\AgdaSpace{}%
\AgdaBound{su}\AgdaSymbol{)}\<%
\\
%
\\[\AgdaEmptyExtraSkip]%
%
\>[2]\AgdaFunction{mconv}\AgdaSpace{}%
\AgdaSymbol{:}\AgdaSpace{}%
\AgdaSymbol{⦃}\AgdaSpace{}%
\AgdaGeneralizable{s}\AgdaSpace{}%
\AgdaOperator{\AgdaFunction{+}}\AgdaSpace{}%
\AgdaGeneralizable{p}\AgdaSpace{}%
\AgdaOperator{\AgdaFunction{≈}}\AgdaSpace{}%
\AgdaGeneralizable{r}\AgdaSpace{}%
\AgdaSymbol{⦄}\AgdaSpace{}%
\AgdaSymbol{→}\AgdaSpace{}%
\AgdaFunction{Ar}\AgdaSpace{}%
\AgdaGeneralizable{r}\AgdaSpace{}%
\AgdaPostulate{ℝ}\AgdaSpace{}%
\AgdaSymbol{→}\AgdaSpace{}%
\AgdaFunction{Ar}\AgdaSpace{}%
\AgdaSymbol{(}\AgdaGeneralizable{u}\AgdaSpace{}%
\AgdaOperator{\AgdaFunction{⊗}}\AgdaSpace{}%
\AgdaGeneralizable{s}\AgdaSymbol{)}\AgdaSpace{}%
\AgdaPostulate{ℝ}\AgdaSpace{}%
\AgdaSymbol{→}\AgdaSpace{}%
\AgdaFunction{Ar}\AgdaSpace{}%
\AgdaGeneralizable{u}\AgdaSpace{}%
\AgdaPostulate{ℝ}\AgdaSpace{}%
\AgdaSymbol{→}\AgdaSpace{}%
\AgdaSymbol{⦃}\AgdaSpace{}%
\AgdaOperator{\AgdaFunction{suc}}\AgdaSpace{}%
\AgdaGeneralizable{p}\AgdaSpace{}%
\AgdaOperator{\AgdaFunction{≈}}\AgdaSpace{}%
\AgdaGeneralizable{q}\AgdaSpace{}%
\AgdaSymbol{⦄}\AgdaSpace{}%
\AgdaSymbol{→}\AgdaSpace{}%
\AgdaFunction{Ar}\AgdaSpace{}%
\AgdaSymbol{(}\AgdaGeneralizable{u}\AgdaSpace{}%
\AgdaOperator{\AgdaFunction{⊗}}\AgdaSpace{}%
\AgdaGeneralizable{q}\AgdaSymbol{)}\AgdaSpace{}%
\AgdaPostulate{ℝ}\<%
\\
%
\>[2]\AgdaFunction{mconv}\AgdaSpace{}%
\AgdaSymbol{⦃}\AgdaSpace{}%
\AgdaBound{sp}\AgdaSpace{}%
\AgdaSymbol{⦄}\AgdaSpace{}%
\AgdaBound{inp}\AgdaSpace{}%
\AgdaBound{w}\AgdaSpace{}%
\AgdaBound{b}\AgdaSpace{}%
\AgdaSymbol{⦃}\AgdaSpace{}%
\AgdaBound{su}\AgdaSpace{}%
\AgdaSymbol{⦄}\AgdaSpace{}%
\AgdaSymbol{=}\AgdaSpace{}%
\AgdaFunction{unnest}\AgdaSpace{}%
\AgdaSymbol{λ}\AgdaSpace{}%
\AgdaBound{i}\AgdaSpace{}%
\AgdaSymbol{→}\AgdaSpace{}%
\AgdaFunction{map}\AgdaSpace{}%
\AgdaSymbol{(}\AgdaBound{b}\AgdaSpace{}%
\AgdaBound{i}\AgdaSpace{}%
\AgdaOperator{\AgdaPrimitive{+\AgdaUnderscore{}}}\AgdaSymbol{)}\AgdaSpace{}%
\AgdaSymbol{(}\AgdaFunction{conv}\AgdaSpace{}%
\AgdaBound{sp}\AgdaSpace{}%
\AgdaBound{inp}\AgdaSpace{}%
\AgdaSymbol{(}\AgdaFunction{nest}\AgdaSpace{}%
\AgdaBound{w}\AgdaSpace{}%
\AgdaBound{i}\AgdaSymbol{)}\AgdaSpace{}%
\AgdaBound{su}\AgdaSymbol{)}\<%
\end{code}
The logistic function computes ${1}/(1 + e^{-x})$ for every element in the array.
\begin{mathpar}
\codeblock{\begin{code}%
%
\>[2]\AgdaFunction{logistic}\AgdaSpace{}%
\AgdaSymbol{:}\AgdaSpace{}%
\AgdaFunction{Ar}\AgdaSpace{}%
\AgdaGeneralizable{s}\AgdaSpace{}%
\AgdaPostulate{ℝ}\AgdaSpace{}%
\AgdaSymbol{→}\AgdaSpace{}%
\AgdaFunction{Ar}\AgdaSpace{}%
\AgdaGeneralizable{s}\AgdaSpace{}%
\AgdaPostulate{ℝ}\<%
\\
%
\>[2]\AgdaFunction{logistic}\AgdaSpace{}%
\AgdaSymbol{=}\AgdaSpace{}%
\AgdaFunction{map}\AgdaSpace{}%
\AgdaSymbol{λ}\AgdaSpace{}%
\AgdaBound{x}\AgdaSpace{}%
\AgdaSymbol{→}\AgdaSpace{}%
\AgdaNumber{1.0}\AgdaSpace{}%
\AgdaOperator{\AgdaPrimitive{÷}}\AgdaSpace{}%
\AgdaSymbol{(}\AgdaNumber{1.0}\AgdaSpace{}%
\AgdaOperator{\AgdaPrimitive{+}}\AgdaSpace{}%
\AgdaOperator{\AgdaPrimitive{e\textasciicircum{}}}\AgdaSpace{}%
\AgdaSymbol{(}\AgdaOperator{\AgdaPrimitive{-}}\AgdaSpace{}%
\AgdaBound{x}\AgdaSymbol{))}\<%
\end{code}}
\end{mathpar}

\paragraph{Average Pooling}
One of the steps of the machine learning algorithm is average pooling which
splits an array into sub-blocks and computes the average for every such
block.  Implementing this pattern generally is tricky as we have to
preserve the local neighbourhood within the blocks.  Working with a
blocked array would be inconvenient as the blocked shape
does not go well with \AF{slides}.  We solve this by introducing
blocked selections \AF{selb} into arrays of shape $(s * p)$ as well
as blocked array constructor \AF{imapb} that builds an array of
shape $(s * p)$ out of $s$ blocks of shape $p$.  Defining these
operations we require pairing and projections of the blocked indices
which is achieved by applying division and modulo operation on the
components.  The types of these operations are as follows:
\begin{mathpar}
\codeblock{\begin{code}%
%
\>[2]\AgdaFunction{ix-div}\AgdaSpace{}%
\AgdaSymbol{:}\AgdaSpace{}%
\AgdaDatatype{P}\AgdaSpace{}%
\AgdaGeneralizable{q}\AgdaSpace{}%
\AgdaSymbol{→}\AgdaSpace{}%
\AgdaGeneralizable{s}\AgdaSpace{}%
\AgdaOperator{\AgdaFunction{*}}\AgdaSpace{}%
\AgdaGeneralizable{p}\AgdaSpace{}%
\AgdaOperator{\AgdaFunction{≈}}\AgdaSpace{}%
\AgdaGeneralizable{q}\AgdaSpace{}%
\AgdaSymbol{→}\AgdaSpace{}%
\AgdaDatatype{P}\AgdaSpace{}%
\AgdaGeneralizable{s}\<%
\end{code}}
\and
\codeblock{\begin{code}%
%
\>[2]\AgdaFunction{ix-mod}\AgdaSpace{}%
\AgdaSymbol{:}\AgdaSpace{}%
\AgdaDatatype{P}\AgdaSpace{}%
\AgdaGeneralizable{q}\AgdaSpace{}%
\AgdaSymbol{→}\AgdaSpace{}%
\AgdaGeneralizable{s}\AgdaSpace{}%
\AgdaOperator{\AgdaFunction{*}}\AgdaSpace{}%
\AgdaGeneralizable{p}\AgdaSpace{}%
\AgdaOperator{\AgdaFunction{≈}}\AgdaSpace{}%
\AgdaGeneralizable{q}\AgdaSpace{}%
\AgdaSymbol{→}\AgdaSpace{}%
\AgdaDatatype{P}\AgdaSpace{}%
\AgdaGeneralizable{p}\<%
\end{code}}
\and
\codeblock{\begin{code}%
%
\>[2]\AgdaFunction{ix-combine}\AgdaSpace{}%
\AgdaSymbol{:}\AgdaSpace{}%
\AgdaDatatype{P}\AgdaSpace{}%
\AgdaGeneralizable{s}\AgdaSpace{}%
\AgdaSymbol{→}\AgdaSpace{}%
\AgdaDatatype{P}\AgdaSpace{}%
\AgdaGeneralizable{p}\AgdaSpace{}%
\AgdaSymbol{→}\AgdaSpace{}%
\AgdaGeneralizable{s}\AgdaSpace{}%
\AgdaOperator{\AgdaFunction{*}}\AgdaSpace{}%
\AgdaGeneralizable{p}\AgdaSpace{}%
\AgdaOperator{\AgdaFunction{≈}}\AgdaSpace{}%
\AgdaGeneralizable{q}\AgdaSpace{}%
\AgdaSymbol{→}\AgdaSpace{}%
\AgdaDatatype{P}\AgdaSpace{}%
\AgdaGeneralizable{q}\<%
\end{code}}
\end{mathpar}
\begin{code}[hide]%
%
\>[2]\AgdaFunction{ix-div}\AgdaSpace{}%
\AgdaBound{is}\AgdaSpace{}%
\AgdaInductiveConstructor{[]}\AgdaSpace{}%
\AgdaSymbol{=}\AgdaSpace{}%
\AgdaBound{is}\<%
\\
%
\>[2]\AgdaFunction{ix-div}\AgdaSpace{}%
\AgdaSymbol{(}\AgdaBound{i}\AgdaSpace{}%
\AgdaOperator{\AgdaInductiveConstructor{∷}}\AgdaSpace{}%
\AgdaBound{is}\AgdaSymbol{)}\AgdaSpace{}%
\AgdaSymbol{(}\AgdaInductiveConstructor{cons}\AgdaSpace{}%
\AgdaSymbol{⦃}\AgdaSpace{}%
\AgdaInductiveConstructor{refl}\AgdaSpace{}%
\AgdaSymbol{⦄}\AgdaSpace{}%
\AgdaSymbol{⦃}\AgdaSpace{}%
\AgdaBound{pf}\AgdaSpace{}%
\AgdaSymbol{⦄)}\<%
\\
\>[2][@{}l@{\AgdaIndent{0}}]%
\>[4]\AgdaSymbol{=}\AgdaSpace{}%
\AgdaField{Prod.proj₁}\AgdaSpace{}%
\AgdaSymbol{(}\AgdaFunction{F.remQuot}\AgdaSpace{}%
\AgdaSymbol{\AgdaUnderscore{}}\AgdaSpace{}%
\AgdaBound{i}\AgdaSymbol{)}\AgdaSpace{}%
\AgdaOperator{\AgdaInductiveConstructor{∷}}\AgdaSpace{}%
\AgdaFunction{ix-div}\AgdaSpace{}%
\AgdaBound{is}\AgdaSpace{}%
\AgdaBound{pf}\<%
\\
%
\\[\AgdaEmptyExtraSkip]%
%
\>[2]\AgdaFunction{ix-mod}\AgdaSpace{}%
\AgdaBound{is}\AgdaSpace{}%
\AgdaInductiveConstructor{[]}\AgdaSpace{}%
\AgdaSymbol{=}\AgdaSpace{}%
\AgdaBound{is}\<%
\\
%
\>[2]\AgdaFunction{ix-mod}\AgdaSpace{}%
\AgdaSymbol{(}\AgdaBound{i}\AgdaSpace{}%
\AgdaOperator{\AgdaInductiveConstructor{∷}}\AgdaSpace{}%
\AgdaBound{is}\AgdaSymbol{)}\AgdaSpace{}%
\AgdaSymbol{(}\AgdaInductiveConstructor{cons}\AgdaSpace{}%
\AgdaSymbol{\{}\AgdaArgument{m}\AgdaSpace{}%
\AgdaSymbol{=}\AgdaSpace{}%
\AgdaBound{m}\AgdaSymbol{\}}\AgdaSpace{}%
\AgdaSymbol{⦃}\AgdaSpace{}%
\AgdaInductiveConstructor{refl}\AgdaSpace{}%
\AgdaSymbol{⦄}\AgdaSpace{}%
\AgdaSymbol{⦃}\AgdaSpace{}%
\AgdaBound{pf}\AgdaSpace{}%
\AgdaSymbol{⦄)}\<%
\\
\>[2][@{}l@{\AgdaIndent{0}}]%
\>[4]\AgdaSymbol{=}\AgdaSpace{}%
\AgdaField{Prod.proj₂}\AgdaSpace{}%
\AgdaSymbol{(}\AgdaFunction{F.remQuot}\AgdaSpace{}%
\AgdaSymbol{\{}\AgdaBound{m}\AgdaSymbol{\}}\AgdaSpace{}%
\AgdaSymbol{\AgdaUnderscore{}}\AgdaSpace{}%
\AgdaBound{i}\AgdaSymbol{)}\AgdaSpace{}%
\AgdaOperator{\AgdaInductiveConstructor{∷}}\AgdaSpace{}%
\AgdaFunction{ix-mod}\AgdaSpace{}%
\AgdaBound{is}\AgdaSpace{}%
\AgdaBound{pf}\<%
\\
%
\\[\AgdaEmptyExtraSkip]%
%
\>[2]\AgdaFunction{ix-combine}\AgdaSpace{}%
\AgdaBound{i}\AgdaSpace{}%
\AgdaBound{j}\AgdaSpace{}%
\AgdaInductiveConstructor{[]}\AgdaSpace{}%
\AgdaSymbol{=}\AgdaSpace{}%
\AgdaBound{j}\<%
\\
%
\>[2]\AgdaFunction{ix-combine}\AgdaSpace{}%
\AgdaSymbol{(}\AgdaBound{i}\AgdaSpace{}%
\AgdaOperator{\AgdaInductiveConstructor{∷}}\AgdaSpace{}%
\AgdaBound{is}\AgdaSymbol{)}\AgdaSpace{}%
\AgdaSymbol{(}\AgdaBound{j}\AgdaSpace{}%
\AgdaOperator{\AgdaInductiveConstructor{∷}}\AgdaSpace{}%
\AgdaBound{js}\AgdaSymbol{)}\AgdaSpace{}%
\AgdaSymbol{(}\AgdaInductiveConstructor{cons}\AgdaSpace{}%
\AgdaSymbol{⦃}\AgdaSpace{}%
\AgdaInductiveConstructor{refl}\AgdaSpace{}%
\AgdaSymbol{⦄}\AgdaSpace{}%
\AgdaSymbol{⦃}\AgdaSpace{}%
\AgdaBound{ps}\AgdaSpace{}%
\AgdaSymbol{⦄)}\<%
\\
\>[2][@{}l@{\AgdaIndent{0}}]%
\>[4]\AgdaSymbol{=}\AgdaSpace{}%
\AgdaFunction{F.combine}\AgdaSpace{}%
\AgdaBound{i}\AgdaSpace{}%
\AgdaBound{j}\AgdaSpace{}%
\AgdaOperator{\AgdaInductiveConstructor{∷}}\AgdaSpace{}%
\AgdaFunction{ix-combine}\AgdaSpace{}%
\AgdaBound{is}\AgdaSpace{}%
\AgdaBound{js}\AgdaSpace{}%
\AgdaBound{ps}\<%
\end{code}
With these operations, definitions of \AF{selb} and \AF{imapb}
are:

\begin{mathpar}
\codeblock{\begin{code}%
%
\>[2]\AgdaFunction{selb}\AgdaSpace{}%
\AgdaSymbol{:}\AgdaSpace{}%
\AgdaFunction{Ar}\AgdaSpace{}%
\AgdaGeneralizable{q}\AgdaSpace{}%
\AgdaGeneralizable{X}\AgdaSpace{}%
\AgdaSymbol{→}\AgdaSpace{}%
\AgdaGeneralizable{s}\AgdaSpace{}%
\AgdaOperator{\AgdaFunction{*}}\AgdaSpace{}%
\AgdaGeneralizable{p}\AgdaSpace{}%
\AgdaOperator{\AgdaFunction{≈}}\AgdaSpace{}%
\AgdaGeneralizable{q}\AgdaSpace{}%
\AgdaSymbol{→}\AgdaSpace{}%
\AgdaFunction{Ar}\AgdaSpace{}%
\AgdaGeneralizable{s}\AgdaSpace{}%
\AgdaSymbol{(}\AgdaFunction{Ar}\AgdaSpace{}%
\AgdaGeneralizable{p}\AgdaSpace{}%
\AgdaGeneralizable{X}\AgdaSymbol{)}\<%
\\
%
\>[2]\AgdaFunction{selb}\AgdaSpace{}%
\AgdaBound{a}\AgdaSpace{}%
\AgdaBound{p}\AgdaSpace{}%
\AgdaBound{i}\AgdaSpace{}%
\AgdaBound{j}\AgdaSpace{}%
\AgdaSymbol{=}\AgdaSpace{}%
\AgdaBound{a}\AgdaSpace{}%
\AgdaSymbol{(}\AgdaFunction{ix-combine}\AgdaSpace{}%
\AgdaBound{i}\AgdaSpace{}%
\AgdaBound{j}\AgdaSpace{}%
\AgdaBound{p}\AgdaSymbol{)}\<%
\end{code}}
\and
\codeblock{\begin{code}%
%
\>[2]\AgdaFunction{imapb}\AgdaSpace{}%
\AgdaSymbol{:}\AgdaSpace{}%
\AgdaFunction{Ar}\AgdaSpace{}%
\AgdaGeneralizable{s}\AgdaSpace{}%
\AgdaSymbol{(}\AgdaFunction{Ar}\AgdaSpace{}%
\AgdaGeneralizable{p}\AgdaSpace{}%
\AgdaGeneralizable{X}\AgdaSymbol{)}\AgdaSpace{}%
\AgdaSymbol{→}\AgdaSpace{}%
\AgdaGeneralizable{s}\AgdaSpace{}%
\AgdaOperator{\AgdaFunction{*}}\AgdaSpace{}%
\AgdaGeneralizable{p}\AgdaSpace{}%
\AgdaOperator{\AgdaFunction{≈}}\AgdaSpace{}%
\AgdaGeneralizable{q}\AgdaSpace{}%
\AgdaSymbol{→}\AgdaSpace{}%
\AgdaFunction{Ar}\AgdaSpace{}%
\AgdaGeneralizable{q}\AgdaSpace{}%
\AgdaGeneralizable{X}\<%
\\
%
\>[2]\AgdaFunction{imapb}\AgdaSpace{}%
\AgdaBound{a}\AgdaSpace{}%
\AgdaBound{p}\AgdaSpace{}%
\AgdaBound{i}\AgdaSpace{}%
\AgdaSymbol{=}\AgdaSpace{}%
\AgdaBound{a}\AgdaSpace{}%
\AgdaSymbol{(}\AgdaFunction{ix-div}\AgdaSpace{}%
\AgdaBound{i}\AgdaSpace{}%
\AgdaBound{p}\AgdaSymbol{)}\AgdaSpace{}%
\AgdaSymbol{(}\AgdaFunction{ix-mod}\AgdaSpace{}%
\AgdaBound{i}\AgdaSpace{}%
\AgdaBound{p}\AgdaSymbol{)}\<%
\end{code}}
\end{mathpar}
We define an average pooling that is specialised to
2-dimensional cases as needed per our running example.
\begin{mathpar}
\codeblock{\begin{code}%
%
\>[2]\AgdaFunction{avgp₂}\AgdaSpace{}%
\AgdaSymbol{:}\AgdaSpace{}%
\AgdaSymbol{(}\AgdaBound{m}\AgdaSpace{}%
\AgdaBound{n}\AgdaSpace{}%
\AgdaSymbol{:}\AgdaSpace{}%
\AgdaDatatype{ℕ}\AgdaSymbol{)}\AgdaSpace{}%
\AgdaSymbol{→}\AgdaSpace{}%
\AgdaFunction{Ar}\AgdaSpace{}%
\AgdaSymbol{(}\AgdaBound{m}\AgdaSpace{}%
\AgdaOperator{\AgdaPrimitive{ℕ.*}}\AgdaSpace{}%
\AgdaNumber{2}\AgdaSpace{}%
\AgdaOperator{\AgdaInductiveConstructor{∷}}\AgdaSpace{}%
\AgdaBound{n}\AgdaSpace{}%
\AgdaOperator{\AgdaPrimitive{ℕ.*}}\AgdaSpace{}%
\AgdaNumber{2}\AgdaSpace{}%
\AgdaOperator{\AgdaInductiveConstructor{∷}}\AgdaSpace{}%
\AgdaInductiveConstructor{[]}\AgdaSymbol{)}\AgdaSpace{}%
\AgdaPostulate{ℝ}\AgdaSpace{}%
\AgdaSymbol{→}\AgdaSpace{}%
\AgdaFunction{Ar}\AgdaSpace{}%
\AgdaSymbol{(}\AgdaBound{m}\AgdaSpace{}%
\AgdaOperator{\AgdaInductiveConstructor{∷}}\AgdaSpace{}%
\AgdaBound{n}\AgdaSpace{}%
\AgdaOperator{\AgdaInductiveConstructor{∷}}\AgdaSpace{}%
\AgdaInductiveConstructor{[]}\AgdaSymbol{)}\AgdaSpace{}%
\AgdaPostulate{ℝ}\<%
\\
%
\>[2]\AgdaFunction{avgp₂}\AgdaSpace{}%
\AgdaBound{m}\AgdaSpace{}%
\AgdaBound{n}\AgdaSpace{}%
\AgdaBound{a}\AgdaSpace{}%
\AgdaSymbol{=}\AgdaSpace{}%
\AgdaFunction{map}\AgdaSpace{}%
\AgdaSymbol{((}\AgdaOperator{\AgdaPrimitive{\AgdaUnderscore{}÷}}\AgdaSpace{}%
\AgdaPrimitive{fromℕ}\AgdaSpace{}%
\AgdaNumber{4}\AgdaSymbol{)}\AgdaSpace{}%
\AgdaOperator{\AgdaFunction{∘}}\AgdaSpace{}%
\AgdaFunction{sum}\AgdaSpace{}%
\AgdaOperator{\AgdaPrimitive{\AgdaUnderscore{}+\AgdaUnderscore{}}}\AgdaSpace{}%
\AgdaNumber{0.0}\AgdaSymbol{)}\AgdaSpace{}%
\AgdaSymbol{(}\AgdaFunction{selb}\AgdaSpace{}%
\AgdaBound{a}\AgdaSpace{}%
\AgdaFunction{it}\AgdaSymbol{)}\<%
\end{code}}
\end{mathpar}
Note that \AF{avgp₂} forces a programmer to provide explicit sizes
of the blocked array, and it will not admit arrays of shape such as
$2 * m \times 2 * n$, because $m * 2$ is not definitionally equal to $2 * m$.

With these primitives we implement a forward part of the CNN
as follows.  The \AB{inp} argument is the image of a hand-written digit, all
the other arguments are weights, and the function returns the 10-element vector
with probabilities which digit that is.  Note that type annotations in let are
purely for documentation --- Agda infers them automatically and these lines
can be removed.  Note also that all the \AF{mconv} applications do not require
explicit proofs as Agda can compute them from the shape information provided
in types.
%\begin{mathpar}
%\codeblock{
\begin{code}%
%
\>[2]\AgdaFunction{forward}%
\>[1730I]\AgdaSymbol{:}\AgdaSpace{}%
\AgdaSymbol{(}\AgdaBound{inp}%
\>[18]\AgdaSymbol{:}%
\>[21]\AgdaFunction{Ar}\AgdaSpace{}%
\AgdaSymbol{(}\AgdaNumber{28}\AgdaSpace{}%
\AgdaOperator{\AgdaInductiveConstructor{∷}}\AgdaSpace{}%
\AgdaNumber{28}\AgdaSpace{}%
\AgdaOperator{\AgdaInductiveConstructor{∷}}\AgdaSpace{}%
\AgdaInductiveConstructor{[]}\AgdaSymbol{)}\AgdaSpace{}%
\AgdaPostulate{ℝ}\AgdaSymbol{)}\AgdaSpace{}%
\AgdaSymbol{→}\AgdaSpace{}%
\AgdaSymbol{(}\AgdaBound{k₁}\AgdaSpace{}%
\AgdaSymbol{:}\AgdaSpace{}%
\AgdaFunction{Ar}\AgdaSpace{}%
\AgdaSymbol{(}\AgdaNumber{6}\AgdaSpace{}%
\AgdaOperator{\AgdaInductiveConstructor{∷}}\AgdaSpace{}%
\AgdaNumber{5}\AgdaSpace{}%
\AgdaOperator{\AgdaInductiveConstructor{∷}}\AgdaSpace{}%
\AgdaNumber{5}\AgdaSpace{}%
\AgdaOperator{\AgdaInductiveConstructor{∷}}\AgdaSpace{}%
\AgdaInductiveConstructor{[]}\AgdaSymbol{)}\AgdaSpace{}%
\AgdaPostulate{ℝ}\AgdaSymbol{)}\<%
\\
\>[.][@{}l@{}]\<[1730I]%
\>[10]\AgdaSymbol{→}\AgdaSpace{}%
\AgdaSymbol{(}\AgdaBound{b₁}%
\>[18]\AgdaSymbol{:}%
\>[21]\AgdaFunction{Ar}\AgdaSpace{}%
\AgdaSymbol{(}\AgdaNumber{6}%
\>[28]\AgdaOperator{\AgdaInductiveConstructor{∷}}\AgdaSpace{}%
\AgdaInductiveConstructor{[]}\AgdaSymbol{)}\AgdaSpace{}%
\AgdaPostulate{ℝ}\AgdaSymbol{)}%
\>[42]\AgdaSymbol{→}\AgdaSpace{}%
\AgdaSymbol{(}\AgdaBound{k₂}\AgdaSpace{}%
\AgdaSymbol{:}\AgdaSpace{}%
\AgdaFunction{Ar}\AgdaSpace{}%
\AgdaSymbol{(}\AgdaNumber{12}\AgdaSpace{}%
\AgdaOperator{\AgdaInductiveConstructor{∷}}\AgdaSpace{}%
\AgdaNumber{6}\AgdaSpace{}%
\AgdaOperator{\AgdaInductiveConstructor{∷}}\AgdaSpace{}%
\AgdaNumber{5}\AgdaSpace{}%
\AgdaOperator{\AgdaInductiveConstructor{∷}}\AgdaSpace{}%
\AgdaNumber{5}\AgdaSpace{}%
\AgdaOperator{\AgdaInductiveConstructor{∷}}\AgdaSpace{}%
\AgdaInductiveConstructor{[]}\AgdaSymbol{)}\AgdaSpace{}%
\AgdaPostulate{ℝ}\AgdaSymbol{)}\<%
\\
%
\>[10]\AgdaSymbol{→}\AgdaSpace{}%
\AgdaSymbol{(}\AgdaBound{b₂}%
\>[18]\AgdaSymbol{:}%
\>[21]\AgdaFunction{Ar}\AgdaSpace{}%
\AgdaSymbol{(}\AgdaNumber{12}\AgdaSpace{}%
\AgdaOperator{\AgdaInductiveConstructor{∷}}\AgdaSpace{}%
\AgdaInductiveConstructor{[]}\AgdaSymbol{)}\AgdaSpace{}%
\AgdaPostulate{ℝ}\AgdaSymbol{)}%
\>[42]\AgdaSymbol{→}\AgdaSpace{}%
\AgdaSymbol{(}\AgdaBound{fc}\AgdaSpace{}%
\AgdaSymbol{:}\AgdaSpace{}%
\AgdaFunction{Ar}\AgdaSpace{}%
\AgdaSymbol{(}\AgdaNumber{10}\AgdaSpace{}%
\AgdaOperator{\AgdaInductiveConstructor{∷}}\AgdaSpace{}%
\AgdaNumber{12}\AgdaSpace{}%
\AgdaOperator{\AgdaInductiveConstructor{∷}}\AgdaSpace{}%
\AgdaNumber{1}\AgdaSpace{}%
\AgdaOperator{\AgdaInductiveConstructor{∷}}\AgdaSpace{}%
\AgdaNumber{4}\AgdaSpace{}%
\AgdaOperator{\AgdaInductiveConstructor{∷}}\AgdaSpace{}%
\AgdaNumber{4}\AgdaSpace{}%
\AgdaOperator{\AgdaInductiveConstructor{∷}}\AgdaSpace{}%
\AgdaInductiveConstructor{[]}\AgdaSymbol{)}\AgdaSpace{}%
\AgdaPostulate{ℝ}\AgdaSymbol{)}\<%
\\
%
\>[10]\AgdaSymbol{→}\AgdaSpace{}%
\AgdaSymbol{(}\AgdaBound{b}%
\>[18]\AgdaSymbol{:}%
\>[21]\AgdaFunction{Ar}\AgdaSpace{}%
\AgdaSymbol{(}\AgdaNumber{10}\AgdaSpace{}%
\AgdaOperator{\AgdaInductiveConstructor{∷}}\AgdaSpace{}%
\AgdaInductiveConstructor{[]}\AgdaSymbol{)}\AgdaSpace{}%
\AgdaPostulate{ℝ}\AgdaSymbol{)}%
\>[42]\AgdaSymbol{→}\AgdaSpace{}%
\AgdaFunction{Ar}\AgdaSpace{}%
\AgdaSymbol{(}\AgdaNumber{10}\AgdaSpace{}%
\AgdaOperator{\AgdaInductiveConstructor{∷}}\AgdaSpace{}%
\AgdaNumber{1}\AgdaSpace{}%
\AgdaOperator{\AgdaInductiveConstructor{∷}}\AgdaSpace{}%
\AgdaNumber{1}\AgdaSpace{}%
\AgdaOperator{\AgdaInductiveConstructor{∷}}\AgdaSpace{}%
\AgdaNumber{1}\AgdaSpace{}%
\AgdaOperator{\AgdaInductiveConstructor{∷}}\AgdaSpace{}%
\AgdaNumber{1}\AgdaSpace{}%
\AgdaOperator{\AgdaInductiveConstructor{∷}}\AgdaSpace{}%
\AgdaInductiveConstructor{[]}\AgdaSymbol{)}\AgdaSpace{}%
\AgdaPostulate{ℝ}\<%
\\
%
\>[2]\AgdaFunction{forward}\AgdaSpace{}%
\AgdaBound{inp}\AgdaSpace{}%
\AgdaBound{k₁}\AgdaSpace{}%
\AgdaBound{b₁}\AgdaSpace{}%
\AgdaBound{k₂}\AgdaSpace{}%
\AgdaBound{b₂}\AgdaSpace{}%
\AgdaBound{fc}\AgdaSpace{}%
\AgdaBound{b}\AgdaSpace{}%
\AgdaSymbol{=}\AgdaSpace{}%
\AgdaKeyword{let}\<%
\\
\>[2][@{}l@{\AgdaIndent{0}}]%
\>[6]\AgdaBound{c₁}\AgdaSpace{}%
\AgdaSymbol{:}\AgdaSpace{}%
\AgdaFunction{Ar}\AgdaSpace{}%
\AgdaSymbol{(}\AgdaNumber{6}\AgdaSpace{}%
\AgdaOperator{\AgdaInductiveConstructor{∷}}\AgdaSpace{}%
\AgdaNumber{24}\AgdaSpace{}%
\AgdaOperator{\AgdaInductiveConstructor{∷}}\AgdaSpace{}%
\AgdaNumber{24}\AgdaSpace{}%
\AgdaOperator{\AgdaInductiveConstructor{∷}}\AgdaSpace{}%
\AgdaInductiveConstructor{[]}\AgdaSymbol{)}\AgdaSpace{}%
\AgdaPostulate{ℝ}\<%
\\
%
\>[6]\AgdaBound{c₁}\AgdaSpace{}%
\AgdaSymbol{=}\AgdaSpace{}%
\AgdaFunction{logistic}\AgdaSpace{}%
\AgdaOperator{\AgdaFunction{\$}}\AgdaSpace{}%
\AgdaFunction{mconv}\AgdaSpace{}%
\AgdaBound{inp}\AgdaSpace{}%
\AgdaBound{k₁}\AgdaSpace{}%
\AgdaBound{b₁}\<%
\\
%
\\[\AgdaEmptyExtraSkip]%
%
\>[6]\AgdaBound{s₁}\AgdaSpace{}%
\AgdaSymbol{:}\AgdaSpace{}%
\AgdaFunction{Ar}\AgdaSpace{}%
\AgdaSymbol{(}\AgdaNumber{6}\AgdaSpace{}%
\AgdaOperator{\AgdaInductiveConstructor{∷}}\AgdaSpace{}%
\AgdaNumber{12}\AgdaSpace{}%
\AgdaOperator{\AgdaInductiveConstructor{∷}}\AgdaSpace{}%
\AgdaNumber{12}\AgdaSpace{}%
\AgdaOperator{\AgdaInductiveConstructor{∷}}\AgdaSpace{}%
\AgdaInductiveConstructor{[]}\AgdaSymbol{)}\AgdaSpace{}%
\AgdaPostulate{ℝ}\<%
\\
%
\>[6]\AgdaBound{s₁}\AgdaSpace{}%
\AgdaSymbol{=}\AgdaSpace{}%
\AgdaFunction{unnest}\AgdaSpace{}%
\AgdaSymbol{\{}\AgdaArgument{s}\AgdaSpace{}%
\AgdaSymbol{=}\AgdaSpace{}%
\AgdaNumber{6}\AgdaSpace{}%
\AgdaOperator{\AgdaInductiveConstructor{∷}}\AgdaSpace{}%
\AgdaInductiveConstructor{[]}\AgdaSymbol{\}}\AgdaSpace{}%
\AgdaOperator{\AgdaFunction{\$}}\AgdaSpace{}%
\AgdaFunction{map}\AgdaSpace{}%
\AgdaSymbol{(}\AgdaFunction{avgp₂}\AgdaSpace{}%
\AgdaNumber{12}\AgdaSpace{}%
\AgdaNumber{12}\AgdaSymbol{)}\AgdaSpace{}%
\AgdaSymbol{(}\AgdaFunction{nest}\AgdaSpace{}%
\AgdaBound{c₁}\AgdaSymbol{)}\<%
\\
%
\\[\AgdaEmptyExtraSkip]%
%
\>[6]\AgdaBound{c₂}\AgdaSpace{}%
\AgdaSymbol{:}\AgdaSpace{}%
\AgdaFunction{Ar}\AgdaSpace{}%
\AgdaSymbol{(}\AgdaNumber{12}\AgdaSpace{}%
\AgdaOperator{\AgdaInductiveConstructor{∷}}\AgdaSpace{}%
\AgdaNumber{1}\AgdaSpace{}%
\AgdaOperator{\AgdaInductiveConstructor{∷}}\AgdaSpace{}%
\AgdaNumber{8}\AgdaSpace{}%
\AgdaOperator{\AgdaInductiveConstructor{∷}}\AgdaSpace{}%
\AgdaNumber{8}\AgdaSpace{}%
\AgdaOperator{\AgdaInductiveConstructor{∷}}\AgdaSpace{}%
\AgdaInductiveConstructor{[]}\AgdaSymbol{)}\AgdaSpace{}%
\AgdaPostulate{ℝ}\<%
\\
%
\>[6]\AgdaBound{c₂}\AgdaSpace{}%
\AgdaSymbol{=}\AgdaSpace{}%
\AgdaFunction{logistic}\AgdaSpace{}%
\AgdaOperator{\AgdaFunction{\$}}\AgdaSpace{}%
\AgdaFunction{mconv}%
\>[29]\AgdaBound{s₁}\AgdaSpace{}%
\AgdaBound{k₂}\AgdaSpace{}%
\AgdaBound{b₂}\<%
\\
%
\\[\AgdaEmptyExtraSkip]%
%
\>[6]\AgdaBound{s₂}\AgdaSpace{}%
\AgdaSymbol{:}\AgdaSpace{}%
\AgdaFunction{Ar}\AgdaSpace{}%
\AgdaSymbol{(}\AgdaNumber{12}\AgdaSpace{}%
\AgdaOperator{\AgdaInductiveConstructor{∷}}\AgdaSpace{}%
\AgdaNumber{1}\AgdaSpace{}%
\AgdaOperator{\AgdaInductiveConstructor{∷}}\AgdaSpace{}%
\AgdaNumber{4}\AgdaSpace{}%
\AgdaOperator{\AgdaInductiveConstructor{∷}}\AgdaSpace{}%
\AgdaNumber{4}\AgdaSpace{}%
\AgdaOperator{\AgdaInductiveConstructor{∷}}\AgdaSpace{}%
\AgdaInductiveConstructor{[]}\AgdaSymbol{)}\AgdaSpace{}%
\AgdaPostulate{ℝ}\<%
\\
%
\>[6]\AgdaBound{s₂}\AgdaSpace{}%
\AgdaSymbol{=}\AgdaSpace{}%
\AgdaFunction{unnest}\AgdaSpace{}%
\AgdaSymbol{\{}\AgdaArgument{s}\AgdaSpace{}%
\AgdaSymbol{=}\AgdaSpace{}%
\AgdaNumber{12}\AgdaSpace{}%
\AgdaOperator{\AgdaInductiveConstructor{∷}}\AgdaSpace{}%
\AgdaNumber{1}\AgdaSpace{}%
\AgdaOperator{\AgdaInductiveConstructor{∷}}\AgdaSpace{}%
\AgdaInductiveConstructor{[]}\AgdaSymbol{\}}\AgdaSpace{}%
\AgdaOperator{\AgdaFunction{\$}}\AgdaSpace{}%
\AgdaFunction{map}\AgdaSpace{}%
\AgdaSymbol{(}\AgdaFunction{avgp₂}\AgdaSpace{}%
\AgdaNumber{4}\AgdaSpace{}%
\AgdaNumber{4}\AgdaSymbol{)}\AgdaSpace{}%
\AgdaSymbol{(}\AgdaFunction{nest}\AgdaSpace{}%
\AgdaBound{c₂}\AgdaSymbol{)}\<%
\\
%
\\[\AgdaEmptyExtraSkip]%
%
\>[6]\AgdaBound{r}\AgdaSpace{}%
\AgdaSymbol{=}\AgdaSpace{}%
\AgdaFunction{logistic}\AgdaSpace{}%
\AgdaOperator{\AgdaFunction{\$}}\AgdaSpace{}%
\AgdaFunction{mconv}\AgdaSpace{}%
\AgdaBound{s₂}\AgdaSpace{}%
\AgdaBound{fc}\AgdaSpace{}%
\AgdaBound{b}\<%
\\
\>[2][@{}l@{\AgdaIndent{0}}]%
\>[4]\AgdaKeyword{in}\AgdaSpace{}%
\AgdaBound{r}\<%
\end{code}
%}
%\end{mathpar}


\begin{code}[hide]%
\>[0]\AgdaSymbol{\{-\#}\AgdaSpace{}%
\AgdaKeyword{OPTIONS}%
\>[13]\AgdaPragma{--backtracking-instance-search}\AgdaSpace{}%
\AgdaSymbol{\#-\}}\<%
\\
\>[0]\AgdaKeyword{open}\AgdaSpace{}%
\AgdaKeyword{import}\AgdaSpace{}%
\AgdaModule{Relation.Binary.PropositionalEquality}\<%
\\
\>[0]\AgdaKeyword{open}\AgdaSpace{}%
\AgdaKeyword{import}\AgdaSpace{}%
\AgdaModule{Relation.Nullary}\<%
\\
\>[0]\AgdaKeyword{open}\AgdaSpace{}%
\AgdaKeyword{import}\AgdaSpace{}%
\AgdaModule{Data.List}\AgdaSpace{}%
\AgdaKeyword{using}\AgdaSpace{}%
\AgdaSymbol{(}\AgdaDatatype{List}\AgdaSymbol{;}\AgdaSpace{}%
\AgdaInductiveConstructor{[]}\AgdaSymbol{;}\AgdaSpace{}%
\AgdaOperator{\AgdaInductiveConstructor{\AgdaUnderscore{}∷\AgdaUnderscore{}}}\AgdaSymbol{)}\<%
\\
\>[0]\AgdaKeyword{open}\AgdaSpace{}%
\AgdaKeyword{import}\AgdaSpace{}%
\AgdaModule{Data.Nat}\AgdaSpace{}%
\AgdaKeyword{using}\AgdaSpace{}%
\AgdaSymbol{(}\AgdaDatatype{ℕ}\AgdaSymbol{;}\AgdaSpace{}%
\AgdaInductiveConstructor{zero}\AgdaSymbol{;}\AgdaSpace{}%
\AgdaInductiveConstructor{suc}\AgdaSymbol{)}\<%
\\
\>[0]\AgdaKeyword{open}\AgdaSpace{}%
\AgdaKeyword{import}\AgdaSpace{}%
\AgdaModule{Data.Empty}\<%
\\
\>[0]\AgdaComment{--open\ import\ Function\ hiding\ (⟨\AgdaUnderscore{}⟩)}\<%
\\
%
\\[\AgdaEmptyExtraSkip]%
\>[0]\AgdaComment{--\ Our\ local\ files.}\<%
\\
\>[0]\AgdaKeyword{open}\AgdaSpace{}%
\AgdaKeyword{import}\AgdaSpace{}%
\AgdaModule{arrays}\<%
\\
\>[0]\AgdaKeyword{module}\AgdaSpace{}%
\AgdaModule{\AgdaUnderscore{}}\AgdaSpace{}%
\AgdaKeyword{where}\<%
\end{code}
\section{Embedded DSL \label{sec:edsl}}

Any implementation of automatic differentiation has to decide which operations
are supported.  Surely, it does not make sense to compute derivatives
of a function that opens a file.  Some systems make it possible to extend
the set of operations via traits or typeclasses, yet when all the instances
are resolved, we end up with a set of operations that
can be seen as a definition of an embedded language.
Once we accept the idea of an embedded language, we may consider to implement
it through deep embedding which gives us the following two advantages.
Firstly, AD, extraction and optimisations can be implemented within the
host language without any compiler modifications.  Secondly, as our host
language is a theorem prover, our implementations can include safety
guarantees of our choice.  This is the approach we are taking here.

One challenge lies in identifying the primitives for the embedded language,
so that it is sufficiently powerful to define CNNs and to express AD.
Finding the right level of abstraction for the primitives is non-trivial,
as it heavily depends on the capabilities of the backend language and the
optimisations that we can perform locally.  However, keeping the DSL,
optimisations and extraction within a single framework makes it possible
to experiment with these choices easily.


We have chosen sufficiently generic primitives so that many interesting
functions can be defined within the DSL, yet all the primitives are easily
implementable in the backend.  Another consideration is the ability to define
AD within the same DSL, which is useful because we can do higher-order
derivatives and we can share optimisations between the programs and their
derivatives.
\begin{code}[hide]%
\>[0]\AgdaKeyword{module}\AgdaSpace{}%
\AgdaModule{Lang}\AgdaSpace{}%
\AgdaKeyword{where}\<%
\\
\>[0][@{}l@{\AgdaIndent{0}}]%
\>[2]\AgdaComment{--open\ Array\ hiding\ (sum;\ slide;\ backslide)}\<%
\\
%
\>[2]\AgdaKeyword{open}\AgdaSpace{}%
\AgdaKeyword{import}\AgdaSpace{}%
\AgdaModule{Data.Nat}\AgdaSpace{}%
\AgdaKeyword{using}\AgdaSpace{}%
\AgdaSymbol{(}\AgdaDatatype{ℕ}\AgdaSymbol{;}\AgdaSpace{}%
\AgdaInductiveConstructor{zero}\AgdaSymbol{;}\AgdaSpace{}%
\AgdaInductiveConstructor{suc}\AgdaSymbol{)}\<%
\\
%
\>[2]\AgdaKeyword{infixl}\AgdaSpace{}%
\AgdaNumber{15}\AgdaSpace{}%
\AgdaOperator{\AgdaInductiveConstructor{\AgdaUnderscore{}▹\AgdaUnderscore{}}}\<%
\\
%
\>[2]\AgdaKeyword{module}\AgdaSpace{}%
\AgdaModule{Ar}\AgdaSpace{}%
\AgdaKeyword{where}\<%
\\
\>[2][@{}l@{\AgdaIndent{0}}]%
\>[4]\AgdaKeyword{open}\AgdaSpace{}%
\AgdaModule{Array}\AgdaSpace{}%
\AgdaKeyword{public}\<%
\\
%
\>[4]\AgdaKeyword{open}\AgdaSpace{}%
\AgdaModule{CNN}\AgdaSpace{}%
\AgdaKeyword{public}\<%
\\
%
\\[\AgdaEmptyExtraSkip]%
%
\>[2]\AgdaKeyword{open}\AgdaSpace{}%
\AgdaModule{Ar}\AgdaSpace{}%
\AgdaKeyword{hiding}\AgdaSpace{}%
\AgdaSymbol{(}\AgdaFunction{sum}\AgdaSymbol{;}\AgdaSpace{}%
\AgdaFunction{slide}\AgdaSymbol{;}\AgdaSpace{}%
\AgdaFunction{backslide}\AgdaSymbol{;}\AgdaSpace{}%
\AgdaFunction{imapb}\AgdaSymbol{;}\AgdaSpace{}%
\AgdaFunction{selb}\AgdaSymbol{;}\AgdaSpace{}%
\AgdaFunction{logistic}\AgdaSymbol{)}\<%
\\
\>[0]\<%
\end{code}

We leverage our dependently-typed setting by making our
embedded language well-scoped and intrinsically typed (shaped).
This is very useful as it eliminates a large class of errors that have to do
with wrong variables uses and ill-typed expressions.
Types are given by \AD{IS}: we have arrays of
shape $s$ (denoted by \AC{ar}) and indices of shape $s$
(denoted by \AC{ix}).  Contexts are given by \AD{Ctx} and they are
snoc-lists of \AF{IS}-es.  We use de Bruijn variables which are given
by the relation \AF{\_∈\_} in the usual way. 
%We also define variables \AB{v₁}, \AB{v₂}, \etc{}
%by iteratively applying \AC{vₛ} to \AC{v₀} (definition not shown).
\begin{mathpar}
\codeblock{\begin{code}%
\>[0][@{}l@{\AgdaIndent{1}}]%
\>[2]\AgdaKeyword{data}\AgdaSpace{}%
\AgdaDatatype{IS}\AgdaSpace{}%
\AgdaSymbol{:}\AgdaSpace{}%
\AgdaPrimitive{Set}\AgdaSpace{}%
\AgdaKeyword{where}\<%
\\
\>[2][@{}l@{\AgdaIndent{0}}]%
\>[4]\AgdaInductiveConstructor{ix}%
\>[8]\AgdaSymbol{:}\AgdaSpace{}%
\AgdaDatatype{S}\AgdaSpace{}%
\AgdaSymbol{→}\AgdaSpace{}%
\AgdaDatatype{IS}\<%
\\
%
\>[4]\AgdaInductiveConstructor{ar}%
\>[8]\AgdaSymbol{:}\AgdaSpace{}%
\AgdaDatatype{S}\AgdaSpace{}%
\AgdaSymbol{→}\AgdaSpace{}%
\AgdaDatatype{IS}\<%
\end{code}}
\and
\codeblock{\begin{code}%
%
\>[2]\AgdaKeyword{data}\AgdaSpace{}%
\AgdaDatatype{Ctx}\AgdaSpace{}%
\AgdaSymbol{:}\AgdaSpace{}%
\AgdaPrimitive{Set}\AgdaSpace{}%
\AgdaKeyword{where}\<%
\\
\>[2][@{}l@{\AgdaIndent{0}}]%
\>[4]\AgdaInductiveConstructor{ε}%
\>[9]\AgdaSymbol{:}\AgdaSpace{}%
\AgdaDatatype{Ctx}\<%
\\
%
\>[4]\AgdaOperator{\AgdaInductiveConstructor{\AgdaUnderscore{}▹\AgdaUnderscore{}}}%
\>[9]\AgdaSymbol{:}\AgdaSpace{}%
\AgdaDatatype{Ctx}\AgdaSpace{}%
\AgdaSymbol{→}\AgdaSpace{}%
\AgdaDatatype{IS}\AgdaSpace{}%
\AgdaSymbol{→}\AgdaSpace{}%
\AgdaDatatype{Ctx}\<%
\end{code}}
\and
\codeblock{\begin{code}[hide]%
%
\>[2]\AgdaKeyword{variable}\<%
\\
\>[2][@{}l@{\AgdaIndent{0}}]%
\>[4]\AgdaGeneralizable{Γ}\AgdaSpace{}%
\AgdaGeneralizable{Δ}\AgdaSpace{}%
\AgdaGeneralizable{Ξ}\AgdaSpace{}%
\AgdaGeneralizable{Ψ}\AgdaSpace{}%
\AgdaSymbol{:}\AgdaSpace{}%
\AgdaDatatype{Ctx}\<%
\\
%
\>[4]\AgdaGeneralizable{is}\AgdaSpace{}%
\AgdaGeneralizable{ip}\AgdaSpace{}%
\AgdaGeneralizable{iq}\AgdaSpace{}%
\AgdaGeneralizable{ir}\AgdaSpace{}%
\AgdaSymbol{:}\AgdaSpace{}%
\AgdaDatatype{IS}\<%
\end{code}
\begin{code}%
%
\>[2]\AgdaKeyword{data}\AgdaSpace{}%
\AgdaOperator{\AgdaDatatype{\AgdaUnderscore{}∈\AgdaUnderscore{}}}\AgdaSpace{}%
\AgdaSymbol{:}\AgdaSpace{}%
\AgdaDatatype{IS}\AgdaSpace{}%
\AgdaSymbol{→}\AgdaSpace{}%
\AgdaDatatype{Ctx}\AgdaSpace{}%
\AgdaSymbol{→}\AgdaSpace{}%
\AgdaPrimitive{Set}\AgdaSpace{}%
\AgdaKeyword{where}\<%
\\
\>[2][@{}l@{\AgdaIndent{0}}]%
\>[4]\AgdaInductiveConstructor{v₀}%
\>[8]\AgdaSymbol{:}\AgdaSpace{}%
\AgdaGeneralizable{is}\AgdaSpace{}%
\AgdaOperator{\AgdaDatatype{∈}}\AgdaSpace{}%
\AgdaSymbol{(}\AgdaGeneralizable{Γ}\AgdaSpace{}%
\AgdaOperator{\AgdaInductiveConstructor{▹}}\AgdaSpace{}%
\AgdaGeneralizable{is}\AgdaSymbol{)}\<%
\\
%
\>[4]\AgdaInductiveConstructor{vₛ}%
\>[8]\AgdaSymbol{:}\AgdaSpace{}%
\AgdaGeneralizable{is}\AgdaSpace{}%
\AgdaOperator{\AgdaDatatype{∈}}\AgdaSpace{}%
\AgdaGeneralizable{Γ}\AgdaSpace{}%
\AgdaSymbol{→}\AgdaSpace{}%
\AgdaGeneralizable{is}\AgdaSpace{}%
\AgdaOperator{\AgdaDatatype{∈}}\AgdaSpace{}%
\AgdaSymbol{(}\AgdaGeneralizable{Γ}\AgdaSpace{}%
\AgdaOperator{\AgdaInductiveConstructor{▹}}\AgdaSpace{}%
\AgdaGeneralizable{ip}\AgdaSymbol{)}\<%
\end{code}}
\end{mathpar}
Note that while our contexts are non-dependent (\ie{} types do not depend on the
members of the context), we use non-trivial dependencies within constructors.
The embedded language does not have a notion of shape as a value, therefore all
the shape dependencies are handled by Agda, keeping our language simply
typed (shaped).  This is helpful when it comes to writing
embedded programs.% XXX: more explanation? 
\begin{code}[hide]%
%
\>[2]\AgdaComment{--pattern\ v₀\ =\ v₀}\<%
\\
\>[0]\AgdaComment{--\ \ pattern\ v₁\ =\ vₛ\ v₀}\<%
\\
\>[0]\AgdaComment{--\ \ pattern\ v₂\ =\ vₛ\ v₁}\<%
\\
\>[0]\AgdaComment{--\ \ pattern\ v₃\ =\ vₛ\ v₂}\<%
\\
\>[0]\AgdaComment{--\ \ pattern\ v₄\ =\ vₛ\ v₃}\<%
\\
\>[0]\AgdaComment{--\ \ pattern\ v₅\ =\ vₛ\ v₄}\<%
\\
\>[0]\AgdaComment{--\ \ pattern\ v₆\ =\ vₛ\ v₅}\<%
\\
\>[0]\AgdaComment{--\ \ pattern\ v₇\ =\ vₛ\ v₆}\<%
\\
\>[0]\AgdaComment{--\ \ pattern\ v₈\ =\ vₛ\ v₇}\<%
\\
\>[0]\AgdaComment{--\ \ pattern\ v₉\ =\ vₛ\ v₈}\<%
\\
%
\\[\AgdaEmptyExtraSkip]%
\>[0][@{}l@{\AgdaIndent{0}}]%
\>[2]\AgdaKeyword{infixl}\AgdaSpace{}%
\AgdaNumber{10}\AgdaSpace{}%
\AgdaOperator{\AgdaInductiveConstructor{\AgdaUnderscore{}⊞\AgdaUnderscore{}}}\<%
\\
%
\>[2]\AgdaKeyword{infixl}\AgdaSpace{}%
\AgdaNumber{15}\AgdaSpace{}%
\AgdaOperator{\AgdaInductiveConstructor{\AgdaUnderscore{}⊠\AgdaUnderscore{}}}\<%
\end{code}

All arrays in our language are assumed to be arrays of reals.  Our contexts
do not carry array element types, and we distinguish
singleton arrays of shape \AF{unit} (\emph{scalars}) that will be mapped
into the type that represents real numbers in the backend \eg{} double.
The language supports two binary operations (addition and multiplication)
that are given by \AD{Bop}.
\begin{mathpar}
\codeblock{\begin{code}%
%
\>[2]\AgdaFunction{unit}\AgdaSpace{}%
\AgdaSymbol{:}\AgdaSpace{}%
\AgdaDatatype{S}\<%
\\
%
\>[2]\AgdaFunction{unit}\AgdaSpace{}%
\AgdaSymbol{=}\AgdaSpace{}%
\AgdaInductiveConstructor{[]}\<%
\end{code}}
\and
\codeblock{\begin{code}%
%
\>[2]\AgdaKeyword{data}\AgdaSpace{}%
\AgdaDatatype{Bop}\AgdaSpace{}%
\AgdaSymbol{:}\AgdaSpace{}%
\AgdaPrimitive{Set}\AgdaSpace{}%
\AgdaKeyword{where}\<%
\\
\>[2][@{}l@{\AgdaIndent{0}}]%
\>[4]\AgdaInductiveConstructor{plus}\AgdaSpace{}%
\AgdaInductiveConstructor{mul}\AgdaSpace{}%
\AgdaSymbol{:}\AgdaSpace{}%
\AgdaDatatype{Bop}\<%
\end{code}}
\end{mathpar}

The embedded language \AF{E} includes: variables \AC{var}; constants 0 and 1
(of arbitrary shape) given by \AC{zero} and \AC{one} correspondingly; three
flavours of array constructor/eliminator pairs given by \AC{imaps}/\AC{sels},
\AC{imap}/\AC{sel} and \AC{imapb}/\AC{selb}; summation \AC{sum}; conditional
\AC{zero-but} where the predicate is fixed to equality of two indices and the
else branch is zero; \AC{slide} and \AC{backslide} exactly as described before;
numerical operations which includes \AC{logistic}, plus,
multiplication, division by a constant \AC{scaledown}, and unary \AC{minus};
finally, let bindings for arrays are given by \AC{let′}.
The definition of the embedded language \AF{E} follows.  We also introduce the
syntax for infix plus and multiplication denoted by \AC{⊞} and \AC{⊠}
correspondingly.
%\begin{mathpar}
%\codeblock{
\begin{code}%
%
\>[2]\AgdaKeyword{data}\AgdaSpace{}%
\AgdaDatatype{E}\AgdaSpace{}%
\AgdaSymbol{:}\AgdaSpace{}%
\AgdaDatatype{Ctx}\AgdaSpace{}%
\AgdaSymbol{→}\AgdaSpace{}%
\AgdaDatatype{IS}\AgdaSpace{}%
\AgdaSymbol{→}\AgdaSpace{}%
\AgdaPrimitive{Set}\AgdaSpace{}%
\AgdaKeyword{where}\<%
\\
\>[2][@{}l@{\AgdaIndent{0}}]%
\>[4]\AgdaInductiveConstructor{var}%
\>[15]\AgdaSymbol{:}\AgdaSpace{}%
\AgdaGeneralizable{is}\AgdaSpace{}%
\AgdaOperator{\AgdaDatatype{∈}}\AgdaSpace{}%
\AgdaGeneralizable{Γ}\AgdaSpace{}%
\AgdaSymbol{→}\AgdaSpace{}%
\AgdaDatatype{E}\AgdaSpace{}%
\AgdaGeneralizable{Γ}\AgdaSpace{}%
\AgdaGeneralizable{is}\<%
\\
%
\>[4]\AgdaInductiveConstructor{zero}%
\>[15]\AgdaSymbol{:}\AgdaSpace{}%
\AgdaDatatype{E}\AgdaSpace{}%
\AgdaGeneralizable{Γ}\AgdaSpace{}%
\AgdaSymbol{(}\AgdaInductiveConstructor{ar}\AgdaSpace{}%
\AgdaGeneralizable{s}\AgdaSymbol{)}\<%
\\
%
\>[4]\AgdaInductiveConstructor{one}%
\>[15]\AgdaSymbol{:}\AgdaSpace{}%
\AgdaDatatype{E}\AgdaSpace{}%
\AgdaGeneralizable{Γ}\AgdaSpace{}%
\AgdaSymbol{(}\AgdaInductiveConstructor{ar}\AgdaSpace{}%
\AgdaGeneralizable{s}\AgdaSymbol{)}\<%
\\
%
\\[\AgdaEmptyExtraSkip]%
%
\>[4]\AgdaInductiveConstructor{imaps}%
\>[15]\AgdaSymbol{:}\AgdaSpace{}%
\AgdaDatatype{E}\AgdaSpace{}%
\AgdaSymbol{(}\AgdaGeneralizable{Γ}\AgdaSpace{}%
\AgdaOperator{\AgdaInductiveConstructor{▹}}\AgdaSpace{}%
\AgdaInductiveConstructor{ix}\AgdaSpace{}%
\AgdaGeneralizable{s}\AgdaSymbol{)}\AgdaSpace{}%
\AgdaSymbol{(}\AgdaInductiveConstructor{ar}\AgdaSpace{}%
\AgdaFunction{unit}\AgdaSymbol{)}\AgdaSpace{}%
\AgdaSymbol{→}\AgdaSpace{}%
\AgdaDatatype{E}\AgdaSpace{}%
\AgdaGeneralizable{Γ}\AgdaSpace{}%
\AgdaSymbol{(}\AgdaInductiveConstructor{ar}\AgdaSpace{}%
\AgdaGeneralizable{s}\AgdaSymbol{)}\<%
\\
%
\>[4]\AgdaInductiveConstructor{sels}%
\>[15]\AgdaSymbol{:}\AgdaSpace{}%
\AgdaDatatype{E}\AgdaSpace{}%
\AgdaGeneralizable{Γ}\AgdaSpace{}%
\AgdaSymbol{(}\AgdaInductiveConstructor{ar}\AgdaSpace{}%
\AgdaGeneralizable{s}\AgdaSymbol{)}\AgdaSpace{}%
\AgdaSymbol{→}\AgdaSpace{}%
\AgdaDatatype{E}\AgdaSpace{}%
\AgdaGeneralizable{Γ}\AgdaSpace{}%
\AgdaSymbol{(}\AgdaInductiveConstructor{ix}\AgdaSpace{}%
\AgdaGeneralizable{s}\AgdaSymbol{)}\AgdaSpace{}%
\AgdaSymbol{→}\AgdaSpace{}%
\AgdaDatatype{E}\AgdaSpace{}%
\AgdaGeneralizable{Γ}\AgdaSpace{}%
\AgdaSymbol{(}\AgdaInductiveConstructor{ar}\AgdaSpace{}%
\AgdaFunction{unit}\AgdaSymbol{)}\<%
\\
%
\\[\AgdaEmptyExtraSkip]%
%
\>[4]\AgdaInductiveConstructor{imap}%
\>[15]\AgdaSymbol{:}\AgdaSpace{}%
\AgdaDatatype{E}\AgdaSpace{}%
\AgdaSymbol{(}\AgdaGeneralizable{Γ}\AgdaSpace{}%
\AgdaOperator{\AgdaInductiveConstructor{▹}}\AgdaSpace{}%
\AgdaInductiveConstructor{ix}\AgdaSpace{}%
\AgdaGeneralizable{s}\AgdaSymbol{)}\AgdaSpace{}%
\AgdaSymbol{(}\AgdaInductiveConstructor{ar}\AgdaSpace{}%
\AgdaGeneralizable{p}\AgdaSymbol{)}\AgdaSpace{}%
\AgdaSymbol{→}\AgdaSpace{}%
\AgdaDatatype{E}\AgdaSpace{}%
\AgdaGeneralizable{Γ}\AgdaSpace{}%
\AgdaSymbol{(}\AgdaInductiveConstructor{ar}\AgdaSpace{}%
\AgdaSymbol{(}\AgdaGeneralizable{s}\AgdaSpace{}%
\AgdaOperator{\AgdaFunction{⊗}}\AgdaSpace{}%
\AgdaGeneralizable{p}\AgdaSymbol{))}\<%
\\
%
\>[4]\AgdaInductiveConstructor{sel}%
\>[15]\AgdaSymbol{:}\AgdaSpace{}%
\AgdaDatatype{E}\AgdaSpace{}%
\AgdaGeneralizable{Γ}\AgdaSpace{}%
\AgdaSymbol{(}\AgdaInductiveConstructor{ar}\AgdaSpace{}%
\AgdaSymbol{(}\AgdaGeneralizable{s}\AgdaSpace{}%
\AgdaOperator{\AgdaFunction{⊗}}\AgdaSpace{}%
\AgdaGeneralizable{p}\AgdaSymbol{))}\AgdaSpace{}%
\AgdaSymbol{→}\AgdaSpace{}%
\AgdaDatatype{E}\AgdaSpace{}%
\AgdaGeneralizable{Γ}\AgdaSpace{}%
\AgdaSymbol{(}\AgdaInductiveConstructor{ix}\AgdaSpace{}%
\AgdaGeneralizable{s}\AgdaSymbol{)}\AgdaSpace{}%
\AgdaSymbol{→}\AgdaSpace{}%
\AgdaDatatype{E}\AgdaSpace{}%
\AgdaGeneralizable{Γ}\AgdaSpace{}%
\AgdaSymbol{(}\AgdaInductiveConstructor{ar}\AgdaSpace{}%
\AgdaGeneralizable{p}\AgdaSymbol{)}\<%
\\
%
\\[\AgdaEmptyExtraSkip]%
%
\>[4]\AgdaInductiveConstructor{imapb}%
\>[15]\AgdaSymbol{:}\AgdaSpace{}%
\AgdaGeneralizable{s}\AgdaSpace{}%
\AgdaOperator{\AgdaFunction{*}}\AgdaSpace{}%
\AgdaGeneralizable{p}\AgdaSpace{}%
\AgdaOperator{\AgdaFunction{≈}}\AgdaSpace{}%
\AgdaGeneralizable{q}\AgdaSpace{}%
\AgdaSymbol{→}\AgdaSpace{}%
\AgdaDatatype{E}\AgdaSpace{}%
\AgdaSymbol{(}\AgdaGeneralizable{Γ}\AgdaSpace{}%
\AgdaOperator{\AgdaInductiveConstructor{▹}}\AgdaSpace{}%
\AgdaInductiveConstructor{ix}\AgdaSpace{}%
\AgdaGeneralizable{s}\AgdaSymbol{)}\AgdaSpace{}%
\AgdaSymbol{(}\AgdaInductiveConstructor{ar}\AgdaSpace{}%
\AgdaGeneralizable{p}\AgdaSymbol{)}\AgdaSpace{}%
\AgdaSymbol{→}\AgdaSpace{}%
\AgdaDatatype{E}\AgdaSpace{}%
\AgdaGeneralizable{Γ}\AgdaSpace{}%
\AgdaSymbol{(}\AgdaInductiveConstructor{ar}\AgdaSpace{}%
\AgdaGeneralizable{q}\AgdaSymbol{)}\<%
\\
%
\>[4]\AgdaInductiveConstructor{selb}%
\>[15]\AgdaSymbol{:}\AgdaSpace{}%
\AgdaGeneralizable{s}\AgdaSpace{}%
\AgdaOperator{\AgdaFunction{*}}\AgdaSpace{}%
\AgdaGeneralizable{p}\AgdaSpace{}%
\AgdaOperator{\AgdaFunction{≈}}\AgdaSpace{}%
\AgdaGeneralizable{q}\AgdaSpace{}%
\AgdaSymbol{→}\AgdaSpace{}%
\AgdaDatatype{E}\AgdaSpace{}%
\AgdaGeneralizable{Γ}\AgdaSpace{}%
\AgdaSymbol{(}\AgdaInductiveConstructor{ar}\AgdaSpace{}%
\AgdaGeneralizable{q}\AgdaSymbol{)}\AgdaSpace{}%
\AgdaSymbol{→}\AgdaSpace{}%
\AgdaDatatype{E}\AgdaSpace{}%
\AgdaGeneralizable{Γ}\AgdaSpace{}%
\AgdaSymbol{(}\AgdaInductiveConstructor{ix}\AgdaSpace{}%
\AgdaGeneralizable{s}\AgdaSymbol{)}\AgdaSpace{}%
\AgdaSymbol{→}\AgdaSpace{}%
\AgdaDatatype{E}\AgdaSpace{}%
\AgdaGeneralizable{Γ}\AgdaSpace{}%
\AgdaSymbol{(}\AgdaInductiveConstructor{ar}\AgdaSpace{}%
\AgdaGeneralizable{p}\AgdaSymbol{)}\<%
\\
%
\\[\AgdaEmptyExtraSkip]%
%
\>[4]\AgdaInductiveConstructor{sum}%
\>[15]\AgdaSymbol{:}\AgdaSpace{}%
\AgdaDatatype{E}\AgdaSpace{}%
\AgdaSymbol{(}\AgdaGeneralizable{Γ}\AgdaSpace{}%
\AgdaOperator{\AgdaInductiveConstructor{▹}}\AgdaSpace{}%
\AgdaInductiveConstructor{ix}\AgdaSpace{}%
\AgdaGeneralizable{s}\AgdaSymbol{)}\AgdaSpace{}%
\AgdaSymbol{(}\AgdaInductiveConstructor{ar}\AgdaSpace{}%
\AgdaGeneralizable{p}\AgdaSymbol{)}\AgdaSpace{}%
\AgdaSymbol{→}\AgdaSpace{}%
\AgdaDatatype{E}\AgdaSpace{}%
\AgdaGeneralizable{Γ}\AgdaSpace{}%
\AgdaSymbol{(}\AgdaInductiveConstructor{ar}\AgdaSpace{}%
\AgdaGeneralizable{p}\AgdaSymbol{)}\<%
\\
%
\>[4]\AgdaInductiveConstructor{zero-but}%
\>[15]\AgdaSymbol{:}\AgdaSpace{}%
\AgdaDatatype{E}\AgdaSpace{}%
\AgdaGeneralizable{Γ}\AgdaSpace{}%
\AgdaSymbol{(}\AgdaInductiveConstructor{ix}\AgdaSpace{}%
\AgdaGeneralizable{s}\AgdaSymbol{)}\AgdaSpace{}%
\AgdaSymbol{→}\AgdaSpace{}%
\AgdaDatatype{E}\AgdaSpace{}%
\AgdaGeneralizable{Γ}\AgdaSpace{}%
\AgdaSymbol{(}\AgdaInductiveConstructor{ix}\AgdaSpace{}%
\AgdaGeneralizable{s}\AgdaSymbol{)}\AgdaSpace{}%
\AgdaSymbol{→}\AgdaSpace{}%
\AgdaDatatype{E}\AgdaSpace{}%
\AgdaGeneralizable{Γ}\AgdaSpace{}%
\AgdaSymbol{(}\AgdaInductiveConstructor{ar}\AgdaSpace{}%
\AgdaGeneralizable{p}\AgdaSymbol{)}\AgdaSpace{}%
\AgdaSymbol{→}\AgdaSpace{}%
\AgdaDatatype{E}\AgdaSpace{}%
\AgdaGeneralizable{Γ}\AgdaSpace{}%
\AgdaSymbol{(}\AgdaInductiveConstructor{ar}\AgdaSpace{}%
\AgdaGeneralizable{p}\AgdaSymbol{)}\<%
\\
%
\\[\AgdaEmptyExtraSkip]%
%
\>[4]\AgdaInductiveConstructor{slide}%
\>[15]\AgdaSymbol{:}\AgdaSpace{}%
\AgdaDatatype{E}\AgdaSpace{}%
\AgdaGeneralizable{Γ}\AgdaSpace{}%
\AgdaSymbol{(}\AgdaInductiveConstructor{ix}\AgdaSpace{}%
\AgdaGeneralizable{s}\AgdaSymbol{)}\AgdaSpace{}%
\AgdaSymbol{→}\AgdaSpace{}%
\AgdaGeneralizable{s}\AgdaSpace{}%
\AgdaOperator{\AgdaFunction{+}}\AgdaSpace{}%
\AgdaGeneralizable{p}\AgdaSpace{}%
\AgdaOperator{\AgdaFunction{≈}}\AgdaSpace{}%
\AgdaGeneralizable{r}\AgdaSpace{}%
\AgdaSymbol{→}\AgdaSpace{}%
\AgdaDatatype{E}\AgdaSpace{}%
\AgdaGeneralizable{Γ}\AgdaSpace{}%
\AgdaSymbol{(}\AgdaInductiveConstructor{ar}\AgdaSpace{}%
\AgdaGeneralizable{r}\AgdaSymbol{)}\AgdaSpace{}%
\AgdaSymbol{→}\AgdaSpace{}%
\AgdaOperator{\AgdaFunction{suc}}\AgdaSpace{}%
\AgdaGeneralizable{p}\AgdaSpace{}%
\AgdaOperator{\AgdaFunction{≈}}\AgdaSpace{}%
\AgdaGeneralizable{u}\AgdaSpace{}%
\AgdaSymbol{→}\AgdaSpace{}%
\AgdaDatatype{E}\AgdaSpace{}%
\AgdaGeneralizable{Γ}\AgdaSpace{}%
\AgdaSymbol{(}\AgdaInductiveConstructor{ar}\AgdaSpace{}%
\AgdaGeneralizable{u}\AgdaSymbol{)}\<%
\\
%
\>[4]\AgdaInductiveConstructor{backslide}%
\>[15]\AgdaSymbol{:}\AgdaSpace{}%
\AgdaDatatype{E}\AgdaSpace{}%
\AgdaGeneralizable{Γ}\AgdaSpace{}%
\AgdaSymbol{(}\AgdaInductiveConstructor{ix}\AgdaSpace{}%
\AgdaGeneralizable{s}\AgdaSymbol{)}\AgdaSpace{}%
\AgdaSymbol{→}\AgdaSpace{}%
\AgdaDatatype{E}\AgdaSpace{}%
\AgdaGeneralizable{Γ}\AgdaSpace{}%
\AgdaSymbol{(}\AgdaInductiveConstructor{ar}\AgdaSpace{}%
\AgdaGeneralizable{u}\AgdaSymbol{)}\AgdaSpace{}%
\AgdaSymbol{→}\AgdaSpace{}%
\AgdaOperator{\AgdaFunction{suc}}\AgdaSpace{}%
\AgdaGeneralizable{p}\AgdaSpace{}%
\AgdaOperator{\AgdaFunction{≈}}\AgdaSpace{}%
\AgdaGeneralizable{u}\AgdaSpace{}%
\AgdaSymbol{→}\AgdaSpace{}%
\AgdaGeneralizable{s}\AgdaSpace{}%
\AgdaOperator{\AgdaFunction{+}}\AgdaSpace{}%
\AgdaGeneralizable{p}\AgdaSpace{}%
\AgdaOperator{\AgdaFunction{≈}}\AgdaSpace{}%
\AgdaGeneralizable{r}\AgdaSpace{}%
\AgdaSymbol{→}\AgdaSpace{}%
\AgdaDatatype{E}\AgdaSpace{}%
\AgdaGeneralizable{Γ}\AgdaSpace{}%
\AgdaSymbol{(}\AgdaInductiveConstructor{ar}\AgdaSpace{}%
\AgdaGeneralizable{r}\AgdaSymbol{)}\<%
\\
%
\\[\AgdaEmptyExtraSkip]%
%
\>[4]\AgdaInductiveConstructor{logistic}%
\>[15]\AgdaSymbol{:}\AgdaSpace{}%
\AgdaDatatype{E}\AgdaSpace{}%
\AgdaGeneralizable{Γ}\AgdaSpace{}%
\AgdaSymbol{(}\AgdaInductiveConstructor{ar}\AgdaSpace{}%
\AgdaGeneralizable{s}\AgdaSymbol{)}\AgdaSpace{}%
\AgdaSymbol{→}\AgdaSpace{}%
\AgdaDatatype{E}\AgdaSpace{}%
\AgdaGeneralizable{Γ}\AgdaSpace{}%
\AgdaSymbol{(}\AgdaInductiveConstructor{ar}\AgdaSpace{}%
\AgdaGeneralizable{s}\AgdaSymbol{)}\<%
\\
%
\>[4]\AgdaInductiveConstructor{bin}%
\>[15]\AgdaSymbol{:}\AgdaSpace{}%
\AgdaDatatype{Bop}\AgdaSpace{}%
\AgdaSymbol{→}\AgdaSpace{}%
\AgdaDatatype{E}\AgdaSpace{}%
\AgdaGeneralizable{Γ}\AgdaSpace{}%
\AgdaSymbol{(}\AgdaInductiveConstructor{ar}\AgdaSpace{}%
\AgdaGeneralizable{s}\AgdaSymbol{)}\AgdaSpace{}%
\AgdaSymbol{→}\AgdaSpace{}%
\AgdaDatatype{E}\AgdaSpace{}%
\AgdaGeneralizable{Γ}\AgdaSpace{}%
\AgdaSymbol{(}\AgdaInductiveConstructor{ar}\AgdaSpace{}%
\AgdaGeneralizable{s}\AgdaSymbol{)}\AgdaSpace{}%
\AgdaSymbol{→}\AgdaSpace{}%
\AgdaDatatype{E}\AgdaSpace{}%
\AgdaGeneralizable{Γ}\AgdaSpace{}%
\AgdaSymbol{(}\AgdaInductiveConstructor{ar}\AgdaSpace{}%
\AgdaGeneralizable{s}\AgdaSymbol{)}\<%
\\
%
\>[4]\AgdaInductiveConstructor{scaledown}%
\>[15]\AgdaSymbol{:}\AgdaSpace{}%
\AgdaDatatype{ℕ}\AgdaSpace{}%
\AgdaSymbol{→}\AgdaSpace{}%
\AgdaDatatype{E}\AgdaSpace{}%
\AgdaGeneralizable{Γ}\AgdaSpace{}%
\AgdaSymbol{(}\AgdaInductiveConstructor{ar}\AgdaSpace{}%
\AgdaGeneralizable{s}\AgdaSymbol{)}\AgdaSpace{}%
\AgdaSymbol{→}\AgdaSpace{}%
\AgdaDatatype{E}\AgdaSpace{}%
\AgdaGeneralizable{Γ}\AgdaSpace{}%
\AgdaSymbol{(}\AgdaInductiveConstructor{ar}\AgdaSpace{}%
\AgdaGeneralizable{s}\AgdaSymbol{)}\<%
\\
%
\>[4]\AgdaInductiveConstructor{minus}%
\>[15]\AgdaSymbol{:}\AgdaSpace{}%
\AgdaDatatype{E}\AgdaSpace{}%
\AgdaGeneralizable{Γ}\AgdaSpace{}%
\AgdaSymbol{(}\AgdaInductiveConstructor{ar}\AgdaSpace{}%
\AgdaGeneralizable{s}\AgdaSymbol{)}\AgdaSpace{}%
\AgdaSymbol{→}\AgdaSpace{}%
\AgdaDatatype{E}\AgdaSpace{}%
\AgdaGeneralizable{Γ}\AgdaSpace{}%
\AgdaSymbol{(}\AgdaInductiveConstructor{ar}\AgdaSpace{}%
\AgdaGeneralizable{s}\AgdaSymbol{)}\<%
\\
%
\>[4]\AgdaInductiveConstructor{let′}%
\>[15]\AgdaSymbol{:}\AgdaSpace{}%
\AgdaDatatype{E}\AgdaSpace{}%
\AgdaGeneralizable{Γ}\AgdaSpace{}%
\AgdaSymbol{(}\AgdaInductiveConstructor{ar}\AgdaSpace{}%
\AgdaGeneralizable{s}\AgdaSymbol{)}\AgdaSpace{}%
\AgdaSymbol{→}\AgdaSpace{}%
\AgdaDatatype{E}\AgdaSpace{}%
\AgdaSymbol{(}\AgdaGeneralizable{Γ}\AgdaSpace{}%
\AgdaOperator{\AgdaInductiveConstructor{▹}}\AgdaSpace{}%
\AgdaInductiveConstructor{ar}\AgdaSpace{}%
\AgdaGeneralizable{s}\AgdaSymbol{)}\AgdaSpace{}%
\AgdaSymbol{(}\AgdaInductiveConstructor{ar}\AgdaSpace{}%
\AgdaGeneralizable{p}\AgdaSymbol{)}\AgdaSpace{}%
\AgdaSymbol{→}\AgdaSpace{}%
\AgdaDatatype{E}\AgdaSpace{}%
\AgdaGeneralizable{Γ}\AgdaSpace{}%
\AgdaSymbol{(}\AgdaInductiveConstructor{ar}\AgdaSpace{}%
\AgdaGeneralizable{p}\AgdaSymbol{)}\<%
\\
%
\\[\AgdaEmptyExtraSkip]%
%
\>[2]\AgdaKeyword{pattern}\AgdaSpace{}%
\AgdaOperator{\AgdaInductiveConstructor{\AgdaUnderscore{}⊠\AgdaUnderscore{}}}\AgdaSpace{}%
\AgdaBound{a}\AgdaSpace{}%
\AgdaBound{b}\AgdaSpace{}%
\AgdaSymbol{=}\AgdaSpace{}%
\AgdaInductiveConstructor{bin}\AgdaSpace{}%
\AgdaInductiveConstructor{mul}\AgdaSpace{}%
\AgdaBound{a}\AgdaSpace{}%
\AgdaBound{b}\<%
\\
%
\>[2]\AgdaKeyword{pattern}\AgdaSpace{}%
\AgdaOperator{\AgdaInductiveConstructor{\AgdaUnderscore{}⊞\AgdaUnderscore{}}}\AgdaSpace{}%
\AgdaBound{a}\AgdaSpace{}%
\AgdaBound{b}\AgdaSpace{}%
\AgdaSymbol{=}\AgdaSpace{}%
\AgdaInductiveConstructor{bin}\AgdaSpace{}%
\AgdaInductiveConstructor{plus}\AgdaSpace{}%
\AgdaBound{a}\AgdaSpace{}%
\AgdaBound{b}\<%
\end{code}
Let us motivate the presence of three flavours of \AC{imap}/\AC{sel}
constructors.  The difference between \AC{imap} and \AC{imapb} follows
from the previous definitions: the former turns an $s$-shaped array
of $p$-shaped arrays into a $(s ⊗ p)$-shaped array, whereas \AF{imapb}
performs tiling based on the $s * p ≈ q$ equation.  Strictly speaking, 
scalar version of imap, that we call \AC{imaps}, is not needed, because
the same functionality can be achieved with \AC{imap}/\AC{sel}.
However, if \AF{imap} computes a scalar in the body, its resulting shape 
is $s ⊗ \AF{unit}$ which is not definitionally equal to $s$.  Using
\AC{sel} for selecting a scalar from an $s$-shaped array requires
casting the shape into $s ⊗ \AF{unit}$.  Hence every scalar imap or
selection will require transporting over the $s ⊗ \AF{unit} ≡ s$ equality,
This significantly clutters expressivity.  One could quotient the shape
type by the above equality, but this requires switching to a more
powerful type theory such as setoid- or cubical type theory.



\subsection{Evaluation}

We give semantics of our language by interpreting \AD{E} expressions
into \AD{Ar} arrays using combinators that we defined earlier.  
This semantics will be also used to prove that optimisations preserve
the meaning of programs.

\subsubsection{Reals}
We parametrise our semantics by the type of reals.
This makes it possible to abstract away from the implementational
details of numerical encoding which are not relevant here.
We define an interface to reals and their basic operations as follows.
\begin{mathpar}
\codeblock{\begin{code}%
\>[0]\AgdaKeyword{record}\AgdaSpace{}%
\AgdaRecord{Real}\AgdaSpace{}%
\AgdaSymbol{:}\AgdaSpace{}%
\AgdaPrimitive{Set₁}\AgdaSpace{}%
\AgdaKeyword{where}\<%
\\
\>[0][@{}l@{\AgdaIndent{0}}]%
\>[2]\AgdaKeyword{field}\<%
\\
\>[2][@{}l@{\AgdaIndent{0}}]%
\>[4]\AgdaField{R}\AgdaSpace{}%
\AgdaSymbol{:}\AgdaSpace{}%
\AgdaPrimitive{Set}\<%
\\
%
\>[4]\AgdaField{fromℕ}\AgdaSpace{}%
\AgdaSymbol{:}\AgdaSpace{}%
\AgdaDatatype{ℕ}\AgdaSpace{}%
\AgdaSymbol{→}\AgdaSpace{}%
\AgdaField{R}\<%
\\
%
\>[4]\AgdaOperator{\AgdaField{\AgdaUnderscore{}+\AgdaUnderscore{}}}\AgdaSpace{}%
\AgdaOperator{\AgdaField{\AgdaUnderscore{}*\AgdaUnderscore{}}}\AgdaSpace{}%
\AgdaOperator{\AgdaField{\AgdaUnderscore{}÷\AgdaUnderscore{}}}\AgdaSpace{}%
\AgdaSymbol{:}\AgdaSpace{}%
\AgdaField{R}\AgdaSpace{}%
\AgdaSymbol{→}\AgdaSpace{}%
\AgdaField{R}\AgdaSpace{}%
\AgdaSymbol{→}\AgdaSpace{}%
\AgdaField{R}\<%
\\
%
\>[4]\AgdaOperator{\AgdaField{-\AgdaUnderscore{}}}\AgdaSpace{}%
\AgdaOperator{\AgdaField{e\textasciicircum{}\AgdaUnderscore{}}}\AgdaSpace{}%
\AgdaSymbol{:}\AgdaSpace{}%
\AgdaField{R}\AgdaSpace{}%
\AgdaSymbol{→}\AgdaSpace{}%
\AgdaField{R}\<%
\end{code}}
\and
\codeblock{\begin{code}[hide]%
%
\>[2]\AgdaKeyword{infixl}\AgdaSpace{}%
\AgdaNumber{10}\AgdaSpace{}%
\AgdaOperator{\AgdaField{\AgdaUnderscore{}+\AgdaUnderscore{}}}\<%
\\
%
\>[2]\AgdaKeyword{infixl}\AgdaSpace{}%
\AgdaNumber{15}\AgdaSpace{}%
\AgdaOperator{\AgdaField{\AgdaUnderscore{}*\AgdaUnderscore{}}}\<%
\\
%
\>[2]\AgdaKeyword{infixl}\AgdaSpace{}%
\AgdaNumber{15}\AgdaSpace{}%
\AgdaOperator{\AgdaField{\AgdaUnderscore{}÷\AgdaUnderscore{}}}\<%
\end{code}
\begin{code}%
%
\>[2]\AgdaFunction{logisticʳ}\AgdaSpace{}%
\AgdaSymbol{:}\AgdaSpace{}%
\AgdaField{R}\AgdaSpace{}%
\AgdaSymbol{→}\AgdaSpace{}%
\AgdaField{R}\<%
\\
%
\>[2]\AgdaFunction{logisticʳ}\AgdaSpace{}%
\AgdaBound{x}\AgdaSpace{}%
\AgdaSymbol{=}\AgdaSpace{}%
\AgdaField{fromℕ}\AgdaSpace{}%
\AgdaNumber{1}\AgdaSpace{}%
\AgdaOperator{\AgdaField{÷}}\AgdaSpace{}%
\AgdaSymbol{(}\AgdaField{fromℕ}\AgdaSpace{}%
\AgdaNumber{1}\AgdaSpace{}%
\AgdaOperator{\AgdaField{+}}\AgdaSpace{}%
\AgdaOperator{\AgdaField{e\textasciicircum{}}}\AgdaSpace{}%
\AgdaSymbol{(}\AgdaOperator{\AgdaField{-}}\AgdaSpace{}%
\AgdaBound{x}\AgdaSymbol{))}\<%
\end{code}}
\end{mathpar}
We require a type of reals that we call \AR{R}; basic arithmetic operations
that include addition, multiplication, division, unary minus, and exponentiation.
Constants such as zero and one can be defined via \AF{fromℕ} which converts
natural numbers to \AR{R}.  With arithmetic operations in place, logistics function
is a derived notion that we define within the same module for convenience.

\begin{code}[hide]%
\>[0]\AgdaKeyword{module}\AgdaSpace{}%
\AgdaModule{Eval}\AgdaSpace{}%
\AgdaSymbol{(}\AgdaBound{real}\AgdaSpace{}%
\AgdaSymbol{:}\AgdaSpace{}%
\AgdaRecord{Real}\AgdaSymbol{)}\AgdaSpace{}%
\AgdaKeyword{where}\<%
\\
\>[0][@{}l@{\AgdaIndent{0}}]%
\>[2]\AgdaComment{--open\ import\ Data.Float\ as\ F\ renaming\ (Float\ to\ ℝ)\ hiding\ (⌊\AgdaUnderscore{}⌋)}\<%
\\
%
\>[2]\AgdaKeyword{open}\AgdaSpace{}%
\AgdaKeyword{import}\AgdaSpace{}%
\AgdaModule{Data.Unit}\<%
\\
%
\>[2]\AgdaKeyword{open}\AgdaSpace{}%
\AgdaKeyword{import}\AgdaSpace{}%
\AgdaModule{Data.Product}\AgdaSpace{}%
\AgdaKeyword{using}\AgdaSpace{}%
\AgdaSymbol{(}\AgdaOperator{\AgdaFunction{\AgdaUnderscore{}×\AgdaUnderscore{}}}\AgdaSymbol{;}\AgdaSpace{}%
\AgdaField{proj₁}\AgdaSymbol{;}\AgdaSpace{}%
\AgdaField{proj₂}\AgdaSymbol{;}\AgdaSpace{}%
\AgdaOperator{\AgdaInductiveConstructor{\AgdaUnderscore{},\AgdaUnderscore{}}}\AgdaSymbol{)}\<%
\\
%
\>[2]\AgdaKeyword{open}\AgdaSpace{}%
\AgdaKeyword{import}\AgdaSpace{}%
\AgdaModule{Data.Fin}\AgdaSpace{}%
\AgdaKeyword{using}\AgdaSpace{}%
\AgdaSymbol{(}\AgdaDatatype{Fin}\AgdaSymbol{;}\AgdaSpace{}%
\AgdaInductiveConstructor{zero}\AgdaSymbol{;}\AgdaSpace{}%
\AgdaInductiveConstructor{suc}\AgdaSymbol{;}\AgdaSpace{}%
\AgdaOperator{\AgdaFunction{\#\AgdaUnderscore{}}}\AgdaSymbol{)}\<%
\\
%
\>[2]\AgdaKeyword{open}\AgdaSpace{}%
\AgdaKeyword{import}\AgdaSpace{}%
\AgdaModule{Relation.Nullary.Decidable}\<%
\\
%
\>[2]\AgdaKeyword{open}\AgdaSpace{}%
\AgdaKeyword{import}\AgdaSpace{}%
\AgdaModule{Data.Bool}\<%
\\
%
\\[\AgdaEmptyExtraSkip]%
%
\>[2]\AgdaKeyword{open}\AgdaSpace{}%
\AgdaModule{Lang}\<%
\\
%
\>[2]\AgdaKeyword{open}\AgdaSpace{}%
\AgdaModule{Array}\<%
\\
%
\>[2]\AgdaKeyword{open}\AgdaSpace{}%
\AgdaModule{Real}\AgdaSpace{}%
\AgdaBound{real}\<%
\end{code}

We interpret expressions in \AF{E} \AB{Γ} \AB{is} as a value of type
(\AF{Val} \AB{is}) in the environment (\AF{Env} \AB{Γ}).  The values are either
arrays or positions of the corresponding shape.  Environments for the given context
\AB{Γ} are tuples of values of the corresponding shapes.  The \AF{lookup} function
translates variables within the context into variables within the environment.
\begin{mathpar}
\codeblock{\begin{code}%
%
\>[2]\AgdaFunction{Val}\AgdaSpace{}%
\AgdaSymbol{:}\AgdaSpace{}%
\AgdaDatatype{IS}\AgdaSpace{}%
\AgdaSymbol{→}\AgdaSpace{}%
\AgdaPrimitive{Set}\<%
\\
%
\>[2]\AgdaFunction{Val}\AgdaSpace{}%
\AgdaSymbol{(}\AgdaInductiveConstructor{ar}\AgdaSpace{}%
\AgdaBound{s}\AgdaSymbol{)}%
\>[14]\AgdaSymbol{=}\AgdaSpace{}%
\AgdaFunction{Ar}\AgdaSpace{}%
\AgdaBound{s}\AgdaSpace{}%
\AgdaField{R}\<%
\\
%
\>[2]\AgdaFunction{Val}\AgdaSpace{}%
\AgdaSymbol{(}\AgdaInductiveConstructor{ix}\AgdaSpace{}%
\AgdaBound{s}\AgdaSymbol{)}%
\>[14]\AgdaSymbol{=}\AgdaSpace{}%
\AgdaDatatype{P}\AgdaSpace{}%
\AgdaBound{s}\<%
\end{code}}
\and
\codeblock{\begin{code}%
%
\>[2]\AgdaFunction{Env}\AgdaSpace{}%
\AgdaSymbol{:}\AgdaSpace{}%
\AgdaDatatype{Ctx}\AgdaSpace{}%
\AgdaSymbol{→}\AgdaSpace{}%
\AgdaPrimitive{Set}\<%
\\
%
\>[2]\AgdaFunction{Env}\AgdaSpace{}%
\AgdaInductiveConstructor{ε}%
\>[16]\AgdaSymbol{=}\AgdaSpace{}%
\AgdaRecord{⊤}\<%
\\
%
\>[2]\AgdaFunction{Env}\AgdaSpace{}%
\AgdaSymbol{(}\AgdaBound{Γ}\AgdaSpace{}%
\AgdaOperator{\AgdaInductiveConstructor{▹}}\AgdaSpace{}%
\AgdaBound{is}\AgdaSymbol{)}%
\>[16]\AgdaSymbol{=}\AgdaSpace{}%
\AgdaFunction{Env}\AgdaSpace{}%
\AgdaBound{Γ}\AgdaSpace{}%
\AgdaOperator{\AgdaFunction{×}}\AgdaSpace{}%
\AgdaFunction{Val}\AgdaSpace{}%
\AgdaBound{is}\<%
\end{code}}
\and
\codeblock{\begin{code}%
%
\>[2]\AgdaFunction{lookup}\AgdaSpace{}%
\AgdaSymbol{:}\AgdaSpace{}%
\AgdaGeneralizable{is}\AgdaSpace{}%
\AgdaOperator{\AgdaDatatype{∈}}\AgdaSpace{}%
\AgdaGeneralizable{Γ}\AgdaSpace{}%
\AgdaSymbol{→}\AgdaSpace{}%
\AgdaFunction{Env}\AgdaSpace{}%
\AgdaGeneralizable{Γ}\AgdaSpace{}%
\AgdaSymbol{→}\AgdaSpace{}%
\AgdaFunction{Val}\AgdaSpace{}%
\AgdaGeneralizable{is}\<%
\\
%
\>[2]\AgdaFunction{lookup}\AgdaSpace{}%
\AgdaInductiveConstructor{v₀}%
\>[17]\AgdaSymbol{(}\AgdaBound{ρ}\AgdaSpace{}%
\AgdaOperator{\AgdaInductiveConstructor{,}}\AgdaSpace{}%
\AgdaBound{x}\AgdaSymbol{)}%
\>[26]\AgdaSymbol{=}\AgdaSpace{}%
\AgdaBound{x}\<%
\\
%
\>[2]\AgdaFunction{lookup}\AgdaSpace{}%
\AgdaSymbol{(}\AgdaInductiveConstructor{vₛ}\AgdaSpace{}%
\AgdaBound{v}\AgdaSymbol{)}%
\>[17]\AgdaSymbol{(}\AgdaBound{ρ}\AgdaSpace{}%
\AgdaOperator{\AgdaInductiveConstructor{,}}\AgdaSpace{}%
\AgdaSymbol{\AgdaUnderscore{})}%
\>[26]\AgdaSymbol{=}\AgdaSpace{}%
\AgdaFunction{lookup}\AgdaSpace{}%
\AgdaBound{v}\AgdaSpace{}%
\AgdaBound{ρ}\<%
\end{code}}
\end{mathpar}

Interpretation is given by \AF{⟦\_⟧} that is defined as follows.  Note that we pass
the environment as instance argument\footnote{For more details on instance arguments
refer to \url{https://agda.readthedocs.io/en/v2.7.0.1/language/instance-arguments.html}.}
which allows us to omit mentioning the environment
in recursive calls when it is passed unchanged.
%  ⟦_⟧ : E Γ is → Env Γ → Val is
%  ⟦ var x               ⟧ ρ  = lookup x ρ
%  ⟦ zero                ⟧ ρ  = K (fromℕ 0)
%  ⟦ one                 ⟧ ρ  = K (fromℕ 1)
%  ⟦ imaps e             ⟧ ρ  = λ i → ⟦ e ⟧ (ρ , i) [] 
%  ⟦ sels e e₁           ⟧ ρ  = K (⟦ e ⟧ ρ (⟦ e₁ ⟧ ρ))
%  ⟦ imap e              ⟧ ρ  = unnest λ i → ⟦ e ⟧ (ρ , i)
%  ⟦ sel e e₁            ⟧ ρ  = nest (⟦ e ⟧ ρ) (⟦ e₁ ⟧ ρ)
%  ⟦ imapb m e           ⟧ ρ  = Ar.imapb (λ i → ⟦ e ⟧ (ρ , i)) m
%  ⟦ selb m e e₁         ⟧ ρ  = Ar.selb (⟦ e ⟧ ρ) m (⟦ e₁ ⟧ ρ)
%  ⟦ sum e               ⟧ ρ  = Ar.sum (Ar.zipWith _+_) (K (fromℕ 0)) (λ i → ⟦ e ⟧ (ρ , i))
%  ⟦ zero-but i j e      ⟧ ρ  = if ⌊ ⟦ i ⟧ ρ ≟ₚ ⟦ j ⟧ ρ ⌋ then ⟦ e ⟧ ρ else K (fromℕ 0)
%  ⟦ slide e p e₁ s      ⟧ ρ  = Ar.slide (⟦ e ⟧ ρ) p (⟦ e₁ ⟧ ρ) s
%  ⟦ backslide e e₁ s p  ⟧ ρ  = Ar.backslide (⟦ e ⟧ ρ) (⟦ e₁ ⟧ ρ) s (fromℕ 0) p
%  ⟦ logistic e          ⟧ ρ  = Ar.map logisticʳ (⟦ e ⟧ ρ)
%  ⟦ e ⊞ e₁              ⟧ ρ  = Ar.zipWith _+_ (⟦ e ⟧ ρ) (⟦ e₁ ⟧ ρ)
%  ⟦ e ⊠ e₁              ⟧ ρ  = Ar.zipWith _*_ (⟦ e ⟧ ρ) (⟦ e₁ ⟧ ρ)
%  ⟦ scaledown n e       ⟧ ρ  = Ar.map (_÷ fromℕ n) (⟦ e ⟧ ρ) 
%  ⟦ minus e             ⟧ ρ  = Ar.map -_ (⟦ e ⟧ ρ)
%  ⟦ let′ e e₁           ⟧ ρ  = ⟦ e₁ ⟧ (ρ , ⟦ e ⟧ ρ)
\begin{code}%
%
\>[2]\AgdaOperator{\AgdaFunction{⟦\AgdaUnderscore{}⟧}}\AgdaSpace{}%
\AgdaSymbol{:}\AgdaSpace{}%
\AgdaDatatype{E}\AgdaSpace{}%
\AgdaGeneralizable{Γ}\AgdaSpace{}%
\AgdaGeneralizable{is}\AgdaSpace{}%
\AgdaSymbol{→}\AgdaSpace{}%
\AgdaSymbol{⦃}\AgdaSpace{}%
\AgdaFunction{Env}\AgdaSpace{}%
\AgdaGeneralizable{Γ}\AgdaSpace{}%
\AgdaSymbol{⦄}\AgdaSpace{}%
\AgdaSymbol{→}\AgdaSpace{}%
\AgdaFunction{Val}\AgdaSpace{}%
\AgdaGeneralizable{is}\<%
\\
%
\>[2]\AgdaOperator{\AgdaFunction{⟦}}\AgdaSpace{}%
\AgdaInductiveConstructor{var}\AgdaSpace{}%
\AgdaBound{x}%
\>[24]\AgdaOperator{\AgdaFunction{⟧}}\AgdaSpace{}%
\AgdaSymbol{⦃}\AgdaSpace{}%
\AgdaBound{ρ}\AgdaSpace{}%
\AgdaSymbol{⦄}%
\>[33]\AgdaSymbol{=}\AgdaSpace{}%
\AgdaFunction{lookup}\AgdaSpace{}%
\AgdaBound{x}\AgdaSpace{}%
\AgdaBound{ρ}\<%
\\
%
\>[2]\AgdaOperator{\AgdaFunction{⟦}}\AgdaSpace{}%
\AgdaInductiveConstructor{zero}%
\>[24]\AgdaOperator{\AgdaFunction{⟧}}\AgdaSpace{}%
\AgdaSymbol{⦃}\AgdaSpace{}%
\AgdaBound{ρ}\AgdaSpace{}%
\AgdaSymbol{⦄}%
\>[33]\AgdaSymbol{=}\AgdaSpace{}%
\AgdaFunction{Ar.K}\AgdaSpace{}%
\AgdaSymbol{(}\AgdaField{fromℕ}\AgdaSpace{}%
\AgdaNumber{0}\AgdaSymbol{)}\<%
\\
%
\>[2]\AgdaOperator{\AgdaFunction{⟦}}\AgdaSpace{}%
\AgdaInductiveConstructor{one}%
\>[24]\AgdaOperator{\AgdaFunction{⟧}}\AgdaSpace{}%
\AgdaSymbol{⦃}\AgdaSpace{}%
\AgdaBound{ρ}\AgdaSpace{}%
\AgdaSymbol{⦄}%
\>[33]\AgdaSymbol{=}\AgdaSpace{}%
\AgdaFunction{Ar.K}\AgdaSpace{}%
\AgdaSymbol{(}\AgdaField{fromℕ}\AgdaSpace{}%
\AgdaNumber{1}\AgdaSymbol{)}\<%
\\
%
\>[2]\AgdaOperator{\AgdaFunction{⟦}}\AgdaSpace{}%
\AgdaInductiveConstructor{imaps}\AgdaSpace{}%
\AgdaBound{e}%
\>[24]\AgdaOperator{\AgdaFunction{⟧}}\AgdaSpace{}%
\AgdaSymbol{⦃}\AgdaSpace{}%
\AgdaBound{ρ}\AgdaSpace{}%
\AgdaSymbol{⦄}%
\>[33]\AgdaSymbol{=}\AgdaSpace{}%
\AgdaSymbol{λ}\AgdaSpace{}%
\AgdaBound{i}\AgdaSpace{}%
\AgdaSymbol{→}\AgdaSpace{}%
\AgdaOperator{\AgdaFunction{⟦}}\AgdaSpace{}%
\AgdaBound{e}\AgdaSpace{}%
\AgdaOperator{\AgdaFunction{⟧}}\AgdaSpace{}%
\AgdaSymbol{⦃}\AgdaSpace{}%
\AgdaBound{ρ}\AgdaSpace{}%
\AgdaOperator{\AgdaInductiveConstructor{,}}\AgdaSpace{}%
\AgdaBound{i}\AgdaSpace{}%
\AgdaSymbol{⦄}\AgdaSpace{}%
\AgdaInductiveConstructor{[]}\<%
\\
%
\>[2]\AgdaOperator{\AgdaFunction{⟦}}\AgdaSpace{}%
\AgdaInductiveConstructor{sels}\AgdaSpace{}%
\AgdaBound{e}\AgdaSpace{}%
\AgdaBound{e₁}%
\>[24]\AgdaOperator{\AgdaFunction{⟧}}\AgdaSpace{}%
\AgdaSymbol{⦃}\AgdaSpace{}%
\AgdaBound{ρ}\AgdaSpace{}%
\AgdaSymbol{⦄}%
\>[33]\AgdaSymbol{=}\AgdaSpace{}%
\AgdaFunction{Ar.K}\AgdaSpace{}%
\AgdaSymbol{(}\AgdaOperator{\AgdaFunction{⟦}}\AgdaSpace{}%
\AgdaBound{e}\AgdaSpace{}%
\AgdaOperator{\AgdaFunction{⟧}}\AgdaSpace{}%
\AgdaOperator{\AgdaFunction{⟦}}\AgdaSpace{}%
\AgdaBound{e₁}\AgdaSpace{}%
\AgdaOperator{\AgdaFunction{⟧}}\AgdaSymbol{)}\<%
\\
%
\>[2]\AgdaOperator{\AgdaFunction{⟦}}\AgdaSpace{}%
\AgdaInductiveConstructor{imap}\AgdaSpace{}%
\AgdaBound{e}%
\>[24]\AgdaOperator{\AgdaFunction{⟧}}\AgdaSpace{}%
\AgdaSymbol{⦃}\AgdaSpace{}%
\AgdaBound{ρ}\AgdaSpace{}%
\AgdaSymbol{⦄}%
\>[33]\AgdaSymbol{=}\AgdaSpace{}%
\AgdaFunction{Ar.unnest}\AgdaSpace{}%
\AgdaSymbol{λ}\AgdaSpace{}%
\AgdaBound{i}\AgdaSpace{}%
\AgdaSymbol{→}\AgdaSpace{}%
\AgdaOperator{\AgdaFunction{⟦}}\AgdaSpace{}%
\AgdaBound{e}\AgdaSpace{}%
\AgdaOperator{\AgdaFunction{⟧}}\AgdaSpace{}%
\AgdaSymbol{⦃}\AgdaSpace{}%
\AgdaBound{ρ}\AgdaSpace{}%
\AgdaOperator{\AgdaInductiveConstructor{,}}\AgdaSpace{}%
\AgdaBound{i}\AgdaSpace{}%
\AgdaSymbol{⦄}\<%
\\
%
\>[2]\AgdaOperator{\AgdaFunction{⟦}}\AgdaSpace{}%
\AgdaInductiveConstructor{sel}\AgdaSpace{}%
\AgdaBound{e}\AgdaSpace{}%
\AgdaBound{e₁}%
\>[24]\AgdaOperator{\AgdaFunction{⟧}}\AgdaSpace{}%
\AgdaSymbol{⦃}\AgdaSpace{}%
\AgdaBound{ρ}\AgdaSpace{}%
\AgdaSymbol{⦄}%
\>[33]\AgdaSymbol{=}\AgdaSpace{}%
\AgdaFunction{Ar.nest}\AgdaSpace{}%
\AgdaOperator{\AgdaFunction{⟦}}\AgdaSpace{}%
\AgdaBound{e}\AgdaSpace{}%
\AgdaOperator{\AgdaFunction{⟧}}\AgdaSpace{}%
\AgdaOperator{\AgdaFunction{⟦}}\AgdaSpace{}%
\AgdaBound{e₁}\AgdaSpace{}%
\AgdaOperator{\AgdaFunction{⟧}}\<%
\\
%
\>[2]\AgdaOperator{\AgdaFunction{⟦}}\AgdaSpace{}%
\AgdaInductiveConstructor{imapb}\AgdaSpace{}%
\AgdaBound{m}\AgdaSpace{}%
\AgdaBound{e}%
\>[24]\AgdaOperator{\AgdaFunction{⟧}}\AgdaSpace{}%
\AgdaSymbol{⦃}\AgdaSpace{}%
\AgdaBound{ρ}\AgdaSpace{}%
\AgdaSymbol{⦄}%
\>[33]\AgdaSymbol{=}\AgdaSpace{}%
\AgdaFunction{Ar.imapb}\AgdaSpace{}%
\AgdaSymbol{(λ}\AgdaSpace{}%
\AgdaBound{i}\AgdaSpace{}%
\AgdaSymbol{→}\AgdaSpace{}%
\AgdaOperator{\AgdaFunction{⟦}}\AgdaSpace{}%
\AgdaBound{e}\AgdaSpace{}%
\AgdaOperator{\AgdaFunction{⟧}}\AgdaSpace{}%
\AgdaSymbol{⦃}\AgdaSpace{}%
\AgdaBound{ρ}\AgdaSpace{}%
\AgdaOperator{\AgdaInductiveConstructor{,}}\AgdaSpace{}%
\AgdaBound{i}\AgdaSpace{}%
\AgdaSymbol{⦄)}\AgdaSpace{}%
\AgdaBound{m}\<%
\\
%
\>[2]\AgdaOperator{\AgdaFunction{⟦}}\AgdaSpace{}%
\AgdaInductiveConstructor{selb}\AgdaSpace{}%
\AgdaBound{m}\AgdaSpace{}%
\AgdaBound{e}\AgdaSpace{}%
\AgdaBound{e₁}%
\>[24]\AgdaOperator{\AgdaFunction{⟧}}\AgdaSpace{}%
\AgdaSymbol{⦃}\AgdaSpace{}%
\AgdaBound{ρ}\AgdaSpace{}%
\AgdaSymbol{⦄}%
\>[33]\AgdaSymbol{=}\AgdaSpace{}%
\AgdaFunction{Ar.selb}\AgdaSpace{}%
\AgdaOperator{\AgdaFunction{⟦}}\AgdaSpace{}%
\AgdaBound{e}\AgdaSpace{}%
\AgdaOperator{\AgdaFunction{⟧}}\AgdaSpace{}%
\AgdaBound{m}\AgdaSpace{}%
\AgdaOperator{\AgdaFunction{⟦}}\AgdaSpace{}%
\AgdaBound{e₁}\AgdaSpace{}%
\AgdaOperator{\AgdaFunction{⟧}}\<%
\\
%
\>[2]\AgdaOperator{\AgdaFunction{⟦}}\AgdaSpace{}%
\AgdaInductiveConstructor{sum}\AgdaSpace{}%
\AgdaBound{e}%
\>[24]\AgdaOperator{\AgdaFunction{⟧}}\AgdaSpace{}%
\AgdaSymbol{⦃}\AgdaSpace{}%
\AgdaBound{ρ}\AgdaSpace{}%
\AgdaSymbol{⦄}%
\>[33]\AgdaSymbol{=}\AgdaSpace{}%
\AgdaFunction{Ar.sum}\AgdaSpace{}%
\AgdaSymbol{(}\AgdaFunction{Ar.zipWith}\AgdaSpace{}%
\AgdaOperator{\AgdaField{\AgdaUnderscore{}+\AgdaUnderscore{}}}\AgdaSymbol{)}\AgdaSpace{}%
\AgdaSymbol{(}\AgdaFunction{Ar.K}\AgdaSpace{}%
\AgdaSymbol{(}\AgdaField{fromℕ}\AgdaSpace{}%
\AgdaNumber{0}\AgdaSymbol{))}\AgdaSpace{}%
\AgdaSymbol{(λ}\AgdaSpace{}%
\AgdaBound{i}\AgdaSpace{}%
\AgdaSymbol{→}\AgdaSpace{}%
\AgdaOperator{\AgdaFunction{⟦}}\AgdaSpace{}%
\AgdaBound{e}\AgdaSpace{}%
\AgdaOperator{\AgdaFunction{⟧}}\AgdaSpace{}%
\AgdaSymbol{⦃}\AgdaSpace{}%
\AgdaBound{ρ}\AgdaSpace{}%
\AgdaOperator{\AgdaInductiveConstructor{,}}\AgdaSpace{}%
\AgdaBound{i}\AgdaSpace{}%
\AgdaSymbol{⦄)}\<%
\\
%
\>[2]\AgdaOperator{\AgdaFunction{⟦}}\AgdaSpace{}%
\AgdaInductiveConstructor{zero-but}\AgdaSpace{}%
\AgdaBound{i}\AgdaSpace{}%
\AgdaBound{j}\AgdaSpace{}%
\AgdaBound{e}%
\>[24]\AgdaOperator{\AgdaFunction{⟧}}\AgdaSpace{}%
\AgdaSymbol{⦃}\AgdaSpace{}%
\AgdaBound{ρ}\AgdaSpace{}%
\AgdaSymbol{⦄}%
\>[33]\AgdaSymbol{=}\AgdaSpace{}%
\AgdaOperator{\AgdaFunction{if}}\AgdaSpace{}%
\AgdaOperator{\AgdaFunction{⌊}}\AgdaSpace{}%
\AgdaOperator{\AgdaFunction{⟦}}\AgdaSpace{}%
\AgdaBound{i}\AgdaSpace{}%
\AgdaOperator{\AgdaFunction{⟧}}\AgdaSpace{}%
\AgdaOperator{\AgdaFunction{≟ₚ}}\AgdaSpace{}%
\AgdaOperator{\AgdaFunction{⟦}}\AgdaSpace{}%
\AgdaBound{j}\AgdaSpace{}%
\AgdaOperator{\AgdaFunction{⟧}}\AgdaSpace{}%
\AgdaOperator{\AgdaFunction{⌋}}\AgdaSpace{}%
\AgdaOperator{\AgdaFunction{then}}\AgdaSpace{}%
\AgdaOperator{\AgdaFunction{⟦}}\AgdaSpace{}%
\AgdaBound{e}\AgdaSpace{}%
\AgdaOperator{\AgdaFunction{⟧}}\AgdaSpace{}%
\AgdaOperator{\AgdaFunction{else}}\AgdaSpace{}%
\AgdaFunction{Ar.K}\AgdaSpace{}%
\AgdaSymbol{(}\AgdaField{fromℕ}\AgdaSpace{}%
\AgdaNumber{0}\AgdaSymbol{)}\<%
\\
%
\>[2]\AgdaOperator{\AgdaFunction{⟦}}\AgdaSpace{}%
\AgdaInductiveConstructor{slide}\AgdaSpace{}%
\AgdaBound{e}\AgdaSpace{}%
\AgdaBound{p}\AgdaSpace{}%
\AgdaBound{e₁}\AgdaSpace{}%
\AgdaBound{s}%
\>[24]\AgdaOperator{\AgdaFunction{⟧}}\AgdaSpace{}%
\AgdaSymbol{⦃}\AgdaSpace{}%
\AgdaBound{ρ}\AgdaSpace{}%
\AgdaSymbol{⦄}%
\>[33]\AgdaSymbol{=}\AgdaSpace{}%
\AgdaFunction{Ar.slide}\AgdaSpace{}%
\AgdaOperator{\AgdaFunction{⟦}}\AgdaSpace{}%
\AgdaBound{e}\AgdaSpace{}%
\AgdaOperator{\AgdaFunction{⟧}}\AgdaSpace{}%
\AgdaBound{p}\AgdaSpace{}%
\AgdaOperator{\AgdaFunction{⟦}}\AgdaSpace{}%
\AgdaBound{e₁}\AgdaSpace{}%
\AgdaOperator{\AgdaFunction{⟧}}\AgdaSpace{}%
\AgdaBound{s}\<%
\\
%
\>[2]\AgdaOperator{\AgdaFunction{⟦}}\AgdaSpace{}%
\AgdaInductiveConstructor{backslide}\AgdaSpace{}%
\AgdaBound{e}\AgdaSpace{}%
\AgdaBound{e₁}\AgdaSpace{}%
\AgdaBound{s}\AgdaSpace{}%
\AgdaBound{p}%
\>[24]\AgdaOperator{\AgdaFunction{⟧}}\AgdaSpace{}%
\AgdaSymbol{⦃}\AgdaSpace{}%
\AgdaBound{ρ}\AgdaSpace{}%
\AgdaSymbol{⦄}%
\>[33]\AgdaSymbol{=}\AgdaSpace{}%
\AgdaFunction{Ar.backslide}\AgdaSpace{}%
\AgdaOperator{\AgdaFunction{⟦}}\AgdaSpace{}%
\AgdaBound{e}\AgdaSpace{}%
\AgdaOperator{\AgdaFunction{⟧}}\AgdaSpace{}%
\AgdaOperator{\AgdaFunction{⟦}}\AgdaSpace{}%
\AgdaBound{e₁}\AgdaSpace{}%
\AgdaOperator{\AgdaFunction{⟧}}\AgdaSpace{}%
\AgdaBound{s}\AgdaSpace{}%
\AgdaSymbol{(}\AgdaField{fromℕ}\AgdaSpace{}%
\AgdaNumber{0}\AgdaSymbol{)}\AgdaSpace{}%
\AgdaBound{p}\<%
\\
%
\>[2]\AgdaOperator{\AgdaFunction{⟦}}\AgdaSpace{}%
\AgdaInductiveConstructor{logistic}\AgdaSpace{}%
\AgdaBound{e}%
\>[24]\AgdaOperator{\AgdaFunction{⟧}}\AgdaSpace{}%
\AgdaSymbol{⦃}\AgdaSpace{}%
\AgdaBound{ρ}\AgdaSpace{}%
\AgdaSymbol{⦄}%
\>[33]\AgdaSymbol{=}\AgdaSpace{}%
\AgdaFunction{Ar.map}\AgdaSpace{}%
\AgdaFunction{logisticʳ}\AgdaSpace{}%
\AgdaOperator{\AgdaFunction{⟦}}\AgdaSpace{}%
\AgdaBound{e}\AgdaSpace{}%
\AgdaOperator{\AgdaFunction{⟧}}\<%
\\
%
\>[2]\AgdaOperator{\AgdaFunction{⟦}}\AgdaSpace{}%
\AgdaBound{e}\AgdaSpace{}%
\AgdaOperator{\AgdaInductiveConstructor{⊞}}\AgdaSpace{}%
\AgdaBound{e₁}%
\>[24]\AgdaOperator{\AgdaFunction{⟧}}\AgdaSpace{}%
\AgdaSymbol{⦃}\AgdaSpace{}%
\AgdaBound{ρ}\AgdaSpace{}%
\AgdaSymbol{⦄}%
\>[33]\AgdaSymbol{=}\AgdaSpace{}%
\AgdaFunction{Ar.zipWith}\AgdaSpace{}%
\AgdaOperator{\AgdaField{\AgdaUnderscore{}+\AgdaUnderscore{}}}\AgdaSpace{}%
\AgdaOperator{\AgdaFunction{⟦}}\AgdaSpace{}%
\AgdaBound{e}\AgdaSpace{}%
\AgdaOperator{\AgdaFunction{⟧}}\AgdaSpace{}%
\AgdaOperator{\AgdaFunction{⟦}}\AgdaSpace{}%
\AgdaBound{e₁}\AgdaSpace{}%
\AgdaOperator{\AgdaFunction{⟧}}\<%
\\
%
\>[2]\AgdaOperator{\AgdaFunction{⟦}}\AgdaSpace{}%
\AgdaBound{e}\AgdaSpace{}%
\AgdaOperator{\AgdaInductiveConstructor{⊠}}\AgdaSpace{}%
\AgdaBound{e₁}%
\>[24]\AgdaOperator{\AgdaFunction{⟧}}\AgdaSpace{}%
\AgdaSymbol{⦃}\AgdaSpace{}%
\AgdaBound{ρ}\AgdaSpace{}%
\AgdaSymbol{⦄}%
\>[33]\AgdaSymbol{=}\AgdaSpace{}%
\AgdaFunction{Ar.zipWith}\AgdaSpace{}%
\AgdaOperator{\AgdaField{\AgdaUnderscore{}*\AgdaUnderscore{}}}\AgdaSpace{}%
\AgdaOperator{\AgdaFunction{⟦}}\AgdaSpace{}%
\AgdaBound{e}\AgdaSpace{}%
\AgdaOperator{\AgdaFunction{⟧}}\AgdaSpace{}%
\AgdaOperator{\AgdaFunction{⟦}}\AgdaSpace{}%
\AgdaBound{e₁}\AgdaSpace{}%
\AgdaOperator{\AgdaFunction{⟧}}\<%
\\
%
\>[2]\AgdaOperator{\AgdaFunction{⟦}}\AgdaSpace{}%
\AgdaInductiveConstructor{scaledown}\AgdaSpace{}%
\AgdaBound{n}\AgdaSpace{}%
\AgdaBound{e}%
\>[24]\AgdaOperator{\AgdaFunction{⟧}}\AgdaSpace{}%
\AgdaSymbol{⦃}\AgdaSpace{}%
\AgdaBound{ρ}\AgdaSpace{}%
\AgdaSymbol{⦄}%
\>[33]\AgdaSymbol{=}\AgdaSpace{}%
\AgdaFunction{Ar.map}\AgdaSpace{}%
\AgdaSymbol{(}\AgdaOperator{\AgdaField{\AgdaUnderscore{}÷}}\AgdaSpace{}%
\AgdaField{fromℕ}\AgdaSpace{}%
\AgdaBound{n}\AgdaSymbol{)}\AgdaSpace{}%
\AgdaOperator{\AgdaFunction{⟦}}\AgdaSpace{}%
\AgdaBound{e}\AgdaSpace{}%
\AgdaOperator{\AgdaFunction{⟧}}\<%
\\
%
\>[2]\AgdaOperator{\AgdaFunction{⟦}}\AgdaSpace{}%
\AgdaInductiveConstructor{minus}\AgdaSpace{}%
\AgdaBound{e}%
\>[24]\AgdaOperator{\AgdaFunction{⟧}}\AgdaSpace{}%
\AgdaSymbol{⦃}\AgdaSpace{}%
\AgdaBound{ρ}\AgdaSpace{}%
\AgdaSymbol{⦄}%
\>[33]\AgdaSymbol{=}\AgdaSpace{}%
\AgdaFunction{Ar.map}\AgdaSpace{}%
\AgdaOperator{\AgdaField{-\AgdaUnderscore{}}}\AgdaSpace{}%
\AgdaOperator{\AgdaFunction{⟦}}\AgdaSpace{}%
\AgdaBound{e}\AgdaSpace{}%
\AgdaOperator{\AgdaFunction{⟧}}\<%
\\
%
\>[2]\AgdaOperator{\AgdaFunction{⟦}}\AgdaSpace{}%
\AgdaInductiveConstructor{let′}\AgdaSpace{}%
\AgdaBound{e}\AgdaSpace{}%
\AgdaBound{e₁}%
\>[24]\AgdaOperator{\AgdaFunction{⟧}}\AgdaSpace{}%
\AgdaSymbol{⦃}\AgdaSpace{}%
\AgdaBound{ρ}\AgdaSpace{}%
\AgdaSymbol{⦄}%
\>[33]\AgdaSymbol{=}\AgdaSpace{}%
\AgdaOperator{\AgdaFunction{⟦}}\AgdaSpace{}%
\AgdaBound{e₁}\AgdaSpace{}%
\AgdaOperator{\AgdaFunction{⟧}}\AgdaSpace{}%
\AgdaSymbol{⦃}\AgdaSpace{}%
\AgdaBound{ρ}\AgdaSpace{}%
\AgdaOperator{\AgdaInductiveConstructor{,}}\AgdaSpace{}%
\AgdaOperator{\AgdaFunction{⟦}}\AgdaSpace{}%
\AgdaBound{e}\AgdaSpace{}%
\AgdaOperator{\AgdaFunction{⟧}}\AgdaSpace{}%
\AgdaSymbol{⦄}\<%
\end{code}
Mostly, the interpretation is a straightforward mapping into the \AF{Ar} constructors.
In the \AC{imaps} case we can see how the implicit conversion from what would be a
shape $s ⊗ \AF{unit}$ into $s$.  In case of \AC{sels} we make a singleton array
using \AF{K}. Note that \AF{sum} has explicit summation index like in a mathematical
$\sum$-notation.  We fix the default value of \AF{backslide} to zero for simplicity.
For arrays of reals, we can get general \AF{backslide} behaviour through masking.
However, this operation can be generalised in case we decide to support arrays of
other element types.


% With the above definition we can better explain the choices of language constructors.
% The most important question to clarify is why do we have three array
% constructors/eliminators.  As the only conceptual datatype of our language is
% an array (of some shape), we do not have any direct way to talk about array elements.
% Therefore, we model the type of array elements (scalars) as arrays of a singleton shape.
% As can be seen, scalar selection \AC{selₛ} returns a singleton array
% (application of \AF{K}) where all the element(s) are equal to the element we are
% selecting.  The corresponding array constructor \AC{imapₛ} makes sure that if we
% compute \AB{s} elements of the shape \AF{unit}, we produce an array of shape \AB{s}
% (and not \AB{s} \AC{⊗} \AF{unit}).  This soves the problem of constructing arays
% from scalars, but how do we construct an array of a product shape?  Given that we
% have an expression in the context (\AB{Γ} \AC{▹} \AC{ix} \AB{s} \AC{▹} \AC{ix} \AB{p}),
% we need to produce an array of \AB{s} \AC{⊗} \AB{p}.  There are several ways how to
% solve this (\eg{} introducing nest/unnest or projections and pairing on indices),
% but it is clear that we need something more than just an \AC{imapₛ}.
% This is the reason to introduce \AC{imap}/\AC{sel} pair which operates on arrays
% of product shapes.  
% As average pooling operates on blocked arrays, we need
% a construction to express this in \AF{E}.  One could introduce explicit 
% \AF{block}/\AF{unblock}, but we merge blocking/unlocking action with
% imap/sel obtaining \AC{imapb}/\AC{selb}.  Our \AC{sum} constructor 
% gets an argument in the extended context which is summation index, so 
% conceptually we generate the values at every summation index before
% summing these values together.  As a result, we only need
% one instance of \AC{sum} which makes our expressions a little tidier.




\subsection{Weakening and Substitution}
We are working with explicit de Bruijn variables (as opposed to HOAS~\cite{hoas}
approaches), and we need to define the notion of weakening and substitution
for constructing expressions and optimising them.
Intrinsic types(shapes) make both definitions challenging.  However, the problem has
been well-understood, and several approaches have been proposed in the
literature~\cite{subst}.

The key structure that we use for weakening is an order-preserving embedding
of contexts given by \AD{\_⊆\_} which is defined inductively.
If \AB{Γ} \AD{⊆} \AB{Δ} then all the
elements of \AB{Γ} can be found in \AB{Δ} in the original order (possibly with some gaps).
Weakening variables according to some context embedding is given by \AF{wkv} which is
defined as follows.
\begin{code}[hide]%
\>[0]\AgdaKeyword{module}\AgdaSpace{}%
\AgdaModule{WkSub}\AgdaSpace{}%
\AgdaKeyword{where}\<%
\\
\>[0][@{}l@{\AgdaIndent{0}}]%
\>[2]\AgdaKeyword{open}\AgdaSpace{}%
\AgdaModule{Lang}\<%
\end{code}
\begin{mathpar}
\codeblock{\begin{code}%
%
\>[2]\AgdaKeyword{data}\AgdaSpace{}%
\AgdaOperator{\AgdaDatatype{\AgdaUnderscore{}⊆\AgdaUnderscore{}}}\AgdaSpace{}%
\AgdaSymbol{:}\AgdaSpace{}%
\AgdaDatatype{Ctx}\AgdaSpace{}%
\AgdaSymbol{→}\AgdaSpace{}%
\AgdaDatatype{Ctx}\AgdaSpace{}%
\AgdaSymbol{→}\AgdaSpace{}%
\AgdaPrimitive{Set}\AgdaSpace{}%
\AgdaKeyword{where}\<%
\\
\>[2][@{}l@{\AgdaIndent{0}}]%
\>[4]\AgdaInductiveConstructor{ε}%
\>[10]\AgdaSymbol{:}\AgdaSpace{}%
\AgdaInductiveConstructor{ε}\AgdaSpace{}%
\AgdaOperator{\AgdaDatatype{⊆}}\AgdaSpace{}%
\AgdaInductiveConstructor{ε}\<%
\\
%
\>[4]\AgdaInductiveConstructor{skip}%
\>[10]\AgdaSymbol{:}\AgdaSpace{}%
\AgdaGeneralizable{Γ}\AgdaSpace{}%
\AgdaOperator{\AgdaDatatype{⊆}}\AgdaSpace{}%
\AgdaGeneralizable{Δ}\AgdaSpace{}%
\AgdaSymbol{→}\AgdaSpace{}%
\AgdaGeneralizable{Γ}\AgdaSpace{}%
\AgdaOperator{\AgdaDatatype{⊆}}\AgdaSpace{}%
\AgdaSymbol{(}\AgdaGeneralizable{Δ}\AgdaSpace{}%
\AgdaOperator{\AgdaInductiveConstructor{▹}}\AgdaSpace{}%
\AgdaGeneralizable{is}\AgdaSymbol{)}\<%
\\
%
\>[4]\AgdaInductiveConstructor{keep}%
\>[10]\AgdaSymbol{:}\AgdaSpace{}%
\AgdaGeneralizable{Γ}\AgdaSpace{}%
\AgdaOperator{\AgdaDatatype{⊆}}\AgdaSpace{}%
\AgdaGeneralizable{Δ}\AgdaSpace{}%
\AgdaSymbol{→}\AgdaSpace{}%
\AgdaSymbol{(}\AgdaGeneralizable{Γ}\AgdaSpace{}%
\AgdaOperator{\AgdaInductiveConstructor{▹}}\AgdaSpace{}%
\AgdaGeneralizable{is}\AgdaSymbol{)}\AgdaSpace{}%
\AgdaOperator{\AgdaDatatype{⊆}}\AgdaSpace{}%
\AgdaSymbol{(}\AgdaGeneralizable{Δ}\AgdaSpace{}%
\AgdaOperator{\AgdaInductiveConstructor{▹}}\AgdaSpace{}%
\AgdaGeneralizable{is}\AgdaSymbol{)}\<%
\end{code}}
\and
\codeblock{\begin{code}  %
%
\>[2]\AgdaFunction{wkv}\AgdaSpace{}%
\AgdaSymbol{:}\AgdaSpace{}%
\AgdaGeneralizable{Γ}\AgdaSpace{}%
\AgdaOperator{\AgdaDatatype{⊆}}\AgdaSpace{}%
\AgdaGeneralizable{Δ}\AgdaSpace{}%
\AgdaSymbol{→}\AgdaSpace{}%
\AgdaGeneralizable{is}\AgdaSpace{}%
\AgdaOperator{\AgdaDatatype{∈}}\AgdaSpace{}%
\AgdaGeneralizable{Γ}\AgdaSpace{}%
\AgdaSymbol{→}\AgdaSpace{}%
\AgdaGeneralizable{is}\AgdaSpace{}%
\AgdaOperator{\AgdaDatatype{∈}}\AgdaSpace{}%
\AgdaGeneralizable{Δ}\<%
\\
%
\>[2]\AgdaFunction{wkv}\AgdaSpace{}%
\AgdaSymbol{(}\AgdaInductiveConstructor{skip}\AgdaSpace{}%
\AgdaBound{s}\AgdaSymbol{)}\AgdaSpace{}%
\AgdaBound{v}%
\>[23]\AgdaSymbol{=}\AgdaSpace{}%
\AgdaInductiveConstructor{vₛ}\AgdaSpace{}%
\AgdaSymbol{(}\AgdaFunction{wkv}\AgdaSpace{}%
\AgdaBound{s}\AgdaSpace{}%
\AgdaBound{v}\AgdaSymbol{)}\<%
\\
%
\>[2]\AgdaFunction{wkv}\AgdaSpace{}%
\AgdaSymbol{(}\AgdaInductiveConstructor{keep}\AgdaSpace{}%
\AgdaBound{s}\AgdaSymbol{)}\AgdaSpace{}%
\AgdaInductiveConstructor{v₀}%
\>[23]\AgdaSymbol{=}\AgdaSpace{}%
\AgdaInductiveConstructor{v₀}\<%
\\
%
\>[2]\AgdaFunction{wkv}\AgdaSpace{}%
\AgdaSymbol{(}\AgdaInductiveConstructor{keep}\AgdaSpace{}%
\AgdaBound{s}\AgdaSymbol{)}\AgdaSpace{}%
\AgdaSymbol{(}\AgdaInductiveConstructor{vₛ}\AgdaSpace{}%
\AgdaBound{v}\AgdaSymbol{)}%
\>[23]\AgdaSymbol{=}\AgdaSpace{}%
\AgdaInductiveConstructor{vₛ}\AgdaSpace{}%
\AgdaSymbol{(}\AgdaFunction{wkv}\AgdaSpace{}%
\AgdaBound{s}\AgdaSpace{}%
\AgdaBound{v}\AgdaSymbol{)}\<%
\end{code}}
\end{mathpar}

Weakening expressions in \AF{E} according to some context embedding is given by \AF{wk}
which type is defined below.  Reflexivity of context embeddings is given by \AF{⊆-eq}.
A common case of weakening expressions into the context with one extra variable
is denoted with \AF{\_↑} and it is defined as follows.
\begin{mathpar}
\codeblock{\begin{code}%
%
\>[2]\AgdaFunction{wk}\AgdaSpace{}%
\AgdaSymbol{:}\AgdaSpace{}%
\AgdaGeneralizable{Γ}\AgdaSpace{}%
\AgdaOperator{\AgdaDatatype{⊆}}\AgdaSpace{}%
\AgdaGeneralizable{Δ}\AgdaSpace{}%
\AgdaSymbol{→}\AgdaSpace{}%
\AgdaDatatype{E}\AgdaSpace{}%
\AgdaGeneralizable{Γ}\AgdaSpace{}%
\AgdaGeneralizable{is}\AgdaSpace{}%
\AgdaSymbol{→}\AgdaSpace{}%
\AgdaDatatype{E}\AgdaSpace{}%
\AgdaGeneralizable{Δ}\AgdaSpace{}%
\AgdaGeneralizable{is}\<%
\end{code}}
\and
\codeblock{\begin{code}%
%
\>[2]\AgdaFunction{⊆-eq}\AgdaSpace{}%
\AgdaSymbol{:}\AgdaSpace{}%
\AgdaGeneralizable{Γ}\AgdaSpace{}%
\AgdaOperator{\AgdaDatatype{⊆}}\AgdaSpace{}%
\AgdaGeneralizable{Γ}\<%
\\
%
\>[2]\AgdaFunction{⊆-eq}\AgdaSpace{}%
\AgdaSymbol{\{}\AgdaInductiveConstructor{ε}\AgdaSymbol{\}}%
\>[16]\AgdaSymbol{=}\AgdaSpace{}%
\AgdaInductiveConstructor{ε}\<%
\\
%
\>[2]\AgdaFunction{⊆-eq}\AgdaSpace{}%
\AgdaSymbol{\{}\AgdaBound{Γ}\AgdaSpace{}%
\AgdaOperator{\AgdaInductiveConstructor{▹}}\AgdaSpace{}%
\AgdaBound{x}\AgdaSymbol{\}}%
\>[16]\AgdaSymbol{=}\AgdaSpace{}%
\AgdaInductiveConstructor{keep}\AgdaSpace{}%
\AgdaFunction{⊆-eq}\<%
\end{code}}
\and
\codeblock{\begin{code}%
%
\>[2]\AgdaOperator{\AgdaFunction{\AgdaUnderscore{}↑}}\AgdaSpace{}%
\AgdaSymbol{:}\AgdaSpace{}%
\AgdaDatatype{E}\AgdaSpace{}%
\AgdaGeneralizable{Γ}\AgdaSpace{}%
\AgdaGeneralizable{is}\AgdaSpace{}%
\AgdaSymbol{→}\AgdaSpace{}%
\AgdaDatatype{E}\AgdaSpace{}%
\AgdaSymbol{(}\AgdaGeneralizable{Γ}\AgdaSpace{}%
\AgdaOperator{\AgdaInductiveConstructor{▹}}\AgdaSpace{}%
\AgdaGeneralizable{ip}\AgdaSymbol{)}\AgdaSpace{}%
\AgdaGeneralizable{is}\<%
\\
%
\>[2]\AgdaOperator{\AgdaFunction{\AgdaUnderscore{}↑}}\AgdaSpace{}%
\AgdaSymbol{=}\AgdaSpace{}%
\AgdaFunction{wk}\AgdaSpace{}%
\AgdaSymbol{(}\AgdaInductiveConstructor{skip}\AgdaSpace{}%
\AgdaFunction{⊆-eq}\AgdaSymbol{)}\<%
\end{code}}
\end{mathpar}

\begin{code}[hide]%
%
\>[2]\AgdaFunction{wk}\AgdaSpace{}%
\AgdaBound{s}\AgdaSpace{}%
\AgdaSymbol{(}\AgdaInductiveConstructor{var}\AgdaSpace{}%
\AgdaBound{x}\AgdaSymbol{)}\AgdaSpace{}%
\AgdaSymbol{=}\AgdaSpace{}%
\AgdaInductiveConstructor{var}\AgdaSpace{}%
\AgdaSymbol{(}\AgdaFunction{wkv}\AgdaSpace{}%
\AgdaBound{s}\AgdaSpace{}%
\AgdaBound{x}\AgdaSymbol{)}\<%
\\
%
\>[2]\AgdaFunction{wk}\AgdaSpace{}%
\AgdaBound{s}\AgdaSpace{}%
\AgdaInductiveConstructor{zero}\AgdaSpace{}%
\AgdaSymbol{=}\AgdaSpace{}%
\AgdaInductiveConstructor{zero}\<%
\\
%
\>[2]\AgdaFunction{wk}\AgdaSpace{}%
\AgdaBound{s}\AgdaSpace{}%
\AgdaInductiveConstructor{one}\AgdaSpace{}%
\AgdaSymbol{=}\AgdaSpace{}%
\AgdaInductiveConstructor{one}\<%
\\
%
\>[2]\AgdaFunction{wk}\AgdaSpace{}%
\AgdaBound{s}\AgdaSpace{}%
\AgdaSymbol{(}\AgdaInductiveConstructor{imaps}\AgdaSpace{}%
\AgdaBound{e}\AgdaSymbol{)}\AgdaSpace{}%
\AgdaSymbol{=}\AgdaSpace{}%
\AgdaInductiveConstructor{imaps}\AgdaSpace{}%
\AgdaSymbol{(}\AgdaFunction{wk}\AgdaSpace{}%
\AgdaSymbol{(}\AgdaInductiveConstructor{keep}\AgdaSpace{}%
\AgdaBound{s}\AgdaSymbol{)}\AgdaSpace{}%
\AgdaBound{e}\AgdaSymbol{)}\<%
\\
%
\>[2]\AgdaFunction{wk}\AgdaSpace{}%
\AgdaBound{s}\AgdaSpace{}%
\AgdaSymbol{(}\AgdaInductiveConstructor{sels}\AgdaSpace{}%
\AgdaBound{e}\AgdaSpace{}%
\AgdaBound{e₁}\AgdaSymbol{)}\AgdaSpace{}%
\AgdaSymbol{=}\AgdaSpace{}%
\AgdaInductiveConstructor{sels}\AgdaSpace{}%
\AgdaSymbol{(}\AgdaFunction{wk}\AgdaSpace{}%
\AgdaBound{s}\AgdaSpace{}%
\AgdaBound{e}\AgdaSymbol{)}\AgdaSpace{}%
\AgdaSymbol{(}\AgdaFunction{wk}\AgdaSpace{}%
\AgdaBound{s}\AgdaSpace{}%
\AgdaBound{e₁}\AgdaSymbol{)}\<%
\\
%
\>[2]\AgdaFunction{wk}\AgdaSpace{}%
\AgdaBound{s}\AgdaSpace{}%
\AgdaSymbol{(}\AgdaInductiveConstructor{imap}\AgdaSpace{}%
\AgdaBound{e}\AgdaSymbol{)}\AgdaSpace{}%
\AgdaSymbol{=}\AgdaSpace{}%
\AgdaInductiveConstructor{imap}\AgdaSpace{}%
\AgdaSymbol{(}\AgdaFunction{wk}\AgdaSpace{}%
\AgdaSymbol{(}\AgdaInductiveConstructor{keep}\AgdaSpace{}%
\AgdaBound{s}\AgdaSymbol{)}\AgdaSpace{}%
\AgdaBound{e}\AgdaSymbol{)}\<%
\\
%
\>[2]\AgdaFunction{wk}\AgdaSpace{}%
\AgdaBound{s}\AgdaSpace{}%
\AgdaSymbol{(}\AgdaInductiveConstructor{sel}\AgdaSpace{}%
\AgdaBound{e}\AgdaSpace{}%
\AgdaBound{e₁}\AgdaSymbol{)}\AgdaSpace{}%
\AgdaSymbol{=}\AgdaSpace{}%
\AgdaInductiveConstructor{sel}\AgdaSpace{}%
\AgdaSymbol{(}\AgdaFunction{wk}\AgdaSpace{}%
\AgdaBound{s}\AgdaSpace{}%
\AgdaBound{e}\AgdaSymbol{)}\AgdaSpace{}%
\AgdaSymbol{(}\AgdaFunction{wk}\AgdaSpace{}%
\AgdaBound{s}\AgdaSpace{}%
\AgdaBound{e₁}\AgdaSymbol{)}\<%
\\
%
\>[2]\AgdaFunction{wk}\AgdaSpace{}%
\AgdaBound{s}\AgdaSpace{}%
\AgdaSymbol{(}\AgdaInductiveConstructor{imapb}\AgdaSpace{}%
\AgdaBound{x}\AgdaSpace{}%
\AgdaBound{e}\AgdaSymbol{)}\AgdaSpace{}%
\AgdaSymbol{=}\AgdaSpace{}%
\AgdaInductiveConstructor{imapb}\AgdaSpace{}%
\AgdaBound{x}\AgdaSpace{}%
\AgdaSymbol{(}\AgdaFunction{wk}\AgdaSpace{}%
\AgdaSymbol{(}\AgdaInductiveConstructor{keep}\AgdaSpace{}%
\AgdaBound{s}\AgdaSymbol{)}\AgdaSpace{}%
\AgdaBound{e}\AgdaSymbol{)}\<%
\\
%
\>[2]\AgdaFunction{wk}\AgdaSpace{}%
\AgdaBound{s}\AgdaSpace{}%
\AgdaSymbol{(}\AgdaInductiveConstructor{selb}\AgdaSpace{}%
\AgdaBound{x}\AgdaSpace{}%
\AgdaBound{e}\AgdaSpace{}%
\AgdaBound{e₁}\AgdaSymbol{)}\AgdaSpace{}%
\AgdaSymbol{=}\AgdaSpace{}%
\AgdaInductiveConstructor{selb}\AgdaSpace{}%
\AgdaBound{x}\AgdaSpace{}%
\AgdaSymbol{(}\AgdaFunction{wk}\AgdaSpace{}%
\AgdaBound{s}\AgdaSpace{}%
\AgdaBound{e}\AgdaSymbol{)}\AgdaSpace{}%
\AgdaSymbol{(}\AgdaFunction{wk}\AgdaSpace{}%
\AgdaBound{s}\AgdaSpace{}%
\AgdaBound{e₁}\AgdaSymbol{)}\<%
\\
%
\>[2]\AgdaFunction{wk}\AgdaSpace{}%
\AgdaBound{s}\AgdaSpace{}%
\AgdaSymbol{(}\AgdaInductiveConstructor{sum}\AgdaSpace{}%
\AgdaBound{e}\AgdaSymbol{)}\AgdaSpace{}%
\AgdaSymbol{=}\AgdaSpace{}%
\AgdaInductiveConstructor{sum}\AgdaSpace{}%
\AgdaSymbol{(}\AgdaFunction{wk}\AgdaSpace{}%
\AgdaSymbol{(}\AgdaInductiveConstructor{keep}\AgdaSpace{}%
\AgdaBound{s}\AgdaSymbol{)}\AgdaSpace{}%
\AgdaBound{e}\AgdaSymbol{)}\<%
\\
%
\>[2]\AgdaFunction{wk}\AgdaSpace{}%
\AgdaBound{s}\AgdaSpace{}%
\AgdaSymbol{(}\AgdaInductiveConstructor{zero-but}\AgdaSpace{}%
\AgdaBound{e}\AgdaSpace{}%
\AgdaBound{e₁}\AgdaSpace{}%
\AgdaBound{e₂}\AgdaSymbol{)}\AgdaSpace{}%
\AgdaSymbol{=}\AgdaSpace{}%
\AgdaInductiveConstructor{zero-but}\AgdaSpace{}%
\AgdaSymbol{(}\AgdaFunction{wk}\AgdaSpace{}%
\AgdaBound{s}\AgdaSpace{}%
\AgdaBound{e}\AgdaSymbol{)}\AgdaSpace{}%
\AgdaSymbol{(}\AgdaFunction{wk}\AgdaSpace{}%
\AgdaBound{s}\AgdaSpace{}%
\AgdaBound{e₁}\AgdaSymbol{)}\AgdaSpace{}%
\AgdaSymbol{(}\AgdaFunction{wk}\AgdaSpace{}%
\AgdaBound{s}\AgdaSpace{}%
\AgdaBound{e₂}\AgdaSymbol{)}\<%
\\
%
\>[2]\AgdaFunction{wk}\AgdaSpace{}%
\AgdaBound{s}\AgdaSpace{}%
\AgdaSymbol{(}\AgdaInductiveConstructor{slide}\AgdaSpace{}%
\AgdaBound{e}\AgdaSpace{}%
\AgdaBound{x}\AgdaSpace{}%
\AgdaBound{e₁}\AgdaSpace{}%
\AgdaBound{x₁}\AgdaSymbol{)}\AgdaSpace{}%
\AgdaSymbol{=}\AgdaSpace{}%
\AgdaInductiveConstructor{slide}\AgdaSpace{}%
\AgdaSymbol{(}\AgdaFunction{wk}\AgdaSpace{}%
\AgdaBound{s}\AgdaSpace{}%
\AgdaBound{e}\AgdaSymbol{)}\AgdaSpace{}%
\AgdaBound{x}\AgdaSpace{}%
\AgdaSymbol{(}\AgdaFunction{wk}\AgdaSpace{}%
\AgdaBound{s}\AgdaSpace{}%
\AgdaBound{e₁}\AgdaSymbol{)}\AgdaSpace{}%
\AgdaBound{x₁}\<%
\\
%
\>[2]\AgdaFunction{wk}\AgdaSpace{}%
\AgdaBound{s}\AgdaSpace{}%
\AgdaSymbol{(}\AgdaInductiveConstructor{backslide}\AgdaSpace{}%
\AgdaBound{e}\AgdaSpace{}%
\AgdaBound{e₁}\AgdaSpace{}%
\AgdaBound{x}\AgdaSpace{}%
\AgdaBound{x₁}\AgdaSymbol{)}\AgdaSpace{}%
\AgdaSymbol{=}\AgdaSpace{}%
\AgdaInductiveConstructor{backslide}\AgdaSpace{}%
\AgdaSymbol{(}\AgdaFunction{wk}\AgdaSpace{}%
\AgdaBound{s}\AgdaSpace{}%
\AgdaBound{e}\AgdaSymbol{)}\AgdaSpace{}%
\AgdaSymbol{(}\AgdaFunction{wk}\AgdaSpace{}%
\AgdaBound{s}\AgdaSpace{}%
\AgdaBound{e₁}\AgdaSymbol{)}\AgdaSpace{}%
\AgdaBound{x}\AgdaSpace{}%
\AgdaBound{x₁}\<%
\\
%
\>[2]\AgdaFunction{wk}\AgdaSpace{}%
\AgdaBound{s}\AgdaSpace{}%
\AgdaSymbol{(}\AgdaInductiveConstructor{logistic}\AgdaSpace{}%
\AgdaBound{e}\AgdaSymbol{)}\AgdaSpace{}%
\AgdaSymbol{=}\AgdaSpace{}%
\AgdaInductiveConstructor{logistic}\AgdaSpace{}%
\AgdaSymbol{(}\AgdaFunction{wk}\AgdaSpace{}%
\AgdaBound{s}\AgdaSpace{}%
\AgdaBound{e}\AgdaSymbol{)}\<%
\\
%
\>[2]\AgdaFunction{wk}\AgdaSpace{}%
\AgdaBound{s}\AgdaSpace{}%
\AgdaSymbol{(}\AgdaInductiveConstructor{bin}\AgdaSpace{}%
\AgdaBound{x}\AgdaSpace{}%
\AgdaBound{e}\AgdaSpace{}%
\AgdaBound{e₁}\AgdaSymbol{)}\AgdaSpace{}%
\AgdaSymbol{=}\AgdaSpace{}%
\AgdaInductiveConstructor{bin}\AgdaSpace{}%
\AgdaBound{x}\AgdaSpace{}%
\AgdaSymbol{(}\AgdaFunction{wk}\AgdaSpace{}%
\AgdaBound{s}\AgdaSpace{}%
\AgdaBound{e}\AgdaSymbol{)}\AgdaSpace{}%
\AgdaSymbol{(}\AgdaFunction{wk}\AgdaSpace{}%
\AgdaBound{s}\AgdaSpace{}%
\AgdaBound{e₁}\AgdaSymbol{)}\<%
\\
%
\>[2]\AgdaFunction{wk}\AgdaSpace{}%
\AgdaBound{s}\AgdaSpace{}%
\AgdaSymbol{(}\AgdaInductiveConstructor{scaledown}\AgdaSpace{}%
\AgdaBound{x}\AgdaSpace{}%
\AgdaBound{e}\AgdaSymbol{)}\AgdaSpace{}%
\AgdaSymbol{=}\AgdaSpace{}%
\AgdaInductiveConstructor{scaledown}\AgdaSpace{}%
\AgdaBound{x}\AgdaSpace{}%
\AgdaSymbol{(}\AgdaFunction{wk}\AgdaSpace{}%
\AgdaBound{s}\AgdaSpace{}%
\AgdaBound{e}\AgdaSymbol{)}\<%
\\
%
\>[2]\AgdaFunction{wk}\AgdaSpace{}%
\AgdaBound{s}\AgdaSpace{}%
\AgdaSymbol{(}\AgdaInductiveConstructor{minus}\AgdaSpace{}%
\AgdaBound{e}\AgdaSymbol{)}\AgdaSpace{}%
\AgdaSymbol{=}\AgdaSpace{}%
\AgdaInductiveConstructor{minus}\AgdaSpace{}%
\AgdaSymbol{(}\AgdaFunction{wk}\AgdaSpace{}%
\AgdaBound{s}\AgdaSpace{}%
\AgdaBound{e}\AgdaSymbol{)}\<%
\\
%
\>[2]\AgdaFunction{wk}\AgdaSpace{}%
\AgdaBound{s}\AgdaSpace{}%
\AgdaSymbol{(}\AgdaInductiveConstructor{let′}\AgdaSpace{}%
\AgdaBound{e}\AgdaSpace{}%
\AgdaBound{e₁}\AgdaSymbol{)}\AgdaSpace{}%
\AgdaSymbol{=}\AgdaSpace{}%
\AgdaInductiveConstructor{let′}\AgdaSpace{}%
\AgdaSymbol{(}\AgdaFunction{wk}\AgdaSpace{}%
\AgdaBound{s}\AgdaSpace{}%
\AgdaBound{e}\AgdaSymbol{)}\AgdaSpace{}%
\AgdaSymbol{(}\AgdaFunction{wk}\AgdaSpace{}%
\AgdaSymbol{(}\AgdaInductiveConstructor{keep}\AgdaSpace{}%
\AgdaBound{s}\AgdaSymbol{)}\AgdaSpace{}%
\AgdaBound{e₁}\AgdaSymbol{)}\<%
\end{code} 

We implement parallel substitution~\cite{par-sub1,par-sub2} in the usual way.
The key structure that
gives rise to the substitution is a mapping of variables in the context \AB{Δ}
into expressions in the context \AB{Γ}.
This is given by \AC{Sub} \AB{Γ} {Δ} and it represents a \AF{Δ}-long
list of (\AF{E} \AB{Γ})-s where each expression is of a type that corresponds to the
variable type at the given position of \AF{Δ}.  We define \AF{wks} which weakens
all the expressions of \AF{Sub} in the following way.
\begin{mathpar}
\codeblock{\begin{code}%
%
\>[2]\AgdaKeyword{data}\AgdaSpace{}%
\AgdaDatatype{Sub}\AgdaSpace{}%
\AgdaSymbol{(}\AgdaBound{Γ}\AgdaSpace{}%
\AgdaSymbol{:}\AgdaSpace{}%
\AgdaDatatype{Ctx}\AgdaSymbol{)}\AgdaSpace{}%
\AgdaSymbol{:}\AgdaSpace{}%
\AgdaDatatype{Ctx}\AgdaSpace{}%
\AgdaSymbol{→}\AgdaSpace{}%
\AgdaPrimitive{Set}\AgdaSpace{}%
\AgdaKeyword{where}\<%
\\
\>[2][@{}l@{\AgdaIndent{0}}]%
\>[4]\AgdaInductiveConstructor{ε}%
\>[9]\AgdaSymbol{:}\AgdaSpace{}%
\AgdaDatatype{Sub}\AgdaSpace{}%
\AgdaBound{Γ}\AgdaSpace{}%
\AgdaInductiveConstructor{ε}\<%
\\
%
\>[4]\AgdaOperator{\AgdaInductiveConstructor{\AgdaUnderscore{}▹\AgdaUnderscore{}}}%
\>[9]\AgdaSymbol{:}\AgdaSpace{}%
\AgdaDatatype{Sub}\AgdaSpace{}%
\AgdaBound{Γ}\AgdaSpace{}%
\AgdaGeneralizable{Δ}\AgdaSpace{}%
\AgdaSymbol{→}\AgdaSpace{}%
\AgdaDatatype{E}\AgdaSpace{}%
\AgdaBound{Γ}\AgdaSpace{}%
\AgdaGeneralizable{is}\AgdaSpace{}%
\AgdaSymbol{→}\AgdaSpace{}%
\AgdaDatatype{Sub}\AgdaSpace{}%
\AgdaBound{Γ}\AgdaSpace{}%
\AgdaSymbol{(}\AgdaGeneralizable{Δ}\AgdaSpace{}%
\AgdaOperator{\AgdaInductiveConstructor{▹}}\AgdaSpace{}%
\AgdaGeneralizable{is}\AgdaSymbol{)}\<%
\end{code}}
\and
\codeblock{\begin{code}%
%
\>[2]\AgdaFunction{wks}\AgdaSpace{}%
\AgdaSymbol{:}\AgdaSpace{}%
\AgdaDatatype{Sub}\AgdaSpace{}%
\AgdaGeneralizable{Γ}\AgdaSpace{}%
\AgdaGeneralizable{Δ}\AgdaSpace{}%
\AgdaSymbol{→}\AgdaSpace{}%
\AgdaGeneralizable{Γ}\AgdaSpace{}%
\AgdaOperator{\AgdaDatatype{⊆}}\AgdaSpace{}%
\AgdaGeneralizable{Ψ}\AgdaSpace{}%
\AgdaSymbol{→}\AgdaSpace{}%
\AgdaDatatype{Sub}\AgdaSpace{}%
\AgdaGeneralizable{Ψ}\AgdaSpace{}%
\AgdaGeneralizable{Δ}\<%
\\
%
\>[2]\AgdaFunction{wks}\AgdaSpace{}%
\AgdaInductiveConstructor{ε}\AgdaSpace{}%
\AgdaBound{p}%
\>[17]\AgdaSymbol{=}\AgdaSpace{}%
\AgdaInductiveConstructor{ε}\<%
\\
%
\>[2]\AgdaFunction{wks}\AgdaSpace{}%
\AgdaSymbol{(}\AgdaBound{s}\AgdaSpace{}%
\AgdaOperator{\AgdaInductiveConstructor{▹}}\AgdaSpace{}%
\AgdaBound{x}\AgdaSymbol{)}\AgdaSpace{}%
\AgdaBound{p}%
\>[17]\AgdaSymbol{=}\AgdaSpace{}%
\AgdaSymbol{(}\AgdaFunction{wks}\AgdaSpace{}%
\AgdaBound{s}\AgdaSpace{}%
\AgdaBound{p}\AgdaSymbol{)}\AgdaSpace{}%
\AgdaOperator{\AgdaInductiveConstructor{▹}}\AgdaSpace{}%
\AgdaFunction{wk}\AgdaSpace{}%
\AgdaBound{p}\AgdaSpace{}%
\AgdaBound{x}\<%
\end{code}}
\end{mathpar}
Using \AF{wks} we define two useful combinators: \AF{sdrop} to lift all
expressions in the \AD{Sub} list into a context that is extended by one variable;
\AF{skeep} to shift the substitution by one variable, keeping the top variable
as is.  With these combinators we define the identity substitution \AF{sub-id}
that has no effect when applying it.  Finally, the type of the actual substitution
that replaces all the variables in \AF{E} according to some \AF{Sub} list is given
by \AF{sub}.
\begin{mathpar}
\codeblock{\begin{code}%
%
\>[2]\AgdaFunction{sdrop}\AgdaSpace{}%
\AgdaSymbol{:}\AgdaSpace{}%
\AgdaDatatype{Sub}\AgdaSpace{}%
\AgdaGeneralizable{Γ}\AgdaSpace{}%
\AgdaGeneralizable{Δ}\AgdaSpace{}%
\AgdaSymbol{→}\AgdaSpace{}%
\AgdaDatatype{Sub}\AgdaSpace{}%
\AgdaSymbol{(}\AgdaGeneralizable{Γ}\AgdaSpace{}%
\AgdaOperator{\AgdaInductiveConstructor{▹}}\AgdaSpace{}%
\AgdaGeneralizable{is}\AgdaSymbol{)}\AgdaSpace{}%
\AgdaGeneralizable{Δ}\<%
\\
%
\>[2]\AgdaFunction{sdrop}\AgdaSpace{}%
\AgdaBound{s}\AgdaSpace{}%
\AgdaSymbol{=}\AgdaSpace{}%
\AgdaFunction{wks}\AgdaSpace{}%
\AgdaBound{s}\AgdaSpace{}%
\AgdaSymbol{(}\AgdaInductiveConstructor{skip}\AgdaSpace{}%
\AgdaFunction{⊆-eq}\AgdaSymbol{)}\<%
\\
%
\\[\AgdaEmptyExtraSkip]%
%
\>[2]\AgdaFunction{skeep}\AgdaSpace{}%
\AgdaSymbol{:}\AgdaSpace{}%
\AgdaDatatype{Sub}\AgdaSpace{}%
\AgdaGeneralizable{Γ}\AgdaSpace{}%
\AgdaGeneralizable{Δ}\AgdaSpace{}%
\AgdaSymbol{→}\AgdaSpace{}%
\AgdaDatatype{Sub}\AgdaSpace{}%
\AgdaSymbol{(}\AgdaGeneralizable{Γ}\AgdaSpace{}%
\AgdaOperator{\AgdaInductiveConstructor{▹}}\AgdaSpace{}%
\AgdaGeneralizable{is}\AgdaSymbol{)}\AgdaSpace{}%
\AgdaSymbol{(}\AgdaGeneralizable{Δ}\AgdaSpace{}%
\AgdaOperator{\AgdaInductiveConstructor{▹}}\AgdaSpace{}%
\AgdaGeneralizable{is}\AgdaSymbol{)}\<%
\\
%
\>[2]\AgdaFunction{skeep}\AgdaSpace{}%
\AgdaBound{s}\AgdaSpace{}%
\AgdaSymbol{=}\AgdaSpace{}%
\AgdaFunction{sdrop}\AgdaSpace{}%
\AgdaBound{s}\AgdaSpace{}%
\AgdaOperator{\AgdaInductiveConstructor{▹}}\AgdaSpace{}%
\AgdaInductiveConstructor{var}\AgdaSpace{}%
\AgdaInductiveConstructor{v₀}\<%
\end{code}}
\and
\codeblock{\begin{code}%
%
\>[2]\AgdaFunction{sub-id}\AgdaSpace{}%
\AgdaSymbol{:}\AgdaSpace{}%
\AgdaDatatype{Sub}\AgdaSpace{}%
\AgdaGeneralizable{Γ}\AgdaSpace{}%
\AgdaGeneralizable{Γ}\<%
\\
%
\>[2]\AgdaFunction{sub-id}\AgdaSpace{}%
\AgdaSymbol{\{}\AgdaInductiveConstructor{ε}\AgdaSymbol{\}}%
\>[18]\AgdaSymbol{=}\AgdaSpace{}%
\AgdaInductiveConstructor{ε}\<%
\\
%
\>[2]\AgdaFunction{sub-id}\AgdaSpace{}%
\AgdaSymbol{\{}\AgdaBound{Γ}\AgdaSpace{}%
\AgdaOperator{\AgdaInductiveConstructor{▹}}\AgdaSpace{}%
\AgdaBound{x}\AgdaSymbol{\}}%
\>[18]\AgdaSymbol{=}\AgdaSpace{}%
\AgdaFunction{skeep}\AgdaSpace{}%
\AgdaFunction{sub-id}\<%
\\
%
\\[\AgdaEmptyExtraSkip]%
%
\>[2]\AgdaFunction{sub}\AgdaSpace{}%
\AgdaSymbol{:}\AgdaSpace{}%
\AgdaDatatype{E}\AgdaSpace{}%
\AgdaGeneralizable{Δ}\AgdaSpace{}%
\AgdaGeneralizable{is}\AgdaSpace{}%
\AgdaSymbol{→}\AgdaSpace{}%
\AgdaDatatype{Sub}\AgdaSpace{}%
\AgdaGeneralizable{Γ}\AgdaSpace{}%
\AgdaGeneralizable{Δ}\AgdaSpace{}%
\AgdaSymbol{→}\AgdaSpace{}%
\AgdaDatatype{E}\AgdaSpace{}%
\AgdaGeneralizable{Γ}\AgdaSpace{}%
\AgdaGeneralizable{is}\<%
\end{code}}
\end{mathpar}
\begin{code}[hide]%
%
\>[2]\AgdaFunction{subv}\AgdaSpace{}%
\AgdaSymbol{:}\AgdaSpace{}%
\AgdaDatatype{Sub}\AgdaSpace{}%
\AgdaGeneralizable{Γ}\AgdaSpace{}%
\AgdaGeneralizable{Δ}\AgdaSpace{}%
\AgdaSymbol{→}\AgdaSpace{}%
\AgdaGeneralizable{is}\AgdaSpace{}%
\AgdaOperator{\AgdaDatatype{∈}}\AgdaSpace{}%
\AgdaGeneralizable{Δ}\AgdaSpace{}%
\AgdaSymbol{→}\AgdaSpace{}%
\AgdaDatatype{E}\AgdaSpace{}%
\AgdaGeneralizable{Γ}\AgdaSpace{}%
\AgdaGeneralizable{is}\<%
\\
%
\>[2]\AgdaFunction{subv}\AgdaSpace{}%
\AgdaSymbol{(}\AgdaBound{s}\AgdaSpace{}%
\AgdaOperator{\AgdaInductiveConstructor{▹}}\AgdaSpace{}%
\AgdaBound{x}\AgdaSymbol{)}\AgdaSpace{}%
\AgdaInductiveConstructor{v₀}%
\>[23]\AgdaSymbol{=}\AgdaSpace{}%
\AgdaBound{x}\<%
\\
%
\>[2]\AgdaFunction{subv}\AgdaSpace{}%
\AgdaSymbol{(}\AgdaBound{s}\AgdaSpace{}%
\AgdaOperator{\AgdaInductiveConstructor{▹}}\AgdaSpace{}%
\AgdaBound{x}\AgdaSymbol{)}\AgdaSpace{}%
\AgdaSymbol{(}\AgdaInductiveConstructor{vₛ}\AgdaSpace{}%
\AgdaBound{v}\AgdaSymbol{)}%
\>[23]\AgdaSymbol{=}\AgdaSpace{}%
\AgdaFunction{subv}\AgdaSpace{}%
\AgdaBound{s}\AgdaSpace{}%
\AgdaBound{v}\<%
\\
\>[0]\<%
\\
%
\>[2]\AgdaFunction{sub}\AgdaSpace{}%
\AgdaSymbol{(}\AgdaInductiveConstructor{var}\AgdaSpace{}%
\AgdaBound{x}\AgdaSymbol{)}\AgdaSpace{}%
\AgdaBound{s}\AgdaSpace{}%
\AgdaSymbol{=}\AgdaSpace{}%
\AgdaFunction{subv}\AgdaSpace{}%
\AgdaBound{s}\AgdaSpace{}%
\AgdaBound{x}\<%
\\
%
\>[2]\AgdaFunction{sub}\AgdaSpace{}%
\AgdaInductiveConstructor{zero}\AgdaSpace{}%
\AgdaBound{s}\AgdaSpace{}%
\AgdaSymbol{=}\AgdaSpace{}%
\AgdaInductiveConstructor{zero}\<%
\\
%
\>[2]\AgdaFunction{sub}\AgdaSpace{}%
\AgdaInductiveConstructor{one}\AgdaSpace{}%
\AgdaBound{s}\AgdaSpace{}%
\AgdaSymbol{=}\AgdaSpace{}%
\AgdaInductiveConstructor{one}\<%
\\
%
\>[2]\AgdaFunction{sub}\AgdaSpace{}%
\AgdaSymbol{(}\AgdaInductiveConstructor{imaps}\AgdaSpace{}%
\AgdaBound{e}\AgdaSymbol{)}\AgdaSpace{}%
\AgdaBound{s}\AgdaSpace{}%
\AgdaSymbol{=}\AgdaSpace{}%
\AgdaInductiveConstructor{imaps}\AgdaSpace{}%
\AgdaSymbol{(}\AgdaFunction{sub}\AgdaSpace{}%
\AgdaBound{e}\AgdaSpace{}%
\AgdaSymbol{(}\AgdaFunction{skeep}\AgdaSpace{}%
\AgdaBound{s}\AgdaSymbol{))}\<%
\\
%
\>[2]\AgdaFunction{sub}\AgdaSpace{}%
\AgdaSymbol{(}\AgdaInductiveConstructor{sels}\AgdaSpace{}%
\AgdaBound{e}\AgdaSpace{}%
\AgdaBound{e₁}\AgdaSymbol{)}\AgdaSpace{}%
\AgdaBound{s}\AgdaSpace{}%
\AgdaSymbol{=}\AgdaSpace{}%
\AgdaInductiveConstructor{sels}\AgdaSpace{}%
\AgdaSymbol{(}\AgdaFunction{sub}\AgdaSpace{}%
\AgdaBound{e}\AgdaSpace{}%
\AgdaBound{s}\AgdaSymbol{)}\AgdaSpace{}%
\AgdaSymbol{(}\AgdaFunction{sub}\AgdaSpace{}%
\AgdaBound{e₁}\AgdaSpace{}%
\AgdaBound{s}\AgdaSymbol{)}\<%
\\
%
\>[2]\AgdaFunction{sub}\AgdaSpace{}%
\AgdaSymbol{(}\AgdaInductiveConstructor{imap}\AgdaSpace{}%
\AgdaBound{e}\AgdaSymbol{)}\AgdaSpace{}%
\AgdaBound{s}\AgdaSpace{}%
\AgdaSymbol{=}\AgdaSpace{}%
\AgdaInductiveConstructor{imap}\AgdaSpace{}%
\AgdaSymbol{(}\AgdaFunction{sub}\AgdaSpace{}%
\AgdaBound{e}\AgdaSpace{}%
\AgdaSymbol{(}\AgdaFunction{skeep}\AgdaSpace{}%
\AgdaBound{s}\AgdaSymbol{))}\<%
\\
%
\>[2]\AgdaFunction{sub}\AgdaSpace{}%
\AgdaSymbol{(}\AgdaInductiveConstructor{sel}\AgdaSpace{}%
\AgdaBound{e}\AgdaSpace{}%
\AgdaBound{e₁}\AgdaSymbol{)}\AgdaSpace{}%
\AgdaBound{s}\AgdaSpace{}%
\AgdaSymbol{=}\AgdaSpace{}%
\AgdaInductiveConstructor{sel}\AgdaSpace{}%
\AgdaSymbol{(}\AgdaFunction{sub}\AgdaSpace{}%
\AgdaBound{e}\AgdaSpace{}%
\AgdaBound{s}\AgdaSymbol{)}\AgdaSpace{}%
\AgdaSymbol{(}\AgdaFunction{sub}\AgdaSpace{}%
\AgdaBound{e₁}\AgdaSpace{}%
\AgdaBound{s}\AgdaSymbol{)}\<%
\\
%
\>[2]\AgdaFunction{sub}\AgdaSpace{}%
\AgdaSymbol{(}\AgdaInductiveConstructor{imapb}\AgdaSpace{}%
\AgdaBound{x}\AgdaSpace{}%
\AgdaBound{e}\AgdaSymbol{)}\AgdaSpace{}%
\AgdaBound{s}\AgdaSpace{}%
\AgdaSymbol{=}\AgdaSpace{}%
\AgdaInductiveConstructor{imapb}\AgdaSpace{}%
\AgdaBound{x}\AgdaSpace{}%
\AgdaSymbol{(}\AgdaFunction{sub}\AgdaSpace{}%
\AgdaBound{e}\AgdaSpace{}%
\AgdaSymbol{(}\AgdaFunction{skeep}\AgdaSpace{}%
\AgdaBound{s}\AgdaSymbol{))}\<%
\\
%
\>[2]\AgdaFunction{sub}\AgdaSpace{}%
\AgdaSymbol{(}\AgdaInductiveConstructor{selb}\AgdaSpace{}%
\AgdaBound{x}\AgdaSpace{}%
\AgdaBound{e}\AgdaSpace{}%
\AgdaBound{e₁}\AgdaSymbol{)}\AgdaSpace{}%
\AgdaBound{s}\AgdaSpace{}%
\AgdaSymbol{=}\AgdaSpace{}%
\AgdaInductiveConstructor{selb}\AgdaSpace{}%
\AgdaBound{x}\AgdaSpace{}%
\AgdaSymbol{(}\AgdaFunction{sub}\AgdaSpace{}%
\AgdaBound{e}\AgdaSpace{}%
\AgdaBound{s}\AgdaSymbol{)}\AgdaSpace{}%
\AgdaSymbol{(}\AgdaFunction{sub}\AgdaSpace{}%
\AgdaBound{e₁}\AgdaSpace{}%
\AgdaBound{s}\AgdaSymbol{)}\<%
\\
%
\>[2]\AgdaFunction{sub}\AgdaSpace{}%
\AgdaSymbol{(}\AgdaInductiveConstructor{sum}\AgdaSpace{}%
\AgdaBound{e}\AgdaSymbol{)}\AgdaSpace{}%
\AgdaBound{s}\AgdaSpace{}%
\AgdaSymbol{=}\AgdaSpace{}%
\AgdaInductiveConstructor{sum}\AgdaSpace{}%
\AgdaSymbol{(}\AgdaFunction{sub}\AgdaSpace{}%
\AgdaBound{e}\AgdaSpace{}%
\AgdaSymbol{(}\AgdaFunction{skeep}\AgdaSpace{}%
\AgdaBound{s}\AgdaSymbol{))}\<%
\\
%
\>[2]\AgdaFunction{sub}\AgdaSpace{}%
\AgdaSymbol{(}\AgdaInductiveConstructor{zero-but}\AgdaSpace{}%
\AgdaBound{e}\AgdaSpace{}%
\AgdaBound{e₁}\AgdaSpace{}%
\AgdaBound{e₂}\AgdaSymbol{)}\AgdaSpace{}%
\AgdaBound{s}\AgdaSpace{}%
\AgdaSymbol{=}\AgdaSpace{}%
\AgdaInductiveConstructor{zero-but}\AgdaSpace{}%
\AgdaSymbol{(}\AgdaFunction{sub}\AgdaSpace{}%
\AgdaBound{e}\AgdaSpace{}%
\AgdaBound{s}\AgdaSymbol{)}\AgdaSpace{}%
\AgdaSymbol{(}\AgdaFunction{sub}\AgdaSpace{}%
\AgdaBound{e₁}\AgdaSpace{}%
\AgdaBound{s}\AgdaSymbol{)}\AgdaSpace{}%
\AgdaSymbol{(}\AgdaFunction{sub}\AgdaSpace{}%
\AgdaBound{e₂}\AgdaSpace{}%
\AgdaBound{s}\AgdaSymbol{)}\<%
\\
%
\>[2]\AgdaFunction{sub}\AgdaSpace{}%
\AgdaSymbol{(}\AgdaInductiveConstructor{slide}\AgdaSpace{}%
\AgdaBound{e}\AgdaSpace{}%
\AgdaBound{x}\AgdaSpace{}%
\AgdaBound{e₁}\AgdaSpace{}%
\AgdaBound{x₁}\AgdaSymbol{)}\AgdaSpace{}%
\AgdaBound{s}\AgdaSpace{}%
\AgdaSymbol{=}\AgdaSpace{}%
\AgdaInductiveConstructor{slide}\AgdaSpace{}%
\AgdaSymbol{(}\AgdaFunction{sub}\AgdaSpace{}%
\AgdaBound{e}\AgdaSpace{}%
\AgdaBound{s}\AgdaSymbol{)}\AgdaSpace{}%
\AgdaBound{x}\AgdaSpace{}%
\AgdaSymbol{(}\AgdaFunction{sub}\AgdaSpace{}%
\AgdaBound{e₁}\AgdaSpace{}%
\AgdaBound{s}\AgdaSymbol{)}\AgdaSpace{}%
\AgdaBound{x₁}\<%
\\
%
\>[2]\AgdaFunction{sub}\AgdaSpace{}%
\AgdaSymbol{(}\AgdaInductiveConstructor{backslide}\AgdaSpace{}%
\AgdaBound{e}\AgdaSpace{}%
\AgdaBound{e₁}\AgdaSpace{}%
\AgdaBound{x}\AgdaSpace{}%
\AgdaBound{x₁}\AgdaSymbol{)}\AgdaSpace{}%
\AgdaBound{s}\AgdaSpace{}%
\AgdaSymbol{=}\AgdaSpace{}%
\AgdaInductiveConstructor{backslide}\AgdaSpace{}%
\AgdaSymbol{(}\AgdaFunction{sub}\AgdaSpace{}%
\AgdaBound{e}\AgdaSpace{}%
\AgdaBound{s}\AgdaSymbol{)}\AgdaSpace{}%
\AgdaSymbol{(}\AgdaFunction{sub}\AgdaSpace{}%
\AgdaBound{e₁}\AgdaSpace{}%
\AgdaBound{s}\AgdaSymbol{)}\AgdaSpace{}%
\AgdaBound{x}\AgdaSpace{}%
\AgdaBound{x₁}\<%
\\
%
\>[2]\AgdaFunction{sub}\AgdaSpace{}%
\AgdaSymbol{(}\AgdaInductiveConstructor{logistic}\AgdaSpace{}%
\AgdaBound{e}\AgdaSymbol{)}\AgdaSpace{}%
\AgdaBound{s}\AgdaSpace{}%
\AgdaSymbol{=}\AgdaSpace{}%
\AgdaInductiveConstructor{logistic}\AgdaSpace{}%
\AgdaSymbol{(}\AgdaFunction{sub}\AgdaSpace{}%
\AgdaBound{e}\AgdaSpace{}%
\AgdaBound{s}\AgdaSymbol{)}\<%
\\
%
\>[2]\AgdaFunction{sub}\AgdaSpace{}%
\AgdaSymbol{(}\AgdaInductiveConstructor{bin}\AgdaSpace{}%
\AgdaBound{x}\AgdaSpace{}%
\AgdaBound{e}\AgdaSpace{}%
\AgdaBound{e₁}\AgdaSymbol{)}\AgdaSpace{}%
\AgdaBound{s}\AgdaSpace{}%
\AgdaSymbol{=}\AgdaSpace{}%
\AgdaInductiveConstructor{bin}\AgdaSpace{}%
\AgdaBound{x}\AgdaSpace{}%
\AgdaSymbol{(}\AgdaFunction{sub}\AgdaSpace{}%
\AgdaBound{e}\AgdaSpace{}%
\AgdaBound{s}\AgdaSymbol{)}\AgdaSpace{}%
\AgdaSymbol{(}\AgdaFunction{sub}\AgdaSpace{}%
\AgdaBound{e₁}\AgdaSpace{}%
\AgdaBound{s}\AgdaSymbol{)}\<%
\\
%
\>[2]\AgdaFunction{sub}\AgdaSpace{}%
\AgdaSymbol{(}\AgdaInductiveConstructor{scaledown}\AgdaSpace{}%
\AgdaBound{x}\AgdaSpace{}%
\AgdaBound{e}\AgdaSymbol{)}\AgdaSpace{}%
\AgdaBound{s}\AgdaSpace{}%
\AgdaSymbol{=}\AgdaSpace{}%
\AgdaInductiveConstructor{scaledown}\AgdaSpace{}%
\AgdaBound{x}\AgdaSpace{}%
\AgdaSymbol{(}\AgdaFunction{sub}\AgdaSpace{}%
\AgdaBound{e}\AgdaSpace{}%
\AgdaBound{s}\AgdaSymbol{)}\<%
\\
%
\>[2]\AgdaFunction{sub}\AgdaSpace{}%
\AgdaSymbol{(}\AgdaInductiveConstructor{minus}\AgdaSpace{}%
\AgdaBound{e}\AgdaSymbol{)}\AgdaSpace{}%
\AgdaBound{s}\AgdaSpace{}%
\AgdaSymbol{=}\AgdaSpace{}%
\AgdaInductiveConstructor{minus}\AgdaSpace{}%
\AgdaSymbol{(}\AgdaFunction{sub}\AgdaSpace{}%
\AgdaBound{e}\AgdaSpace{}%
\AgdaBound{s}\AgdaSymbol{)}\<%
\\
%
\>[2]\AgdaFunction{sub}\AgdaSpace{}%
\AgdaSymbol{(}\AgdaInductiveConstructor{let′}\AgdaSpace{}%
\AgdaBound{e}\AgdaSpace{}%
\AgdaBound{e₁}\AgdaSymbol{)}\AgdaSpace{}%
\AgdaBound{s}\AgdaSpace{}%
\AgdaSymbol{=}\AgdaSpace{}%
\AgdaInductiveConstructor{let′}\AgdaSpace{}%
\AgdaSymbol{(}\AgdaFunction{sub}\AgdaSpace{}%
\AgdaBound{e}\AgdaSpace{}%
\AgdaBound{s}\AgdaSymbol{)}\AgdaSpace{}%
\AgdaSymbol{(}\AgdaFunction{sub}\AgdaSpace{}%
\AgdaBound{e₁}\AgdaSpace{}%
\AgdaSymbol{(}\AgdaFunction{skeep}\AgdaSpace{}%
\AgdaBound{s}\AgdaSymbol{))}\<%
\\
%
\\[\AgdaEmptyExtraSkip]%
%
\>[2]\AgdaOperator{\AgdaFunction{\AgdaUnderscore{}∙ˢ\AgdaUnderscore{}}}\AgdaSpace{}%
\AgdaSymbol{:}\AgdaSpace{}%
\AgdaDatatype{Sub}\AgdaSpace{}%
\AgdaGeneralizable{Δ}\AgdaSpace{}%
\AgdaGeneralizable{Ψ}\AgdaSpace{}%
\AgdaSymbol{→}\AgdaSpace{}%
\AgdaDatatype{Sub}\AgdaSpace{}%
\AgdaGeneralizable{Γ}\AgdaSpace{}%
\AgdaGeneralizable{Δ}\AgdaSpace{}%
\AgdaSymbol{→}\AgdaSpace{}%
\AgdaDatatype{Sub}\AgdaSpace{}%
\AgdaGeneralizable{Γ}\AgdaSpace{}%
\AgdaGeneralizable{Ψ}\<%
\\
%
\>[2]\AgdaInductiveConstructor{ε}\AgdaSpace{}%
\AgdaOperator{\AgdaFunction{∙ˢ}}\AgdaSpace{}%
\AgdaBound{t}\AgdaSpace{}%
\AgdaSymbol{=}\AgdaSpace{}%
\AgdaInductiveConstructor{ε}\<%
\\
%
\>[2]\AgdaSymbol{(}\AgdaBound{s}\AgdaSpace{}%
\AgdaOperator{\AgdaInductiveConstructor{▹}}\AgdaSpace{}%
\AgdaBound{x}\AgdaSymbol{)}\AgdaSpace{}%
\AgdaOperator{\AgdaFunction{∙ˢ}}\AgdaSpace{}%
\AgdaBound{t}\AgdaSpace{}%
\AgdaSymbol{=}\AgdaSpace{}%
\AgdaSymbol{(}\AgdaBound{s}\AgdaSpace{}%
\AgdaOperator{\AgdaFunction{∙ˢ}}\AgdaSpace{}%
\AgdaBound{t}\AgdaSymbol{)}\AgdaSpace{}%
\AgdaOperator{\AgdaInductiveConstructor{▹}}\AgdaSpace{}%
\AgdaFunction{sub}\AgdaSpace{}%
\AgdaBound{x}\AgdaSpace{}%
\AgdaBound{t}\<%
\end{code}
As our contexts are not dependent (\eg{} the type of the variables does not
depend on previous variables) we can define a substitution that swaps two top
variables in the context.  This substitution is given by \AF{sub-swap} and it
will be used in optimisations.
\begin{mathpar}
\codeblock{\begin{code}%
%
\>[2]\AgdaFunction{sub-swap}\AgdaSpace{}%
\AgdaSymbol{:}\AgdaSpace{}%
\AgdaDatatype{Sub}\AgdaSpace{}%
\AgdaSymbol{(}\AgdaGeneralizable{Γ}\AgdaSpace{}%
\AgdaOperator{\AgdaInductiveConstructor{▹}}\AgdaSpace{}%
\AgdaGeneralizable{is}\AgdaSpace{}%
\AgdaOperator{\AgdaInductiveConstructor{▹}}\AgdaSpace{}%
\AgdaGeneralizable{ip}\AgdaSymbol{)}\AgdaSpace{}%
\AgdaSymbol{(}\AgdaGeneralizable{Γ}\AgdaSpace{}%
\AgdaOperator{\AgdaInductiveConstructor{▹}}\AgdaSpace{}%
\AgdaGeneralizable{ip}\AgdaSpace{}%
\AgdaOperator{\AgdaInductiveConstructor{▹}}\AgdaSpace{}%
\AgdaGeneralizable{is}\AgdaSymbol{)}\<%
\\
%
\>[2]\AgdaFunction{sub-swap}\AgdaSpace{}%
\AgdaSymbol{=}\AgdaSpace{}%
\AgdaFunction{sdrop}\AgdaSpace{}%
\AgdaSymbol{(}\AgdaFunction{sdrop}\AgdaSpace{}%
\AgdaFunction{sub-id}\AgdaSymbol{)}\AgdaSpace{}%
\AgdaOperator{\AgdaInductiveConstructor{▹}}\AgdaSpace{}%
\AgdaInductiveConstructor{var}\AgdaSpace{}%
\AgdaInductiveConstructor{v₀}\AgdaSpace{}%
\AgdaOperator{\AgdaInductiveConstructor{▹}}\AgdaSpace{}%
\AgdaInductiveConstructor{var}\AgdaSpace{}%
\AgdaSymbol{(}\AgdaInductiveConstructor{vₛ}\AgdaSpace{}%
\AgdaInductiveConstructor{v₀}\AgdaSymbol{)}\<%
\end{code}}
\end{mathpar}

\subsection{Syntax}
Deeply-embedded DSLs with intrinsic de Bruijn variables guarantee well-scopedness,
but they are not intuitive for humans.  This section proposes a mechanism
to overcome the encoding burden by providing HOAS-like syntactic wrappers for
\AD{E}.

Our goal is to replace de Bruijn variables with Agda's variables.  One immediate
difficulty with this approach is that whenever we go under binders such as 
\AC{imap} or \AC{let′}, all the variables/expressions we defined before have to be
lifted into the extended context.  We tackle this problem by forcing Agda to
compute such lifted expressions automatically by (ab)using Agda's
instance resolution mechanism.

We only ever need to lift expressions/variables into
contexts where a certain number of variables were added at the end, \ie{} the
prefix of the extended context always correspond to the context of the original
expression.  Therefore, we define a binary relation \AF{Prefix} \AB{Γ} \AB{Δ}
that determines whether the \AB{Γ} is a prefix of \AB{Δ}.  We annotate
constructors of \AF{Prefix} with the \AK{instance} keyword, and we wrap
the argument of the \AC{suc} constructor into double braces, turning this into
an instance argument.  Simultaneously, \AF{prefix-⊆} defines a translation from
\AD{Prefix} into order-preserving embeddings \AD{⊆} so that we can
leverage weakening.
\begin{mathpar}
\codeblock{\begin{code}[hide]%
\>[0]\AgdaKeyword{module}\AgdaSpace{}%
\AgdaModule{Syntax}\AgdaSpace{}%
\AgdaKeyword{where}\<%
\\
\>[0][@{}l@{\AgdaIndent{0}}]%
\>[2]\AgdaKeyword{open}\AgdaSpace{}%
\AgdaModule{Lang}\<%
\\
%
\>[2]\AgdaKeyword{open}\AgdaSpace{}%
\AgdaKeyword{import}\AgdaSpace{}%
\AgdaModule{Data.List}\AgdaSpace{}%
\AgdaSymbol{as}\AgdaSpace{}%
\AgdaModule{L}\AgdaSpace{}%
\AgdaKeyword{using}\AgdaSpace{}%
\AgdaSymbol{(}\AgdaDatatype{List}\AgdaSymbol{;}\AgdaSpace{}%
\AgdaInductiveConstructor{[]}\AgdaSymbol{;}\AgdaSpace{}%
\AgdaOperator{\AgdaInductiveConstructor{\AgdaUnderscore{}∷\AgdaUnderscore{}}}\AgdaSymbol{)}\<%
\\
%
\>[2]\AgdaKeyword{open}\AgdaSpace{}%
\AgdaModule{Array}\AgdaSpace{}%
\AgdaKeyword{hiding}\AgdaSpace{}%
\AgdaSymbol{(}\AgdaFunction{sum}\AgdaSymbol{)}\<%
\\
%
\>[2]\AgdaKeyword{open}\AgdaSpace{}%
\AgdaModule{WkSub}\<%
\end{code}
\begin{code}%
%
\>[2]\AgdaKeyword{data}\AgdaSpace{}%
\AgdaDatatype{Prefix}\AgdaSpace{}%
\AgdaSymbol{:}\AgdaSpace{}%
\AgdaSymbol{(}\AgdaBound{Γ}\AgdaSpace{}%
\AgdaBound{Δ}\AgdaSpace{}%
\AgdaSymbol{:}\AgdaSpace{}%
\AgdaDatatype{Ctx}\AgdaSymbol{)}\AgdaSpace{}%
\AgdaSymbol{→}\AgdaSpace{}%
\AgdaPrimitive{Set}\AgdaSpace{}%
\AgdaKeyword{where}\<%
\\
\>[2][@{}l@{\AgdaIndent{0}}]%
\>[4]\AgdaKeyword{instance}\<%
\\
\>[4][@{}l@{\AgdaIndent{0}}]%
\>[6]\AgdaInductiveConstructor{zero}\AgdaSpace{}%
\AgdaSymbol{:}\AgdaSpace{}%
\AgdaDatatype{Prefix}\AgdaSpace{}%
\AgdaGeneralizable{Γ}\AgdaSpace{}%
\AgdaGeneralizable{Γ}\<%
\\
%
\>[6]\AgdaInductiveConstructor{suc}%
\>[11]\AgdaSymbol{:}\AgdaSpace{}%
\AgdaSymbol{⦃}\AgdaSpace{}%
\AgdaDatatype{Prefix}\AgdaSpace{}%
\AgdaGeneralizable{Γ}\AgdaSpace{}%
\AgdaGeneralizable{Δ}\AgdaSpace{}%
\AgdaSymbol{⦄}\AgdaSpace{}%
\AgdaSymbol{→}\AgdaSpace{}%
\AgdaDatatype{Prefix}\AgdaSpace{}%
\AgdaGeneralizable{Γ}\AgdaSpace{}%
\AgdaSymbol{(}\AgdaGeneralizable{Δ}\AgdaSpace{}%
\AgdaOperator{\AgdaInductiveConstructor{▹}}\AgdaSpace{}%
\AgdaGeneralizable{is}\AgdaSymbol{)}\<%
\end{code}}
\and
\codeblock{\begin{code}%
%
\>[2]\AgdaFunction{prefix-⊆}\AgdaSpace{}%
\AgdaSymbol{:}\AgdaSpace{}%
\AgdaDatatype{Prefix}\AgdaSpace{}%
\AgdaGeneralizable{Γ}\AgdaSpace{}%
\AgdaGeneralizable{Δ}\AgdaSpace{}%
\AgdaSymbol{→}\AgdaSpace{}%
\AgdaGeneralizable{Γ}\AgdaSpace{}%
\AgdaOperator{\AgdaDatatype{⊆}}\AgdaSpace{}%
\AgdaGeneralizable{Δ}\<%
\\
%
\>[2]\AgdaFunction{prefix-⊆}\AgdaSpace{}%
\AgdaInductiveConstructor{zero}%
\>[24]\AgdaSymbol{=}\AgdaSpace{}%
\AgdaFunction{⊆-eq}\<%
\\
%
\>[2]\AgdaFunction{prefix-⊆}\AgdaSpace{}%
\AgdaSymbol{(}\AgdaInductiveConstructor{suc}\AgdaSpace{}%
\AgdaSymbol{⦃}\AgdaSpace{}%
\AgdaBound{p}\AgdaSpace{}%
\AgdaSymbol{⦄)}%
\>[24]\AgdaSymbol{=}\AgdaSpace{}%
\AgdaInductiveConstructor{skip}\AgdaSpace{}%
\AgdaSymbol{(}\AgdaFunction{prefix-⊆}\AgdaSpace{}%
\AgdaBound{p}\AgdaSymbol{)}\<%
\end{code}}
\end{mathpar}

As a result, Agda is able to construct an element of
\AC{Prefix} \AB{Γ} (\AB{Γ} \AC{▹} \AB{is} ▹ \AB{ip} ▹ \AB{iq}) automatically.
Similarly
to hidden arguments, there is no guarantee that the solution will be found
in all cases --- Agda will report an error in case of failure.
Note that instance resolution is looking for the \emph{unique} solution.
This is the reason why we could not use \AD{⊆} instead of \AD{Prefix} easily,
even thought the former is more general than the latter.
Contexts (\AB{Γ} \AC{▹} \AB{is}) and (\AB{Γ} \AC{▹} \AB{is} \AC{▹} \AB{is}),
are related by \AD{Prefix} in a unique way.  However, these two contexts are
related by \AD{⊆} in two different ways (by keeping the first or the
second variable).

We introduce generalised variables \AF{GV} and expressions \AF{GV} that
are defined in context \AB{Γ} but can be lifted into any context \AB{Δ}, given
that \AB{Γ} is a prefix of \AB{Δ}.  The trick here is that both types define
a function of two hidden arguments that Agda will be able to fill-in automatically.
\begin{mathpar}
\codeblock{\begin{code}%
%
\>[2]\AgdaFunction{GE}\AgdaSpace{}%
\AgdaSymbol{:}\AgdaSpace{}%
\AgdaDatatype{Ctx}\AgdaSpace{}%
\AgdaSymbol{→}\AgdaSpace{}%
\AgdaDatatype{IS}\AgdaSpace{}%
\AgdaSymbol{→}\AgdaSpace{}%
\AgdaPrimitive{Set}\<%
\\
%
\>[2]\AgdaFunction{GE}\AgdaSpace{}%
\AgdaBound{Γ}\AgdaSpace{}%
\AgdaBound{is}\AgdaSpace{}%
\AgdaSymbol{=}\AgdaSpace{}%
\AgdaSymbol{∀}\AgdaSpace{}%
\AgdaSymbol{\{}\AgdaBound{Δ}\AgdaSymbol{\}}\AgdaSpace{}%
\AgdaSymbol{→}\AgdaSpace{}%
\AgdaSymbol{⦃}\AgdaSpace{}%
\AgdaDatatype{Prefix}\AgdaSpace{}%
\AgdaBound{Γ}\AgdaSpace{}%
\AgdaBound{Δ}\AgdaSpace{}%
\AgdaSymbol{⦄}\AgdaSpace{}%
\AgdaSymbol{→}\AgdaSpace{}%
\AgdaDatatype{E}\AgdaSpace{}%
\AgdaBound{Δ}\AgdaSpace{}%
\AgdaBound{is}\<%
\end{code}}
\and
\codeblock{\begin{code}%
%
\>[2]\AgdaFunction{GVar}\AgdaSpace{}%
\AgdaSymbol{:}\AgdaSpace{}%
\AgdaDatatype{Ctx}\AgdaSpace{}%
\AgdaSymbol{→}\AgdaSpace{}%
\AgdaDatatype{IS}\AgdaSpace{}%
\AgdaSymbol{→}\AgdaSpace{}%
\AgdaPrimitive{Set}\<%
\\
%
\>[2]\AgdaFunction{GVar}\AgdaSpace{}%
\AgdaBound{Γ}\AgdaSpace{}%
\AgdaBound{is}\AgdaSpace{}%
\AgdaSymbol{=}\AgdaSpace{}%
\AgdaSymbol{∀}\AgdaSpace{}%
\AgdaSymbol{\{}\AgdaBound{Δ}\AgdaSymbol{\}}\AgdaSpace{}%
\AgdaSymbol{→}\AgdaSpace{}%
\AgdaSymbol{⦃}\AgdaSpace{}%
\AgdaBound{p}\AgdaSpace{}%
\AgdaSymbol{:}\AgdaSpace{}%
\AgdaDatatype{Prefix}\AgdaSpace{}%
\AgdaBound{Γ}\AgdaSpace{}%
\AgdaBound{Δ}\AgdaSpace{}%
\AgdaSymbol{⦄}\AgdaSpace{}%
\AgdaSymbol{→}\AgdaSpace{}%
\AgdaBound{is}\AgdaSpace{}%
\AgdaOperator{\AgdaDatatype{∈}}\AgdaSpace{}%
\AgdaBound{Δ}\<%
\end{code}}
\end{mathpar}
We can lift expressions into generalised expressions in two steps: 
(i) we translate prefixes into \AC{⊆}; (ii) we use weakening that we defined earlier.
\begin{mathpar}
\codeblock{\begin{code}%
%
\>[2]\AgdaOperator{\AgdaFunction{⟨\AgdaUnderscore{}⟩}}\AgdaSpace{}%
\AgdaSymbol{:}\AgdaSpace{}%
\AgdaDatatype{E}\AgdaSpace{}%
\AgdaGeneralizable{Γ}\AgdaSpace{}%
\AgdaGeneralizable{is}\AgdaSpace{}%
\AgdaSymbol{→}\AgdaSpace{}%
\AgdaFunction{GE}\AgdaSpace{}%
\AgdaGeneralizable{Γ}\AgdaSpace{}%
\AgdaGeneralizable{is}\<%
\\
%
\>[2]\AgdaOperator{\AgdaFunction{⟨\AgdaUnderscore{}⟩}}\AgdaSpace{}%
\AgdaBound{t}\AgdaSpace{}%
\AgdaSymbol{\{}\AgdaBound{Δ}\AgdaSymbol{\}}\AgdaSpace{}%
\AgdaSymbol{⦃}\AgdaSpace{}%
\AgdaBound{p}\AgdaSpace{}%
\AgdaSymbol{⦄}\AgdaSpace{}%
\AgdaSymbol{=}\AgdaSpace{}%
\AgdaFunction{wk}\AgdaSpace{}%
\AgdaSymbol{(}\AgdaFunction{prefix-⊆}\AgdaSpace{}%
\AgdaBound{p}\AgdaSymbol{)}\AgdaSpace{}%
\AgdaBound{t}\<%
\end{code}}
\and
\codeblock{\begin{code}%
%
\>[2]\AgdaOperator{\AgdaFunction{⟨\AgdaUnderscore{}⟩ᵛ}}\AgdaSpace{}%
\AgdaSymbol{:}\AgdaSpace{}%
\AgdaGeneralizable{is}\AgdaSpace{}%
\AgdaOperator{\AgdaDatatype{∈}}\AgdaSpace{}%
\AgdaGeneralizable{Γ}\AgdaSpace{}%
\AgdaSymbol{→}\AgdaSpace{}%
\AgdaFunction{GVar}\AgdaSpace{}%
\AgdaGeneralizable{Γ}\AgdaSpace{}%
\AgdaGeneralizable{is}\<%
\\
%
\>[2]\AgdaOperator{\AgdaFunction{⟨\AgdaUnderscore{}⟩ᵛ}}\AgdaSpace{}%
\AgdaBound{v}\AgdaSpace{}%
\AgdaSymbol{⦃}\AgdaSpace{}%
\AgdaBound{p}\AgdaSpace{}%
\AgdaSymbol{⦄}\AgdaSpace{}%
\AgdaSymbol{=}\AgdaSpace{}%
\AgdaFunction{wkv}\AgdaSpace{}%
\AgdaSymbol{(}\AgdaFunction{prefix-⊆}\AgdaSpace{}%
\AgdaBound{p}\AgdaSymbol{)}\AgdaSpace{}%
\AgdaBound{v}\<%
\end{code}}
\end{mathpar}
With these operations we define HOAS-like wrappers for the \AF{E} binders
such as \AC{imap} family, \AC{sum} and \AC{let′}.  Consider
a wrapper for \AC{imap} defined as follows.
\begin{code}%
%
\>[2]\AgdaFunction{Imap}\AgdaSpace{}%
\AgdaSymbol{:}\AgdaSpace{}%
\AgdaSymbol{(}\AgdaFunction{GE}\AgdaSpace{}%
\AgdaSymbol{(}\AgdaGeneralizable{Γ}\AgdaSpace{}%
\AgdaOperator{\AgdaInductiveConstructor{▹}}\AgdaSpace{}%
\AgdaInductiveConstructor{ix}\AgdaSpace{}%
\AgdaGeneralizable{s}\AgdaSymbol{)}\AgdaSpace{}%
\AgdaSymbol{(}\AgdaInductiveConstructor{ix}\AgdaSpace{}%
\AgdaGeneralizable{s}\AgdaSymbol{)}\AgdaSpace{}%
\AgdaSymbol{→}\AgdaSpace{}%
\AgdaDatatype{E}\AgdaSpace{}%
\AgdaSymbol{(}\AgdaGeneralizable{Γ}\AgdaSpace{}%
\AgdaOperator{\AgdaInductiveConstructor{▹}}\AgdaSpace{}%
\AgdaInductiveConstructor{ix}\AgdaSpace{}%
\AgdaGeneralizable{s}\AgdaSymbol{)}\AgdaSpace{}%
\AgdaSymbol{(}\AgdaInductiveConstructor{ar}\AgdaSpace{}%
\AgdaGeneralizable{p}\AgdaSymbol{))}\AgdaSpace{}%
\AgdaSymbol{→}\AgdaSpace{}%
\AgdaDatatype{E}\AgdaSpace{}%
\AgdaGeneralizable{Γ}\AgdaSpace{}%
\AgdaSymbol{(}\AgdaInductiveConstructor{ar}\AgdaSpace{}%
\AgdaSymbol{(}\AgdaGeneralizable{s}\AgdaSpace{}%
\AgdaOperator{\AgdaFunction{⊗}}\AgdaSpace{}%
\AgdaGeneralizable{p}\AgdaSymbol{))}\<%
\\
%
\>[2]\AgdaFunction{Imap}\AgdaSpace{}%
\AgdaBound{f}\AgdaSpace{}%
\AgdaSymbol{=}\AgdaSpace{}%
\AgdaInductiveConstructor{imap}\AgdaSpace{}%
\AgdaSymbol{(}\AgdaBound{f}\AgdaSpace{}%
\AgdaSymbol{(}\AgdaInductiveConstructor{var}\AgdaSpace{}%
\AgdaOperator{\AgdaFunction{⟨}}\AgdaSpace{}%
\AgdaInductiveConstructor{v₀}\AgdaSpace{}%
\AgdaOperator{\AgdaFunction{⟩ᵛ}}\AgdaSymbol{))}\<%
\end{code}
The first argument is a function in Agda's function space where the
argument is a generalised expression of type $\AC{ix}\ s$ in the context
extended by $\AC{ix}\ s$ (the imap index); and the return type is the array
expression that will be computed in the body of the imap.  The implementation
of the wrapper constructs an \AC{imap}, lifting the index variable
$v₀$ of the context \AB{Γ} \AC{▹} $\AC{ix}\ s$ into a larger context that is determined
by the hidden/instance arguments of $f$.  This means that within the body of $f$
we can use the argument under further binders as follows:
\begin{code}%
%
\>[2]\AgdaFunction{\AgdaUnderscore{}}\AgdaSpace{}%
\AgdaSymbol{:}\AgdaSpace{}%
\AgdaDatatype{E}\AgdaSpace{}%
\AgdaInductiveConstructor{ε}\AgdaSpace{}%
\AgdaSymbol{\AgdaUnderscore{}}\<%
\\
%
\>[2]\AgdaSymbol{\AgdaUnderscore{}}\AgdaSpace{}%
\AgdaSymbol{=}\AgdaSpace{}%
\AgdaFunction{Imap}\AgdaSpace{}%
\AgdaSymbol{\{}\AgdaArgument{s}\AgdaSpace{}%
\AgdaSymbol{=}\AgdaSpace{}%
\AgdaInductiveConstructor{ι}\AgdaSpace{}%
\AgdaNumber{5}\AgdaSymbol{\}}\AgdaSpace{}%
\AgdaSymbol{λ}\AgdaSpace{}%
\AgdaBound{i}\AgdaSpace{}%
\AgdaSymbol{→}\AgdaSpace{}%
\AgdaFunction{Imap}\AgdaSpace{}%
\AgdaSymbol{\{}\AgdaArgument{s}\AgdaSpace{}%
\AgdaSymbol{=}\AgdaSpace{}%
\AgdaInductiveConstructor{ι}\AgdaSpace{}%
\AgdaNumber{5}\AgdaSymbol{\}}\AgdaSpace{}%
\AgdaSymbol{λ}\AgdaSpace{}%
\AgdaBound{j}\AgdaSpace{}%
\AgdaSymbol{→}\AgdaSpace{}%
\AgdaInductiveConstructor{sels}\AgdaSpace{}%
\AgdaSymbol{(}\AgdaInductiveConstructor{sel}\AgdaSpace{}%
\AgdaInductiveConstructor{one}\AgdaSpace{}%
\AgdaBound{j}\AgdaSymbol{)}\AgdaSpace{}%
\AgdaBound{i}\<%
\end{code}
The code for wrappers for \AC{sum}, \AC{imaps}, \AC{imapb} looks very similar
so we omit it here.  However, when defining a wrapper for \AC{let′} we
use Agda's \AK{syntax} feature\footnote{See
\url{https://agda.readthedocs.io/en/v2.7.0.1/language/syntax-declarations.html}
for more details.} to define a familiar syntax for let bindings in \AF{E}.
\begin{code}[hide]%
%
\>[2]\AgdaFunction{Sum}%
\>[1664I]\AgdaSymbol{:}\AgdaSpace{}%
\AgdaSymbol{∀}\AgdaSpace{}%
\AgdaSymbol{\{}\AgdaBound{Γ}\AgdaSymbol{\}}\<%
\\
\>[1664I][@{}l@{\AgdaIndent{0}}]%
\>[7]\AgdaSymbol{→}\AgdaSpace{}%
\AgdaSymbol{(}\AgdaFunction{GE}\AgdaSpace{}%
\AgdaSymbol{(}\AgdaBound{Γ}\AgdaSpace{}%
\AgdaOperator{\AgdaInductiveConstructor{▹}}\AgdaSpace{}%
\AgdaInductiveConstructor{ix}\AgdaSpace{}%
\AgdaGeneralizable{s}\AgdaSymbol{)}\AgdaSpace{}%
\AgdaSymbol{(}\AgdaInductiveConstructor{ix}\AgdaSpace{}%
\AgdaGeneralizable{s}\AgdaSymbol{)}\AgdaSpace{}%
\AgdaSymbol{→}\AgdaSpace{}%
\AgdaDatatype{E}\AgdaSpace{}%
\AgdaSymbol{(}\AgdaBound{Γ}\AgdaSpace{}%
\AgdaOperator{\AgdaInductiveConstructor{▹}}\AgdaSpace{}%
\AgdaInductiveConstructor{ix}\AgdaSpace{}%
\AgdaGeneralizable{s}\AgdaSymbol{)}\AgdaSpace{}%
\AgdaSymbol{(}\AgdaInductiveConstructor{ar}\AgdaSpace{}%
\AgdaGeneralizable{p}\AgdaSymbol{))}\<%
\\
%
\>[7]\AgdaSymbol{→}\AgdaSpace{}%
\AgdaDatatype{E}\AgdaSpace{}%
\AgdaBound{Γ}\AgdaSpace{}%
\AgdaSymbol{(}\AgdaInductiveConstructor{ar}\AgdaSpace{}%
\AgdaGeneralizable{p}\AgdaSymbol{)}\<%
\\
%
\>[2]\AgdaFunction{Sum}\AgdaSpace{}%
\AgdaBound{f}\AgdaSpace{}%
\AgdaSymbol{=}\AgdaSpace{}%
\AgdaInductiveConstructor{sum}\AgdaSpace{}%
\AgdaSymbol{(}\AgdaBound{f}\AgdaSpace{}%
\AgdaSymbol{λ}\AgdaSpace{}%
\AgdaSymbol{\{}\AgdaBound{Δ}\AgdaSymbol{\}}\AgdaSpace{}%
\AgdaSymbol{⦃}\AgdaSpace{}%
\AgdaBound{p}\AgdaSpace{}%
\AgdaSymbol{⦄}\AgdaSpace{}%
\AgdaSymbol{→}\AgdaSpace{}%
\AgdaInductiveConstructor{var}\AgdaSpace{}%
\AgdaOperator{\AgdaFunction{⟨}}\AgdaSpace{}%
\AgdaInductiveConstructor{v₀}\AgdaSpace{}%
\AgdaOperator{\AgdaFunction{⟩ᵛ}}\AgdaSymbol{)}\<%
\\
%
\\[\AgdaEmptyExtraSkip]%
%
\>[2]\AgdaFunction{Imaps}%
\>[1700I]\AgdaSymbol{:}\AgdaSpace{}%
\AgdaSymbol{∀}\AgdaSpace{}%
\AgdaSymbol{\{}\AgdaBound{Γ}\AgdaSymbol{\}}\<%
\\
\>[.][@{}l@{}]\<[1700I]%
\>[8]\AgdaSymbol{→}\AgdaSpace{}%
\AgdaSymbol{(}\AgdaFunction{GE}\AgdaSpace{}%
\AgdaSymbol{(}\AgdaBound{Γ}\AgdaSpace{}%
\AgdaOperator{\AgdaInductiveConstructor{▹}}\AgdaSpace{}%
\AgdaInductiveConstructor{ix}\AgdaSpace{}%
\AgdaGeneralizable{s}\AgdaSymbol{)}\AgdaSpace{}%
\AgdaSymbol{(}\AgdaInductiveConstructor{ix}\AgdaSpace{}%
\AgdaGeneralizable{s}\AgdaSymbol{)}\AgdaSpace{}%
\AgdaSymbol{→}\AgdaSpace{}%
\AgdaDatatype{E}\AgdaSpace{}%
\AgdaSymbol{(}\AgdaBound{Γ}\AgdaSpace{}%
\AgdaOperator{\AgdaInductiveConstructor{▹}}\AgdaSpace{}%
\AgdaInductiveConstructor{ix}\AgdaSpace{}%
\AgdaGeneralizable{s}\AgdaSymbol{)}\AgdaSpace{}%
\AgdaSymbol{(}\AgdaInductiveConstructor{ar}\AgdaSpace{}%
\AgdaFunction{unit}\AgdaSymbol{))}\<%
\\
%
\>[8]\AgdaSymbol{→}\AgdaSpace{}%
\AgdaDatatype{E}\AgdaSpace{}%
\AgdaBound{Γ}\AgdaSpace{}%
\AgdaSymbol{(}\AgdaInductiveConstructor{ar}\AgdaSpace{}%
\AgdaGeneralizable{s}\AgdaSymbol{)}\<%
\\
%
\>[2]\AgdaFunction{Imaps}\AgdaSpace{}%
\AgdaBound{f}\AgdaSpace{}%
\AgdaSymbol{=}\AgdaSpace{}%
\AgdaInductiveConstructor{imaps}\AgdaSpace{}%
\AgdaSymbol{(}\AgdaBound{f}\AgdaSpace{}%
\AgdaSymbol{λ}\AgdaSpace{}%
\AgdaSymbol{\{}\AgdaBound{Δ}\AgdaSymbol{\}}\AgdaSpace{}%
\AgdaSymbol{⦃}\AgdaSpace{}%
\AgdaBound{p}\AgdaSpace{}%
\AgdaSymbol{⦄}\AgdaSpace{}%
\AgdaSymbol{→}\AgdaSpace{}%
\AgdaInductiveConstructor{var}\AgdaSpace{}%
\AgdaOperator{\AgdaFunction{⟨}}\AgdaSpace{}%
\AgdaInductiveConstructor{v₀}\AgdaSpace{}%
\AgdaOperator{\AgdaFunction{⟩ᵛ}}\AgdaSymbol{)}\<%
\\
%
\\[\AgdaEmptyExtraSkip]%
%
\>[2]\AgdaFunction{Imapb}%
\>[1736I]\AgdaSymbol{:}\AgdaSpace{}%
\AgdaSymbol{∀}\AgdaSpace{}%
\AgdaSymbol{\{}\AgdaBound{Γ}\AgdaSymbol{\}}\<%
\\
\>[.][@{}l@{}]\<[1736I]%
\>[8]\AgdaSymbol{→}\AgdaSpace{}%
\AgdaGeneralizable{s}\AgdaSpace{}%
\AgdaOperator{\AgdaFunction{*}}\AgdaSpace{}%
\AgdaGeneralizable{p}\AgdaSpace{}%
\AgdaOperator{\AgdaFunction{≈}}\AgdaSpace{}%
\AgdaGeneralizable{q}\<%
\\
%
\>[8]\AgdaSymbol{→}\AgdaSpace{}%
\AgdaSymbol{(}\AgdaFunction{GE}\AgdaSpace{}%
\AgdaSymbol{(}\AgdaBound{Γ}\AgdaSpace{}%
\AgdaOperator{\AgdaInductiveConstructor{▹}}\AgdaSpace{}%
\AgdaInductiveConstructor{ix}\AgdaSpace{}%
\AgdaGeneralizable{s}\AgdaSymbol{)}\AgdaSpace{}%
\AgdaSymbol{(}\AgdaInductiveConstructor{ix}\AgdaSpace{}%
\AgdaGeneralizable{s}\AgdaSymbol{)}\AgdaSpace{}%
\AgdaSymbol{→}\AgdaSpace{}%
\AgdaDatatype{E}\AgdaSpace{}%
\AgdaSymbol{(}\AgdaBound{Γ}\AgdaSpace{}%
\AgdaOperator{\AgdaInductiveConstructor{▹}}\AgdaSpace{}%
\AgdaInductiveConstructor{ix}\AgdaSpace{}%
\AgdaGeneralizable{s}\AgdaSymbol{)}\AgdaSpace{}%
\AgdaSymbol{(}\AgdaInductiveConstructor{ar}\AgdaSpace{}%
\AgdaGeneralizable{p}\AgdaSymbol{))}\<%
\\
%
\>[8]\AgdaSymbol{→}\AgdaSpace{}%
\AgdaDatatype{E}\AgdaSpace{}%
\AgdaBound{Γ}\AgdaSpace{}%
\AgdaSymbol{(}\AgdaInductiveConstructor{ar}\AgdaSpace{}%
\AgdaGeneralizable{q}\AgdaSymbol{)}\<%
\\
%
\>[2]\AgdaFunction{Imapb}\AgdaSpace{}%
\AgdaBound{p}\AgdaSpace{}%
\AgdaBound{f}\AgdaSpace{}%
\AgdaSymbol{=}\AgdaSpace{}%
\AgdaInductiveConstructor{imapb}\AgdaSpace{}%
\AgdaBound{p}\AgdaSpace{}%
\AgdaSymbol{(}\AgdaBound{f}\AgdaSpace{}%
\AgdaSymbol{λ}\AgdaSpace{}%
\AgdaSymbol{\{}\AgdaBound{Δ}\AgdaSymbol{\}}\AgdaSpace{}%
\AgdaSymbol{⦃}\AgdaSpace{}%
\AgdaBound{p}\AgdaSpace{}%
\AgdaSymbol{⦄}\AgdaSpace{}%
\AgdaSymbol{→}\AgdaSpace{}%
\AgdaInductiveConstructor{var}\AgdaSpace{}%
\AgdaOperator{\AgdaFunction{⟨}}\AgdaSpace{}%
\AgdaInductiveConstructor{v₀}\AgdaSpace{}%
\AgdaOperator{\AgdaFunction{⟩ᵛ}}\AgdaSymbol{)}\<%
\end{code}
\begin{code}%
%
\>[2]\AgdaFunction{Let-syntax}\AgdaSpace{}%
\AgdaSymbol{:}\AgdaSpace{}%
\AgdaDatatype{E}\AgdaSpace{}%
\AgdaGeneralizable{Γ}\AgdaSpace{}%
\AgdaSymbol{(}\AgdaInductiveConstructor{ar}\AgdaSpace{}%
\AgdaGeneralizable{s}\AgdaSymbol{)}\AgdaSpace{}%
\AgdaSymbol{→}\AgdaSpace{}%
\AgdaSymbol{(}\AgdaFunction{GE}\AgdaSpace{}%
\AgdaSymbol{(}\AgdaGeneralizable{Γ}\AgdaSpace{}%
\AgdaOperator{\AgdaInductiveConstructor{▹}}\AgdaSpace{}%
\AgdaSymbol{(}\AgdaInductiveConstructor{ar}\AgdaSpace{}%
\AgdaGeneralizable{s}\AgdaSymbol{))}\AgdaSpace{}%
\AgdaSymbol{(}\AgdaInductiveConstructor{ar}\AgdaSpace{}%
\AgdaGeneralizable{s}\AgdaSymbol{)}\AgdaSpace{}%
\AgdaSymbol{→}\AgdaSpace{}%
\AgdaDatatype{E}\AgdaSpace{}%
\AgdaSymbol{(}\AgdaGeneralizable{Γ}\AgdaSpace{}%
\AgdaOperator{\AgdaInductiveConstructor{▹}}\AgdaSpace{}%
\AgdaSymbol{(}\AgdaInductiveConstructor{ar}\AgdaSpace{}%
\AgdaGeneralizable{s}\AgdaSymbol{))}\AgdaSpace{}%
\AgdaSymbol{(}\AgdaInductiveConstructor{ar}\AgdaSpace{}%
\AgdaGeneralizable{p}\AgdaSymbol{))}\AgdaSpace{}%
\AgdaSymbol{→}\AgdaSpace{}%
\AgdaDatatype{E}\AgdaSpace{}%
\AgdaGeneralizable{Γ}\AgdaSpace{}%
\AgdaSymbol{(}\AgdaInductiveConstructor{ar}\AgdaSpace{}%
\AgdaGeneralizable{p}\AgdaSymbol{)}\<%
\\
%
\>[2]\AgdaFunction{Let-syntax}\AgdaSpace{}%
\AgdaBound{x}\AgdaSpace{}%
\AgdaBound{f}\AgdaSpace{}%
\AgdaSymbol{=}\AgdaSpace{}%
\AgdaInductiveConstructor{let′}\AgdaSpace{}%
\AgdaBound{x}\AgdaSpace{}%
\AgdaSymbol{(}\AgdaBound{f}\AgdaSpace{}%
\AgdaSymbol{(}\AgdaInductiveConstructor{var}\AgdaSpace{}%
\AgdaOperator{\AgdaFunction{⟨}}\AgdaSpace{}%
\AgdaInductiveConstructor{v₀}\AgdaSpace{}%
\AgdaOperator{\AgdaFunction{⟩ᵛ}}\AgdaSymbol{))}\<%
\\
%
\>[2]\AgdaKeyword{syntax}\AgdaSpace{}%
\AgdaFunction{Let-syntax}\AgdaSpace{}%
\AgdaBound{e}\AgdaSpace{}%
\AgdaSymbol{(λ}\AgdaSpace{}%
\AgdaBound{x}\AgdaSpace{}%
\AgdaSymbol{→}\AgdaSpace{}%
\AgdaBound{e'}\AgdaSymbol{)}\AgdaSpace{}%
\AgdaSymbol{=}\AgdaSpace{}%
\AgdaFunction{Let}\AgdaSpace{}%
\AgdaBound{x}\AgdaSpace{}%
\AgdaFunction{:=}\AgdaSpace{}%
\AgdaBound{e}\AgdaSpace{}%
\AgdaFunction{In}\AgdaSpace{}%
\AgdaBound{e'}\<%
\end{code}
With these definitions we can write expressions with let binding as follows:
\begin{code}%
%
\>[2]\AgdaFunction{\AgdaUnderscore{}}\AgdaSpace{}%
\AgdaSymbol{:}\AgdaSpace{}%
\AgdaDatatype{E}\AgdaSpace{}%
\AgdaInductiveConstructor{ε}\AgdaSpace{}%
\AgdaSymbol{(}\AgdaInductiveConstructor{ar}\AgdaSpace{}%
\AgdaInductiveConstructor{[]}\AgdaSymbol{)}\<%
\\
%
\>[2]\AgdaSymbol{\AgdaUnderscore{}}\AgdaSpace{}%
\AgdaSymbol{=}\AgdaSpace{}%
\AgdaFunction{Let}\AgdaSpace{}%
\AgdaBound{x}\AgdaSpace{}%
\AgdaFunction{:=}\AgdaSpace{}%
\AgdaInductiveConstructor{one}\AgdaSpace{}%
\AgdaFunction{In}\AgdaSpace{}%
\AgdaFunction{Let}\AgdaSpace{}%
\AgdaBound{y}\AgdaSpace{}%
\AgdaFunction{:=}\AgdaSpace{}%
\AgdaBound{x}\AgdaSpace{}%
\AgdaOperator{\AgdaInductiveConstructor{⊞}}\AgdaSpace{}%
\AgdaInductiveConstructor{one}\AgdaSpace{}%
\AgdaFunction{In}\AgdaSpace{}%
\AgdaSymbol{(}\AgdaBound{x}\AgdaSpace{}%
\AgdaOperator{\AgdaInductiveConstructor{⊞}}\AgdaSpace{}%
\AgdaBound{y}\AgdaSymbol{)}\AgdaSpace{}%
\AgdaOperator{\AgdaInductiveConstructor{⊠}}\AgdaSpace{}%
\AgdaBound{x}\<%
\end{code}
One final syntactical convenience is the ability to
represent contexts in the HOAS style.
First of all, we define a helper function \AF{ext} that appends a list of \AF{IS}
at the end of some context \AF{Γ}.
Secondly, we define \AF{lfun} that for the given list of \AF{IS}-es
(l = [is₁, \dots, isₙ]), some context \AF{Γ} and some \AF{IS}
ip computes an Agda function of type (\AF{GE} (\AF{ext} \AF{Γ} l) is₁ →
\dots → \AF{GE} (\AF{ext} \AF{Γ} l) isₙ → \AD{E} (\AF{ext} Γ l) ip).
The function \AF{lvar} lifts a variable in some context Γ into
a generalised expression within the \AF{ext}ended context.
\begin{mathpar}
\codeblock{\begin{code}[hide]%
%
\>[2]\AgdaKeyword{infixl}\AgdaSpace{}%
\AgdaNumber{3}\AgdaSpace{}%
\AgdaFunction{Let-syntax}\<%
\\
%
\\[\AgdaEmptyExtraSkip]%
%
\>[2]\AgdaComment{--\ Extend\ context\ with\ a\ list\ of\ types}\<%
\\
%
\>[2]\AgdaComment{--\ (List\ is\ a\ context\ that\ grows\ to\ the\ left)}\<%
\end{code}
\begin{code}%
%
\>[2]\AgdaFunction{ext}\AgdaSpace{}%
\AgdaSymbol{:}\AgdaSpace{}%
\AgdaDatatype{Ctx}\AgdaSpace{}%
\AgdaSymbol{→}\AgdaSpace{}%
\AgdaDatatype{List}\AgdaSpace{}%
\AgdaDatatype{IS}\AgdaSpace{}%
\AgdaSymbol{→}\AgdaSpace{}%
\AgdaDatatype{Ctx}\<%
\\
%
\>[2]\AgdaFunction{ext}\AgdaSpace{}%
\AgdaBound{Γ}\AgdaSpace{}%
\AgdaInductiveConstructor{[]}%
\>[16]\AgdaSymbol{=}\AgdaSpace{}%
\AgdaBound{Γ}\<%
\\
%
\>[2]\AgdaFunction{ext}\AgdaSpace{}%
\AgdaBound{Γ}\AgdaSpace{}%
\AgdaSymbol{(}\AgdaBound{x}\AgdaSpace{}%
\AgdaOperator{\AgdaInductiveConstructor{∷}}\AgdaSpace{}%
\AgdaBound{l}\AgdaSymbol{)}\AgdaSpace{}%
\AgdaSymbol{=}\AgdaSpace{}%
\AgdaFunction{ext}\AgdaSpace{}%
\AgdaSymbol{(}\AgdaBound{Γ}\AgdaSpace{}%
\AgdaOperator{\AgdaInductiveConstructor{▹}}\AgdaSpace{}%
\AgdaBound{x}\AgdaSymbol{)}\AgdaSpace{}%
\AgdaBound{l}\<%
\end{code}}
\and
\codeblock{\begin{code}%
%
\>[2]\AgdaFunction{lfun}\AgdaSpace{}%
\AgdaSymbol{:}\AgdaSpace{}%
\AgdaSymbol{(}\AgdaBound{l}\AgdaSpace{}%
\AgdaSymbol{:}\AgdaSpace{}%
\AgdaDatatype{List}\AgdaSpace{}%
\AgdaDatatype{IS}\AgdaSymbol{)}%
\>[24]\AgdaSymbol{(}\AgdaBound{Γ}\AgdaSpace{}%
\AgdaSymbol{:}\AgdaSpace{}%
\AgdaDatatype{Ctx}\AgdaSymbol{)}\AgdaSpace{}%
\AgdaSymbol{(}\AgdaBound{is}\AgdaSpace{}%
\AgdaSymbol{:}\AgdaSpace{}%
\AgdaDatatype{IS}\AgdaSymbol{)}\AgdaSpace{}%
\AgdaSymbol{→}\AgdaSpace{}%
\AgdaPrimitive{Set}\<%
\\
%
\>[2]\AgdaFunction{lvar}\AgdaSpace{}%
\AgdaSymbol{:}\AgdaSpace{}%
\AgdaSymbol{∀}\AgdaSpace{}%
\AgdaBound{l}\AgdaSpace{}%
\AgdaSymbol{→}\AgdaSpace{}%
\AgdaGeneralizable{is}\AgdaSpace{}%
\AgdaOperator{\AgdaDatatype{∈}}\AgdaSpace{}%
\AgdaGeneralizable{Γ}\AgdaSpace{}%
\AgdaSymbol{→}\AgdaSpace{}%
\AgdaFunction{GE}\AgdaSpace{}%
\AgdaSymbol{(}\AgdaFunction{ext}\AgdaSpace{}%
\AgdaGeneralizable{Γ}\AgdaSpace{}%
\AgdaBound{l}\AgdaSymbol{)}\AgdaSpace{}%
\AgdaGeneralizable{is}\<%
\end{code}}
\end{mathpar}
\begin{code}[hide]%
%
\>[2]\AgdaComment{--\ Turn\ the\ list\ of\ IS\ into\ the\ following\ function:}\<%
\\
%
\>[2]\AgdaComment{--\ \ \ l\ =\ [a,\ b,\ c]}\<%
\\
%
\>[2]\AgdaComment{--\ \ \ X\ =\ X}\<%
\\
%
\>[2]\AgdaComment{--\ \ \ Γ\ =\ Γ}\<%
\\
%
\>[2]\AgdaComment{--\ \ \ ----------------------------}\<%
\\
%
\>[2]\AgdaComment{--\ \ \ GE\ Γ\ a\ →\ GE\ Γ\ b\ →\ GE\ Γ\ c\ →\ X}\<%
\\
%
\>[2]\AgdaFunction{lfunh}\AgdaSpace{}%
\AgdaSymbol{:}\AgdaSpace{}%
\AgdaSymbol{(}\AgdaBound{l}\AgdaSpace{}%
\AgdaSymbol{:}\AgdaSpace{}%
\AgdaDatatype{List}\AgdaSpace{}%
\AgdaDatatype{IS}\AgdaSymbol{)}\AgdaSpace{}%
\AgdaSymbol{(}\AgdaBound{X}\AgdaSpace{}%
\AgdaSymbol{:}\AgdaSpace{}%
\AgdaPrimitive{Set}\AgdaSymbol{)}\AgdaSpace{}%
\AgdaSymbol{(}\AgdaBound{Γ}\AgdaSpace{}%
\AgdaSymbol{:}\AgdaSpace{}%
\AgdaDatatype{Ctx}\AgdaSymbol{)}\AgdaSpace{}%
\AgdaSymbol{→}\AgdaSpace{}%
\AgdaPrimitive{Set}\<%
\\
%
\>[2]\AgdaFunction{lfunh}\AgdaSpace{}%
\AgdaInductiveConstructor{[]}\AgdaSpace{}%
\AgdaBound{X}\AgdaSpace{}%
\AgdaBound{Γ}\AgdaSpace{}%
\AgdaSymbol{=}\AgdaSpace{}%
\AgdaBound{X}\<%
\\
%
\>[2]\AgdaFunction{lfunh}\AgdaSpace{}%
\AgdaSymbol{(}\AgdaBound{a}\AgdaSpace{}%
\AgdaOperator{\AgdaInductiveConstructor{∷}}\AgdaSpace{}%
\AgdaBound{l}\AgdaSymbol{)}\AgdaSpace{}%
\AgdaBound{X}\AgdaSpace{}%
\AgdaBound{Γ}\AgdaSpace{}%
\AgdaSymbol{=}\AgdaSpace{}%
\AgdaFunction{GE}\AgdaSpace{}%
\AgdaBound{Γ}\AgdaSpace{}%
\AgdaBound{a}\AgdaSpace{}%
\AgdaSymbol{→}\AgdaSpace{}%
\AgdaFunction{lfunh}\AgdaSpace{}%
\AgdaBound{l}\AgdaSpace{}%
\AgdaBound{X}\AgdaSpace{}%
\AgdaBound{Γ}\<%
\\
%
\\[\AgdaEmptyExtraSkip]%
%
\>[2]\AgdaComment{--\ Diagonalise\ lfunh:}\<%
\\
%
\>[2]\AgdaComment{--\ \ \ l\ =\ [a,\ b]}\<%
\\
%
\>[2]\AgdaComment{--\ \ \ Γ\ =\ Γ}\<%
\\
%
\>[2]\AgdaComment{--\ \ \ is\ =\ is}\<%
\\
%
\>[2]\AgdaComment{--\ \ \ ---------------------------------------------}\<%
\\
%
\>[2]\AgdaComment{--\ \ \ GE\ (ext\ Γ\ l)\ a\ →\ GE\ (ext\ Γ\ l)\ →\ E\ (ext\ Γ\ l)\ is}\<%
\\
%
\>[2]\AgdaFunction{lfun}\AgdaSpace{}%
\AgdaBound{l}\AgdaSpace{}%
\AgdaBound{Γ}\AgdaSpace{}%
\AgdaBound{τ}\AgdaSpace{}%
\AgdaSymbol{=}\AgdaSpace{}%
\AgdaFunction{lfunh}\AgdaSpace{}%
\AgdaBound{l}\AgdaSpace{}%
\AgdaSymbol{(}\AgdaDatatype{E}\AgdaSpace{}%
\AgdaSymbol{(}\AgdaFunction{ext}\AgdaSpace{}%
\AgdaBound{Γ}\AgdaSpace{}%
\AgdaBound{l}\AgdaSymbol{)}\AgdaSpace{}%
\AgdaBound{τ}\AgdaSymbol{)}\AgdaSpace{}%
\AgdaSymbol{(}\AgdaFunction{ext}\AgdaSpace{}%
\AgdaBound{Γ}\AgdaSpace{}%
\AgdaBound{l}\AgdaSymbol{)}\<%
\\
%
\>[2]\AgdaFunction{lvar}\AgdaSpace{}%
\AgdaInductiveConstructor{[]}\AgdaSpace{}%
\AgdaBound{v}\AgdaSpace{}%
\AgdaSymbol{=}\AgdaSpace{}%
\AgdaInductiveConstructor{var}\AgdaSpace{}%
\AgdaOperator{\AgdaFunction{⟨}}\AgdaSpace{}%
\AgdaBound{v}\AgdaSpace{}%
\AgdaOperator{\AgdaFunction{⟩ᵛ}}\<%
\\
%
\>[2]\AgdaFunction{lvar}\AgdaSpace{}%
\AgdaSymbol{(}\AgdaBound{x}\AgdaSpace{}%
\AgdaOperator{\AgdaInductiveConstructor{∷}}\AgdaSpace{}%
\AgdaBound{l}\AgdaSymbol{)}\AgdaSpace{}%
\AgdaBound{v}\AgdaSpace{}%
\AgdaSymbol{=}\AgdaSpace{}%
\AgdaFunction{lvar}\AgdaSpace{}%
\AgdaBound{l}\AgdaSpace{}%
\AgdaSymbol{(}\AgdaInductiveConstructor{vₛ}\AgdaSpace{}%
\AgdaBound{v}\AgdaSymbol{)}\<%
\end{code}
With these helper functions we define \AF{Lcon} which computes
an expression in the context extended by $l$ from the function
of $l$ arguments:
\begin{code}%
%
\>[2]\AgdaFunction{Lcon}\AgdaSpace{}%
\AgdaSymbol{:}\AgdaSpace{}%
\AgdaSymbol{∀}\AgdaSpace{}%
\AgdaBound{l}\AgdaSpace{}%
\AgdaBound{is}\AgdaSpace{}%
\AgdaBound{Γ}\AgdaSpace{}%
\AgdaSymbol{→}\AgdaSpace{}%
\AgdaSymbol{(}\AgdaBound{f}\AgdaSpace{}%
\AgdaSymbol{:}\AgdaSpace{}%
\AgdaFunction{lfun}\AgdaSpace{}%
\AgdaBound{l}\AgdaSpace{}%
\AgdaBound{Γ}\AgdaSpace{}%
\AgdaBound{is}\AgdaSymbol{)}\AgdaSpace{}%
\AgdaSymbol{→}\AgdaSpace{}%
\AgdaDatatype{E}\AgdaSpace{}%
\AgdaSymbol{(}\AgdaFunction{ext}\AgdaSpace{}%
\AgdaBound{Γ}\AgdaSpace{}%
\AgdaBound{l}\AgdaSymbol{)}\AgdaSpace{}%
\AgdaBound{is}\<%
\\
%
\>[2]\AgdaFunction{Lcon}\AgdaSpace{}%
\AgdaInductiveConstructor{[]}%
\>[15]\AgdaBound{is}\AgdaSpace{}%
\AgdaBound{Γ}\AgdaSpace{}%
\AgdaBound{f}%
\>[23]\AgdaSymbol{=}\AgdaSpace{}%
\AgdaBound{f}\<%
\\
%
\>[2]\AgdaFunction{Lcon}\AgdaSpace{}%
\AgdaSymbol{(}\AgdaBound{x}\AgdaSpace{}%
\AgdaOperator{\AgdaInductiveConstructor{∷}}\AgdaSpace{}%
\AgdaBound{l}\AgdaSymbol{)}\AgdaSpace{}%
\AgdaBound{is}\AgdaSpace{}%
\AgdaBound{Γ}\AgdaSpace{}%
\AgdaBound{f}%
\>[23]\AgdaSymbol{=}\AgdaSpace{}%
\AgdaFunction{Lcon}\AgdaSpace{}%
\AgdaBound{l}\AgdaSpace{}%
\AgdaBound{is}\AgdaSpace{}%
\AgdaSymbol{(}\AgdaBound{Γ}\AgdaSpace{}%
\AgdaOperator{\AgdaInductiveConstructor{▹}}\AgdaSpace{}%
\AgdaBound{x}\AgdaSymbol{)}\AgdaSpace{}%
\AgdaSymbol{(}\AgdaBound{f}\AgdaSpace{}%
\AgdaSymbol{(}\AgdaFunction{lvar}\AgdaSpace{}%
\AgdaBound{l}\AgdaSpace{}%
\AgdaInductiveConstructor{v₀}\AgdaSymbol{))}\<%
\end{code}
In practice, this allows one to bind the last $n$ elements of the
context to Agda variables and use them under binders without weakening.
For example, consider the expression:
\begin{code}%
%
\>[2]\AgdaFunction{\AgdaUnderscore{}}\AgdaSpace{}%
\AgdaSymbol{:}\AgdaSpace{}%
\AgdaDatatype{E}\AgdaSpace{}%
\AgdaSymbol{\AgdaUnderscore{}}\AgdaSpace{}%
\AgdaSymbol{\AgdaUnderscore{}}\<%
\\
%
\>[2]\AgdaSymbol{\AgdaUnderscore{}}\AgdaSpace{}%
\AgdaSymbol{=}%
\>[2003I]\AgdaFunction{Lcon}\AgdaSpace{}%
\AgdaSymbol{(}\AgdaInductiveConstructor{ar}\AgdaSpace{}%
\AgdaSymbol{(}\AgdaInductiveConstructor{ι}\AgdaSpace{}%
\AgdaNumber{5}\AgdaSymbol{)}\AgdaSpace{}%
\AgdaOperator{\AgdaInductiveConstructor{∷}}\AgdaSpace{}%
\AgdaInductiveConstructor{ar}\AgdaSpace{}%
\AgdaSymbol{(}\AgdaNumber{5}\AgdaSpace{}%
\AgdaOperator{\AgdaInductiveConstructor{∷}}\AgdaSpace{}%
\AgdaNumber{5}\AgdaSpace{}%
\AgdaOperator{\AgdaInductiveConstructor{∷}}\AgdaSpace{}%
\AgdaInductiveConstructor{[]}\AgdaSymbol{)}\AgdaSpace{}%
\AgdaOperator{\AgdaInductiveConstructor{∷}}\AgdaSpace{}%
\AgdaInductiveConstructor{[]}\AgdaSymbol{)}\AgdaSpace{}%
\AgdaSymbol{(}\AgdaInductiveConstructor{ar}\AgdaSpace{}%
\AgdaInductiveConstructor{[]}\AgdaSymbol{)}\AgdaSpace{}%
\AgdaInductiveConstructor{ε}\<%
\\
\>[.][@{}l@{}]\<[2003I]%
\>[6]\AgdaSymbol{λ}\AgdaSpace{}%
\AgdaBound{a}\AgdaSpace{}%
\AgdaBound{b}\AgdaSpace{}%
\AgdaSymbol{→}\AgdaSpace{}%
\AgdaFunction{Sum}\AgdaSpace{}%
\AgdaSymbol{λ}\AgdaSpace{}%
\AgdaBound{i}\AgdaSpace{}%
\AgdaSymbol{→}\AgdaSpace{}%
\AgdaInductiveConstructor{sels}\AgdaSpace{}%
\AgdaBound{a}\AgdaSpace{}%
\AgdaBound{i}\AgdaSpace{}%
\AgdaOperator{\AgdaInductiveConstructor{⊞}}\AgdaSpace{}%
\AgdaInductiveConstructor{sels}\AgdaSpace{}%
\AgdaSymbol{(}\AgdaInductiveConstructor{sel}\AgdaSpace{}%
\AgdaBound{b}\AgdaSpace{}%
\AgdaBound{i}\AgdaSymbol{)}\AgdaSpace{}%
\AgdaBound{i}\<%
\end{code}
where \AB{a} and \AB{b} are Agda's variables that represent
de Bruijn variable 1 and 0{} in the context (ε ▹ 5 ∷ [] ▹ 5 ∷ 5 ∷ []).



\subsection{CNN Primitives in \AD{E}}
The built-in operations of \AD{E} are not specific to the CNN that we
are defining.  Therefore, similarly to Sec.~\ref{sec:ar-cnn-prim},
the primitives required for the running example have to be
implemented in terms of \AD{E}.  Syntactic wrappers help us
to achieve similarity between the operations defined below and the
\AF{Ar} primitives.

Implementation of \AF{conv}, \AF{mconv} and \AF{avgp₂} is a direct
translation of the respective \AD{Ar} counterparts, the only visible
differences are: lack of \AF{map} combinator and
rank-polymorphic addition and multiplication.
\begin{code}[hide]%
\>[0]\AgdaKeyword{module}\AgdaSpace{}%
\AgdaModule{Primitives}\AgdaSpace{}%
\AgdaKeyword{where}\<%
\\
%
\\[\AgdaEmptyExtraSkip]%
\>[0][@{}l@{\AgdaIndent{0}}]%
\>[2]\AgdaKeyword{open}\AgdaSpace{}%
\AgdaKeyword{import}\AgdaSpace{}%
\AgdaModule{Data.List}\AgdaSpace{}%
\AgdaSymbol{as}\AgdaSpace{}%
\AgdaModule{L}\AgdaSpace{}%
\AgdaKeyword{using}\AgdaSpace{}%
\AgdaSymbol{(}\AgdaDatatype{List}\AgdaSymbol{;}\AgdaSpace{}%
\AgdaInductiveConstructor{[]}\AgdaSymbol{;}\AgdaSpace{}%
\AgdaOperator{\AgdaInductiveConstructor{\AgdaUnderscore{}∷\AgdaUnderscore{}}}\AgdaSymbol{)}\<%
\\
%
\>[2]\AgdaKeyword{open}\AgdaSpace{}%
\AgdaKeyword{import}\AgdaSpace{}%
\AgdaModule{Data.Nat}\AgdaSpace{}%
\AgdaSymbol{as}\AgdaSpace{}%
\AgdaModule{ℕ}\AgdaSpace{}%
\AgdaKeyword{using}\AgdaSpace{}%
\AgdaSymbol{(}\AgdaDatatype{ℕ}\AgdaSymbol{;}\AgdaSpace{}%
\AgdaInductiveConstructor{zero}\AgdaSymbol{;}\AgdaSpace{}%
\AgdaInductiveConstructor{suc}\AgdaSymbol{)}\<%
\\
%
\>[2]\AgdaKeyword{open}\AgdaSpace{}%
\AgdaKeyword{import}\AgdaSpace{}%
\AgdaModule{Function}\AgdaSpace{}%
\AgdaKeyword{using}\AgdaSpace{}%
\AgdaSymbol{(}\AgdaOperator{\AgdaFunction{\AgdaUnderscore{}\$\AgdaUnderscore{}}}\AgdaSymbol{;}\AgdaSpace{}%
\AgdaFunction{it}\AgdaSymbol{;}\AgdaSpace{}%
\AgdaOperator{\AgdaFunction{\AgdaUnderscore{}∋\AgdaUnderscore{}}}\AgdaSymbol{)}\<%
\\
%
\>[2]\AgdaKeyword{open}\AgdaSpace{}%
\AgdaKeyword{import}\AgdaSpace{}%
\AgdaModule{Relation.Binary.PropositionalEquality}\<%
\\
%
\>[2]\AgdaKeyword{open}\AgdaSpace{}%
\AgdaModule{Array}\AgdaSpace{}%
\AgdaKeyword{hiding}\AgdaSpace{}%
\AgdaSymbol{(}\AgdaFunction{slide}\AgdaSymbol{)}\<%
\\
%
\>[2]\AgdaKeyword{open}\AgdaSpace{}%
\AgdaModule{Syntax}\<%
\\
%
\>[2]\AgdaKeyword{open}\AgdaSpace{}%
\AgdaModule{WkSub}\<%
\\
%
\>[2]\AgdaKeyword{open}\AgdaSpace{}%
\AgdaModule{Lang}\<%
\\
%
\\[\AgdaEmptyExtraSkip]%
\>[0]\<%
\end{code}
\begin{code}%
\>[0][@{}l@{\AgdaIndent{1}}]%
\>[2]\AgdaFunction{conv}\AgdaSpace{}%
\AgdaSymbol{:}\AgdaSpace{}%
\AgdaDatatype{E}\AgdaSpace{}%
\AgdaGeneralizable{Γ}\AgdaSpace{}%
\AgdaSymbol{(}\AgdaInductiveConstructor{ar}\AgdaSpace{}%
\AgdaGeneralizable{r}\AgdaSymbol{)}\AgdaSpace{}%
\AgdaSymbol{→}\AgdaSpace{}%
\AgdaSymbol{⦃}\AgdaSpace{}%
\AgdaGeneralizable{s}\AgdaSpace{}%
\AgdaOperator{\AgdaFunction{+}}\AgdaSpace{}%
\AgdaGeneralizable{p}\AgdaSpace{}%
\AgdaOperator{\AgdaFunction{≈}}\AgdaSpace{}%
\AgdaGeneralizable{r}\AgdaSpace{}%
\AgdaSymbol{⦄}\AgdaSpace{}%
\AgdaSymbol{→}\AgdaSpace{}%
\AgdaDatatype{E}\AgdaSpace{}%
\AgdaGeneralizable{Γ}\AgdaSpace{}%
\AgdaSymbol{(}\AgdaInductiveConstructor{ar}\AgdaSpace{}%
\AgdaGeneralizable{s}\AgdaSymbol{)}\AgdaSpace{}%
\AgdaSymbol{→}\AgdaSpace{}%
\AgdaSymbol{⦃}\AgdaSpace{}%
\AgdaOperator{\AgdaFunction{suc}}\AgdaSpace{}%
\AgdaGeneralizable{p}\AgdaSpace{}%
\AgdaOperator{\AgdaFunction{≈}}\AgdaSpace{}%
\AgdaGeneralizable{u}\AgdaSpace{}%
\AgdaSymbol{⦄}\AgdaSpace{}%
\AgdaSymbol{→}\AgdaSpace{}%
\AgdaDatatype{E}\AgdaSpace{}%
\AgdaGeneralizable{Γ}\AgdaSpace{}%
\AgdaSymbol{(}\AgdaInductiveConstructor{ar}\AgdaSpace{}%
\AgdaGeneralizable{u}\AgdaSymbol{)}\<%
\\
%
\>[2]\AgdaFunction{conv}\AgdaSpace{}%
\AgdaBound{f}\AgdaSpace{}%
\AgdaSymbol{⦃}\AgdaSpace{}%
\AgdaBound{s+p}\AgdaSpace{}%
\AgdaSymbol{⦄}\AgdaSpace{}%
\AgdaBound{g}\AgdaSpace{}%
\AgdaSymbol{⦃}\AgdaSpace{}%
\AgdaBound{ss}\AgdaSpace{}%
\AgdaSymbol{⦄}\AgdaSpace{}%
\AgdaSymbol{=}\AgdaSpace{}%
\AgdaFunction{Sum}\AgdaSpace{}%
\AgdaSymbol{λ}\AgdaSpace{}%
\AgdaBound{i}\AgdaSpace{}%
\AgdaSymbol{→}\AgdaSpace{}%
\AgdaSymbol{(}\AgdaInductiveConstructor{slide}\AgdaSpace{}%
\AgdaBound{i}\AgdaSpace{}%
\AgdaBound{s+p}\AgdaSpace{}%
\AgdaOperator{\AgdaFunction{⟨}}\AgdaSpace{}%
\AgdaBound{f}\AgdaSpace{}%
\AgdaOperator{\AgdaFunction{⟩}}\AgdaSpace{}%
\AgdaBound{ss}\AgdaSymbol{)}\AgdaSpace{}%
\AgdaOperator{\AgdaInductiveConstructor{⊠}}\AgdaSpace{}%
\AgdaFunction{Imaps}\AgdaSpace{}%
\AgdaSymbol{λ}\AgdaSpace{}%
\AgdaBound{j}\AgdaSpace{}%
\AgdaSymbol{→}\AgdaSpace{}%
\AgdaInductiveConstructor{sels}\AgdaSpace{}%
\AgdaOperator{\AgdaFunction{⟨}}\AgdaSpace{}%
\AgdaBound{g}\AgdaSpace{}%
\AgdaOperator{\AgdaFunction{⟩}}\AgdaSpace{}%
\AgdaBound{i}\<%
\\
%
\\[\AgdaEmptyExtraSkip]%
%
\>[2]\AgdaFunction{mconv}\AgdaSpace{}%
\AgdaSymbol{:}%
\>[2128I]\AgdaSymbol{⦃}\AgdaSpace{}%
\AgdaGeneralizable{s}\AgdaSpace{}%
\AgdaOperator{\AgdaFunction{+}}\AgdaSpace{}%
\AgdaGeneralizable{p}\AgdaSpace{}%
\AgdaOperator{\AgdaFunction{≈}}\AgdaSpace{}%
\AgdaGeneralizable{r}\AgdaSpace{}%
\AgdaSymbol{⦄}\AgdaSpace{}%
\AgdaSymbol{→}\AgdaSpace{}%
\AgdaSymbol{(}\AgdaBound{inp}\AgdaSpace{}%
\AgdaSymbol{:}\AgdaSpace{}%
\AgdaDatatype{E}\AgdaSpace{}%
\AgdaGeneralizable{Γ}\AgdaSpace{}%
\AgdaSymbol{(}\AgdaInductiveConstructor{ar}\AgdaSpace{}%
\AgdaGeneralizable{r}\AgdaSymbol{))}\AgdaSpace{}%
\AgdaSymbol{(}\AgdaBound{ws}\AgdaSpace{}%
\AgdaSymbol{:}\AgdaSpace{}%
\AgdaDatatype{E}\AgdaSpace{}%
\AgdaGeneralizable{Γ}\AgdaSpace{}%
\AgdaSymbol{(}\AgdaInductiveConstructor{ar}\AgdaSpace{}%
\AgdaSymbol{(}\AgdaGeneralizable{u}\AgdaSpace{}%
\AgdaOperator{\AgdaFunction{⊗}}\AgdaSpace{}%
\AgdaGeneralizable{s}\AgdaSymbol{)))}\<%
\\
\>[.][@{}l@{}]\<[2128I]%
\>[10]\AgdaSymbol{(}\AgdaBound{bᵥ}\AgdaSpace{}%
\AgdaSymbol{:}\AgdaSpace{}%
\AgdaDatatype{E}\AgdaSpace{}%
\AgdaGeneralizable{Γ}\AgdaSpace{}%
\AgdaSymbol{(}\AgdaInductiveConstructor{ar}\AgdaSpace{}%
\AgdaGeneralizable{u}\AgdaSymbol{))}\AgdaSpace{}%
\AgdaSymbol{→}\AgdaSpace{}%
\AgdaSymbol{⦃}\AgdaSpace{}%
\AgdaOperator{\AgdaFunction{suc}}\AgdaSpace{}%
\AgdaGeneralizable{p}\AgdaSpace{}%
\AgdaOperator{\AgdaFunction{≈}}\AgdaSpace{}%
\AgdaGeneralizable{w}\AgdaSpace{}%
\AgdaSymbol{⦄}\AgdaSpace{}%
\AgdaSymbol{→}\AgdaSpace{}%
\AgdaDatatype{E}\AgdaSpace{}%
\AgdaGeneralizable{Γ}\AgdaSpace{}%
\AgdaSymbol{(}\AgdaInductiveConstructor{ar}\AgdaSpace{}%
\AgdaSymbol{(}\AgdaGeneralizable{u}\AgdaSpace{}%
\AgdaOperator{\AgdaFunction{⊗}}\AgdaSpace{}%
\AgdaGeneralizable{w}\AgdaSymbol{))}\<%
\\
%
\>[2]\AgdaFunction{mconv}\AgdaSpace{}%
\AgdaSymbol{⦃}\AgdaSpace{}%
\AgdaBound{sp}\AgdaSpace{}%
\AgdaSymbol{⦄}\AgdaSpace{}%
\AgdaBound{inp}\AgdaSpace{}%
\AgdaBound{wᵥ}\AgdaSpace{}%
\AgdaBound{bᵥ}\AgdaSpace{}%
\AgdaSymbol{⦃}\AgdaSpace{}%
\AgdaBound{su}\AgdaSpace{}%
\AgdaSymbol{⦄}\AgdaSpace{}%
\AgdaSymbol{=}\AgdaSpace{}%
\AgdaFunction{Imap}\AgdaSpace{}%
\AgdaSymbol{λ}\AgdaSpace{}%
\AgdaBound{i}\AgdaSpace{}%
\AgdaSymbol{→}\AgdaSpace{}%
\AgdaFunction{conv}\AgdaSpace{}%
\AgdaOperator{\AgdaFunction{⟨}}\AgdaSpace{}%
\AgdaBound{inp}\AgdaSpace{}%
\AgdaOperator{\AgdaFunction{⟩}}\AgdaSpace{}%
\AgdaSymbol{(}\AgdaInductiveConstructor{sel}\AgdaSpace{}%
\AgdaOperator{\AgdaFunction{⟨}}\AgdaSpace{}%
\AgdaBound{wᵥ}\AgdaSpace{}%
\AgdaOperator{\AgdaFunction{⟩}}\AgdaSpace{}%
\AgdaBound{i}\AgdaSymbol{)}\AgdaSpace{}%
\AgdaOperator{\AgdaInductiveConstructor{⊞}}\AgdaSpace{}%
\AgdaFunction{Imaps}\AgdaSpace{}%
\AgdaSymbol{λ}\AgdaSpace{}%
\AgdaBound{\AgdaUnderscore{}}\AgdaSpace{}%
\AgdaSymbol{→}\AgdaSpace{}%
\AgdaInductiveConstructor{sels}\AgdaSpace{}%
\AgdaOperator{\AgdaFunction{⟨}}\AgdaSpace{}%
\AgdaBound{bᵥ}\AgdaSpace{}%
\AgdaOperator{\AgdaFunction{⟩}}\AgdaSpace{}%
\AgdaBound{i}\<%
\\
%
\\[\AgdaEmptyExtraSkip]%
%
\>[2]\AgdaFunction{avgp₂}\AgdaSpace{}%
\AgdaSymbol{:}\AgdaSpace{}%
\AgdaSymbol{∀}\AgdaSpace{}%
\AgdaBound{m}\AgdaSpace{}%
\AgdaBound{n}\AgdaSpace{}%
\AgdaSymbol{→}\AgdaSpace{}%
\AgdaSymbol{(}\AgdaBound{a}\AgdaSpace{}%
\AgdaSymbol{:}\AgdaSpace{}%
\AgdaDatatype{E}\AgdaSpace{}%
\AgdaGeneralizable{Γ}\AgdaSpace{}%
\AgdaSymbol{(}\AgdaInductiveConstructor{ar}\AgdaSpace{}%
\AgdaSymbol{(}\AgdaBound{m}\AgdaSpace{}%
\AgdaOperator{\AgdaPrimitive{ℕ.*}}\AgdaSpace{}%
\AgdaNumber{2}\AgdaSpace{}%
\AgdaOperator{\AgdaInductiveConstructor{∷}}\AgdaSpace{}%
\AgdaBound{n}\AgdaSpace{}%
\AgdaOperator{\AgdaPrimitive{ℕ.*}}\AgdaSpace{}%
\AgdaNumber{2}\AgdaSpace{}%
\AgdaOperator{\AgdaInductiveConstructor{∷}}\AgdaSpace{}%
\AgdaInductiveConstructor{[]}\AgdaSymbol{)))}\AgdaSpace{}%
\AgdaSymbol{→}\AgdaSpace{}%
\AgdaDatatype{E}\AgdaSpace{}%
\AgdaGeneralizable{Γ}\AgdaSpace{}%
\AgdaSymbol{(}\AgdaInductiveConstructor{ar}\AgdaSpace{}%
\AgdaSymbol{(}\AgdaBound{m}\AgdaSpace{}%
\AgdaOperator{\AgdaInductiveConstructor{∷}}\AgdaSpace{}%
\AgdaBound{n}\AgdaSpace{}%
\AgdaOperator{\AgdaInductiveConstructor{∷}}\AgdaSpace{}%
\AgdaInductiveConstructor{[]}\AgdaSymbol{))}\<%
\\
%
\>[2]\AgdaFunction{avgp₂}\AgdaSpace{}%
\AgdaBound{m}\AgdaSpace{}%
\AgdaBound{n}\AgdaSpace{}%
\AgdaBound{a}\AgdaSpace{}%
\AgdaSymbol{=}\AgdaSpace{}%
\AgdaFunction{Imaps}\AgdaSpace{}%
\AgdaSymbol{λ}\AgdaSpace{}%
\AgdaBound{i}\AgdaSpace{}%
\AgdaSymbol{→}\AgdaSpace{}%
\AgdaInductiveConstructor{scaledown}\AgdaSpace{}%
\AgdaNumber{4}\AgdaSpace{}%
\AgdaOperator{\AgdaFunction{\$}}\AgdaSpace{}%
\AgdaFunction{Sum}\AgdaSpace{}%
\AgdaSymbol{λ}\AgdaSpace{}%
\AgdaBound{j}\AgdaSpace{}%
\AgdaSymbol{→}\AgdaSpace{}%
\AgdaInductiveConstructor{sels}\AgdaSpace{}%
\AgdaSymbol{(}\AgdaInductiveConstructor{selb}\AgdaSpace{}%
\AgdaFunction{it}\AgdaSpace{}%
\AgdaOperator{\AgdaFunction{⟨}}\AgdaSpace{}%
\AgdaBound{a}\AgdaSpace{}%
\AgdaOperator{\AgdaFunction{⟩}}\AgdaSpace{}%
\AgdaBound{i}\AgdaSymbol{)}\AgdaSpace{}%
\AgdaBound{j}\<%
\end{code}
The mean squared error function \AF{meansqerr} computes
$\sum_i \frac{1}{2}(r_i - o_i)^2$ for the argument arrays $r$ and $o$
which must be of the same shape.
\begin{code}%
%
\>[2]\AgdaFunction{sqerr}\AgdaSpace{}%
\AgdaSymbol{:}\AgdaSpace{}%
\AgdaSymbol{(}\AgdaBound{r}\AgdaSpace{}%
\AgdaBound{o}\AgdaSpace{}%
\AgdaSymbol{:}\AgdaSpace{}%
\AgdaDatatype{E}\AgdaSpace{}%
\AgdaGeneralizable{Γ}\AgdaSpace{}%
\AgdaSymbol{(}\AgdaInductiveConstructor{ar}\AgdaSpace{}%
\AgdaInductiveConstructor{[]}\AgdaSymbol{))}\AgdaSpace{}%
\AgdaSymbol{→}\AgdaSpace{}%
\AgdaDatatype{E}\AgdaSpace{}%
\AgdaGeneralizable{Γ}\AgdaSpace{}%
\AgdaSymbol{(}\AgdaInductiveConstructor{ar}\AgdaSpace{}%
\AgdaInductiveConstructor{[]}\AgdaSymbol{)}\<%
\\
%
\>[2]\AgdaFunction{sqerr}\AgdaSpace{}%
\AgdaBound{r}\AgdaSpace{}%
\AgdaBound{o}\AgdaSpace{}%
\AgdaSymbol{=}\AgdaSpace{}%
\AgdaInductiveConstructor{scaledown}\AgdaSpace{}%
\AgdaNumber{2}\AgdaSpace{}%
\AgdaSymbol{((}\AgdaBound{r}\AgdaSpace{}%
\AgdaOperator{\AgdaInductiveConstructor{⊞}}\AgdaSpace{}%
\AgdaSymbol{(}\AgdaInductiveConstructor{minus}\AgdaSpace{}%
\AgdaBound{o}\AgdaSymbol{))}\AgdaSpace{}%
\AgdaOperator{\AgdaInductiveConstructor{⊠}}\AgdaSpace{}%
\AgdaSymbol{(}\AgdaBound{r}\AgdaSpace{}%
\AgdaOperator{\AgdaInductiveConstructor{⊞}}\AgdaSpace{}%
\AgdaSymbol{(}\AgdaInductiveConstructor{minus}\AgdaSpace{}%
\AgdaBound{o}\AgdaSymbol{)))}\<%
\\
%
\\[\AgdaEmptyExtraSkip]%
%
\>[2]\AgdaFunction{meansqerr}\AgdaSpace{}%
\AgdaSymbol{:}\AgdaSpace{}%
\AgdaSymbol{(}\AgdaBound{r}\AgdaSpace{}%
\AgdaBound{o}\AgdaSpace{}%
\AgdaSymbol{:}\AgdaSpace{}%
\AgdaDatatype{E}\AgdaSpace{}%
\AgdaGeneralizable{Γ}\AgdaSpace{}%
\AgdaSymbol{(}\AgdaInductiveConstructor{ar}\AgdaSpace{}%
\AgdaGeneralizable{s}\AgdaSymbol{))}\AgdaSpace{}%
\AgdaSymbol{→}\AgdaSpace{}%
\AgdaDatatype{E}\AgdaSpace{}%
\AgdaGeneralizable{Γ}\AgdaSpace{}%
\AgdaSymbol{(}\AgdaInductiveConstructor{ar}\AgdaSpace{}%
\AgdaInductiveConstructor{[]}\AgdaSymbol{)}\<%
\\
%
\>[2]\AgdaFunction{meansqerr}\AgdaSpace{}%
\AgdaBound{r}\AgdaSpace{}%
\AgdaBound{o}\AgdaSpace{}%
\AgdaSymbol{=}\AgdaSpace{}%
\AgdaFunction{Sum}\AgdaSpace{}%
\AgdaSymbol{λ}\AgdaSpace{}%
\AgdaBound{i}\AgdaSpace{}%
\AgdaSymbol{→}\AgdaSpace{}%
\AgdaFunction{sqerr}\AgdaSpace{}%
\AgdaSymbol{(}\AgdaInductiveConstructor{sels}\AgdaSpace{}%
\AgdaOperator{\AgdaFunction{⟨}}\AgdaSpace{}%
\AgdaBound{r}\AgdaSpace{}%
\AgdaOperator{\AgdaFunction{⟩}}\AgdaSpace{}%
\AgdaBound{i}\AgdaSymbol{)}\AgdaSpace{}%
\AgdaSymbol{(}\AgdaInductiveConstructor{sels}\AgdaSpace{}%
\AgdaOperator{\AgdaFunction{⟨}}\AgdaSpace{}%
\AgdaBound{o}\AgdaSpace{}%
\AgdaOperator{\AgdaFunction{⟩}}\AgdaSpace{}%
\AgdaBound{i}\AgdaSymbol{)}\<%
\end{code}
With these primitives, we embed our running example in $E$ as follows:
\begin{code}%
%
\>[2]\AgdaFunction{cnn}\AgdaSpace{}%
\AgdaSymbol{:}\AgdaSpace{}%
\AgdaDatatype{E}\AgdaSpace{}%
\AgdaSymbol{\AgdaUnderscore{}}\AgdaSpace{}%
\AgdaSymbol{\AgdaUnderscore{}}\<%
\\
%
\>[2]\AgdaFunction{cnn}\AgdaSpace{}%
\AgdaSymbol{=}%
\>[2316I]\AgdaFunction{Lcon}%
\>[2317I]\AgdaSymbol{(}%
\>[16]\AgdaInductiveConstructor{ar}\AgdaSpace{}%
\AgdaSymbol{(}\AgdaNumber{28}\AgdaSpace{}%
\AgdaOperator{\AgdaInductiveConstructor{∷}}\AgdaSpace{}%
\AgdaNumber{28}\AgdaSpace{}%
\AgdaOperator{\AgdaInductiveConstructor{∷}}\AgdaSpace{}%
\AgdaInductiveConstructor{[]}\AgdaSymbol{)}\AgdaSpace{}%
\AgdaOperator{\AgdaInductiveConstructor{∷}}\AgdaSpace{}%
\AgdaInductiveConstructor{ar}\AgdaSpace{}%
\AgdaSymbol{(}\AgdaNumber{6}\AgdaSpace{}%
\AgdaOperator{\AgdaInductiveConstructor{∷}}\AgdaSpace{}%
\AgdaNumber{5}\AgdaSpace{}%
\AgdaOperator{\AgdaInductiveConstructor{∷}}\AgdaSpace{}%
\AgdaNumber{5}\AgdaSpace{}%
\AgdaOperator{\AgdaInductiveConstructor{∷}}\AgdaSpace{}%
\AgdaInductiveConstructor{[]}\AgdaSymbol{)}\<%
\\
\>[2317I][@{}l@{\AgdaIndent{0}}]%
\>[14]\AgdaOperator{\AgdaInductiveConstructor{∷}}\AgdaSpace{}%
\AgdaInductiveConstructor{ar}\AgdaSpace{}%
\AgdaSymbol{(}\AgdaNumber{6}\AgdaSpace{}%
\AgdaOperator{\AgdaInductiveConstructor{∷}}\AgdaSpace{}%
\AgdaInductiveConstructor{[]}\AgdaSymbol{)}%
\>[34]\AgdaOperator{\AgdaInductiveConstructor{∷}}\AgdaSpace{}%
\AgdaInductiveConstructor{ar}\AgdaSpace{}%
\AgdaSymbol{(}\AgdaNumber{12}\AgdaSpace{}%
\AgdaOperator{\AgdaInductiveConstructor{∷}}\AgdaSpace{}%
\AgdaNumber{6}\AgdaSpace{}%
\AgdaOperator{\AgdaInductiveConstructor{∷}}\AgdaSpace{}%
\AgdaNumber{5}\AgdaSpace{}%
\AgdaOperator{\AgdaInductiveConstructor{∷}}\AgdaSpace{}%
\AgdaNumber{5}\AgdaSpace{}%
\AgdaOperator{\AgdaInductiveConstructor{∷}}\AgdaSpace{}%
\AgdaInductiveConstructor{[]}\AgdaSymbol{)}\<%
\\
%
\>[14]\AgdaOperator{\AgdaInductiveConstructor{∷}}\AgdaSpace{}%
\AgdaInductiveConstructor{ar}\AgdaSpace{}%
\AgdaSymbol{(}\AgdaNumber{12}\AgdaSpace{}%
\AgdaOperator{\AgdaInductiveConstructor{∷}}\AgdaSpace{}%
\AgdaInductiveConstructor{[]}\AgdaSymbol{)}%
\>[34]\AgdaOperator{\AgdaInductiveConstructor{∷}}\AgdaSpace{}%
\AgdaInductiveConstructor{ar}\AgdaSpace{}%
\AgdaSymbol{(}\AgdaNumber{10}\AgdaSpace{}%
\AgdaOperator{\AgdaInductiveConstructor{∷}}\AgdaSpace{}%
\AgdaNumber{12}\AgdaSpace{}%
\AgdaOperator{\AgdaInductiveConstructor{∷}}\AgdaSpace{}%
\AgdaNumber{1}\AgdaSpace{}%
\AgdaOperator{\AgdaInductiveConstructor{∷}}\AgdaSpace{}%
\AgdaNumber{4}\AgdaSpace{}%
\AgdaOperator{\AgdaInductiveConstructor{∷}}\AgdaSpace{}%
\AgdaNumber{4}\AgdaSpace{}%
\AgdaOperator{\AgdaInductiveConstructor{∷}}\AgdaSpace{}%
\AgdaInductiveConstructor{[]}\AgdaSymbol{)}\<%
\\
%
\>[14]\AgdaOperator{\AgdaInductiveConstructor{∷}}\AgdaSpace{}%
\AgdaInductiveConstructor{ar}\AgdaSpace{}%
\AgdaSymbol{(}\AgdaNumber{10}\AgdaSpace{}%
\AgdaOperator{\AgdaInductiveConstructor{∷}}\AgdaSpace{}%
\AgdaInductiveConstructor{[]}\AgdaSymbol{)}%
\>[34]\AgdaOperator{\AgdaInductiveConstructor{∷}}\AgdaSpace{}%
\AgdaInductiveConstructor{ar}\AgdaSpace{}%
\AgdaSymbol{(}\AgdaNumber{10}\AgdaSpace{}%
\AgdaOperator{\AgdaInductiveConstructor{∷}}\AgdaSpace{}%
\AgdaNumber{1}\AgdaSpace{}%
\AgdaOperator{\AgdaInductiveConstructor{∷}}\AgdaSpace{}%
\AgdaNumber{1}\AgdaSpace{}%
\AgdaOperator{\AgdaInductiveConstructor{∷}}\AgdaSpace{}%
\AgdaNumber{1}\AgdaSpace{}%
\AgdaOperator{\AgdaInductiveConstructor{∷}}\AgdaSpace{}%
\AgdaNumber{1}\AgdaSpace{}%
\AgdaOperator{\AgdaInductiveConstructor{∷}}\AgdaSpace{}%
\AgdaInductiveConstructor{[]}\AgdaSymbol{)}\<%
\\
%
\>[14]\AgdaOperator{\AgdaInductiveConstructor{∷}}\AgdaSpace{}%
\AgdaInductiveConstructor{[]}\AgdaSymbol{)}\<%
\\
\>[.][@{}l@{}]\<[2317I]%
\>[13]\AgdaSymbol{(}\AgdaInductiveConstructor{ar}\AgdaSpace{}%
\AgdaInductiveConstructor{[]}\AgdaSymbol{)}\AgdaSpace{}%
\AgdaInductiveConstructor{ε}\<%
\\
\>[.][@{}l@{}]\<[2316I]%
\>[8]\AgdaSymbol{λ}\AgdaSpace{}%
\AgdaBound{inp}\AgdaSpace{}%
\AgdaBound{k₁}\AgdaSpace{}%
\AgdaBound{b₁}\AgdaSpace{}%
\AgdaBound{k₂}\AgdaSpace{}%
\AgdaBound{b₂}\AgdaSpace{}%
\AgdaBound{fc}\AgdaSpace{}%
\AgdaBound{b}\AgdaSpace{}%
\AgdaBound{target}\AgdaSpace{}%
\AgdaSymbol{→}\<%
\\
%
\>[8]\AgdaFunction{Let}\AgdaSpace{}%
\AgdaBound{c₁₁}\AgdaSpace{}%
\AgdaFunction{:=}\AgdaSpace{}%
\AgdaFunction{mconv}\AgdaSpace{}%
\AgdaBound{inp}\AgdaSpace{}%
\AgdaBound{k₁}\AgdaSpace{}%
\AgdaBound{b₁}%
\>[36]\AgdaFunction{In}\<%
\\
%
\>[8]\AgdaFunction{Let}\AgdaSpace{}%
\AgdaBound{c₁}%
\>[16]\AgdaFunction{:=}\AgdaSpace{}%
\AgdaInductiveConstructor{logistic}\AgdaSpace{}%
\AgdaBound{c₁₁}\AgdaSpace{}%
\AgdaFunction{In}\<%
\\
%
\>[8]\AgdaFunction{Let}\AgdaSpace{}%
\AgdaBound{s₁}%
\>[16]\AgdaFunction{:=}\AgdaSpace{}%
\AgdaSymbol{(}\AgdaFunction{Imap}\AgdaSpace{}%
\AgdaSymbol{\{}\AgdaArgument{s}\AgdaSpace{}%
\AgdaSymbol{=}\AgdaSpace{}%
\AgdaNumber{6}\AgdaSpace{}%
\AgdaOperator{\AgdaInductiveConstructor{∷}}\AgdaSpace{}%
\AgdaInductiveConstructor{[]}\AgdaSymbol{\}}\AgdaSpace{}%
\AgdaSymbol{λ}\AgdaSpace{}%
\AgdaBound{i}\AgdaSpace{}%
\AgdaSymbol{→}\AgdaSpace{}%
\AgdaFunction{avgp₂}\AgdaSpace{}%
\AgdaNumber{12}\AgdaSpace{}%
\AgdaNumber{12}\AgdaSpace{}%
\AgdaSymbol{(}\AgdaInductiveConstructor{sel}\AgdaSpace{}%
\AgdaBound{c₁}\AgdaSpace{}%
\AgdaBound{i}\AgdaSymbol{))}\AgdaSpace{}%
\AgdaFunction{In}\<%
\\
%
\>[8]\AgdaFunction{Let}\AgdaSpace{}%
\AgdaBound{c₂₁}\AgdaSpace{}%
\AgdaFunction{:=}\AgdaSpace{}%
\AgdaFunction{mconv}\AgdaSpace{}%
\AgdaBound{s₁}\AgdaSpace{}%
\AgdaBound{k₂}\AgdaSpace{}%
\AgdaBound{b₂}\AgdaSpace{}%
\AgdaFunction{In}\<%
\\
%
\>[8]\AgdaFunction{Let}\AgdaSpace{}%
\AgdaBound{c₂}%
\>[16]\AgdaFunction{:=}\AgdaSpace{}%
\AgdaInductiveConstructor{logistic}\AgdaSpace{}%
\AgdaBound{c₂₁}\AgdaSpace{}%
\AgdaFunction{In}\<%
\\
%
\>[8]\AgdaFunction{Let}\AgdaSpace{}%
\AgdaBound{s₂}%
\>[16]\AgdaFunction{:=}\AgdaSpace{}%
\AgdaSymbol{(}\AgdaFunction{Imap}\AgdaSpace{}%
\AgdaSymbol{\{}\AgdaArgument{s}\AgdaSpace{}%
\AgdaSymbol{=}\AgdaSpace{}%
\AgdaNumber{12}\AgdaSpace{}%
\AgdaOperator{\AgdaInductiveConstructor{∷}}\AgdaSpace{}%
\AgdaNumber{1}\AgdaSpace{}%
\AgdaOperator{\AgdaInductiveConstructor{∷}}\AgdaSpace{}%
\AgdaInductiveConstructor{[]}\AgdaSymbol{\}}\AgdaSpace{}%
\AgdaSymbol{λ}\AgdaSpace{}%
\AgdaBound{i}\AgdaSpace{}%
\AgdaSymbol{→}\AgdaSpace{}%
\AgdaFunction{avgp₂}\AgdaSpace{}%
\AgdaNumber{4}\AgdaSpace{}%
\AgdaNumber{4}\AgdaSpace{}%
\AgdaSymbol{(}\AgdaInductiveConstructor{sel}\AgdaSpace{}%
\AgdaBound{c₂}\AgdaSpace{}%
\AgdaBound{i}\AgdaSymbol{))}\AgdaSpace{}%
\AgdaFunction{In}\<%
\\
%
\>[8]\AgdaFunction{Let}\AgdaSpace{}%
\AgdaBound{o₁}%
\>[16]\AgdaFunction{:=}\AgdaSpace{}%
\AgdaFunction{mconv}\AgdaSpace{}%
\AgdaBound{s₂}\AgdaSpace{}%
\AgdaBound{fc}\AgdaSpace{}%
\AgdaBound{b}\AgdaSpace{}%
\AgdaFunction{In}\<%
\\
%
\>[8]\AgdaFunction{Let}\AgdaSpace{}%
\AgdaBound{o}%
\>[16]\AgdaFunction{:=}\AgdaSpace{}%
\AgdaInductiveConstructor{logistic}\AgdaSpace{}%
\AgdaBound{o₁}\AgdaSpace{}%
\AgdaFunction{In}\<%
\\
%
\>[8]\AgdaFunction{Let}\AgdaSpace{}%
\AgdaBound{e}%
\>[16]\AgdaFunction{:=}\AgdaSpace{}%
\AgdaFunction{meansqerr}\AgdaSpace{}%
\AgdaBound{target}\AgdaSpace{}%
\AgdaBound{o}\AgdaSpace{}%
\AgdaFunction{In}\<%
\\
%
\>[8]\AgdaBound{e}\<%
\end{code}
Note that with the proposed syntax, the above definition looks very similar
to the one we defined directly in Agda.  We use more let bindings in this
definition, which is not an arbitrary choice, and we will come back to this
discussion in the next section.

% \paragraph{Building Blocks}
% Now we implement the remaining building blocks in \AD{E} that are needed
% to define our CNN.
% \begin{code}[hide]
% module BB where
%   open import Data.Nat as ℕ using (ℕ; zero; suc)
%   open Array hiding (sum; slide; backslide)
%   open Lang
%   open SubWk using (wk; ↑_; ↑↑_)
% 
%   --_⊞_ _⊠_ : (a b : E Γ (ar s)) → E Γ (ar s)
%   Imapₛ : (E (Γ ▹ ix s) (ix s) → E (Γ ▹ ix s) (ar unit)) → E Γ (ar s)
%   Imap : (E (Γ ▹ ix s) (ix s) → E (Γ ▹ ix s) (ar p)) → E Γ (ar (s ⊗ p))
%   Sum : (E (Γ ▹ ix s) (ix s) → E (Γ ▹ ix s) (ar p)) → E Γ (ar p)
% \end{code}
% We start with a several convenience functions that wrap \AC{imap}s and \AC{sum}
% such that when we write (\AF{Imap} \AB{λ} \AB{i} \AB{→} \AB{⋯}), Agda's variable
% $i$ is mapped to the \AF{E}'s variable \AC{v₀}.
% \begin{mathpar}
% \codeblock{\begin{code}
%   Imapₛ f = imapₛ (f (var v₀))
% \end{code}}
% \and
% \codeblock{\begin{code}
%   Imap f = imap (f (var v₀))
% \end{code}}
% \and
% \codeblock{\begin{code}
%   Sum f = sum (f (var v₀))
% \end{code}}
% \end{mathpar}
% 
% The remaining operations are \AF{conv}, \AF{mconv} and \AF{avgp₂} which
% can be defined as functions on \AF{E} as follows.
% \begin{code}
%   conv : E Γ (ar r) → s + p ≈ r → E Γ (ar s) → suc p ≈ u → E Γ (ar u)
%   conv f sp g su = Sum λ i → slide i sp (↑ f) su ⊠ Imapₛ λ _ → selₛ (↑↑ g) (↑ i)
% 
%   mconv : s + p ≈ r → (inp : E Γ (ar r)) (we : E Γ (ar (u ⊗ s))) (b : E Γ (ar u))
%         → suc p ≈ w → E Γ (ar (u ⊗ w))
%   mconv sp inp we b su = Imap λ i → conv (↑ inp) sp (sel (↑ we) i) su ⊞ Imapₛ λ _ → selₛ (↑↑ b) (↑ i)
% 
%   avgp₂ : ∀ m n → (a : E Γ (ar (ι (m ℕ.* 2) ⊗ ι (n ℕ.* 2)))) → E Γ (ar (ι m ⊗ ι n))
%   avgp₂ m n a = Imapₛ λ i → scaledown 4 $ Sum λ j → selₛ (selb (ι ⊗ ι) (↑↑ a) (↑ i)) j
% 
% \end{code}
% Note that these definitions are not very different from those found in
% Section~\ref{sec:array-theory}.  Some operations such as \AF{nest} and \AF{unnest}
% got inlined into \AF{E}'s operators, and all we really have to take care of is 
% weakening of the expressions whenever we go under binders.


\begin{code}[hide]%
\>[0]\AgdaKeyword{open}\AgdaSpace{}%
\AgdaKeyword{import}\AgdaSpace{}%
\AgdaModule{Relation.Binary.PropositionalEquality}\<%
\\
\>[0]\AgdaKeyword{open}\AgdaSpace{}%
\AgdaKeyword{import}\AgdaSpace{}%
\AgdaModule{Relation.Nullary}\<%
\\
\>[0]\AgdaKeyword{open}\AgdaSpace{}%
\AgdaKeyword{import}\AgdaSpace{}%
\AgdaModule{Data.List}\AgdaSpace{}%
\AgdaKeyword{using}\AgdaSpace{}%
\AgdaSymbol{(}\AgdaDatatype{List}\AgdaSymbol{;}\AgdaSpace{}%
\AgdaInductiveConstructor{[]}\AgdaSymbol{;}\AgdaSpace{}%
\AgdaOperator{\AgdaInductiveConstructor{\AgdaUnderscore{}∷\AgdaUnderscore{}}}\AgdaSymbol{)}\<%
\\
\>[0]\AgdaKeyword{open}\AgdaSpace{}%
\AgdaKeyword{import}\AgdaSpace{}%
\AgdaModule{Data.Empty}\<%
\\
\>[0]\AgdaKeyword{open}\AgdaSpace{}%
\AgdaKeyword{import}\AgdaSpace{}%
\AgdaModule{Function}\<%
\\
%
\\[\AgdaEmptyExtraSkip]%
\>[0]\AgdaComment{--\ Our\ local\ files.}\<%
\\
\>[0]\AgdaKeyword{open}\AgdaSpace{}%
\AgdaKeyword{import}\AgdaSpace{}%
\AgdaModule{arrays}\<%
\\
\>[0]\AgdaKeyword{open}\AgdaSpace{}%
\AgdaKeyword{import}\AgdaSpace{}%
\AgdaModule{lang}\<%
\\
\>[0]\AgdaKeyword{module}\AgdaSpace{}%
\AgdaModule{\AgdaUnderscore{}}\AgdaSpace{}%
\AgdaKeyword{where}\<%
\end{code}

\section{Automatic Differentiation\label{sec:ad}}

We implement automatic differentiation in reverse mode
for expressions in \AF{E}.  We focus on reverse mode because it is
of most interest in machine learning, and it is more challenging to implement.
We start with a brief introduction of the AD, for much more in-depth
explanations refer to~\cite{autodiff-survey, backprop-stlc}.   Consider differentiating
a function composition consisting of three functions:
\[ 
   y = (f \circ g \circ h)\ x
\]
rewrite it using temporary variables:
\begin{eqnarray*}
  w_0 &=& x \\
  w_1 &=& h\ w_0 \\
  w_2 &=& g\ w_1 \\
  w_3 &=& f\ w_2 = y
\end{eqnarray*}
The chain rule gives us 
$\frac{\partial y}{\partial x} 
  = \frac{\partial y}{\partial w_2}
    \frac{\partial w_2}{\partial w_1}
    \frac{\partial w_1}{\partial x}$.  The difference between the forward and reverse
    mode lies in the direction that we traverse the chain rule.  In forward mode we
    traverse the chain inside-out, and the revers mode traverses the chain outside-in
    thus computing recursive relation:
$\frac{\partial y}{\partial w_i}
  = \frac{\partial y}{\partial w_{i+1}}
    \frac{\partial w_{i+1}}{\partial w_i}$.  For our example, we compute
$\frac{\partial y}{\partial w_2}$, then $\frac{\partial w_2}{\partial w_1}$ and
finally $\frac{\partial w_1}{\partial x}$.  While there seem to be no difference for
functions of one variable, there is a big difference for functions of $n$ variables
as we can compute derivatives of all the non-dependent variables simultaneously.
Consider an example of the $z = f\ x\ y = sin(xy + x)$:
\begin{eqnarray*}
  w_0 &=& x \\
  w_1 &=& y \\
  w_2 &=& w_1w_2\\
  w_3 &=& w_2 + w_0 \\
  w_4 &=& \sin w_3 = z
\end{eqnarray*}
We compute the adjoints $\bar{w}_i = \frac{\partial y}{\partial w_i}$ using the following
rule.  If $w_i$ has successors in the computational graph, we can apply the chain rule
as follows:
\[ 
    \bar{w}_i = \sum_{j \in succ\ i} \bar{w}_j\frac{\partial w_j}{\partial w_i}
\]
For our example:
\begin{eqnarray*}
  \bar{w}_4 &=& 1 = \frac{\partial z}{\partial z} \\
  \bar{w}_3 &=& \bar{w}_4 \cos w_3\\
  \bar{w}_2 &=& \bar{w}_3 \cdot 1 \\
  \bar{w}_1 &=& \bar{w}_2 w_0 \\
  \bar{w}_0 &=& \bar{w}_3 + \bar{w}_2 w_1
\end{eqnarray*}
If we inline all the $\bar{w}_i$ definitions and inspect the values of partial derivatives
with respect to $x$ and $y$ we obtain expected results:
$\frac{\partial z}{\partial x} = \cos (xy + x)(y + 1)$ and
$\frac{\partial z}{\partial y} = \cos (xy + x)x$.


In the implementation of the AD for \AF{E} in some context \AB{Γ}, we would like to obtain
all the partial derivatives with respect to the variables in context \AB{Γ}.  Each partial
derivative is itself an expression \AF{E} in context \AF{Γ}.  Therefore, we need to define
a data type for an environment of \AB{Γ}-many expressions in context \AB{Γ}.  We call this
environment \AF{Env} which is defined below.  This is a standard construction that is
similar to our parallel substitution \AF{Sub}, except we ignore the values of index
types --- they never contribute to the computation of values, so we do not need to
compute partial derivatives for them.  However, the presence of lets in \AF{E}
means that the let-bound expressions may be shared by several partial derivative
expressions.  While we could replicate the let binding for every partial derivative
that needs it, this leads to unnecessary code duplication which in turn leads to
inefficient performance.  As a solution, we allow let bindings for the entire \AF{Env},
which is achieved by the \AF{EE} type defined as follows.
\begin{mathpar}
\codeblock{\begin{code}[hide]%
\>[0]\AgdaKeyword{module}\AgdaSpace{}%
\AgdaModule{AD}\AgdaSpace{}%
\AgdaKeyword{where}\<%
\\
\>[0][@{}l@{\AgdaIndent{0}}]%
\>[2]\AgdaKeyword{open}\AgdaSpace{}%
\AgdaKeyword{import}\AgdaSpace{}%
\AgdaModule{Data.Unit}\<%
\\
%
\>[2]\AgdaKeyword{open}\AgdaSpace{}%
\AgdaKeyword{import}\AgdaSpace{}%
\AgdaModule{Data.Product}\AgdaSpace{}%
\AgdaSymbol{as}\AgdaSpace{}%
\AgdaModule{Prod}\<%
\\
%
\>[2]\AgdaKeyword{open}\AgdaSpace{}%
\AgdaModule{Array}\AgdaSpace{}%
\AgdaKeyword{hiding}\AgdaSpace{}%
\AgdaSymbol{(}\AgdaFunction{sum}\AgdaSymbol{;}\AgdaSpace{}%
\AgdaFunction{backslide}\AgdaSymbol{;}\AgdaSpace{}%
\AgdaFunction{slide}\AgdaSymbol{)}\<%
\\
%
\>[2]\AgdaKeyword{open}\AgdaSpace{}%
\AgdaModule{WkSub}\<%
\\
%
\>[2]\AgdaKeyword{open}\AgdaSpace{}%
\AgdaModule{Lang}\<%
\end{code}
\begin{code}%
%
\>[2]\AgdaKeyword{data}\AgdaSpace{}%
\AgdaDatatype{Env}\AgdaSpace{}%
\AgdaSymbol{:}\AgdaSpace{}%
\AgdaDatatype{Ctx}\AgdaSpace{}%
\AgdaSymbol{→}\AgdaSpace{}%
\AgdaDatatype{Ctx}\AgdaSpace{}%
\AgdaSymbol{→}\AgdaSpace{}%
\AgdaPrimitive{Set}\AgdaSpace{}%
\AgdaKeyword{where}\<%
\\
\>[2][@{}l@{\AgdaIndent{0}}]%
\>[4]\AgdaInductiveConstructor{ε}%
\>[10]\AgdaSymbol{:}\AgdaSpace{}%
\AgdaDatatype{Env}\AgdaSpace{}%
\AgdaInductiveConstructor{ε}\AgdaSpace{}%
\AgdaGeneralizable{Γ}\<%
\\
%
\>[4]\AgdaInductiveConstructor{skip}%
\>[10]\AgdaSymbol{:}\AgdaSpace{}%
\AgdaDatatype{Env}\AgdaSpace{}%
\AgdaGeneralizable{Γ}\AgdaSpace{}%
\AgdaGeneralizable{Δ}\AgdaSpace{}%
\AgdaSymbol{→}\AgdaSpace{}%
\AgdaDatatype{Env}\AgdaSpace{}%
\AgdaSymbol{(}\AgdaGeneralizable{Γ}\AgdaSpace{}%
\AgdaOperator{\AgdaInductiveConstructor{▹}}\AgdaSpace{}%
\AgdaInductiveConstructor{ix}\AgdaSpace{}%
\AgdaGeneralizable{s}\AgdaSymbol{)}\AgdaSpace{}%
\AgdaGeneralizable{Δ}\<%
\\
%
\>[4]\AgdaOperator{\AgdaInductiveConstructor{\AgdaUnderscore{}▹\AgdaUnderscore{}}}%
\>[10]\AgdaSymbol{:}\AgdaSpace{}%
\AgdaDatatype{Env}\AgdaSpace{}%
\AgdaGeneralizable{Γ}\AgdaSpace{}%
\AgdaGeneralizable{Δ}\AgdaSpace{}%
\AgdaSymbol{→}\AgdaSpace{}%
\AgdaDatatype{E}\AgdaSpace{}%
\AgdaGeneralizable{Δ}\AgdaSpace{}%
\AgdaSymbol{(}\AgdaInductiveConstructor{ar}\AgdaSpace{}%
\AgdaGeneralizable{s}\AgdaSymbol{)}\AgdaSpace{}%
\AgdaSymbol{→}\AgdaSpace{}%
\AgdaDatatype{Env}\AgdaSpace{}%
\AgdaSymbol{(}\AgdaGeneralizable{Γ}\AgdaSpace{}%
\AgdaOperator{\AgdaInductiveConstructor{▹}}\AgdaSpace{}%
\AgdaInductiveConstructor{ar}\AgdaSpace{}%
\AgdaGeneralizable{s}\AgdaSymbol{)}\AgdaSpace{}%
\AgdaGeneralizable{Δ}\<%
\end{code}}
\and
\codeblock{\begin{code}%
%
\>[2]\AgdaKeyword{data}\AgdaSpace{}%
\AgdaDatatype{EE}\AgdaSpace{}%
\AgdaSymbol{:}\AgdaSpace{}%
\AgdaDatatype{Ctx}\AgdaSpace{}%
\AgdaSymbol{→}\AgdaSpace{}%
\AgdaDatatype{Ctx}\AgdaSpace{}%
\AgdaSymbol{→}\AgdaSpace{}%
\AgdaPrimitive{Set}\AgdaSpace{}%
\AgdaKeyword{where}\<%
\\
\>[2][@{}l@{\AgdaIndent{0}}]%
\>[4]\AgdaInductiveConstructor{env}%
\>[10]\AgdaSymbol{:}\AgdaSpace{}%
\AgdaDatatype{Env}\AgdaSpace{}%
\AgdaGeneralizable{Γ}\AgdaSpace{}%
\AgdaGeneralizable{Δ}\AgdaSpace{}%
\AgdaSymbol{→}\AgdaSpace{}%
\AgdaDatatype{EE}\AgdaSpace{}%
\AgdaGeneralizable{Γ}\AgdaSpace{}%
\AgdaGeneralizable{Δ}\<%
\\
%
\>[4]\AgdaInductiveConstructor{let′}%
\>[10]\AgdaSymbol{:}\AgdaSpace{}%
\AgdaDatatype{E}\AgdaSpace{}%
\AgdaGeneralizable{Δ}\AgdaSpace{}%
\AgdaSymbol{(}\AgdaInductiveConstructor{ar}\AgdaSpace{}%
\AgdaGeneralizable{s}\AgdaSymbol{)}\AgdaSpace{}%
\AgdaSymbol{→}\AgdaSpace{}%
\AgdaDatatype{EE}\AgdaSpace{}%
\AgdaGeneralizable{Γ}\AgdaSpace{}%
\AgdaSymbol{(}\AgdaGeneralizable{Δ}\AgdaSpace{}%
\AgdaOperator{\AgdaInductiveConstructor{▹}}\AgdaSpace{}%
\AgdaInductiveConstructor{ar}\AgdaSpace{}%
\AgdaGeneralizable{s}\AgdaSymbol{)}\<%
\\
%
\>[10]\AgdaSymbol{→}\AgdaSpace{}%
\AgdaDatatype{EE}\AgdaSpace{}%
\AgdaGeneralizable{Γ}\AgdaSpace{}%
\AgdaGeneralizable{Δ}\<%
\end{code}}
\end{mathpar}

We briefly explain a few useful combinators that manipulate (let-extended) environments.
We can weaken environments in two ways: \AF{ee-wk} weakens each element of the environment;
whereas \AF{ee-wk-zero} extends the length of the environment by inserting \AC{zero}
elements according to the \AF{⊆}-argument.  We can add two environments with \AF{ee-plus}
which adds elements of the environments point-wise combines two let chains into a single
one.  For the environment where elements are in the context where zero-th variable
is some (\AC{ix} s), \AF{ee-map-sum} applies \AF{sum} to all its elements.  Note that
here we inline let-bindings of the environment into the elements, because the bindings
may refer to the index.  This potentially leads to code duplication, but for now, we
assume that further optimisations will be able to deal with this.  We may reconsider
this choice later.  We can create an empty environment where all the elements are
\AC{zero} using \AF{ee-zero}.  We remove the top element of the environment with
\AF{ee-tail}.  We use \AF{ee-update+} \AB{ρ} \AB{i} \AB{e} to add to the $i$-th
element of the environment (this returns the new environment with the updated element).
Finally, we can extend the environment by adding \AC{zero} as the top element with
\AF{\_▹𝟘}.
\begin{mathpar}
\codeblock{\begin{code}%
%
\>[2]\AgdaFunction{ee-wk}%
\>[14]\AgdaSymbol{:}\AgdaSpace{}%
\AgdaGeneralizable{Δ}\AgdaSpace{}%
\AgdaOperator{\AgdaDatatype{⊆}}\AgdaSpace{}%
\AgdaGeneralizable{Ψ}\AgdaSpace{}%
\AgdaSymbol{→}\AgdaSpace{}%
\AgdaDatatype{EE}\AgdaSpace{}%
\AgdaGeneralizable{Γ}\AgdaSpace{}%
\AgdaGeneralizable{Δ}\AgdaSpace{}%
\AgdaSymbol{→}\AgdaSpace{}%
\AgdaDatatype{EE}\AgdaSpace{}%
\AgdaGeneralizable{Γ}\AgdaSpace{}%
\AgdaGeneralizable{Ψ}\<%
\\
%
\>[2]\AgdaFunction{ee-wk-zero}%
\>[14]\AgdaSymbol{:}\AgdaSpace{}%
\AgdaDatatype{EE}\AgdaSpace{}%
\AgdaGeneralizable{Γ}\AgdaSpace{}%
\AgdaGeneralizable{Δ}\AgdaSpace{}%
\AgdaSymbol{→}\AgdaSpace{}%
\AgdaGeneralizable{Γ}\AgdaSpace{}%
\AgdaOperator{\AgdaDatatype{⊆}}\AgdaSpace{}%
\AgdaGeneralizable{Ψ}\AgdaSpace{}%
\AgdaSymbol{→}\AgdaSpace{}%
\AgdaDatatype{EE}\AgdaSpace{}%
\AgdaGeneralizable{Ψ}\AgdaSpace{}%
\AgdaGeneralizable{Δ}\<%
\end{code}}
\and
\codeblock{\begin{code}%
%
\>[2]\AgdaFunction{ee-plus}%
\>[14]\AgdaSymbol{:}\AgdaSpace{}%
\AgdaSymbol{(}\AgdaBound{ρ}\AgdaSpace{}%
\AgdaBound{ν}\AgdaSpace{}%
\AgdaSymbol{:}\AgdaSpace{}%
\AgdaDatatype{EE}\AgdaSpace{}%
\AgdaGeneralizable{Γ}\AgdaSpace{}%
\AgdaGeneralizable{Δ}\AgdaSymbol{)}\AgdaSpace{}%
\AgdaSymbol{→}\AgdaSpace{}%
\AgdaDatatype{EE}\AgdaSpace{}%
\AgdaGeneralizable{Γ}\AgdaSpace{}%
\AgdaGeneralizable{Δ}\<%
\\
%
\>[2]\AgdaFunction{ee-map-sum}%
\>[14]\AgdaSymbol{:}\AgdaSpace{}%
\AgdaDatatype{EE}\AgdaSpace{}%
\AgdaGeneralizable{Γ}\AgdaSpace{}%
\AgdaSymbol{(}\AgdaGeneralizable{Δ}\AgdaSpace{}%
\AgdaOperator{\AgdaInductiveConstructor{▹}}\AgdaSpace{}%
\AgdaInductiveConstructor{ix}\AgdaSpace{}%
\AgdaGeneralizable{s}\AgdaSymbol{)}\AgdaSpace{}%
\AgdaSymbol{→}\AgdaSpace{}%
\AgdaDatatype{EE}\AgdaSpace{}%
\AgdaGeneralizable{Γ}\AgdaSpace{}%
\AgdaGeneralizable{Δ}\<%
\end{code}}
\and
\codeblock{\begin{code}%
%
\>[2]\AgdaFunction{ee-tail}%
\>[14]\AgdaSymbol{:}\AgdaSpace{}%
\AgdaDatatype{EE}\AgdaSpace{}%
\AgdaSymbol{(}\AgdaGeneralizable{Γ}\AgdaSpace{}%
\AgdaOperator{\AgdaInductiveConstructor{▹}}\AgdaSpace{}%
\AgdaGeneralizable{is}\AgdaSymbol{)}\AgdaSpace{}%
\AgdaGeneralizable{Δ}\AgdaSpace{}%
\AgdaSymbol{→}\AgdaSpace{}%
\AgdaDatatype{EE}\AgdaSpace{}%
\AgdaGeneralizable{Γ}\AgdaSpace{}%
\AgdaGeneralizable{Δ}\<%
\\
%
\>[2]\AgdaFunction{zero-ee}%
\>[14]\AgdaSymbol{:}\AgdaSpace{}%
\AgdaDatatype{EE}\AgdaSpace{}%
\AgdaGeneralizable{Γ}\AgdaSpace{}%
\AgdaGeneralizable{Δ}\<%
\end{code}}
\and
\codeblock{\begin{code}%
%
\>[2]\AgdaFunction{ee-update+}%
\>[14]\AgdaSymbol{:}\AgdaSpace{}%
\AgdaDatatype{EE}\AgdaSpace{}%
\AgdaGeneralizable{Γ}\AgdaSpace{}%
\AgdaGeneralizable{Δ}\AgdaSpace{}%
\AgdaSymbol{→}\AgdaSpace{}%
\AgdaInductiveConstructor{ar}\AgdaSpace{}%
\AgdaGeneralizable{s}\AgdaSpace{}%
\AgdaOperator{\AgdaDatatype{∈}}\AgdaSpace{}%
\AgdaGeneralizable{Γ}\AgdaSpace{}%
\AgdaSymbol{→}\AgdaSpace{}%
\AgdaDatatype{E}\AgdaSpace{}%
\AgdaGeneralizable{Δ}\AgdaSpace{}%
\AgdaSymbol{(}\AgdaInductiveConstructor{ar}\AgdaSpace{}%
\AgdaGeneralizable{s}\AgdaSymbol{)}\AgdaSpace{}%
\AgdaSymbol{→}\AgdaSpace{}%
\AgdaDatatype{EE}\AgdaSpace{}%
\AgdaGeneralizable{Γ}\AgdaSpace{}%
\AgdaGeneralizable{Δ}\<%
\\
%
\>[2]\AgdaOperator{\AgdaFunction{\AgdaUnderscore{}▹𝟘}}%
\>[14]\AgdaSymbol{:}\AgdaSpace{}%
\AgdaDatatype{EE}\AgdaSpace{}%
\AgdaGeneralizable{Γ}\AgdaSpace{}%
\AgdaGeneralizable{Δ}\AgdaSpace{}%
\AgdaSymbol{→}\AgdaSpace{}%
\AgdaDatatype{EE}\AgdaSpace{}%
\AgdaSymbol{(}\AgdaGeneralizable{Γ}\AgdaSpace{}%
\AgdaOperator{\AgdaInductiveConstructor{▹}}\AgdaSpace{}%
\AgdaInductiveConstructor{ar}\AgdaSpace{}%
\AgdaGeneralizable{s}\AgdaSymbol{)}\AgdaSpace{}%
\AgdaSymbol{(}\AgdaGeneralizable{Δ}\AgdaSpace{}%
\AgdaOperator{\AgdaInductiveConstructor{▹}}\AgdaSpace{}%
\AgdaInductiveConstructor{ar}\AgdaSpace{}%
\AgdaGeneralizable{s}\AgdaSymbol{)}\<%
\end{code}}
\end{mathpar}
\begin{code}[hide]%
%
\>[2]\AgdaComment{--\ Weaken\ all\ expressions\ in\ the\ Env\ enironment}\<%
\\
%
\>[2]\AgdaFunction{env-wk}\AgdaSpace{}%
\AgdaSymbol{:}\AgdaSpace{}%
\AgdaGeneralizable{Δ}\AgdaSpace{}%
\AgdaOperator{\AgdaDatatype{⊆}}\AgdaSpace{}%
\AgdaGeneralizable{Ψ}\AgdaSpace{}%
\AgdaSymbol{→}\AgdaSpace{}%
\AgdaDatatype{Env}\AgdaSpace{}%
\AgdaGeneralizable{Γ}\AgdaSpace{}%
\AgdaGeneralizable{Δ}\AgdaSpace{}%
\AgdaSymbol{→}\AgdaSpace{}%
\AgdaDatatype{Env}\AgdaSpace{}%
\AgdaGeneralizable{Γ}\AgdaSpace{}%
\AgdaGeneralizable{Ψ}\<%
\\
%
\>[2]\AgdaFunction{env-wk}\AgdaSpace{}%
\AgdaBound{w}\AgdaSpace{}%
\AgdaInductiveConstructor{ε}\AgdaSpace{}%
\AgdaSymbol{=}\AgdaSpace{}%
\AgdaInductiveConstructor{ε}\<%
\\
%
\>[2]\AgdaFunction{env-wk}\AgdaSpace{}%
\AgdaBound{w}\AgdaSpace{}%
\AgdaSymbol{(}\AgdaInductiveConstructor{skip}\AgdaSpace{}%
\AgdaBound{ρ}\AgdaSymbol{)}\AgdaSpace{}%
\AgdaSymbol{=}\AgdaSpace{}%
\AgdaInductiveConstructor{skip}\AgdaSpace{}%
\AgdaSymbol{(}\AgdaFunction{env-wk}\AgdaSpace{}%
\AgdaBound{w}\AgdaSpace{}%
\AgdaBound{ρ}\AgdaSymbol{)}\<%
\\
%
\>[2]\AgdaFunction{env-wk}\AgdaSpace{}%
\AgdaBound{w}\AgdaSpace{}%
\AgdaSymbol{(}\AgdaBound{ρ}\AgdaSpace{}%
\AgdaOperator{\AgdaInductiveConstructor{▹}}\AgdaSpace{}%
\AgdaBound{x}\AgdaSymbol{)}\AgdaSpace{}%
\AgdaSymbol{=}\AgdaSpace{}%
\AgdaFunction{env-wk}\AgdaSpace{}%
\AgdaBound{w}\AgdaSpace{}%
\AgdaBound{ρ}\AgdaSpace{}%
\AgdaOperator{\AgdaInductiveConstructor{▹}}\AgdaSpace{}%
\AgdaFunction{wk}\AgdaSpace{}%
\AgdaBound{w}\AgdaSpace{}%
\AgdaBound{x}\<%
\\
%
\\[\AgdaEmptyExtraSkip]%
%
\>[2]\AgdaComment{--\ Weaken\ all\ expressions\ in\ the\ EE\ environment}\<%
\\
%
\>[2]\AgdaFunction{ee-wk}\AgdaSpace{}%
\AgdaBound{w}\AgdaSpace{}%
\AgdaSymbol{(}\AgdaInductiveConstructor{env}\AgdaSpace{}%
\AgdaBound{x}\AgdaSymbol{)}\AgdaSpace{}%
\AgdaSymbol{=}\AgdaSpace{}%
\AgdaInductiveConstructor{env}\AgdaSpace{}%
\AgdaSymbol{(}\AgdaFunction{env-wk}\AgdaSpace{}%
\AgdaBound{w}\AgdaSpace{}%
\AgdaBound{x}\AgdaSymbol{)}\<%
\\
%
\>[2]\AgdaFunction{ee-wk}\AgdaSpace{}%
\AgdaBound{w}\AgdaSpace{}%
\AgdaSymbol{(}\AgdaInductiveConstructor{let′}\AgdaSpace{}%
\AgdaBound{x}\AgdaSpace{}%
\AgdaBound{ρ}\AgdaSymbol{)}\AgdaSpace{}%
\AgdaSymbol{=}\AgdaSpace{}%
\AgdaInductiveConstructor{let′}\AgdaSpace{}%
\AgdaSymbol{(}\AgdaFunction{wk}\AgdaSpace{}%
\AgdaBound{w}\AgdaSpace{}%
\AgdaBound{x}\AgdaSymbol{)}\AgdaSpace{}%
\AgdaSymbol{(}\AgdaFunction{ee-wk}\AgdaSpace{}%
\AgdaSymbol{(}\AgdaInductiveConstructor{keep}\AgdaSpace{}%
\AgdaBound{w}\AgdaSymbol{)}\AgdaSpace{}%
\AgdaBound{ρ}\AgdaSymbol{)}\<%
\\
%
\\[\AgdaEmptyExtraSkip]%
%
\>[2]\AgdaComment{--\ Throw\ away\ the\ last\ element}\<%
\\
%
\>[2]\AgdaFunction{ee-tail}\AgdaSpace{}%
\AgdaSymbol{(}\AgdaInductiveConstructor{env}\AgdaSpace{}%
\AgdaSymbol{(}\AgdaInductiveConstructor{skip}\AgdaSpace{}%
\AgdaBound{ρ}\AgdaSymbol{))}\AgdaSpace{}%
\AgdaSymbol{=}\AgdaSpace{}%
\AgdaInductiveConstructor{env}\AgdaSpace{}%
\AgdaBound{ρ}\<%
\\
%
\>[2]\AgdaFunction{ee-tail}\AgdaSpace{}%
\AgdaSymbol{(}\AgdaInductiveConstructor{env}\AgdaSpace{}%
\AgdaSymbol{(}\AgdaBound{ρ}\AgdaSpace{}%
\AgdaOperator{\AgdaInductiveConstructor{▹}}\AgdaSpace{}%
\AgdaSymbol{\AgdaUnderscore{}))}\AgdaSpace{}%
\AgdaSymbol{=}\AgdaSpace{}%
\AgdaInductiveConstructor{env}\AgdaSpace{}%
\AgdaBound{ρ}\<%
\\
%
\>[2]\AgdaFunction{ee-tail}\AgdaSpace{}%
\AgdaSymbol{(}\AgdaInductiveConstructor{let′}\AgdaSpace{}%
\AgdaBound{x}\AgdaSpace{}%
\AgdaBound{ρ}\AgdaSymbol{)}\AgdaSpace{}%
\AgdaSymbol{=}\AgdaSpace{}%
\AgdaInductiveConstructor{let′}\AgdaSpace{}%
\AgdaBound{x}\AgdaSpace{}%
\AgdaSymbol{(}\AgdaFunction{ee-tail}\AgdaSpace{}%
\AgdaBound{ρ}\AgdaSymbol{)}\<%
\\
%
\\[\AgdaEmptyExtraSkip]%
%
\>[2]\AgdaComment{--\ Insert\ zeroes\ in\ the\ environment\ Env\ according\ to\ the\ ⊆\ content}\<%
\\
%
\>[2]\AgdaFunction{env-wk-zero}\AgdaSpace{}%
\AgdaSymbol{:}\AgdaSpace{}%
\AgdaDatatype{Env}\AgdaSpace{}%
\AgdaGeneralizable{Γ}\AgdaSpace{}%
\AgdaGeneralizable{Δ}\AgdaSpace{}%
\AgdaSymbol{→}\AgdaSpace{}%
\AgdaGeneralizable{Γ}\AgdaSpace{}%
\AgdaOperator{\AgdaDatatype{⊆}}\AgdaSpace{}%
\AgdaGeneralizable{Ψ}\AgdaSpace{}%
\AgdaSymbol{→}\AgdaSpace{}%
\AgdaDatatype{Env}\AgdaSpace{}%
\AgdaGeneralizable{Ψ}\AgdaSpace{}%
\AgdaGeneralizable{Δ}\<%
\\
%
\>[2]\AgdaFunction{env-wk-zero}\AgdaSpace{}%
\AgdaBound{ρ}\AgdaSpace{}%
\AgdaInductiveConstructor{ε}\AgdaSpace{}%
\AgdaSymbol{=}\AgdaSpace{}%
\AgdaBound{ρ}\<%
\\
%
\>[2]\AgdaFunction{env-wk-zero}\AgdaSpace{}%
\AgdaBound{ρ}\AgdaSpace{}%
\AgdaSymbol{(}\AgdaInductiveConstructor{skip}\AgdaSpace{}%
\AgdaSymbol{\{}\AgdaArgument{is}\AgdaSpace{}%
\AgdaSymbol{=}\AgdaSpace{}%
\AgdaInductiveConstructor{ix}\AgdaSpace{}%
\AgdaBound{x}\AgdaSymbol{\}}\AgdaSpace{}%
\AgdaBound{w}\AgdaSymbol{)}\AgdaSpace{}%
\AgdaSymbol{=}\AgdaSpace{}%
\AgdaInductiveConstructor{skip}\AgdaSpace{}%
\AgdaSymbol{(}\AgdaFunction{env-wk-zero}\AgdaSpace{}%
\AgdaBound{ρ}\AgdaSpace{}%
\AgdaBound{w}\AgdaSymbol{)}\<%
\\
%
\>[2]\AgdaFunction{env-wk-zero}\AgdaSpace{}%
\AgdaBound{ρ}\AgdaSpace{}%
\AgdaSymbol{(}\AgdaInductiveConstructor{skip}\AgdaSpace{}%
\AgdaSymbol{\{}\AgdaArgument{is}\AgdaSpace{}%
\AgdaSymbol{=}\AgdaSpace{}%
\AgdaInductiveConstructor{ar}\AgdaSpace{}%
\AgdaBound{x}\AgdaSymbol{\}}\AgdaSpace{}%
\AgdaBound{w}\AgdaSymbol{)}\AgdaSpace{}%
\AgdaSymbol{=}\AgdaSpace{}%
\AgdaFunction{env-wk-zero}\AgdaSpace{}%
\AgdaBound{ρ}\AgdaSpace{}%
\AgdaBound{w}\AgdaSpace{}%
\AgdaOperator{\AgdaInductiveConstructor{▹}}\AgdaSpace{}%
\AgdaInductiveConstructor{zero}\<%
\\
%
\>[2]\AgdaFunction{env-wk-zero}\AgdaSpace{}%
\AgdaSymbol{(}\AgdaInductiveConstructor{skip}\AgdaSpace{}%
\AgdaBound{ρ}\AgdaSymbol{)}\AgdaSpace{}%
\AgdaSymbol{(}\AgdaInductiveConstructor{keep}\AgdaSpace{}%
\AgdaSymbol{\{}\AgdaArgument{is}\AgdaSpace{}%
\AgdaSymbol{=}\AgdaSpace{}%
\AgdaInductiveConstructor{ix}\AgdaSpace{}%
\AgdaBound{x}\AgdaSymbol{\}}\AgdaSpace{}%
\AgdaBound{w}\AgdaSymbol{)}\AgdaSpace{}%
\AgdaSymbol{=}\AgdaSpace{}%
\AgdaInductiveConstructor{skip}\AgdaSpace{}%
\AgdaSymbol{(}\AgdaFunction{env-wk-zero}\AgdaSpace{}%
\AgdaBound{ρ}\AgdaSpace{}%
\AgdaBound{w}\AgdaSymbol{)}\<%
\\
%
\>[2]\AgdaFunction{env-wk-zero}\AgdaSpace{}%
\AgdaSymbol{(}\AgdaBound{ρ}\AgdaSpace{}%
\AgdaOperator{\AgdaInductiveConstructor{▹}}\AgdaSpace{}%
\AgdaBound{x₁}\AgdaSymbol{)}\AgdaSpace{}%
\AgdaSymbol{(}\AgdaInductiveConstructor{keep}\AgdaSpace{}%
\AgdaSymbol{\{}\AgdaArgument{is}\AgdaSpace{}%
\AgdaSymbol{=}\AgdaSpace{}%
\AgdaInductiveConstructor{ar}\AgdaSpace{}%
\AgdaBound{x}\AgdaSymbol{\}}\AgdaSpace{}%
\AgdaBound{w}\AgdaSymbol{)}\AgdaSpace{}%
\AgdaSymbol{=}\AgdaSpace{}%
\AgdaFunction{env-wk-zero}\AgdaSpace{}%
\AgdaBound{ρ}\AgdaSpace{}%
\AgdaBound{w}\AgdaSpace{}%
\AgdaOperator{\AgdaInductiveConstructor{▹}}\AgdaSpace{}%
\AgdaBound{x₁}\<%
\\
%
\\[\AgdaEmptyExtraSkip]%
%
\>[2]\AgdaComment{--\ Insert\ zeroes\ in\ the\ environment\ EE\ according\ to\ the\ ⊆\ content}\<%
\\
%
\>[2]\AgdaFunction{ee-wk-zero}\AgdaSpace{}%
\AgdaSymbol{(}\AgdaInductiveConstructor{env}\AgdaSpace{}%
\AgdaBound{ρ}\AgdaSymbol{)}\AgdaSpace{}%
\AgdaBound{w}\AgdaSpace{}%
\AgdaSymbol{=}\AgdaSpace{}%
\AgdaInductiveConstructor{env}\AgdaSpace{}%
\AgdaSymbol{(}\AgdaFunction{env-wk-zero}\AgdaSpace{}%
\AgdaBound{ρ}\AgdaSpace{}%
\AgdaBound{w}\AgdaSymbol{)}\<%
\\
%
\>[2]\AgdaFunction{ee-wk-zero}\AgdaSpace{}%
\AgdaSymbol{(}\AgdaInductiveConstructor{let′}\AgdaSpace{}%
\AgdaBound{x}\AgdaSpace{}%
\AgdaBound{ρ}\AgdaSymbol{)}\AgdaSpace{}%
\AgdaBound{w}\AgdaSpace{}%
\AgdaSymbol{=}\AgdaSpace{}%
\AgdaInductiveConstructor{let′}\AgdaSpace{}%
\AgdaBound{x}\AgdaSpace{}%
\AgdaSymbol{(}\AgdaFunction{ee-wk-zero}\AgdaSpace{}%
\AgdaBound{ρ}\AgdaSpace{}%
\AgdaBound{w}\AgdaSymbol{)}\<%
\\
%
\\[\AgdaEmptyExtraSkip]%
%
\>[2]\AgdaComment{--\ Add\ zero\ to\ the\ end\ of\ EE\ (wrapper\ for\ ee-wk-zero)}\<%
\\
%
\>[2]\AgdaFunction{ee-push-zero}\AgdaSpace{}%
\AgdaSymbol{:}\AgdaSpace{}%
\AgdaDatatype{EE}\AgdaSpace{}%
\AgdaGeneralizable{Γ}\AgdaSpace{}%
\AgdaGeneralizable{Δ}\AgdaSpace{}%
\AgdaSymbol{→}\AgdaSpace{}%
\AgdaDatatype{EE}\AgdaSpace{}%
\AgdaSymbol{(}\AgdaGeneralizable{Γ}\AgdaSpace{}%
\AgdaOperator{\AgdaInductiveConstructor{▹}}\AgdaSpace{}%
\AgdaInductiveConstructor{ar}\AgdaSpace{}%
\AgdaGeneralizable{s}\AgdaSymbol{)}\AgdaSpace{}%
\AgdaGeneralizable{Δ}\<%
\\
%
\>[2]\AgdaFunction{ee-push-zero}\AgdaSpace{}%
\AgdaBound{ρ}\AgdaSpace{}%
\AgdaSymbol{=}\AgdaSpace{}%
\AgdaFunction{ee-wk-zero}\AgdaSpace{}%
\AgdaBound{ρ}\AgdaSpace{}%
\AgdaSymbol{(}\AgdaInductiveConstructor{skip}\AgdaSpace{}%
\AgdaFunction{⊆-eq}\AgdaSymbol{)}\<%
\\
%
\\[\AgdaEmptyExtraSkip]%
%
\>[2]\AgdaFunction{zero-env}\AgdaSpace{}%
\AgdaSymbol{:}\AgdaSpace{}%
\AgdaDatatype{Env}\AgdaSpace{}%
\AgdaGeneralizable{Γ}\AgdaSpace{}%
\AgdaGeneralizable{Δ}\<%
\\
%
\>[2]\AgdaFunction{zero-env}\AgdaSpace{}%
\AgdaSymbol{\{}\AgdaInductiveConstructor{ε}\AgdaSymbol{\}}\AgdaSpace{}%
\AgdaSymbol{=}\AgdaSpace{}%
\AgdaInductiveConstructor{ε}\<%
\\
%
\>[2]\AgdaFunction{zero-env}\AgdaSpace{}%
\AgdaSymbol{\{}\AgdaBound{Γ}\AgdaSpace{}%
\AgdaOperator{\AgdaInductiveConstructor{▹}}\AgdaSpace{}%
\AgdaInductiveConstructor{ix}\AgdaSpace{}%
\AgdaBound{x}\AgdaSymbol{\}}\AgdaSpace{}%
\AgdaSymbol{=}\AgdaSpace{}%
\AgdaInductiveConstructor{skip}\AgdaSpace{}%
\AgdaFunction{zero-env}\<%
\\
%
\>[2]\AgdaFunction{zero-env}\AgdaSpace{}%
\AgdaSymbol{\{}\AgdaBound{Γ}\AgdaSpace{}%
\AgdaOperator{\AgdaInductiveConstructor{▹}}\AgdaSpace{}%
\AgdaInductiveConstructor{ar}\AgdaSpace{}%
\AgdaBound{x}\AgdaSymbol{\}}\AgdaSpace{}%
\AgdaSymbol{=}\AgdaSpace{}%
\AgdaFunction{zero-env}\AgdaSpace{}%
\AgdaOperator{\AgdaInductiveConstructor{▹}}\AgdaSpace{}%
\AgdaInductiveConstructor{zero}\<%
\\
%
\\[\AgdaEmptyExtraSkip]%
%
\>[2]\AgdaFunction{zero-ee}\AgdaSpace{}%
\AgdaSymbol{=}\AgdaSpace{}%
\AgdaInductiveConstructor{env}\AgdaSpace{}%
\AgdaSymbol{(}\AgdaFunction{zero-env}\AgdaSymbol{)}\<%
\\
%
\\[\AgdaEmptyExtraSkip]%
%
\>[2]\AgdaFunction{env-update+}\AgdaSpace{}%
\AgdaSymbol{:}\AgdaSpace{}%
\AgdaDatatype{Env}\AgdaSpace{}%
\AgdaGeneralizable{Γ}\AgdaSpace{}%
\AgdaGeneralizable{Δ}\AgdaSpace{}%
\AgdaSymbol{→}\AgdaSpace{}%
\AgdaSymbol{(}\AgdaBound{v}\AgdaSpace{}%
\AgdaSymbol{:}\AgdaSpace{}%
\AgdaInductiveConstructor{ar}\AgdaSpace{}%
\AgdaGeneralizable{s}\AgdaSpace{}%
\AgdaOperator{\AgdaDatatype{∈}}\AgdaSpace{}%
\AgdaGeneralizable{Γ}\AgdaSymbol{)}\AgdaSpace{}%
\AgdaSymbol{→}\AgdaSpace{}%
\AgdaSymbol{(}\AgdaBound{t}\AgdaSpace{}%
\AgdaSymbol{:}\AgdaSpace{}%
\AgdaDatatype{E}\AgdaSpace{}%
\AgdaGeneralizable{Δ}\AgdaSpace{}%
\AgdaSymbol{(}\AgdaInductiveConstructor{ar}\AgdaSpace{}%
\AgdaGeneralizable{s}\AgdaSymbol{))}\AgdaSpace{}%
\AgdaSymbol{→}\AgdaSpace{}%
\AgdaDatatype{Env}\AgdaSpace{}%
\AgdaGeneralizable{Γ}\AgdaSpace{}%
\AgdaGeneralizable{Δ}\<%
\\
%
\>[2]\AgdaFunction{env-update+}\AgdaSpace{}%
\AgdaSymbol{(}\AgdaBound{ρ}\AgdaSpace{}%
\AgdaOperator{\AgdaInductiveConstructor{▹}}\AgdaSpace{}%
\AgdaBound{x}\AgdaSymbol{)}\AgdaSpace{}%
\AgdaInductiveConstructor{v₀}\AgdaSpace{}%
\AgdaBound{t}\AgdaSpace{}%
\AgdaSymbol{=}\AgdaSpace{}%
\AgdaBound{ρ}\AgdaSpace{}%
\AgdaOperator{\AgdaInductiveConstructor{▹}}\AgdaSpace{}%
\AgdaSymbol{(}\AgdaBound{x}\AgdaSpace{}%
\AgdaOperator{\AgdaInductiveConstructor{⊞}}\AgdaSpace{}%
\AgdaBound{t}\AgdaSymbol{)}\<%
\\
%
\>[2]\AgdaFunction{env-update+}\AgdaSpace{}%
\AgdaSymbol{(}\AgdaInductiveConstructor{skip}\AgdaSpace{}%
\AgdaBound{ρ}\AgdaSymbol{)}\AgdaSpace{}%
\AgdaSymbol{(}\AgdaInductiveConstructor{vₛ}\AgdaSpace{}%
\AgdaBound{v}\AgdaSymbol{)}\AgdaSpace{}%
\AgdaBound{t}\AgdaSpace{}%
\AgdaSymbol{=}\AgdaSpace{}%
\AgdaInductiveConstructor{skip}\AgdaSpace{}%
\AgdaSymbol{(}\AgdaFunction{env-update+}\AgdaSpace{}%
\AgdaBound{ρ}\AgdaSpace{}%
\AgdaBound{v}\AgdaSpace{}%
\AgdaBound{t}\AgdaSymbol{)}\<%
\\
%
\>[2]\AgdaFunction{env-update+}\AgdaSpace{}%
\AgdaSymbol{(}\AgdaBound{ρ}\AgdaSpace{}%
\AgdaOperator{\AgdaInductiveConstructor{▹}}\AgdaSpace{}%
\AgdaBound{x}\AgdaSymbol{)}\AgdaSpace{}%
\AgdaSymbol{(}\AgdaInductiveConstructor{vₛ}\AgdaSpace{}%
\AgdaBound{v}\AgdaSymbol{)}\AgdaSpace{}%
\AgdaBound{t}\AgdaSpace{}%
\AgdaSymbol{=}\AgdaSpace{}%
\AgdaFunction{env-update+}\AgdaSpace{}%
\AgdaBound{ρ}\AgdaSpace{}%
\AgdaBound{v}\AgdaSpace{}%
\AgdaBound{t}\AgdaSpace{}%
\AgdaOperator{\AgdaInductiveConstructor{▹}}\AgdaSpace{}%
\AgdaBound{x}\<%
\\
%
\\[\AgdaEmptyExtraSkip]%
%
\>[2]\AgdaFunction{ee-update+}\AgdaSpace{}%
\AgdaSymbol{(}\AgdaInductiveConstructor{env}\AgdaSpace{}%
\AgdaBound{ρ}\AgdaSymbol{)}\AgdaSpace{}%
\AgdaBound{v}\AgdaSpace{}%
\AgdaBound{t}\AgdaSpace{}%
\AgdaSymbol{=}\AgdaSpace{}%
\AgdaInductiveConstructor{env}\AgdaSpace{}%
\AgdaSymbol{(}\AgdaFunction{env-update+}\AgdaSpace{}%
\AgdaBound{ρ}\AgdaSpace{}%
\AgdaBound{v}\AgdaSpace{}%
\AgdaBound{t}\AgdaSymbol{)}\<%
\\
%
\>[2]\AgdaFunction{ee-update+}\AgdaSpace{}%
\AgdaSymbol{(}\AgdaInductiveConstructor{let′}\AgdaSpace{}%
\AgdaBound{x}\AgdaSpace{}%
\AgdaBound{ρ}\AgdaSymbol{)}\AgdaSpace{}%
\AgdaBound{v}\AgdaSpace{}%
\AgdaBound{t}\AgdaSpace{}%
\AgdaSymbol{=}\AgdaSpace{}%
\AgdaInductiveConstructor{let′}\AgdaSpace{}%
\AgdaBound{x}\AgdaSpace{}%
\AgdaSymbol{(}\AgdaFunction{ee-update+}\AgdaSpace{}%
\AgdaBound{ρ}\AgdaSpace{}%
\AgdaBound{v}\AgdaSpace{}%
\AgdaSymbol{(}\AgdaBound{t}\AgdaSpace{}%
\AgdaOperator{\AgdaFunction{↑}}\AgdaSymbol{))}\<%
\\
\>[0]\<%
\\
%
\>[2]\AgdaFunction{env-map-sum}\AgdaSpace{}%
\AgdaSymbol{:}\AgdaSpace{}%
\AgdaDatatype{Env}\AgdaSpace{}%
\AgdaGeneralizable{Γ}\AgdaSpace{}%
\AgdaSymbol{(}\AgdaGeneralizable{Δ}\AgdaSpace{}%
\AgdaOperator{\AgdaInductiveConstructor{▹}}\AgdaSpace{}%
\AgdaInductiveConstructor{ix}\AgdaSpace{}%
\AgdaGeneralizable{s}\AgdaSymbol{)}\AgdaSpace{}%
\AgdaSymbol{→}\AgdaSpace{}%
\AgdaDatatype{Env}\AgdaSpace{}%
\AgdaGeneralizable{Γ}\AgdaSpace{}%
\AgdaGeneralizable{Δ}\<%
\\
%
\>[2]\AgdaFunction{env-map-sum}\AgdaSpace{}%
\AgdaInductiveConstructor{ε}\AgdaSpace{}%
\AgdaSymbol{=}\AgdaSpace{}%
\AgdaInductiveConstructor{ε}\<%
\\
%
\>[2]\AgdaFunction{env-map-sum}\AgdaSpace{}%
\AgdaSymbol{(}\AgdaInductiveConstructor{skip}\AgdaSpace{}%
\AgdaBound{ρ}\AgdaSymbol{)}\AgdaSpace{}%
\AgdaSymbol{=}\AgdaSpace{}%
\AgdaInductiveConstructor{skip}\AgdaSpace{}%
\AgdaSymbol{(}\AgdaFunction{env-map-sum}\AgdaSpace{}%
\AgdaBound{ρ}\AgdaSymbol{)}\<%
\\
%
\>[2]\AgdaFunction{env-map-sum}\AgdaSpace{}%
\AgdaSymbol{(}\AgdaBound{ρ}\AgdaSpace{}%
\AgdaOperator{\AgdaInductiveConstructor{▹}}\AgdaSpace{}%
\AgdaBound{x}\AgdaSymbol{)}\AgdaSpace{}%
\AgdaSymbol{=}\AgdaSpace{}%
\AgdaFunction{env-map-sum}\AgdaSpace{}%
\AgdaBound{ρ}\AgdaSpace{}%
\AgdaOperator{\AgdaInductiveConstructor{▹}}\AgdaSpace{}%
\AgdaInductiveConstructor{E.sum}\AgdaSpace{}%
\AgdaBound{x}\<%
\\
%
\\[\AgdaEmptyExtraSkip]%
%
\>[2]\AgdaFunction{ee-fold}\AgdaSpace{}%
\AgdaSymbol{:}\AgdaSpace{}%
\AgdaDatatype{EE}\AgdaSpace{}%
\AgdaGeneralizable{Γ}\AgdaSpace{}%
\AgdaGeneralizable{Δ}\AgdaSpace{}%
\AgdaSymbol{→}\AgdaSpace{}%
\AgdaDatatype{Env}\AgdaSpace{}%
\AgdaGeneralizable{Γ}\AgdaSpace{}%
\AgdaGeneralizable{Δ}\<%
\\
%
\>[2]\AgdaFunction{ee-fold}\AgdaSpace{}%
\AgdaSymbol{(}\AgdaInductiveConstructor{env}\AgdaSpace{}%
\AgdaBound{x}\AgdaSymbol{)}\AgdaSpace{}%
\AgdaSymbol{=}\AgdaSpace{}%
\AgdaBound{x}\<%
\\
%
\>[2]\AgdaFunction{ee-fold}\AgdaSpace{}%
\AgdaSymbol{\{}\AgdaArgument{Δ}\AgdaSpace{}%
\AgdaSymbol{=}\AgdaSpace{}%
\AgdaBound{Δ}\AgdaSymbol{\}}\AgdaSpace{}%
\AgdaSymbol{(}\AgdaInductiveConstructor{let′}\AgdaSpace{}%
\AgdaSymbol{\{}\AgdaArgument{s}\AgdaSpace{}%
\AgdaSymbol{=}\AgdaSpace{}%
\AgdaBound{s}\AgdaSymbol{\}}\AgdaSpace{}%
\AgdaBound{x}\AgdaSpace{}%
\AgdaBound{ρ}\AgdaSymbol{)}\AgdaSpace{}%
\AgdaSymbol{=}\AgdaSpace{}%
\AgdaFunction{map-let}\AgdaSpace{}%
\AgdaSymbol{(}\AgdaFunction{ee-fold}\AgdaSpace{}%
\AgdaBound{ρ}\AgdaSymbol{)}\<%
\\
\>[2][@{}l@{\AgdaIndent{0}}]%
\>[4]\AgdaKeyword{where}%
\>[525I]\AgdaFunction{map-let}\AgdaSpace{}%
\AgdaSymbol{:}\AgdaSpace{}%
\AgdaSymbol{∀}\AgdaSpace{}%
\AgdaSymbol{\{}\AgdaBound{Γ}\AgdaSymbol{\}}\AgdaSpace{}%
\AgdaSymbol{→}\AgdaSpace{}%
\AgdaDatatype{Env}\AgdaSpace{}%
\AgdaBound{Γ}\AgdaSpace{}%
\AgdaSymbol{(}\AgdaBound{Δ}\AgdaSpace{}%
\AgdaOperator{\AgdaInductiveConstructor{▹}}\AgdaSpace{}%
\AgdaInductiveConstructor{ar}\AgdaSpace{}%
\AgdaBound{s}\AgdaSymbol{)}\AgdaSpace{}%
\AgdaSymbol{→}\AgdaSpace{}%
\AgdaDatatype{Env}\AgdaSpace{}%
\AgdaBound{Γ}\AgdaSpace{}%
\AgdaBound{Δ}\<%
\\
\>[.][@{}l@{}]\<[525I]%
\>[10]\AgdaFunction{map-let}\AgdaSpace{}%
\AgdaInductiveConstructor{ε}\AgdaSpace{}%
\AgdaSymbol{=}\AgdaSpace{}%
\AgdaInductiveConstructor{ε}\<%
\\
%
\>[10]\AgdaFunction{map-let}\AgdaSpace{}%
\AgdaSymbol{(}\AgdaInductiveConstructor{skip}\AgdaSpace{}%
\AgdaBound{ν}\AgdaSymbol{)}\AgdaSpace{}%
\AgdaSymbol{=}\AgdaSpace{}%
\AgdaInductiveConstructor{skip}\AgdaSpace{}%
\AgdaSymbol{(}\AgdaFunction{map-let}\AgdaSpace{}%
\AgdaBound{ν}\AgdaSymbol{)}\<%
\\
%
\>[10]\AgdaFunction{map-let}\AgdaSpace{}%
\AgdaSymbol{(}\AgdaBound{ν}\AgdaSpace{}%
\AgdaOperator{\AgdaInductiveConstructor{▹}}\AgdaSpace{}%
\AgdaBound{e}\AgdaSymbol{)}\AgdaSpace{}%
\AgdaSymbol{=}\AgdaSpace{}%
\AgdaFunction{map-let}\AgdaSpace{}%
\AgdaBound{ν}\AgdaSpace{}%
\AgdaOperator{\AgdaInductiveConstructor{▹}}\AgdaSpace{}%
\AgdaInductiveConstructor{let′}\AgdaSpace{}%
\AgdaBound{x}\AgdaSpace{}%
\AgdaBound{e}\<%
\\
%
\\[\AgdaEmptyExtraSkip]%
%
\>[2]\AgdaFunction{ee-map-sum}\AgdaSpace{}%
\AgdaBound{ρ}\AgdaSpace{}%
\AgdaSymbol{=}\AgdaSpace{}%
\AgdaInductiveConstructor{env}\AgdaSpace{}%
\AgdaSymbol{(}\AgdaFunction{env-map-sum}\AgdaSpace{}%
\AgdaSymbol{(}\AgdaFunction{ee-fold}\AgdaSpace{}%
\AgdaBound{ρ}\AgdaSymbol{))}\<%
\\
%
\\[\AgdaEmptyExtraSkip]%
%
\>[2]\AgdaFunction{env-plus}\AgdaSpace{}%
\AgdaSymbol{:}\AgdaSpace{}%
\AgdaSymbol{(}\AgdaBound{ρ}\AgdaSpace{}%
\AgdaBound{ν}\AgdaSpace{}%
\AgdaSymbol{:}\AgdaSpace{}%
\AgdaDatatype{Env}\AgdaSpace{}%
\AgdaGeneralizable{Γ}\AgdaSpace{}%
\AgdaGeneralizable{Δ}\AgdaSymbol{)}\AgdaSpace{}%
\AgdaSymbol{→}\AgdaSpace{}%
\AgdaDatatype{Env}\AgdaSpace{}%
\AgdaGeneralizable{Γ}\AgdaSpace{}%
\AgdaGeneralizable{Δ}\<%
\\
%
\>[2]\AgdaFunction{env-plus}\AgdaSpace{}%
\AgdaInductiveConstructor{ε}\AgdaSpace{}%
\AgdaBound{ν}\AgdaSpace{}%
\AgdaSymbol{=}\AgdaSpace{}%
\AgdaBound{ν}\<%
\\
%
\>[2]\AgdaFunction{env-plus}\AgdaSpace{}%
\AgdaSymbol{(}\AgdaInductiveConstructor{skip}\AgdaSpace{}%
\AgdaBound{ρ}\AgdaSymbol{)}\AgdaSpace{}%
\AgdaSymbol{(}\AgdaInductiveConstructor{skip}\AgdaSpace{}%
\AgdaBound{ν}\AgdaSymbol{)}\AgdaSpace{}%
\AgdaSymbol{=}\AgdaSpace{}%
\AgdaInductiveConstructor{skip}\AgdaSpace{}%
\AgdaSymbol{(}\AgdaFunction{env-plus}\AgdaSpace{}%
\AgdaBound{ρ}\AgdaSpace{}%
\AgdaBound{ν}\AgdaSymbol{)}\<%
\\
%
\>[2]\AgdaFunction{env-plus}\AgdaSpace{}%
\AgdaSymbol{(}\AgdaBound{ρ}\AgdaSpace{}%
\AgdaOperator{\AgdaInductiveConstructor{▹}}\AgdaSpace{}%
\AgdaBound{x}\AgdaSymbol{)}\AgdaSpace{}%
\AgdaSymbol{(}\AgdaBound{ν}\AgdaSpace{}%
\AgdaOperator{\AgdaInductiveConstructor{▹}}\AgdaSpace{}%
\AgdaBound{y}\AgdaSymbol{)}\AgdaSpace{}%
\AgdaSymbol{=}\AgdaSpace{}%
\AgdaFunction{env-plus}\AgdaSpace{}%
\AgdaBound{ρ}\AgdaSpace{}%
\AgdaBound{ν}\AgdaSpace{}%
\AgdaOperator{\AgdaInductiveConstructor{▹}}\AgdaSpace{}%
\AgdaSymbol{(}\AgdaBound{x}\AgdaSpace{}%
\AgdaOperator{\AgdaInductiveConstructor{⊞}}\AgdaSpace{}%
\AgdaBound{y}\AgdaSymbol{)}\<%
\\
%
\\[\AgdaEmptyExtraSkip]%
%
\>[2]\AgdaSymbol{\{-\#}\AgdaSpace{}%
\AgdaKeyword{TERMINATING}\AgdaSpace{}%
\AgdaSymbol{\#-\}}%
\>[23]\AgdaComment{--\ See\ GradTerm.agda\ file\ where\ this\ terminates}\<%
\\
%
\>[23]\AgdaComment{--\ here\ we\ simple\ present\ a\ more\ readable\ version}\<%
\\
%
\>[2]\AgdaFunction{ee-plus}\AgdaSpace{}%
\AgdaSymbol{(}\AgdaInductiveConstructor{env}\AgdaSpace{}%
\AgdaBound{ρ}\AgdaSymbol{)}\AgdaSpace{}%
\AgdaSymbol{(}\AgdaInductiveConstructor{env}\AgdaSpace{}%
\AgdaBound{ν}\AgdaSymbol{)}\AgdaSpace{}%
\AgdaSymbol{=}\AgdaSpace{}%
\AgdaInductiveConstructor{env}\AgdaSpace{}%
\AgdaSymbol{(}\AgdaFunction{env-plus}\AgdaSpace{}%
\AgdaBound{ρ}\AgdaSpace{}%
\AgdaBound{ν}\AgdaSymbol{)}\<%
\\
%
\>[2]\AgdaFunction{ee-plus}\AgdaSpace{}%
\AgdaSymbol{(}\AgdaInductiveConstructor{env}\AgdaSpace{}%
\AgdaBound{ρ}\AgdaSymbol{)}\AgdaSpace{}%
\AgdaSymbol{(}\AgdaInductiveConstructor{let′}\AgdaSpace{}%
\AgdaBound{x}\AgdaSpace{}%
\AgdaBound{ν}\AgdaSymbol{)}\AgdaSpace{}%
\AgdaSymbol{=}\AgdaSpace{}%
\AgdaInductiveConstructor{let′}\AgdaSpace{}%
\AgdaBound{x}\AgdaSpace{}%
\AgdaSymbol{(}\AgdaFunction{ee-plus}\AgdaSpace{}%
\AgdaSymbol{(}\AgdaFunction{ee-wk}\AgdaSpace{}%
\AgdaSymbol{(}\AgdaInductiveConstructor{skip}\AgdaSpace{}%
\AgdaFunction{⊆-eq}\AgdaSymbol{)}\AgdaSpace{}%
\AgdaSymbol{(}\AgdaInductiveConstructor{env}\AgdaSpace{}%
\AgdaBound{ρ}\AgdaSymbol{))}\AgdaSpace{}%
\AgdaBound{ν}\AgdaSymbol{)}\<%
\\
%
\>[2]\AgdaFunction{ee-plus}\AgdaSpace{}%
\AgdaSymbol{(}\AgdaInductiveConstructor{let′}\AgdaSpace{}%
\AgdaBound{x}\AgdaSpace{}%
\AgdaBound{ρ}\AgdaSymbol{)}\AgdaSpace{}%
\AgdaBound{ν}\AgdaSpace{}%
\AgdaSymbol{=}\AgdaSpace{}%
\AgdaInductiveConstructor{let′}\AgdaSpace{}%
\AgdaBound{x}\AgdaSpace{}%
\AgdaSymbol{(}\AgdaFunction{ee-plus}\AgdaSpace{}%
\AgdaBound{ρ}\AgdaSpace{}%
\AgdaSymbol{(}\AgdaFunction{ee-wk}\AgdaSpace{}%
\AgdaSymbol{(}\AgdaInductiveConstructor{skip}\AgdaSpace{}%
\AgdaFunction{⊆-eq}\AgdaSymbol{)}\AgdaSpace{}%
\AgdaBound{ν}\AgdaSymbol{))}\<%
\\
%
\\[\AgdaEmptyExtraSkip]%
%
\>[2]\AgdaBound{δ}\AgdaSpace{}%
\AgdaOperator{\AgdaFunction{▹𝟘}}\AgdaSpace{}%
\AgdaSymbol{=}\AgdaSpace{}%
\AgdaFunction{ee-push-zero}\AgdaSpace{}%
\AgdaOperator{\AgdaFunction{\$}}\AgdaSpace{}%
\AgdaFunction{ee-wk}\AgdaSpace{}%
\AgdaSymbol{(}\AgdaInductiveConstructor{skip}\AgdaSpace{}%
\AgdaFunction{⊆-eq}\AgdaSymbol{)}\AgdaSpace{}%
\AgdaBound{δ}\<%
\end{code}

\todo[inline]{Discuss whether it is inefficient to use env-zero in Σ∇}
We define\footnote{Agda does not recognise that the definition of \AF{∇} that
we give here terminates.  We fix this in the supplementary materials by choosing
an inductively decreasing invariant.  However we keep this definition in the
paper for readability.} the function \AF{∇} that takes an expression \AF{E}
and the seed (initially set to one) and we compute a function that
updates the environment of partial derivatives (initial environment is \AF{ee-zero}).
We use two helper functions: \AF{∇Σ} for summing up environments and \AF{∇}
for dealing with the derivative of let expressions.  The code is presented
below and the explanation of how it works follow.
\begin{code}[hide]%
%
\>[2]\AgdaSymbol{\{-\#}\AgdaSpace{}%
\AgdaKeyword{TERMINATING}\AgdaSpace{}%
\AgdaSymbol{\#-\}}%
\>[23]\AgdaComment{--\ See\ GradTerm.agda\ file\ where\ this\ terminates}\<%
\\
%
\>[23]\AgdaComment{--\ here\ we\ simply\ present\ a\ more\ readable\ version.}\<%
\end{code}
\begin{code}%
%
\>[2]\AgdaFunction{∇ₗ}\AgdaSpace{}%
\AgdaSymbol{:}\AgdaSpace{}%
\AgdaDatatype{E}\AgdaSpace{}%
\AgdaGeneralizable{Γ}\AgdaSpace{}%
\AgdaSymbol{(}\AgdaInductiveConstructor{ar}\AgdaSpace{}%
\AgdaGeneralizable{s}\AgdaSymbol{)}\AgdaSpace{}%
\AgdaSymbol{→}\AgdaSpace{}%
\AgdaDatatype{EE}\AgdaSpace{}%
\AgdaSymbol{(}\AgdaGeneralizable{Γ}\AgdaSpace{}%
\AgdaOperator{\AgdaInductiveConstructor{▹}}\AgdaSpace{}%
\AgdaInductiveConstructor{ar}\AgdaSpace{}%
\AgdaGeneralizable{s}\AgdaSymbol{)}\AgdaSpace{}%
\AgdaGeneralizable{Γ}\AgdaSpace{}%
\AgdaSymbol{→}\AgdaSpace{}%
\AgdaDatatype{EE}\AgdaSpace{}%
\AgdaGeneralizable{Γ}\AgdaSpace{}%
\AgdaGeneralizable{Γ}\<%
\\
%
\>[2]\AgdaFunction{∇Σ}\AgdaSpace{}%
\AgdaSymbol{:}\AgdaSpace{}%
\AgdaSymbol{(}\AgdaBound{e}\AgdaSpace{}%
\AgdaBound{s}\AgdaSpace{}%
\AgdaSymbol{:}\AgdaSpace{}%
\AgdaDatatype{E}\AgdaSpace{}%
\AgdaSymbol{(}\AgdaGeneralizable{Γ}\AgdaSpace{}%
\AgdaOperator{\AgdaInductiveConstructor{▹}}\AgdaSpace{}%
\AgdaInductiveConstructor{ix}\AgdaSpace{}%
\AgdaGeneralizable{s}\AgdaSymbol{)}\AgdaSpace{}%
\AgdaSymbol{(}\AgdaInductiveConstructor{ar}\AgdaSpace{}%
\AgdaGeneralizable{p}\AgdaSymbol{))}\AgdaSpace{}%
\AgdaSymbol{→}\AgdaSpace{}%
\AgdaDatatype{EE}\AgdaSpace{}%
\AgdaGeneralizable{Γ}\AgdaSpace{}%
\AgdaGeneralizable{Γ}\AgdaSpace{}%
\AgdaSymbol{→}\AgdaSpace{}%
\AgdaDatatype{EE}\AgdaSpace{}%
\AgdaGeneralizable{Γ}\AgdaSpace{}%
\AgdaGeneralizable{Γ}\<%
\\
%
\\[\AgdaEmptyExtraSkip]%
%
\>[2]\AgdaFunction{∇}\AgdaSpace{}%
\AgdaSymbol{:}\AgdaSpace{}%
\AgdaSymbol{(}\AgdaBound{e}\AgdaSpace{}%
\AgdaBound{s}\AgdaSpace{}%
\AgdaSymbol{:}\AgdaSpace{}%
\AgdaDatatype{E}\AgdaSpace{}%
\AgdaGeneralizable{Γ}\AgdaSpace{}%
\AgdaGeneralizable{is}\AgdaSymbol{)}\AgdaSpace{}%
\AgdaSymbol{→}\AgdaSpace{}%
\AgdaDatatype{EE}\AgdaSpace{}%
\AgdaGeneralizable{Γ}\AgdaSpace{}%
\AgdaGeneralizable{Γ}\AgdaSpace{}%
\AgdaSymbol{→}\AgdaSpace{}%
\AgdaDatatype{EE}\AgdaSpace{}%
\AgdaGeneralizable{Γ}\AgdaSpace{}%
\AgdaGeneralizable{Γ}\<%
\\
%
\>[2]\AgdaFunction{∇}\AgdaSpace{}%
\AgdaSymbol{\{}\AgdaArgument{is}\AgdaSpace{}%
\AgdaSymbol{=}\AgdaSpace{}%
\AgdaInductiveConstructor{ix}\AgdaSpace{}%
\AgdaSymbol{\AgdaUnderscore{}\}}\AgdaSpace{}%
\AgdaSymbol{(}\AgdaInductiveConstructor{var}\AgdaSpace{}%
\AgdaBound{x}\AgdaSymbol{)}%
\>[27]\AgdaBound{s}%
\>[31]\AgdaSymbol{=}\AgdaSpace{}%
\AgdaFunction{id}\<%
\\
%
\>[2]\AgdaFunction{∇}\AgdaSpace{}%
\AgdaSymbol{\{}\AgdaArgument{is}\AgdaSpace{}%
\AgdaSymbol{=}\AgdaSpace{}%
\AgdaInductiveConstructor{ar}\AgdaSpace{}%
\AgdaSymbol{\AgdaUnderscore{}\}}\AgdaSpace{}%
\AgdaSymbol{(}\AgdaInductiveConstructor{var}\AgdaSpace{}%
\AgdaBound{x}\AgdaSymbol{)}%
\>[27]\AgdaBound{s}%
\>[31]\AgdaSymbol{=}\AgdaSpace{}%
\AgdaSymbol{λ}\AgdaSpace{}%
\AgdaBound{δ}\AgdaSpace{}%
\AgdaSymbol{→}\AgdaSpace{}%
\AgdaFunction{ee-update+}\AgdaSpace{}%
\AgdaBound{δ}\AgdaSpace{}%
\AgdaBound{x}\AgdaSpace{}%
\AgdaBound{s}\<%
\\
%
\>[2]\AgdaFunction{∇}\AgdaSpace{}%
\AgdaInductiveConstructor{zero}%
\>[27]\AgdaBound{s}%
\>[31]\AgdaSymbol{=}\AgdaSpace{}%
\AgdaFunction{id}\<%
\\
%
\>[2]\AgdaFunction{∇}\AgdaSpace{}%
\AgdaInductiveConstructor{one}%
\>[27]\AgdaBound{s}%
\>[31]\AgdaSymbol{=}\AgdaSpace{}%
\AgdaFunction{id}\<%
\\
%
\\[\AgdaEmptyExtraSkip]%
%
\>[2]\AgdaFunction{∇}\AgdaSpace{}%
\AgdaSymbol{(}\AgdaInductiveConstructor{imaps}\AgdaSpace{}%
\AgdaBound{e}\AgdaSymbol{)}%
\>[27]\AgdaBound{s}%
\>[31]\AgdaSymbol{=}\AgdaSpace{}%
\AgdaFunction{∇Σ}\AgdaSpace{}%
\AgdaBound{e}\AgdaSpace{}%
\AgdaSymbol{(}\AgdaInductiveConstructor{sels}%
\>[48]\AgdaSymbol{(}\AgdaBound{s}\AgdaSpace{}%
\AgdaOperator{\AgdaFunction{↑}}\AgdaSymbol{)}\AgdaSpace{}%
\AgdaSymbol{(}\AgdaInductiveConstructor{var}\AgdaSpace{}%
\AgdaInductiveConstructor{v₀}\AgdaSymbol{))}\<%
\\
%
\>[2]\AgdaFunction{∇}\AgdaSpace{}%
\AgdaSymbol{(}\AgdaInductiveConstructor{imap}\AgdaSpace{}%
\AgdaBound{e}\AgdaSymbol{)}%
\>[27]\AgdaBound{s}%
\>[31]\AgdaSymbol{=}\AgdaSpace{}%
\AgdaFunction{∇Σ}\AgdaSpace{}%
\AgdaBound{e}\AgdaSpace{}%
\AgdaSymbol{(}\AgdaInductiveConstructor{sel}%
\>[48]\AgdaSymbol{(}\AgdaBound{s}\AgdaSpace{}%
\AgdaOperator{\AgdaFunction{↑}}\AgdaSymbol{)}\AgdaSpace{}%
\AgdaSymbol{(}\AgdaInductiveConstructor{var}\AgdaSpace{}%
\AgdaInductiveConstructor{v₀}\AgdaSymbol{))}\<%
\\
%
\>[2]\AgdaFunction{∇}\AgdaSpace{}%
\AgdaSymbol{(}\AgdaInductiveConstructor{E.imapb}\AgdaSpace{}%
\AgdaBound{m}\AgdaSpace{}%
\AgdaBound{e}\AgdaSymbol{)}%
\>[27]\AgdaBound{s}%
\>[31]\AgdaSymbol{=}\AgdaSpace{}%
\AgdaFunction{∇Σ}\AgdaSpace{}%
\AgdaBound{e}\AgdaSpace{}%
\AgdaSymbol{(}\AgdaInductiveConstructor{E.selb}\AgdaSpace{}%
\AgdaBound{m}\AgdaSpace{}%
\AgdaSymbol{(}\AgdaBound{s}\AgdaSpace{}%
\AgdaOperator{\AgdaFunction{↑}}\AgdaSymbol{)}\AgdaSpace{}%
\AgdaSymbol{(}\AgdaInductiveConstructor{var}\AgdaSpace{}%
\AgdaInductiveConstructor{v₀}\AgdaSymbol{))}\<%
\\
%
\\[\AgdaEmptyExtraSkip]%
%
\>[2]\AgdaFunction{∇}\AgdaSpace{}%
\AgdaSymbol{(}\AgdaInductiveConstructor{sels}\AgdaSpace{}%
\AgdaBound{e}\AgdaSpace{}%
\AgdaBound{i}\AgdaSymbol{)}%
\>[27]\AgdaBound{s}%
\>[31]\AgdaSymbol{=}\AgdaSpace{}%
\AgdaFunction{∇}\AgdaSpace{}%
\AgdaBound{e}\AgdaSpace{}%
\AgdaSymbol{(}\AgdaInductiveConstructor{imaps}%
\>[48]\AgdaSymbol{(}\AgdaInductiveConstructor{zero-but}\AgdaSpace{}%
\AgdaSymbol{(}\AgdaInductiveConstructor{var}\AgdaSpace{}%
\AgdaInductiveConstructor{v₀}\AgdaSymbol{)}\AgdaSpace{}%
\AgdaSymbol{(}\AgdaBound{i}\AgdaSpace{}%
\AgdaOperator{\AgdaFunction{↑}}\AgdaSymbol{)}\AgdaSpace{}%
\AgdaSymbol{(}\AgdaBound{s}\AgdaSpace{}%
\AgdaOperator{\AgdaFunction{↑}}\AgdaSymbol{)))}\<%
\\
%
\>[2]\AgdaFunction{∇}\AgdaSpace{}%
\AgdaSymbol{(}\AgdaInductiveConstructor{sel}\AgdaSpace{}%
\AgdaBound{e}\AgdaSpace{}%
\AgdaBound{i}\AgdaSymbol{)}%
\>[27]\AgdaBound{s}%
\>[31]\AgdaSymbol{=}\AgdaSpace{}%
\AgdaFunction{∇}\AgdaSpace{}%
\AgdaBound{e}\AgdaSpace{}%
\AgdaSymbol{(}\AgdaInductiveConstructor{imap}%
\>[48]\AgdaSymbol{(}\AgdaInductiveConstructor{zero-but}\AgdaSpace{}%
\AgdaSymbol{(}\AgdaInductiveConstructor{var}\AgdaSpace{}%
\AgdaInductiveConstructor{v₀}\AgdaSymbol{)}\AgdaSpace{}%
\AgdaSymbol{(}\AgdaBound{i}\AgdaSpace{}%
\AgdaOperator{\AgdaFunction{↑}}\AgdaSymbol{)}\AgdaSpace{}%
\AgdaSymbol{(}\AgdaBound{s}\AgdaSpace{}%
\AgdaOperator{\AgdaFunction{↑}}\AgdaSymbol{)))}\<%
\\
%
\>[2]\AgdaFunction{∇}\AgdaSpace{}%
\AgdaSymbol{(}\AgdaInductiveConstructor{E.selb}\AgdaSpace{}%
\AgdaBound{m}\AgdaSpace{}%
\AgdaBound{e}\AgdaSpace{}%
\AgdaBound{i}\AgdaSymbol{)}%
\>[27]\AgdaBound{s}%
\>[31]\AgdaSymbol{=}\AgdaSpace{}%
\AgdaFunction{∇}\AgdaSpace{}%
\AgdaBound{e}\AgdaSpace{}%
\AgdaSymbol{(}\AgdaInductiveConstructor{E.imapb}\AgdaSpace{}%
\AgdaBound{m}\AgdaSpace{}%
\AgdaSymbol{(}\AgdaInductiveConstructor{zero-but}\AgdaSpace{}%
\AgdaSymbol{(}\AgdaInductiveConstructor{var}\AgdaSpace{}%
\AgdaInductiveConstructor{v₀}\AgdaSymbol{)}\AgdaSpace{}%
\AgdaSymbol{(}\AgdaBound{i}\AgdaSpace{}%
\AgdaOperator{\AgdaFunction{↑}}\AgdaSymbol{)}\AgdaSpace{}%
\AgdaSymbol{(}\AgdaBound{s}\AgdaSpace{}%
\AgdaOperator{\AgdaFunction{↑}}\AgdaSymbol{)))}\<%
\\
%
\\[\AgdaEmptyExtraSkip]%
%
\>[2]\AgdaFunction{∇}\AgdaSpace{}%
\AgdaSymbol{(}\AgdaInductiveConstructor{E.sum}\AgdaSpace{}%
\AgdaBound{e}\AgdaSymbol{)}%
\>[27]\AgdaBound{s}%
\>[31]\AgdaSymbol{=}\AgdaSpace{}%
\AgdaFunction{∇Σ}\AgdaSpace{}%
\AgdaBound{e}\AgdaSpace{}%
\AgdaSymbol{(}\AgdaBound{s}\AgdaSpace{}%
\AgdaOperator{\AgdaFunction{↑}}\AgdaSymbol{)}\<%
\\
%
\>[2]\AgdaFunction{∇}\AgdaSpace{}%
\AgdaSymbol{(}\AgdaInductiveConstructor{zero-but}\AgdaSpace{}%
\AgdaBound{i}\AgdaSpace{}%
\AgdaBound{j}\AgdaSpace{}%
\AgdaBound{e}\AgdaSymbol{)}%
\>[27]\AgdaBound{s}%
\>[31]\AgdaSymbol{=}\AgdaSpace{}%
\AgdaFunction{∇}\AgdaSpace{}%
\AgdaBound{e}\AgdaSpace{}%
\AgdaSymbol{(}\AgdaInductiveConstructor{zero-but}\AgdaSpace{}%
\AgdaBound{i}\AgdaSpace{}%
\AgdaBound{j}\AgdaSpace{}%
\AgdaBound{s}\AgdaSymbol{)}\<%
\\
%
\\[\AgdaEmptyExtraSkip]%
%
\>[2]\AgdaFunction{∇}\AgdaSpace{}%
\AgdaSymbol{(}\AgdaInductiveConstructor{E.slide}\AgdaSpace{}%
\AgdaBound{i}\AgdaSpace{}%
\AgdaBound{p}\AgdaSpace{}%
\AgdaBound{e}\AgdaSpace{}%
\AgdaBound{su}\AgdaSymbol{)}%
\>[27]\AgdaBound{s}%
\>[31]\AgdaSymbol{=}\AgdaSpace{}%
\AgdaFunction{∇}\AgdaSpace{}%
\AgdaBound{e}\AgdaSpace{}%
\AgdaSymbol{(}\AgdaInductiveConstructor{E.backslide}\AgdaSpace{}%
\AgdaBound{i}\AgdaSpace{}%
\AgdaBound{s}\AgdaSpace{}%
\AgdaBound{su}\AgdaSpace{}%
\AgdaBound{p}\AgdaSymbol{)}\<%
\\
%
\>[2]\AgdaFunction{∇}\AgdaSpace{}%
\AgdaSymbol{(}\AgdaInductiveConstructor{E.backslide}\AgdaSpace{}%
\AgdaBound{i}\AgdaSpace{}%
\AgdaBound{e}\AgdaSpace{}%
\AgdaBound{su}\AgdaSpace{}%
\AgdaBound{p}\AgdaSymbol{)}\AgdaSpace{}%
\AgdaBound{s}%
\>[31]\AgdaSymbol{=}\AgdaSpace{}%
\AgdaFunction{∇}\AgdaSpace{}%
\AgdaBound{e}\AgdaSpace{}%
\AgdaSymbol{(}\AgdaInductiveConstructor{E.slide}\AgdaSpace{}%
\AgdaBound{i}\AgdaSpace{}%
\AgdaBound{p}\AgdaSpace{}%
\AgdaBound{s}\AgdaSpace{}%
\AgdaBound{su}\AgdaSymbol{)}\<%
\\
%
\\[\AgdaEmptyExtraSkip]%
%
\>[2]\AgdaFunction{∇}\AgdaSpace{}%
\AgdaSymbol{(}\AgdaBound{e}\AgdaSpace{}%
\AgdaOperator{\AgdaInductiveConstructor{⊞}}\AgdaSpace{}%
\AgdaBound{e₁}\AgdaSymbol{)}%
\>[27]\AgdaBound{s}%
\>[31]\AgdaSymbol{=}\AgdaSpace{}%
\AgdaFunction{∇}\AgdaSpace{}%
\AgdaBound{e}\AgdaSpace{}%
\AgdaBound{s}\AgdaSpace{}%
\AgdaOperator{\AgdaFunction{∘}}\AgdaSpace{}%
\AgdaFunction{∇}\AgdaSpace{}%
\AgdaBound{e₁}\AgdaSpace{}%
\AgdaBound{s}\<%
\\
%
\>[2]\AgdaFunction{∇}\AgdaSpace{}%
\AgdaSymbol{(}\AgdaBound{e}\AgdaSpace{}%
\AgdaOperator{\AgdaInductiveConstructor{⊠}}\AgdaSpace{}%
\AgdaBound{e₁}\AgdaSymbol{)}%
\>[27]\AgdaBound{s}%
\>[31]\AgdaSymbol{=}\AgdaSpace{}%
\AgdaFunction{∇}\AgdaSpace{}%
\AgdaBound{e}\AgdaSpace{}%
\AgdaSymbol{(}\AgdaBound{s}\AgdaSpace{}%
\AgdaOperator{\AgdaInductiveConstructor{⊠}}\AgdaSpace{}%
\AgdaBound{e₁}\AgdaSymbol{)}\AgdaSpace{}%
\AgdaOperator{\AgdaFunction{∘}}\AgdaSpace{}%
\AgdaFunction{∇}\AgdaSpace{}%
\AgdaBound{e₁}\AgdaSpace{}%
\AgdaSymbol{(}\AgdaBound{s}\AgdaSpace{}%
\AgdaOperator{\AgdaInductiveConstructor{⊠}}\AgdaSpace{}%
\AgdaBound{e}\AgdaSymbol{)}\<%
\\
%
\>[2]\AgdaFunction{∇}\AgdaSpace{}%
\AgdaSymbol{(}\AgdaInductiveConstructor{scaledown}\AgdaSpace{}%
\AgdaBound{x}\AgdaSpace{}%
\AgdaBound{e}\AgdaSymbol{)}%
\>[27]\AgdaBound{s}%
\>[31]\AgdaSymbol{=}\AgdaSpace{}%
\AgdaFunction{∇}\AgdaSpace{}%
\AgdaBound{e}\AgdaSpace{}%
\AgdaSymbol{(}\AgdaInductiveConstructor{scaledown}\AgdaSpace{}%
\AgdaBound{x}\AgdaSpace{}%
\AgdaBound{s}\AgdaSymbol{)}\<%
\\
%
\>[2]\AgdaFunction{∇}\AgdaSpace{}%
\AgdaSymbol{(}\AgdaInductiveConstructor{minus}\AgdaSpace{}%
\AgdaBound{e}\AgdaSymbol{)}%
\>[27]\AgdaBound{s}%
\>[31]\AgdaSymbol{=}\AgdaSpace{}%
\AgdaFunction{∇}\AgdaSpace{}%
\AgdaBound{e}\AgdaSpace{}%
\AgdaSymbol{(}\AgdaInductiveConstructor{minus}\AgdaSpace{}%
\AgdaBound{s}\AgdaSymbol{)}\<%
\\
%
\>[2]\AgdaFunction{∇}\AgdaSpace{}%
\AgdaSymbol{(}\AgdaInductiveConstructor{logistic}\AgdaSpace{}%
\AgdaBound{e}\AgdaSymbol{)}%
\>[27]\AgdaBound{s}%
\>[31]\AgdaSymbol{=}\AgdaSpace{}%
\AgdaFunction{∇}\AgdaSpace{}%
\AgdaBound{e}\AgdaSpace{}%
\AgdaSymbol{(}\AgdaInductiveConstructor{let′}\AgdaSpace{}%
\AgdaSymbol{(}\AgdaInductiveConstructor{logistic}\AgdaSpace{}%
\AgdaBound{e}\AgdaSymbol{)}\AgdaSpace{}%
\AgdaSymbol{((}\AgdaBound{s}\AgdaSpace{}%
\AgdaOperator{\AgdaFunction{↑}}\AgdaSymbol{)}\AgdaSpace{}%
\AgdaOperator{\AgdaInductiveConstructor{⊠}}\AgdaSpace{}%
\AgdaInductiveConstructor{var}\AgdaSpace{}%
\AgdaInductiveConstructor{v₀}\AgdaSpace{}%
\AgdaOperator{\AgdaInductiveConstructor{⊠}}\AgdaSpace{}%
\AgdaSymbol{(}\AgdaInductiveConstructor{one}\AgdaSpace{}%
\AgdaOperator{\AgdaInductiveConstructor{⊞}}\AgdaSpace{}%
\AgdaInductiveConstructor{minus}\AgdaSpace{}%
\AgdaSymbol{(}\AgdaInductiveConstructor{var}\AgdaSpace{}%
\AgdaInductiveConstructor{v₀}\AgdaSymbol{))))}\<%
\\
\>[0]\<%
\\
%
\>[2]\AgdaFunction{∇}\AgdaSpace{}%
\AgdaSymbol{(}\AgdaInductiveConstructor{let′}\AgdaSpace{}%
\AgdaBound{e}\AgdaSpace{}%
\AgdaBound{e₁}\AgdaSymbol{)}%
\>[27]\AgdaBound{s}%
\>[31]\AgdaSymbol{=}\AgdaSpace{}%
\AgdaSymbol{λ}\AgdaSpace{}%
\AgdaBound{δ}\AgdaSpace{}%
\AgdaSymbol{→}\AgdaSpace{}%
\AgdaFunction{∇ₗ}\AgdaSpace{}%
\AgdaBound{e}\AgdaSpace{}%
\AgdaSymbol{(}\AgdaInductiveConstructor{let′}\AgdaSpace{}%
\AgdaBound{e}\AgdaSpace{}%
\AgdaSymbol{(}\AgdaFunction{∇}\AgdaSpace{}%
\AgdaBound{e₁}\AgdaSpace{}%
\AgdaSymbol{(}\AgdaBound{s}\AgdaSpace{}%
\AgdaOperator{\AgdaFunction{↑}}\AgdaSymbol{)}\AgdaSpace{}%
\AgdaSymbol{(}\AgdaBound{δ}\AgdaSpace{}%
\AgdaOperator{\AgdaFunction{▹𝟘}}\AgdaSymbol{)))}\<%
\\
%
\\[\AgdaEmptyExtraSkip]%
%
\>[2]\AgdaFunction{∇Σ}\AgdaSpace{}%
\AgdaBound{e}\AgdaSpace{}%
\AgdaBound{s}\AgdaSpace{}%
\AgdaBound{δ}\AgdaSpace{}%
\AgdaSymbol{=}\AgdaSpace{}%
\AgdaFunction{ee-plus}\AgdaSpace{}%
\AgdaBound{δ}\AgdaSpace{}%
\AgdaOperator{\AgdaFunction{\$}}\AgdaSpace{}%
\AgdaFunction{ee-tail}\AgdaSpace{}%
\AgdaOperator{\AgdaFunction{\$}}\AgdaSpace{}%
\AgdaFunction{ee-map-sum}\AgdaSpace{}%
\AgdaSymbol{(}\AgdaFunction{∇}\AgdaSpace{}%
\AgdaBound{e}\AgdaSpace{}%
\AgdaBound{s}\AgdaSpace{}%
\AgdaFunction{zero-ee}\AgdaSymbol{)}\<%
\\
%
\\[\AgdaEmptyExtraSkip]%
%
\>[2]\AgdaFunction{∇ₗ}\AgdaSpace{}%
\AgdaBound{e}\AgdaSpace{}%
\AgdaSymbol{(}\AgdaInductiveConstructor{env}\AgdaSpace{}%
\AgdaSymbol{(}\AgdaBound{ρ}\AgdaSpace{}%
\AgdaOperator{\AgdaInductiveConstructor{▹}}\AgdaSpace{}%
\AgdaBound{x}\AgdaSymbol{))}%
\>[22]\AgdaSymbol{=}\AgdaSpace{}%
\AgdaFunction{ee-tail}\AgdaSpace{}%
\AgdaOperator{\AgdaFunction{\$}}\AgdaSpace{}%
\AgdaInductiveConstructor{let′}\AgdaSpace{}%
\AgdaBound{x}\AgdaSpace{}%
\AgdaSymbol{(}\AgdaFunction{∇}\AgdaSpace{}%
\AgdaSymbol{(}\AgdaBound{e}\AgdaSpace{}%
\AgdaOperator{\AgdaFunction{↑}}\AgdaSymbol{)}\AgdaSpace{}%
\AgdaSymbol{(}\AgdaInductiveConstructor{var}\AgdaSpace{}%
\AgdaInductiveConstructor{v₀}\AgdaSymbol{)}\AgdaSpace{}%
\AgdaSymbol{(}\AgdaInductiveConstructor{env}\AgdaSpace{}%
\AgdaBound{ρ}\AgdaSpace{}%
\AgdaOperator{\AgdaFunction{▹𝟘}}\AgdaSymbol{))}\<%
\\
%
\>[2]\AgdaFunction{∇ₗ}\AgdaSpace{}%
\AgdaBound{e}\AgdaSpace{}%
\AgdaSymbol{(}\AgdaInductiveConstructor{let′}\AgdaSpace{}%
\AgdaBound{x}\AgdaSpace{}%
\AgdaBound{ρ}\AgdaSymbol{)}%
\>[22]\AgdaSymbol{=}\AgdaSpace{}%
\AgdaInductiveConstructor{let′}\AgdaSpace{}%
\AgdaBound{x}\AgdaSpace{}%
\AgdaSymbol{(}\AgdaFunction{ee-tail}\AgdaSpace{}%
\AgdaOperator{\AgdaFunction{\$}}\AgdaSpace{}%
\AgdaFunction{∇ₗ}\AgdaSpace{}%
\AgdaSymbol{(}\AgdaBound{e}\AgdaSpace{}%
\AgdaOperator{\AgdaFunction{↑}}\AgdaSymbol{)}\AgdaSpace{}%
\AgdaSymbol{(}\AgdaFunction{ee-wk-zero}\AgdaSpace{}%
\AgdaBound{ρ}\AgdaSpace{}%
\AgdaSymbol{(}\AgdaInductiveConstructor{keep}\AgdaSpace{}%
\AgdaSymbol{(}\AgdaInductiveConstructor{skip}\AgdaSpace{}%
\AgdaFunction{⊆-eq}\AgdaSymbol{))))}\<%
\end{code}
Derivative of constants (\AC{zero} and \AC{one})
is zero, so nothing needs to be updated in the environment.  Index variables are
not stored in the environment, so no updates are needed either.  If we differentiate
the variable $x$ with some seed \AB{s}, we update the $x$-th position in the environment
by adding \AB{s} to it.

Differentiation of \AC{imap}s proceeds as follows: if we inline \AF{∇Σ}, we
recursively apply \AF{∇} to $e$ 
with the element of the original seed \AB{s} selected at the top variable in
the \AF{ee-zero} environment.  Recall that $s$ is an array of the shape that
corresponds to the result of \AC{imap} computation, and $e$ is defined in
the context of shape (\AB{Γ} \AC{▹} (\AC{ix} \AB{p})).  Then we apply \AF{sum}
to all the elements of the resulting environment (\ie{} we sum-up all the
changes to the environment caused by computations of the individual array
elements).  Finally we add the current environment \AB{δ} to the result.
Note that the use of \AF{ee-tail} in \AF{∇Σ} is only needed to remove the
trivial \AC{ix} element from the environment.  This strategy applies to all
three \AF{imap}s, we just use the right kind of selection into the seed $s$.

When differentiating selections we recurse on the array we are
selecting from with the seed that is zero everywhere except the index we were
selecting at.  Differentiating
conditionals is straight-forward, as $i$ and $j$ must be in the context, we can
simply differentiate $e$ with the condition on the seed.  If indices were equal, we will
compute the update, otherwise we will differentiate with seed \AC{zero} which
has no effect.  As we are operating in a total language, there is no need to worry
about pulling the expression out of conditional.

The argument of \AC{sum}
lives is in the extended context, so we apply the same rules as for the \AC{imap} family,
except we propagate the original seed to all the summands.  Addition and multiplication
rules are straight-forward application of rules of symbolic differentiation.
The \AC{slide}/\AC{backslide}
pair forms a satisfying \AF{∇}-symmetry.  Finally, \AC{scaledown}, \AC{minus} and
\AC{logistic} follow the rules of differentiation.
\todo[inline]{reviewer 3 (2024): I was surprised that the backpropagation of
imaps was defined through selections, is there no way to define it as a
whole-array operation with less temporary values (i.e. how does it compare to
other definitions of AD on array languages)? if this is unnecessary due to the
later optimizations, it might be useful to mention this briefly at this stage.

I was surprised to see that AD is defined with backpropagation on a context,
rather than adjoint functions. Does this mean that the AD process needs to know
the global context and is not compositional (can the results from AD on
multiple sub-expressions or functions be easily combined together), or am I
mistaken?
}

\paragraph{Let expressions} The rules for the let case look complicated due to
encoding, but it is easy to understand what is going on from the following example.
Consider an expression in one variable $a$ that binds a local variable $x$ to $a^2$
and the initial environment for $\partial a$ that is set to $0$:
\[ 
   \AF{∇}\ (\AC{let}\ x = a^2\ \AC{in}\ (a + a)x)\ 1\ \langle 0 \rangle
\]
The first step is to apply \AF{∇} to the body of the let.  This
requires extending our environment with an element for $\partial x$,
computing:
\[ 
   \AF{∇}\ ((a + a)x)\ 1\ (\AC{let}\ x = a^2\ \AC{in}\ \langle 0, 0 \rangle)
   = \AC{let}\ x = a^2\ \AC{in}\ \langle x+x, a+a \rangle
\]
in the environment that preserves the $x$-bound expression.  This exactly
the call we do in the \AC{let′} case of \AF{∇}.
The next step is to apply the chain rule, computing the derivative of the
$x$-bound expression using the result of the previous computation as seed:
\[ 
   \AF{∇}\ a^2\ (a+a)\ (\AC{let}\ x = a^2\ \AC{in}\ \langle x+x \rangle)
   = \AC{let}\ x = a^2\ \AC{in}\ \langle x+x + a(a+a) + a(a+a) \rangle
\]
which gives the expected result $6a^2$ (we were differentiating $2a^3$ written
in a funny way).  However, direct use of $(a+a)$ as a seed in the last step
inlines the computation of the $(a+a)$ expression.  Instead, we can share 
this computation by defining a new let-binding and rearranging the call to
\AF{∇} as follows:
\[
   \AC{let}\ x = a^2\ \AC{in}\ 
   \AC{let}\ y = a+a\ \AC{in}\ 
   (\AF{∇}\ a^2\ y\ \langle x \rangle)
   = 
   \AC{let}\ x = a^2\ \AC{in}\ 
   \AC{let}\ y = a+a\ \AC{in}\
   \langle x+x + ya + ya \rangle
\]
this is exactly what \AF{∇ₗ} is doing --- traverse under the chain let
and share the seed by introducing a new variable.

 
 
 
 
 
\subsection{Optimisation\label{sec:opt}}
\todo[inline]{reviewer 3 (2024): optimizations are performed by normalization
(assuming that all peep-hole optimizations are always beneficial), however ML
frameworks typically explore optimization spaces with complex trade-offs (e.g.
sometimes not fusing a computation and keeping a temporary array may be a
better choice), have you considered how to deal with this difficulty? clarify
this}
Our rule-based AD algorithm from the previous section guarantees that its
output preserves correct shapes and that all the variables are well-scoped.
However, direct compilation of the AD-generated expressions may be
computationally inefficient.
While we can hope that the backend will take care of this, it is relatively
easy to implement a number of rewriting rules prior that will be applied
during prior compilation.  The motivation is two-fold: (i) designing
optimisations for a small DSL is much easier than for a general-purpose
language; (ii) as we have semantics of \AF{E}, we can make sure that our
optimisations are correct (semantics-preserving).
We are only going to demonstrate here the general setting, please refer
to supplementary materials for further details.

Firstly, as our semantics is defined on abstract reals, we require some
properties of their properties to prove semantics preservation.
For the optimisations that we implement,
we only need neutrality of addition and multiplication:
\begin{code}[hide]%
\>[0]\AgdaKeyword{module}\AgdaSpace{}%
\AgdaModule{Opt}\AgdaSpace{}%
\AgdaKeyword{where}\<%
\\
\>[0][@{}l@{\AgdaIndent{0}}]%
\>[2]\AgdaKeyword{open}\AgdaSpace{}%
\AgdaKeyword{import}\AgdaSpace{}%
\AgdaModule{Data.Nat}\AgdaSpace{}%
\AgdaSymbol{as}\AgdaSpace{}%
\AgdaModule{ℕ}\AgdaSpace{}%
\AgdaKeyword{using}\AgdaSpace{}%
\AgdaSymbol{(}\AgdaDatatype{ℕ}\AgdaSymbol{;}\AgdaSpace{}%
\AgdaInductiveConstructor{zero}\AgdaSymbol{;}\AgdaSpace{}%
\AgdaInductiveConstructor{suc}\AgdaSymbol{)}\<%
\\
%
\>[2]\AgdaKeyword{open}\AgdaSpace{}%
\AgdaKeyword{import}\AgdaSpace{}%
\AgdaModule{Data.Product}\<%
\\
%
\>[2]\AgdaKeyword{open}\AgdaSpace{}%
\AgdaModule{Lang}\<%
\\
%
\>[2]\AgdaKeyword{open}\AgdaSpace{}%
\AgdaModule{WkSub}\<%
\end{code}
\begin{code}%
%
\>[2]\AgdaKeyword{record}\AgdaSpace{}%
\AgdaRecord{RealProp}\AgdaSpace{}%
\AgdaSymbol{(}\AgdaBound{r}\AgdaSpace{}%
\AgdaSymbol{:}\AgdaSpace{}%
\AgdaRecord{Real}\AgdaSymbol{)}\AgdaSpace{}%
\AgdaSymbol{:}\AgdaSpace{}%
\AgdaPrimitive{Set}\AgdaSpace{}%
\AgdaKeyword{where}\<%
\\
\>[2][@{}l@{\AgdaIndent{0}}]%
\>[4]\AgdaKeyword{open}\AgdaSpace{}%
\AgdaModule{Real}\AgdaSpace{}%
\AgdaBound{r}\AgdaSymbol{;}\AgdaSpace{}%
\AgdaKeyword{field}\<%
\\
\>[4][@{}l@{\AgdaIndent{0}}]%
\>[6]\AgdaField{+-neutˡ}\AgdaSpace{}%
\AgdaSymbol{:}\AgdaSpace{}%
\AgdaSymbol{∀}\AgdaSpace{}%
\AgdaSymbol{\{}\AgdaBound{x}\AgdaSymbol{\}}\AgdaSpace{}%
\AgdaSymbol{→}\AgdaSpace{}%
\AgdaFunction{fromℕ}\AgdaSpace{}%
\AgdaNumber{0}\AgdaSpace{}%
\AgdaOperator{\AgdaFunction{+}}\AgdaSpace{}%
\AgdaBound{x}%
\>[37]\AgdaOperator{\AgdaDatatype{≡}}\AgdaSpace{}%
\AgdaBound{x}\AgdaSymbol{;}%
\>[43]\AgdaField{+-neutʳ}\AgdaSpace{}%
\AgdaSymbol{:}\AgdaSpace{}%
\AgdaSymbol{∀}\AgdaSpace{}%
\AgdaSymbol{\{}\AgdaBound{x}\AgdaSymbol{\}}\AgdaSpace{}%
\AgdaSymbol{→}\AgdaSpace{}%
\AgdaBound{x}\AgdaSpace{}%
\AgdaOperator{\AgdaFunction{+}}\AgdaSpace{}%
\AgdaFunction{fromℕ}\AgdaSpace{}%
\AgdaNumber{0}%
\>[74]\AgdaOperator{\AgdaDatatype{≡}}\AgdaSpace{}%
\AgdaBound{x}\<%
\\
%
\>[6]\AgdaField{*-neutˡ}\AgdaSpace{}%
\AgdaSymbol{:}\AgdaSpace{}%
\AgdaSymbol{∀}\AgdaSpace{}%
\AgdaSymbol{\{}\AgdaBound{x}\AgdaSymbol{\}}\AgdaSpace{}%
\AgdaSymbol{→}\AgdaSpace{}%
\AgdaFunction{fromℕ}\AgdaSpace{}%
\AgdaNumber{1}\AgdaSpace{}%
\AgdaOperator{\AgdaFunction{*}}\AgdaSpace{}%
\AgdaBound{x}%
\>[37]\AgdaOperator{\AgdaDatatype{≡}}\AgdaSpace{}%
\AgdaBound{x}\AgdaSymbol{;}%
\>[43]\AgdaField{*-neutʳ}\AgdaSpace{}%
\AgdaSymbol{:}\AgdaSpace{}%
\AgdaSymbol{∀}\AgdaSpace{}%
\AgdaSymbol{\{}\AgdaBound{x}\AgdaSymbol{\}}\AgdaSpace{}%
\AgdaSymbol{→}\AgdaSpace{}%
\AgdaBound{x}\AgdaSpace{}%
\AgdaOperator{\AgdaFunction{*}}\AgdaSpace{}%
\AgdaFunction{fromℕ}\AgdaSpace{}%
\AgdaNumber{1}%
\>[74]\AgdaOperator{\AgdaDatatype{≡}}\AgdaSpace{}%
\AgdaBound{x}\<%
\end{code}
We define the meaning of semantics preservation by means of the \AF{\_≈ᵉ\_}
relation, which says that two expressions are equivalent if they evaluate
to equivalent values.  Equivalence of values is given by the propositional
equality of indices and extensional equality of arrays.  The type of
semantics-preserving \AF{opt}imisation function is given as follows.
\begin{code}[hide]%
%
\>[2]\AgdaKeyword{postulate}\<%
\\
\>[2][@{}l@{\AgdaIndent{0}}]%
\>[4]\AgdaPostulate{real}\AgdaSpace{}%
\AgdaSymbol{:}\AgdaSpace{}%
\AgdaRecord{Real}\<%
\\
%
\\[\AgdaEmptyExtraSkip]%
%
\>[2]\AgdaKeyword{open}\AgdaSpace{}%
\AgdaModule{Eval}\AgdaSpace{}%
\AgdaPostulate{real}\<%
\end{code}
\begin{mathpar}
\codeblock{\begin{code}%
%
\>[2]\AgdaOperator{\AgdaFunction{\AgdaUnderscore{}≈ᵛ\AgdaUnderscore{}}}\AgdaSpace{}%
\AgdaSymbol{:}\AgdaSpace{}%
\AgdaSymbol{(}\AgdaBound{a}\AgdaSpace{}%
\AgdaBound{b}\AgdaSpace{}%
\AgdaSymbol{:}\AgdaSpace{}%
\AgdaFunction{Val}\AgdaSpace{}%
\AgdaGeneralizable{is}\AgdaSymbol{)}\AgdaSpace{}%
\AgdaSymbol{→}\AgdaSpace{}%
\AgdaPrimitive{Set}\<%
\\
%
\>[2]\AgdaOperator{\AgdaFunction{\AgdaUnderscore{}≈ᵛ\AgdaUnderscore{}}}\AgdaSpace{}%
\AgdaSymbol{\{}\AgdaInductiveConstructor{ix}\AgdaSpace{}%
\AgdaBound{s}\AgdaSymbol{\}}\AgdaSpace{}%
\AgdaBound{a}\AgdaSpace{}%
\AgdaBound{b}\AgdaSpace{}%
\AgdaSymbol{=}\AgdaSpace{}%
\AgdaBound{a}\AgdaSpace{}%
\AgdaOperator{\AgdaDatatype{≡}}\AgdaSpace{}%
\AgdaBound{b}\<%
\\
%
\>[2]\AgdaOperator{\AgdaFunction{\AgdaUnderscore{}≈ᵛ\AgdaUnderscore{}}}\AgdaSpace{}%
\AgdaSymbol{\{}\AgdaInductiveConstructor{ar}\AgdaSpace{}%
\AgdaBound{s}\AgdaSymbol{\}}\AgdaSpace{}%
\AgdaBound{a}\AgdaSpace{}%
\AgdaBound{b}\AgdaSpace{}%
\AgdaSymbol{=}\AgdaSpace{}%
\AgdaSymbol{∀}\AgdaSpace{}%
\AgdaBound{i}\AgdaSpace{}%
\AgdaSymbol{→}\AgdaSpace{}%
\AgdaBound{a}\AgdaSpace{}%
\AgdaBound{i}\AgdaSpace{}%
\AgdaOperator{\AgdaDatatype{≡}}\AgdaSpace{}%
\AgdaBound{b}\AgdaSpace{}%
\AgdaBound{i}\<%
\end{code}}
\and
\codeblock{\begin{code}%
%
\>[2]\AgdaOperator{\AgdaFunction{\AgdaUnderscore{}≈ᵉ\AgdaUnderscore{}}}\AgdaSpace{}%
\AgdaSymbol{:}\AgdaSpace{}%
\AgdaDatatype{E}\AgdaSpace{}%
\AgdaGeneralizable{Γ}\AgdaSpace{}%
\AgdaGeneralizable{is}\AgdaSpace{}%
\AgdaSymbol{→}\AgdaSpace{}%
\AgdaDatatype{E}\AgdaSpace{}%
\AgdaGeneralizable{Γ}\AgdaSpace{}%
\AgdaGeneralizable{is}\AgdaSpace{}%
\AgdaSymbol{→}\AgdaSpace{}%
\AgdaPrimitive{Set}\<%
\\
%
\>[2]\AgdaOperator{\AgdaFunction{\AgdaUnderscore{}≈ᵉ\AgdaUnderscore{}}}\AgdaSpace{}%
\AgdaSymbol{\{}\AgdaBound{Γ}\AgdaSymbol{\}}\AgdaSpace{}%
\AgdaBound{a}\AgdaSpace{}%
\AgdaBound{b}\AgdaSpace{}%
\AgdaSymbol{=}\AgdaSpace{}%
\AgdaSymbol{⦃}\AgdaSpace{}%
\AgdaBound{ρ}\AgdaSpace{}%
\AgdaSymbol{:}\AgdaSpace{}%
\AgdaFunction{Env}\AgdaSpace{}%
\AgdaBound{Γ}\AgdaSpace{}%
\AgdaSymbol{⦄}\AgdaSpace{}%
\AgdaSymbol{→}\AgdaSpace{}%
\AgdaOperator{\AgdaFunction{⟦}}\AgdaSpace{}%
\AgdaBound{a}\AgdaSpace{}%
\AgdaOperator{\AgdaFunction{⟧}}\AgdaSpace{}%
\AgdaOperator{\AgdaFunction{≈ᵛ}}\AgdaSpace{}%
\AgdaOperator{\AgdaFunction{⟦}}\AgdaSpace{}%
\AgdaBound{b}\AgdaSpace{}%
\AgdaOperator{\AgdaFunction{⟧}}\<%
\\
\>[0]\<%
\\
%
\>[2]\AgdaFunction{opt}\AgdaSpace{}%
\AgdaSymbol{:}\AgdaSpace{}%
\AgdaSymbol{(}\AgdaBound{e}\AgdaSpace{}%
\AgdaSymbol{:}\AgdaSpace{}%
\AgdaDatatype{E}\AgdaSpace{}%
\AgdaGeneralizable{Γ}\AgdaSpace{}%
\AgdaGeneralizable{is}\AgdaSymbol{)}\AgdaSpace{}%
\AgdaSymbol{→}\AgdaSpace{}%
\AgdaFunction{∃}\AgdaSpace{}%
\AgdaSymbol{λ}\AgdaSpace{}%
\AgdaBound{e′}\AgdaSpace{}%
\AgdaSymbol{→}\AgdaSpace{}%
\AgdaSymbol{(}\AgdaBound{e}\AgdaSpace{}%
\AgdaOperator{\AgdaFunction{≈ᵉ}}\AgdaSpace{}%
\AgdaBound{e′}\AgdaSymbol{)}\<%
\end{code}}
\end{mathpar}
\begin{code}[hide]%
%
\>[2]\AgdaFunction{reflᵉ}\AgdaSpace{}%
\AgdaSymbol{:}\AgdaSpace{}%
\AgdaSymbol{∀}\AgdaSpace{}%
\AgdaSymbol{(}\AgdaBound{e}\AgdaSpace{}%
\AgdaSymbol{:}\AgdaSpace{}%
\AgdaDatatype{E}\AgdaSpace{}%
\AgdaGeneralizable{Γ}\AgdaSpace{}%
\AgdaGeneralizable{is}\AgdaSymbol{)}\AgdaSpace{}%
\AgdaSymbol{→}\AgdaSpace{}%
\AgdaBound{e}\AgdaSpace{}%
\AgdaOperator{\AgdaFunction{≈ᵉ}}\AgdaSpace{}%
\AgdaBound{e}\<%
\\
%
\>[2]\AgdaFunction{reflᵉ}\AgdaSpace{}%
\AgdaSymbol{\{}\AgdaBound{is}\AgdaSymbol{\}}\AgdaSpace{}%
\AgdaSymbol{\{}\AgdaInductiveConstructor{ix}\AgdaSpace{}%
\AgdaBound{x}\AgdaSymbol{\}}\AgdaSpace{}%
\AgdaSymbol{=}\AgdaSpace{}%
\AgdaSymbol{λ}\AgdaSpace{}%
\AgdaBound{e}\AgdaSpace{}%
\AgdaSymbol{→}\AgdaSpace{}%
\AgdaInductiveConstructor{refl}\<%
\\
%
\>[2]\AgdaFunction{reflᵉ}\AgdaSpace{}%
\AgdaSymbol{\{}\AgdaBound{is}\AgdaSymbol{\}}\AgdaSpace{}%
\AgdaSymbol{\{}\AgdaInductiveConstructor{ar}\AgdaSpace{}%
\AgdaBound{x}\AgdaSymbol{\}}\AgdaSpace{}%
\AgdaSymbol{=}\AgdaSpace{}%
\AgdaSymbol{λ}\AgdaSpace{}%
\AgdaBound{e}\AgdaSpace{}%
\AgdaBound{i}\AgdaSpace{}%
\AgdaSymbol{→}\AgdaSpace{}%
\AgdaInductiveConstructor{refl}\<%
\\
%
\>[2]\AgdaFunction{opt}\AgdaSpace{}%
\AgdaBound{e}\AgdaSpace{}%
\AgdaSymbol{=}\AgdaSpace{}%
\AgdaBound{e}\AgdaSpace{}%
\AgdaOperator{\AgdaInductiveConstructor{,}}\AgdaSpace{}%
\AgdaFunction{reflᵉ}\AgdaSpace{}%
\AgdaBound{e}\<%
\end{code}

Consider several examples of the rewrites that we are implementing.
We omit the proofs for readability, but they are available in the
supplementary materials.
\begin{mathpar}
   \AC{sels}\ zero\ e \rightsquigarrow \AC{zero}
   \and
   \AC{sels}\ one\ e \rightsquigarrow \AC{one}
   \and
   \AC{sels}\ (\AC{sum} e)\ i \rightsquigarrow \AC{sum}\ (\AC{sels}\ e\ (i\ \AF{↑}))
   \and
   \AC{sum}\ \AC{zero} \rightsquigarrow \AC{zero}
   \and
   \AC{sum}\ (\AC{imap*}\ e) \rightsquigarrow 
   \AC{imap*}\ (\AC{sum}\ (\AF{sub}\ e\ \AF{sub-swap}))
\end{mathpar}
Semantic preservation becomes especially useful in the following cases which
are not immediately obvious:
\begin{align*}
   \AC{sum}\ (\AC{zero-but}\ (\AC{var}\ i)\ (\AC{var}\ j)\ e)
   &\mathop{|} i = v_0 \wedge j = v_0
   \rightsquigarrow 
   \AC{sum}\ e
   \\
   \AC{sum}\ (\AC{zero-but}\ (\AC{var}\ i)\ (\AC{var}\ j)\ e)
   &\mathop{|} i \neq v_0 \wedge j = v_0
   \rightsquigarrow 
   \AF{sub}\ e\ (\AF{sub-id}\ \AC{▹}\ \AC{var}\ i)
   \\
   \AC{sum}\ (\AC{zero-but}\ (\AC{var}\ i)\ (\AC{var}\ j)\ e)
   &\mathop{|} i = v_0 \wedge j \neq v_0
   \rightsquigarrow 
   \AF{sub}\ e\ (\AF{sub-id}\ \AC{▹}\ \AC{var}\ j)
\end{align*}
which tell us that if we are summing-up comparisons of indices that
happen to be variables, we can compare whether either of the variables
is \AC{v₀} (the one that \AC{sum} binds) and potentially avoid comparison
or summation.

There is a number of rewrite rules that we left out.  Note that we are
not claiming that these optimisations are complete in any sense.  We have
implemented enough rewrites for the chosen example and backend compiler.
However, it is straight-forward to add more rewrite rules if they are
needed in other contexts.

Additionally to rewrites described above, we implemented a pass that
identifies whether let bodies re-define expressions that are bound to
the let variable.  If this is the case, then the expression is substituted
by the variable.  The main reason for this is the expression \AC{logistic} $e$,
which recomputes \AF{logistic} $e$ as a part of its derivative.  While
this is correct mathematically, this creates code duplication in cases
such as (\AC{let′} (\AF{logistic} $e$) $\dots$).  Instead of reusing
the variable that is bound in let, it recomputes the entire expression.
As it is difficult to tell whether the call to logistic has been bound
somewhere before, we implement a generally useful deduplication that
solves this problem.



% \begin{code}[hide]
% module Opt where
%   open import Data.Nat as ℕ using (ℕ; zero; suc)
%   open Lang
%   open SubWk
%   --open Eval using (sub; ctx-swap; ↑_; ↑↑_; eq?)
%   open Array hiding (sum; slide;backslide)
%   open BB
%   open AD
% \end{code}
% \begin{code}
%   opt : E Γ is → E Γ is
%   opt (selₛ e e₁) with opt e | opt e₁
%   ... | zero            | i = zero
%   ... | one             | i = one
%   ... | imapₛ e         | i = sub v₀ e i
%   ... | bin op a b      | i = bin op (selₛ a i) (selₛ b i)
%   ... | sum e           | i = sum (selₛ e (↑ i))
%   ... | zero-but i j a  | k = zero-but i j (selₛ a k)
%   ... | a               | i = selₛ a i
% 
%   opt (sum e) with opt e
%   ... | zero            = zero
%   ... | imap a          = imap     (sum (ctx-swap v₁ a))
%   ... | imapₛ a         = imapₛ    (sum (ctx-swap v₁ a))
%   ... | imapb m a       = imapb m  (sum (ctx-swap v₁ a))
%   ... | zero-but (var i) (var j) a with eq? v₀ i | eq? v₀ j
%   ... | eq        | eq        = sum a
%   ... | neq _ i′  | eq        = sub v₀ a (var i′)
%   ... | eq        | neq _ j′  = sub v₀ a (var j′)
%   ... | neq _ i′  | neq _ j′  = zero-but (var i′) (var j′) (sum a)
%   opt (sum e) | a = sum a
%   -- ⋯
% \end{code}
% Selection into \AC{zero} and \AC{one} is \AF{zero} and \AC{one}, as our constants
% are shape-polymorphic.  Selection into an \AF{imapₛ} is evaluation of the \AC{imapₛ}
% body at the given index (this is an array version of the $\beta$-rule).  Selection
% from the binary operation is a binary operation of selections.  Selection into \AC{sum}
% is the \AC{sum} of selections.  Selection into conditional is the same as conditional
% over selection.  Summing \AC{zero} is \AC{zero}.  Summing $s$-many $p$-shaped arrays
% is the same as computing the sum of $i$-th index of every array for all $p$ indices.
% If we have a sum of the conditional with the predicate is the equality of indices
% $i$ and $j$ and we know that $i$ and $j$ are variables, we can compare the index
% variable of the \AC{sum} with $i$ and $j$.  If they match, then conditional will
% be triggered at every iteration so it can be removed.  If only one of them match,
% and we are comparing variables of the same shape, there will be exactly one case
% (for non-empty shapes) where this conditional will be triggered.  Therefore, all
% the iterations except the one at the non-matching variable will turn to zero, and
% we can simply return the expressions substituted at this variable.  If the shape
% of the index variables is empty, we are in the absurd case, as we cannot possibly
% create an element of an empty type.  Finally, if none of the variables match,
% the iteration within the \AC{sum} do not affect the result of the predicate ---
% it will be either true or false for all the iterations.  Therefore, we can lift
% the conditional outside of the sum.
% \begin{code}[hide]
%   opt zero = zero
%   opt one = one
%   
%   opt (var x) = var x
%   
%   opt (imapₛ e) = imapₛ (opt e)
%   
%   -- Literal copy of the above, replaing scalar versions
%   -- with normal one
%   opt (imap e) = imap (opt e)
%   opt (sel e e₁) with opt e | opt e₁
%   ... | zero | i = zero
%   ... | one | i = one
%   ... | imap e | i = sub v₀ e i
%   --... | imapb m e | i = ?
%   ... | bin op a b | i = bin op (sel a i) (sel b i)
%   ... | sum e | i = sum (sel e (wk v₀ i))
%   ... | zero-but i j a | k = zero-but i j (sel a k)
%   ... | a | i = sel a i
%   
%   -- Literal copy of the above for the blocked version
%   opt (imapb m e) = imapb m (opt e)
%   opt (selb m e k) with opt e
%   ... | zero = zero
%   ... | one = one
%   ... | sum e = sum (selb m e (↑ k {- var $ vₛ k-}))
%   ... | zero-but i j a = zero-but i j (selb m a k)
%   ... | bin op a b = bin op (selb m a k) (selb m b k)
%   opt (selb m e j) | a = selb m a j
%   
%   
%   opt (zero-but (var i) (var j) e) with opt e
%   ... | a with eq? i j
%   ... | eq = a
%   ... | neq _ _ = zero-but (var i) (var j) a
%   --opt (zero-but i j e) = zero-but i j (opt e)
%   
%   opt (bin plus e e₁) with opt e | opt e₁
%   ... | zero | b = b
%   ... | a | zero = a
%   ... | (zero-but i j e) | b = zero-but i j (bin plus e b)
%   ... | a | (zero-but i j e) = zero-but i j (bin plus a e)
% 
%   ... | imapₛ a | b = imapₛ (bin plus a (selₛ (↑ b) (var v₀)))
%   ... | a | imapₛ b = imapₛ (bin plus (selₛ (↑ a) (var v₀)) b)
%   ... | imap a | b = imap (bin plus a (sel (↑ b) (var v₀)))
%   ... | a | imap b = imap (bin plus (sel (↑ a) (var v₀)) b)
%   ... | imapb m a | b = imapb m (bin plus a (selb m (↑ b) (var v₀)))
%   ... | a | imapb m b = imapb m (bin plus (selb m (↑ a) (var v₀)) b)
% 
%   ... | a | b = bin plus a b
%   opt (bin mul e e₁) with opt e | opt e₁
%   ... | zero | b = zero
%   ... | a | zero = zero
%   ... | one | b = b
%   ... | a | one = a
%   ... | (zero-but i j e) | b = zero-but i j (bin mul e b)
%   ... | a | (zero-but i j e) = zero-but i j (bin mul a e)
%   
%   ... | imapₛ a | b = imapₛ (bin mul a (selₛ (↑ b) (var v₀)))
%   ... | a | imapₛ b = imapₛ (bin mul (selₛ (↑ a) (var v₀)) b)
%   ... | imap a | b = imap (bin mul a (sel (↑ b) (var v₀)))
%   ... | a | imap b = imap (bin mul (sel (↑ a) (var v₀)) b)
%   ... | imapb m a | b = imapb m (bin mul a (selb m (↑ b) (var v₀)))
%   ... | a | imapb m b = imapb m (bin mul (selb m (↑ a) (var v₀)) b)
%   
%   ... | a | b = bin mul a b
%   
%   -- XXX: not calling opt on e, as this is index
%   opt (slide i pl e su) with opt e
%   ... | zero = zero
%   ... | a = slide i pl a su
%   opt (backslide i e su pl) with opt e
%   ... | zero = zero
%   ... | a = backslide i a su pl
%   opt (scaledown x e) with opt e
%   ... | scaledown y a = scaledown (x ℕ.* y) a
%   ... | a = scaledown x a
%   -- TODO: propogate minues inside of +, *, imap, etc.
%   opt (minus e) with opt e
%   ... | minus a = a
%   ... | imapₛ a = imapₛ (minus a)
%   ... | imap a = imap (minus a)
%   ... | imapb m a = imapb m (minus a)
%   ... | sum e = sum (minus e)
%   ... | bin plus a b = bin plus (minus a) (minus b)
%   ... | bin mul a b = bin plus (minus a) b
%   ... | a = minus a
%   opt (logistic e) with opt e
%   ... | imapₛ a = imapₛ (logistic a)
%   ... | imap a = imap (logistic a)
%   ... | a = logistic a
% 
% 
%   multiopt : ℕ → E Γ is → E Γ is
%   multiopt zero e = e
%   multiopt (suc n) e = opt (multiopt n e)
% 
%   module TryOpt where
% \end{code}
% 
% Let us observe optimisation effects when computing derivatives of
% the scalar dot-product defined as follows.
% \begin{code}
%     dotp : E Γ (ar s) → E Γ (ar s) → E Γ (ar unit)
%     dotp a b = Sum λ i → selₛ (↑ a) i ⊠ selₛ (↑ b) i
% \end{code}
% \begin{code}[hide]
%     C : Ctx 
%     a : E C _ 
%     b : E C _
%     seed : E C _
% \end{code}
% We define the context \AF{C} where two top variables are of 5-element vector shape
% and the last variable (\AC{v₂}) is of scalar shape.  We bind these variables to Agda
% variables for convenience.
% \begin{code}
%     C = ε ▹  ar (ι 1)       ▹  ar (ι 5)    ▹  ar (ι 5);
%              seed = var v₂  ;  a = var v₁  ;  b  = var v₀
% \end{code}
% \begin{code}[hide]
%     ∂a     = env-ix {C} (∇ {C} (dotp a b) seed (env-zero {C})) v₁
%     ∂a′    = multiopt 3 ∂a
% \end{code}
% We compute the derivatives of \AF{dotp a b} with seed \AF{seed} and we inspect
% the $a$-th position in the returned environment that we call \AF{∂a}.  Then we repeatedly
% apply \AF{opt} (three times) to \AF{∂a} and save it in \AF{∂a′}.  We force Agda to
% verify that the content of the variables is as follows:
% \begin{code}
%     non-opt   : ∂a   ≡ (Sum λ i → zero ⊞ Imapₛ λ j → zero-but j (↑ i) (↑↑ seed ⊠ selₛ (↑↑ b) (↑ i))) ⊞ zero
%     with-opt  : ∂a′  ≡ Imapₛ λ i → (↑ seed ⊠ selₛ (↑ b) i)
% \end{code}
% \begin{code}[hide]
%     non-opt = refl
%     with-opt = refl
% -- open Lang
% -- open SubWk
% \end{code}
% As it can be seen, \AF{∂a} sums-up the arrays, where only one element is non-zero at
% every iteration.  Such a computation is highly inefficient when executed directly,
% as it needs to compute all the inner arrays before summing them up.  However, the
% optimised version correctly rewrites \AF{∂a} into \AC{imap} that multiplies
% the \AB{seed} by $b$, which is the expected answer.  This reduces complexity
% of the expression form squared to linear.
% 
\subsection{Extraction}

We had two reasons to define the embedded language \AF{E}.
Firstly, \AF{E} makes it possible to implement automatic differentiation
within Agda, as we described in the previous section.
Secondly, we extract expressions in \AF{E} into
a programming language that can produce efficient code.  This
section describes extraction process into Futhark.

Futhark is a functional language with automatic memory management and
a built-in type for arrays.  Futhark provides key array combinators such as
map and reduce, which makes the translation process straightforward.
The only boilerplate code we require from Futhark in order
to run the generated code is: implementations of operation on reals
from \AM{Real} (these are mapped into 32-bit floating point operations);
and rank-$n$ versions of the imap and sum combinators.  The latter is defined
as follows:
\begin{Verbatim}
def imap1 'a : (n: i64) -> (i64 -> a) -> [n]a =
  \n f -> map f (iota n)
def imap2 'a : (m: i64) -> (n: i64) -> (i64 -> i64 -> a) -> [m][n]a =
  \m n f -> imap m (\i -> imap n (f i))
...
def isum1 : (m: i64) -> (i64 -> real) -> real =
  \m f -> loop r = zero for i < m do r F.+ f i
def isum2 : (m: i64) -> (n: i64)
          -> (i64 -> i64 -> real) -> real =
  \m n f -> loop r = zero for i < m do r F.+ isum1 n (f i)
...
\end{Verbatim}


\paragraph{Static Ranks} As Futhark does not support rank polymorphism, we must define imap and sum variants for every needed array rank. This also means that
it is not possible to translate an arbitrary expression in \AF{E} into
Futhark, because \AF{E} can define a function that abstracts over shapes
(which, in turn, means abstraction over ranks).  For the purpose of
extraction, we assume that all the ranks are known statically, and we
resolve possible shape abstractions during extraction.  The assumption about
static ranks holds for many numerical applications including our
running example.  Relaxing this assumption is an interesting future work.

\paragraph{Normalisation} Consider translating an expression like
\AC{sel} (\AC{imap} λ i → \AB{e}) \AB{u}.  If we were to treat arrays
as functions and selections as applications, then the above expression
could be normalised into $e[i := u]$.  One could hope that Futhark could do
such a $\beta$-reduction on the generated code, but this is not the case.
The intuition for this choice is that in Futhark arrays are tabulated
functions, and inlining arbitrary evaluation of array elements may
have a significant performance cost.  For example, in the expression
\texttt{let a = imap \textbackslash i -> }$e$ \texttt{in imap \textbackslash j -> a[f j]}, Futhark
allocates memory for $a$ and manifests elements in memory, and within the
body of the let, selection fetches from memory.  If we were
to inline $a$ by replacing $a[f\ j]$ with $e[i := f\ j]$, we loose sharing
by potentially recomputing $e$ much more often than needed
(e.g. assume that $i$ ranges over 10 elements, but $j$ over $10^5$).
Resolving when such inlining is beneficial for performance is non-trivial,
therefore Futhark (and many other array languages) do not inline 
computation of array elements.  For our running example, naive translation
results in too many cases when arrays are constructed just to select
an element from them.  Therefore, we need some notion of normalisation
prior to extraction.


We combine normalisation and extraction in a single step,
resulting in an approach that is similar to normalisation by evaluation.
We model Futhark arrays as Agda functions, which makes it
easy to encode normalisation steps.
\begin{code}[hide]%
\>[0]\AgdaKeyword{module}\AgdaSpace{}%
\AgdaModule{Futhark}\AgdaSpace{}%
\AgdaKeyword{where}\<%
\\
\>[0][@{}l@{\AgdaIndent{0}}]%
\>[2]\AgdaKeyword{open}\AgdaSpace{}%
\AgdaKeyword{import}\AgdaSpace{}%
\AgdaModule{Data.Nat.Show}\AgdaSpace{}%
\AgdaKeyword{using}\AgdaSpace{}%
\AgdaSymbol{()}\AgdaSpace{}%
\AgdaKeyword{renaming}\AgdaSpace{}%
\AgdaSymbol{(}\AgdaFunction{show}\AgdaSpace{}%
\AgdaSymbol{to}\AgdaSpace{}%
\AgdaFunction{show-nat}\AgdaSymbol{)}\<%
\\
%
\>[2]\AgdaKeyword{open}\AgdaSpace{}%
\AgdaKeyword{import}\AgdaSpace{}%
\AgdaModule{Data.List}\AgdaSpace{}%
\AgdaSymbol{as}\AgdaSpace{}%
\AgdaModule{L}\AgdaSpace{}%
\AgdaKeyword{using}\AgdaSpace{}%
\AgdaSymbol{(}\AgdaDatatype{List}\AgdaSymbol{;}\AgdaSpace{}%
\AgdaInductiveConstructor{[]}\AgdaSymbol{;}\AgdaSpace{}%
\AgdaOperator{\AgdaInductiveConstructor{\AgdaUnderscore{}∷\AgdaUnderscore{}}}\AgdaSymbol{)}\<%
\\
%
\>[2]\AgdaKeyword{open}\AgdaSpace{}%
\AgdaKeyword{import}\AgdaSpace{}%
\AgdaModule{Data.List.Relation.Unary.All}\AgdaSpace{}%
\AgdaSymbol{as}\AgdaSpace{}%
\AgdaModule{All}\AgdaSpace{}%
\AgdaKeyword{using}\AgdaSpace{}%
\AgdaSymbol{(}\AgdaDatatype{All}\AgdaSymbol{;}\AgdaSpace{}%
\AgdaInductiveConstructor{[]}\AgdaSymbol{;}\AgdaSpace{}%
\AgdaOperator{\AgdaInductiveConstructor{\AgdaUnderscore{}∷\AgdaUnderscore{}}}\AgdaSymbol{)}\<%
\\
%
\>[2]\AgdaKeyword{open}\AgdaSpace{}%
\AgdaKeyword{import}\AgdaSpace{}%
\AgdaModule{Relation.Binary.PropositionalEquality}\<%
\\
%
\>[2]\AgdaKeyword{open}\AgdaSpace{}%
\AgdaKeyword{import}\AgdaSpace{}%
\AgdaModule{Data.String}\<%
\\
%
\>[2]\AgdaKeyword{open}\AgdaSpace{}%
\AgdaKeyword{import}\AgdaSpace{}%
\AgdaModule{Text.Printf}\<%
\\
%
\>[2]\AgdaKeyword{open}\AgdaSpace{}%
\AgdaKeyword{import}\AgdaSpace{}%
\AgdaModule{Data.Unit}\<%
\\
%
\>[2]\AgdaKeyword{open}\AgdaSpace{}%
\AgdaKeyword{import}\AgdaSpace{}%
\AgdaModule{Data.Product}\AgdaSpace{}%
\AgdaSymbol{as}\AgdaSpace{}%
\AgdaModule{Prod}\AgdaSpace{}%
\AgdaKeyword{hiding}\AgdaSpace{}%
\AgdaSymbol{(}\AgdaOperator{\AgdaFunction{\AgdaUnderscore{}<*>\AgdaUnderscore{}}}\AgdaSymbol{)}\<%
\\
%
\>[2]\AgdaKeyword{open}\AgdaSpace{}%
\AgdaKeyword{import}\AgdaSpace{}%
\AgdaModule{Data.Nat}\AgdaSpace{}%
\AgdaKeyword{using}\AgdaSpace{}%
\AgdaSymbol{(}\AgdaDatatype{ℕ}\AgdaSymbol{;}\AgdaSpace{}%
\AgdaInductiveConstructor{zero}\AgdaSymbol{;}\AgdaSpace{}%
\AgdaInductiveConstructor{suc}\AgdaSymbol{)}\<%
\\
%
\>[2]\AgdaKeyword{open}\AgdaSpace{}%
\AgdaKeyword{import}\AgdaSpace{}%
\AgdaModule{arrays}\<%
\\
%
\>[2]\AgdaKeyword{open}\AgdaSpace{}%
\AgdaKeyword{import}\AgdaSpace{}%
\AgdaModule{lang}\<%
\\
%
\>[2]\AgdaKeyword{open}\AgdaSpace{}%
\AgdaKeyword{import}\AgdaSpace{}%
\AgdaModule{Function}\<%
\\
%
\>[2]\AgdaKeyword{open}\AgdaSpace{}%
\AgdaModule{Array}\AgdaSpace{}%
\AgdaKeyword{hiding}\AgdaSpace{}%
\AgdaSymbol{(}\AgdaFunction{Ix}\AgdaSymbol{)}\<%
\\
%
\>[2]\AgdaKeyword{open}\AgdaSpace{}%
\AgdaModule{Lang}\<%
\\
%
\\[\AgdaEmptyExtraSkip]%
%
\>[2]\AgdaKeyword{open}\AgdaSpace{}%
\AgdaKeyword{import}\AgdaSpace{}%
\AgdaModule{Effect.Monad.State}\<%
\\
%
\>[2]\AgdaKeyword{open}\AgdaSpace{}%
\AgdaKeyword{import}\AgdaSpace{}%
\AgdaModule{Effect.Monad}\AgdaSpace{}%
\AgdaKeyword{using}\AgdaSpace{}%
\AgdaSymbol{(}\AgdaRecord{RawMonad}\AgdaSymbol{)}\<%
\\
%
\>[2]\AgdaKeyword{open}\AgdaSpace{}%
\AgdaModule{RawMonadState}\AgdaSpace{}%
\AgdaSymbol{\{\{...\}\}}\AgdaSpace{}%
\AgdaComment{--\ public}\<%
\\
%
\>[2]\AgdaKeyword{open}\AgdaSpace{}%
\AgdaModule{RawMonad}\AgdaSpace{}%
\AgdaSymbol{\{\{...\}\}}\AgdaSpace{}%
\AgdaComment{--\ public}\<%
\\
\>[0]\<%
\\
%
\>[2]\AgdaKeyword{instance}\<%
\\
\>[2][@{}l@{\AgdaIndent{0}}]%
\>[4]\AgdaFunction{\AgdaUnderscore{}}\AgdaSpace{}%
\AgdaSymbol{=}\AgdaSpace{}%
\AgdaFunction{monad}\<%
\\
%
\>[4]\AgdaFunction{\AgdaUnderscore{}}\AgdaSpace{}%
\AgdaSymbol{=}\AgdaSpace{}%
\AgdaFunction{applicative}\<%
\\
%
\>[4]\AgdaFunction{\AgdaUnderscore{}}\AgdaSpace{}%
\AgdaSymbol{=}\AgdaSpace{}%
\AgdaFunction{monadState}\<%
\end{code}
Futhark indices for an array of shape $s$ are given by the type \AD{Ix} which
is simply a list of strings (the name of the index) per dimension.
The \AF{Sem} function gives an interpretation to types of \AF{E} expressions.
Indices are just interpreted as \AF{Ix} of the corresponding shape.  Array
types are morally functions fro indices to strings.  However, in the definition
the type is a little more complicated:
\begin{mathpar}
\codeblock{\begin{code}%
%
\>[2]\AgdaKeyword{data}\AgdaSpace{}%
\AgdaDatatype{Ix}\AgdaSpace{}%
\AgdaSymbol{:}\AgdaSpace{}%
\AgdaDatatype{S}\AgdaSpace{}%
\AgdaSymbol{→}\AgdaSpace{}%
\AgdaPrimitive{Set}\AgdaSpace{}%
\AgdaKeyword{where}\<%
\\
\>[2][@{}l@{\AgdaIndent{0}}]%
\>[4]\AgdaInductiveConstructor{[]}%
\>[8]\AgdaSymbol{:}\AgdaSpace{}%
\AgdaDatatype{Ix}\AgdaSpace{}%
\AgdaInductiveConstructor{[]}\<%
\\
%
\>[4]\AgdaOperator{\AgdaInductiveConstructor{\AgdaUnderscore{}∷\AgdaUnderscore{}}}\AgdaSpace{}%
\AgdaSymbol{:}\AgdaSpace{}%
\AgdaPostulate{String}\AgdaSpace{}%
\AgdaSymbol{→}\AgdaSpace{}%
\AgdaDatatype{Ix}\AgdaSpace{}%
\AgdaGeneralizable{s}\AgdaSpace{}%
\AgdaSymbol{→}\AgdaSpace{}%
\AgdaDatatype{Ix}\AgdaSpace{}%
\AgdaSymbol{(}\AgdaGeneralizable{n}\AgdaSpace{}%
\AgdaOperator{\AgdaInductiveConstructor{∷}}\AgdaSpace{}%
\AgdaGeneralizable{s}\AgdaSymbol{)}\<%
\end{code}}
\and
\codeblock{\begin{code}%
%
\>[2]\AgdaFunction{Sem}\AgdaSpace{}%
\AgdaSymbol{:}\AgdaSpace{}%
\AgdaDatatype{IS}\AgdaSpace{}%
\AgdaSymbol{→}\AgdaSpace{}%
\AgdaPrimitive{Set}\<%
\\
%
\>[2]\AgdaFunction{Sem}\AgdaSpace{}%
\AgdaSymbol{(}\AgdaInductiveConstructor{ar}\AgdaSpace{}%
\AgdaBound{s}\AgdaSymbol{)}\AgdaSpace{}%
\AgdaSymbol{=}\AgdaSpace{}%
\AgdaSymbol{(}\AgdaDatatype{Ix}\AgdaSpace{}%
\AgdaBound{s}\AgdaSpace{}%
\AgdaSymbol{→}\AgdaSpace{}%
\AgdaFunction{State}\AgdaSpace{}%
\AgdaDatatype{ℕ}\AgdaSpace{}%
\AgdaSymbol{((}\AgdaPostulate{String}\AgdaSpace{}%
\AgdaSymbol{→}\AgdaSpace{}%
\AgdaPostulate{String}\AgdaSymbol{)}\AgdaSpace{}%
\AgdaOperator{\AgdaFunction{×}}\AgdaSpace{}%
\AgdaPostulate{String}\AgdaSymbol{))}\<%
\\
%
\>[2]\AgdaFunction{Sem}\AgdaSpace{}%
\AgdaSymbol{(}\AgdaInductiveConstructor{ix}\AgdaSpace{}%
\AgdaBound{s}\AgdaSymbol{)}\AgdaSpace{}%
\AgdaSymbol{=}\AgdaSpace{}%
\AgdaDatatype{Ix}\AgdaSpace{}%
\AgdaBound{s}\<%
\end{code}}
\end{mathpar}
Let us explain the complexity of the array type.  First of all, the codomain
of the array is wrapped into a state monad which gives a source of fresh variable
names.  Within the monad we have a pair we have a function which represents
a context for the actual array computation which is the second argument.
This context is needed because of the interplay between let bindings and
imaps.  Consider for a moment that we do not have explicit context in the
type for \AC{ar} and we are compiling an expression
\AF{Let} z \AF{:=} \AC{zero} \AF{in} \AF{Imaps} λ i → z which can result in
something like:
\begin{code}%
%
\>[2]\AgdaFunction{f}\AgdaSpace{}%
\AgdaSymbol{:}\AgdaSpace{}%
\AgdaDatatype{Ix}\AgdaSpace{}%
\AgdaGeneralizable{s}\AgdaSpace{}%
\AgdaSymbol{→}\AgdaSpace{}%
\AgdaFunction{State}\AgdaSpace{}%
\AgdaDatatype{ℕ}\AgdaSpace{}%
\AgdaPostulate{String}\<%
\\
%
\>[2]\AgdaFunction{f}\AgdaSpace{}%
\AgdaBound{i}\AgdaSpace{}%
\AgdaSymbol{=}\AgdaSpace{}%
\AgdaFunction{return}\AgdaSpace{}%
\AgdaSymbol{(}\AgdaString{"let\ z\ =\ 0\ in\ "}\AgdaSpace{}%
\AgdaOperator{\AgdaFunction{++}}\AgdaSpace{}%
\AgdaSymbol{(λ}\AgdaSpace{}%
\AgdaBound{j}\AgdaSpace{}%
\AgdaSymbol{→}\AgdaSpace{}%
\AgdaString{"z"}\AgdaSymbol{)}\AgdaSpace{}%
\AgdaBound{i}\AgdaSymbol{)}\<%
\end{code}
If we select into this array (by applying it to some index expression)
or compose it with other functions, this works as expected.  However,
at a certain point we may need to turn this expression into the actual
array, which looks something like \AS{"imap λ i → "} \AF{++} f \AS{"i"}.
However, this expression evaluates to \AS{"imap λ i → let z = 0\ in z"},
but this inlines computation of let binding in the body of the imap,
which may have a serious performance impact if let binds a non-trivial
computation.  By introducing contexts in \AF{Sem}, we just control where
the imap code is injected.  Generally speaking, our strategy here is to
preserve sharing that is introduced by let bindings, yet normalise
bound expressions and bodies.

For the actual extraction we define the environment of Futhark values
called \AF{FEnv}.  Two functions that actually do the translation are
\AF{to-fut} which computes the \AF{Sem} value, and \AF{to-str} that
calls \AF{to-fut} and wraps the result with \AF{imap} as we described
above.
\begin{mathpar}
\codeblock{\begin{code}%
%
\>[2]\AgdaFunction{FEnv}\AgdaSpace{}%
\AgdaSymbol{:}\AgdaSpace{}%
\AgdaDatatype{Ctx}\AgdaSpace{}%
\AgdaSymbol{→}\AgdaSpace{}%
\AgdaPrimitive{Set}\<%
\\
%
\>[2]\AgdaFunction{FEnv}\AgdaSpace{}%
\AgdaInductiveConstructor{ε}\AgdaSpace{}%
\AgdaSymbol{=}\AgdaSpace{}%
\AgdaRecord{⊤}\<%
\\
%
\>[2]\AgdaFunction{FEnv}\AgdaSpace{}%
\AgdaSymbol{(}\AgdaBound{Γ}\AgdaSpace{}%
\AgdaOperator{\AgdaInductiveConstructor{▹}}\AgdaSpace{}%
\AgdaBound{is}\AgdaSymbol{)}\AgdaSpace{}%
\AgdaSymbol{=}\AgdaSpace{}%
\AgdaFunction{FEnv}\AgdaSpace{}%
\AgdaBound{Γ}\AgdaSpace{}%
\AgdaOperator{\AgdaFunction{×}}\AgdaSpace{}%
\AgdaFunction{Sem}\AgdaSpace{}%
\AgdaBound{is}\<%
\end{code}}
\and
\codeblock{\begin{code}[hide]%
%
\>[2]\AgdaFunction{lookup}\AgdaSpace{}%
\AgdaSymbol{:}\AgdaSpace{}%
\AgdaGeneralizable{is}\AgdaSpace{}%
\AgdaOperator{\AgdaDatatype{∈}}\AgdaSpace{}%
\AgdaGeneralizable{Γ}\AgdaSpace{}%
\AgdaSymbol{→}\AgdaSpace{}%
\AgdaFunction{FEnv}\AgdaSpace{}%
\AgdaGeneralizable{Γ}\AgdaSpace{}%
\AgdaSymbol{→}\AgdaSpace{}%
\AgdaFunction{Sem}\AgdaSpace{}%
\AgdaGeneralizable{is}\<%
\\
%
\>[2]\AgdaFunction{lookup}\AgdaSpace{}%
\AgdaInductiveConstructor{v₀}\AgdaSpace{}%
\AgdaSymbol{(}\AgdaBound{ρ}\AgdaSpace{}%
\AgdaOperator{\AgdaInductiveConstructor{,}}\AgdaSpace{}%
\AgdaBound{e}\AgdaSymbol{)}\AgdaSpace{}%
\AgdaSymbol{=}\AgdaSpace{}%
\AgdaBound{e}\<%
\\
%
\>[2]\AgdaFunction{lookup}\AgdaSpace{}%
\AgdaSymbol{(}\AgdaInductiveConstructor{vₛ}\AgdaSpace{}%
\AgdaBound{x}\AgdaSymbol{)}\AgdaSpace{}%
\AgdaSymbol{(}\AgdaBound{ρ}\AgdaSpace{}%
\AgdaOperator{\AgdaInductiveConstructor{,}}\AgdaSpace{}%
\AgdaBound{e}\AgdaSymbol{)}\AgdaSpace{}%
\AgdaSymbol{=}\AgdaSpace{}%
\AgdaFunction{lookup}\AgdaSpace{}%
\AgdaBound{x}\AgdaSpace{}%
\AgdaBound{ρ}\<%
\\
%
\\[\AgdaEmptyExtraSkip]%
%
\>[2]\AgdaComment{--show-shape\ :\ S\ →\ String}\<%
\\
%
\>[2]\AgdaComment{--show-shape\ s\ =\ printf\ "[\%s]"\ \$\ intersperse\ ",\ "\ \$\ L.map\ show-nat\ s}\<%
\\
%
\\[\AgdaEmptyExtraSkip]%
%
\>[2]\AgdaFunction{s-list}\AgdaSpace{}%
\AgdaSymbol{:}\AgdaSpace{}%
\AgdaDatatype{S}\AgdaSpace{}%
\AgdaSymbol{→}\AgdaSpace{}%
\AgdaDatatype{List}\AgdaSpace{}%
\AgdaDatatype{ℕ}\<%
\\
%
\>[2]\AgdaFunction{s-list}\AgdaSpace{}%
\AgdaInductiveConstructor{[]}\AgdaSpace{}%
\AgdaSymbol{=}\AgdaSpace{}%
\AgdaInductiveConstructor{[]}\<%
\\
%
\>[2]\AgdaFunction{s-list}\AgdaSpace{}%
\AgdaSymbol{(}\AgdaBound{n}\AgdaSpace{}%
\AgdaOperator{\AgdaInductiveConstructor{∷}}\AgdaSpace{}%
\AgdaBound{ns}\AgdaSymbol{)}\AgdaSpace{}%
\AgdaSymbol{=}\AgdaSpace{}%
\AgdaBound{n}\AgdaSpace{}%
\AgdaOperator{\AgdaInductiveConstructor{∷}}\AgdaSpace{}%
\AgdaFunction{s-list}\AgdaSpace{}%
\AgdaBound{ns}\<%
\\
%
\\[\AgdaEmptyExtraSkip]%
%
\>[2]\AgdaFunction{list-s}\AgdaSpace{}%
\AgdaSymbol{:}\AgdaSpace{}%
\AgdaDatatype{List}\AgdaSpace{}%
\AgdaDatatype{ℕ}\AgdaSpace{}%
\AgdaSymbol{→}\AgdaSpace{}%
\AgdaDatatype{S}\<%
\\
%
\>[2]\AgdaFunction{list-s}\AgdaSpace{}%
\AgdaInductiveConstructor{[]}\AgdaSpace{}%
\AgdaSymbol{=}\AgdaSpace{}%
\AgdaInductiveConstructor{[]}\<%
\\
%
\>[2]\AgdaFunction{list-s}\AgdaSpace{}%
\AgdaSymbol{(}\AgdaBound{n}\AgdaSpace{}%
\AgdaOperator{\AgdaInductiveConstructor{∷}}\AgdaSpace{}%
\AgdaBound{ns}\AgdaSymbol{)}\AgdaSpace{}%
\AgdaSymbol{=}\AgdaSpace{}%
\AgdaBound{n}\AgdaSpace{}%
\AgdaOperator{\AgdaInductiveConstructor{∷}}\AgdaSpace{}%
\AgdaFunction{list-s}\AgdaSpace{}%
\AgdaBound{ns}\<%
\\
%
\\[\AgdaEmptyExtraSkip]%
%
\>[2]\AgdaFunction{shape-args}\AgdaSpace{}%
\AgdaSymbol{:}\AgdaSpace{}%
\AgdaDatatype{S}\AgdaSpace{}%
\AgdaSymbol{→}\AgdaSpace{}%
\AgdaPostulate{String}\<%
\\
%
\>[2]\AgdaFunction{shape-args}\AgdaSpace{}%
\AgdaSymbol{=}\AgdaSpace{}%
\AgdaFunction{intersperse}\AgdaSpace{}%
\AgdaString{"\ "}\AgdaSpace{}%
\AgdaOperator{\AgdaFunction{∘}}\AgdaSpace{}%
\AgdaFunction{L.map}\AgdaSpace{}%
\AgdaFunction{show-nat}\AgdaSpace{}%
\AgdaOperator{\AgdaFunction{∘}}\AgdaSpace{}%
\AgdaFunction{s-list}\<%
\\
%
\\[\AgdaEmptyExtraSkip]%
%
\>[2]\AgdaFunction{dim}\AgdaSpace{}%
\AgdaSymbol{:}\AgdaSpace{}%
\AgdaDatatype{S}\AgdaSpace{}%
\AgdaSymbol{→}\AgdaSpace{}%
\AgdaDatatype{ℕ}\<%
\\
%
\>[2]\AgdaFunction{dim}\AgdaSpace{}%
\AgdaSymbol{=}\AgdaSpace{}%
\AgdaFunction{L.length}\AgdaSpace{}%
\AgdaOperator{\AgdaFunction{∘}}\AgdaSpace{}%
\AgdaFunction{s-list}\<%
\\
%
\\[\AgdaEmptyExtraSkip]%
%
\>[2]\AgdaFunction{fresh-var}\AgdaSpace{}%
\AgdaSymbol{:}\AgdaSpace{}%
\AgdaDatatype{ℕ}\AgdaSpace{}%
\AgdaSymbol{→}\AgdaSpace{}%
\AgdaPostulate{String}\<%
\\
%
\>[2]\AgdaFunction{fresh-var}\AgdaSpace{}%
\AgdaBound{n}\AgdaSpace{}%
\AgdaSymbol{=}\AgdaSpace{}%
\AgdaString{"x"}\AgdaSpace{}%
\AgdaOperator{\AgdaFunction{++}}\AgdaSpace{}%
\AgdaFunction{show-nat}\AgdaSpace{}%
\AgdaBound{n}\<%
\\
\>[0]\<%
\\
%
\>[2]\AgdaFunction{fresh-ix}\AgdaSpace{}%
\AgdaSymbol{:}\AgdaSpace{}%
\AgdaPostulate{String}\AgdaSpace{}%
\AgdaSymbol{→}\AgdaSpace{}%
\AgdaDatatype{Ix}\AgdaSpace{}%
\AgdaGeneralizable{s}\<%
\\
%
\>[2]\AgdaFunction{fresh-ix}\AgdaSpace{}%
\AgdaBound{n}\AgdaSpace{}%
\AgdaSymbol{=}\AgdaSpace{}%
\AgdaField{proj₂}\AgdaSpace{}%
\AgdaSymbol{(}\AgdaFunction{runState}\AgdaSpace{}%
\AgdaSymbol{(}\AgdaFunction{go}\AgdaSpace{}%
\AgdaBound{n}\AgdaSymbol{)}\AgdaSpace{}%
\AgdaNumber{0}\AgdaSymbol{)}\<%
\\
\>[2][@{}l@{\AgdaIndent{0}}]%
\>[4]\AgdaKeyword{where}\<%
\\
\>[4][@{}l@{\AgdaIndent{0}}]%
\>[6]\AgdaFunction{go}\AgdaSpace{}%
\AgdaSymbol{:}\AgdaSpace{}%
\AgdaPostulate{String}\AgdaSpace{}%
\AgdaSymbol{→}\AgdaSpace{}%
\AgdaFunction{State}\AgdaSpace{}%
\AgdaDatatype{ℕ}\AgdaSpace{}%
\AgdaSymbol{(}\AgdaDatatype{Ix}\AgdaSpace{}%
\AgdaGeneralizable{s}\AgdaSymbol{)}\<%
\\
%
\>[6]\AgdaFunction{go}\AgdaSpace{}%
\AgdaSymbol{\{}\AgdaInductiveConstructor{[]}\AgdaSymbol{\}}\AgdaSpace{}%
\AgdaBound{n}\AgdaSpace{}%
\AgdaSymbol{=}\AgdaSpace{}%
\AgdaFunction{return}\AgdaSpace{}%
\AgdaInductiveConstructor{[]}\<%
\\
%
\>[6]\AgdaFunction{go}\AgdaSpace{}%
\AgdaSymbol{\{}\AgdaBound{x}\AgdaSpace{}%
\AgdaOperator{\AgdaInductiveConstructor{∷}}\AgdaSpace{}%
\AgdaBound{s}\AgdaSymbol{\}}\AgdaSpace{}%
\AgdaBound{n}\AgdaSpace{}%
\AgdaSymbol{=}\AgdaSpace{}%
\AgdaKeyword{do}\<%
\\
\>[6][@{}l@{\AgdaIndent{0}}]%
\>[8]\AgdaBound{c}\AgdaSpace{}%
\AgdaOperator{\AgdaField{←}}\AgdaSpace{}%
\AgdaFunction{get}\<%
\\
%
\>[8]\AgdaField{modify}\AgdaSpace{}%
\AgdaInductiveConstructor{suc}\<%
\\
%
\>[8]\AgdaBound{is}\AgdaSpace{}%
\AgdaOperator{\AgdaField{←}}\AgdaSpace{}%
\AgdaFunction{go}\AgdaSpace{}%
\AgdaSymbol{\{}\AgdaBound{s}\AgdaSymbol{\}}\AgdaSpace{}%
\AgdaBound{n}\<%
\\
%
\>[8]\AgdaFunction{return}\AgdaSpace{}%
\AgdaSymbol{(}\AgdaFunction{printf}\AgdaSpace{}%
\AgdaString{"\%s\AgdaUnderscore{}\%u"}\AgdaSpace{}%
\AgdaBound{n}\AgdaSpace{}%
\AgdaBound{c}\AgdaSpace{}%
\AgdaOperator{\AgdaInductiveConstructor{∷}}\AgdaSpace{}%
\AgdaBound{is}\AgdaSymbol{)}\<%
\\
%
\\[\AgdaEmptyExtraSkip]%
%
\>[2]\AgdaFunction{iv}\AgdaSpace{}%
\AgdaSymbol{:}\AgdaSpace{}%
\AgdaSymbol{(}\AgdaBound{s}\AgdaSpace{}%
\AgdaSymbol{:}\AgdaSpace{}%
\AgdaDatatype{S}\AgdaSymbol{)}\AgdaSpace{}%
\AgdaSymbol{→}\AgdaSpace{}%
\AgdaFunction{State}\AgdaSpace{}%
\AgdaDatatype{ℕ}\AgdaSpace{}%
\AgdaSymbol{(}\AgdaDatatype{Ix}\AgdaSpace{}%
\AgdaBound{s}\AgdaSymbol{)}\<%
\\
%
\>[2]\AgdaFunction{iv}\AgdaSpace{}%
\AgdaBound{s}\AgdaSpace{}%
\AgdaSymbol{=}\AgdaSpace{}%
\AgdaKeyword{do}\<%
\\
\>[2][@{}l@{\AgdaIndent{0}}]%
\>[4]\AgdaBound{c}\AgdaSpace{}%
\AgdaOperator{\AgdaField{←}}\AgdaSpace{}%
\AgdaFunction{get}\<%
\\
%
\>[4]\AgdaField{modify}\AgdaSpace{}%
\AgdaInductiveConstructor{suc}\<%
\\
%
\>[4]\AgdaFunction{return}\AgdaSpace{}%
\AgdaSymbol{(}\AgdaFunction{fresh-ix}\AgdaSpace{}%
\AgdaSymbol{(}\AgdaFunction{fresh-var}\AgdaSpace{}%
\AgdaBound{c}\AgdaSymbol{))}\<%
\\
%
\\[\AgdaEmptyExtraSkip]%
\>[0]\<%
\\
%
\>[2]\AgdaFunction{bop}\AgdaSpace{}%
\AgdaSymbol{:}\AgdaSpace{}%
\AgdaDatatype{Bop}\AgdaSpace{}%
\AgdaSymbol{->}\AgdaSpace{}%
\AgdaPostulate{String}\<%
\\
%
\>[2]\AgdaFunction{bop}\AgdaSpace{}%
\AgdaInductiveConstructor{plus}\AgdaSpace{}%
\AgdaSymbol{=}\AgdaSpace{}%
\AgdaString{"F.+"}\<%
\\
%
\>[2]\AgdaFunction{bop}\AgdaSpace{}%
\AgdaInductiveConstructor{mul}\AgdaSpace{}%
\AgdaSymbol{=}\AgdaSpace{}%
\AgdaString{"F.*"}\<%
\\
%
\\[\AgdaEmptyExtraSkip]%
%
\>[2]\AgdaFunction{show-array-type}\AgdaSpace{}%
\AgdaSymbol{:}\AgdaSpace{}%
\AgdaDatatype{S}\AgdaSpace{}%
\AgdaSymbol{→}\AgdaSpace{}%
\AgdaPostulate{String}\<%
\\
%
\>[2]\AgdaFunction{show-array-type}\AgdaSpace{}%
\AgdaInductiveConstructor{[]}\AgdaSpace{}%
\AgdaSymbol{=}\AgdaSpace{}%
\AgdaString{"f32"}\<%
\\
%
\>[2]\AgdaCatchallClause{\AgdaFunction{show-array-type}}\AgdaSpace{}%
\AgdaCatchallClause{\AgdaBound{s}}\AgdaSpace{}%
\AgdaSymbol{=}\AgdaSpace{}%
\AgdaFunction{printf}\AgdaSpace{}%
\AgdaString{"\%sf32"}\AgdaSpace{}%
\AgdaOperator{\AgdaFunction{\$}}\AgdaSpace{}%
\AgdaFunction{intersperse}\AgdaSpace{}%
\AgdaString{""}\AgdaSpace{}%
\AgdaOperator{\AgdaFunction{\$}}\AgdaSpace{}%
\AgdaFunction{L.map}\AgdaSpace{}%
\AgdaSymbol{(}\AgdaFunction{printf}\AgdaSpace{}%
\AgdaString{"[\%s]"}\AgdaSpace{}%
\AgdaOperator{\AgdaFunction{∘}}\AgdaSpace{}%
\AgdaFunction{show-nat}\AgdaSymbol{)}\AgdaSpace{}%
\AgdaSymbol{(}\AgdaFunction{s-list}\AgdaSpace{}%
\AgdaBound{s}\AgdaSymbol{)}\<%
\\
%
\\[\AgdaEmptyExtraSkip]%
%
\>[2]\AgdaOperator{\AgdaFunction{\AgdaUnderscore{}⊗ⁱ\AgdaUnderscore{}}}\AgdaSpace{}%
\AgdaSymbol{:}\AgdaSpace{}%
\AgdaDatatype{Ix}\AgdaSpace{}%
\AgdaGeneralizable{s}\AgdaSpace{}%
\AgdaSymbol{→}\AgdaSpace{}%
\AgdaDatatype{Ix}\AgdaSpace{}%
\AgdaGeneralizable{p}\AgdaSpace{}%
\AgdaSymbol{→}\AgdaSpace{}%
\AgdaDatatype{Ix}\AgdaSpace{}%
\AgdaSymbol{(}\AgdaGeneralizable{s}\AgdaSpace{}%
\AgdaOperator{\AgdaFunction{Ar.⊗}}\AgdaSpace{}%
\AgdaGeneralizable{p}\AgdaSymbol{)}\<%
\\
%
\>[2]\AgdaInductiveConstructor{[]}\AgdaSpace{}%
\AgdaOperator{\AgdaFunction{⊗ⁱ}}\AgdaSpace{}%
\AgdaBound{js}\AgdaSpace{}%
\AgdaSymbol{=}\AgdaSpace{}%
\AgdaBound{js}\<%
\\
%
\>[2]\AgdaSymbol{(}\AgdaBound{i}\AgdaSpace{}%
\AgdaOperator{\AgdaInductiveConstructor{∷}}\AgdaSpace{}%
\AgdaBound{is}\AgdaSymbol{)}\AgdaSpace{}%
\AgdaOperator{\AgdaFunction{⊗ⁱ}}\AgdaSpace{}%
\AgdaBound{js}\AgdaSpace{}%
\AgdaSymbol{=}\AgdaSpace{}%
\AgdaBound{i}\AgdaSpace{}%
\AgdaOperator{\AgdaInductiveConstructor{∷}}\AgdaSpace{}%
\AgdaSymbol{(}\AgdaBound{is}\AgdaSpace{}%
\AgdaOperator{\AgdaFunction{⊗ⁱ}}\AgdaSpace{}%
\AgdaBound{js}\AgdaSymbol{)}\<%
\\
\>[0]\<%
\\
%
\>[2]\AgdaFunction{splitⁱ}\AgdaSpace{}%
\AgdaSymbol{:}\AgdaSpace{}%
\AgdaSymbol{(}\AgdaBound{ij}\AgdaSpace{}%
\AgdaSymbol{:}\AgdaSpace{}%
\AgdaDatatype{Ix}\AgdaSpace{}%
\AgdaSymbol{(}\AgdaGeneralizable{s}\AgdaSpace{}%
\AgdaOperator{\AgdaFunction{Ar.⊗}}\AgdaSpace{}%
\AgdaGeneralizable{p}\AgdaSymbol{))}\AgdaSpace{}%
\AgdaSymbol{→}\AgdaSpace{}%
\AgdaRecord{Σ}\AgdaSpace{}%
\AgdaSymbol{(}\AgdaDatatype{Ix}\AgdaSpace{}%
\AgdaGeneralizable{s}\AgdaSymbol{)}\AgdaSpace{}%
\AgdaSymbol{λ}\AgdaSpace{}%
\AgdaBound{i}\AgdaSpace{}%
\AgdaSymbol{→}\AgdaSpace{}%
\AgdaRecord{Σ}\AgdaSpace{}%
\AgdaSymbol{(}\AgdaDatatype{Ix}\AgdaSpace{}%
\AgdaGeneralizable{p}\AgdaSymbol{)}\AgdaSpace{}%
\AgdaSymbol{λ}\AgdaSpace{}%
\AgdaBound{j}\AgdaSpace{}%
\AgdaSymbol{→}\AgdaSpace{}%
\AgdaBound{i}\AgdaSpace{}%
\AgdaOperator{\AgdaFunction{⊗ⁱ}}\AgdaSpace{}%
\AgdaBound{j}\AgdaSpace{}%
\AgdaOperator{\AgdaDatatype{≡}}\AgdaSpace{}%
\AgdaBound{ij}\<%
\\
%
\>[2]\AgdaFunction{splitⁱ}\AgdaSpace{}%
\AgdaSymbol{\{}\AgdaInductiveConstructor{[]}\AgdaSymbol{\}}\AgdaSpace{}%
\AgdaBound{ij}\AgdaSpace{}%
\AgdaSymbol{=}\AgdaSpace{}%
\AgdaInductiveConstructor{[]}\AgdaSpace{}%
\AgdaOperator{\AgdaInductiveConstructor{,}}\AgdaSpace{}%
\AgdaBound{ij}\AgdaSpace{}%
\AgdaOperator{\AgdaInductiveConstructor{,}}\AgdaSpace{}%
\AgdaInductiveConstructor{refl}\<%
\\
%
\>[2]\AgdaFunction{splitⁱ}\AgdaSpace{}%
\AgdaSymbol{\{\AgdaUnderscore{}}\AgdaSpace{}%
\AgdaOperator{\AgdaInductiveConstructor{∷}}\AgdaSpace{}%
\AgdaBound{s}\AgdaSymbol{\}}\AgdaSpace{}%
\AgdaSymbol{(}\AgdaBound{x}\AgdaSpace{}%
\AgdaOperator{\AgdaInductiveConstructor{∷}}\AgdaSpace{}%
\AgdaBound{ij}\AgdaSymbol{)}\AgdaSpace{}%
\AgdaKeyword{with}\AgdaSpace{}%
\AgdaFunction{splitⁱ}\AgdaSpace{}%
\AgdaSymbol{\{}\AgdaBound{s}\AgdaSymbol{\}}\AgdaSpace{}%
\AgdaBound{ij}\<%
\\
%
\>[2]\AgdaSymbol{...}\AgdaSpace{}%
\AgdaSymbol{|}\AgdaSpace{}%
\AgdaBound{i}\AgdaSpace{}%
\AgdaOperator{\AgdaInductiveConstructor{,}}\AgdaSpace{}%
\AgdaBound{j}\AgdaSpace{}%
\AgdaOperator{\AgdaInductiveConstructor{,}}\AgdaSpace{}%
\AgdaInductiveConstructor{refl}\AgdaSpace{}%
\AgdaSymbol{=}\AgdaSpace{}%
\AgdaSymbol{(}\AgdaBound{x}\AgdaSpace{}%
\AgdaOperator{\AgdaInductiveConstructor{∷}}\AgdaSpace{}%
\AgdaBound{i}\AgdaSymbol{)}\AgdaSpace{}%
\AgdaOperator{\AgdaInductiveConstructor{,}}\AgdaSpace{}%
\AgdaBound{j}\AgdaSpace{}%
\AgdaOperator{\AgdaInductiveConstructor{,}}\AgdaSpace{}%
\AgdaInductiveConstructor{refl}\<%
\\
%
\\[\AgdaEmptyExtraSkip]%
%
\>[2]\AgdaFunction{ix-curry}\AgdaSpace{}%
\AgdaSymbol{:}\AgdaSpace{}%
\AgdaSymbol{(}\AgdaDatatype{Ix}\AgdaSpace{}%
\AgdaSymbol{(}\AgdaGeneralizable{s}\AgdaSpace{}%
\AgdaOperator{\AgdaFunction{Ar.⊗}}\AgdaSpace{}%
\AgdaGeneralizable{p}\AgdaSymbol{)}\AgdaSpace{}%
\AgdaSymbol{→}\AgdaSpace{}%
\AgdaGeneralizable{X}\AgdaSymbol{)}\AgdaSpace{}%
\AgdaSymbol{→}\AgdaSpace{}%
\AgdaDatatype{Ix}\AgdaSpace{}%
\AgdaGeneralizable{s}\AgdaSpace{}%
\AgdaSymbol{→}\AgdaSpace{}%
\AgdaDatatype{Ix}\AgdaSpace{}%
\AgdaGeneralizable{p}\AgdaSpace{}%
\AgdaSymbol{→}\AgdaSpace{}%
\AgdaGeneralizable{X}\<%
\\
%
\>[2]\AgdaFunction{ix-curry}\AgdaSpace{}%
\AgdaBound{f}\AgdaSpace{}%
\AgdaBound{i}\AgdaSpace{}%
\AgdaBound{j}\AgdaSpace{}%
\AgdaSymbol{=}\AgdaSpace{}%
\AgdaBound{f}\AgdaSpace{}%
\AgdaSymbol{(}\AgdaBound{i}\AgdaSpace{}%
\AgdaOperator{\AgdaFunction{⊗ⁱ}}\AgdaSpace{}%
\AgdaBound{j}\AgdaSymbol{)}\<%
\\
%
\\[\AgdaEmptyExtraSkip]%
%
\>[2]\AgdaFunction{ix-uncurry}\AgdaSpace{}%
\AgdaSymbol{:}\AgdaSpace{}%
\AgdaSymbol{(}\AgdaDatatype{Ix}\AgdaSpace{}%
\AgdaGeneralizable{s}\AgdaSpace{}%
\AgdaSymbol{→}\AgdaSpace{}%
\AgdaDatatype{Ix}\AgdaSpace{}%
\AgdaGeneralizable{p}\AgdaSpace{}%
\AgdaSymbol{→}\AgdaSpace{}%
\AgdaGeneralizable{X}\AgdaSymbol{)}\AgdaSpace{}%
\AgdaSymbol{→}\AgdaSpace{}%
\AgdaDatatype{Ix}\AgdaSpace{}%
\AgdaSymbol{(}\AgdaGeneralizable{s}\AgdaSpace{}%
\AgdaOperator{\AgdaFunction{Ar.⊗}}\AgdaSpace{}%
\AgdaGeneralizable{p}\AgdaSymbol{)}\AgdaSpace{}%
\AgdaSymbol{→}\AgdaSpace{}%
\AgdaGeneralizable{X}\<%
\\
%
\>[2]\AgdaFunction{ix-uncurry}\AgdaSpace{}%
\AgdaSymbol{\{}\AgdaArgument{s}\AgdaSpace{}%
\AgdaSymbol{=}\AgdaSpace{}%
\AgdaBound{s}\AgdaSymbol{\}}\AgdaSpace{}%
\AgdaBound{f}\AgdaSpace{}%
\AgdaBound{ij}\AgdaSpace{}%
\AgdaKeyword{with}\AgdaSpace{}%
\AgdaFunction{splitⁱ}\AgdaSpace{}%
\AgdaSymbol{\{}\AgdaBound{s}\AgdaSymbol{\}}\AgdaSpace{}%
\AgdaBound{ij}\<%
\\
%
\>[2]\AgdaSymbol{...}\AgdaSpace{}%
\AgdaSymbol{|}\AgdaSpace{}%
\AgdaBound{i}\AgdaSpace{}%
\AgdaOperator{\AgdaInductiveConstructor{,}}\AgdaSpace{}%
\AgdaBound{j}\AgdaSpace{}%
\AgdaOperator{\AgdaInductiveConstructor{,}}\AgdaSpace{}%
\AgdaInductiveConstructor{refl}\AgdaSpace{}%
\AgdaSymbol{=}\AgdaSpace{}%
\AgdaBound{f}\AgdaSpace{}%
\AgdaBound{i}\AgdaSpace{}%
\AgdaBound{j}\<%
\\
%
\\[\AgdaEmptyExtraSkip]%
%
\>[2]\AgdaFunction{ix-map}\AgdaSpace{}%
\AgdaSymbol{:}\AgdaSpace{}%
\AgdaSymbol{(}\AgdaPostulate{String}\AgdaSpace{}%
\AgdaSymbol{→}\AgdaSpace{}%
\AgdaPostulate{String}\AgdaSymbol{)}\AgdaSpace{}%
\AgdaSymbol{→}\AgdaSpace{}%
\AgdaDatatype{Ix}\AgdaSpace{}%
\AgdaGeneralizable{s}\AgdaSpace{}%
\AgdaSymbol{→}\AgdaSpace{}%
\AgdaDatatype{Ix}\AgdaSpace{}%
\AgdaGeneralizable{s}\<%
\\
%
\>[2]\AgdaFunction{ix-map}\AgdaSpace{}%
\AgdaBound{f}\AgdaSpace{}%
\AgdaInductiveConstructor{[]}\AgdaSpace{}%
\AgdaSymbol{=}\AgdaSpace{}%
\AgdaInductiveConstructor{[]}\<%
\\
%
\>[2]\AgdaFunction{ix-map}\AgdaSpace{}%
\AgdaBound{f}\AgdaSpace{}%
\AgdaSymbol{(}\AgdaBound{x}\AgdaSpace{}%
\AgdaOperator{\AgdaInductiveConstructor{∷}}\AgdaSpace{}%
\AgdaBound{i}\AgdaSymbol{)}\AgdaSpace{}%
\AgdaSymbol{=}\AgdaSpace{}%
\AgdaBound{f}\AgdaSpace{}%
\AgdaBound{x}\AgdaSpace{}%
\AgdaOperator{\AgdaInductiveConstructor{∷}}\AgdaSpace{}%
\AgdaFunction{ix-map}\AgdaSpace{}%
\AgdaBound{f}\AgdaSpace{}%
\AgdaBound{i}\<%
\\
\>[0]\<%
\\
%
\>[2]\AgdaFunction{ix-zipwith}\AgdaSpace{}%
\AgdaSymbol{:}\AgdaSpace{}%
\AgdaSymbol{((}\AgdaBound{a}\AgdaSpace{}%
\AgdaBound{b}\AgdaSpace{}%
\AgdaSymbol{:}\AgdaSpace{}%
\AgdaPostulate{String}\AgdaSymbol{)}\AgdaSpace{}%
\AgdaSymbol{→}\AgdaSpace{}%
\AgdaPostulate{String}\AgdaSymbol{)}\AgdaSpace{}%
\AgdaSymbol{→}\AgdaSpace{}%
\AgdaDatatype{Ix}\AgdaSpace{}%
\AgdaGeneralizable{s}\AgdaSpace{}%
\AgdaSymbol{→}\AgdaSpace{}%
\AgdaDatatype{Ix}\AgdaSpace{}%
\AgdaGeneralizable{s}\AgdaSpace{}%
\AgdaSymbol{→}\AgdaSpace{}%
\AgdaDatatype{Ix}\AgdaSpace{}%
\AgdaGeneralizable{s}\<%
\\
%
\>[2]\AgdaFunction{ix-zipwith}\AgdaSpace{}%
\AgdaBound{f}\AgdaSpace{}%
\AgdaInductiveConstructor{[]}\AgdaSpace{}%
\AgdaInductiveConstructor{[]}\AgdaSpace{}%
\AgdaSymbol{=}\AgdaSpace{}%
\AgdaInductiveConstructor{[]}\<%
\\
%
\>[2]\AgdaFunction{ix-zipwith}\AgdaSpace{}%
\AgdaBound{f}\AgdaSpace{}%
\AgdaSymbol{(}\AgdaBound{x}\AgdaSpace{}%
\AgdaOperator{\AgdaInductiveConstructor{∷}}\AgdaSpace{}%
\AgdaBound{i}\AgdaSymbol{)}\AgdaSpace{}%
\AgdaSymbol{(}\AgdaBound{y}\AgdaSpace{}%
\AgdaOperator{\AgdaInductiveConstructor{∷}}\AgdaSpace{}%
\AgdaBound{j}\AgdaSymbol{)}\AgdaSpace{}%
\AgdaSymbol{=}\AgdaSpace{}%
\AgdaBound{f}\AgdaSpace{}%
\AgdaBound{x}\AgdaSpace{}%
\AgdaBound{y}\AgdaSpace{}%
\AgdaOperator{\AgdaInductiveConstructor{∷}}\AgdaSpace{}%
\AgdaFunction{ix-zipwith}\AgdaSpace{}%
\AgdaBound{f}\AgdaSpace{}%
\AgdaBound{i}\AgdaSpace{}%
\AgdaBound{j}\<%
\\
%
\\[\AgdaEmptyExtraSkip]%
%
\\[\AgdaEmptyExtraSkip]%
%
\>[2]\AgdaFunction{ix-join}\AgdaSpace{}%
\AgdaSymbol{:}\AgdaSpace{}%
\AgdaDatatype{Ix}\AgdaSpace{}%
\AgdaGeneralizable{s}\AgdaSpace{}%
\AgdaSymbol{→}\AgdaSpace{}%
\AgdaPostulate{String}\AgdaSpace{}%
\AgdaSymbol{→}\AgdaSpace{}%
\AgdaPostulate{String}\<%
\\
%
\>[2]\AgdaFunction{ix-join}\AgdaSpace{}%
\AgdaInductiveConstructor{[]}\AgdaSpace{}%
\AgdaBound{d}\AgdaSpace{}%
\AgdaSymbol{=}\AgdaSpace{}%
\AgdaString{""}\<%
\\
%
\>[2]\AgdaFunction{ix-join}\AgdaSpace{}%
\AgdaSymbol{(}\AgdaBound{x}\AgdaSpace{}%
\AgdaOperator{\AgdaInductiveConstructor{∷}}\AgdaSpace{}%
\AgdaInductiveConstructor{[]}\AgdaSymbol{)}\AgdaSpace{}%
\AgdaBound{d}\AgdaSpace{}%
\AgdaSymbol{=}\AgdaSpace{}%
\AgdaBound{x}\<%
\\
%
\>[2]\AgdaFunction{ix-join}\AgdaSpace{}%
\AgdaSymbol{\{}\AgdaArgument{s}\AgdaSpace{}%
\AgdaSymbol{=}\AgdaSpace{}%
\AgdaSymbol{\AgdaUnderscore{}}\AgdaSpace{}%
\AgdaOperator{\AgdaInductiveConstructor{∷}}\AgdaSpace{}%
\AgdaBound{s}\AgdaSymbol{\}}\AgdaSpace{}%
\AgdaSymbol{(}\AgdaBound{x}\AgdaSpace{}%
\AgdaOperator{\AgdaInductiveConstructor{∷}}\AgdaSpace{}%
\AgdaBound{y}\AgdaSpace{}%
\AgdaOperator{\AgdaInductiveConstructor{∷}}\AgdaSpace{}%
\AgdaBound{xs}\AgdaSymbol{)}\AgdaSpace{}%
\AgdaBound{d}\AgdaSpace{}%
\AgdaSymbol{=}\AgdaSpace{}%
\AgdaBound{x}\AgdaSpace{}%
\AgdaOperator{\AgdaFunction{++}}\AgdaSpace{}%
\AgdaBound{d}\AgdaSpace{}%
\AgdaOperator{\AgdaFunction{++}}\AgdaSpace{}%
\AgdaFunction{ix-join}\AgdaSpace{}%
\AgdaSymbol{\{}\AgdaBound{s}\AgdaSymbol{\}}\AgdaSpace{}%
\AgdaSymbol{(}\AgdaBound{y}\AgdaSpace{}%
\AgdaOperator{\AgdaInductiveConstructor{∷}}\AgdaSpace{}%
\AgdaBound{xs}\AgdaSymbol{)}\AgdaSpace{}%
\AgdaBound{d}\<%
\\
%
\\[\AgdaEmptyExtraSkip]%
%
\>[2]\AgdaFunction{ix-to-list}\AgdaSpace{}%
\AgdaSymbol{:}\AgdaSpace{}%
\AgdaDatatype{Ix}\AgdaSpace{}%
\AgdaGeneralizable{s}\AgdaSpace{}%
\AgdaSymbol{→}\AgdaSpace{}%
\AgdaDatatype{List}\AgdaSpace{}%
\AgdaPostulate{String}\<%
\\
%
\>[2]\AgdaFunction{ix-to-list}\AgdaSpace{}%
\AgdaInductiveConstructor{[]}\AgdaSpace{}%
\AgdaSymbol{=}\AgdaSpace{}%
\AgdaInductiveConstructor{[]}\<%
\\
%
\>[2]\AgdaFunction{ix-to-list}\AgdaSpace{}%
\AgdaSymbol{(}\AgdaBound{x}\AgdaSpace{}%
\AgdaOperator{\AgdaInductiveConstructor{∷}}\AgdaSpace{}%
\AgdaBound{xs}\AgdaSymbol{)}\AgdaSpace{}%
\AgdaSymbol{=}\AgdaSpace{}%
\AgdaBound{x}\AgdaSpace{}%
\AgdaOperator{\AgdaInductiveConstructor{∷}}\AgdaSpace{}%
\AgdaFunction{ix-to-list}\AgdaSpace{}%
\AgdaBound{xs}\<%
\\
%
\\[\AgdaEmptyExtraSkip]%
%
\\[\AgdaEmptyExtraSkip]%
%
\>[2]\AgdaFunction{to-sel}\AgdaSpace{}%
\AgdaSymbol{:}\AgdaSpace{}%
\AgdaDatatype{Ix}\AgdaSpace{}%
\AgdaGeneralizable{s}\AgdaSpace{}%
\AgdaSymbol{→}\AgdaSpace{}%
\AgdaPostulate{String}\AgdaSpace{}%
\AgdaSymbol{→}\AgdaSpace{}%
\AgdaPostulate{String}\<%
\\
%
\>[2]\AgdaFunction{to-sel}\AgdaSpace{}%
\AgdaBound{i}\AgdaSpace{}%
\AgdaBound{a}\AgdaSpace{}%
\AgdaSymbol{=}\AgdaSpace{}%
\AgdaBound{a}\AgdaSpace{}%
\AgdaOperator{\AgdaFunction{++}}\AgdaSpace{}%
\AgdaFunction{ix-join}\AgdaSpace{}%
\AgdaSymbol{(}\AgdaFunction{ix-map}\AgdaSpace{}%
\AgdaSymbol{(}\AgdaFunction{printf}\AgdaSpace{}%
\AgdaString{"[\%s]"}\AgdaSymbol{)}\AgdaSpace{}%
\AgdaBound{i}\AgdaSymbol{)}\AgdaSpace{}%
\AgdaString{""}\<%
\\
%
\\[\AgdaEmptyExtraSkip]%
%
\\[\AgdaEmptyExtraSkip]%
%
\>[2]\AgdaFunction{to-imap}\AgdaSpace{}%
\AgdaSymbol{:}\AgdaSpace{}%
\AgdaSymbol{(}\AgdaBound{s}\AgdaSpace{}%
\AgdaSymbol{:}\AgdaSpace{}%
\AgdaDatatype{S}\AgdaSymbol{)}\AgdaSpace{}%
\AgdaSymbol{→}\AgdaSpace{}%
\AgdaSymbol{(}\AgdaBound{i}\AgdaSpace{}%
\AgdaSymbol{:}\AgdaSpace{}%
\AgdaDatatype{Ix}\AgdaSpace{}%
\AgdaBound{s}\AgdaSymbol{)}\AgdaSpace{}%
\AgdaSymbol{→}\AgdaSpace{}%
\AgdaSymbol{(}\AgdaBound{e}\AgdaSpace{}%
\AgdaSymbol{:}\AgdaSpace{}%
\AgdaPostulate{String}\AgdaSymbol{)}\AgdaSpace{}%
\AgdaSymbol{→}\AgdaSpace{}%
\AgdaPostulate{String}\<%
\\
%
\>[2]\AgdaFunction{to-imap}\AgdaSpace{}%
\AgdaBound{s}\AgdaSpace{}%
\AgdaBound{i}\AgdaSpace{}%
\AgdaBound{e}\AgdaSpace{}%
\AgdaSymbol{=}%
\>[1742I]\AgdaFunction{printf}\AgdaSpace{}%
\AgdaString{"(imap\%u\ \%s\ (\textbackslash{}\textbackslash{}\ \%s\ ->\ \%s))"}\<%
\\
\>[1742I][@{}l@{\AgdaIndent{0}}]%
\>[19]\AgdaSymbol{(}\AgdaFunction{dim}\AgdaSpace{}%
\AgdaBound{s}\AgdaSymbol{)}\AgdaSpace{}%
\AgdaSymbol{(}\AgdaFunction{shape-args}\AgdaSpace{}%
\AgdaBound{s}\AgdaSymbol{)}\AgdaSpace{}%
\AgdaSymbol{(}\AgdaFunction{ix-join}\AgdaSpace{}%
\AgdaBound{i}\AgdaSpace{}%
\AgdaString{"\ "}\AgdaSymbol{)}\<%
\\
%
\>[19]\AgdaBound{e}\<%
\\
%
\>[2]\AgdaComment{--to-sum\ :\ (s\ :\ S)\ →\ (i\ :\ Ix\ s)\ →\ (e\ :\ String)\ →\ String}\<%
\\
%
\>[2]\AgdaComment{--to-sum\ []\ i\ e\ =\ e}\<%
\\
%
\>[2]\AgdaComment{--to-sum\ s\ \ i\ e\ =\ printf\ "(sum\%ud\ \%s)"\ (dim\ s)\ (to-imap\ s\ i\ e)}\<%
\\
%
\\[\AgdaEmptyExtraSkip]%
%
\>[2]\AgdaFunction{to-sum}\AgdaSpace{}%
\AgdaSymbol{:}\AgdaSpace{}%
\AgdaSymbol{(}\AgdaBound{s}\AgdaSpace{}%
\AgdaSymbol{:}\AgdaSpace{}%
\AgdaDatatype{S}\AgdaSymbol{)}\AgdaSpace{}%
\AgdaSymbol{→}\AgdaSpace{}%
\AgdaSymbol{(}\AgdaBound{i}\AgdaSpace{}%
\AgdaSymbol{:}\AgdaSpace{}%
\AgdaDatatype{Ix}\AgdaSpace{}%
\AgdaBound{s}\AgdaSymbol{)}\AgdaSpace{}%
\AgdaSymbol{→}\AgdaSpace{}%
\AgdaSymbol{(}\AgdaBound{e}\AgdaSpace{}%
\AgdaSymbol{:}\AgdaSpace{}%
\AgdaPostulate{String}\AgdaSymbol{)}\AgdaSpace{}%
\AgdaSymbol{→}\AgdaSpace{}%
\AgdaPostulate{String}\<%
\\
%
\>[2]\AgdaFunction{to-sum}\AgdaSpace{}%
\AgdaInductiveConstructor{[]}\AgdaSpace{}%
\AgdaBound{i}\AgdaSpace{}%
\AgdaBound{e}\AgdaSpace{}%
\AgdaSymbol{=}\AgdaSpace{}%
\AgdaBound{e}\<%
\\
%
\>[2]\AgdaCatchallClause{\AgdaFunction{to-sum}}\AgdaSpace{}%
\AgdaCatchallClause{\AgdaBound{s}}%
\>[12]\AgdaCatchallClause{\AgdaBound{i}}\AgdaSpace{}%
\AgdaCatchallClause{\AgdaBound{e}}\AgdaSpace{}%
\AgdaSymbol{=}\AgdaSpace{}%
\AgdaFunction{printf}%
\>[1774I]\AgdaString{"(isum\%u\ \%s\ (\textbackslash{}\textbackslash{}\ \%s\ ->\ \%s))"}\AgdaSpace{}%
\AgdaSymbol{(}\AgdaFunction{dim}\AgdaSpace{}%
\AgdaBound{s}\AgdaSymbol{)}\AgdaSpace{}%
\AgdaSymbol{(}\AgdaFunction{shape-args}\AgdaSpace{}%
\AgdaBound{s}\AgdaSymbol{)}\<%
\\
\>[.][@{}l@{}]\<[1774I]%
\>[25]\AgdaSymbol{(}\AgdaFunction{ix-join}\AgdaSpace{}%
\AgdaBound{i}\AgdaSpace{}%
\AgdaString{"\ "}\AgdaSymbol{)}\AgdaSpace{}%
\AgdaBound{e}\<%
\\
%
\\[\AgdaEmptyExtraSkip]%
%
\>[2]\AgdaFunction{ix-plus}%
\>[1782I]\AgdaSymbol{:}\AgdaSpace{}%
\AgdaGeneralizable{s}\AgdaSpace{}%
\AgdaOperator{\AgdaFunction{+}}\AgdaSpace{}%
\AgdaGeneralizable{p}\AgdaSpace{}%
\AgdaOperator{\AgdaFunction{≈}}\AgdaSpace{}%
\AgdaGeneralizable{r}\AgdaSpace{}%
\AgdaSymbol{→}\AgdaSpace{}%
\AgdaSymbol{(}\AgdaOperator{\AgdaFunction{suc\AgdaUnderscore{}≈\AgdaUnderscore{}}}\AgdaSpace{}%
\AgdaGeneralizable{p}\AgdaSpace{}%
\AgdaGeneralizable{u}\AgdaSymbol{)}\<%
\\
\>[.][@{}l@{}]\<[1782I]%
\>[10]\AgdaSymbol{→}\AgdaSpace{}%
\AgdaSymbol{(}\AgdaBound{i}\AgdaSpace{}%
\AgdaSymbol{:}\AgdaSpace{}%
\AgdaDatatype{Ix}\AgdaSpace{}%
\AgdaGeneralizable{s}\AgdaSymbol{)}\<%
\\
%
\>[10]\AgdaSymbol{→}\AgdaSpace{}%
\AgdaSymbol{(}\AgdaBound{j}\AgdaSpace{}%
\AgdaSymbol{:}\AgdaSpace{}%
\AgdaDatatype{Ix}\AgdaSpace{}%
\AgdaGeneralizable{u}\AgdaSymbol{)}\<%
\\
%
\>[10]\AgdaSymbol{→}\AgdaSpace{}%
\AgdaDatatype{Ix}\AgdaSpace{}%
\AgdaGeneralizable{r}\<%
\\
%
\>[2]\AgdaFunction{ix-plus}\AgdaSpace{}%
\AgdaInductiveConstructor{[]}%
\>[14]\AgdaInductiveConstructor{[]}\AgdaSpace{}%
\AgdaInductiveConstructor{[]}\AgdaSpace{}%
\AgdaInductiveConstructor{[]}\AgdaSpace{}%
\AgdaSymbol{=}\AgdaSpace{}%
\AgdaInductiveConstructor{[]}\<%
\\
%
\>[2]\AgdaFunction{ix-plus}\AgdaSpace{}%
\AgdaSymbol{(}\AgdaInductiveConstructor{cons}\AgdaSpace{}%
\AgdaSymbol{⦃}\AgdaSpace{}%
\AgdaSymbol{\AgdaUnderscore{}}\AgdaSpace{}%
\AgdaSymbol{⦄}\AgdaSpace{}%
\AgdaSymbol{⦃}\AgdaSpace{}%
\AgdaBound{s+p}\AgdaSpace{}%
\AgdaSymbol{⦄)}\AgdaSpace{}%
\AgdaSymbol{(}\AgdaInductiveConstructor{cons}\AgdaSpace{}%
\AgdaSymbol{⦃}\AgdaSpace{}%
\AgdaSymbol{\AgdaUnderscore{}}\AgdaSpace{}%
\AgdaSymbol{⦄}\AgdaSpace{}%
\AgdaSymbol{⦃}\AgdaSpace{}%
\AgdaBound{sp}\AgdaSpace{}%
\AgdaSymbol{⦄)}\AgdaSpace{}%
\AgdaSymbol{(}\AgdaBound{i}\AgdaSpace{}%
\AgdaOperator{\AgdaInductiveConstructor{∷}}\AgdaSpace{}%
\AgdaBound{is}\AgdaSymbol{)}\AgdaSpace{}%
\AgdaSymbol{(}\AgdaBound{j}\AgdaSpace{}%
\AgdaOperator{\AgdaInductiveConstructor{∷}}\AgdaSpace{}%
\AgdaBound{js}\AgdaSymbol{)}\AgdaSpace{}%
\AgdaSymbol{=}\<%
\\
\>[2][@{}l@{\AgdaIndent{0}}]%
\>[4]\AgdaFunction{printf}\AgdaSpace{}%
\AgdaString{"(\%s\ +\ \%s)"}\AgdaSpace{}%
\AgdaBound{i}\AgdaSpace{}%
\AgdaBound{j}\AgdaSpace{}%
\AgdaOperator{\AgdaInductiveConstructor{∷}}\AgdaSpace{}%
\AgdaFunction{ix-plus}\AgdaSpace{}%
\AgdaBound{s+p}\AgdaSpace{}%
\AgdaBound{sp}\AgdaSpace{}%
\AgdaBound{is}\AgdaSpace{}%
\AgdaBound{js}\<%
\\
%
\\[\AgdaEmptyExtraSkip]%
%
\>[2]\AgdaFunction{ix-eq}\AgdaSpace{}%
\AgdaSymbol{:}\AgdaSpace{}%
\AgdaSymbol{(}\AgdaBound{i}\AgdaSpace{}%
\AgdaBound{j}\AgdaSpace{}%
\AgdaSymbol{:}\AgdaSpace{}%
\AgdaDatatype{Ix}\AgdaSpace{}%
\AgdaGeneralizable{s}\AgdaSymbol{)}\AgdaSpace{}%
\AgdaSymbol{→}\AgdaSpace{}%
\AgdaPostulate{String}\<%
\\
%
\>[2]\AgdaFunction{ix-eq}\AgdaSpace{}%
\AgdaBound{i}\AgdaSpace{}%
\AgdaBound{j}\AgdaSpace{}%
\AgdaSymbol{=}\AgdaSpace{}%
\AgdaFunction{ix-join}\AgdaSpace{}%
\AgdaSymbol{(}\AgdaFunction{ix-zipwith}\AgdaSpace{}%
\AgdaSymbol{(}\AgdaFunction{printf}\AgdaSpace{}%
\AgdaString{"(\%s\ ==\ \%s)"}\AgdaSymbol{)}\AgdaSpace{}%
\AgdaBound{i}\AgdaSpace{}%
\AgdaBound{j}\AgdaSymbol{)}\AgdaSpace{}%
\AgdaString{"\ \&\&\ "}\<%
\\
%
\\[\AgdaEmptyExtraSkip]%
%
\>[2]\AgdaFunction{ix-minus}%
\>[1855I]\AgdaSymbol{:}\AgdaSpace{}%
\AgdaGeneralizable{s}\AgdaSpace{}%
\AgdaOperator{\AgdaFunction{+}}\AgdaSpace{}%
\AgdaGeneralizable{p}\AgdaSpace{}%
\AgdaOperator{\AgdaFunction{≈}}\AgdaSpace{}%
\AgdaGeneralizable{r}\AgdaSpace{}%
\AgdaSymbol{→}\AgdaSpace{}%
\AgdaSymbol{(}\AgdaOperator{\AgdaFunction{suc\AgdaUnderscore{}≈\AgdaUnderscore{}}}\AgdaSpace{}%
\AgdaGeneralizable{p}\AgdaSpace{}%
\AgdaGeneralizable{u}\AgdaSymbol{)}\<%
\\
\>[.][@{}l@{}]\<[1855I]%
\>[11]\AgdaSymbol{→}\AgdaSpace{}%
\AgdaSymbol{(}\AgdaBound{i}\AgdaSpace{}%
\AgdaSymbol{:}\AgdaSpace{}%
\AgdaDatatype{Ix}\AgdaSpace{}%
\AgdaGeneralizable{r}\AgdaSymbol{)}\<%
\\
%
\>[11]\AgdaSymbol{→}\AgdaSpace{}%
\AgdaSymbol{(}\AgdaBound{j}\AgdaSpace{}%
\AgdaSymbol{:}\AgdaSpace{}%
\AgdaDatatype{Ix}\AgdaSpace{}%
\AgdaGeneralizable{s}\AgdaSymbol{)}\<%
\\
%
\>[11]\AgdaSymbol{→}\AgdaSpace{}%
\AgdaDatatype{Ix}\AgdaSpace{}%
\AgdaGeneralizable{u}\<%
\\
%
\>[2]\AgdaFunction{ix-minus}\AgdaSpace{}%
\AgdaInductiveConstructor{[]}%
\>[15]\AgdaInductiveConstructor{[]}\AgdaSpace{}%
\AgdaInductiveConstructor{[]}\AgdaSpace{}%
\AgdaInductiveConstructor{[]}\AgdaSpace{}%
\AgdaSymbol{=}\AgdaSpace{}%
\AgdaInductiveConstructor{[]}\<%
\\
%
\>[2]\AgdaFunction{ix-minus}\AgdaSpace{}%
\AgdaSymbol{(}\AgdaInductiveConstructor{cons}\AgdaSpace{}%
\AgdaSymbol{⦃}\AgdaSpace{}%
\AgdaSymbol{\AgdaUnderscore{}}\AgdaSpace{}%
\AgdaSymbol{⦄}\AgdaSpace{}%
\AgdaSymbol{⦃}\AgdaSpace{}%
\AgdaBound{s+p}\AgdaSpace{}%
\AgdaSymbol{⦄)}\AgdaSpace{}%
\AgdaSymbol{(}\AgdaInductiveConstructor{cons}\AgdaSpace{}%
\AgdaSymbol{⦃}\AgdaSpace{}%
\AgdaSymbol{\AgdaUnderscore{}}\AgdaSpace{}%
\AgdaSymbol{⦄}\AgdaSpace{}%
\AgdaSymbol{⦃}\AgdaSpace{}%
\AgdaBound{sp}\AgdaSpace{}%
\AgdaSymbol{⦄)}\AgdaSpace{}%
\AgdaSymbol{(}\AgdaBound{i}\AgdaSpace{}%
\AgdaOperator{\AgdaInductiveConstructor{∷}}\AgdaSpace{}%
\AgdaBound{is}\AgdaSymbol{)}\AgdaSpace{}%
\AgdaSymbol{(}\AgdaBound{j}\AgdaSpace{}%
\AgdaOperator{\AgdaInductiveConstructor{∷}}\AgdaSpace{}%
\AgdaBound{js}\AgdaSymbol{)}\AgdaSpace{}%
\AgdaSymbol{=}\<%
\\
\>[2][@{}l@{\AgdaIndent{0}}]%
\>[4]\AgdaFunction{printf}\AgdaSpace{}%
\AgdaString{"(\%s\ -\ \%s)"}\AgdaSpace{}%
\AgdaBound{i}\AgdaSpace{}%
\AgdaBound{j}\AgdaSpace{}%
\AgdaOperator{\AgdaInductiveConstructor{∷}}\AgdaSpace{}%
\AgdaFunction{ix-minus}\AgdaSpace{}%
\AgdaBound{s+p}\AgdaSpace{}%
\AgdaBound{sp}\AgdaSpace{}%
\AgdaBound{is}\AgdaSpace{}%
\AgdaBound{js}\<%
\\
%
\\[\AgdaEmptyExtraSkip]%
%
\\[\AgdaEmptyExtraSkip]%
%
\>[2]\AgdaFunction{to-div-mod}%
\>[1910I]\AgdaSymbol{:}\AgdaSpace{}%
\AgdaGeneralizable{s}\AgdaSpace{}%
\AgdaOperator{\AgdaFunction{*}}\AgdaSpace{}%
\AgdaGeneralizable{p}\AgdaSpace{}%
\AgdaOperator{\AgdaFunction{≈}}\AgdaSpace{}%
\AgdaGeneralizable{q}\AgdaSpace{}%
\AgdaSymbol{→}\AgdaSpace{}%
\AgdaDatatype{Ix}\AgdaSpace{}%
\AgdaGeneralizable{q}\<%
\\
\>[.][@{}l@{}]\<[1910I]%
\>[13]\AgdaSymbol{→}\AgdaSpace{}%
\AgdaDatatype{Ix}\AgdaSpace{}%
\AgdaGeneralizable{s}\AgdaSpace{}%
\AgdaOperator{\AgdaFunction{×}}\AgdaSpace{}%
\AgdaDatatype{Ix}\AgdaSpace{}%
\AgdaGeneralizable{p}\<%
\\
%
\>[2]\AgdaFunction{to-div-mod}\AgdaSpace{}%
\AgdaInductiveConstructor{[]}%
\>[18]\AgdaInductiveConstructor{[]}\AgdaSpace{}%
\AgdaSymbol{=}\AgdaSpace{}%
\AgdaInductiveConstructor{[]}\AgdaSpace{}%
\AgdaOperator{\AgdaInductiveConstructor{,}}\AgdaSpace{}%
\AgdaInductiveConstructor{[]}\<%
\\
%
\>[2]\AgdaFunction{to-div-mod}\AgdaSpace{}%
\AgdaSymbol{(}\AgdaInductiveConstructor{cons}\AgdaSpace{}%
\AgdaSymbol{\{}\AgdaArgument{n}\AgdaSpace{}%
\AgdaSymbol{=}\AgdaSpace{}%
\AgdaBound{n}\AgdaSymbol{\}}\AgdaSpace{}%
\AgdaSymbol{⦃}\AgdaSpace{}%
\AgdaSymbol{\AgdaUnderscore{}}\AgdaSpace{}%
\AgdaSymbol{⦄}\AgdaSpace{}%
\AgdaSymbol{⦃}\AgdaSpace{}%
\AgdaBound{eq}\AgdaSpace{}%
\AgdaSymbol{⦄)}\AgdaSpace{}%
\AgdaSymbol{(}\AgdaBound{x}\AgdaSpace{}%
\AgdaOperator{\AgdaInductiveConstructor{∷}}\AgdaSpace{}%
\AgdaBound{i}\AgdaSymbol{)}\AgdaSpace{}%
\AgdaSymbol{=}\<%
\\
\>[2][@{}l@{\AgdaIndent{0}}]%
\>[4]\AgdaComment{--\ (i:\ Fin\ (m*n))\ →\ [p,q]\ :\ Fin\ [m,n]\ =>\ p=i/n\ q=i\%n}\<%
\\
%
\>[4]\AgdaFunction{Prod.map}%
\>[1943I]\AgdaSymbol{(}\AgdaFunction{printf}\AgdaSpace{}%
\AgdaString{"(\%s\ /\ \%s)"}\AgdaSpace{}%
\AgdaBound{x}\AgdaSpace{}%
\AgdaSymbol{(}\AgdaFunction{show-nat}\AgdaSpace{}%
\AgdaBound{n}\AgdaSymbol{)}\AgdaSpace{}%
\AgdaOperator{\AgdaInductiveConstructor{∷\AgdaUnderscore{}}}\AgdaSymbol{)}\<%
\\
\>[.][@{}l@{}]\<[1943I]%
\>[13]\AgdaSymbol{(}\AgdaFunction{printf}\AgdaSpace{}%
\AgdaString{"(\%s\ \%\%\ \%s)"}\AgdaSpace{}%
\AgdaBound{x}\AgdaSpace{}%
\AgdaSymbol{(}\AgdaFunction{show-nat}\AgdaSpace{}%
\AgdaBound{n}\AgdaSymbol{)}\AgdaSpace{}%
\AgdaOperator{\AgdaInductiveConstructor{∷\AgdaUnderscore{}}}\AgdaSymbol{)}\<%
\\
%
\>[13]\AgdaSymbol{(}\AgdaFunction{to-div-mod}\AgdaSpace{}%
\AgdaBound{eq}\AgdaSpace{}%
\AgdaBound{i}\AgdaSymbol{)}\<%
\\
%
\\[\AgdaEmptyExtraSkip]%
%
\>[2]\AgdaFunction{from-div-mod}%
\>[1956I]\AgdaSymbol{:}\AgdaSpace{}%
\AgdaGeneralizable{s}\AgdaSpace{}%
\AgdaOperator{\AgdaFunction{*}}\AgdaSpace{}%
\AgdaGeneralizable{p}\AgdaSpace{}%
\AgdaOperator{\AgdaFunction{≈}}\AgdaSpace{}%
\AgdaGeneralizable{q}\<%
\\
\>[.][@{}l@{}]\<[1956I]%
\>[15]\AgdaSymbol{→}\AgdaSpace{}%
\AgdaDatatype{Ix}\AgdaSpace{}%
\AgdaGeneralizable{s}\AgdaSpace{}%
\AgdaSymbol{→}\AgdaSpace{}%
\AgdaDatatype{Ix}\AgdaSpace{}%
\AgdaGeneralizable{p}\<%
\\
%
\>[15]\AgdaSymbol{→}\AgdaSpace{}%
\AgdaDatatype{Ix}\AgdaSpace{}%
\AgdaGeneralizable{q}\<%
\\
%
\>[2]\AgdaFunction{from-div-mod}\AgdaSpace{}%
\AgdaInductiveConstructor{[]}\AgdaSpace{}%
\AgdaInductiveConstructor{[]}\AgdaSpace{}%
\AgdaInductiveConstructor{[]}\AgdaSpace{}%
\AgdaSymbol{=}\AgdaSpace{}%
\AgdaInductiveConstructor{[]}\<%
\\
%
\>[2]\AgdaFunction{from-div-mod}\AgdaSpace{}%
\AgdaSymbol{(}\AgdaInductiveConstructor{cons}\AgdaSpace{}%
\AgdaSymbol{\{}\AgdaArgument{n}\AgdaSpace{}%
\AgdaSymbol{=}\AgdaSpace{}%
\AgdaBound{n}\AgdaSymbol{\}}\AgdaSpace{}%
\AgdaSymbol{⦃}\AgdaSpace{}%
\AgdaSymbol{\AgdaUnderscore{}}\AgdaSpace{}%
\AgdaSymbol{⦄}\AgdaSpace{}%
\AgdaSymbol{⦃}\AgdaSpace{}%
\AgdaBound{eq}\AgdaSpace{}%
\AgdaSymbol{⦄)}\AgdaSpace{}%
\AgdaSymbol{(}\AgdaBound{i}\AgdaSpace{}%
\AgdaOperator{\AgdaInductiveConstructor{∷}}\AgdaSpace{}%
\AgdaBound{is}\AgdaSymbol{)}\AgdaSpace{}%
\AgdaSymbol{(}\AgdaBound{j}\AgdaSpace{}%
\AgdaOperator{\AgdaInductiveConstructor{∷}}\AgdaSpace{}%
\AgdaBound{js}\AgdaSymbol{)}\AgdaSpace{}%
\AgdaSymbol{=}\<%
\\
\>[2][@{}l@{\AgdaIndent{0}}]%
\>[4]\AgdaComment{--\ (i\ :\ Fin\ m)\ (j\ :\ Fin\ n)\ \ (k\ :\ Fin\ (m\ *\ n))\ \ k\ =\ i\ *\ n\ +\ j\ \ }\<%
\\
%
\>[4]\AgdaFunction{printf}\AgdaSpace{}%
\AgdaString{"((\%s\ *\ \%s)\ +\ \%s)"}\AgdaSpace{}%
\AgdaBound{i}\AgdaSpace{}%
\AgdaSymbol{(}\AgdaFunction{show-nat}\AgdaSpace{}%
\AgdaBound{n}\AgdaSymbol{)}\AgdaSpace{}%
\AgdaBound{j}\<%
\\
%
\>[4]\AgdaOperator{\AgdaInductiveConstructor{∷}}\AgdaSpace{}%
\AgdaFunction{from-div-mod}\AgdaSpace{}%
\AgdaBound{eq}\AgdaSpace{}%
\AgdaBound{is}\AgdaSpace{}%
\AgdaBound{js}\<%
\\
%
\\[\AgdaEmptyExtraSkip]%
%
\>[2]\AgdaComment{--\ Generate\ a\ new\ name\ for\ an\ external\ array}\<%
\\
%
\>[2]\AgdaFunction{mkar}\AgdaSpace{}%
\AgdaSymbol{:}\AgdaSpace{}%
\AgdaPostulate{String}\AgdaSpace{}%
\AgdaSymbol{→}\AgdaSpace{}%
\AgdaDatatype{Ix}\AgdaSpace{}%
\AgdaGeneralizable{s}\AgdaSpace{}%
\AgdaSymbol{→}\AgdaSpace{}%
\AgdaFunction{State}\AgdaSpace{}%
\AgdaDatatype{ℕ}\AgdaSpace{}%
\AgdaSymbol{((}\AgdaPostulate{String}\AgdaSpace{}%
\AgdaSymbol{→}\AgdaSpace{}%
\AgdaPostulate{String}\AgdaSymbol{)}\AgdaSpace{}%
\AgdaOperator{\AgdaFunction{×}}\AgdaSpace{}%
\AgdaPostulate{String}\AgdaSymbol{)}\<%
\\
%
\>[2]\AgdaFunction{mkar}\AgdaSpace{}%
\AgdaBound{a}\AgdaSpace{}%
\AgdaBound{i}\AgdaSpace{}%
\AgdaSymbol{=}\AgdaSpace{}%
\AgdaFunction{return}\AgdaSpace{}%
\AgdaSymbol{(}\AgdaFunction{id}\AgdaSpace{}%
\AgdaOperator{\AgdaInductiveConstructor{,}}\AgdaSpace{}%
\AgdaFunction{to-sel}\AgdaSpace{}%
\AgdaBound{i}\AgdaSpace{}%
\AgdaBound{a}\AgdaSymbol{)}\<%
\end{code}
\begin{code}%
%
\>[2]\AgdaFunction{to-fut}\AgdaSpace{}%
\AgdaSymbol{:}\AgdaSpace{}%
\AgdaDatatype{E}\AgdaSpace{}%
\AgdaGeneralizable{Γ}\AgdaSpace{}%
\AgdaGeneralizable{is}\AgdaSpace{}%
\AgdaSymbol{→}\AgdaSpace{}%
\AgdaFunction{FEnv}\AgdaSpace{}%
\AgdaGeneralizable{Γ}\AgdaSpace{}%
\AgdaSymbol{→}\AgdaSpace{}%
\AgdaFunction{State}\AgdaSpace{}%
\AgdaDatatype{ℕ}\AgdaSpace{}%
\AgdaSymbol{(}\AgdaFunction{Sem}\AgdaSpace{}%
\AgdaGeneralizable{is}\AgdaSymbol{)}\<%
\\
%
\>[2]\AgdaFunction{to-str}\AgdaSpace{}%
\AgdaSymbol{:}\AgdaSpace{}%
\AgdaDatatype{E}\AgdaSpace{}%
\AgdaGeneralizable{Γ}\AgdaSpace{}%
\AgdaSymbol{(}\AgdaInductiveConstructor{ar}\AgdaSpace{}%
\AgdaGeneralizable{s}\AgdaSymbol{)}\AgdaSpace{}%
\AgdaSymbol{→}\AgdaSpace{}%
\AgdaFunction{FEnv}\AgdaSpace{}%
\AgdaGeneralizable{Γ}\AgdaSpace{}%
\AgdaSymbol{→}\AgdaSpace{}%
\AgdaFunction{State}\AgdaSpace{}%
\AgdaDatatype{ℕ}\AgdaSpace{}%
\AgdaPostulate{String}\<%
\end{code}}
\end{mathpar}
Consider two cases of \AF{to-fut} for \AC{imap} an \AC{sel}.
In both cases the array we are constructing or selecting from is
of shape $s ++ p$.  We use two helper functions \AF{ix-curry}
and \AF{ix-uncurry} that translate between functions of type
\AD{Ix (s ++ p)} → X and \AD{Ix} s → \AD{Ix p} → X.  In the
\AC{imap} case we generate a function keeping potential let
chains within the imap expression.  In case of \AF{sel}, we
are computing the value of the array we are selecting from (i.e. $a$)
and within the returned expression we apply $a$ to the corresponding
indices --- this is normalisation step.
\begin{mathpar}
\codeblock{\begin{code}%
%
\>[2]\AgdaFunction{to-fut}\AgdaSpace{}%
\AgdaSymbol{(}\AgdaInductiveConstructor{imap}\AgdaSpace{}%
\AgdaSymbol{\{}\AgdaArgument{s}\AgdaSpace{}%
\AgdaSymbol{=}\AgdaSpace{}%
\AgdaBound{s}\AgdaSymbol{\}}\AgdaSpace{}%
\AgdaBound{e}\AgdaSymbol{)}\AgdaSpace{}%
\AgdaBound{ρ}\AgdaSpace{}%
\AgdaSymbol{=}\<%
\\
\>[2][@{}l@{\AgdaIndent{0}}]%
\>[4]\AgdaFunction{return}\AgdaSpace{}%
\AgdaOperator{\AgdaFunction{\$}}\AgdaSpace{}%
\AgdaFunction{ix-uncurry}\AgdaSpace{}%
\AgdaSymbol{\{}\AgdaBound{s}\AgdaSymbol{\}}\AgdaSpace{}%
\AgdaSymbol{λ}\AgdaSpace{}%
\AgdaBound{i}\AgdaSpace{}%
\AgdaBound{j}\AgdaSpace{}%
\AgdaSymbol{→}\AgdaSpace{}%
\AgdaKeyword{do}\<%
\\
\>[4][@{}l@{\AgdaIndent{0}}]%
\>[6]\AgdaBound{b}\AgdaSpace{}%
\AgdaOperator{\AgdaField{←}}\AgdaSpace{}%
\AgdaFunction{to-fut}\AgdaSpace{}%
\AgdaBound{e}\AgdaSpace{}%
\AgdaSymbol{(}\AgdaBound{ρ}\AgdaSpace{}%
\AgdaOperator{\AgdaInductiveConstructor{,}}\AgdaSpace{}%
\AgdaBound{i}\AgdaSymbol{)}\<%
\\
%
\>[6]\AgdaBound{f}\AgdaSpace{}%
\AgdaOperator{\AgdaInductiveConstructor{,}}\AgdaSpace{}%
\AgdaBound{b′}\AgdaSpace{}%
\AgdaOperator{\AgdaField{←}}\AgdaSpace{}%
\AgdaBound{b}\AgdaSpace{}%
\AgdaBound{j}\<%
\\
%
\>[6]\AgdaFunction{return}\AgdaSpace{}%
\AgdaSymbol{(}\AgdaFunction{id}\AgdaSpace{}%
\AgdaOperator{\AgdaInductiveConstructor{,}}\AgdaSpace{}%
\AgdaBound{f}\AgdaSpace{}%
\AgdaBound{b′}\AgdaSymbol{)}\<%
\end{code}}
\and
\codeblock{\begin{code}%
%
\>[2]\AgdaFunction{to-fut}\AgdaSpace{}%
\AgdaSymbol{(}\AgdaInductiveConstructor{sel}\AgdaSpace{}%
\AgdaBound{e}\AgdaSpace{}%
\AgdaBound{e₁}\AgdaSymbol{)}\AgdaSpace{}%
\AgdaBound{ρ}\AgdaSpace{}%
\AgdaSymbol{=}\AgdaSpace{}%
\AgdaKeyword{do}\<%
\\
\>[2][@{}l@{\AgdaIndent{0}}]%
\>[5]\AgdaBound{a}\AgdaSpace{}%
\AgdaOperator{\AgdaField{←}}\AgdaSpace{}%
\AgdaFunction{to-fut}\AgdaSpace{}%
\AgdaBound{e}\AgdaSpace{}%
\AgdaBound{ρ}\<%
\\
%
\>[5]\AgdaBound{i}\AgdaSpace{}%
\AgdaOperator{\AgdaField{←}}\AgdaSpace{}%
\AgdaFunction{to-fut}\AgdaSpace{}%
\AgdaBound{e₁}\AgdaSpace{}%
\AgdaBound{ρ}\<%
\\
%
\>[5]\AgdaFunction{return}\AgdaSpace{}%
\AgdaSymbol{λ}\AgdaSpace{}%
\AgdaBound{j}\AgdaSpace{}%
\AgdaSymbol{→}\AgdaSpace{}%
\AgdaKeyword{do}\<%
\\
\>[5][@{}l@{\AgdaIndent{0}}]%
\>[7]\AgdaBound{f}\AgdaSpace{}%
\AgdaOperator{\AgdaInductiveConstructor{,}}\AgdaSpace{}%
\AgdaBound{a′}\AgdaSpace{}%
\AgdaOperator{\AgdaField{←}}\AgdaSpace{}%
\AgdaFunction{ix-curry}\AgdaSpace{}%
\AgdaBound{a}\AgdaSpace{}%
\AgdaBound{i}\AgdaSpace{}%
\AgdaBound{j}\<%
\\
%
\>[7]\AgdaFunction{return}\AgdaSpace{}%
\AgdaSymbol{(}\AgdaBound{f}\AgdaSpace{}%
\AgdaOperator{\AgdaInductiveConstructor{,}}\AgdaSpace{}%
\AgdaBound{a′}\AgdaSymbol{)}\<%
\end{code}}
\end{mathpar}
\begin{code}[hide]%
%
\>[2]\AgdaFunction{to-fut}\AgdaSpace{}%
\AgdaSymbol{(}\AgdaInductiveConstructor{var}\AgdaSpace{}%
\AgdaBound{x}\AgdaSymbol{)}\AgdaSpace{}%
\AgdaBound{ρ}\AgdaSpace{}%
\AgdaSymbol{=}\AgdaSpace{}%
\AgdaFunction{return}\AgdaSpace{}%
\AgdaOperator{\AgdaFunction{\$}}\AgdaSpace{}%
\AgdaFunction{lookup}\AgdaSpace{}%
\AgdaBound{x}\AgdaSpace{}%
\AgdaBound{ρ}\<%
\\
%
\>[2]\AgdaFunction{to-fut}\AgdaSpace{}%
\AgdaInductiveConstructor{zero}\AgdaSpace{}%
\AgdaBound{ρ}\AgdaSpace{}%
\AgdaSymbol{=}\AgdaSpace{}%
\AgdaFunction{return}\AgdaSpace{}%
\AgdaSymbol{(λ}\AgdaSpace{}%
\AgdaBound{\AgdaUnderscore{}}\AgdaSpace{}%
\AgdaSymbol{→}\AgdaSpace{}%
\AgdaFunction{return}\AgdaSpace{}%
\AgdaSymbol{(}\AgdaFunction{id}\AgdaSpace{}%
\AgdaOperator{\AgdaInductiveConstructor{,}}\AgdaSpace{}%
\AgdaString{"zero"}\AgdaSymbol{))}\<%
\\
%
\>[2]\AgdaFunction{to-fut}\AgdaSpace{}%
\AgdaInductiveConstructor{one}\AgdaSpace{}%
\AgdaBound{ρ}\AgdaSpace{}%
\AgdaSymbol{=}\AgdaSpace{}%
\AgdaFunction{return}\AgdaSpace{}%
\AgdaSymbol{(λ}\AgdaSpace{}%
\AgdaBound{\AgdaUnderscore{}}\AgdaSpace{}%
\AgdaSymbol{→}\AgdaSpace{}%
\AgdaFunction{return}\AgdaSpace{}%
\AgdaSymbol{(}\AgdaFunction{id}\AgdaSpace{}%
\AgdaOperator{\AgdaInductiveConstructor{,}}\AgdaSpace{}%
\AgdaString{"one"}\AgdaSymbol{))}\<%
\\
%
\>[2]\AgdaFunction{to-fut}\AgdaSpace{}%
\AgdaSymbol{(}\AgdaInductiveConstructor{imaps}\AgdaSpace{}%
\AgdaBound{e}\AgdaSymbol{)}\AgdaSpace{}%
\AgdaBound{ρ}\AgdaSpace{}%
\AgdaSymbol{=}\AgdaSpace{}%
\AgdaFunction{return}\AgdaSpace{}%
\AgdaSymbol{λ}\AgdaSpace{}%
\AgdaBound{i}\AgdaSpace{}%
\AgdaSymbol{→}\AgdaSpace{}%
\AgdaKeyword{do}\<%
\\
\>[2][@{}l@{\AgdaIndent{0}}]%
\>[5]\AgdaBound{b}\AgdaSpace{}%
\AgdaOperator{\AgdaField{←}}\AgdaSpace{}%
\AgdaFunction{to-fut}\AgdaSpace{}%
\AgdaBound{e}\AgdaSpace{}%
\AgdaSymbol{(}\AgdaBound{ρ}\AgdaSpace{}%
\AgdaOperator{\AgdaInductiveConstructor{,}}\AgdaSpace{}%
\AgdaBound{i}\AgdaSymbol{)}\<%
\\
%
\>[5]\AgdaBound{f}\AgdaSpace{}%
\AgdaOperator{\AgdaInductiveConstructor{,}}\AgdaSpace{}%
\AgdaBound{b′}\AgdaSpace{}%
\AgdaOperator{\AgdaField{←}}\AgdaSpace{}%
\AgdaBound{b}\AgdaSpace{}%
\AgdaInductiveConstructor{[]}\<%
\\
%
\>[5]\AgdaFunction{return}\AgdaSpace{}%
\AgdaSymbol{(}\AgdaFunction{id}\AgdaSpace{}%
\AgdaOperator{\AgdaInductiveConstructor{,}}\AgdaSpace{}%
\AgdaBound{f}\AgdaSpace{}%
\AgdaBound{b′}\AgdaSymbol{)}\<%
\\
%
\\[\AgdaEmptyExtraSkip]%
%
\>[5]\AgdaComment{--λ\ i\ →\ let\ k\ =\ to-fut\ e\ (ρ\ ,\ i)\ ;\ r\ =\ (\AgdaUnderscore{}\$\ [])\ <\$>\ k\ in\ join\ r}\<%
\\
%
\>[2]\AgdaFunction{to-fut}\AgdaSpace{}%
\AgdaSymbol{(}\AgdaInductiveConstructor{sels}\AgdaSpace{}%
\AgdaBound{e}\AgdaSpace{}%
\AgdaBound{e₁}\AgdaSymbol{)}\AgdaSpace{}%
\AgdaBound{ρ}\AgdaSpace{}%
\AgdaSymbol{=}\AgdaSpace{}%
\AgdaKeyword{do}\<%
\\
\>[2][@{}l@{\AgdaIndent{0}}]%
\>[5]\AgdaBound{a}\AgdaSpace{}%
\AgdaOperator{\AgdaField{←}}\AgdaSpace{}%
\AgdaFunction{to-fut}\AgdaSpace{}%
\AgdaBound{e}\AgdaSpace{}%
\AgdaBound{ρ}\<%
\\
%
\>[5]\AgdaBound{x}\AgdaSpace{}%
\AgdaOperator{\AgdaField{←}}\AgdaSpace{}%
\AgdaFunction{to-fut}\AgdaSpace{}%
\AgdaBound{e₁}\AgdaSpace{}%
\AgdaBound{ρ}\<%
\\
%
\>[5]\AgdaFunction{return}\AgdaSpace{}%
\AgdaSymbol{λ}\AgdaSpace{}%
\AgdaBound{i}\AgdaSpace{}%
\AgdaSymbol{→}\AgdaSpace{}%
\AgdaKeyword{do}\<%
\\
\>[5][@{}l@{\AgdaIndent{0}}]%
\>[7]\AgdaBound{f}\AgdaSpace{}%
\AgdaOperator{\AgdaInductiveConstructor{,}}\AgdaSpace{}%
\AgdaBound{a′}\AgdaSpace{}%
\AgdaOperator{\AgdaField{←}}\AgdaSpace{}%
\AgdaBound{a}\AgdaSpace{}%
\AgdaBound{x}\<%
\\
%
\>[7]\AgdaFunction{return}\AgdaSpace{}%
\AgdaSymbol{(}\AgdaBound{f}\AgdaSpace{}%
\AgdaOperator{\AgdaInductiveConstructor{,}}\AgdaSpace{}%
\AgdaBound{a′}\AgdaSymbol{)}\<%
\\
%
\>[5]\AgdaComment{--return\ λ\ \AgdaUnderscore{}\ →\ f\ x}\<%
\\
%
\>[2]\AgdaFunction{to-fut}\AgdaSpace{}%
\AgdaSymbol{(}\AgdaInductiveConstructor{E.imapb}\AgdaSpace{}%
\AgdaBound{x}\AgdaSpace{}%
\AgdaBound{e}\AgdaSymbol{)}\AgdaSpace{}%
\AgdaBound{ρ}\AgdaSpace{}%
\AgdaSymbol{=}\AgdaSpace{}%
\AgdaFunction{return}\AgdaSpace{}%
\AgdaSymbol{λ}\AgdaSpace{}%
\AgdaBound{i}\AgdaSpace{}%
\AgdaSymbol{→}\AgdaSpace{}%
\AgdaKeyword{do}\<%
\\
\>[2][@{}l@{\AgdaIndent{0}}]%
\>[4]\AgdaKeyword{let}\AgdaSpace{}%
\AgdaBound{j}\AgdaSpace{}%
\AgdaOperator{\AgdaInductiveConstructor{,}}\AgdaSpace{}%
\AgdaBound{k}\AgdaSpace{}%
\AgdaSymbol{=}\AgdaSpace{}%
\AgdaFunction{to-div-mod}\AgdaSpace{}%
\AgdaBound{x}\AgdaSpace{}%
\AgdaBound{i}\<%
\\
%
\>[4]\AgdaBound{b}\AgdaSpace{}%
\AgdaOperator{\AgdaField{←}}\AgdaSpace{}%
\AgdaFunction{to-fut}\AgdaSpace{}%
\AgdaBound{e}\AgdaSpace{}%
\AgdaSymbol{(}\AgdaBound{ρ}\AgdaSpace{}%
\AgdaOperator{\AgdaInductiveConstructor{,}}\AgdaSpace{}%
\AgdaBound{j}\AgdaSymbol{)}\<%
\\
%
\>[4]\AgdaBound{f}\AgdaSpace{}%
\AgdaOperator{\AgdaInductiveConstructor{,}}\AgdaSpace{}%
\AgdaBound{b′}\AgdaSpace{}%
\AgdaOperator{\AgdaField{←}}\AgdaSpace{}%
\AgdaBound{b}\AgdaSpace{}%
\AgdaBound{k}\<%
\\
%
\>[4]\AgdaFunction{return}\AgdaSpace{}%
\AgdaSymbol{(}\AgdaFunction{id}\AgdaSpace{}%
\AgdaOperator{\AgdaInductiveConstructor{,}}\AgdaSpace{}%
\AgdaBound{f}\AgdaSpace{}%
\AgdaBound{b′}\AgdaSymbol{)}\<%
\\
%
\>[2]\AgdaFunction{to-fut}\AgdaSpace{}%
\AgdaSymbol{(}\AgdaInductiveConstructor{E.selb}\AgdaSpace{}%
\AgdaBound{x}\AgdaSpace{}%
\AgdaBound{e}\AgdaSpace{}%
\AgdaBound{e₁}\AgdaSymbol{)}\AgdaSpace{}%
\AgdaBound{ρ}\AgdaSpace{}%
\AgdaSymbol{=}\AgdaSpace{}%
\AgdaKeyword{do}\<%
\\
\>[2][@{}l@{\AgdaIndent{0}}]%
\>[4]\AgdaBound{a}\AgdaSpace{}%
\AgdaOperator{\AgdaField{←}}\AgdaSpace{}%
\AgdaFunction{to-fut}\AgdaSpace{}%
\AgdaBound{e}\AgdaSpace{}%
\AgdaBound{ρ}\<%
\\
%
\>[4]\AgdaBound{i}\AgdaSpace{}%
\AgdaOperator{\AgdaField{←}}\AgdaSpace{}%
\AgdaFunction{to-fut}\AgdaSpace{}%
\AgdaBound{e₁}\AgdaSpace{}%
\AgdaBound{ρ}\<%
\\
%
\>[4]\AgdaFunction{return}\AgdaSpace{}%
\AgdaSymbol{λ}\AgdaSpace{}%
\AgdaBound{j}\AgdaSpace{}%
\AgdaSymbol{→}\AgdaSpace{}%
\AgdaKeyword{do}\<%
\\
\>[4][@{}l@{\AgdaIndent{0}}]%
\>[6]\AgdaKeyword{let}\AgdaSpace{}%
\AgdaBound{k}\AgdaSpace{}%
\AgdaSymbol{=}\AgdaSpace{}%
\AgdaFunction{from-div-mod}\AgdaSpace{}%
\AgdaBound{x}\AgdaSpace{}%
\AgdaBound{i}\AgdaSpace{}%
\AgdaBound{j}\<%
\\
%
\>[6]\AgdaBound{f}\AgdaSpace{}%
\AgdaOperator{\AgdaInductiveConstructor{,}}\AgdaSpace{}%
\AgdaBound{a′}\AgdaSpace{}%
\AgdaOperator{\AgdaField{←}}\AgdaSpace{}%
\AgdaBound{a}\AgdaSpace{}%
\AgdaBound{k}\<%
\\
%
\>[6]\AgdaFunction{return}\AgdaSpace{}%
\AgdaSymbol{(}\AgdaBound{f}\AgdaSpace{}%
\AgdaOperator{\AgdaInductiveConstructor{,}}\AgdaSpace{}%
\AgdaBound{a′}\AgdaSymbol{)}\<%
\\
%
\>[2]\AgdaFunction{to-fut}\AgdaSpace{}%
\AgdaSymbol{(}\AgdaInductiveConstructor{E.sum}\AgdaSpace{}%
\AgdaSymbol{\{}\AgdaArgument{s}\AgdaSpace{}%
\AgdaSymbol{=}\AgdaSpace{}%
\AgdaBound{s}\AgdaSymbol{\}}\AgdaSpace{}%
\AgdaBound{e}\AgdaSymbol{)}\AgdaSpace{}%
\AgdaBound{ρ}\AgdaSpace{}%
\AgdaSymbol{=}\AgdaSpace{}%
\AgdaKeyword{do}\<%
\\
\>[2][@{}l@{\AgdaIndent{0}}]%
\>[4]\AgdaBound{i}\AgdaSpace{}%
\AgdaOperator{\AgdaField{←}}\AgdaSpace{}%
\AgdaFunction{iv}\AgdaSpace{}%
\AgdaBound{s}\<%
\\
%
\>[4]\AgdaBound{b}\AgdaSpace{}%
\AgdaOperator{\AgdaField{←}}\AgdaSpace{}%
\AgdaFunction{to-fut}\AgdaSpace{}%
\AgdaBound{e}\AgdaSpace{}%
\AgdaSymbol{(}\AgdaBound{ρ}\AgdaSpace{}%
\AgdaOperator{\AgdaInductiveConstructor{,}}\AgdaSpace{}%
\AgdaBound{i}\AgdaSymbol{)}\<%
\\
%
\>[4]\AgdaFunction{return}\AgdaSpace{}%
\AgdaSymbol{λ}\AgdaSpace{}%
\AgdaBound{j}\AgdaSpace{}%
\AgdaSymbol{→}\AgdaSpace{}%
\AgdaKeyword{do}\<%
\\
\>[4][@{}l@{\AgdaIndent{0}}]%
\>[6]\AgdaBound{f}\AgdaSpace{}%
\AgdaOperator{\AgdaInductiveConstructor{,}}\AgdaSpace{}%
\AgdaBound{b′}\AgdaSpace{}%
\AgdaOperator{\AgdaField{←}}\AgdaSpace{}%
\AgdaBound{b}\AgdaSpace{}%
\AgdaBound{j}\<%
\\
%
\>[6]\AgdaFunction{return}\AgdaSpace{}%
\AgdaSymbol{(}\AgdaFunction{id}\AgdaSpace{}%
\AgdaOperator{\AgdaInductiveConstructor{,}}\AgdaSpace{}%
\AgdaFunction{to-sum}\AgdaSpace{}%
\AgdaBound{s}\AgdaSpace{}%
\AgdaBound{i}\AgdaSpace{}%
\AgdaSymbol{(}\AgdaBound{f}\AgdaSpace{}%
\AgdaBound{b′}\AgdaSymbol{))}\<%
\\
%
\>[2]\AgdaFunction{to-fut}\AgdaSpace{}%
\AgdaSymbol{(}\AgdaInductiveConstructor{zero-but}\AgdaSpace{}%
\AgdaBound{e}\AgdaSpace{}%
\AgdaBound{e₁}\AgdaSpace{}%
\AgdaBound{e₂}\AgdaSymbol{)}\AgdaSpace{}%
\AgdaBound{ρ}\AgdaSpace{}%
\AgdaSymbol{=}\AgdaSpace{}%
\AgdaKeyword{do}\<%
\\
\>[2][@{}l@{\AgdaIndent{0}}]%
\>[4]\AgdaBound{i}\AgdaSpace{}%
\AgdaOperator{\AgdaField{←}}\AgdaSpace{}%
\AgdaFunction{to-fut}\AgdaSpace{}%
\AgdaBound{e}\AgdaSpace{}%
\AgdaBound{ρ}\<%
\\
%
\>[4]\AgdaBound{j}\AgdaSpace{}%
\AgdaOperator{\AgdaField{←}}\AgdaSpace{}%
\AgdaFunction{to-fut}\AgdaSpace{}%
\AgdaBound{e₁}\AgdaSpace{}%
\AgdaBound{ρ}\<%
\\
%
\>[4]\AgdaBound{a}\AgdaSpace{}%
\AgdaOperator{\AgdaField{←}}\AgdaSpace{}%
\AgdaFunction{to-fut}\AgdaSpace{}%
\AgdaBound{e₂}\AgdaSpace{}%
\AgdaBound{ρ}\<%
\\
%
\>[4]\AgdaFunction{return}\AgdaSpace{}%
\AgdaSymbol{λ}\AgdaSpace{}%
\AgdaBound{k}\AgdaSpace{}%
\AgdaSymbol{→}\AgdaSpace{}%
\AgdaKeyword{do}\<%
\\
\>[4][@{}l@{\AgdaIndent{0}}]%
\>[6]\AgdaBound{f}\AgdaSpace{}%
\AgdaOperator{\AgdaInductiveConstructor{,}}\AgdaSpace{}%
\AgdaBound{a′}\AgdaSpace{}%
\AgdaOperator{\AgdaField{←}}\AgdaSpace{}%
\AgdaBound{a}\AgdaSpace{}%
\AgdaBound{k}\<%
\\
%
\>[6]\AgdaComment{--\ move\ context\ under\ if,\ so\ that\ we\ do\ not\ evaluate\ stuff\ that\ we\ do\ not\ need.}\<%
\\
%
\>[6]\AgdaFunction{return}\AgdaSpace{}%
\AgdaSymbol{(}\AgdaFunction{id}\AgdaSpace{}%
\AgdaOperator{\AgdaInductiveConstructor{,}}\AgdaSpace{}%
\AgdaFunction{printf}\AgdaSpace{}%
\AgdaString{"(if\ (\%s)\ then\ \%s\ else\ zero)"}\AgdaSpace{}%
\AgdaSymbol{(}\AgdaFunction{ix-eq}\AgdaSpace{}%
\AgdaBound{i}\AgdaSpace{}%
\AgdaBound{j}\AgdaSymbol{)}\AgdaSpace{}%
\AgdaSymbol{(}\AgdaBound{f}\AgdaSpace{}%
\AgdaBound{a′}\AgdaSymbol{))}\<%
\\
%
\>[2]\AgdaFunction{to-fut}\AgdaSpace{}%
\AgdaSymbol{(}\AgdaInductiveConstructor{E.slide}\AgdaSpace{}%
\AgdaBound{e}\AgdaSpace{}%
\AgdaBound{x}\AgdaSpace{}%
\AgdaBound{e₁}\AgdaSpace{}%
\AgdaBound{x₁}\AgdaSymbol{)}\AgdaSpace{}%
\AgdaBound{ρ}\AgdaSpace{}%
\AgdaSymbol{=}\AgdaSpace{}%
\AgdaKeyword{do}\<%
\\
\>[2][@{}l@{\AgdaIndent{0}}]%
\>[4]\AgdaBound{i}\AgdaSpace{}%
\AgdaOperator{\AgdaField{←}}\AgdaSpace{}%
\AgdaFunction{to-fut}\AgdaSpace{}%
\AgdaBound{e}\AgdaSpace{}%
\AgdaBound{ρ}\<%
\\
%
\>[4]\AgdaBound{a}\AgdaSpace{}%
\AgdaOperator{\AgdaField{←}}\AgdaSpace{}%
\AgdaFunction{to-fut}\AgdaSpace{}%
\AgdaBound{e₁}\AgdaSpace{}%
\AgdaBound{ρ}\<%
\\
%
\>[4]\AgdaFunction{return}\AgdaSpace{}%
\AgdaSymbol{λ}\AgdaSpace{}%
\AgdaBound{j}\AgdaSpace{}%
\AgdaSymbol{→}\AgdaSpace{}%
\AgdaKeyword{do}\<%
\\
\>[4][@{}l@{\AgdaIndent{0}}]%
\>[6]\AgdaBound{f}\AgdaSpace{}%
\AgdaOperator{\AgdaInductiveConstructor{,}}\AgdaSpace{}%
\AgdaBound{a′}\AgdaSpace{}%
\AgdaOperator{\AgdaField{←}}\AgdaSpace{}%
\AgdaBound{a}\AgdaSpace{}%
\AgdaSymbol{(}\AgdaFunction{ix-plus}\AgdaSpace{}%
\AgdaBound{x}\AgdaSpace{}%
\AgdaBound{x₁}\AgdaSpace{}%
\AgdaBound{i}\AgdaSpace{}%
\AgdaBound{j}\AgdaSymbol{)}\<%
\\
%
\>[6]\AgdaFunction{return}\AgdaSpace{}%
\AgdaSymbol{(}\AgdaBound{f}\AgdaSpace{}%
\AgdaOperator{\AgdaInductiveConstructor{,}}\AgdaSpace{}%
\AgdaBound{a′}\AgdaSymbol{)}\<%
\\
%
\>[2]\AgdaFunction{to-fut}\AgdaSpace{}%
\AgdaSymbol{(}\AgdaInductiveConstructor{E.backslide}\AgdaSpace{}%
\AgdaSymbol{\{}\AgdaArgument{u}\AgdaSpace{}%
\AgdaSymbol{=}\AgdaSpace{}%
\AgdaBound{u}\AgdaSymbol{\}}\AgdaSpace{}%
\AgdaBound{e}\AgdaSpace{}%
\AgdaBound{e₁}\AgdaSpace{}%
\AgdaBound{x}\AgdaSpace{}%
\AgdaBound{x₁}\AgdaSymbol{)}\AgdaSpace{}%
\AgdaBound{ρ}\AgdaSpace{}%
\AgdaSymbol{=}\AgdaSpace{}%
\AgdaKeyword{do}\<%
\\
\>[2][@{}l@{\AgdaIndent{0}}]%
\>[4]\AgdaBound{i}\AgdaSpace{}%
\AgdaOperator{\AgdaField{←}}\AgdaSpace{}%
\AgdaFunction{to-fut}\AgdaSpace{}%
\AgdaBound{e}\AgdaSpace{}%
\AgdaBound{ρ}\<%
\\
%
\>[4]\AgdaBound{a}\AgdaSpace{}%
\AgdaOperator{\AgdaField{←}}\AgdaSpace{}%
\AgdaFunction{to-fut}\AgdaSpace{}%
\AgdaBound{e₁}\AgdaSpace{}%
\AgdaBound{ρ}\<%
\\
%
\>[4]\AgdaFunction{return}\AgdaSpace{}%
\AgdaSymbol{λ}\AgdaSpace{}%
\AgdaBound{j}\AgdaSpace{}%
\AgdaSymbol{→}\AgdaSpace{}%
\AgdaKeyword{do}\<%
\\
\>[4][@{}l@{\AgdaIndent{0}}]%
\>[6]\AgdaKeyword{let}\AgdaSpace{}%
\AgdaBound{j-i}\AgdaSpace{}%
\AgdaSymbol{=}\AgdaSpace{}%
\AgdaFunction{ix-minus}\AgdaSpace{}%
\AgdaBound{x₁}\AgdaSpace{}%
\AgdaBound{x}\AgdaSpace{}%
\AgdaBound{j}\AgdaSpace{}%
\AgdaBound{i}\<%
\\
%
\>[6]\AgdaKeyword{let}\AgdaSpace{}%
\AgdaBound{j≥i}\AgdaSpace{}%
\AgdaSymbol{=}\AgdaSpace{}%
\AgdaFunction{intersperse}\AgdaSpace{}%
\AgdaString{"\ \&\&\ "}\AgdaSpace{}%
\AgdaSymbol{(}\AgdaFunction{L.zipWith}\AgdaSpace{}%
\AgdaSymbol{(}\AgdaFunction{printf}\AgdaSpace{}%
\AgdaString{"\%s\ >=\ \%s"}\AgdaSymbol{)}\AgdaSpace{}%
\AgdaSymbol{(}\AgdaFunction{ix-to-list}\AgdaSpace{}%
\AgdaBound{j}\AgdaSymbol{)}\AgdaSpace{}%
\AgdaSymbol{(}\AgdaFunction{ix-to-list}\AgdaSpace{}%
\AgdaBound{i}\AgdaSymbol{))}\<%
\\
%
\>[6]\AgdaKeyword{let}\AgdaSpace{}%
\AgdaBound{j-i<u}\AgdaSpace{}%
\AgdaSymbol{=}\AgdaSpace{}%
\AgdaFunction{intersperse}\AgdaSpace{}%
\AgdaString{"\ \&\&\ "}\AgdaSpace{}%
\AgdaSymbol{(}\AgdaFunction{L.zipWith}\AgdaSpace{}%
\AgdaSymbol{(}\AgdaFunction{printf}\AgdaSpace{}%
\AgdaString{"\%s\ <\ \%u"}\AgdaSymbol{)}\AgdaSpace{}%
\AgdaSymbol{(}\AgdaFunction{ix-to-list}\AgdaSpace{}%
\AgdaBound{j-i}\AgdaSymbol{)}\AgdaSpace{}%
\AgdaSymbol{(}\AgdaFunction{s-list}\AgdaSpace{}%
\AgdaBound{u}\AgdaSymbol{))}\<%
\\
%
\\[\AgdaEmptyExtraSkip]%
%
\>[6]\AgdaBound{f}\AgdaSpace{}%
\AgdaOperator{\AgdaInductiveConstructor{,}}\AgdaSpace{}%
\AgdaBound{a′}\AgdaSpace{}%
\AgdaOperator{\AgdaField{←}}\AgdaSpace{}%
\AgdaBound{a}\AgdaSpace{}%
\AgdaBound{j-i}\<%
\\
%
\>[6]\AgdaComment{--\ Again,\ move\ the\ context\ under\ if.}\<%
\\
%
\>[6]\AgdaKeyword{let}\AgdaSpace{}%
\AgdaBound{b}\AgdaSpace{}%
\AgdaSymbol{=}\AgdaSpace{}%
\AgdaFunction{printf}%
\>[2412I]\AgdaString{"if\ (\%s\ \&\&\ \%s)\ then\ \%s\ else\ zero"}\<%
\\
\>[.][@{}l@{}]\<[2412I]%
\>[21]\AgdaBound{j≥i}\AgdaSpace{}%
\AgdaBound{j-i<u}\AgdaSpace{}%
\AgdaSymbol{(}\AgdaBound{f}\AgdaSpace{}%
\AgdaBound{a′}\AgdaSymbol{)}\<%
\\
%
\\[\AgdaEmptyExtraSkip]%
%
\>[6]\AgdaFunction{return}\AgdaSpace{}%
\AgdaSymbol{(}\AgdaFunction{id}\AgdaSpace{}%
\AgdaOperator{\AgdaInductiveConstructor{,}}\AgdaSpace{}%
\AgdaBound{b}\AgdaSymbol{)}\<%
\\
%
\>[2]\AgdaFunction{to-fut}\AgdaSpace{}%
\AgdaSymbol{(}\AgdaInductiveConstructor{logistic}\AgdaSpace{}%
\AgdaBound{e}\AgdaSymbol{)}\AgdaSpace{}%
\AgdaBound{ρ}\AgdaSpace{}%
\AgdaSymbol{=}\AgdaSpace{}%
\AgdaKeyword{do}\<%
\\
\>[2][@{}l@{\AgdaIndent{0}}]%
\>[4]\AgdaBound{a}\AgdaSpace{}%
\AgdaOperator{\AgdaField{←}}\AgdaSpace{}%
\AgdaFunction{to-fut}\AgdaSpace{}%
\AgdaBound{e}\AgdaSpace{}%
\AgdaBound{ρ}\<%
\\
%
\>[4]\AgdaFunction{return}\AgdaSpace{}%
\AgdaSymbol{λ}\AgdaSpace{}%
\AgdaBound{i}\AgdaSpace{}%
\AgdaSymbol{→}\AgdaSpace{}%
\AgdaKeyword{do}\<%
\\
\>[4][@{}l@{\AgdaIndent{0}}]%
\>[6]\AgdaBound{f}\AgdaSpace{}%
\AgdaOperator{\AgdaInductiveConstructor{,}}\AgdaSpace{}%
\AgdaBound{a′}\AgdaSpace{}%
\AgdaOperator{\AgdaField{←}}\AgdaSpace{}%
\AgdaBound{a}\AgdaSpace{}%
\AgdaBound{i}\<%
\\
%
\>[6]\AgdaFunction{return}\AgdaSpace{}%
\AgdaSymbol{(}\AgdaBound{f}\AgdaSpace{}%
\AgdaOperator{\AgdaInductiveConstructor{,}}%
\>[19]\AgdaFunction{printf}\AgdaSpace{}%
\AgdaString{"(logistics\ \%s)"}\AgdaSpace{}%
\AgdaBound{a′}\AgdaSymbol{)}\<%
\\
%
\>[2]\AgdaFunction{to-fut}\AgdaSpace{}%
\AgdaSymbol{(}\AgdaBound{e}\AgdaSpace{}%
\AgdaOperator{\AgdaInductiveConstructor{⊞}}\AgdaSpace{}%
\AgdaBound{e₁}\AgdaSymbol{)}\AgdaSpace{}%
\AgdaBound{ρ}\AgdaSpace{}%
\AgdaSymbol{=}\AgdaSpace{}%
\AgdaKeyword{do}\<%
\\
\>[2][@{}l@{\AgdaIndent{0}}]%
\>[4]\AgdaBound{l}\AgdaSpace{}%
\AgdaOperator{\AgdaField{←}}\AgdaSpace{}%
\AgdaFunction{to-fut}\AgdaSpace{}%
\AgdaBound{e}\AgdaSpace{}%
\AgdaBound{ρ}\<%
\\
%
\>[4]\AgdaBound{r}\AgdaSpace{}%
\AgdaOperator{\AgdaField{←}}\AgdaSpace{}%
\AgdaFunction{to-fut}\AgdaSpace{}%
\AgdaBound{e₁}\AgdaSpace{}%
\AgdaBound{ρ}\<%
\\
%
\>[4]\AgdaFunction{return}\AgdaSpace{}%
\AgdaSymbol{λ}\AgdaSpace{}%
\AgdaBound{i}\AgdaSpace{}%
\AgdaSymbol{→}\AgdaSpace{}%
\AgdaKeyword{do}\<%
\\
\>[4][@{}l@{\AgdaIndent{0}}]%
\>[6]\AgdaBound{f}\AgdaSpace{}%
\AgdaOperator{\AgdaInductiveConstructor{,}}\AgdaSpace{}%
\AgdaBound{l′}\AgdaSpace{}%
\AgdaOperator{\AgdaField{←}}\AgdaSpace{}%
\AgdaBound{l}\AgdaSpace{}%
\AgdaBound{i}\<%
\\
%
\>[6]\AgdaBound{g}\AgdaSpace{}%
\AgdaOperator{\AgdaInductiveConstructor{,}}\AgdaSpace{}%
\AgdaBound{r′}\AgdaSpace{}%
\AgdaOperator{\AgdaField{←}}\AgdaSpace{}%
\AgdaBound{r}\AgdaSpace{}%
\AgdaBound{i}\<%
\\
%
\>[6]\AgdaFunction{return}\AgdaSpace{}%
\AgdaSymbol{(}\AgdaBound{f}\AgdaSpace{}%
\AgdaOperator{\AgdaFunction{∘}}\AgdaSpace{}%
\AgdaBound{g}\AgdaSpace{}%
\AgdaOperator{\AgdaInductiveConstructor{,}}\AgdaSpace{}%
\AgdaFunction{printf}\AgdaSpace{}%
\AgdaString{"(\%s\ F.+\ \%s)"}\AgdaSpace{}%
\AgdaBound{l′}\AgdaSpace{}%
\AgdaBound{r′}\AgdaSymbol{)}\<%
\\
%
\\[\AgdaEmptyExtraSkip]%
%
\>[2]\AgdaFunction{to-fut}\AgdaSpace{}%
\AgdaSymbol{(}\AgdaBound{e}\AgdaSpace{}%
\AgdaOperator{\AgdaInductiveConstructor{⊠}}\AgdaSpace{}%
\AgdaBound{e₁}\AgdaSymbol{)}\AgdaSpace{}%
\AgdaBound{ρ}\AgdaSpace{}%
\AgdaSymbol{=}\AgdaSpace{}%
\AgdaKeyword{do}\<%
\\
\>[2][@{}l@{\AgdaIndent{0}}]%
\>[4]\AgdaBound{l}\AgdaSpace{}%
\AgdaOperator{\AgdaField{←}}\AgdaSpace{}%
\AgdaFunction{to-fut}\AgdaSpace{}%
\AgdaBound{e}\AgdaSpace{}%
\AgdaBound{ρ}\<%
\\
%
\>[4]\AgdaBound{r}\AgdaSpace{}%
\AgdaOperator{\AgdaField{←}}\AgdaSpace{}%
\AgdaFunction{to-fut}\AgdaSpace{}%
\AgdaBound{e₁}\AgdaSpace{}%
\AgdaBound{ρ}\<%
\\
%
\>[4]\AgdaFunction{return}\AgdaSpace{}%
\AgdaSymbol{λ}\AgdaSpace{}%
\AgdaBound{i}\AgdaSpace{}%
\AgdaSymbol{→}\AgdaSpace{}%
\AgdaKeyword{do}\<%
\\
\>[4][@{}l@{\AgdaIndent{0}}]%
\>[6]\AgdaBound{f}\AgdaSpace{}%
\AgdaOperator{\AgdaInductiveConstructor{,}}\AgdaSpace{}%
\AgdaBound{l′}\AgdaSpace{}%
\AgdaOperator{\AgdaField{←}}\AgdaSpace{}%
\AgdaBound{l}\AgdaSpace{}%
\AgdaBound{i}\<%
\\
%
\>[6]\AgdaBound{g}\AgdaSpace{}%
\AgdaOperator{\AgdaInductiveConstructor{,}}\AgdaSpace{}%
\AgdaBound{r′}\AgdaSpace{}%
\AgdaOperator{\AgdaField{←}}\AgdaSpace{}%
\AgdaBound{r}\AgdaSpace{}%
\AgdaBound{i}\<%
\\
%
\>[6]\AgdaFunction{return}\AgdaSpace{}%
\AgdaSymbol{(}\AgdaBound{f}\AgdaSpace{}%
\AgdaOperator{\AgdaFunction{∘}}\AgdaSpace{}%
\AgdaBound{g}\AgdaSpace{}%
\AgdaOperator{\AgdaInductiveConstructor{,}}\AgdaSpace{}%
\AgdaFunction{printf}\AgdaSpace{}%
\AgdaString{"(\%s\ F.*\ \%s)"}\AgdaSpace{}%
\AgdaBound{l′}\AgdaSpace{}%
\AgdaBound{r′}\AgdaSymbol{)}\<%
\\
%
\\[\AgdaEmptyExtraSkip]%
%
\>[2]\AgdaFunction{to-fut}\AgdaSpace{}%
\AgdaSymbol{(}\AgdaInductiveConstructor{scaledown}\AgdaSpace{}%
\AgdaBound{x}\AgdaSpace{}%
\AgdaBound{e}\AgdaSymbol{)}\AgdaSpace{}%
\AgdaBound{ρ}\AgdaSpace{}%
\AgdaSymbol{=}\AgdaSpace{}%
\AgdaKeyword{do}\<%
\\
\>[2][@{}l@{\AgdaIndent{0}}]%
\>[4]\AgdaBound{a}\AgdaSpace{}%
\AgdaOperator{\AgdaField{←}}\AgdaSpace{}%
\AgdaFunction{to-fut}\AgdaSpace{}%
\AgdaBound{e}\AgdaSpace{}%
\AgdaBound{ρ}\<%
\\
%
\>[4]\AgdaFunction{return}\AgdaSpace{}%
\AgdaSymbol{λ}\AgdaSpace{}%
\AgdaBound{i}\AgdaSpace{}%
\AgdaSymbol{→}\AgdaSpace{}%
\AgdaKeyword{do}\<%
\\
\>[4][@{}l@{\AgdaIndent{0}}]%
\>[6]\AgdaBound{f}\AgdaSpace{}%
\AgdaOperator{\AgdaInductiveConstructor{,}}\AgdaSpace{}%
\AgdaBound{a′}\AgdaSpace{}%
\AgdaOperator{\AgdaField{←}}\AgdaSpace{}%
\AgdaBound{a}\AgdaSpace{}%
\AgdaBound{i}\<%
\\
%
\>[6]\AgdaFunction{return}\AgdaSpace{}%
\AgdaSymbol{(}\AgdaBound{f}\AgdaSpace{}%
\AgdaOperator{\AgdaInductiveConstructor{,}}%
\>[19]\AgdaFunction{printf}\AgdaSpace{}%
\AgdaString{"(\%s\ F./\ fromi64\ \%s)"}\AgdaSpace{}%
\AgdaBound{a′}\AgdaSpace{}%
\AgdaSymbol{(}\AgdaFunction{show-nat}\AgdaSpace{}%
\AgdaBound{x}\AgdaSymbol{))}\<%
\\
%
\\[\AgdaEmptyExtraSkip]%
%
\\[\AgdaEmptyExtraSkip]%
%
\>[2]\AgdaFunction{to-fut}\AgdaSpace{}%
\AgdaSymbol{(}\AgdaInductiveConstructor{minus}\AgdaSpace{}%
\AgdaBound{e}\AgdaSymbol{)}\AgdaSpace{}%
\AgdaBound{ρ}\AgdaSpace{}%
\AgdaSymbol{=}\AgdaSpace{}%
\AgdaKeyword{do}\<%
\\
\>[2][@{}l@{\AgdaIndent{0}}]%
\>[4]\AgdaBound{a}\AgdaSpace{}%
\AgdaOperator{\AgdaField{←}}\AgdaSpace{}%
\AgdaFunction{to-fut}\AgdaSpace{}%
\AgdaBound{e}\AgdaSpace{}%
\AgdaBound{ρ}\<%
\\
%
\>[4]\AgdaFunction{return}\AgdaSpace{}%
\AgdaSymbol{λ}\AgdaSpace{}%
\AgdaBound{i}\AgdaSpace{}%
\AgdaSymbol{→}\AgdaSpace{}%
\AgdaKeyword{do}\<%
\\
\>[4][@{}l@{\AgdaIndent{0}}]%
\>[6]\AgdaBound{f}\AgdaSpace{}%
\AgdaOperator{\AgdaInductiveConstructor{,}}\AgdaSpace{}%
\AgdaBound{a′}\AgdaSpace{}%
\AgdaOperator{\AgdaField{←}}\AgdaSpace{}%
\AgdaBound{a}\AgdaSpace{}%
\AgdaBound{i}\<%
\\
%
\>[6]\AgdaFunction{return}\AgdaSpace{}%
\AgdaSymbol{(}\AgdaBound{f}\AgdaSpace{}%
\AgdaOperator{\AgdaInductiveConstructor{,}}%
\>[19]\AgdaFunction{printf}\AgdaSpace{}%
\AgdaString{"(F.neg\ \%s)"}\AgdaSpace{}%
\AgdaBound{a′}\AgdaSymbol{)}\<%
\\
%
\\[\AgdaEmptyExtraSkip]%
%
\>[2]\AgdaFunction{to-fut}\AgdaSpace{}%
\AgdaSymbol{(}\AgdaInductiveConstructor{let′}\AgdaSpace{}%
\AgdaBound{e}\AgdaSpace{}%
\AgdaBound{e₁}\AgdaSymbol{)}\AgdaSpace{}%
\AgdaBound{ρ}\AgdaSpace{}%
\AgdaSymbol{=}\AgdaSpace{}%
\AgdaKeyword{do}\<%
\\
\>[2][@{}l@{\AgdaIndent{0}}]%
\>[4]\AgdaBound{c}\AgdaSpace{}%
\AgdaOperator{\AgdaField{←}}\AgdaSpace{}%
\AgdaFunction{get}\<%
\\
%
\>[4]\AgdaField{modify}\AgdaSpace{}%
\AgdaInductiveConstructor{suc}\<%
\\
%
\>[4]\AgdaKeyword{let}\AgdaSpace{}%
\AgdaBound{n}\AgdaSpace{}%
\AgdaSymbol{=}\AgdaSpace{}%
\AgdaFunction{fresh-var}\AgdaSpace{}%
\AgdaBound{c}\<%
\\
%
\>[4]\AgdaBound{b}\AgdaSpace{}%
\AgdaOperator{\AgdaField{←}}\AgdaSpace{}%
\AgdaFunction{to-fut}\AgdaSpace{}%
\AgdaBound{e₁}\AgdaSpace{}%
\AgdaSymbol{(}\AgdaBound{ρ}\AgdaSpace{}%
\AgdaOperator{\AgdaInductiveConstructor{,}}\AgdaSpace{}%
\AgdaSymbol{(}\AgdaFunction{mkar}\AgdaSpace{}%
\AgdaBound{n}\AgdaSymbol{))}\<%
\\
%
\>[4]\AgdaFunction{return}\AgdaSpace{}%
\AgdaSymbol{λ}\AgdaSpace{}%
\AgdaBound{i}\AgdaSpace{}%
\AgdaSymbol{→}\AgdaSpace{}%
\AgdaKeyword{do}\<%
\\
\>[4][@{}l@{\AgdaIndent{0}}]%
\>[6]\AgdaBound{x}\AgdaSpace{}%
\AgdaOperator{\AgdaField{←}}\AgdaSpace{}%
\AgdaFunction{to-str}\AgdaSpace{}%
\AgdaBound{e}\AgdaSpace{}%
\AgdaBound{ρ}\<%
\\
%
\>[6]\AgdaBound{f}\AgdaSpace{}%
\AgdaOperator{\AgdaInductiveConstructor{,}}\AgdaSpace{}%
\AgdaBound{b′}\AgdaSpace{}%
\AgdaOperator{\AgdaField{←}}\AgdaSpace{}%
\AgdaBound{b}\AgdaSpace{}%
\AgdaBound{i}\<%
\\
%
\>[6]\AgdaFunction{return}\AgdaSpace{}%
\AgdaSymbol{(}\AgdaFunction{printf}\AgdaSpace{}%
\AgdaString{"(let\ \%s\ =\ \%s\textbackslash{}nin\ \%s)"}\AgdaSpace{}%
\AgdaBound{n}\AgdaSpace{}%
\AgdaBound{x}\AgdaSpace{}%
\AgdaOperator{\AgdaFunction{∘}}\AgdaSpace{}%
\AgdaBound{f}\AgdaSpace{}%
\AgdaOperator{\AgdaInductiveConstructor{,}}%
\>[55]\AgdaBound{b′}\AgdaSymbol{)}\<%
\\
%
\\[\AgdaEmptyExtraSkip]%
%
\\[\AgdaEmptyExtraSkip]%
%
\>[2]\AgdaFunction{to-str}\AgdaSpace{}%
\AgdaSymbol{\{}\AgdaArgument{s}\AgdaSpace{}%
\AgdaSymbol{=}\AgdaSpace{}%
\AgdaInductiveConstructor{[]}\AgdaSymbol{\}}\AgdaSpace{}%
\AgdaBound{e}\AgdaSpace{}%
\AgdaBound{ρ}\AgdaSpace{}%
\AgdaSymbol{=}\AgdaSpace{}%
\AgdaKeyword{do}\<%
\\
\>[2][@{}l@{\AgdaIndent{0}}]%
\>[4]\AgdaBound{p}\AgdaSpace{}%
\AgdaOperator{\AgdaField{←}}\AgdaSpace{}%
\AgdaFunction{to-fut}\AgdaSpace{}%
\AgdaBound{e}\AgdaSpace{}%
\AgdaBound{ρ}\<%
\\
%
\>[4]\AgdaBound{f}\AgdaSpace{}%
\AgdaOperator{\AgdaInductiveConstructor{,}}\AgdaSpace{}%
\AgdaBound{b}\AgdaSpace{}%
\AgdaOperator{\AgdaField{←}}\AgdaSpace{}%
\AgdaBound{p}\AgdaSpace{}%
\AgdaInductiveConstructor{[]}\<%
\\
%
\>[4]\AgdaFunction{return}\AgdaSpace{}%
\AgdaSymbol{(}\AgdaBound{f}\AgdaSpace{}%
\AgdaBound{b}\AgdaSymbol{)}\<%
\\
%
\>[2]\AgdaCatchallClause{\AgdaFunction{to-str}}\AgdaSpace{}%
\AgdaCatchallClause{\AgdaSymbol{\{}}\AgdaCatchallClause{\AgdaArgument{s}}\AgdaSpace{}%
\AgdaCatchallClause{\AgdaSymbol{=}}\AgdaSpace{}%
\AgdaCatchallClause{\AgdaBound{s}}\AgdaCatchallClause{\AgdaSymbol{\}}}\AgdaSpace{}%
\AgdaCatchallClause{\AgdaBound{e}}\AgdaSpace{}%
\AgdaCatchallClause{\AgdaBound{ρ}}\AgdaSpace{}%
\AgdaSymbol{=}\AgdaSpace{}%
\AgdaKeyword{do}\<%
\\
\>[2][@{}l@{\AgdaIndent{0}}]%
\>[4]\AgdaBound{p}\AgdaSpace{}%
\AgdaOperator{\AgdaField{←}}\AgdaSpace{}%
\AgdaFunction{to-fut}\AgdaSpace{}%
\AgdaBound{e}\AgdaSpace{}%
\AgdaBound{ρ}\<%
\\
%
\>[4]\AgdaBound{i}\AgdaSpace{}%
\AgdaOperator{\AgdaField{←}}\AgdaSpace{}%
\AgdaFunction{iv}\AgdaSpace{}%
\AgdaBound{s}\<%
\\
%
\>[4]\AgdaBound{f}\AgdaSpace{}%
\AgdaOperator{\AgdaInductiveConstructor{,}}\AgdaSpace{}%
\AgdaBound{b}\AgdaSpace{}%
\AgdaOperator{\AgdaField{←}}\AgdaSpace{}%
\AgdaBound{p}\AgdaSpace{}%
\AgdaBound{i}\<%
\\
%
\>[4]\AgdaFunction{return}\AgdaSpace{}%
\AgdaSymbol{(}\AgdaBound{f}\AgdaSpace{}%
\AgdaSymbol{(}\AgdaFunction{to-imap}\AgdaSpace{}%
\AgdaBound{s}\AgdaSpace{}%
\AgdaBound{i}\AgdaSpace{}%
\AgdaBound{b}\AgdaSymbol{))}\<%
\end{code}
The rest of the code generator looks very similar, therefore we omit it
here but the full code is available in the supplementary materials.




% 
% Next, we have to take care of shapes.  Array shapes in \AF{E} are binary trees,
% but array shapes in SaC are 1-dimensional arrays (flattened binary trees).
% When some expression in \AF{E} is of product shape, we usually have to
% supply left or right subshapes of the product to SaC. These are always available
% through implicit arguments of \AF{E} constructors. Assuming that by the
% time we come to extraction, all the \AF{E} shapes are constants, we can
% always generate shape expressions in SaC.  This is implemented in \AF{show-shape}.
% Relaxing the assumption about constant shapes is possible but requires
% extension of \AF{E} so that we can always bind the shapes used in \AF{E}
% to some expressions in SaC.
% 
% We also need a source of fresh variables so that we can generate indices
% for \AC{imap} expressions.  We define a stateful function \AF{iv} that
% generates a fresh index variable.  
% 
% Extraction is given by \AF{to-sac} that translates the expression $e$ in
% the environment $\rho$.  The function is stateful so that we can generate
% fresh variables when needed.
% 
% The definitions of \AF{SEnv}, \AF{iv}, {\AF{show-shape}, and \AF{to-sac} follow.
% \begin{code}[hide]
% module Sac where
%   open import Data.Unit
%   open import Data.Product
%   open import Data.List as L using (List; []; _∷_; _++_)
%   open import Data.Nat as ℕ using (ℕ; zero; suc)
%   open import Data.Nat.Show using () renaming (show to show-nat)
%   open import Data.String hiding (_++_)
%   open import Text.Printf
%   open import Category.Monad.State --using (State; StateMonad; RawMonadState)
%   open import Category.Monad using (RawMonad)
%   --open RawMonad {{...}} public
%   open RawMonadState {{...}} public
%   open Lang
%   open Array hiding (sum; slide; backslide)
%   open SubWk
% 
%   instance
%     -- stateMon : ∀ {S : Set} → RawMonad (State S)
%     -- stateMon {S} = StateMonad S
% 
%     stateMonState : ∀ {S : Set} → RawMonadState S (State S)
%     stateMonState {S} = StateMonadState S
% \end{code}
% \begin{mathpar}
% \codeblock{\begin{code}
%   SEnv : Ctx → Set
%   SEnv ε         = ⊤
%   SEnv (Γ ▹ is)  = SEnv Γ × String
% \end{code}}
% \and
% \codeblock{\begin{code}
%   iv : S → State ℕ String
%   iv s = do  v ← get
%              modify suc
%              return $ printf "x%u" v
% \end{code}
% \begin{code}[hide]
% 
%   lookup : is ∈ Γ → SEnv Γ → String
%   lookup v₀      (ρ , e) = e
%   lookup (vₛ x)  (ρ , e) = lookup x ρ
% 
% 
%   -- show-shape : S → String
%   -- show-shape (ι x) = show-nat x
%   -- show-shape (s S.⊗ p) = printf "⟨%s, %s⟩" (show-shape s) (show-shape p)
% 
%   fresh-var : ℕ → String
%   fresh-var n = printf "x%u" n
% 
%   bop : Bop -> String
%   bop plus = "+"
%   bop mul = "*"
% 
%   dim : S → ℕ
%   dim (ι _) = 1
%   dim (s Array.⊗ p) = dim s ℕ.+ dim p
% 
%   ivl : S → State ℕ (List String)
%   ivl (ι _) = do
%     v ← get
%     modify suc
%     return $ (fresh-var v ∷ [])
%   ivl (s S.⊗ p) = do
%     l ← ivl s
%     r ← ivl p
%     return $ l L.++ r
%   
%   --iv s = printf "[%s]" ∘ intersperse ", " <$> ivl s
% \end{code}}
% \and
% \codeblock{\begin{code}
%   show-shape : S → String
%   show-shape s = printf "[%s]" 
%                $ intersperse ", " 
%                $ go s
%     where
%       go : S → List String
%       go (ι x)    = show-nat x ∷ []
%       go (s ⊗ p)  = go s ++ go p
% \end{code}}
% \and
% \codeblock{\begin{code}
%   to-sac : (e : E Γ is) → (ρ : SEnv Γ) → State ℕ String
%   to-sac (imap {s = s} e) ρ = do
%      i ← iv s
%      b ← to-sac e (ρ , i)
%      return $ printf "{ %s -> %s | %s < %s }"
%                      i b i (show-shape s)
%   to-sac (sel e e₁) ρ = 
%      printf "(%s)[%s]" <$> to-sac e ρ ⊛ to-sac e₁ ρ
%   -- ⋯
% \end{code}}
% \end{mathpar}
% \begin{code}[hide]
%   to-sac zero ρ = return "zero"
%   to-sac one ρ = return "one"
%   to-sac (var x) ρ = return $ lookup x ρ
%   to-sac (imapₛ {s = s} e) ρ = do
%      i ← iv s
%      b ← to-sac e (ρ , i)
%      let sh = show-shape s
%      --return $ printf "{ %s -> %s | %s < %s }" i b i sh
%      return $ printf "IMAPS(%s, (%s), (%s))" i b sh
%   to-sac (selₛ e e₁) ρ = do
%      a ← to-sac e ρ
%      i ← to-sac e₁ ρ
%      --return $ printf "(%s)[%s]" a i
%      return $ printf "sels(%s, %s)" a i
% 
%   -- Copy-paste from scalar versions
% 
%   -- Copy-paste from scalar versions
%   to-sac (imapb {s = s}{p} m e) ρ = do
%      i ← iv s
%      b ← to-sac e (ρ , i)
%      let sh-s = show-shape s
%      let sh-p = show-shape p
%      return $ printf "unblock({ %s -> %s | %s < %s }, %s)" i b i sh-s sh-p
%   to-sac (selb {p = p} m e e₁) ρ = do
%      a ← to-sac e ρ
%      i ← to-sac e₁ ρ
%      let sh-p = show-shape p
%      return $ printf "selb(%s, %s, %s)" a i sh-p
% 
%   to-sac (zero-but i j e) ρ 
%      = printf "%s == %s ? %s : zero" <$> (to-sac i ρ) ⊛ (to-sac j ρ) ⊛ (to-sac e ρ)
%   to-sac (sum {s = s} {p = p} e) ρ = do
%      -- outer index 
%      i ← iv s
%      -- inner index which is juts a fresh name
%      j ← iv p
%      b ← to-sac e (ρ , i)
%      -- `s` is outer shape, and `p` is the inner one
%      let sh-s = show-shape s
%      let sh-p = show-shape p
%      --return $ printf "sumOuter(%u, { %s -> %s | %s < %s})" (dim s) i b i sh-s
%      -- sumOuter(ivOuter, ivInner, e, shOuter, shInner)
%      return $ printf "sumOuter(%s, %s, %s, (%s), (%s))" i j b sh-s sh-p
%   to-sac (bin x e e₁) ρ = do
%      a ← to-sac e ρ
%      b ← to-sac e₁ ρ
%      return $ printf "(%s) %s (%s)" a (bop x) b
%   to-sac (slide {p = p} e pl e₁ su) ρ = do
%      i ← to-sac e ρ
%      a ← to-sac e₁ ρ
%      let sh-p = show-shape p
%      return $ printf "slide(%s, %s, %s)" i a sh-p
%   to-sac (backslide {r = r} e e₁ su pl) ρ = do
%      i ← to-sac e ρ
%      a ← to-sac e₁ ρ
%      let sh-sp = show-shape r
%      return $ printf "backlide(%s, %s, %s)" i a sh-sp
% 
%   to-sac (scaledown x e) ρ = do
%      a ← to-sac e ρ
%      return $ printf "(%s) / %s" a (show-nat x)
% 
%   to-sac (minus e) ρ = printf "-(%s)" <$> to-sac e ρ 
%   to-sac (logistic e) ρ = printf "logistics(%s)" <$> to-sac e ρ
% 
% 
%   -- This can be made stateful, but we are assuming that
%   -- vₛ is no need to make imap/sum index variables unique.
%   env-sac : AD.Env Γ Δ → (vars : SEnv Δ) → SEnv Γ
%   env-sac {ε} ρ σ = _
%   env-sac {Γ ▹ ix s} ρ σ = env-sac ρ σ , "--"
%   env-sac {Γ ▹ ar s} (ρ , e) σ = env-sac ρ σ , proj₁ (to-sac e σ 1)
% 
%   -- Reversed environment to list
%   env-rev-list : SEnv Γ → List String
%   env-rev-list {ε}     ρ = []
%   env-rev-list {Γ ▹ _} (ρ , x) = x ∷ env-rev-list ρ
%  
%   -- zipWith for Environments
%   zip-env : (String → String → String) → SEnv Γ → SEnv Γ → SEnv Γ
%   zip-env {ε}     f tt      tt      = tt
%   zip-env {Γ ▹ x} f (ν , n) (ρ , e) = zip-env f ν ρ , f n e
% \end{code}
% 
% \subsubsection{SaC Primitives\label{sec:sac-primitives}}
% As can be seen from the two cases of \AF{to-sac}, the extraction process is
% not complicated. In essence, we define a small snippet of SaC code for 
% each \AF{E} constructor.  Consider the \AC{imap}/\AC{sel}
% family from the code snippet.  The \AC{imap} constructor maps directly to SaC's
% tensor comprehensions~\cite{tensor-comp} expressed as: \texttt{\{ iv -> e | iv < s \}}.
% This expression constructs arrays by evaluating \texttt{e} for every array non-negative index
% vector
% \texttt{iv} whose components are element-wise smaller than the shape \texttt{s}.  The shape of the resulting
% array is concatenation of \texttt{s} and whatever the shape of \texttt{e} is.
% Selections \AC{sel} correspond to the built-in array selection using
% C-like syntax \texttt{e[iv]} where \texttt{e} is the array we are selecting
% from and \texttt{iv} is the index vector.   Shape constraints are exactly as in
% \AF{E}: if \texttt{e} is of shape \texttt{s ++ p}, and \texttt{iv} is bounded
% by \texttt{s} then \texttt{e[iv]} is of shape \texttt{p}.
% 
% Scalar versions of imap/sel require a little wrapping.  For \AC{imapₛ} we
% generate a tensor comprehension that selects inner expressions (they are
% 1-element vectors) at zero-th position.  For \AC{selₛ} we make selection into
% an array and we wrap the result in a 1-d vector:
% \begin{mathpar}
% {\begin{varwidth}{0.9\textwidth}
% \begin{lstlisting}[linewidth=.4\textwidth]
% #define IMAPS(iv, e, shp) \
%   {iv -> (e)[[0]] | iv < shp}
% \end{lstlisting}
% \end{varwidth}}
% \and
% {\begin{varwidth}{0.9\textwidth}
% \begin{lstlisting}[linewidth=.55\textwidth]
% inline float[1]
% sels(float[d:shp] x, int[d] iv)
% {
%   return [x[iv]];
% }
% \end{lstlisting}
% \end{varwidth}}
% \end{mathpar}
% When translating (\AC{imapₛ} \{ \AB{s} \} \AB{e}) we pick a fresh index variable
% \texttt{iv}, then we translate \AB{e} (in the environment extended with \texttt{iv})
% into \texttt{e'} and we generate \texttt{IMAPS(iv, e', shp)}, where \texttt{shp} is
% a translation of \texttt{s}.  On the side of SaC we expand this macro as shown
% above.  We could have expanded this macro on the Agda side, but this abstraction
% makes it possible to make adjustments in the generated code without running Agda.
% We map \AC{selₛ} into the \texttt{sels} function.  Consider the type of \texttt{sels}
% which uses the recently added feature of SaC that makes it possible to encode
% shape constraints in types~\cite{type-pattern}.  While these constraints are potentially checked at runtime,
% they are very useful for readability and they provide some confidence about the
% generated code.  The meaning of the type \texttt{float[d:shp]} is that it is
% an array of base type \texttt{float} of rank \texttt{d} and shape \texttt{shp}.
% When a variable of the same name is used within different arguments, it automatically
% triggers the equality constraint between the corresponding ranks/shapes.
% 
% \paragraph{Blocking} Implementation of \AC{selb}/\AC{imapb} pair relies on
% the notion of blocking, so we introduce the analogue to \AF{block}/\AF{unblock}
% functionality in SaC as follows:
% \begin{mathpar}
% {\begin{varwidth}{0.9\textwidth}
% \begin{lstlisting}[linewidth=.44\textwidth]
% inline float[n:s,n:p]
% block(float[n:sp] x, int[n] p)
%      | all(s*p == sp)
%      , all(p   >= 0)
% {
%   return { iv -> tile(p, iv * p, x) 
%          | iv < sp / p};
% }
% \end{lstlisting}
% \end{varwidth}}
% \and
% {\begin{varwidth}{0.9\textwidth}
% \begin{lstlisting}[linewidth=.55\textwidth]
% inline float[n:sp] 
% unblock(float[n:s,n:p] a, int[n] p)
%        | all(s*p == sp)
%        , all(p   >= 0)
% {
%   return { iv -> a[(iv / p) ++ mod (iv, p)]
%          | iv < s*p};
% }
% \end{lstlisting}
% \end{varwidth}}
% \end{mathpar}
% The type \texttt{float[n:s,n:p]} denotes an array of the shape \texttt{s ++ p}
% where \texttt{s} and \texttt{p} are of length \texttt{n}.  This is a product
% shape in terms of our array theory.  As \texttt{sp} is just a variable that
% is not related to \texttt{s} or \texttt{p}, we add two constraints (expressions
% behind the bar after the function definition) saying that: (i) \texttt{sp} is
% a point-wise product of \texttt{s} and \texttt{p}; (ii) all the elements of
% the \texttt{p}-shape are greater than zero.  Keep in mind that these are potential
% runtime constraints, they may be proved or flagged as disproved during compilation
% but they do not provide a static guarantee. The implementation of block uses the \texttt{tile}
% operation from the standard library of SaC. It selects a sub-array of the given shape at the given position.
% In \texttt{unblock} we use a division and a modulo operation to remap the indices.
% When translating \AC{selb}, we simply select into \texttt{block}-ed array.
% When translating \AC{imapb}, we use the tensor comprehension as in case of
% \AC{imap} to compute blocked array and then we call \texttt{unblock} on it.
% 
% \paragraph{Sliding} Slides and backslides are translated into calls to
% the following SaC functions:
% \begin{mathpar}
% {\begin{varwidth}{0.9\textwidth}
% \begin{lstlisting}
% inline float[d:n1] 
% slide(int[d] i, float[d:mn] x, int[d] n)       | all(n1        == n + 1)
%                                                , all(n + 1 + i <= mn)
% {
%   return { iv -> x[iv + i] | iv < n + 1 };
% }
% 
% inline float[d:mn]
% backslide(int[d] i, float[d:n1] y, int[d] mn)  | all(i < 1 + mn - n1)
% {
%   return { iv -> y[iv - i] | i <= iv < n1 + i;
%            iv -> 0f        |      iv < mn };
% }
% \end{lstlisting}
% \end{varwidth}}
% \end{mathpar}
% Shape constraints become a little bit involved here because we implicitly
% reconstruct the proof objects such as \AB{m} \AF{+} \AB{n} \AF{≈} \AB{mn}
% and \AF{suc} \AB{n} \AF{≈} \AB{n1}.  Otherwise, \texttt{slide} selects a
% sub-array of the shape (\texttt{n+1}) starting at the index \texttt{i}.
% The \texttt{backslide} populates the sub-array with the elements of
% \texttt{y} and the second partition of the tensor comprehension specifies
% that all the other indices evaluate to zero.  Translation of \AC{slide}
% and \AC{backslide} maps the arguments one-to-one, additionally providing
% the $n$-shape in case of slide and the $(m+n)$ shape in case of backslide.
% 
% \paragraph{Summation} When translating (\AC{sum} \{\AB{s}\} \AB{e}), where
% \AB{e} is of shape \AB{p} (and the index variable within the \AC{sum} is
% bounded by \AB{s}), we map these arguments into the following SaC function:
% \begin{lstlisting}
% inline float[n:p] sumOuter(float[m:s,n:p] a, int[m] s, int[n] p) {
%   return { jv -> sum({iv -> a[iv++jv] | iv < s}) | jv < p };
% }
% \end{lstlisting}
% We use SaC's builtin \texttt{sum} function that sums-up all the elements
% of the given array.
% 
% The rest of the constructions are mapped into regular arithmetic operations
% that are provided by SaC.
% 
% 
% \subsection{Local Variables}
% 
% The framework that we built so far computes derivatives of the variables in
% the context.  This means that for complex expressions in \AF{E} (such as \AF{forward}),
% all the let bindings will be inlined.  This is often not desirable both for performance
% and readability.  Here we present a mechanism that introduce local variables
% and preserves them during AD.
% \begin{code}[hide]
% module DoubleChain where
%   -- In this module I want to preserve derivatives
%   -- of the local variables in the chain (instead of inlining them)
%   open import Data.String
%   open import Text.Printf
%   open import Data.Product --using (Σ; _×_; _,_)
%   open import Data.Unit
%   open import Data.Nat as ℕ using (ℕ; zero; suc)
%   open import Data.List as L using (List; []; _∷_)
%   open Array hiding (sum; slide; backslide)
%   open Lang
%   open SubWk
%   open AD
%   open Opt
%   open BB
% 
%   Env′ : Ctx → Set
%   Env′ Γ = Env Γ Γ
% \end{code}
% 
% The key data structure that makes it possible to introduce local variables
% is called \AF{Chain} which has two constructors.  The empty chain consists
% of the names for all the variables in the context \AB{Γ}.  This represents the
% case where no local variables have been introduced.  The \AC{\_▹\_} constructor
% takes a chain in context \AB{Δ} and the array expression of shape \AB{p} in
% the same context together with the variable name.  This produces the chain
% in the context extended by two variables.  One variable is a place-holder
% for the expression and the other variable is a placeholder for the derivative
% of that expression.
% \begin{code}
%   data Chain : Ctx → Set where
%     ε    : Sac.SEnv Γ → Chain Γ
%     _▹_  : Chain Δ → (String × E Δ (ar p)) → Chain (Δ ▹ ar p ▹ ar p)
% \end{code}
% 
% The computation of the derivative in \AF{Chain}s follows the following
% simple idea.  Consider the chain with two variables $a$ and
% $b$ in the initial context \AB{Γ}, and two local variables $x$ and $y$.
% Here is what happens when we compute the derivative of some expression
% $e$ (that may depend on $a$, $b$, $x$, $y$) with some seed $s$ in the
% empty $\delta_0$ environment. 
% 
% %\begin{table}
% \begin{center}
% \begin{tabular}{cc|cccc|l}
%    $a$         &$b$         &$\partial{x}$& $x$         &$\partial{y}$&$y$       & \text{compute $\nabla\ e\ s\ \delta_0$}\\
%    \hline
%    $\delta_a$  &$\delta_b$  &-            & $\delta_x$  &-            &$\delta_y$& \text{assign $\delta_y$ to $\partial{y}$}\\
%    $\delta_a$  &$\delta_b$  &-            & $\delta_x$  &$\delta_y$   &$\delta_y$& \text{compute $\nabla\ y_e\ \partial{y}$}\\
%    $\delta'_a$ &$\delta'_b$ &-            & $\delta'_x$ &$\delta_y$   &$\delta_y$& \text{assign $\delta'_x$ to $\partial{x}$}\\
%    $\delta'_a$ &$\delta'_b$ &-            & $\delta'_x$ &$\delta_y$   &$\delta_y$& \text{compute $\nabla\ x_e\ \partial{x}$}\\
%    $\delta''_a$ &$\delta''_b$ &$\delta'_x$  & $\delta'_x$ &$\delta_y$   &$\delta_y$& \text{done}
% \end{tabular}
% \end{center}
% %\end{table}
% 
% First of all, the computation of $e$ returns the environment $\delta$ that can
% be found in the first line of the table.  Then we repeat the following steps while
% traversing the chain backwards: we copy the $y$-th position of the $\delta$-environment
% to the $\partial{y}$-th position, and we compute the expression $y_e$ that is assigned to $y$
% ($xx$ in this case) with the seed $\partial{y}$-th variable.  Just to clarify, the seed
% is the variable $\partial{y}$ and not its value.  Then we repeat the same process
% for $x$ and potentially all the other remaining local variables (not in this case) until
% we hit the beginning of the chain.
% 
% At the end of the process we obtain an environment where derivatives for $a$ and
% $b$ are expressed in terms of $\partial{x}$ and $\partial{y}$.  The remaining step
% is to collect the values of $\partial{x}$ and $\partial{y}$ which can be found
% at the corresponding positions in the $\delta$-environment.
% \begin{code}[hide]
%   data LCtx : Set where
%     []  : LCtx
%     _◃_ : IS → LCtx → LCtx
% 
%   _<><_ : Ctx → LCtx → Ctx
%   Γ <>< [] = Γ
%   Γ <>< (x ◃ Δ) = (Γ ▹ x) <>< Δ
% 
%   data LEnv : LCtx → Ctx → Set where
%     []  : LEnv [] Γ
%     _◃_ : ∀ {Δ′} → E Γ (ar s) → LEnv Δ′ Γ → LEnv (ar s ◃ Δ′) Γ
% 
%   data Postfix : Ctx → Ctx → Set where
%     done : Postfix ε Γ
%     next : Postfix Γ Δ → Postfix (Γ ▹ ar s) (Δ ▹ ar s)
% 
%   double-ctx : Ctx → Ctx
%   double-ctx ε = ε
%   double-ctx (Γ ▹ x) = double-ctx Γ ▹ x ▹ x
% 
%   chain-to-env : Chain Γ → Σ Ctx λ Δ → Env (double-ctx Δ) Γ × Postfix (double-ctx Δ) Γ
%   chain-to-env (ε x)   = ε , tt , done
%   chain-to-env (_▹_ {p = p} c (_ , x)) = let
%     Δ , ρ , po = chain-to-env c
%     in (Δ ▹ ar p) , ((env-map {Γ = double-ctx Δ} (↑↑_) ρ , zero) , (↑ ↑ x)) , (next (next po))
% 
%   pstep : ∀ {Δ′} → Postfix ((Δ ▹ ar s) <>< Δ′) Γ → Postfix (Δ <>< (ar s ◃ Δ′)) Γ
%   pstep {Δ′ = []} (next p) = next p
%   pstep {Δ′ = x ◃ Δ′} p = p
% 
%   post-var : ∀ {Δ′} → Postfix (Δ <>< Δ′) Γ → is ∈ Δ → is ∈ Γ
%   post-var {Δ′ = []} (next p) v₀ = v₀
%   post-var {Δ′ = []} (next p) (vₛ x) = vₛ (post-var {Δ′ = []} p x)
%   post-var {Δ′ = is ◃ Δ′} p x = post-var {Δ′ = Δ′} p (vₛ x)
% 
%   no-ix : ix s ∈ Δ → ¬ Postfix Δ Γ
%   no-ix v₀ = λ ()
%   no-ix (vₛ v) (next p) = no-ix v p
% 
%   post-fish : ∀ Δ′ → is ∈ Δ → is ∈ (Δ <>< Δ′)
%   post-fish [] v = v
%   post-fish (x ◃ Δ′) v = post-fish Δ′ (vₛ v)
% 
%   gradc : ∀ {Δ′} → Env (double-ctx Δ) Γ → LEnv Δ′ Γ 
%             → Postfix ((double-ctx Δ) <>< Δ′) Γ →  Env′ Γ → Env′ Γ
%   gradc {ε}        {Γ} {Δ′} ρ ρ′ p δ = δ
%   gradc {Δ ▹ ix x} {Γ} {Δ′} ρ ρ′ p δ = ⊥-elim (no-ix (post-fish Δ′ v₀) p)
%   gradc {Δ ▹ ar x} {Γ} {Δ′} ((ρ , z) , e) ρ′ p δ =
%     let
%     ve = post-var {Δ′ = Δ′} p v₀  -- variable for e in Γ
%     vz = post-var {Δ′ = Δ′} p v₁  -- variable for z in Γ
%     s  = env-ix δ ve
%     δ₁ = update δ vz (const s)    -- save s in the z's position
%     δ₂ = ∇ e (var vz) δ₁          -- use vz position as seed
%     in gradc {Δ} ρ (z ◃ (e ◃ ρ′)) (pstep {Δ′ = ar x ◃ Δ′} (pstep {Δ′ = Δ′} p)) δ₂
% 
%   chain-grad : Chain (Γ ▹ ar s) → E (Γ ▹ ar s) (ar s) → Env′ (Γ ▹ ar s)
%   chain-grad {Γ} {s} c seed = let
%     -- Well, this is a choice I suppose
%     --δ = ∇ seed one (env-imap {Γ = Γ ▹ ar s} (const zero))
%     δ = env-imap {Γ = Γ} (const zero) , seed
%     Δ , ρ , po = chain-to-env c
%     in env-map {Γ = Γ ▹ ar s} (multiopt 10) $ gradc ρ [] po δ
% 
%   chain-sac-ctx : Chain Γ → Sac.SEnv Γ
%   chain-sac-ctx (ε x) = x
%   chain-sac-ctx (c ▹ (v , _)) = chain-sac-ctx c ,, ("∂/∂" ++ v) ,, v
%   
%   filter-grad : Chain Γ → Sac.SEnv Γ → List String 
%   filter-grad (ε x)   δ = Sac.env-rev-list δ
%   filter-grad (c ▹ _) ((δ , _), x) = x ∷ filter-grad c δ
% 
%   chain-grad-sac : Chain Γ → Env′ Γ → String
%   chain-grad-sac {Γ} c δ = let
%     vars = chain-sac-ctx c
%     vals = Sac.env-sac {Γ} δ vars
%     assignments = filter-grad c $ Sac.zip-env (printf "∂/∂%s = %s;") vars vals
%     in intersperse "\n" assignments
% 
%   chain-sac-l : Chain Γ → ℕ → List String 
%   chain-sac-l (ε x) _ = []
%   chain-sac-l (c ▹ (v , e)) n = let r , n′ = Sac.to-sac (multiopt 10 e) (chain-sac-ctx c) n 
%                                 in printf "%s = %s;" v r ∷ chain-sac-l c n′
% 
%   chain-sac : Chain Γ → String
%   chain-sac c = intersperse "\n" $ L.reverse $ chain-sac-l c 1
% 
% 
%   -- test-chain : Chain _ --(ε ▹ ar (ι 3))
%   -- test-chain = ε {Γ = ε ▹ ar (ι 3)} (_ ,, "a") 
%   --            ▹ ("r" , mul-test)
%   --            ▹ ("r₁" , (var v₀) ⊠ (var v₂))
% 
%   -- test-grad : String
%   -- test-grad = chain-sac test-chain 
%   --             ++ "\n" ++ chain-grad-sac test-chain (chain-grad test-chain one)
% \end{code}
% 
% Let us consider a small example to see this in action.  We start with a little
% convenience data structure \AF{ChainCtx} that keeps the shapes and the variable names
% together.  We also define the function \AF{ce-split} that splits 
% \AF{ChainCtx} into the context and the environment with variable names in that context:
% \begin{code}
%   data ChainCtx : Set where
%     ε : ChainCtx
%     _▹_ : ChainCtx → String × S → ChainCtx
% 
%   ce-split : ChainCtx → Σ Ctx Sac.SEnv
% \end{code}
% \begin{code}[hide]
%   ce-split ε = ε , tt
%   ce-split (cx ▹ (v , s)) = let Δ , ρ = ce-split cx in (Δ ▹ ar s) , (ρ , v)
% 
%   Product : ℕ → Set → Set
%   Product 0       A = ⊤
%   Product 1       A = A
%   Product (suc n) A = A × Product n A
% 
%   Es : ∀ {Γ : Ctx} → (n : ℕ) → {Product n IS} → Set
%   Es {Γ} 0             {is} = ⊤
%   Es {Γ} 1             {is} = E Γ is
%   Es {Γ} (suc (suc n)) {is , p}  = E Γ is × Es {Γ} (suc n) {p}
% 
%   ↑↑ₙ : ∀ {Γ : Ctx} {is} n {p : Product n IS} → Es {Γ} n {p} → Es {Γ ▹ is ▹ is} n {p}
%   ↑↑ₙ 0 es = _
%   ↑↑ₙ 1 e  = ↑↑ e
%   ↑↑ₙ (suc (suc n)) (e , es) = ↑↑ e , ↑↑ₙ (suc n) es
% \end{code}
% Consider an initial environment of two 5-element vectors $a$ and $b$; local
% computations $x = ab$ and $y = xx$; and the generated code when computing derivative
% of $y$ (\AC{var v₀}) on the right.
% \begin{mathpar}
% \codeblock{\begin{code}
%   test-chain : Chain _
%   test-chain = let
%     Γ , ρ = ce-split (ε ▹ ("a" , ι 5) ▹ ("b" , ι 5))
%     a = var v₁; b = var v₀
%     C₁ = ε {Γ} ρ  ▹ ("x" , a ⊠ b)
%     x = var v₀
%     C₂ = C₁       ▹ ("y" , x ⊠ x)
%     in C₂
% \end{code}}
% \and
% {\begin{varwidth}{0.9\textwidth}
% \begin{lstlisting}[linewidth=.44\textwidth]
% x = (a) * (b);
% y = (x) * (x);
% ddy = one;
% ddx = ((ddy) * (x)) + ((ddy) * (x));
% ddb = (ddx) * (a);
% dda = (ddx) * (b);
% \end{lstlisting}
% \end{varwidth}}
% \end{mathpar}
% Let us convince ourselves that the result is correct.  Our expression is $abab = a^2b^2$,
% and its partial derivatives $\frac{\partial}{\partial a} = 2ab^2$,
% $\frac{\partial}{\partial b} = 2ba^2$.  If we fold the assignments, we get:
% \begin{eqnarray*}
%    \text{dda} &= (x + x)b = (ab + ab)b = 2ab^2\\
%    \text{ddb} &= (x + x)a = (ab + ab)a = 2ba^2
% \end{eqnarray*}
% Note that computations in $x$ and \texttt{ddx} are shared in further computations
% which was the main goal of introducing this mechanism.
% 
% There are two inconveniences in the above implementation that we would like to
% mention:
% \begin{enumerate}
% \item There is no restriction on using the placeholders for derivatives in the 
% chain expressions, so in principle, one could write expression in terms of
% variables and their derivatives.  However, this is not being handled and likely
% to generate bogus terms.  If this is a useful feature, it requires more thinking
% on how exactly it should work.  Otherwise it is easy to introduce restrictions
% that rule out such cases.
% \item If we define variables in the chain that do not contribute to the final
% expression, we may introduce extra computations.  We do not compromise correctness,
% as all inaccessible terms will get zero value.  However, direct execution of the
% resulting expressions may introduce redundant computations.
% \end{enumerate}
% Both of these are future work.  For now, we make an assumption that placeholders
% are not used in the expressions and that we do not insert bindings that do not
% contribute to the final result.
% 
% \begin{code}[hide]
%   test-chain-sac : String
%   test-chain-sac
%     = chain-sac test-chain 
%              ++ "\n" ++ chain-grad-sac test-chain (chain-grad test-chain (one))
% 
% \end{code}
% 
% We present the specification of our case study in \AF{E} using \AF{Chain}.  We start
% with the context \AF{cnn-ctx} that contains the \texttt{target} digit that
% is depicted on the image, the input image \texttt{inp} and the weights of the network.
% The definition of the chain is a one-to-one copy of the definition found in
% Section~\ref{sec:cnn}.  The only real difference is that we have to take care of
% maintaining bindings between Agda variables and the variables in \AF{E}.  Fortunately,
% let expressions in Agda make it possible to shadow the binding, which comes very
% useful in this case.
% 
% {\small
% \begin{code}
%   cnn-ctx : ChainCtx
%   cnn-ctx  = ε
%            ▹ ("target"  , ι 10 ⊗ (ι 1 ⊗ (ι 1 ⊗ (ι 1 ⊗ ι 1))))     -- 7
%            ▹ ("inp"     , ι 28 ⊗ ι 28)                            -- 6
%            ▹ ("k₁"      , ι 6 ⊗ (ι 5 ⊗ ι 5))                      -- 5
%            ▹ ("b₁"      , ι 6)                                    -- 4
%            ▹ ("k₂"      , ι 12 ⊗ (ι 6 ⊗ (ι 5 ⊗ ι 5)))             -- 3
%            ▹ ("b₂"      , ι 12)                                   -- 2
%            ▹ ("fc"      , ι 10 ⊗ (ι 12 ⊗ (ι 1 ⊗ (ι 4 ⊗ ι 4))))    -- 1
%            ▹ ("b"       , ι 10)                                   -- 0
% 
%   cnn-chain : Chain _
%   cnn-chain = let 
%       Γ , ρ = ce-split cnn-ctx 
%       inp = var v₆; k₁ = var v₅; b₁ = var v₄; k₂ = var v₃; b₂ = var v₂; fc = var v₁; b = var v₀
%       C₁ = ε {Γ} ρ ▹ ("c₁₁" , mconv (ι ⊗ ι) inp k₁ b₁ (ι ⊗ ι));        k₂ = ↑↑ k₂; b₂ = ↑↑ b₂;  fc = ↑↑ fc; b = ↑↑ b; c₁₁ = var v₀
%       C₂ = C₁ ▹ ("c₁"  , logistic c₁₁);                                k₂ = ↑↑ k₂; b₂ = ↑↑ b₂;  fc = ↑↑ fc; b = ↑↑ b; c₁ = var v₀
%       C₃ = C₂ ▹ ("s₁"  , Imap λ i → avgp₂ 12 12 (sel (↑ c₁) i));       k₂ = ↑↑ k₂; b₂ = ↑↑ b₂;  fc = ↑↑ fc; b = ↑↑ b; s₁ = var v₀
%       C₄ = C₃ ▹ ("c₂₁" , mconv (ι ⊗ (ι ⊗ ι)) s₁ k₂ b₂ (ι ⊗ (ι ⊗ ι)));                           fc = ↑↑ fc; b = ↑↑ b; c₂₁ = var v₀
%       C₅ = C₄ ▹ ("c₂"  , logistic c₂₁);                                                         fc = ↑↑ fc; b = ↑↑ b; c₂ = var v₀
%       C₆ = C₅ ▹ ("s₂"  , Imap λ i → Imap λ j → avgp₂ 4 4 (sel (sel (↑↑ c₂) (↑ i)) j));          fc = ↑↑ fc; b = ↑↑ b; s₂ = var v₀
%       C₇ = C₆ ▹ ("r₁"  , mconv (ι ⊗ (ι ⊗ (ι ⊗ ι))) s₂ fc b (ι ⊗ (ι ⊗ (ι ⊗ ι))));                r₁ = var v₀
%       C₈ = C₇ ▹ ("r"   , logistic r₁)
%       in C₈
% \end{code}
% 
% \begin{code}[hide]
%   test-cnn : String
%   test-cnn 
%     = let
%         -- 2*8 + 7 = 23
%         target = ↑↑ ↑↑ ↑↑ ↑↑ ↑↑  ↑↑ ↑↑ ↑↑ ↑↑ ↑↑  ↑↑ ↑ (var v₀)
%       in chain-sac cnn-chain 
%              ++ "\n" ++ chain-grad-sac cnn-chain (chain-grad cnn-chain (var v₀ ⊞ minus target))
% \end{code}
% }
% 

\section{Performance\label{sec:performance}}

% Notes for Troels:
%
% - Leave Agda-side optimisations for Artem
% - Only backend: Futhark
% - Compared with:
%   - TensorFlow (on GPU)
%   - Hand-written Futhark (lower priority)
% - Will try for multicore numbers as well, if does not complicate story


One of the goals of this work is to demonstrate that it is possible to formulate
the problem in a proof assistant and then pass it on to the other system that can
run the algorithm efficiently.  In order to substantiate this claim, we compare
the running times of the code that we generate from the specification at the end of the
Section~\ref{sec:ad} and the hand-written SaC code from~\cite{cnn-array}.
We are not interested in an exhaustive performance study similar to what is provided
in \cite{cnn-array}. Instead, we take the version from that paper as reference point 
and we aim to find out whether we are in the same ballpark.

We take the code from~\cite{cnn-array}, make sure that it runs, and we replace
the hand-written CNN training with the Agda-generated one. 
Our first observation is that both versions 
produce the same results, and none of the shape constraints
that we defined in Section~\ref{sec:sac-primitives} fired.  This means that our
code generation is working.  Unfortunately, the runtime comparison revealed that
our version is about 10$\times$ slower than the hand-written SaC version.

We got in touch with the SaC team who provided a lot of support in identifying
the causes of the disappointing difference in performance.  It turns out that the main culprit has
to do with the inability to optimise away selections into tensor comprehensions in a few situations.
With-Loop-Folding~\cite{wlf}, SaC's mechanism for fusing tensor comprehensions fails to fold
tensor comprehensions that are nested and cannot be flattened statically.
In simple terms, the expression \texttt{\{iv -> e(iv)\}[jv]} in some situation does not reduce to
\texttt{e(jv)} which, in our generated code, is essential to match the hand-written performance.
As a result, several arrays were created
simply to make a single selection into them.  The original code never ran into this
problem as the hand-written code avoided such patterns.  Our \AF{E} optimiser
from Section~\ref{sec:opt} could not help either, because the problem was occurring after
the SaC primitives such as slide and block were inlined.

After numerous attempts on altering \AF{E} to fit SaC requirements and the SaC
team trying to implement some of the missing optimisations, on the February 23rd
(6 days before the ICFP deadline) we realised that the best runtime we can
obtain is still 6$\times$ slower than the hand-written code.  The compiler is too
sensitive to the flavour of the code that we pass to it, and when certain patterns
are not recognised, there is very little one can do other than trying to fit
those patterns.  However, this is not always possible with the generated code.
Performance \emph{is} frustrating!

\subsection{Generating C}

After overcoming the natural instinct to give up, we realised that the real
power of the proposed approach lies in the ability to modify any part of the
code generation pipeline.  This includes swapping the back-end language of choice
to something else.  Therefore, we decided to try generating C code instead of
SaC code.

While C is a canonical low-level language, it has excellent support for
multi-dimensional arrays, given that the ranks are statically known.
At runtime these arrays are flattened vectors, they do not have to live
on the stack, and the language takes care of multi-dimensional indexing.

However, the key difference between the C and SaC is memory management.
SaC is a functional language that uses reference counting to automate
operations on allocating and freeing memory.  In C these decisions are
manual, and as we have seen before, excessive memory allocation is detrimental
for the runtime.  For our use case we avoid memory management problem
entirely, by assuming that all the variables in the \AF{Chain} have
to be preallocated, and if we need any temporary array variables when
extracting array values, we fail extraction.  This way we guarantee
that no memory allocation is ever needed.

Meeting such a requirement means that we need to optimise away operations
like \AC{slide}/\AC{backslide} as they require conceptual array allocation.
The same goes for \AC{imap}s appearing in some of the argument positions.
Putting these considerations together, we extended \AF{E} with the following
explicit operations on indices:
\begin{code}[hide]%
\>[0]\AgdaKeyword{open}\AgdaSpace{}%
\AgdaKeyword{import}\AgdaSpace{}%
\AgdaModule{Data.Nat}\AgdaSpace{}%
\AgdaSymbol{as}\AgdaSpace{}%
\AgdaModule{ℕ}\AgdaSpace{}%
\AgdaKeyword{using}\AgdaSpace{}%
\AgdaSymbol{(}\AgdaDatatype{ℕ}\AgdaSymbol{;}\AgdaSpace{}%
\AgdaInductiveConstructor{zero}\AgdaSymbol{;}\AgdaSpace{}%
\AgdaInductiveConstructor{suc}\AgdaSymbol{)}\<%
\\
\>[0]\AgdaKeyword{open}\AgdaSpace{}%
\AgdaKeyword{import}\AgdaSpace{}%
\AgdaModule{Data.Unit}\<%
\\
\>[0]\AgdaKeyword{open}\AgdaSpace{}%
\AgdaKeyword{import}\AgdaSpace{}%
\AgdaModule{Data.Empty}\<%
\\
\>[0]\AgdaKeyword{open}\AgdaSpace{}%
\AgdaKeyword{import}\AgdaSpace{}%
\AgdaModule{Data.Product}\AgdaSpace{}%
\AgdaSymbol{as}\AgdaSpace{}%
\AgdaModule{Prod}\AgdaSpace{}%
\AgdaKeyword{using}\AgdaSpace{}%
\AgdaSymbol{(}\AgdaRecord{Σ}\AgdaSymbol{;}\AgdaSpace{}%
\AgdaFunction{∃}\AgdaSymbol{;}\AgdaSpace{}%
\AgdaOperator{\AgdaInductiveConstructor{\AgdaUnderscore{},\AgdaUnderscore{}}}\AgdaSymbol{;}\AgdaSpace{}%
\AgdaOperator{\AgdaFunction{\AgdaUnderscore{}×\AgdaUnderscore{}}}\AgdaSymbol{;}\AgdaSpace{}%
\AgdaField{proj₁}\AgdaSymbol{;}\AgdaSpace{}%
\AgdaField{proj₂}\AgdaSymbol{)}\<%
\\
\>[0]\AgdaKeyword{open}\AgdaSpace{}%
\AgdaKeyword{import}\AgdaSpace{}%
\AgdaModule{Relation.Nullary}\<%
\\
\>[0]\AgdaKeyword{open}\AgdaSpace{}%
\AgdaKeyword{import}\AgdaSpace{}%
\AgdaModule{Relation.Binary.PropositionalEquality}\AgdaSpace{}%
\AgdaKeyword{hiding}\AgdaSpace{}%
\AgdaSymbol{(}\AgdaOperator{\AgdaInductiveConstructor{[\AgdaUnderscore{}]}}\AgdaSymbol{)}\<%
\\
\>[0]\AgdaKeyword{open}\AgdaSpace{}%
\AgdaKeyword{import}\AgdaSpace{}%
\AgdaModule{Data.List}\AgdaSpace{}%
\AgdaSymbol{as}\AgdaSpace{}%
\AgdaModule{L}\AgdaSpace{}%
\AgdaKeyword{using}\AgdaSpace{}%
\AgdaSymbol{(}\AgdaDatatype{List}\AgdaSymbol{;}\AgdaSpace{}%
\AgdaInductiveConstructor{[]}\AgdaSymbol{;}\AgdaSpace{}%
\AgdaOperator{\AgdaInductiveConstructor{\AgdaUnderscore{}∷\AgdaUnderscore{}}}\AgdaSymbol{)}\<%
\\
\>[0]\AgdaKeyword{open}\AgdaSpace{}%
\AgdaKeyword{import}\AgdaSpace{}%
\AgdaModule{Function}\<%
\\
\>[0]\AgdaKeyword{open}\AgdaSpace{}%
\AgdaKeyword{import}\AgdaSpace{}%
\AgdaModule{arrays}\<%
\\
\>[0]\AgdaKeyword{open}\AgdaSpace{}%
\AgdaModule{Array}\AgdaSpace{}%
\AgdaKeyword{hiding}\AgdaSpace{}%
\AgdaSymbol{(}\AgdaFunction{sum}\AgdaSymbol{;}\AgdaSpace{}%
\AgdaFunction{slide}\AgdaSymbol{;}\AgdaSpace{}%
\AgdaFunction{backslide}\AgdaSymbol{)}\<%
\\
%
\\[\AgdaEmptyExtraSkip]%
\>[0]\AgdaKeyword{data}\AgdaSpace{}%
\AgdaDatatype{IS}\AgdaSpace{}%
\AgdaSymbol{:}\AgdaSpace{}%
\AgdaPrimitive{Set}\AgdaSpace{}%
\AgdaKeyword{where}\<%
\\
\>[0][@{}l@{\AgdaIndent{0}}]%
\>[2]\AgdaInductiveConstructor{ix}\AgdaSpace{}%
\AgdaSymbol{:}\AgdaSpace{}%
\AgdaDatatype{S}\AgdaSpace{}%
\AgdaSymbol{→}\AgdaSpace{}%
\AgdaDatatype{IS}\<%
\\
%
\>[2]\AgdaInductiveConstructor{ar}\AgdaSpace{}%
\AgdaSymbol{:}\AgdaSpace{}%
\AgdaDatatype{S}\AgdaSpace{}%
\AgdaSymbol{→}\AgdaSpace{}%
\AgdaDatatype{IS}\<%
\\
%
\\[\AgdaEmptyExtraSkip]%
\>[0]\AgdaKeyword{infixl}\AgdaSpace{}%
\AgdaNumber{15}\AgdaSpace{}%
\AgdaOperator{\AgdaInductiveConstructor{\AgdaUnderscore{}▹\AgdaUnderscore{}}}\<%
\\
\>[0]\AgdaKeyword{data}\AgdaSpace{}%
\AgdaDatatype{Ctx}\AgdaSpace{}%
\AgdaSymbol{:}\AgdaSpace{}%
\AgdaPrimitive{Set}\AgdaSpace{}%
\AgdaKeyword{where}\<%
\\
\>[0][@{}l@{\AgdaIndent{0}}]%
\>[2]\AgdaInductiveConstructor{ε}\AgdaSpace{}%
\AgdaSymbol{:}\AgdaSpace{}%
\AgdaDatatype{Ctx}\<%
\\
%
\>[2]\AgdaOperator{\AgdaInductiveConstructor{\AgdaUnderscore{}▹\AgdaUnderscore{}}}\AgdaSpace{}%
\AgdaSymbol{:}\AgdaSpace{}%
\AgdaDatatype{Ctx}\AgdaSpace{}%
\AgdaSymbol{→}\AgdaSpace{}%
\AgdaDatatype{IS}\AgdaSpace{}%
\AgdaSymbol{→}\AgdaSpace{}%
\AgdaDatatype{Ctx}\<%
\\
%
\\[\AgdaEmptyExtraSkip]%
\>[0]\AgdaKeyword{variable}\<%
\\
\>[0][@{}l@{\AgdaIndent{0}}]%
\>[2]\AgdaGeneralizable{Γ}\AgdaSpace{}%
\AgdaGeneralizable{Δ}\AgdaSpace{}%
\AgdaGeneralizable{Ξ}\AgdaSpace{}%
\AgdaGeneralizable{Ψ}\AgdaSpace{}%
\AgdaSymbol{:}\AgdaSpace{}%
\AgdaDatatype{Ctx}\<%
\\
%
\>[2]\AgdaGeneralizable{is}\AgdaSpace{}%
\AgdaGeneralizable{ip}\AgdaSpace{}%
\AgdaGeneralizable{iq}\AgdaSpace{}%
\AgdaGeneralizable{ir}\AgdaSpace{}%
\AgdaSymbol{:}\AgdaSpace{}%
\AgdaDatatype{IS}\<%
\\
%
\\[\AgdaEmptyExtraSkip]%
\>[0]\AgdaKeyword{data}\AgdaSpace{}%
\AgdaOperator{\AgdaDatatype{\AgdaUnderscore{}∈\AgdaUnderscore{}}}\AgdaSpace{}%
\AgdaSymbol{:}\AgdaSpace{}%
\AgdaDatatype{IS}\AgdaSpace{}%
\AgdaSymbol{→}\AgdaSpace{}%
\AgdaDatatype{Ctx}\AgdaSpace{}%
\AgdaSymbol{→}\AgdaSpace{}%
\AgdaPrimitive{Set}\AgdaSpace{}%
\AgdaKeyword{where}\<%
\\
\>[0][@{}l@{\AgdaIndent{0}}]%
\>[2]\AgdaInductiveConstructor{here}\AgdaSpace{}%
\AgdaSymbol{:}\AgdaSpace{}%
\AgdaGeneralizable{is}\AgdaSpace{}%
\AgdaOperator{\AgdaDatatype{∈}}\AgdaSpace{}%
\AgdaSymbol{(}\AgdaGeneralizable{Γ}\AgdaSpace{}%
\AgdaOperator{\AgdaInductiveConstructor{▹}}\AgdaSpace{}%
\AgdaGeneralizable{is}\AgdaSymbol{)}\<%
\\
%
\>[2]\AgdaInductiveConstructor{there}\AgdaSpace{}%
\AgdaSymbol{:}\AgdaSpace{}%
\AgdaGeneralizable{is}\AgdaSpace{}%
\AgdaOperator{\AgdaDatatype{∈}}\AgdaSpace{}%
\AgdaGeneralizable{Γ}\AgdaSpace{}%
\AgdaSymbol{→}\AgdaSpace{}%
\AgdaGeneralizable{is}\AgdaSpace{}%
\AgdaOperator{\AgdaDatatype{∈}}\AgdaSpace{}%
\AgdaSymbol{(}\AgdaGeneralizable{Γ}\AgdaSpace{}%
\AgdaOperator{\AgdaInductiveConstructor{▹}}\AgdaSpace{}%
\AgdaGeneralizable{ip}\AgdaSymbol{)}\<%
\\
%
\\[\AgdaEmptyExtraSkip]%
\>[0]\AgdaKeyword{pattern}\AgdaSpace{}%
\AgdaInductiveConstructor{v₀}\AgdaSpace{}%
\AgdaSymbol{=}\AgdaSpace{}%
\AgdaInductiveConstructor{here}\<%
\\
\>[0]\AgdaKeyword{pattern}\AgdaSpace{}%
\AgdaInductiveConstructor{v₁}\AgdaSpace{}%
\AgdaSymbol{=}\AgdaSpace{}%
\AgdaInductiveConstructor{there}\AgdaSpace{}%
\AgdaInductiveConstructor{v₀}\<%
\\
\>[0]\AgdaKeyword{pattern}\AgdaSpace{}%
\AgdaInductiveConstructor{v₂}\AgdaSpace{}%
\AgdaSymbol{=}\AgdaSpace{}%
\AgdaInductiveConstructor{there}\AgdaSpace{}%
\AgdaInductiveConstructor{v₁}\<%
\\
\>[0]\AgdaKeyword{pattern}\AgdaSpace{}%
\AgdaInductiveConstructor{v₃}\AgdaSpace{}%
\AgdaSymbol{=}\AgdaSpace{}%
\AgdaInductiveConstructor{there}\AgdaSpace{}%
\AgdaInductiveConstructor{v₂}\<%
\\
\>[0]\AgdaKeyword{pattern}\AgdaSpace{}%
\AgdaInductiveConstructor{v₄}\AgdaSpace{}%
\AgdaSymbol{=}\AgdaSpace{}%
\AgdaInductiveConstructor{there}\AgdaSpace{}%
\AgdaInductiveConstructor{v₃}\<%
\\
\>[0]\AgdaKeyword{pattern}\AgdaSpace{}%
\AgdaInductiveConstructor{v₅}\AgdaSpace{}%
\AgdaSymbol{=}\AgdaSpace{}%
\AgdaInductiveConstructor{there}\AgdaSpace{}%
\AgdaInductiveConstructor{v₄}\<%
\\
\>[0]\AgdaKeyword{pattern}\AgdaSpace{}%
\AgdaInductiveConstructor{v₆}\AgdaSpace{}%
\AgdaSymbol{=}\AgdaSpace{}%
\AgdaInductiveConstructor{there}\AgdaSpace{}%
\AgdaInductiveConstructor{v₅}\<%
\\
\>[0]\AgdaKeyword{pattern}\AgdaSpace{}%
\AgdaInductiveConstructor{v₇}\AgdaSpace{}%
\AgdaSymbol{=}\AgdaSpace{}%
\AgdaInductiveConstructor{there}\AgdaSpace{}%
\AgdaInductiveConstructor{v₆}\<%
\\
\>[0]\AgdaKeyword{pattern}\AgdaSpace{}%
\AgdaInductiveConstructor{v₈}\AgdaSpace{}%
\AgdaSymbol{=}\AgdaSpace{}%
\AgdaInductiveConstructor{there}\AgdaSpace{}%
\AgdaInductiveConstructor{v₇}\<%
\\
\>[0]\AgdaKeyword{pattern}\AgdaSpace{}%
\AgdaInductiveConstructor{v₉}\AgdaSpace{}%
\AgdaSymbol{=}\AgdaSpace{}%
\AgdaInductiveConstructor{there}\AgdaSpace{}%
\AgdaInductiveConstructor{v₈}\<%
\\
%
\\[\AgdaEmptyExtraSkip]%
\>[0]\AgdaFunction{unit}\AgdaSpace{}%
\AgdaSymbol{:}\AgdaSpace{}%
\AgdaDatatype{S}\<%
\\
\>[0]\AgdaFunction{unit}\AgdaSpace{}%
\AgdaSymbol{=}\AgdaSpace{}%
\AgdaInductiveConstructor{ι}\AgdaSpace{}%
\AgdaNumber{1}\<%
\\
%
\\[\AgdaEmptyExtraSkip]%
\>[0]\AgdaKeyword{data}\AgdaSpace{}%
\AgdaDatatype{Bop}\AgdaSpace{}%
\AgdaSymbol{:}\AgdaSpace{}%
\AgdaPrimitive{Set}\AgdaSpace{}%
\AgdaKeyword{where}\<%
\\
\>[0][@{}l@{\AgdaIndent{0}}]%
\>[2]\AgdaInductiveConstructor{plus}\AgdaSpace{}%
\AgdaInductiveConstructor{mul}\AgdaSpace{}%
\AgdaSymbol{:}\AgdaSpace{}%
\AgdaDatatype{Bop}\<%
\end{code}
\begin{code}%
\>[0]\AgdaKeyword{data}\AgdaSpace{}%
\AgdaDatatype{E}\AgdaSpace{}%
\AgdaSymbol{:}\AgdaSpace{}%
\AgdaDatatype{Ctx}\AgdaSpace{}%
\AgdaSymbol{→}\AgdaSpace{}%
\AgdaDatatype{IS}\AgdaSpace{}%
\AgdaSymbol{→}\AgdaSpace{}%
\AgdaPrimitive{Set}\AgdaSpace{}%
\AgdaKeyword{where}\<%
\\
\>[0][@{}l@{\AgdaIndent{0}}]%
\>[2]\AgdaInductiveConstructor{div}%
\>[13]\AgdaSymbol{:}\AgdaSpace{}%
\AgdaGeneralizable{s}\AgdaSpace{}%
\AgdaOperator{\AgdaFunction{*}}\AgdaSpace{}%
\AgdaGeneralizable{p}\AgdaSpace{}%
\AgdaOperator{\AgdaFunction{≈}}\AgdaSpace{}%
\AgdaGeneralizable{q}\AgdaSpace{}%
\AgdaSymbol{→}\AgdaSpace{}%
\AgdaSymbol{(}\AgdaBound{i}\AgdaSpace{}%
\AgdaSymbol{:}\AgdaSpace{}%
\AgdaDatatype{E}\AgdaSpace{}%
\AgdaGeneralizable{Γ}\AgdaSpace{}%
\AgdaSymbol{(}\AgdaInductiveConstructor{ix}\AgdaSpace{}%
\AgdaGeneralizable{q}\AgdaSymbol{))}\AgdaSpace{}%
\AgdaSymbol{→}\AgdaSpace{}%
\AgdaDatatype{E}\AgdaSpace{}%
\AgdaGeneralizable{Γ}\AgdaSpace{}%
\AgdaSymbol{(}\AgdaInductiveConstructor{ix}\AgdaSpace{}%
\AgdaGeneralizable{s}\AgdaSymbol{)}\<%
\\
%
\>[2]\AgdaInductiveConstructor{mod}%
\>[13]\AgdaSymbol{:}\AgdaSpace{}%
\AgdaGeneralizable{s}\AgdaSpace{}%
\AgdaOperator{\AgdaFunction{*}}\AgdaSpace{}%
\AgdaGeneralizable{p}\AgdaSpace{}%
\AgdaOperator{\AgdaFunction{≈}}\AgdaSpace{}%
\AgdaGeneralizable{q}\AgdaSpace{}%
\AgdaSymbol{→}\AgdaSpace{}%
\AgdaSymbol{(}\AgdaBound{i}\AgdaSpace{}%
\AgdaSymbol{:}\AgdaSpace{}%
\AgdaDatatype{E}\AgdaSpace{}%
\AgdaGeneralizable{Γ}\AgdaSpace{}%
\AgdaSymbol{(}\AgdaInductiveConstructor{ix}\AgdaSpace{}%
\AgdaGeneralizable{q}\AgdaSymbol{))}\AgdaSpace{}%
\AgdaSymbol{→}\AgdaSpace{}%
\AgdaDatatype{E}\AgdaSpace{}%
\AgdaGeneralizable{Γ}\AgdaSpace{}%
\AgdaSymbol{(}\AgdaInductiveConstructor{ix}\AgdaSpace{}%
\AgdaGeneralizable{p}\AgdaSymbol{)}\<%
\\
%
\>[2]\AgdaInductiveConstructor{ix-plus}%
\>[13]\AgdaSymbol{:}\AgdaSpace{}%
\AgdaSymbol{(}\AgdaBound{i}\AgdaSpace{}%
\AgdaSymbol{:}\AgdaSpace{}%
\AgdaDatatype{E}\AgdaSpace{}%
\AgdaGeneralizable{Γ}\AgdaSpace{}%
\AgdaSymbol{(}\AgdaInductiveConstructor{ix}\AgdaSpace{}%
\AgdaGeneralizable{s}\AgdaSymbol{))}\AgdaSpace{}%
\AgdaSymbol{→}\AgdaSpace{}%
\AgdaSymbol{(}\AgdaBound{j}\AgdaSpace{}%
\AgdaSymbol{:}\AgdaSpace{}%
\AgdaDatatype{E}\AgdaSpace{}%
\AgdaGeneralizable{Γ}\AgdaSpace{}%
\AgdaSymbol{(}\AgdaInductiveConstructor{ix}\AgdaSpace{}%
\AgdaGeneralizable{u}\AgdaSymbol{))}\AgdaSpace{}%
\AgdaSymbol{→}\AgdaSpace{}%
\AgdaOperator{\AgdaFunction{suc}}\AgdaSpace{}%
\AgdaGeneralizable{p}\AgdaSpace{}%
\AgdaOperator{\AgdaFunction{≈}}\AgdaSpace{}%
\AgdaGeneralizable{u}\AgdaSpace{}%
\AgdaSymbol{→}\AgdaSpace{}%
\AgdaGeneralizable{s}\AgdaSpace{}%
\AgdaOperator{\AgdaFunction{+}}\AgdaSpace{}%
\AgdaGeneralizable{p}\AgdaSpace{}%
\AgdaOperator{\AgdaFunction{≈}}\AgdaSpace{}%
\AgdaGeneralizable{r}\AgdaSpace{}%
\AgdaSymbol{→}\AgdaSpace{}%
\AgdaDatatype{E}\AgdaSpace{}%
\AgdaGeneralizable{Γ}\AgdaSpace{}%
\AgdaSymbol{(}\AgdaInductiveConstructor{ix}\AgdaSpace{}%
\AgdaGeneralizable{r}\AgdaSymbol{)}\<%
\\
%
\>[2]\AgdaInductiveConstructor{ix-minus}%
\>[13]\AgdaSymbol{:}\AgdaSpace{}%
\AgdaSymbol{(}\AgdaBound{i}\AgdaSpace{}%
\AgdaSymbol{:}\AgdaSpace{}%
\AgdaDatatype{E}\AgdaSpace{}%
\AgdaGeneralizable{Γ}\AgdaSpace{}%
\AgdaSymbol{(}\AgdaInductiveConstructor{ix}\AgdaSpace{}%
\AgdaGeneralizable{r}\AgdaSymbol{))}\AgdaSpace{}%
\AgdaSymbol{→}\AgdaSpace{}%
\AgdaSymbol{(}\AgdaBound{j}\AgdaSpace{}%
\AgdaSymbol{:}\AgdaSpace{}%
\AgdaDatatype{E}\AgdaSpace{}%
\AgdaGeneralizable{Γ}\AgdaSpace{}%
\AgdaSymbol{(}\AgdaInductiveConstructor{ix}\AgdaSpace{}%
\AgdaGeneralizable{s}\AgdaSymbol{))}\AgdaSpace{}%
\AgdaSymbol{→}\AgdaSpace{}%
\AgdaGeneralizable{s}\AgdaSpace{}%
\AgdaOperator{\AgdaFunction{+}}\AgdaSpace{}%
\AgdaGeneralizable{p}\AgdaSpace{}%
\AgdaOperator{\AgdaFunction{≈}}\AgdaSpace{}%
\AgdaGeneralizable{r}\AgdaSpace{}%
\AgdaSymbol{→}\AgdaSpace{}%
\AgdaOperator{\AgdaFunction{suc}}\AgdaSpace{}%
\AgdaGeneralizable{p}\AgdaSpace{}%
\AgdaOperator{\AgdaFunction{≈}}\AgdaSpace{}%
\AgdaGeneralizable{u}\<%
\\
%
\>[13]\AgdaSymbol{→}\AgdaSpace{}%
\AgdaSymbol{(}\AgdaBound{e}\AgdaSpace{}%
\AgdaSymbol{:}\AgdaSpace{}%
\AgdaDatatype{E}\AgdaSpace{}%
\AgdaSymbol{(}\AgdaGeneralizable{Γ}\AgdaSpace{}%
\AgdaOperator{\AgdaInductiveConstructor{▹}}\AgdaSpace{}%
\AgdaInductiveConstructor{ix}\AgdaSpace{}%
\AgdaGeneralizable{u}\AgdaSymbol{)}\AgdaSpace{}%
\AgdaSymbol{(}\AgdaInductiveConstructor{ar}\AgdaSpace{}%
\AgdaGeneralizable{q}\AgdaSymbol{))}\AgdaSpace{}%
\AgdaSymbol{→}\AgdaSpace{}%
\AgdaDatatype{E}\AgdaSpace{}%
\AgdaGeneralizable{Γ}\AgdaSpace{}%
\AgdaSymbol{(}\AgdaInductiveConstructor{ar}\AgdaSpace{}%
\AgdaGeneralizable{q}\AgdaSymbol{)}\<%
\\
%
\>[2]\AgdaInductiveConstructor{ix-minusᵣ}%
\>[13]\AgdaSymbol{:}\AgdaSpace{}%
\AgdaSymbol{(}\AgdaBound{i}\AgdaSpace{}%
\AgdaSymbol{:}\AgdaSpace{}%
\AgdaDatatype{E}\AgdaSpace{}%
\AgdaGeneralizable{Γ}\AgdaSpace{}%
\AgdaSymbol{(}\AgdaInductiveConstructor{ix}\AgdaSpace{}%
\AgdaGeneralizable{r}\AgdaSymbol{))}\AgdaSpace{}%
\AgdaSymbol{→}\AgdaSpace{}%
\AgdaSymbol{(}\AgdaBound{j}\AgdaSpace{}%
\AgdaSymbol{:}\AgdaSpace{}%
\AgdaDatatype{E}\AgdaSpace{}%
\AgdaGeneralizable{Γ}\AgdaSpace{}%
\AgdaSymbol{(}\AgdaInductiveConstructor{ix}\AgdaSpace{}%
\AgdaGeneralizable{u}\AgdaSymbol{))}\AgdaSpace{}%
\AgdaSymbol{→}\AgdaSpace{}%
\AgdaGeneralizable{s}\AgdaSpace{}%
\AgdaOperator{\AgdaFunction{+}}\AgdaSpace{}%
\AgdaGeneralizable{p}\AgdaSpace{}%
\AgdaOperator{\AgdaFunction{≈}}\AgdaSpace{}%
\AgdaGeneralizable{r}\AgdaSpace{}%
\AgdaSymbol{→}\AgdaSpace{}%
\AgdaOperator{\AgdaFunction{suc}}\AgdaSpace{}%
\AgdaGeneralizable{p}\AgdaSpace{}%
\AgdaOperator{\AgdaFunction{≈}}\AgdaSpace{}%
\AgdaGeneralizable{u}\<%
\\
%
\>[13]\AgdaSymbol{→}\AgdaSpace{}%
\AgdaSymbol{(}\AgdaBound{e}\AgdaSpace{}%
\AgdaSymbol{:}\AgdaSpace{}%
\AgdaDatatype{E}\AgdaSpace{}%
\AgdaSymbol{(}\AgdaGeneralizable{Γ}\AgdaSpace{}%
\AgdaOperator{\AgdaInductiveConstructor{▹}}\AgdaSpace{}%
\AgdaInductiveConstructor{ix}\AgdaSpace{}%
\AgdaGeneralizable{s}\AgdaSymbol{)}\AgdaSpace{}%
\AgdaSymbol{(}\AgdaInductiveConstructor{ar}\AgdaSpace{}%
\AgdaGeneralizable{q}\AgdaSymbol{))}\AgdaSpace{}%
\AgdaSymbol{→}\AgdaSpace{}%
\AgdaDatatype{E}\AgdaSpace{}%
\AgdaGeneralizable{Γ}\AgdaSpace{}%
\AgdaSymbol{(}\AgdaInductiveConstructor{ar}\AgdaSpace{}%
\AgdaGeneralizable{q}\AgdaSymbol{)}\<%
\\
%
\>[2]\AgdaComment{--\ ...}\<%
\end{code}
\begin{code}[hide]%
%
\>[2]\AgdaInductiveConstructor{zero}\AgdaSpace{}%
\AgdaInductiveConstructor{one}\AgdaSpace{}%
\AgdaSymbol{:}\AgdaSpace{}%
\AgdaDatatype{E}\AgdaSpace{}%
\AgdaGeneralizable{Γ}\AgdaSpace{}%
\AgdaSymbol{(}\AgdaInductiveConstructor{ar}\AgdaSpace{}%
\AgdaGeneralizable{s}\AgdaSymbol{)}\<%
\\
%
\>[2]\AgdaInductiveConstructor{var}\AgdaSpace{}%
\AgdaSymbol{:}\AgdaSpace{}%
\AgdaGeneralizable{is}\AgdaSpace{}%
\AgdaOperator{\AgdaDatatype{∈}}\AgdaSpace{}%
\AgdaGeneralizable{Γ}\AgdaSpace{}%
\AgdaSymbol{→}\AgdaSpace{}%
\AgdaDatatype{E}\AgdaSpace{}%
\AgdaGeneralizable{Γ}\AgdaSpace{}%
\AgdaGeneralizable{is}\<%
\\
%
\\[\AgdaEmptyExtraSkip]%
%
\>[2]\AgdaInductiveConstructor{imapₛ}\AgdaSpace{}%
\AgdaSymbol{:}\AgdaSpace{}%
\AgdaDatatype{E}\AgdaSpace{}%
\AgdaSymbol{(}\AgdaGeneralizable{Γ}\AgdaSpace{}%
\AgdaOperator{\AgdaInductiveConstructor{▹}}\AgdaSpace{}%
\AgdaInductiveConstructor{ix}\AgdaSpace{}%
\AgdaGeneralizable{s}\AgdaSymbol{)}\AgdaSpace{}%
\AgdaSymbol{(}\AgdaInductiveConstructor{ar}\AgdaSpace{}%
\AgdaFunction{unit}\AgdaSymbol{)}\AgdaSpace{}%
\AgdaSymbol{→}\AgdaSpace{}%
\AgdaDatatype{E}\AgdaSpace{}%
\AgdaGeneralizable{Γ}\AgdaSpace{}%
\AgdaSymbol{(}\AgdaInductiveConstructor{ar}\AgdaSpace{}%
\AgdaGeneralizable{s}\AgdaSymbol{)}\<%
\\
%
\>[2]\AgdaInductiveConstructor{selₛ}\AgdaSpace{}%
\AgdaSymbol{:}\AgdaSpace{}%
\AgdaDatatype{E}\AgdaSpace{}%
\AgdaGeneralizable{Γ}\AgdaSpace{}%
\AgdaSymbol{(}\AgdaInductiveConstructor{ar}\AgdaSpace{}%
\AgdaGeneralizable{s}\AgdaSymbol{)}\AgdaSpace{}%
\AgdaSymbol{→}\AgdaSpace{}%
\AgdaDatatype{E}\AgdaSpace{}%
\AgdaGeneralizable{Γ}\AgdaSpace{}%
\AgdaSymbol{(}\AgdaInductiveConstructor{ix}\AgdaSpace{}%
\AgdaGeneralizable{s}\AgdaSymbol{)}\AgdaSpace{}%
\AgdaSymbol{→}\AgdaSpace{}%
\AgdaDatatype{E}\AgdaSpace{}%
\AgdaGeneralizable{Γ}\AgdaSpace{}%
\AgdaSymbol{(}\AgdaInductiveConstructor{ar}\AgdaSpace{}%
\AgdaFunction{unit}\AgdaSymbol{)}\<%
\\
%
\\[\AgdaEmptyExtraSkip]%
%
\>[2]\AgdaInductiveConstructor{imap}\AgdaSpace{}%
\AgdaSymbol{:}\AgdaSpace{}%
\AgdaDatatype{E}\AgdaSpace{}%
\AgdaSymbol{(}\AgdaGeneralizable{Γ}\AgdaSpace{}%
\AgdaOperator{\AgdaInductiveConstructor{▹}}\AgdaSpace{}%
\AgdaInductiveConstructor{ix}\AgdaSpace{}%
\AgdaGeneralizable{s}\AgdaSymbol{)}\AgdaSpace{}%
\AgdaSymbol{(}\AgdaInductiveConstructor{ar}\AgdaSpace{}%
\AgdaGeneralizable{p}\AgdaSymbol{)}\AgdaSpace{}%
\AgdaSymbol{→}\AgdaSpace{}%
\AgdaDatatype{E}\AgdaSpace{}%
\AgdaGeneralizable{Γ}\AgdaSpace{}%
\AgdaSymbol{(}\AgdaInductiveConstructor{ar}\AgdaSpace{}%
\AgdaSymbol{(}\AgdaGeneralizable{s}\AgdaSpace{}%
\AgdaOperator{\AgdaFunction{⊗}}\AgdaSpace{}%
\AgdaGeneralizable{p}\AgdaSymbol{))}\<%
\\
%
\>[2]\AgdaInductiveConstructor{sel}\AgdaSpace{}%
\AgdaSymbol{:}\AgdaSpace{}%
\AgdaDatatype{E}\AgdaSpace{}%
\AgdaGeneralizable{Γ}\AgdaSpace{}%
\AgdaSymbol{(}\AgdaInductiveConstructor{ar}\AgdaSpace{}%
\AgdaSymbol{(}\AgdaGeneralizable{s}\AgdaSpace{}%
\AgdaOperator{\AgdaFunction{⊗}}\AgdaSpace{}%
\AgdaGeneralizable{p}\AgdaSymbol{))}\AgdaSpace{}%
\AgdaSymbol{→}\AgdaSpace{}%
\AgdaDatatype{E}\AgdaSpace{}%
\AgdaGeneralizable{Γ}\AgdaSpace{}%
\AgdaSymbol{(}\AgdaInductiveConstructor{ix}\AgdaSpace{}%
\AgdaGeneralizable{s}\AgdaSymbol{)}\AgdaSpace{}%
\AgdaSymbol{→}\AgdaSpace{}%
\AgdaDatatype{E}\AgdaSpace{}%
\AgdaGeneralizable{Γ}\AgdaSpace{}%
\AgdaSymbol{(}\AgdaInductiveConstructor{ar}\AgdaSpace{}%
\AgdaGeneralizable{p}\AgdaSymbol{)}\<%
\\
%
\\[\AgdaEmptyExtraSkip]%
%
\>[2]\AgdaComment{--\ Blocked\ operations\ for\ avgpool\ }\<%
\\
%
\>[2]\AgdaInductiveConstructor{imapb}\AgdaSpace{}%
\AgdaSymbol{:}\AgdaSpace{}%
\AgdaGeneralizable{s}\AgdaSpace{}%
\AgdaOperator{\AgdaFunction{*}}\AgdaSpace{}%
\AgdaGeneralizable{p}\AgdaSpace{}%
\AgdaOperator{\AgdaFunction{≈}}\AgdaSpace{}%
\AgdaGeneralizable{q}\AgdaSpace{}%
\AgdaSymbol{→}\AgdaSpace{}%
\AgdaDatatype{E}\AgdaSpace{}%
\AgdaSymbol{(}\AgdaGeneralizable{Γ}\AgdaSpace{}%
\AgdaOperator{\AgdaInductiveConstructor{▹}}\AgdaSpace{}%
\AgdaInductiveConstructor{ix}\AgdaSpace{}%
\AgdaGeneralizable{s}\AgdaSymbol{)}\AgdaSpace{}%
\AgdaSymbol{(}\AgdaInductiveConstructor{ar}\AgdaSpace{}%
\AgdaGeneralizable{p}\AgdaSymbol{)}\AgdaSpace{}%
\AgdaSymbol{→}\AgdaSpace{}%
\AgdaDatatype{E}\AgdaSpace{}%
\AgdaGeneralizable{Γ}\AgdaSpace{}%
\AgdaSymbol{(}\AgdaInductiveConstructor{ar}\AgdaSpace{}%
\AgdaGeneralizable{q}\AgdaSymbol{)}\<%
\\
%
\>[2]\AgdaInductiveConstructor{selb}\AgdaSpace{}%
\AgdaSymbol{:}\AgdaSpace{}%
\AgdaGeneralizable{s}\AgdaSpace{}%
\AgdaOperator{\AgdaFunction{*}}\AgdaSpace{}%
\AgdaGeneralizable{p}\AgdaSpace{}%
\AgdaOperator{\AgdaFunction{≈}}\AgdaSpace{}%
\AgdaGeneralizable{q}\AgdaSpace{}%
\AgdaSymbol{→}\AgdaSpace{}%
\AgdaDatatype{E}\AgdaSpace{}%
\AgdaGeneralizable{Γ}\AgdaSpace{}%
\AgdaSymbol{(}\AgdaInductiveConstructor{ar}\AgdaSpace{}%
\AgdaGeneralizable{q}\AgdaSymbol{)}\AgdaSpace{}%
\AgdaSymbol{→}\AgdaSpace{}%
\AgdaDatatype{E}\AgdaSpace{}%
\AgdaGeneralizable{Γ}\AgdaSpace{}%
\AgdaSymbol{(}\AgdaInductiveConstructor{ix}\AgdaSpace{}%
\AgdaGeneralizable{s}\AgdaSymbol{)}\AgdaSpace{}%
\AgdaSymbol{→}\AgdaSpace{}%
\AgdaDatatype{E}\AgdaSpace{}%
\AgdaGeneralizable{Γ}\AgdaSpace{}%
\AgdaSymbol{(}\AgdaInductiveConstructor{ar}\AgdaSpace{}%
\AgdaGeneralizable{p}\AgdaSymbol{)}\<%
\\
%
\\[\AgdaEmptyExtraSkip]%
%
\>[2]\AgdaComment{--\ zero-but\ i\ j\ e\ =\ i\ ==\ j\ ?\ e\ :\ 0}\<%
\\
%
\>[2]\AgdaInductiveConstructor{zero-but}\AgdaSpace{}%
\AgdaSymbol{:}\AgdaSpace{}%
\AgdaDatatype{E}\AgdaSpace{}%
\AgdaGeneralizable{Γ}\AgdaSpace{}%
\AgdaSymbol{(}\AgdaInductiveConstructor{ix}\AgdaSpace{}%
\AgdaGeneralizable{s}\AgdaSymbol{)}\AgdaSpace{}%
\AgdaSymbol{→}\AgdaSpace{}%
\AgdaDatatype{E}\AgdaSpace{}%
\AgdaGeneralizable{Γ}\AgdaSpace{}%
\AgdaSymbol{(}\AgdaInductiveConstructor{ix}\AgdaSpace{}%
\AgdaGeneralizable{s}\AgdaSymbol{)}\AgdaSpace{}%
\AgdaSymbol{→}\AgdaSpace{}%
\AgdaDatatype{E}\AgdaSpace{}%
\AgdaGeneralizable{Γ}\AgdaSpace{}%
\AgdaSymbol{(}\AgdaInductiveConstructor{ar}\AgdaSpace{}%
\AgdaGeneralizable{p}\AgdaSymbol{)}\AgdaSpace{}%
\AgdaSymbol{→}\AgdaSpace{}%
\AgdaDatatype{E}\AgdaSpace{}%
\AgdaGeneralizable{Γ}\AgdaSpace{}%
\AgdaSymbol{(}\AgdaInductiveConstructor{ar}\AgdaSpace{}%
\AgdaGeneralizable{p}\AgdaSymbol{)}\<%
\\
%
\>[2]\AgdaInductiveConstructor{sum}\AgdaSpace{}%
\AgdaSymbol{:}\AgdaSpace{}%
\AgdaDatatype{E}\AgdaSpace{}%
\AgdaSymbol{(}\AgdaGeneralizable{Γ}\AgdaSpace{}%
\AgdaOperator{\AgdaInductiveConstructor{▹}}\AgdaSpace{}%
\AgdaInductiveConstructor{ix}\AgdaSpace{}%
\AgdaGeneralizable{s}\AgdaSymbol{)}\AgdaSpace{}%
\AgdaSymbol{(}\AgdaInductiveConstructor{ar}\AgdaSpace{}%
\AgdaGeneralizable{p}\AgdaSymbol{)}\AgdaSpace{}%
\AgdaSymbol{→}\AgdaSpace{}%
\AgdaDatatype{E}\AgdaSpace{}%
\AgdaGeneralizable{Γ}\AgdaSpace{}%
\AgdaSymbol{(}\AgdaInductiveConstructor{ar}\AgdaSpace{}%
\AgdaGeneralizable{p}\AgdaSymbol{)}\<%
\\
%
\>[2]\AgdaInductiveConstructor{bin}\AgdaSpace{}%
\AgdaSymbol{:}\AgdaSpace{}%
\AgdaDatatype{Bop}\AgdaSpace{}%
\AgdaSymbol{→}\AgdaSpace{}%
\AgdaDatatype{E}\AgdaSpace{}%
\AgdaGeneralizable{Γ}\AgdaSpace{}%
\AgdaSymbol{(}\AgdaInductiveConstructor{ar}\AgdaSpace{}%
\AgdaGeneralizable{s}\AgdaSymbol{)}\AgdaSpace{}%
\AgdaSymbol{→}\AgdaSpace{}%
\AgdaDatatype{E}\AgdaSpace{}%
\AgdaGeneralizable{Γ}\AgdaSpace{}%
\AgdaSymbol{(}\AgdaInductiveConstructor{ar}\AgdaSpace{}%
\AgdaGeneralizable{s}\AgdaSymbol{)}\AgdaSpace{}%
\AgdaSymbol{→}\AgdaSpace{}%
\AgdaDatatype{E}\AgdaSpace{}%
\AgdaGeneralizable{Γ}\AgdaSpace{}%
\AgdaSymbol{(}\AgdaInductiveConstructor{ar}\AgdaSpace{}%
\AgdaGeneralizable{s}\AgdaSymbol{)}\<%
\\
%
\\[\AgdaEmptyExtraSkip]%
%
\>[2]\AgdaInductiveConstructor{slide}%
\>[468I]\AgdaSymbol{:}\AgdaSpace{}%
\AgdaDatatype{E}\AgdaSpace{}%
\AgdaGeneralizable{Γ}\AgdaSpace{}%
\AgdaSymbol{(}\AgdaInductiveConstructor{ix}\AgdaSpace{}%
\AgdaGeneralizable{s}\AgdaSymbol{)}\AgdaSpace{}%
\AgdaSymbol{→}\AgdaSpace{}%
\AgdaGeneralizable{s}\AgdaSpace{}%
\AgdaOperator{\AgdaFunction{+}}\AgdaSpace{}%
\AgdaGeneralizable{p}\AgdaSpace{}%
\AgdaOperator{\AgdaFunction{≈}}\AgdaSpace{}%
\AgdaGeneralizable{r}\AgdaSpace{}%
\AgdaSymbol{→}\AgdaSpace{}%
\AgdaDatatype{E}\AgdaSpace{}%
\AgdaGeneralizable{Γ}\AgdaSpace{}%
\AgdaSymbol{(}\AgdaInductiveConstructor{ar}\AgdaSpace{}%
\AgdaGeneralizable{r}\AgdaSymbol{)}\<%
\\
\>[.][@{}l@{}]\<[468I]%
\>[8]\AgdaSymbol{→}\AgdaSpace{}%
\AgdaOperator{\AgdaFunction{suc}}\AgdaSpace{}%
\AgdaGeneralizable{p}\AgdaSpace{}%
\AgdaOperator{\AgdaFunction{≈}}\AgdaSpace{}%
\AgdaGeneralizable{u}\AgdaSpace{}%
\AgdaSymbol{→}\AgdaSpace{}%
\AgdaDatatype{E}\AgdaSpace{}%
\AgdaGeneralizable{Γ}\AgdaSpace{}%
\AgdaSymbol{(}\AgdaInductiveConstructor{ar}\AgdaSpace{}%
\AgdaGeneralizable{u}\AgdaSymbol{)}\<%
\\
%
\>[2]\AgdaInductiveConstructor{backslide}%
\>[493I]\AgdaSymbol{:}\AgdaSpace{}%
\AgdaDatatype{E}\AgdaSpace{}%
\AgdaGeneralizable{Γ}\AgdaSpace{}%
\AgdaSymbol{(}\AgdaInductiveConstructor{ix}\AgdaSpace{}%
\AgdaGeneralizable{s}\AgdaSymbol{)}\AgdaSpace{}%
\AgdaSymbol{→}\AgdaSpace{}%
\AgdaDatatype{E}\AgdaSpace{}%
\AgdaGeneralizable{Γ}\AgdaSpace{}%
\AgdaSymbol{(}\AgdaInductiveConstructor{ar}\AgdaSpace{}%
\AgdaGeneralizable{u}\AgdaSymbol{)}\AgdaSpace{}%
\AgdaSymbol{→}\AgdaSpace{}%
\AgdaOperator{\AgdaFunction{suc}}\AgdaSpace{}%
\AgdaGeneralizable{p}\AgdaSpace{}%
\AgdaOperator{\AgdaFunction{≈}}\AgdaSpace{}%
\AgdaGeneralizable{u}\<%
\\
\>[.][@{}l@{}]\<[493I]%
\>[12]\AgdaSymbol{→}\AgdaSpace{}%
\AgdaGeneralizable{s}\AgdaSpace{}%
\AgdaOperator{\AgdaFunction{+}}\AgdaSpace{}%
\AgdaGeneralizable{p}\AgdaSpace{}%
\AgdaOperator{\AgdaFunction{≈}}\AgdaSpace{}%
\AgdaGeneralizable{r}\AgdaSpace{}%
\AgdaSymbol{→}\AgdaSpace{}%
\AgdaDatatype{E}\AgdaSpace{}%
\AgdaGeneralizable{Γ}\AgdaSpace{}%
\AgdaSymbol{(}\AgdaInductiveConstructor{ar}\AgdaSpace{}%
\AgdaGeneralizable{r}\AgdaSymbol{)}\<%
\\
\>[0]\<%
\\
%
\>[2]\AgdaInductiveConstructor{scaledown}\AgdaSpace{}%
\AgdaSymbol{:}\AgdaSpace{}%
\AgdaDatatype{ℕ}\AgdaSpace{}%
\AgdaSymbol{→}\AgdaSpace{}%
\AgdaDatatype{E}\AgdaSpace{}%
\AgdaGeneralizable{Γ}\AgdaSpace{}%
\AgdaSymbol{(}\AgdaInductiveConstructor{ar}\AgdaSpace{}%
\AgdaGeneralizable{s}\AgdaSymbol{)}\AgdaSpace{}%
\AgdaSymbol{→}\AgdaSpace{}%
\AgdaDatatype{E}\AgdaSpace{}%
\AgdaGeneralizable{Γ}\AgdaSpace{}%
\AgdaSymbol{(}\AgdaInductiveConstructor{ar}\AgdaSpace{}%
\AgdaGeneralizable{s}\AgdaSymbol{)}\<%
\\
%
\>[2]\AgdaInductiveConstructor{minus}\AgdaSpace{}%
\AgdaSymbol{:}\AgdaSpace{}%
\AgdaDatatype{E}\AgdaSpace{}%
\AgdaGeneralizable{Γ}\AgdaSpace{}%
\AgdaSymbol{(}\AgdaInductiveConstructor{ar}\AgdaSpace{}%
\AgdaGeneralizable{s}\AgdaSymbol{)}\AgdaSpace{}%
\AgdaSymbol{→}\AgdaSpace{}%
\AgdaDatatype{E}\AgdaSpace{}%
\AgdaGeneralizable{Γ}\AgdaSpace{}%
\AgdaSymbol{(}\AgdaInductiveConstructor{ar}\AgdaSpace{}%
\AgdaGeneralizable{s}\AgdaSymbol{)}\<%
\\
%
\\[\AgdaEmptyExtraSkip]%
%
\>[2]\AgdaInductiveConstructor{logistic}\AgdaSpace{}%
\AgdaSymbol{:}\AgdaSpace{}%
\AgdaDatatype{E}\AgdaSpace{}%
\AgdaGeneralizable{Γ}\AgdaSpace{}%
\AgdaSymbol{(}\AgdaInductiveConstructor{ar}\AgdaSpace{}%
\AgdaGeneralizable{s}\AgdaSymbol{)}\AgdaSpace{}%
\AgdaSymbol{→}\AgdaSpace{}%
\AgdaDatatype{E}\AgdaSpace{}%
\AgdaGeneralizable{Γ}\AgdaSpace{}%
\AgdaSymbol{(}\AgdaInductiveConstructor{ar}\AgdaSpace{}%
\AgdaGeneralizable{s}\AgdaSymbol{)}\<%
\end{code}
The \AC{div} and \AC{mod} constructors perform point-wise division or modulo
operation on the index $i$ and the shape $p$.  This is needed to express selections
into blocked arrays as we have seen in Section~\ref{sec:sac-primitives}.
The \AC{ix-plus} is a point-wise addition of $i$ and $j$.  The \AC{ix-minus} and
\AC{ix-minusᵣ} correspond to left and right subtraction from the Section~\ref{sec:cnn}.
The meaning of these constructors is follows: if $j$ can be subtracted from $i$
(in the sense of existence of inverse to \AF{⊕ₚ} exists) then we evaluate $e$ at that index,
otherwise we return zero.

\subsubsection{Optimisations}
We add the following optimisations to facilitate removal of temporary arrays in
the generated code.  We show the only ones that we added, all the optimisations
we defined before are still valid.
\begin{code}[hide]%
\>[0]\AgdaOperator{\AgdaFunction{\AgdaUnderscore{}/\AgdaUnderscore{}}}\AgdaSpace{}%
\AgdaSymbol{:}\AgdaSpace{}%
\AgdaSymbol{(}\AgdaBound{Γ}\AgdaSpace{}%
\AgdaSymbol{:}\AgdaSpace{}%
\AgdaDatatype{Ctx}\AgdaSymbol{)}\AgdaSpace{}%
\AgdaSymbol{→}\AgdaSpace{}%
\AgdaGeneralizable{is}\AgdaSpace{}%
\AgdaOperator{\AgdaDatatype{∈}}\AgdaSpace{}%
\AgdaBound{Γ}\AgdaSpace{}%
\AgdaSymbol{→}\AgdaSpace{}%
\AgdaDatatype{Ctx}\<%
\\
\>[0]\AgdaSymbol{(}\AgdaBound{Γ}\AgdaSpace{}%
\AgdaOperator{\AgdaInductiveConstructor{▹}}\AgdaSpace{}%
\AgdaBound{x}\AgdaSymbol{)}\AgdaSpace{}%
\AgdaOperator{\AgdaFunction{/}}\AgdaSpace{}%
\AgdaInductiveConstructor{here}\AgdaSpace{}%
\AgdaSymbol{=}\AgdaSpace{}%
\AgdaBound{Γ}\<%
\\
\>[0]\AgdaSymbol{(}\AgdaBound{Γ}\AgdaSpace{}%
\AgdaOperator{\AgdaInductiveConstructor{▹}}\AgdaSpace{}%
\AgdaBound{x}\AgdaSymbol{)}\AgdaSpace{}%
\AgdaOperator{\AgdaFunction{/}}\AgdaSpace{}%
\AgdaInductiveConstructor{there}\AgdaSpace{}%
\AgdaBound{v}\AgdaSpace{}%
\AgdaSymbol{=}\AgdaSpace{}%
\AgdaSymbol{(}\AgdaBound{Γ}\AgdaSpace{}%
\AgdaOperator{\AgdaFunction{/}}\AgdaSpace{}%
\AgdaBound{v}\AgdaSymbol{)}\AgdaSpace{}%
\AgdaOperator{\AgdaInductiveConstructor{▹}}\AgdaSpace{}%
\AgdaBound{x}\<%
\\
%
\\[\AgdaEmptyExtraSkip]%
\>[0]\AgdaComment{--\ See\ the\ actual\ definition\ in\ the\ ./code\ directory\ in\ the}\<%
\\
\>[0]\AgdaComment{--\ root\ of\ the\ repo,\ here\ we\ just\ make\ a\ stub\ to\ explain\ the}\<%
\\
\>[0]\AgdaComment{--\ code\ below.}\<%
\\
\>[0]\AgdaKeyword{postulate}\<%
\\
\>[0][@{}l@{\AgdaIndent{0}}]%
\>[2]\AgdaPostulate{wkv}\AgdaSpace{}%
\AgdaSymbol{:}\AgdaSpace{}%
\AgdaSymbol{(}\AgdaBound{v}\AgdaSpace{}%
\AgdaSymbol{:}\AgdaSpace{}%
\AgdaGeneralizable{is}\AgdaSpace{}%
\AgdaOperator{\AgdaDatatype{∈}}\AgdaSpace{}%
\AgdaGeneralizable{Γ}\AgdaSymbol{)}\AgdaSpace{}%
\AgdaSymbol{→}\AgdaSpace{}%
\AgdaGeneralizable{ip}\AgdaSpace{}%
\AgdaOperator{\AgdaDatatype{∈}}\AgdaSpace{}%
\AgdaSymbol{(}\AgdaGeneralizable{Γ}\AgdaSpace{}%
\AgdaOperator{\AgdaFunction{/}}\AgdaSpace{}%
\AgdaBound{v}\AgdaSymbol{)}\AgdaSpace{}%
\AgdaSymbol{→}\AgdaSpace{}%
\AgdaGeneralizable{ip}\AgdaSpace{}%
\AgdaOperator{\AgdaDatatype{∈}}\AgdaSpace{}%
\AgdaGeneralizable{Γ}\<%
\\
%
\>[2]\AgdaPostulate{wk}\AgdaSpace{}%
\AgdaSymbol{:}\AgdaSpace{}%
\AgdaSymbol{(}\AgdaBound{v}\AgdaSpace{}%
\AgdaSymbol{:}\AgdaSpace{}%
\AgdaGeneralizable{is}\AgdaSpace{}%
\AgdaOperator{\AgdaDatatype{∈}}\AgdaSpace{}%
\AgdaGeneralizable{Γ}\AgdaSymbol{)}\AgdaSpace{}%
\AgdaSymbol{→}\AgdaSpace{}%
\AgdaDatatype{E}\AgdaSpace{}%
\AgdaSymbol{(}\AgdaGeneralizable{Γ}\AgdaSpace{}%
\AgdaOperator{\AgdaFunction{/}}\AgdaSpace{}%
\AgdaBound{v}\AgdaSymbol{)}\AgdaSpace{}%
\AgdaGeneralizable{ip}\AgdaSpace{}%
\AgdaSymbol{→}\AgdaSpace{}%
\AgdaDatatype{E}\AgdaSpace{}%
\AgdaGeneralizable{Γ}\AgdaSpace{}%
\AgdaGeneralizable{ip}\<%
\\
%
\\[\AgdaEmptyExtraSkip]%
\>[0]\AgdaComment{--\ Nicer\ syntax\ for\ common\ case:}\<%
\\
\>[0]\AgdaKeyword{infixr}\AgdaSpace{}%
\AgdaNumber{18}\AgdaSpace{}%
\AgdaOperator{\AgdaFunction{↑\AgdaUnderscore{}}}\<%
\\
\>[0]\AgdaOperator{\AgdaFunction{↑\AgdaUnderscore{}}}\AgdaSpace{}%
\AgdaSymbol{:}\AgdaSpace{}%
\AgdaDatatype{E}\AgdaSpace{}%
\AgdaGeneralizable{Γ}\AgdaSpace{}%
\AgdaGeneralizable{is}\AgdaSpace{}%
\AgdaSymbol{→}\AgdaSpace{}%
\AgdaDatatype{E}\AgdaSpace{}%
\AgdaSymbol{(}\AgdaGeneralizable{Γ}\AgdaSpace{}%
\AgdaOperator{\AgdaInductiveConstructor{▹}}\AgdaSpace{}%
\AgdaGeneralizable{ip}\AgdaSymbol{)}\AgdaSpace{}%
\AgdaGeneralizable{is}\<%
\\
\>[0]\AgdaOperator{\AgdaFunction{↑\AgdaUnderscore{}}}\AgdaSpace{}%
\AgdaSymbol{=}\AgdaSpace{}%
\AgdaPostulate{wk}\AgdaSpace{}%
\AgdaInductiveConstructor{here}\<%
\\
%
\\[\AgdaEmptyExtraSkip]%
\>[0]\AgdaKeyword{infixr}\AgdaSpace{}%
\AgdaNumber{18}\AgdaSpace{}%
\AgdaOperator{\AgdaFunction{↑↑\AgdaUnderscore{}}}\<%
\\
\>[0]\AgdaOperator{\AgdaFunction{↑↑\AgdaUnderscore{}}}\AgdaSpace{}%
\AgdaSymbol{:}\AgdaSpace{}%
\AgdaDatatype{E}\AgdaSpace{}%
\AgdaGeneralizable{Γ}\AgdaSpace{}%
\AgdaGeneralizable{is}\AgdaSpace{}%
\AgdaSymbol{→}\AgdaSpace{}%
\AgdaDatatype{E}\AgdaSpace{}%
\AgdaSymbol{(}\AgdaGeneralizable{Γ}\AgdaSpace{}%
\AgdaOperator{\AgdaInductiveConstructor{▹}}\AgdaSpace{}%
\AgdaGeneralizable{ip}\AgdaSpace{}%
\AgdaOperator{\AgdaInductiveConstructor{▹}}\AgdaSpace{}%
\AgdaGeneralizable{iq}\AgdaSymbol{)}\AgdaSpace{}%
\AgdaGeneralizable{is}\<%
\\
\>[0]\AgdaOperator{\AgdaFunction{↑↑\AgdaUnderscore{}}}\AgdaSpace{}%
\AgdaSymbol{=}\AgdaSpace{}%
\AgdaOperator{\AgdaFunction{↑\AgdaUnderscore{}}}\AgdaSpace{}%
\AgdaOperator{\AgdaFunction{∘}}\AgdaSpace{}%
\AgdaOperator{\AgdaFunction{↑\AgdaUnderscore{}}}\<%
\\
%
\\[\AgdaEmptyExtraSkip]%
\>[0]\AgdaKeyword{data}\AgdaSpace{}%
\AgdaDatatype{Eq}\AgdaSpace{}%
\AgdaSymbol{:}\AgdaSpace{}%
\AgdaGeneralizable{is}\AgdaSpace{}%
\AgdaOperator{\AgdaDatatype{∈}}\AgdaSpace{}%
\AgdaGeneralizable{Γ}\AgdaSpace{}%
\AgdaSymbol{→}\AgdaSpace{}%
\AgdaGeneralizable{ip}\AgdaSpace{}%
\AgdaOperator{\AgdaDatatype{∈}}\AgdaSpace{}%
\AgdaGeneralizable{Γ}\AgdaSpace{}%
\AgdaSymbol{→}\AgdaSpace{}%
\AgdaPrimitive{Set}\AgdaSpace{}%
\AgdaKeyword{where}\<%
\\
\>[0][@{}l@{\AgdaIndent{0}}]%
\>[2]\AgdaInductiveConstructor{eq}\AgdaSpace{}%
\AgdaSymbol{:}\AgdaSpace{}%
\AgdaSymbol{\{}\AgdaBound{x}\AgdaSpace{}%
\AgdaSymbol{:}\AgdaSpace{}%
\AgdaGeneralizable{is}\AgdaSpace{}%
\AgdaOperator{\AgdaDatatype{∈}}\AgdaSpace{}%
\AgdaGeneralizable{Γ}\AgdaSymbol{\}}\AgdaSpace{}%
\AgdaSymbol{→}\AgdaSpace{}%
\AgdaDatatype{Eq}\AgdaSpace{}%
\AgdaBound{x}\AgdaSpace{}%
\AgdaBound{x}\<%
\\
%
\>[2]\AgdaInductiveConstructor{neq}\AgdaSpace{}%
\AgdaSymbol{:}\AgdaSpace{}%
\AgdaSymbol{(}\AgdaBound{x}\AgdaSpace{}%
\AgdaSymbol{:}\AgdaSpace{}%
\AgdaGeneralizable{is}\AgdaSpace{}%
\AgdaOperator{\AgdaDatatype{∈}}\AgdaSpace{}%
\AgdaGeneralizable{Γ}\AgdaSymbol{)}\AgdaSpace{}%
\AgdaSymbol{→}\AgdaSpace{}%
\AgdaSymbol{(}\AgdaBound{y}\AgdaSpace{}%
\AgdaSymbol{:}\AgdaSpace{}%
\AgdaGeneralizable{ip}\AgdaSpace{}%
\AgdaOperator{\AgdaDatatype{∈}}\AgdaSpace{}%
\AgdaSymbol{(}\AgdaGeneralizable{Γ}\AgdaSpace{}%
\AgdaOperator{\AgdaFunction{/}}\AgdaSpace{}%
\AgdaBound{x}\AgdaSymbol{))}\AgdaSpace{}%
\AgdaSymbol{→}\AgdaSpace{}%
\AgdaDatatype{Eq}\AgdaSpace{}%
\AgdaBound{x}\AgdaSpace{}%
\AgdaSymbol{(}\AgdaPostulate{wkv}\AgdaSpace{}%
\AgdaBound{x}\AgdaSpace{}%
\AgdaBound{y}\AgdaSymbol{)}\<%
\\
%
\\[\AgdaEmptyExtraSkip]%
\>[0]\AgdaKeyword{postulate}\<%
\\
\>[0][@{}l@{\AgdaIndent{0}}]%
\>[2]\AgdaPostulate{eq?}\AgdaSpace{}%
\AgdaSymbol{:}\AgdaSpace{}%
\AgdaSymbol{(}\AgdaBound{x}\AgdaSpace{}%
\AgdaSymbol{:}\AgdaSpace{}%
\AgdaGeneralizable{is}\AgdaSpace{}%
\AgdaOperator{\AgdaDatatype{∈}}\AgdaSpace{}%
\AgdaGeneralizable{Γ}\AgdaSymbol{)}\AgdaSpace{}%
\AgdaSymbol{→}\AgdaSpace{}%
\AgdaSymbol{(}\AgdaBound{y}\AgdaSpace{}%
\AgdaSymbol{:}\AgdaSpace{}%
\AgdaGeneralizable{ip}\AgdaSpace{}%
\AgdaOperator{\AgdaDatatype{∈}}\AgdaSpace{}%
\AgdaGeneralizable{Γ}\AgdaSymbol{)}\AgdaSpace{}%
\AgdaSymbol{→}\AgdaSpace{}%
\AgdaDatatype{Eq}\AgdaSpace{}%
\AgdaBound{x}\AgdaSpace{}%
\AgdaBound{y}\<%
\\
%
\>[2]\AgdaPostulate{sub}\AgdaSpace{}%
\AgdaSymbol{:}\AgdaSpace{}%
\AgdaSymbol{(}\AgdaBound{v}\AgdaSpace{}%
\AgdaSymbol{:}\AgdaSpace{}%
\AgdaGeneralizable{is}\AgdaSpace{}%
\AgdaOperator{\AgdaDatatype{∈}}\AgdaSpace{}%
\AgdaGeneralizable{Γ}\AgdaSymbol{)}\AgdaSpace{}%
\AgdaSymbol{→}\AgdaSpace{}%
\AgdaDatatype{E}\AgdaSpace{}%
\AgdaGeneralizable{Γ}\AgdaSpace{}%
\AgdaGeneralizable{ip}\AgdaSpace{}%
\AgdaSymbol{→}\AgdaSpace{}%
\AgdaDatatype{E}\AgdaSpace{}%
\AgdaSymbol{(}\AgdaGeneralizable{Γ}\AgdaSpace{}%
\AgdaOperator{\AgdaFunction{/}}\AgdaSpace{}%
\AgdaBound{v}\AgdaSymbol{)}\AgdaSpace{}%
\AgdaGeneralizable{is}\AgdaSpace{}%
\AgdaSymbol{→}\AgdaSpace{}%
\AgdaDatatype{E}\AgdaSpace{}%
\AgdaSymbol{(}\AgdaGeneralizable{Γ}\AgdaSpace{}%
\AgdaOperator{\AgdaFunction{/}}\AgdaSpace{}%
\AgdaBound{v}\AgdaSymbol{)}\AgdaSpace{}%
\AgdaGeneralizable{ip}\<%
\\
\>[0]\<%
\end{code}
\begin{code}%
\>[0]\AgdaFunction{opt}\AgdaSpace{}%
\AgdaSymbol{:}\AgdaSpace{}%
\AgdaDatatype{E}\AgdaSpace{}%
\AgdaGeneralizable{Γ}\AgdaSpace{}%
\AgdaGeneralizable{is}\AgdaSpace{}%
\AgdaSymbol{→}\AgdaSpace{}%
\AgdaDatatype{E}\AgdaSpace{}%
\AgdaGeneralizable{Γ}\AgdaSpace{}%
\AgdaGeneralizable{is}\<%
\\
\>[0]\AgdaFunction{opt}\AgdaSpace{}%
\AgdaSymbol{(}\AgdaInductiveConstructor{selₛ}\AgdaSpace{}%
\AgdaBound{e}\AgdaSpace{}%
\AgdaBound{e₁}\AgdaSymbol{)}\AgdaSpace{}%
\AgdaKeyword{with}\AgdaSpace{}%
\AgdaFunction{opt}\AgdaSpace{}%
\AgdaBound{e}\AgdaSpace{}%
\AgdaSymbol{|}\AgdaSpace{}%
\AgdaFunction{opt}\AgdaSpace{}%
\AgdaBound{e₁}\<%
\\
\>[0]\AgdaSymbol{...}\AgdaSpace{}%
\AgdaSymbol{|}\AgdaSpace{}%
\AgdaInductiveConstructor{imapb}\AgdaSpace{}%
\AgdaBound{m}\AgdaSpace{}%
\AgdaBound{e}%
\>[24]\AgdaSymbol{|}\AgdaSpace{}%
\AgdaBound{i}\AgdaSpace{}%
\AgdaSymbol{=}\AgdaSpace{}%
\AgdaInductiveConstructor{selₛ}\AgdaSpace{}%
\AgdaSymbol{(}\AgdaPostulate{sub}\AgdaSpace{}%
\AgdaInductiveConstructor{v₀}\AgdaSpace{}%
\AgdaBound{e}\AgdaSpace{}%
\AgdaSymbol{(}\AgdaInductiveConstructor{div}\AgdaSpace{}%
\AgdaBound{m}\AgdaSpace{}%
\AgdaBound{i}\AgdaSymbol{))}\AgdaSpace{}%
\AgdaSymbol{(}\AgdaInductiveConstructor{mod}\AgdaSpace{}%
\AgdaBound{m}\AgdaSpace{}%
\AgdaBound{i}\AgdaSymbol{)}\<%
\\
\>[0]\AgdaSymbol{...}\AgdaSpace{}%
\AgdaSymbol{|}\AgdaSpace{}%
\AgdaInductiveConstructor{slide}\AgdaSpace{}%
\AgdaBound{i}\AgdaSpace{}%
\AgdaBound{pl}\AgdaSpace{}%
\AgdaBound{a}\AgdaSpace{}%
\AgdaBound{su}%
\>[24]\AgdaSymbol{|}\AgdaSpace{}%
\AgdaBound{k}\AgdaSpace{}%
\AgdaSymbol{=}\AgdaSpace{}%
\AgdaInductiveConstructor{selₛ}\AgdaSpace{}%
\AgdaBound{a}\AgdaSpace{}%
\AgdaSymbol{(}\AgdaInductiveConstructor{ix-plus}\AgdaSpace{}%
\AgdaBound{i}\AgdaSpace{}%
\AgdaBound{k}\AgdaSpace{}%
\AgdaBound{su}\AgdaSpace{}%
\AgdaBound{pl}\AgdaSymbol{)}\<%
\\
\>[0]\AgdaComment{---\ |\ ...\ as\ before\ ...}\<%
\end{code}
Here we optimise away scalar selections into blocked imaps.  Recall that $m$ tells us
that we have an array of shape $s * p$, and $e$ computes blocks of shape $p$.  If we
are selecting into such a blocked array at the index $i$, we know that we are selecting
$(i / p)$-th block, and from that block we are selecting $(i \% p)$ element.  Existence
of explicit \AC{div} and \AC{mod} operations on indices makes it possible to implement
this rewrite rule that is again very similar to $\beta$-reduction.
\begin{code}[hide]%
\>[0]\AgdaCatchallClause{\AgdaSymbol{...}}\AgdaSpace{}%
\AgdaCatchallClause{\AgdaSymbol{|}}\AgdaSpace{}%
\AgdaCatchallClause{\AgdaBound{a}}%
\>[24]\AgdaCatchallClause{\AgdaSymbol{|}}\AgdaSpace{}%
\AgdaCatchallClause{\AgdaBound{i}}\AgdaSpace{}%
\AgdaSymbol{=}\AgdaSpace{}%
\AgdaInductiveConstructor{selₛ}\AgdaSpace{}%
\AgdaBound{a}\AgdaSpace{}%
\AgdaBound{i}\<%
\end{code}
\begin{code}%
\>[0]\AgdaFunction{opt}\AgdaSpace{}%
\AgdaSymbol{(}\AgdaInductiveConstructor{sum}\AgdaSpace{}%
\AgdaBound{e}\AgdaSymbol{)}\AgdaSpace{}%
\AgdaKeyword{with}\AgdaSpace{}%
\AgdaFunction{opt}\AgdaSpace{}%
\AgdaBound{e}\<%
\\
\>[0]\AgdaSymbol{...}\AgdaSpace{}%
\AgdaSymbol{|}\AgdaSpace{}%
\AgdaInductiveConstructor{zero-but}\AgdaSpace{}%
\AgdaSymbol{(}\AgdaInductiveConstructor{var}\AgdaSpace{}%
\AgdaBound{i}\AgdaSymbol{)}\AgdaSpace{}%
\AgdaSymbol{(}\AgdaInductiveConstructor{ix-plus}\AgdaSpace{}%
\AgdaSymbol{(}\AgdaInductiveConstructor{var}\AgdaSpace{}%
\AgdaBound{j}\AgdaSymbol{)}\AgdaSpace{}%
\AgdaSymbol{(}\AgdaInductiveConstructor{var}\AgdaSpace{}%
\AgdaBound{k}\AgdaSymbol{)}\AgdaSpace{}%
\AgdaBound{su}\AgdaSpace{}%
\AgdaBound{pl}\AgdaSymbol{)}\AgdaSpace{}%
\AgdaBound{a}%
\>[58]\AgdaKeyword{with}\AgdaSpace{}%
\AgdaPostulate{eq?}\AgdaSpace{}%
\AgdaInductiveConstructor{v₀}\AgdaSpace{}%
\AgdaBound{i}\AgdaSpace{}%
\AgdaSymbol{|}\AgdaSpace{}%
\AgdaPostulate{eq?}\AgdaSpace{}%
\AgdaInductiveConstructor{v₀}\AgdaSpace{}%
\AgdaBound{j}\AgdaSpace{}%
\AgdaSymbol{|}\AgdaSpace{}%
\AgdaPostulate{eq?}\AgdaSpace{}%
\AgdaInductiveConstructor{v₀}\AgdaSpace{}%
\AgdaBound{k}\<%
\\
\>[0]\AgdaSymbol{...}\AgdaSpace{}%
\AgdaSymbol{|}\AgdaSpace{}%
\AgdaInductiveConstructor{neq}\AgdaSpace{}%
\AgdaSymbol{\AgdaUnderscore{}}\AgdaSpace{}%
\AgdaBound{i′}%
\>[16]\AgdaSymbol{|}\AgdaSpace{}%
\AgdaInductiveConstructor{neq}\AgdaSpace{}%
\AgdaSymbol{\AgdaUnderscore{}}\AgdaSpace{}%
\AgdaBound{j′}%
\>[28]\AgdaSymbol{|}\AgdaSpace{}%
\AgdaInductiveConstructor{eq}%
\>[40]\AgdaSymbol{=}\AgdaSpace{}%
\AgdaInductiveConstructor{ix-minus}%
\>[53]\AgdaSymbol{(}\AgdaInductiveConstructor{var}\AgdaSpace{}%
\AgdaBound{i′}\AgdaSymbol{)}\AgdaSpace{}%
\AgdaSymbol{(}\AgdaInductiveConstructor{var}\AgdaSpace{}%
\AgdaBound{j′}\AgdaSymbol{)}\AgdaSpace{}%
\AgdaBound{pl}\AgdaSpace{}%
\AgdaBound{su}\AgdaSpace{}%
\AgdaBound{a}\<%
\\
\>[0]\AgdaSymbol{...}\AgdaSpace{}%
\AgdaSymbol{|}\AgdaSpace{}%
\AgdaInductiveConstructor{neq}\AgdaSpace{}%
\AgdaSymbol{\AgdaUnderscore{}}\AgdaSpace{}%
\AgdaBound{i′}%
\>[16]\AgdaSymbol{|}\AgdaSpace{}%
\AgdaInductiveConstructor{eq}%
\>[28]\AgdaSymbol{|}\AgdaSpace{}%
\AgdaInductiveConstructor{neq}\AgdaSpace{}%
\AgdaSymbol{\AgdaUnderscore{}}\AgdaSpace{}%
\AgdaBound{k′}%
\>[40]\AgdaSymbol{=}\AgdaSpace{}%
\AgdaInductiveConstructor{ix-minusᵣ}%
\>[53]\AgdaSymbol{(}\AgdaInductiveConstructor{var}\AgdaSpace{}%
\AgdaBound{i′}\AgdaSymbol{)}\AgdaSpace{}%
\AgdaSymbol{(}\AgdaInductiveConstructor{var}\AgdaSpace{}%
\AgdaBound{k′}\AgdaSymbol{)}\AgdaSpace{}%
\AgdaBound{pl}\AgdaSpace{}%
\AgdaBound{su}\AgdaSpace{}%
\AgdaBound{a}\<%
\\
\>[0]\AgdaCatchallClause{\AgdaSymbol{...}}\AgdaSpace{}%
\AgdaCatchallClause{\AgdaSymbol{|}}\AgdaSpace{}%
\AgdaCatchallClause{\AgdaSymbol{\AgdaUnderscore{}}}%
\>[16]\AgdaCatchallClause{\AgdaSymbol{|}}\AgdaSpace{}%
\AgdaCatchallClause{\AgdaSymbol{\AgdaUnderscore{}}}%
\>[28]\AgdaCatchallClause{\AgdaSymbol{|}}\AgdaSpace{}%
\AgdaCatchallClause{\AgdaSymbol{\AgdaUnderscore{}}}%
\>[40]\AgdaSymbol{=}\AgdaSpace{}%
\AgdaInductiveConstructor{sum}\AgdaSpace{}%
\AgdaSymbol{(}\AgdaInductiveConstructor{zero-but}\AgdaSpace{}%
\AgdaSymbol{(}\AgdaInductiveConstructor{var}\AgdaSpace{}%
\AgdaBound{i}\AgdaSymbol{)}\AgdaSpace{}%
\AgdaSymbol{(}\AgdaInductiveConstructor{ix-plus}\AgdaSpace{}%
\AgdaSymbol{(}\AgdaInductiveConstructor{var}\AgdaSpace{}%
\AgdaBound{j}\AgdaSymbol{)}\AgdaSpace{}%
\AgdaSymbol{(}\AgdaInductiveConstructor{var}\AgdaSpace{}%
\AgdaBound{k}\AgdaSymbol{)}\AgdaSpace{}%
\AgdaBound{su}\AgdaSpace{}%
\AgdaBound{pl}\AgdaSymbol{)}\AgdaSpace{}%
\AgdaBound{a}\AgdaSymbol{)}\<%
\\
\>[0]\AgdaComment{---\ |\ ...\ as\ before\ ...}\<%
\end{code}
\begin{code}[hide]%
\>[0]\AgdaCatchallClause{\AgdaFunction{opt}}\AgdaSpace{}%
\AgdaCatchallClause{\AgdaSymbol{(}}\AgdaCatchallClause{\AgdaInductiveConstructor{sum}}\AgdaSpace{}%
\AgdaCatchallClause{\AgdaBound{e}}\AgdaCatchallClause{\AgdaSymbol{)}}\AgdaSpace{}%
\AgdaCatchallClause{\AgdaSymbol{|}}\AgdaSpace{}%
\AgdaCatchallClause{\AgdaBound{a}}\AgdaSpace{}%
\AgdaSymbol{=}\AgdaSpace{}%
\AgdaInductiveConstructor{sum}\AgdaSpace{}%
\AgdaBound{a}\<%
\end{code}
Here we are dealing with the sum over summation index $t$ where the inner expression is
a conditional on indices of the form \texttt{i == j + k ? e : 0}.  Here we apply the
same comparison of index variables as before.  If $k$ happens to be the variable $t$,
then overall sum will only add one non-zero element at $(i-j)$-th index, given that this
(left) subtraction is possible in the sense of existence of the inverse to \AF{\_⊕ₚ\_} operation
defined in Section~\ref{sec:general-ix-ops}.  The same happens when the summation index
$t$ is equal to $j$, we only need to consider $(i-k)$-th element given that this (right)
subtraction is possible.  One could cover other cases where $t$ is equal to $i$, or
when $i$ and $j+k$ are swapped, but these are not occurring in our running example.

\begin{code}%
\>[0]\AgdaFunction{opt}\AgdaSpace{}%
\AgdaSymbol{(}\AgdaInductiveConstructor{scaledown}\AgdaSpace{}%
\AgdaBound{x}\AgdaSpace{}%
\AgdaBound{e}\AgdaSymbol{)}\AgdaSpace{}%
\AgdaKeyword{with}\AgdaSpace{}%
\AgdaFunction{opt}\AgdaSpace{}%
\AgdaBound{e}\<%
\\
\>[0]\AgdaSymbol{...}\AgdaSpace{}%
\AgdaSymbol{|}\AgdaSpace{}%
\AgdaInductiveConstructor{sum}\AgdaSpace{}%
\AgdaBound{a}\AgdaSpace{}%
\AgdaSymbol{=}\AgdaSpace{}%
\AgdaInductiveConstructor{sum}\AgdaSpace{}%
\AgdaSymbol{(}\AgdaInductiveConstructor{scaledown}\AgdaSpace{}%
\AgdaBound{x}\AgdaSpace{}%
\AgdaBound{a}\AgdaSymbol{)}\<%
\\
\>[0]\AgdaComment{---\ |\ ...\ as\ before\ ...}\<%
\end{code}
Finally, here is a rule that looks very innocent in the high-level language, yet
becomes of importance in the low-level one.  The rule says that if we are summing
the array and then dividing it by a constant, we should move division inside the
summation.  The reason for this rewrite rule being important is when the result
of the sum is non-scalar, we need to create a temporary array, before scaling down
all its elements.  A language with first class arrays can obviously take care of
such minor details, but in C we have to be explicit about it.
\begin{code}[hide]%
\>[0]\AgdaCatchallClause{\AgdaSymbol{...}}\AgdaSpace{}%
\AgdaCatchallClause{\AgdaSymbol{|}}\AgdaSpace{}%
\AgdaCatchallClause{\AgdaBound{a}}\AgdaSpace{}%
\AgdaSymbol{=}\AgdaSpace{}%
\AgdaInductiveConstructor{scaledown}\AgdaSpace{}%
\AgdaBound{x}\AgdaSpace{}%
\AgdaBound{a}\<%
\\
\>[0]\AgdaCatchallClause{\AgdaFunction{opt}}\AgdaSpace{}%
\AgdaCatchallClause{\AgdaBound{e}}\AgdaSpace{}%
\AgdaSymbol{=}\AgdaSpace{}%
\AgdaBound{e}\<%
\end{code}

\subsubsection{Code Generation}
Due to space limitations, we only consider the basic mechanisms we used in the
code generator, all the code is available in supplementary materials.  We use
heap-allocated multi-dimensional arrays that can be defined as follows:
\begin{lstlisting}[language=C]
  float(*k1)[6][5][5] = malloc(sizeof(*k1));
\end{lstlisting}
This ensures that \texttt{k1} is represented as a continuous region of memory
of size $6*5*5$ floats.  When such arrays are indexed (\eg{} \texttt{(*k1)[i][j][k]}),
the indices are translated into a single offset into the continuous memory.
Therefore, there is no pointer chasing which makes this approach efficient at
runtime.  As C uses row-major order to compute the offsets, we do obtain
partial array selections on the left, \eg \texttt{(*k1)[i]} is a $5\times 5$
array that can be further indexed or passed to \texttt{sizeof} that correctly
identifies the size of this subarray.  Surely, this is a pointer into the \texttt{k1}
array, so all the modifications to \texttt{(*k1)[i]} will modify \texttt{k1}.
As a great convenience feature, C compiler tracks the ranges of the indices
and produces warnings in cases when it figures out that ranges of indices
and the array we are indexing do not match.

Whenever we translate some $e$ in \AF{E} into C, we have to provide a storage
where $e$ has to be written to.  In case of compiling the \AF{Chain} every
local variable becomes such a storage for the bound expression.  Therefore,
our extractor always has a result variable as an argument.

For example, let us consider an expression $a ⊞ a$ of shape
(\AC{ι} 5 \AC{⊗} \AC{ι} 5), where $a$ is mapped to the C variable
\texttt{float (*a)[5][5]} that is written to the result variable
\texttt{float (*r)[5][5]}.  Here is the code that we generate:
\begin{lstlisting}[language=C]
  for (size_t x1_1 = 0; x1_1 < 5; x1_1++) { 
    for (size_t x1_2 = 0; x1_2 < 5; x1_2++) { 
      (*r)[x1_1][x1_2] = ((*a)[x1_1][x1_2] + (*a)[x1_1][x1_2]);
    }}
\end{lstlisting}
We started with checking that $a ⊞ a$ is a \emph{selectable} expression.
This means that we can always generate expression at the given index.
As we know that the shape of $a ⊞ a$ is (\AC{ι} 5 \AC{⊗} \AC{ι} 5),
we need to generate a loop nest of that shape that assigns where
we assign the expression at the given index to the result at the given
index.

We need to distinguish whether we are writing into the result or adding
into it as in cases when dealing with \AF{sum}.  Consider the code that
is generated for (\AC{sum} (\AC{selₛ} (\AB{a} (\AC{var v₀}))) where
we are adding all the elements of the array $a$ into result variable
\texttt{float (*r)[1]}:
\begin{lstlisting}[language=C]
  for (size_t x2_1 = 0; x2_1 < 5; x2_1++) {
    for (size_t x2_2 = 0; x2_2 < 5; x2_2++) {
      for (size_t x3_1 = 0; x3_1 < 1; x3_1++) { 
        (*r)[x3_1] += (&(*a)[x2_1][x2_2])[x3_1];
      }}}
\end{lstlisting}
Two things are happening here, first we generate \texttt{+=} assignment
and we make an implicit assumption that resulting variables are initialised
to zero.  In the extractor, additionally to the resulting variable we
track whether we need to do an assignment or assignment with addition.
Secondly, while $a$ is two-dimensional, we have three-dimensional loop
nest.  The latter comes from the representation of scalars as 1-element
vectors.  When we resolved the two-dimensional summation index \texttt{x2},
we know that we need to assign into the object of shape (\AC{ι} 1), but
the left-hand-side is a scalar (float).  The trick here is that in C we
can always turn scalars into 1-element vectors by simply taking the address
of the scalar.  This is why we have this 1-iteration for-loop over
\texttt{x3\_1} that will be immediately optimised away by the C compiler.

Finally, when we it comes to the operation on indices, such as addition,
subtraction, division or modulo, we generate the corresponding operation
on the individual loop indices.

Remaining details of the code generation take care of traversing through
the structure of \AF{E} with some plumbing that has to do with generating
loop-nests around expressions and checking that they are selectable.

\subsubsection{Running the Generated C Code}
In order to run the generated C code we translate the boilerplate code
from SaC to C.  While doing so, we made sure that our code can be run
in parallel.  While the  SaC compiler does this automatically, there is one
obvious loop that requires parallelisation which is computation of
the batch.  When we train the CNN, we take a batch of images and the
weights and we compute gradients for those weights per every image.
After that we average all the gradients in the batch, and we update
the weights, after all the batch is processed.  Clearly, all the
gradient computations in the batch can run in parallel.  We achieve
this by organising the batch loop such that all the gradients are
stored in a separate memory region, and we parallelise this loop
using OpenMP annotations.

We verify that the code that we generate compute the same results
as the hand-written SaC code.  Then we replicate the experiment from
the~\cite{cnn-array} using 40 epochs, 100 images in the batch, and
feeding 10000 training images.  We run the experiment on the 18-core
13th Gen Intel(R) Core(TM) i5-13600K machine using sac2c version
\texttt{1.3.3-MijasCosta-1161-gb543c} and the GCC compiler
version \texttt{12.2.0}.  The first thing that we learn is that
our generated C code is sensible (factor of 3 running time)
to the compilation flags that we enable.  We identified the set
of flags that when passed to both compilers\footnote{SaC compiler
generates C code, so we can control what flags it uses when
compiling it.}, the runtime at the
largest number of cores are 11s for the hand-written SaC implementation 
and 13.5s for the generated C code, with
very little variance.  This 20\% difference is orthogonal
to parallel execution, as it is also observed when running
the code on a single core.  The set of flags has to do with
floating point operations: \texttt{-fno-signed-zeros} ignores
the distinction between negative and positive zeroes that is given
by IEEE 754 standard, allowing to reduce (-0.0*x) to 0.0;
\texttt{-fno-math-errno} does not set errno after calling math functions;
\texttt{-fno-trapping-math} and \texttt{-fassociative-math} make
sure that we can assume associativity of floating point operations
which does not hold according to the IEEE 754.

The main performance difference comes from the fact that
compiled SaC code uses less intermediate arrays, significantly reducing the number
of memory writes.  There are numerous ways how to improve the performance
of the generated C code, but for the purposes of this paper we consider that getting within
20\% of the hand-written SaC code is sufficient evidence for our hypothesis
that the two-languages approach seems viable for achieving proved
correctness and performance.
We have automatic differentiation in the safe environment
that generates the C code that runs almost as fast as the hand-written
SaC code.



















\section{Related Work\label{sec:relatedwork}}

In the following we relate the specifics of our contribution to prior
work. One important distinction however, which is easily obscured by
this detail-oriented comparison, is that our work combines parts that
have been subject to prior work, but are not commonly found together.

% - "Verified Tensor-Program Optimization Via High-level Scheduling Rewrites"
% - "Efficient Differentiable Programming in a Functional Array-Processing Language"
% - "Verified tensor-program optimization via high-level scheduling rewrites"
% - "Indexed Streams: A Formal Intermediate Representation for Fused Contraction Programs"
% - "You only linearize once"
% - Dependent ML
% - ATS
% - Jax
% - Remora

\subsection{Automatic Differentiation}

Automatic differentiation has been around for many decades~\cite{early-ad1, early-ad2},
so it is well-understood at a conceptual level.  However,
a number of questions related to bringing AD into the context of
programming languages remain open.  Recent successes in machine learning
have spurred further interest in AD which has led to several new developments.
For the context of this paper, we focus on recent work that contributes to 
the perspective of balancing correctness guarantees and performance.
Our selection here is by no means exhaustive, for
a broader scope we refer the reader to~\cite{autodiff-survey}.

There has been a number of programming-language-oriented approaches that explain
how to add AD to a programming language of choice. Examples of these include
Futhark~\cite{futhark/sc22ad}, Haskell~\cite{ad-haskell}, and
Jax~\cite{ad-jax,radul2023you}. Furthermore, a number of machine learning
frameworks that incorporate AD have been proposed in recent years: TensorFlow~\cite{ad-tf},
PyTorch~\cite{ad-pytorch}, MXNet~\cite{ad-mxnet} and many more.
While in particular the dedicated frameworks tend to find widespread 
acceptance by practitioners, both, correctness and performance leave
two open questions: (i) how is it possible to
ensure that the AD algorithm is implemented correctly?
\todo[inline]{reviewer 3 (2024) notices: This paper does not really answer this question besides enforcing consistency of shapes and accesses? Besides, final extraction could break these properties?  this is correct, we have to clarify this.}
(ii) if the
language or the framework do not perform as expected, what are the
options to solve this?  Unfortunately, for many cases the answer to
both questions is unsatisfying.  Most of the languages/frameworks do not
come with formal correctness guarantees, so one has to trust the
implementers of these tools.  One can run tests as well to gain trust 
in the implementation but that is far from a 
formal guarantee.  If one relies
on the AD provided by a chosen language/framework, and the generated code does not
perform well, one has to modify the language/framework, as these solutions
are tightly integrated with the tools. The problem here is that most of of these tools
have very large and sophisticated implementations typically comprising
of hundreds of thousands of lines of code.

Another line of work studies high-level principles of AD using
category theory~\cite{ad-theor1, ad-theor2, ad-theor3}.
While this indeed comes with great correctness guarantees due to
some naturality principles, it is not always clear how to implement
this in a way that leads to efficiently executable specifications.  Also, the
entire treatment of index-safe tensors is unclear.

In~\cite{ad-elliott} the author proposes to view AD problem using
the language of cartesian categories.  It has been shown that
this approach can be used in practice by implementing the proposed
technique in Haskell.  Type classes are a vehicle to restrict expressions
that are differentiable.  There is a Haskell plugin that translates
expressions that are instances of the mentioned type classes into
categorical primitives, AD is performed on these and the code is reflected
back to Haskell.  This is a nice approach that makes it easy
to verify the correctness of the algorithm.  However, the treatment
of tensors and general extractability remains a little unclear.
While it is briefly mentioned that representable functors
are supported, it is unclear whether this is sufficient to
represent rank-polymorphic arrays with strict bound checks.
Also, correctness guarantees are inevitably restricted by the
Haskell type system, so we are likely to find invariants that
are inexpressible in that setup.

\subsection{Verified high performance computing}

One popular line of research is based on the idea of separating the
high level \emph{specification} of a problem from the \emph{schedule}
that describe how it is to be executed. Verifying that such schedules
are semantics-preserving is the topic of several recent
publications~\cite{10.1145/3527328,10.1145/3498717}. Our approach
differs by not using a scheduling language, and instead focuses on
verifying the specification itself, and also by having a particularly
expressive specification language.

Thiemann and Chakravarty demonstrated a prototype embedding of
Accelerate~\cite{10.1145/1926354.1926358} (a Haskell library for
accelerator programming) in Agda~\cite{thiemann2013agda}. Their
approach was based on dynamic code generation by invoking Accelerate
through Agda's foreign function interface, which posed various
unresolved implementation challenges. Even though our for support rank
polymorphism results in having even more complicated types, our code
extraction, which is essentially ahead-of-time compilation, is fairly
unproblematic.

\subsection{Type systems}

Verifying index operations is a classic application of dependent
types, as seen in for example Dependent
ML~\cite{10.1145/292540.292560}, although most work assumes static
ranks. Our target language, Futhark, supports size-dependent
types~\cite{10.1145/3609024.3609412}, but this information is used
only to ensure shape conformance (e.g., that the operands to a matrix
multiplication have the same size), not to ensure the correctness of
array indexing. Single-Assignment C~\cite{sac2} (SaC) is a statically
typed language with support for rank polymorphism, including dynamic
ranks, but it does not guarantee that absence of indexing errors.

Gibbons showed how to express rank polymorphism in Haskell, through
the use of \emph{Naperian functors}~\cite{10.1145/2976022.2976023}.
This form of rank polymorphism is somewhat more limited than ours; in
particular, it imposes static ranks. Further, Gibbons' work focused
solely on the embedding in Haskell, while our system also supports
code extraction and demonstrates good real-world performance.
Similarly, Remora~\cite{rank-poly} is a dependently typed
rank-polymorphic language that also supports only static ranks.

\section{Conclusions\label{sec:conclusions}}

The paper demonstrates a technique of developing high performance
applications with strong correctness guarantees, including the absence
of out-of-bound indexing, certain functions being inverses, as well as
well-scopedness and well-typedness of our DSL.

The key insight lies in using a proof assistant in cooperation with a
high-performance language of choice. This gives a clear separation of
concerns that is very difficult to achieve within a single language.
The proof assistant is used to design a specification, prove all the
correctness invariants of interest and performs an extraction into a
high-performance language, in our case Futhark, although only a
relatively small part of our work is specific to that language.

Having a trusted specification as well as entire code-generation
pipeline within a single dependently-typed framework is incredibly
powerful. As we have demonstrated at the example of the neural
network, we can introduce domain-specific optimisations and
transformations, such as automatic differentiation. For our example,
the entire framework that includes array theory, DSL, optimisations
and extraction is about 2000 lines of Agda code.

The asymptotic efficiency of our AD implementation depends crucially
on ``ad-hoc'' optimisations of summations of arrays with only a single
nonzero element. Although effective in many cases, this can be seen as
a somewhat fragile technique, and will likely not work for a richer
language with more complicated control flow, but this limitation is
not fundamental to our approach: in the future, we could for example
use an sparse representation of adjoint arrays instead.

A lot of pieces that we have developed in this paper can be reused in
other numerical applications. However, there are many more
opportunities that we did not explore. For example, one can prove the
correctness of optimisations, relating evaluation of optimised an
non-optimised expressions. We can provide more guarantees when we run
extraction, \eg{} we can formalise some aspects of the backend
language and relate them to our DSL. As for the DSL itself, the exact
choice of supported primitives, and its implications for extracting
high performance code or performing optimisation or transformation,
remains an interesting question.

There are indeed plenty of opportunities, but the key point is this.
Correctness and performance are competing requirements when it comes
to application design. Therefore, such a cooperation between
correctness-oriented and performance-oriented tools is likely to
persist. With this work we demonstrate that such cooperation can be
done using fairly straightforward means, by Agda standards, and obtain
compelling practical performance.


\bibliographystyle{ACM-Reference-Format}
\bibliography{paper}

\end{document}
