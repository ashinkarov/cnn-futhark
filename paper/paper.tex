\documentclass[acmsmall,screen]{acmart}
\usepackage{listings}
\usepackage{pgfplots}
\usepackage{pgfplotstable}
\usepackage{microtype}
\usepackage{mathtools} % needed for \doublecolon symbol
\usepackage{bbold} % For extended blackboard bold symbols
\usepackage{fancyvrb}

\usepackage{todonotes}

% Add definitions for abbreviations like e.g.; i.e; etc. with
% correct spacing depending on the parameters.  In this case
% `all' exposes all the definitions of the package, and `british'
% makes sure that there is no comma after e.g. or i.e.
\usepackage[all,british]{foreign}

\lstdefinelanguage{SaC}[]{}{%
    language=C,
    morekeywords={inline,return,for,shape,dim,with,fold,modarray,genarray},
    otherkeywords={?,:,->},
    comment=[l][\color{green!50!brown}]{//},
    morecomment=[s][\color{blue!70!gray}]{int[}{]},
    morecomment=[s][\color{blue!70!gray}]{double[}{]}
}
\lstset{%
    basicstyle=\fontsize{8}{8}\selectfont\ttfamily,
    language=SaC,
    captionpos=b,
    keywordstyle=\color{blue!70!gray},
    commentstyle=\color{green!50!brown},
    stringstyle=\color{violet!40!magenta},
    %showstringspaces=false,
}

% https://tex.stackexchange.com/questions/318142/typeseeting-a-multiset-with-double-curly-braces
\newcommand*{\xlbrace}{[\mskip-5mu[}
\newcommand*{\xrbrace}{]\mskip-5mu]}

\newcommand*{\ldblbrace}{\{\mskip-5mu\{}
\newcommand*{\rdblbrace}{\}\mskip-5mu\}}


\usepackage{bm}
\usepackage{agda}
\usepackage{newunicodechar}

\usepackage{mathpartir}
\usepackage{varwidth}
% Do we need this?
\usepackage{microtype} 

%\usepackage{tikz}
\usepackage{float}
\usepackage{wrapfig}

% Fixing overfull lines
\emergencystretch=2pt % default 0.0pt
% This removes vertical gaps in the wrapfigure, not sure
% whether there is a more elegant way of doing this.
\setlength{\intextsep}{0pt}


\newcommand\codeblock[1]{%
  %{\fbox{\begin{varwidth}{0.9\textwidth}#1\end{varwidth}}}
  {\begin{varwidth}{0.9\textwidth}#1\end{varwidth}}
} 


\newunicodechar{₀}{\ensuremath{_0}}
\newunicodechar{₁}{\ensuremath{_1}}
\newunicodechar{₂}{\ensuremath{_2}}
\newunicodechar{₃}{\ensuremath{_3}}
\newunicodechar{₄}{\ensuremath{_4}}
\newunicodechar{₅}{\ensuremath{_5}}
\newunicodechar{₆}{\ensuremath{_6}}
\newunicodechar{₇}{\ensuremath{_7}}
\newunicodechar{₈}{\ensuremath{_8}}
\newunicodechar{₉}{\ensuremath{_9}}
\newunicodechar{ι}{\ensuremath{\iota}}
\newunicodechar{ₗ}{\ensuremath{_l}}
\newunicodechar{ₐ}{\ensuremath{_a}}
\newunicodechar{ₖ}{\ensuremath{_k}}
\newunicodechar{ᵢ}{\ensuremath{_i}}
\newunicodechar{ⱼ}{\ensuremath{_j}}
\newunicodechar{ᵥ}{\ensuremath{_v}}
\newunicodechar{ₕ}{\ensuremath{_h}}
\newunicodechar{ₚ}{\ensuremath{_p}}
\newunicodechar{ₛ}{\ensuremath{_s}}
\newunicodechar{ₒ}{\ensuremath{_o}}
\newunicodechar{ₙ}{\ensuremath{_n}}


\newunicodechar{ᵣ}{\ensuremath{_r}}
\newunicodechar{ʳ}{\ensuremath{^r}}
\newunicodechar{ˡ}{\ensuremath{^l}}
\newunicodechar{ᵛ}{\ensuremath{^v}}
\newunicodechar{ᵉ}{\ensuremath{^e}}
\newunicodechar{ℕ}{\ensuremath{\mathbb{N}}}
\newunicodechar{ℝ}{\ensuremath{\mathbb{R}}}
\newunicodechar{𝟘}{\ensuremath{\mathbb{0}}}

\newunicodechar{∀}{\ensuremath{\forall}}
\newunicodechar{∃}{\ensuremath{\exists}}
\newunicodechar{≡}{\ensuremath{\equiv}}
\newunicodechar{≈}{\ensuremath{\approx}}
\newunicodechar{≟}{\ensuremath{\stackrel{?}{=}}}
\newunicodechar{∎}{\ensuremath{\blacksquare}}
\newunicodechar{⊛}{\ensuremath{\circledast}}
\newunicodechar{⊗}{\ensuremath{\otimes}}
\newunicodechar{⊕}{\ensuremath{\oplus}}
\newunicodechar{⊝}{\ensuremath{\ominus}}
\newunicodechar{≤}{\ensuremath{\le}}
\newunicodechar{φ}{\ensuremath{\phi}}
\newunicodechar{ψ}{\ensuremath{\psi}}
\newunicodechar{ε}{\ensuremath{\epsilon}}
\newunicodechar{δ}{\ensuremath{\delta}}
\newunicodechar{λ}{\ensuremath{\lambda}}
\newunicodechar{σ}{\ensuremath{\sigma}}
\newunicodechar{ρ}{\ensuremath{\rho}}
\newunicodechar{ν}{\ensuremath{\nu}}
\newunicodechar{∂}{\ensuremath{\partial}}

\newunicodechar{Γ}{\ensuremath{\Gamma}}
\newunicodechar{Δ}{\ensuremath{\Delta}}
\newunicodechar{Ψ}{\ensuremath{\Phi}}
\newunicodechar{Ξ}{\ensuremath{\Xi}}

\newunicodechar{′}{{'}}
\newunicodechar{∷}{\ensuremath{\dblcolon}}
\newunicodechar{↔}{\ensuremath{\leftrightarrow}}
\newunicodechar{↦}{\ensuremath{\mapsto}}
\newunicodechar{∘}{\ensuremath{\circ}}
\newunicodechar{⁻}{\ensuremath{^{-}}}
\newunicodechar{∙}{\ensuremath{\boldsymbol{\cdot}}}
\newunicodechar{▹}{\ensuremath{\triangleright}}
\newunicodechar{◃}{\ensuremath{\triangleleft}}
\newunicodechar{⋯}{\ensuremath{\cdots}}
\newunicodechar{∇}{\ensuremath{\nabla}}
\newunicodechar{Σ}{\ensuremath{\Sigma}}
\newunicodechar{∈}{\ensuremath{\in}}
\newunicodechar{⟦}{\ensuremath{\llbracket}}
\newunicodechar{⟧}{\ensuremath{\rrbracket}}
\newunicodechar{⦃}{\ensuremath{\ldblbrace}}
\newunicodechar{⦄}{\ensuremath{\rdblbrace}}
\newunicodechar{⌊}{\ensuremath{\lfloor}}
\newunicodechar{⌋}{\ensuremath{\rfloor}}
  

\newunicodechar{⇒}{\ensuremath{\Rightarrow}}
\newunicodechar{⟪}{\ensuremath{\xlbrace}}
\newunicodechar{⟫}{\ensuremath{\xrbrace}}
\newunicodechar{⊠}{\ensuremath{\boxtimes}}
\newunicodechar{⊞}{\ensuremath{\boxplus}}
\newunicodechar{⊟}{\ensuremath{\boxminus}}
\newunicodechar{∋}{\ensuremath{\ni}}
\newunicodechar{∶}{\ensuremath{\bm{:}}}
\newunicodechar{↦}{\ensuremath{\mapsto}}
\newunicodechar{⊤}{\ensuremath{\top}}
\newunicodechar{⊆}{\ensuremath{\subseteq}}
\newunicodechar{⊎}{\ensuremath{\uplus}}





% Some shortcut commands for agda symbols
\newcommand{\AD}[1]{\AgdaDatatype{#1}}
\newcommand{\AC}[1]{\AgdaInductiveConstructor{#1}}
\newcommand{\AF}[1]{\AgdaFunction{#1}}
\newcommand{\AB}[1]{\AgdaBound{#1}}
\newcommand{\AK}[1]{\AgdaKeyword{#1}}
\newcommand{\AR}[1]{\AgdaField{#1}}
\newcommand{\AM}[1]{\AgdaModule{#1}}
\newcommand{\AN}[1]{\AgdaNumber{#1}}
\newcommand{\AS}[1]{\AgdaString{#1}}

\renewcommand{\AgdaCommentFontStyle}[1]{\textrm{#1}}
\renewcommand{\AgdaFontStyle}[1]{\textrm{#1}}
\renewcommand{\AgdaKeywordFontStyle}[1]{\textrm{#1}}

\pgfplotstableset{col sep=comma}


%%% The following is specific to ICFP '25 and the paper
%%% 'Correctness Meets High Performance in Agda'
%%% by Artjoms Šinkarovs and Troels Henriksen.
%%%
\setcopyright{cc}
\setcctype{by-sa}
\acmDOI{10.1145/3747524}
\acmYear{2025}
\acmJournal{PACMPL}
\acmVolume{9}
\acmNumber{ICFP}
\acmArticle{255}
\acmMonth{8}
\received{2025-02-27}
\received[accepted]{2025-06-27}


%\renewcommand{\shortauthors}%
%  {A.~{\v{S}}inkarovs, T.~Koopman, and S.~Scholz}


\pgfkeys{/pgf/number format/.cd,fixed,precision=0}


\keywords{Dependent Types, Agda, Array Programming, Automatic Differentiation, Futhark}

\begin{CCSXML}
<ccs2012>
<concept>
<concept_id>10003752.10003790.10011740</concept_id>
<concept_desc>Theory of computation~Type theory</concept_desc>
<concept_significance>500</concept_significance>
</concept>
<concept>
<concept_id>10011007.10011006.10011039</concept_id>
<concept_desc>Software and its engineering~Formal language definitions</concept_desc>
<concept_significance>500</concept_significance>
</concept>
<concept>
<concept_id>10011007.10011006.10011041</concept_id>
<concept_desc>Software and its engineering~Compilers</concept_desc>
<concept_significance>500</concept_significance>
</concept>
<concept>
<concept_id>10010147.10010169.10010175</concept_id>
<concept_desc>Computing methodologies~Parallel programming languages</concept_desc>
<concept_significance>500</concept_significance>
</concept>
</ccs2012>
\end{CCSXML}

\ccsdesc[500]{Theory of computation~Type theory}
\ccsdesc[500]{Software and its engineering~Formal language definitions}
\ccsdesc[500]{Software and its engineering~Compilers}
\ccsdesc[500]{Computing methodologies~Parallel programming languages}

\begin{document}

\title{Correctness Meets Performance: From Agda to Futhark}

\author{Artjoms Šinkarovs}
\orcid{0000-0003-3292-2985}
\affiliation{%
  \institution{University of Southampton}
  \city{Southampton}
  \country{United Kingdom}
}
\email{artjoms.sinkarovs@pm.me}

\author{Troels Henriksen}
\orcid{0000-0002-1195-9722}
\affiliation{%
  \institution{University of Copenhagen}
  \city{Copenhagen}
  \country{Denmark}
}
\email{athas@sigkill.dk}

\begin{abstract}
In this paper we demonstrate a technique for developing high performance applications
with strong correctness guarantees.  Using a theorem prover, we derive a high-level
specification of the application that includes correctness invariants of our choice.
After that, within the same theorem prover, we implement an extraction of the
specified application into a high-performance language of our choice.  Concretely,
we are using Agda to specify a framework for automatic differentiation (reverse mode)
that is focused on index-safe tensors.  This framework comes
with an optimiser for tensor expressions and the ability to translate these
expressions into Futhark.  We specify a canonical convolutional neural network
within the proposed framework, compute the derivatives needed for the training
phase and then demonstrate that the generated code approaches the performance of TensorFlow
code when running on a GPU.
\end{abstract}

\maketitle

\section{Introduction\label{sec:intro}}

The year is 2025, and programmers still face a trade-off between correctness and
performance. Low-level languages like C or Fortran allow careful control over
hardware, but offer few guarantees about safety or program structure.
Dependently-typed languages like Agda or Lean, on the other hand, make it
possible to encode precise invariants, but typically lack the infrastructure for
high-performance computation.

Rather than chasing a mythical language that promises both complete correctness
and high performance, we explore a practical compromise: leveraging a
dependently-typed proof assistant for the parts of a scientific programming
pipeline where full verification is most crucial, and integrating it with a
high-performance backend for execution. Specifically, we investigate how Agda
can be used not only for specification and reasoning, but also for
transformation, optimisation, and code generation in a realistic machine
learning workload.

To explore this idea, we investigate the concrete problem of automatic
differentiation (AD) which is often found in machine learning applications. This
is a convenient case study as it comes with the following challenges. From the
correctness perspective, it is crucially important to track the shapes and ranks
of the tensors, guaranteeing the absence of out-of-bound indexing. This is a
very common source of errors that can be difficult to find. Secondly, we have to
compute derivatives of the given tensor expressions, preserve safe indexing
guarantees while we do so, and we have to be able to translate the computed
expressions into some high-performance language. As machine learning
applications are known to be numerically intensive problems, our performance
challenge lies in running the program as fast as we can on the chosen hardware
architecture.

In an ideal world, all parts would of course come with full formal
specifications and accompanying proof, but in practice some parts remain an
entirely separate and nontrivial research effort to verify for functional
correctness, such as AD. Other parts---such as code generation--can be handled
using known techniques, but are somewhat time consuming. Despite all this, we
show that even in those cases where resources preclude full verification, Agda
can still be used as a practical tool, with full verification employed where it
is deemed desirable and practical.

We follow~\cite{cnn-array} which demonstrates that it is possible to implement
one of the canonical convolutional neural network (CNN) in the array language
SaC~\cite{sac1, sac2}, obtaining good sequential and parallel performance that
is competitive with TensorFlow~\cite{ad-tf} and PyTorch~\cite{ad-pytorch}.
Focusing on correctness, we propose a theory of rank-polymorphic
arrays~\cite{rank-poly} in Agda~\cite{agda-2-6-3}. Within this framework, we
encode the CNN from~\cite{cnn-array} and lift it into an embedded DSL. We
implement AD (reverse mode) and domain-specific optimisations for expressions in
that DSL. Finally, we implement an extraction into Futhark (a functional array
language), apply the CNN to the MNIST digit recognition problem, and compare
performance with TensorFlow.

As a result, we demonstrate an approach where the entire specification,
optimiser, AD and code generation are available to us within a proof assistant
of our choice. We can prove facts about all the stages of the pipeline and
easily adjust them to our liking. We also demonstrate that some components can
be left only partially verified, where the effort of verification is judged
disproportionate to the benefit. We argue that such a liberating approach is
feasible in practice, at least for the times of dialectic of correctness and
performance. The overall goals is to demonstrate how to formulate a
computationally intensive problem in a proof assistant, transform it with
verification of key safety properties, and then pass it on to the other system
that can run the algorithm efficiently.

The contributions of this paper are as follows:
\begin{enumerate}
  \item a rank-polymorphic array theory and implementation of
        the CNN from~\cite{cnn-array} in Agda;
  \item a deeply-embedded DSL with HOAS wrappers in Agda which supports AD (reverse mode);
  \item an extraction mechanism for generating Futhark code from the DSL; and
  \item an experimental evaluation of the generated code.
\end{enumerate}

This paper is written in literate Agda, which guarantees that all the code
snippets have been type-checked.  All the code is available at \url{https://github.com/ashinkarov/cnn-futhark}.



\section{Array Theory\label{sec:array-theory}}

\begin{code}[hide]%
\>[0]\AgdaKeyword{open}\AgdaSpace{}%
\AgdaKeyword{import}\AgdaSpace{}%
\AgdaModule{Relation.Binary.PropositionalEquality}\<%
\\
\>[0]\AgdaKeyword{open}\AgdaSpace{}%
\AgdaKeyword{import}\AgdaSpace{}%
\AgdaModule{Relation.Nullary}\<%
\\
\>[0]\AgdaKeyword{open}\AgdaSpace{}%
\AgdaKeyword{import}\AgdaSpace{}%
\AgdaModule{Data.List}\AgdaSpace{}%
\AgdaKeyword{using}\AgdaSpace{}%
\AgdaSymbol{(}\AgdaDatatype{List}\AgdaSymbol{;}\AgdaSpace{}%
\AgdaInductiveConstructor{[]}\AgdaSymbol{;}\AgdaSpace{}%
\AgdaOperator{\AgdaInductiveConstructor{\AgdaUnderscore{}∷\AgdaUnderscore{}}}\AgdaSymbol{)}\<%
\\
\>[0]\AgdaKeyword{open}\AgdaSpace{}%
\AgdaKeyword{import}\AgdaSpace{}%
\AgdaModule{Data.Empty}\<%
\\
\>[0]\AgdaKeyword{open}\AgdaSpace{}%
\AgdaKeyword{import}\AgdaSpace{}%
\AgdaModule{Function}\<%
\\
%
\\[\AgdaEmptyExtraSkip]%
\>[0]\AgdaKeyword{module}\AgdaSpace{}%
\AgdaModule{\AgdaUnderscore{}}\AgdaSpace{}%
\AgdaKeyword{where}\<%
\\
\>[0]\AgdaKeyword{module}\AgdaSpace{}%
\AgdaModule{Array}\AgdaSpace{}%
\AgdaKeyword{where}\<%
\\
\>[0][@{}l@{\AgdaIndent{0}}]%
\>[2]\AgdaKeyword{open}\AgdaSpace{}%
\AgdaKeyword{import}\AgdaSpace{}%
\AgdaModule{Data.Nat}\AgdaSpace{}%
\AgdaKeyword{using}\AgdaSpace{}%
\AgdaSymbol{(}\AgdaInductiveConstructor{zero}\AgdaSymbol{;}\AgdaSpace{}%
\AgdaInductiveConstructor{suc}\AgdaSymbol{;}\AgdaSpace{}%
\AgdaDatatype{ℕ}\AgdaSymbol{;}\AgdaSpace{}%
\AgdaOperator{\AgdaPrimitive{\AgdaUnderscore{}+\AgdaUnderscore{}}}\AgdaSymbol{;}\AgdaSpace{}%
\AgdaOperator{\AgdaPrimitive{\AgdaUnderscore{}*\AgdaUnderscore{}}}\AgdaSymbol{;}\AgdaSpace{}%
\AgdaOperator{\AgdaDatatype{\AgdaUnderscore{}≤\AgdaUnderscore{}}}\AgdaSymbol{;}\AgdaSpace{}%
\AgdaInductiveConstructor{s≤s}\AgdaSymbol{;}\AgdaSpace{}%
\AgdaInductiveConstructor{z≤n}\AgdaSymbol{;}\AgdaSpace{}%
\AgdaOperator{\AgdaFunction{\AgdaUnderscore{}<\AgdaUnderscore{}}}\AgdaSymbol{)}\<%
\\
%
\>[2]\AgdaKeyword{open}\AgdaSpace{}%
\AgdaKeyword{import}\AgdaSpace{}%
\AgdaModule{Data.Nat.Properties}\AgdaSpace{}%
\AgdaKeyword{using}\AgdaSpace{}%
\AgdaSymbol{(}\AgdaFunction{+-mono-≤}\AgdaSymbol{;}\AgdaSpace{}%
\AgdaFunction{≤-step}\AgdaSymbol{;}\AgdaSpace{}%
\AgdaFunction{≤-pred}\AgdaSymbol{;}\AgdaSpace{}%
\AgdaOperator{\AgdaFunction{\AgdaUnderscore{}≟\AgdaUnderscore{}}}\AgdaSymbol{;}\AgdaSpace{}%
\AgdaFunction{+-comm}\AgdaSymbol{;}\AgdaSpace{}%
\AgdaFunction{+-suc}\AgdaSymbol{)}\<%
\\
%
\>[2]\AgdaKeyword{open}\AgdaSpace{}%
\AgdaKeyword{import}\AgdaSpace{}%
\AgdaModule{Data.Fin}\AgdaSpace{}%
\AgdaSymbol{as}\AgdaSpace{}%
\AgdaModule{F}\AgdaSpace{}%
\AgdaKeyword{using}\AgdaSpace{}%
\AgdaSymbol{(}\AgdaInductiveConstructor{zero}\AgdaSymbol{;}\AgdaSpace{}%
\AgdaInductiveConstructor{suc}\AgdaSymbol{;}\AgdaSpace{}%
\AgdaDatatype{Fin}\AgdaSymbol{;}\AgdaSpace{}%
\AgdaFunction{combine}\AgdaSymbol{;}\AgdaSpace{}%
\AgdaFunction{remQuot}\AgdaSymbol{;}\AgdaSpace{}%
\AgdaFunction{fromℕ<}\AgdaSymbol{;}\AgdaSpace{}%
\AgdaFunction{inject+}\AgdaSymbol{;}\AgdaSpace{}%
\AgdaFunction{splitAt}\AgdaSymbol{)}\<%
\\
%
\>[2]\AgdaKeyword{open}\AgdaSpace{}%
\AgdaKeyword{import}\AgdaSpace{}%
\AgdaModule{Data.Fin.Properties}\AgdaSpace{}%
\AgdaKeyword{using}\AgdaSpace{}%
\AgdaSymbol{(}\AgdaFunction{suc-injective}\AgdaSymbol{;}\AgdaSpace{}%
\AgdaFunction{toℕ<n}\AgdaSymbol{;}\AgdaSpace{}%
\AgdaFunction{splitAt-inject+}\AgdaSymbol{)}\<%
\\
%
\>[2]\AgdaComment{--open\ import\ Fin2\ using\ (Fin;\ \#\AgdaUnderscore{};\ combine;\ remQuot;\ zerof;\ sucf;\ \AgdaUnderscore{}⊕\AgdaUnderscore{};\ \AgdaUnderscore{}⊝\AgdaUnderscore{})}\<%
\\
%
\>[2]\AgdaKeyword{open}\AgdaSpace{}%
\AgdaKeyword{import}\AgdaSpace{}%
\AgdaModule{Data.Sum}\AgdaSpace{}%
\AgdaKeyword{using}\AgdaSpace{}%
\AgdaSymbol{(}\AgdaOperator{\AgdaDatatype{\AgdaUnderscore{}⊎\AgdaUnderscore{}}}\AgdaSymbol{;}\AgdaSpace{}%
\AgdaInductiveConstructor{inj₁}\AgdaSymbol{;}\AgdaSpace{}%
\AgdaInductiveConstructor{inj₂}\AgdaSymbol{)}\<%
\\
%
\>[2]\AgdaKeyword{open}\AgdaSpace{}%
\AgdaKeyword{import}\AgdaSpace{}%
\AgdaModule{Data.Product}\AgdaSpace{}%
\AgdaSymbol{as}\AgdaSpace{}%
\AgdaModule{Prod}\AgdaSpace{}%
\AgdaKeyword{using}\AgdaSpace{}%
\AgdaSymbol{(}\AgdaFunction{∃}\AgdaSymbol{;}\AgdaSpace{}%
\AgdaOperator{\AgdaInductiveConstructor{\AgdaUnderscore{},\AgdaUnderscore{}}}\AgdaSymbol{;}\AgdaSpace{}%
\AgdaOperator{\AgdaFunction{\AgdaUnderscore{}×\AgdaUnderscore{}}}\AgdaSymbol{;}\AgdaSpace{}%
\AgdaFunction{uncurry}\AgdaSymbol{)}\<%
\end{code}

The central data structure of our case study is a multi-dimensional array (ML
uses the term \emph{tensor}).  This section presents a minimalist array theory in Agda
which is well-suited for specifying numerical applications such as CNNs.

The work in the rest of the paper is presented in Agda, with which we assume some
familiarity.
For gentle introductions to Agda we refer to one of the tutorials that are freely available
online.\footnote{See \url{https://agda.readthedocs.io/en/v2.7.0.1/getting-started/tutorial-list.html}.}

The conciseness of the CNN specification
in~\cite{cnn-array} relies on rank-polymorphism, which is the ability to operate
on arrays of arbitrary ranks.  Our array theory is rank polymorphic
which distinguishes it from most existing approaches.
The central consideration when working with dependent types is how to represent data.
Some encodings are better suited for reasoning, others are more efficient
at runtime.  Due to our two-language setup, our choice of representation is
driven by proof considerations only.
This is why we represent arrays as functions from indices to values.

Absence of out-of-bound errors means that all array indices fall within
the shapes of the arrays that they are selecting from.
The shape of array describes the extent of each of its axes.  We represent
shapes as lists of natural numbers using the data type \AD{S}.
The \AC{[]} shape describes an array of rank zero that contains exactly one
element (arrays of such shape are often called \emph{scalars} and we use this
terminology in the rest of the paper).
The cons operation \AC{\_∷\_} prepends a new axis to the left of the shape.
Note on the notation: underscores in \AC{\_∷\_} specify positions where
arguments go, turning \AC{∷} into an infix binary operation.

Array positions (indices) are given by the dependent type \AD{P} which
is indexed by shapes \AD{S}.  A position within an array of shape \AB{s}
is a list of natural numbers of the same length as $s$ where all elements
are less than the corresponding elements of $s$.

Arrays are given by the type \AF{Ar} \AB{s} \AB{X} where $s$ is a shape of the
array and $X$ is the type of array elements. We allow shapes to be empty, in
which case the array represents a scalar. Formal definitions of \AF{S}, \AF{P}
and \AF{Ar} are as follows:

\begin{mathpar}
\codeblock{\begin{code}%
%
\>[2]\AgdaKeyword{data}\AgdaSpace{}%
\AgdaDatatype{S}\AgdaSpace{}%
\AgdaSymbol{:}\AgdaSpace{}%
\AgdaPrimitive{Set}\AgdaSpace{}%
\AgdaKeyword{where}\<%
\\
\>[2][@{}l@{\AgdaIndent{0}}]%
\>[4]\AgdaInductiveConstructor{[]}%
\>[9]\AgdaSymbol{:}\AgdaSpace{}%
\AgdaDatatype{S}\<%
\\
%
\>[4]\AgdaOperator{\AgdaInductiveConstructor{\AgdaUnderscore{}∷\AgdaUnderscore{}}}%
\>[9]\AgdaSymbol{:}\AgdaSpace{}%
\AgdaDatatype{ℕ}\AgdaSpace{}%
\AgdaSymbol{→}\AgdaSpace{}%
\AgdaDatatype{S}\AgdaSpace{}%
\AgdaSymbol{→}\AgdaSpace{}%
\AgdaDatatype{S}\<%
\end{code}
\begin{code}[hide]%
%
\>[2]\AgdaKeyword{variable}\<%
\\
\>[2][@{}l@{\AgdaIndent{0}}]%
\>[4]\AgdaGeneralizable{m}\AgdaSpace{}%
\AgdaGeneralizable{n}\AgdaSpace{}%
\AgdaGeneralizable{k}\AgdaSpace{}%
\AgdaSymbol{:}\AgdaSpace{}%
\AgdaDatatype{ℕ}\<%
\\
%
\>[4]\AgdaGeneralizable{s}\AgdaSpace{}%
\AgdaGeneralizable{p}\AgdaSpace{}%
\AgdaGeneralizable{q}\AgdaSpace{}%
\AgdaGeneralizable{r}\AgdaSpace{}%
\AgdaGeneralizable{u}\AgdaSpace{}%
\AgdaGeneralizable{w}\AgdaSpace{}%
\AgdaSymbol{:}\AgdaSpace{}%
\AgdaDatatype{S}\<%
\\
%
\>[4]\AgdaGeneralizable{X}\AgdaSpace{}%
\AgdaGeneralizable{Y}\AgdaSpace{}%
\AgdaGeneralizable{Z}\AgdaSpace{}%
\AgdaSymbol{:}\AgdaSpace{}%
\AgdaPrimitive{Set}\<%
\end{code}}
\and
\codeblock{\begin{code}%
%
\>[2]\AgdaKeyword{data}\AgdaSpace{}%
\AgdaDatatype{P}\AgdaSpace{}%
\AgdaSymbol{:}\AgdaSpace{}%
\AgdaDatatype{S}\AgdaSpace{}%
\AgdaSymbol{→}\AgdaSpace{}%
\AgdaPrimitive{Set}\AgdaSpace{}%
\AgdaKeyword{where}\<%
\\
\>[2][@{}l@{\AgdaIndent{0}}]%
\>[4]\AgdaInductiveConstructor{[]}%
\>[9]\AgdaSymbol{:}\AgdaSpace{}%
\AgdaDatatype{P}\AgdaSpace{}%
\AgdaInductiveConstructor{[]}\<%
\\
%
\>[4]\AgdaOperator{\AgdaInductiveConstructor{\AgdaUnderscore{}∷\AgdaUnderscore{}}}%
\>[9]\AgdaSymbol{:}\AgdaSpace{}%
\AgdaDatatype{Fin}\AgdaSpace{}%
\AgdaGeneralizable{n}\AgdaSpace{}%
\AgdaSymbol{→}\AgdaSpace{}%
\AgdaDatatype{P}\AgdaSpace{}%
\AgdaGeneralizable{s}\AgdaSpace{}%
\AgdaSymbol{→}\AgdaSpace{}%
\AgdaDatatype{P}\AgdaSpace{}%
\AgdaSymbol{(}\AgdaGeneralizable{n}\AgdaSpace{}%
\AgdaOperator{\AgdaInductiveConstructor{∷}}\AgdaSpace{}%
\AgdaGeneralizable{s}\AgdaSymbol{)}\<%
\end{code}}
\and
\codeblock{\begin{code}%
%
\>[2]\AgdaFunction{Ar}\AgdaSpace{}%
\AgdaSymbol{:}\AgdaSpace{}%
\AgdaDatatype{S}\AgdaSpace{}%
\AgdaSymbol{→}\AgdaSpace{}%
\AgdaPrimitive{Set}\AgdaSpace{}%
\AgdaSymbol{→}\AgdaSpace{}%
\AgdaPrimitive{Set}\<%
\\
%
\>[2]\AgdaFunction{Ar}\AgdaSpace{}%
\AgdaBound{s}\AgdaSpace{}%
\AgdaBound{X}\AgdaSpace{}%
\AgdaSymbol{=}\AgdaSpace{}%
\AgdaDatatype{P}\AgdaSpace{}%
\AgdaBound{s}\AgdaSpace{}%
\AgdaSymbol{→}\AgdaSpace{}%
\AgdaBound{X}\<%
\end{code}}
\end{mathpar}
The type \AF{Fin} $n$ represents natural numbers bounded by $n$.
As arrays are functions, selections are function applications and
the array constructor is a function definition (\eg{} via $\lambda$-abstraction).

\paragraph{Array Combinators} It is helpful to invest a little time
in defining array combinators.  First, we can observe that \AD{Ar} of
a fixed shape is an applicative functor~\cite{applicative}, so we can trivially derive:
\AF{K}\ \AB{x} to produce a constant array; \AF{map}\ \AB{f}\ \AB{a}
to apply \AB{f} to all the elements of \AB{a}; and \AF{zipWith}\ \AB{f}
\ \AB{a}\ \AB{b} to point-wise apply the binary operation 
\AB{f} to \AB{a} and \AB{b}.
\begin{mathpar}
\codeblock{\begin{code}%
%
\>[2]\AgdaFunction{K}\AgdaSpace{}%
\AgdaSymbol{:}\AgdaSpace{}%
\AgdaGeneralizable{X}\AgdaSpace{}%
\AgdaSymbol{→}\AgdaSpace{}%
\AgdaFunction{Ar}\AgdaSpace{}%
\AgdaGeneralizable{s}\AgdaSpace{}%
\AgdaGeneralizable{X}\<%
\\
%
\>[2]\AgdaFunction{K}\AgdaSpace{}%
\AgdaBound{x}\AgdaSpace{}%
\AgdaBound{i}\AgdaSpace{}%
\AgdaSymbol{=}\AgdaSpace{}%
\AgdaBound{x}\<%
\end{code}}
\and
\codeblock{\begin{code}%
%
\>[2]\AgdaFunction{map}\AgdaSpace{}%
\AgdaSymbol{:}\AgdaSpace{}%
\AgdaSymbol{(}\AgdaGeneralizable{X}\AgdaSpace{}%
\AgdaSymbol{→}\AgdaSpace{}%
\AgdaGeneralizable{Y}\AgdaSymbol{)}\AgdaSpace{}%
\AgdaSymbol{→}\AgdaSpace{}%
\AgdaFunction{Ar}\AgdaSpace{}%
\AgdaGeneralizable{s}\AgdaSpace{}%
\AgdaGeneralizable{X}\AgdaSpace{}%
\AgdaSymbol{→}\AgdaSpace{}%
\AgdaFunction{Ar}\AgdaSpace{}%
\AgdaGeneralizable{s}\AgdaSpace{}%
\AgdaGeneralizable{Y}\<%
\\
%
\>[2]\AgdaFunction{map}\AgdaSpace{}%
\AgdaBound{f}\AgdaSpace{}%
\AgdaBound{a}\AgdaSpace{}%
\AgdaBound{i}\AgdaSpace{}%
\AgdaSymbol{=}\AgdaSpace{}%
\AgdaBound{f}\AgdaSpace{}%
\AgdaSymbol{(}\AgdaBound{a}\AgdaSpace{}%
\AgdaBound{i}\AgdaSymbol{)}\<%
\end{code}}
\and
\codeblock{\begin{code}%
%
\>[2]\AgdaFunction{zipWith}\AgdaSpace{}%
\AgdaSymbol{:}\AgdaSpace{}%
\AgdaSymbol{(}\AgdaGeneralizable{X}\AgdaSpace{}%
\AgdaSymbol{→}\AgdaSpace{}%
\AgdaGeneralizable{Y}\AgdaSpace{}%
\AgdaSymbol{→}\AgdaSpace{}%
\AgdaGeneralizable{Z}\AgdaSymbol{)}\AgdaSpace{}%
\AgdaSymbol{→}\AgdaSpace{}%
\AgdaFunction{Ar}\AgdaSpace{}%
\AgdaGeneralizable{s}\AgdaSpace{}%
\AgdaGeneralizable{X}\AgdaSpace{}%
\AgdaSymbol{→}\AgdaSpace{}%
\AgdaFunction{Ar}\AgdaSpace{}%
\AgdaGeneralizable{s}\AgdaSpace{}%
\AgdaGeneralizable{Y}\AgdaSpace{}%
\AgdaSymbol{→}\AgdaSpace{}%
\AgdaFunction{Ar}\AgdaSpace{}%
\AgdaGeneralizable{s}\AgdaSpace{}%
\AgdaGeneralizable{Z}\<%
\\
%
\>[2]\AgdaFunction{zipWith}\AgdaSpace{}%
\AgdaBound{f}\AgdaSpace{}%
\AgdaBound{a}\AgdaSpace{}%
\AgdaBound{b}\AgdaSpace{}%
\AgdaBound{i}\AgdaSpace{}%
\AgdaSymbol{=}\AgdaSpace{}%
\AgdaBound{f}\AgdaSpace{}%
\AgdaSymbol{(}\AgdaBound{a}\AgdaSpace{}%
\AgdaBound{i}\AgdaSymbol{)}\AgdaSpace{}%
\AgdaSymbol{(}\AgdaBound{b}\AgdaSpace{}%
\AgdaBound{i}\AgdaSymbol{)}\<%
\end{code}}
\end{mathpar}

Array shapes can be concatenated as lists.  We call this operation
\emph{shape product} and we denote it with \AF{\_⊗\_} (because this
corresponds to the shape of tensor product).  Positions of sub-shapes
can be joined into a position of a product shape using the \AF{\_⊗ₚ\_}
operation.  Dually, positions of a product shape can be split into
positions of the corresponding subshapes using \AF{split}.  The types
of these three operations are as follows.
\begin{mathpar}
\codeblock{\begin{code}%
%
\>[2]\AgdaOperator{\AgdaFunction{\AgdaUnderscore{}⊗\AgdaUnderscore{}}}\AgdaSpace{}%
\AgdaSymbol{:}\AgdaSpace{}%
\AgdaDatatype{S}\AgdaSpace{}%
\AgdaSymbol{→}\AgdaSpace{}%
\AgdaDatatype{S}\AgdaSpace{}%
\AgdaSymbol{→}\AgdaSpace{}%
\AgdaDatatype{S}\<%
\end{code}}
\and
\codeblock{\begin{code}%
%
\>[2]\AgdaOperator{\AgdaFunction{\AgdaUnderscore{}⊗ₚ\AgdaUnderscore{}}}\AgdaSpace{}%
\AgdaSymbol{:}\AgdaSpace{}%
\AgdaDatatype{P}\AgdaSpace{}%
\AgdaGeneralizable{s}\AgdaSpace{}%
\AgdaSymbol{→}\AgdaSpace{}%
\AgdaDatatype{P}\AgdaSpace{}%
\AgdaGeneralizable{p}\AgdaSpace{}%
\AgdaSymbol{→}\AgdaSpace{}%
\AgdaDatatype{P}\AgdaSpace{}%
\AgdaSymbol{(}\AgdaGeneralizable{s}\AgdaSpace{}%
\AgdaOperator{\AgdaFunction{⊗}}\AgdaSpace{}%
\AgdaGeneralizable{p}\AgdaSymbol{)}\<%
\end{code}}
\and
\codeblock{\begin{code}%
%
\>[2]\AgdaFunction{split}\AgdaSpace{}%
\AgdaSymbol{:}\AgdaSpace{}%
\AgdaDatatype{P}\AgdaSpace{}%
\AgdaSymbol{(}\AgdaGeneralizable{s}\AgdaSpace{}%
\AgdaOperator{\AgdaFunction{⊗}}\AgdaSpace{}%
\AgdaGeneralizable{p}\AgdaSymbol{)}\AgdaSpace{}%
\AgdaSymbol{→}\AgdaSpace{}%
\AgdaDatatype{P}\AgdaSpace{}%
\AgdaGeneralizable{s}\AgdaSpace{}%
\AgdaOperator{\AgdaFunction{×}}\AgdaSpace{}%
\AgdaDatatype{P}\AgdaSpace{}%
\AgdaGeneralizable{p}\<%
\end{code}}
\end{mathpar}
\begin{code}[hide]%
%
\>[2]\AgdaInductiveConstructor{[]}\AgdaSpace{}%
\AgdaOperator{\AgdaFunction{⊗}}\AgdaSpace{}%
\AgdaBound{p}\AgdaSpace{}%
\AgdaSymbol{=}\AgdaSpace{}%
\AgdaBound{p}\<%
\\
%
\>[2]\AgdaSymbol{(}\AgdaBound{n}\AgdaSpace{}%
\AgdaOperator{\AgdaInductiveConstructor{∷}}\AgdaSpace{}%
\AgdaBound{s}\AgdaSymbol{)}\AgdaSpace{}%
\AgdaOperator{\AgdaFunction{⊗}}\AgdaSpace{}%
\AgdaBound{p}\AgdaSpace{}%
\AgdaSymbol{=}\AgdaSpace{}%
\AgdaBound{n}\AgdaSpace{}%
\AgdaOperator{\AgdaInductiveConstructor{∷}}\AgdaSpace{}%
\AgdaSymbol{(}\AgdaBound{s}\AgdaSpace{}%
\AgdaOperator{\AgdaFunction{⊗}}\AgdaSpace{}%
\AgdaBound{p}\AgdaSymbol{)}\<%
\\
%
\\[\AgdaEmptyExtraSkip]%
%
\>[2]\AgdaInductiveConstructor{[]}\AgdaSpace{}%
\AgdaOperator{\AgdaFunction{⊗ₚ}}\AgdaSpace{}%
\AgdaBound{jv}\AgdaSpace{}%
\AgdaSymbol{=}\AgdaSpace{}%
\AgdaBound{jv}\<%
\\
%
\>[2]\AgdaSymbol{(}\AgdaBound{i}\AgdaSpace{}%
\AgdaOperator{\AgdaInductiveConstructor{∷}}\AgdaSpace{}%
\AgdaBound{iv}\AgdaSymbol{)}\AgdaSpace{}%
\AgdaOperator{\AgdaFunction{⊗ₚ}}\AgdaSpace{}%
\AgdaBound{jv}\AgdaSpace{}%
\AgdaSymbol{=}\AgdaSpace{}%
\AgdaBound{i}\AgdaSpace{}%
\AgdaOperator{\AgdaInductiveConstructor{∷}}\AgdaSpace{}%
\AgdaSymbol{(}\AgdaBound{iv}\AgdaSpace{}%
\AgdaOperator{\AgdaFunction{⊗ₚ}}\AgdaSpace{}%
\AgdaBound{jv}\AgdaSymbol{)}\<%
\\
%
\\[\AgdaEmptyExtraSkip]%
%
\>[2]\AgdaFunction{split}\AgdaSpace{}%
\AgdaSymbol{\{}\AgdaArgument{s}\AgdaSpace{}%
\AgdaSymbol{=}\AgdaSpace{}%
\AgdaInductiveConstructor{[]}\AgdaSymbol{\}}%
\>[20]\AgdaBound{is}\AgdaSpace{}%
\AgdaSymbol{=}\AgdaSpace{}%
\AgdaInductiveConstructor{[]}\AgdaSpace{}%
\AgdaOperator{\AgdaInductiveConstructor{,}}\AgdaSpace{}%
\AgdaBound{is}\<%
\\
%
\>[2]\AgdaFunction{split}\AgdaSpace{}%
\AgdaSymbol{\{}\AgdaArgument{s}\AgdaSpace{}%
\AgdaSymbol{=}\AgdaSpace{}%
\AgdaBound{x}\AgdaSpace{}%
\AgdaOperator{\AgdaInductiveConstructor{∷}}\AgdaSpace{}%
\AgdaBound{s}\AgdaSymbol{\}}\AgdaSpace{}%
\AgdaSymbol{(}\AgdaBound{i}\AgdaSpace{}%
\AgdaOperator{\AgdaInductiveConstructor{∷}}\AgdaSpace{}%
\AgdaBound{is}\AgdaSymbol{)}\AgdaSpace{}%
\AgdaSymbol{=}\AgdaSpace{}%
\AgdaFunction{Prod.map₁}\AgdaSpace{}%
\AgdaSymbol{(}\AgdaBound{i}\AgdaSpace{}%
\AgdaOperator{\AgdaInductiveConstructor{∷\AgdaUnderscore{}}}\AgdaSymbol{)}\AgdaSpace{}%
\AgdaSymbol{(}\AgdaFunction{split}\AgdaSpace{}%
\AgdaBound{is}\AgdaSymbol{)}\<%
\\
%
\\[\AgdaEmptyExtraSkip]%
%
\>[2]\AgdaOperator{\AgdaFunction{\AgdaUnderscore{}≟ₚ\AgdaUnderscore{}}}\AgdaSpace{}%
\AgdaSymbol{:}\AgdaSpace{}%
\AgdaSymbol{(}\AgdaBound{i}\AgdaSpace{}%
\AgdaBound{j}\AgdaSpace{}%
\AgdaSymbol{:}\AgdaSpace{}%
\AgdaDatatype{P}\AgdaSpace{}%
\AgdaGeneralizable{s}\AgdaSymbol{)}\AgdaSpace{}%
\AgdaSymbol{→}\AgdaSpace{}%
\AgdaRecord{Dec}\AgdaSpace{}%
\AgdaSymbol{(}\AgdaBound{i}\AgdaSpace{}%
\AgdaOperator{\AgdaDatatype{≡}}\AgdaSpace{}%
\AgdaBound{j}\AgdaSymbol{)}\<%
\\
%
\>[2]\AgdaOperator{\AgdaFunction{\AgdaUnderscore{}≟ₚ\AgdaUnderscore{}}}\AgdaSpace{}%
\AgdaSymbol{\{}\AgdaInductiveConstructor{[]}\AgdaSymbol{\}}\AgdaSpace{}%
\AgdaInductiveConstructor{[]}\AgdaSpace{}%
\AgdaInductiveConstructor{[]}\AgdaSpace{}%
\AgdaSymbol{=}\AgdaSpace{}%
\AgdaInductiveConstructor{yes}\AgdaSpace{}%
\AgdaInductiveConstructor{refl}\<%
\\
%
\>[2]\AgdaOperator{\AgdaFunction{\AgdaUnderscore{}≟ₚ\AgdaUnderscore{}}}\AgdaSpace{}%
\AgdaSymbol{\{}\AgdaBound{x}\AgdaSpace{}%
\AgdaOperator{\AgdaInductiveConstructor{∷}}\AgdaSpace{}%
\AgdaBound{s}\AgdaSymbol{\}}\AgdaSpace{}%
\AgdaSymbol{(}\AgdaBound{i}\AgdaSpace{}%
\AgdaOperator{\AgdaInductiveConstructor{∷}}\AgdaSpace{}%
\AgdaBound{is}\AgdaSymbol{)}\AgdaSpace{}%
\AgdaSymbol{(}\AgdaBound{j}\AgdaSpace{}%
\AgdaOperator{\AgdaInductiveConstructor{∷}}\AgdaSpace{}%
\AgdaBound{js}\AgdaSymbol{)}\AgdaSpace{}%
\AgdaKeyword{with}\AgdaSpace{}%
\AgdaBound{i}\AgdaSpace{}%
\AgdaOperator{\AgdaFunction{F.≟}}\AgdaSpace{}%
\AgdaBound{j}\<%
\\
%
\>[2]\AgdaSymbol{...}\AgdaSpace{}%
\AgdaSymbol{|}\AgdaSpace{}%
\AgdaInductiveConstructor{no}\AgdaSpace{}%
\AgdaBound{¬p}\AgdaSpace{}%
\AgdaSymbol{=}\AgdaSpace{}%
\AgdaInductiveConstructor{no}\AgdaSpace{}%
\AgdaSymbol{λ}\AgdaSpace{}%
\AgdaSymbol{\{}\AgdaSpace{}%
\AgdaInductiveConstructor{refl}\AgdaSpace{}%
\AgdaSymbol{→}\AgdaSpace{}%
\AgdaBound{¬p}\AgdaSpace{}%
\AgdaInductiveConstructor{refl}\AgdaSpace{}%
\AgdaSymbol{\}}\<%
\\
%
\>[2]\AgdaSymbol{...}\AgdaSpace{}%
\AgdaSymbol{|}\AgdaSpace{}%
\AgdaInductiveConstructor{yes}\AgdaSpace{}%
\AgdaInductiveConstructor{refl}\AgdaSpace{}%
\AgdaKeyword{with}\AgdaSpace{}%
\AgdaBound{is}\AgdaSpace{}%
\AgdaOperator{\AgdaFunction{≟ₚ}}\AgdaSpace{}%
\AgdaBound{js}\<%
\\
%
\>[2]\AgdaSymbol{...}\AgdaSpace{}%
\AgdaSymbol{|}\AgdaSpace{}%
\AgdaInductiveConstructor{no}\AgdaSpace{}%
\AgdaBound{¬q}\AgdaSpace{}%
\AgdaSymbol{=}\AgdaSpace{}%
\AgdaInductiveConstructor{no}\AgdaSpace{}%
\AgdaSymbol{λ}\AgdaSpace{}%
\AgdaSymbol{\{}\AgdaSpace{}%
\AgdaInductiveConstructor{refl}\AgdaSpace{}%
\AgdaSymbol{→}\AgdaSpace{}%
\AgdaBound{¬q}\AgdaSpace{}%
\AgdaInductiveConstructor{refl}\AgdaSpace{}%
\AgdaSymbol{\}}\<%
\\
%
\>[2]\AgdaSymbol{...}\AgdaSpace{}%
\AgdaSymbol{|}\AgdaSpace{}%
\AgdaInductiveConstructor{yes}\AgdaSpace{}%
\AgdaInductiveConstructor{refl}\AgdaSpace{}%
\AgdaSymbol{=}\AgdaSpace{}%
\AgdaInductiveConstructor{yes}\AgdaSpace{}%
\AgdaInductiveConstructor{refl}\<%
\end{code}

Arrays are homogeneously nested, \ie{} the shapes of all the sub-arrays
have to be the same.  Therefore, we can switch between the array of a product
shape and the nested array (array of arrays).  This operation is very similar
to currying except it happens at the level of shapes.  The combinators that
achieve this are named \AF{nest} and \AF{unnest} and their definitions are:
\begin{mathpar}
\codeblock{\begin{code}%
%
\>[2]\AgdaFunction{nest}\AgdaSpace{}%
\AgdaSymbol{:}\AgdaSpace{}%
\AgdaFunction{Ar}\AgdaSpace{}%
\AgdaSymbol{(}\AgdaGeneralizable{s}\AgdaSpace{}%
\AgdaOperator{\AgdaFunction{⊗}}\AgdaSpace{}%
\AgdaGeneralizable{p}\AgdaSymbol{)}\AgdaSpace{}%
\AgdaGeneralizable{X}\AgdaSpace{}%
\AgdaSymbol{→}\AgdaSpace{}%
\AgdaFunction{Ar}\AgdaSpace{}%
\AgdaGeneralizable{s}\AgdaSpace{}%
\AgdaSymbol{(}\AgdaFunction{Ar}\AgdaSpace{}%
\AgdaGeneralizable{p}\AgdaSpace{}%
\AgdaGeneralizable{X}\AgdaSymbol{)}\<%
\\
%
\>[2]\AgdaFunction{nest}\AgdaSpace{}%
\AgdaBound{a}\AgdaSpace{}%
\AgdaBound{i}\AgdaSpace{}%
\AgdaBound{j}\AgdaSpace{}%
\AgdaSymbol{=}\AgdaSpace{}%
\AgdaBound{a}\AgdaSpace{}%
\AgdaSymbol{(}\AgdaBound{i}\AgdaSpace{}%
\AgdaOperator{\AgdaFunction{⊗ₚ}}\AgdaSpace{}%
\AgdaBound{j}\AgdaSymbol{)}\<%
\end{code}}
\and
\codeblock{\begin{code} %
%
\>[2]\AgdaFunction{unnest}\AgdaSpace{}%
\AgdaSymbol{:}\AgdaSpace{}%
\AgdaFunction{Ar}\AgdaSpace{}%
\AgdaGeneralizable{s}\AgdaSpace{}%
\AgdaSymbol{(}\AgdaFunction{Ar}\AgdaSpace{}%
\AgdaGeneralizable{p}\AgdaSpace{}%
\AgdaGeneralizable{X}\AgdaSymbol{)}\AgdaSpace{}%
\AgdaSymbol{→}\AgdaSpace{}%
\AgdaFunction{Ar}\AgdaSpace{}%
\AgdaSymbol{(}\AgdaGeneralizable{s}\AgdaSpace{}%
\AgdaOperator{\AgdaFunction{⊗}}\AgdaSpace{}%
\AgdaGeneralizable{p}\AgdaSymbol{)}\AgdaSpace{}%
\AgdaGeneralizable{X}\<%
\\
%
\>[2]\AgdaFunction{unnest}\AgdaSpace{}%
\AgdaBound{a}\AgdaSpace{}%
\AgdaBound{i}\AgdaSpace{}%
\AgdaSymbol{=}\AgdaSpace{}%
\AgdaFunction{uncurry}\AgdaSpace{}%
\AgdaBound{a}\AgdaSpace{}%
\AgdaSymbol{(}\AgdaFunction{split}\AgdaSpace{}%
\AgdaBound{i}\AgdaSymbol{)}\<%
\end{code}}
\end{mathpar}


\paragraph{Reduction} We implement reduction of the binary operations
over arrays in two steps.  Firstly, we define 1-d reductions  that
we call \AD{sum₁} which is similar to right fold on lists.
Arrays of higher ranks iterate \AF{sum₁} bottom-up.  The definition
of the primitives are as follows:
\begin{mathpar}
\codeblock{\begin{code}%
%
\>[2]\AgdaKeyword{pattern}\AgdaSpace{}%
\AgdaInductiveConstructor{ι}\AgdaSpace{}%
\AgdaBound{n}\AgdaSpace{}%
\AgdaSymbol{=}\AgdaSpace{}%
\AgdaBound{n}\AgdaSpace{}%
\AgdaOperator{\AgdaInductiveConstructor{∷}}\AgdaSpace{}%
\AgdaInductiveConstructor{[]}\<%
\\
%
\\[\AgdaEmptyExtraSkip]%
%
\>[2]\AgdaFunction{ιsuc}\AgdaSpace{}%
\AgdaSymbol{:}\AgdaSpace{}%
\AgdaDatatype{P}\AgdaSpace{}%
\AgdaSymbol{(}\AgdaInductiveConstructor{ι}\AgdaSpace{}%
\AgdaGeneralizable{n}\AgdaSymbol{)}\AgdaSpace{}%
\AgdaSymbol{→}\AgdaSpace{}%
\AgdaDatatype{P}\AgdaSpace{}%
\AgdaSymbol{(}\AgdaInductiveConstructor{ι}\AgdaSpace{}%
\AgdaSymbol{(}\AgdaInductiveConstructor{suc}\AgdaSpace{}%
\AgdaGeneralizable{n}\AgdaSymbol{))}\<%
\\
%
\>[2]\AgdaFunction{ιsuc}\AgdaSpace{}%
\AgdaSymbol{(}\AgdaInductiveConstructor{ι}\AgdaSpace{}%
\AgdaBound{i}\AgdaSymbol{)}\AgdaSpace{}%
\AgdaSymbol{=}\AgdaSpace{}%
\AgdaInductiveConstructor{ι}\AgdaSpace{}%
\AgdaSymbol{(}\AgdaInductiveConstructor{suc}\AgdaSpace{}%
\AgdaBound{i}\AgdaSymbol{)}\<%
\end{code}}
\and
\codeblock{\begin{code}%
%
\>[2]\AgdaFunction{sum₁}\AgdaSpace{}%
\AgdaSymbol{:}\AgdaSpace{}%
\AgdaSymbol{(}\AgdaGeneralizable{X}\AgdaSpace{}%
\AgdaSymbol{→}\AgdaSpace{}%
\AgdaGeneralizable{X}\AgdaSpace{}%
\AgdaSymbol{→}\AgdaSpace{}%
\AgdaGeneralizable{X}\AgdaSymbol{)}\AgdaSpace{}%
\AgdaSymbol{→}\AgdaSpace{}%
\AgdaGeneralizable{X}\AgdaSpace{}%
\AgdaSymbol{→}\AgdaSpace{}%
\AgdaFunction{Ar}\AgdaSpace{}%
\AgdaSymbol{(}\AgdaInductiveConstructor{ι}\AgdaSpace{}%
\AgdaGeneralizable{n}\AgdaSymbol{)}\AgdaSpace{}%
\AgdaGeneralizable{X}\AgdaSpace{}%
\AgdaSymbol{→}\AgdaSpace{}%
\AgdaGeneralizable{X}\<%
\\
%
\>[2]\AgdaFunction{sum₁}\AgdaSpace{}%
\AgdaSymbol{\{}\AgdaArgument{n}\AgdaSpace{}%
\AgdaSymbol{=}\AgdaSpace{}%
\AgdaInductiveConstructor{zero}\AgdaSymbol{\}}%
\>[20]\AgdaBound{f}\AgdaSpace{}%
\AgdaBound{ε}\AgdaSpace{}%
\AgdaBound{a}\AgdaSpace{}%
\AgdaSymbol{=}\AgdaSpace{}%
\AgdaBound{ε}\<%
\\
%
\>[2]\AgdaFunction{sum₁}\AgdaSpace{}%
\AgdaSymbol{\{}\AgdaArgument{n}\AgdaSpace{}%
\AgdaSymbol{=}\AgdaSpace{}%
\AgdaInductiveConstructor{suc}\AgdaSpace{}%
\AgdaBound{n}\AgdaSymbol{\}}%
\>[20]\AgdaBound{f}\AgdaSpace{}%
\AgdaBound{ε}\AgdaSpace{}%
\AgdaBound{a}\AgdaSpace{}%
\AgdaSymbol{=}\AgdaSpace{}%
\AgdaBound{f}\AgdaSpace{}%
\AgdaSymbol{(}\AgdaBound{a}\AgdaSpace{}%
\AgdaSymbol{(}\AgdaInductiveConstructor{ι}\AgdaSpace{}%
\AgdaInductiveConstructor{zero}\AgdaSymbol{))}\AgdaSpace{}%
\AgdaSymbol{(}\AgdaFunction{sum₁}\AgdaSpace{}%
\AgdaBound{f}\AgdaSpace{}%
\AgdaBound{ε}\AgdaSpace{}%
\AgdaSymbol{(}\AgdaBound{a}\AgdaSpace{}%
\AgdaOperator{\AgdaFunction{∘}}\AgdaSpace{}%
\AgdaFunction{ιsuc}\AgdaSymbol{))}\<%
\end{code}}
\and
\codeblock{\begin{code}%
%
\>[2]\AgdaFunction{sum}\AgdaSpace{}%
\AgdaSymbol{:}\AgdaSpace{}%
\AgdaSymbol{(}\AgdaGeneralizable{X}\AgdaSpace{}%
\AgdaSymbol{→}\AgdaSpace{}%
\AgdaGeneralizable{X}\AgdaSpace{}%
\AgdaSymbol{→}\AgdaSpace{}%
\AgdaGeneralizable{X}\AgdaSymbol{)}\AgdaSpace{}%
\AgdaSymbol{→}\AgdaSpace{}%
\AgdaGeneralizable{X}\AgdaSpace{}%
\AgdaSymbol{→}\AgdaSpace{}%
\AgdaFunction{Ar}\AgdaSpace{}%
\AgdaGeneralizable{s}\AgdaSpace{}%
\AgdaGeneralizable{X}\AgdaSpace{}%
\AgdaSymbol{→}\AgdaSpace{}%
\AgdaGeneralizable{X}\<%
\\
%
\>[2]\AgdaFunction{sum}\AgdaSpace{}%
\AgdaSymbol{\{}\AgdaArgument{s}\AgdaSpace{}%
\AgdaSymbol{=}\AgdaSpace{}%
\AgdaInductiveConstructor{[]}\AgdaSymbol{\}}%
\>[19]\AgdaBound{f}\AgdaSpace{}%
\AgdaBound{ε}\AgdaSpace{}%
\AgdaBound{a}\AgdaSpace{}%
\AgdaSymbol{=}\AgdaSpace{}%
\AgdaBound{f}\AgdaSpace{}%
\AgdaBound{ε}\AgdaSpace{}%
\AgdaSymbol{(}\AgdaBound{a}\AgdaSpace{}%
\AgdaInductiveConstructor{[]}\AgdaSymbol{)}\<%
\\
%
\>[2]\AgdaFunction{sum}\AgdaSpace{}%
\AgdaSymbol{\{}\AgdaArgument{s}\AgdaSpace{}%
\AgdaSymbol{=}\AgdaSpace{}%
\AgdaBound{x}\AgdaSpace{}%
\AgdaOperator{\AgdaInductiveConstructor{∷}}\AgdaSpace{}%
\AgdaBound{s}\AgdaSymbol{\}}%
\>[19]\AgdaBound{f}\AgdaSpace{}%
\AgdaBound{ε}\AgdaSpace{}%
\AgdaBound{a}\AgdaSpace{}%
\AgdaSymbol{=}\AgdaSpace{}%
\AgdaFunction{sum₁}\AgdaSpace{}%
\AgdaBound{f}\AgdaSpace{}%
\AgdaBound{ε}\AgdaSpace{}%
\AgdaOperator{\AgdaFunction{\$}}\AgdaSpace{}%
\AgdaFunction{map}\AgdaSpace{}%
\AgdaSymbol{(}\AgdaFunction{sum}\AgdaSpace{}%
\AgdaBound{f}\AgdaSpace{}%
\AgdaBound{ε}\AgdaSymbol{)}\AgdaSpace{}%
\AgdaSymbol{(}\AgdaFunction{nest}\AgdaSpace{}%
\AgdaBound{a}\AgdaSymbol{)}\<%
\end{code}}
\end{mathpar}

Note that our reduction forces the types of the arguments of the binary
operation to be the same, which is different from usual definitions of foldr.
While this generality is not required for our example,
it is worth noting that the standard behaviour can be recovered\footnote{
We recover regular fold behaviour by running \AD{sum} over function composition:
\begin{code}%
%
\>[2]\AgdaFunction{sum′}\AgdaSpace{}%
\AgdaSymbol{:}\AgdaSpace{}%
\AgdaSymbol{(}\AgdaGeneralizable{X}\AgdaSpace{}%
\AgdaSymbol{→}\AgdaSpace{}%
\AgdaGeneralizable{Y}\AgdaSpace{}%
\AgdaSymbol{→}\AgdaSpace{}%
\AgdaGeneralizable{Y}\AgdaSymbol{)}\AgdaSpace{}%
\AgdaSymbol{→}\AgdaSpace{}%
\AgdaGeneralizable{Y}\AgdaSpace{}%
\AgdaSymbol{→}\AgdaSpace{}%
\AgdaFunction{Ar}\AgdaSpace{}%
\AgdaGeneralizable{s}\AgdaSpace{}%
\AgdaGeneralizable{X}\AgdaSpace{}%
\AgdaSymbol{→}\AgdaSpace{}%
\AgdaGeneralizable{Y}\<%
\\
%
\>[2]\AgdaFunction{sum′}\AgdaSpace{}%
\AgdaBound{f}\AgdaSpace{}%
\AgdaBound{ε}\AgdaSpace{}%
\AgdaBound{a}\AgdaSpace{}%
\AgdaSymbol{=}\AgdaSpace{}%
\AgdaFunction{sum}\AgdaSpace{}%
\AgdaOperator{\AgdaFunction{\AgdaUnderscore{}∘′\AgdaUnderscore{}}}\AgdaSpace{}%
\AgdaFunction{id}\AgdaSpace{}%
\AgdaSymbol{(}\AgdaFunction{map}\AgdaSpace{}%
\AgdaBound{f}\AgdaSpace{}%
\AgdaBound{a}\AgdaSymbol{)}\AgdaSpace{}%
\AgdaBound{ε}\<%
\end{code}
} through reduction of function composition.

% \paragraph{Reshaping}
% One common operation on arrays is element-preserving change of shape.  We call
% such an operation \AF{reshape}.  It is clear that array elements can be preserved only in
% cases when the number of elements in the original array and the reshaped one
% is the same.  We define an inductive relation \AF{Reshape} that relates
% only those shapes that preserve the number of array elements.  
% \begin{code}[hide]
%   infixr 5 _∙_
%   --infixl 10 _,_
% \end{code}
% \begin{mathpar}
% \codeblock{\begin{code}
%   data Reshape : S → S → Set where
%     eq      : Reshape s s
%     _∙_     : Reshape p q → Reshape s p → Reshape s q
%     _,_     : Reshape s p → Reshape q r → Reshape (s ⊗ q) (p ⊗ r)
%     split   : Reshape (ι (m * n)) (ι m ⊗ ι n)
%     flat    : Reshape (ι m ⊗ ι n) (ι (m * n))
%     swap    : Reshape (s ⊗ p) (p ⊗ s)
%     assocl  : Reshape (s ⊗ (p ⊗ q)) ((s ⊗ p) ⊗ q)
%     assocr  : Reshape ((s ⊗ p) ⊗ q) (s ⊗ (p ⊗ q))
% \end{code}}
% \end{mathpar}
% Any expression $r$ of
% the type (\AF{Reshape} \AB{s} \AB{p}) comes with a straight-forward action on
% indices that we denote \AF{\_⟨\_⟩} (its definition is omitted).
% Such a (contravariant) action translates
% the index within the shape \AB{p} into the index within the shape \AB{s}.
% Given this translation, we can easily define \AF{reshape} as shown below.
% \AF{Reshape} is constructed such that if $s$ and $p$ are related, then 
% $p$ and $s$ are related too.  This fact is given by the function \AF{rev}
% (its definition is omitted) and it immediately implies that all the
% actions on indices as well as array \AF{reshape}s are invertible.
% 
% Note that two shapes can be related by \AF{Reshape} in more than
% one way, which results in different array reshapes.  
% For example, consider \AF{Reshape} (\AC{ι} 5 \AC{⊗} \AC{ι} 4) (\AC{ι} 5 \AC{⊗} \AC{ι} 4)
% given by \AC{swap} or through (\AC{split} \AC{∙} \AC{flat}).  While the former transposes 
% the array elements, the latter does not.
% \begin{mathpar}
% \codeblock{\begin{code}
%   _⟨_⟩ : P p → Reshape s p → P s
% \end{code}}
% \and
% \codeblock{\begin{code}
%   reshape : Reshape s p → Ar s X → Ar p X
%   reshape r a = λ ix → a (ix ⟨ r ⟩)
% \end{code}}
% \and
% \codeblock{\begin{code}
%   rev : Reshape s p → Reshape p s
% \end{code}}
% \end{mathpar}
% From the perspective of category theory, if \AF{S} is an object then \AF{Reshape}
% is a Hom set, where \AC{eq} is identity and \AC{\_∙\_} is a composition with
% the expected properties.  In the language of containers~\cite{containers}, \AF{Ar} is
% a container and \AF{Reshape} is an inductive subset of cartesian container morphisms.
% 


% \begin{code}[hide]
%   i ⟨ eq ⟩ = i
%   (i ⊗ j) ⟨ r , r₁ ⟩ = (i ⟨ r ⟩) ⊗ (j ⟨ r₁ ⟩)
%   i ⟨ r ∙ r₁ ⟩ = i ⟨ r ⟩ ⟨ r₁ ⟩
%   (ι i ⊗ ι j) ⟨ split ⟩ = ι (combine i j)
%   ι i ⟨ flat ⟩ = let a , b = remQuot _ i in ι a ⊗ ι b
%   (i ⊗ j) ⟨ swap ⟩ = j ⊗ i
%   ((i ⊗ j) ⊗ k) ⟨ assocl ⟩ = i ⊗ (j ⊗ k)
%   (i ⊗ (j ⊗ k)) ⟨ assocr ⟩ = (i ⊗ j) ⊗ k
%   
%   
%   rev eq = eq
%   rev (r , r₁) = rev r , rev r₁
%   rev (r ∙ r₁) = rev r₁ ∙ rev r
%   rev split = flat
%   rev flat = split
%   rev swap = swap
%   rev assocl = assocr
%   rev assocr = assocl
% \end{code}


\section{CNN Building Blocks\label{sec:cnn}}

Using the array theory and combinators from the previous section we
define the primitives that are needed for the CNN.

\subsection{One-dimensional convolution}
We start with plus and minus operations for 1-d indices which are
prerequisites for defining convolution:
\begin{code}[hide]%
%
\>[2]\AgdaFunction{inject-left}\AgdaSpace{}%
\AgdaSymbol{:}\AgdaSpace{}%
\AgdaDatatype{Fin}\AgdaSpace{}%
\AgdaSymbol{(}\AgdaInductiveConstructor{suc}\AgdaSpace{}%
\AgdaGeneralizable{m}\AgdaSymbol{)}\AgdaSpace{}%
\AgdaSymbol{→}\AgdaSpace{}%
\AgdaDatatype{Fin}\AgdaSpace{}%
\AgdaSymbol{(}\AgdaInductiveConstructor{suc}\AgdaSpace{}%
\AgdaSymbol{(}\AgdaGeneralizable{n}\AgdaSpace{}%
\AgdaOperator{\AgdaPrimitive{+}}\AgdaSpace{}%
\AgdaGeneralizable{m}\AgdaSymbol{))}\<%
\\
%
\>[2]\AgdaFunction{inject-left}\AgdaSpace{}%
\AgdaSymbol{\{}\AgdaBound{m}\AgdaSymbol{\}}\AgdaSpace{}%
\AgdaSymbol{\{}\AgdaBound{n}\AgdaSymbol{\}}\AgdaSpace{}%
\AgdaBound{i}\AgdaSpace{}%
\AgdaKeyword{rewrite}\AgdaSpace{}%
\AgdaFunction{+-comm}\AgdaSpace{}%
\AgdaBound{n}\AgdaSpace{}%
\AgdaBound{m}%
\>[44]\AgdaSymbol{=}\AgdaSpace{}%
\AgdaFunction{inject+}\AgdaSpace{}%
\AgdaSymbol{\AgdaUnderscore{}}\AgdaSpace{}%
\AgdaBound{i}\<%
\\
\>[0]\<%
\\
%
\>[2]\AgdaFunction{split-inj₁}\AgdaSpace{}%
\AgdaSymbol{:}\AgdaSpace{}%
\AgdaSymbol{(}\AgdaBound{i}\AgdaSpace{}%
\AgdaSymbol{:}\AgdaSpace{}%
\AgdaDatatype{Fin}\AgdaSpace{}%
\AgdaSymbol{(}\AgdaGeneralizable{m}\AgdaSpace{}%
\AgdaOperator{\AgdaPrimitive{+}}\AgdaSpace{}%
\AgdaGeneralizable{n}\AgdaSymbol{))}\AgdaSpace{}%
\AgdaSymbol{(}\AgdaBound{k}\AgdaSpace{}%
\AgdaSymbol{:}\AgdaSpace{}%
\AgdaDatatype{Fin}\AgdaSpace{}%
\AgdaGeneralizable{m}\AgdaSymbol{)}\AgdaSpace{}%
\AgdaSymbol{→}\AgdaSpace{}%
\AgdaFunction{splitAt}\AgdaSpace{}%
\AgdaGeneralizable{m}\AgdaSpace{}%
\AgdaBound{i}\AgdaSpace{}%
\AgdaOperator{\AgdaDatatype{≡}}\AgdaSpace{}%
\AgdaInductiveConstructor{inj₁}\AgdaSpace{}%
\AgdaBound{k}\AgdaSpace{}%
\AgdaSymbol{→}\AgdaSpace{}%
\AgdaFunction{inject+}\AgdaSpace{}%
\AgdaSymbol{\AgdaUnderscore{}}\AgdaSpace{}%
\AgdaBound{k}\AgdaSpace{}%
\AgdaOperator{\AgdaDatatype{≡}}\AgdaSpace{}%
\AgdaBound{i}\<%
\\
%
\>[2]\AgdaFunction{split-inj₁}\AgdaSpace{}%
\AgdaSymbol{\{}\AgdaInductiveConstructor{suc}\AgdaSpace{}%
\AgdaBound{m}\AgdaSymbol{\}}\AgdaSpace{}%
\AgdaInductiveConstructor{zero}\AgdaSpace{}%
\AgdaDottedPattern{\AgdaSymbol{.}}\AgdaDottedPattern{\AgdaInductiveConstructor{zero}}\AgdaSpace{}%
\AgdaInductiveConstructor{refl}\AgdaSpace{}%
\AgdaSymbol{=}\AgdaSpace{}%
\AgdaInductiveConstructor{refl}\<%
\\
%
\>[2]\AgdaFunction{split-inj₁}\AgdaSpace{}%
\AgdaSymbol{\{}\AgdaInductiveConstructor{suc}\AgdaSpace{}%
\AgdaBound{m}\AgdaSymbol{\}}\AgdaSpace{}%
\AgdaSymbol{(}\AgdaInductiveConstructor{suc}\AgdaSpace{}%
\AgdaBound{i}\AgdaSymbol{)}\AgdaSpace{}%
\AgdaInductiveConstructor{zero}\AgdaSpace{}%
\AgdaBound{p}\AgdaSpace{}%
\AgdaKeyword{with}\AgdaSpace{}%
\AgdaFunction{splitAt}\AgdaSpace{}%
\AgdaBound{m}\AgdaSpace{}%
\AgdaBound{i}\AgdaSpace{}%
\AgdaSymbol{|}\AgdaSpace{}%
\AgdaFunction{inspect}\AgdaSpace{}%
\AgdaSymbol{(}\AgdaFunction{splitAt}\AgdaSpace{}%
\AgdaBound{m}\AgdaSymbol{)}\AgdaSpace{}%
\AgdaBound{i}\<%
\\
%
\>[2]\AgdaFunction{split-inj₁}\AgdaSpace{}%
\AgdaSymbol{\{}\AgdaInductiveConstructor{suc}\AgdaSpace{}%
\AgdaBound{m}\AgdaSymbol{\}}\AgdaSpace{}%
\AgdaSymbol{(}\AgdaInductiveConstructor{suc}\AgdaSpace{}%
\AgdaBound{i}\AgdaSymbol{)}\AgdaSpace{}%
\AgdaInductiveConstructor{zero}\AgdaSpace{}%
\AgdaSymbol{()}\AgdaSpace{}%
\AgdaSymbol{|}\AgdaSpace{}%
\AgdaInductiveConstructor{inj₁}\AgdaSpace{}%
\AgdaBound{x}\AgdaSpace{}%
\AgdaSymbol{|}\AgdaSpace{}%
\AgdaOperator{\AgdaInductiveConstructor{[}}\AgdaSpace{}%
\AgdaBound{r}\AgdaSpace{}%
\AgdaOperator{\AgdaInductiveConstructor{]}}\<%
\\
%
\>[2]\AgdaFunction{split-inj₁}\AgdaSpace{}%
\AgdaSymbol{\{}\AgdaInductiveConstructor{suc}\AgdaSpace{}%
\AgdaBound{m}\AgdaSymbol{\}}\AgdaSpace{}%
\AgdaSymbol{(}\AgdaInductiveConstructor{suc}\AgdaSpace{}%
\AgdaBound{i}\AgdaSymbol{)}\AgdaSpace{}%
\AgdaInductiveConstructor{zero}\AgdaSpace{}%
\AgdaSymbol{()}\AgdaSpace{}%
\AgdaSymbol{|}\AgdaSpace{}%
\AgdaInductiveConstructor{inj₂}\AgdaSpace{}%
\AgdaBound{y}\AgdaSpace{}%
\AgdaSymbol{|}\AgdaSpace{}%
\AgdaOperator{\AgdaInductiveConstructor{[}}\AgdaSpace{}%
\AgdaBound{r}\AgdaSpace{}%
\AgdaOperator{\AgdaInductiveConstructor{]}}\<%
\\
%
\>[2]\AgdaFunction{split-inj₁}\AgdaSpace{}%
\AgdaSymbol{\{}\AgdaInductiveConstructor{suc}\AgdaSpace{}%
\AgdaBound{m}\AgdaSymbol{\}}\AgdaSpace{}%
\AgdaSymbol{(}\AgdaInductiveConstructor{suc}\AgdaSpace{}%
\AgdaBound{i}\AgdaSymbol{)}\AgdaSpace{}%
\AgdaSymbol{(}\AgdaInductiveConstructor{suc}\AgdaSpace{}%
\AgdaBound{k}\AgdaSymbol{)}\AgdaSpace{}%
\AgdaBound{p}\AgdaSpace{}%
\AgdaKeyword{with}\AgdaSpace{}%
\AgdaFunction{splitAt}\AgdaSpace{}%
\AgdaBound{m}\AgdaSpace{}%
\AgdaBound{i}\AgdaSpace{}%
\AgdaSymbol{|}\AgdaSpace{}%
\AgdaFunction{inspect}\AgdaSpace{}%
\AgdaSymbol{(}\AgdaFunction{splitAt}\AgdaSpace{}%
\AgdaBound{m}\AgdaSymbol{)}\AgdaSpace{}%
\AgdaBound{i}\<%
\\
%
\>[2]\AgdaFunction{split-inj₁}\AgdaSpace{}%
\AgdaSymbol{\{}\AgdaInductiveConstructor{suc}\AgdaSpace{}%
\AgdaBound{m}\AgdaSymbol{\}}\AgdaSpace{}%
\AgdaSymbol{(}\AgdaInductiveConstructor{suc}\AgdaSpace{}%
\AgdaBound{i}\AgdaSymbol{)}\AgdaSpace{}%
\AgdaSymbol{(}\AgdaInductiveConstructor{suc}\AgdaSpace{}%
\AgdaDottedPattern{\AgdaSymbol{.}}\AgdaDottedPattern{\AgdaBound{x}}\AgdaSymbol{)}\AgdaSpace{}%
\AgdaInductiveConstructor{refl}\AgdaSpace{}%
\AgdaSymbol{|}\AgdaSpace{}%
\AgdaInductiveConstructor{inj₁}\AgdaSpace{}%
\AgdaBound{x}\AgdaSpace{}%
\AgdaSymbol{|}\AgdaSpace{}%
\AgdaOperator{\AgdaInductiveConstructor{[}}\AgdaSpace{}%
\AgdaBound{r}\AgdaSpace{}%
\AgdaOperator{\AgdaInductiveConstructor{]}}\AgdaSpace{}%
\AgdaSymbol{=}\AgdaSpace{}%
\AgdaFunction{cong}\AgdaSpace{}%
\AgdaInductiveConstructor{suc}\AgdaSpace{}%
\AgdaSymbol{(}\AgdaFunction{split-inj₁}\AgdaSpace{}%
\AgdaBound{i}\AgdaSpace{}%
\AgdaBound{x}\AgdaSpace{}%
\AgdaBound{r}\AgdaSymbol{)}\<%
\\
\>[0]\<%
\\
%
\>[2]\AgdaFunction{inj₁₂}\AgdaSpace{}%
\AgdaSymbol{:}\AgdaSpace{}%
\AgdaSymbol{\{}\AgdaBound{A}\AgdaSpace{}%
\AgdaBound{B}\AgdaSpace{}%
\AgdaSymbol{:}\AgdaSpace{}%
\AgdaPrimitive{Set}\AgdaSymbol{\}\{}\AgdaBound{x}\AgdaSpace{}%
\AgdaSymbol{:}\AgdaSpace{}%
\AgdaBound{A}\AgdaSymbol{\}\{}\AgdaBound{y}\AgdaSpace{}%
\AgdaSymbol{:}\AgdaSpace{}%
\AgdaBound{B}\AgdaSymbol{\}}\AgdaSpace{}%
\AgdaSymbol{→}\AgdaSpace{}%
\AgdaInductiveConstructor{inj₁}\AgdaSpace{}%
\AgdaBound{x}\AgdaSpace{}%
\AgdaOperator{\AgdaDatatype{≡}}\AgdaSpace{}%
\AgdaInductiveConstructor{inj₂}\AgdaSpace{}%
\AgdaBound{y}\AgdaSpace{}%
\AgdaSymbol{→}\AgdaSpace{}%
\AgdaFunction{⊥}\<%
\\
%
\>[2]\AgdaFunction{inj₁₂}\AgdaSpace{}%
\AgdaSymbol{()}\<%
\end{code}
\begin{mathpar}
\codeblock{\begin{code}%
%
\>[2]\AgdaOperator{\AgdaFunction{\AgdaUnderscore{}⊕\AgdaUnderscore{}}}\AgdaSpace{}%
\AgdaSymbol{:}\AgdaSpace{}%
\AgdaDatatype{Fin}\AgdaSpace{}%
\AgdaGeneralizable{m}\AgdaSpace{}%
\AgdaSymbol{→}\AgdaSpace{}%
\AgdaDatatype{Fin}\AgdaSpace{}%
\AgdaSymbol{(}\AgdaNumber{1}\AgdaSpace{}%
\AgdaOperator{\AgdaPrimitive{+}}\AgdaSpace{}%
\AgdaGeneralizable{n}\AgdaSymbol{)}\AgdaSpace{}%
\AgdaSymbol{→}\AgdaSpace{}%
\AgdaDatatype{Fin}\AgdaSpace{}%
\AgdaSymbol{(}\AgdaGeneralizable{m}\AgdaSpace{}%
\AgdaOperator{\AgdaPrimitive{+}}\AgdaSpace{}%
\AgdaGeneralizable{n}\AgdaSymbol{)}\<%
\\
%
\>[2]\AgdaInductiveConstructor{zero}%
\>[9]\AgdaOperator{\AgdaFunction{⊕}}\AgdaSpace{}%
\AgdaBound{j}\AgdaSpace{}%
\AgdaSymbol{=}\AgdaSpace{}%
\AgdaFunction{inject-left}\AgdaSpace{}%
\AgdaBound{j}\<%
\\
%
\>[2]\AgdaInductiveConstructor{suc}\AgdaSpace{}%
\AgdaBound{i}%
\>[9]\AgdaOperator{\AgdaFunction{⊕}}\AgdaSpace{}%
\AgdaBound{j}\AgdaSpace{}%
\AgdaSymbol{=}\AgdaSpace{}%
\AgdaInductiveConstructor{suc}\AgdaSpace{}%
\AgdaSymbol{(}\AgdaBound{i}\AgdaSpace{}%
\AgdaOperator{\AgdaFunction{⊕}}\AgdaSpace{}%
\AgdaBound{j}\AgdaSymbol{)}\<%
\end{code}}
\and
\codeblock{\begin{code}%
%
\>[2]\AgdaOperator{\AgdaFunction{\AgdaUnderscore{}⊝\AgdaUnderscore{}}}%
\>[667I]\AgdaSymbol{:}\AgdaSpace{}%
\AgdaSymbol{(}\AgdaBound{i}\AgdaSpace{}%
\AgdaSymbol{:}\AgdaSpace{}%
\AgdaDatatype{Fin}\AgdaSpace{}%
\AgdaSymbol{(}\AgdaGeneralizable{m}\AgdaSpace{}%
\AgdaOperator{\AgdaPrimitive{+}}\AgdaSpace{}%
\AgdaGeneralizable{n}\AgdaSymbol{))}\AgdaSpace{}%
\AgdaSymbol{(}\AgdaBound{j}\AgdaSpace{}%
\AgdaSymbol{:}\AgdaSpace{}%
\AgdaDatatype{Fin}\AgdaSpace{}%
\AgdaGeneralizable{m}\AgdaSymbol{)}\<%
\\
\>[.][@{}l@{}]\<[667I]%
\>[6]\AgdaSymbol{→}\AgdaSpace{}%
\AgdaRecord{Dec}\AgdaSpace{}%
\AgdaSymbol{(}\AgdaFunction{∃}\AgdaSpace{}%
\AgdaSymbol{λ}\AgdaSpace{}%
\AgdaBound{k}\AgdaSpace{}%
\AgdaSymbol{→}\AgdaSpace{}%
\AgdaBound{j}\AgdaSpace{}%
\AgdaOperator{\AgdaFunction{⊕}}\AgdaSpace{}%
\AgdaBound{k}\AgdaSpace{}%
\AgdaOperator{\AgdaDatatype{≡}}\AgdaSpace{}%
\AgdaBound{i}\AgdaSymbol{)}\<%
\end{code}}
\end{mathpar}
\begin{code}[hide]%
%
\>[2]\AgdaOperator{\AgdaFunction{\AgdaUnderscore{}⊝\AgdaUnderscore{}}}\AgdaSpace{}%
\AgdaSymbol{\{}\AgdaInductiveConstructor{suc}\AgdaSpace{}%
\AgdaBound{m}\AgdaSymbol{\}}\AgdaSpace{}%
\AgdaSymbol{\{}\AgdaBound{n}\AgdaSymbol{\}}\AgdaSpace{}%
\AgdaBound{i}\AgdaSpace{}%
\AgdaInductiveConstructor{zero}\AgdaSpace{}%
\AgdaKeyword{rewrite}\AgdaSpace{}%
\AgdaFunction{+-comm}\AgdaSpace{}%
\AgdaBound{m}\AgdaSpace{}%
\AgdaBound{n}\AgdaSpace{}%
\AgdaKeyword{with}\AgdaSpace{}%
\AgdaFunction{splitAt}\AgdaSpace{}%
\AgdaSymbol{(}\AgdaInductiveConstructor{suc}\AgdaSpace{}%
\AgdaBound{n}\AgdaSymbol{)}\AgdaSpace{}%
\AgdaBound{i}\AgdaSpace{}%
\AgdaSymbol{|}\AgdaSpace{}%
\AgdaFunction{inspect}\AgdaSpace{}%
\AgdaSymbol{(}\AgdaFunction{splitAt}\AgdaSpace{}%
\AgdaSymbol{(}\AgdaInductiveConstructor{suc}\AgdaSpace{}%
\AgdaBound{n}\AgdaSymbol{))}\AgdaSpace{}%
\AgdaBound{i}\<%
\\
%
\>[2]\AgdaSymbol{...}\AgdaSpace{}%
\AgdaSymbol{|}\AgdaSpace{}%
\AgdaInductiveConstructor{inj₁}\AgdaSpace{}%
\AgdaBound{k}\AgdaSpace{}%
\AgdaSymbol{|}\AgdaSpace{}%
\AgdaOperator{\AgdaInductiveConstructor{[}}\AgdaSpace{}%
\AgdaBound{r}\AgdaSpace{}%
\AgdaOperator{\AgdaInductiveConstructor{]}}\AgdaSpace{}%
\AgdaSymbol{=}\AgdaSpace{}%
\AgdaInductiveConstructor{yes}\AgdaSpace{}%
\AgdaSymbol{(}\AgdaBound{k}\AgdaSpace{}%
\AgdaOperator{\AgdaInductiveConstructor{,}}\AgdaSpace{}%
\AgdaFunction{split-inj₁}\AgdaSpace{}%
\AgdaBound{i}\AgdaSpace{}%
\AgdaBound{k}\AgdaSpace{}%
\AgdaBound{r}\AgdaSymbol{)}\<%
\\
%
\>[2]\AgdaSymbol{...}\AgdaSpace{}%
\AgdaSymbol{|}\AgdaSpace{}%
\AgdaInductiveConstructor{inj₂}\AgdaSpace{}%
\AgdaBound{k}\AgdaSpace{}%
\AgdaSymbol{|}\AgdaSpace{}%
\AgdaOperator{\AgdaInductiveConstructor{[}}\AgdaSpace{}%
\AgdaBound{r}\AgdaSpace{}%
\AgdaOperator{\AgdaInductiveConstructor{]}}\AgdaSpace{}%
\AgdaSymbol{=}\AgdaSpace{}%
\AgdaInductiveConstructor{no}\AgdaSpace{}%
\AgdaFunction{reason}\<%
\\
\>[2][@{}l@{\AgdaIndent{0}}]%
\>[4]\AgdaKeyword{where}\<%
\\
\>[4][@{}l@{\AgdaIndent{0}}]%
\>[6]\AgdaFunction{reason}\AgdaSpace{}%
\AgdaSymbol{:}\AgdaSpace{}%
\AgdaSymbol{\AgdaUnderscore{}}\<%
\\
%
\>[6]\AgdaFunction{reason}\AgdaSpace{}%
\AgdaSymbol{(}\AgdaBound{k}\AgdaSpace{}%
\AgdaOperator{\AgdaInductiveConstructor{,}}\AgdaSpace{}%
\AgdaInductiveConstructor{refl}\AgdaSymbol{)}\AgdaSpace{}%
\AgdaKeyword{rewrite}\AgdaSpace{}%
\AgdaFunction{splitAt-inject+}\AgdaSpace{}%
\AgdaSymbol{(}\AgdaInductiveConstructor{suc}\AgdaSpace{}%
\AgdaBound{n}\AgdaSymbol{)}\AgdaSpace{}%
\AgdaBound{m}\AgdaSpace{}%
\AgdaBound{k}\AgdaSpace{}%
\AgdaSymbol{=}\AgdaSpace{}%
\AgdaFunction{inj₁₂}\AgdaSpace{}%
\AgdaBound{r}\<%
\\
%
\>[2]\AgdaInductiveConstructor{zero}\AgdaSpace{}%
\AgdaOperator{\AgdaFunction{⊝}}\AgdaSpace{}%
\AgdaInductiveConstructor{suc}\AgdaSpace{}%
\AgdaBound{j}\AgdaSpace{}%
\AgdaSymbol{=}\AgdaSpace{}%
\AgdaInductiveConstructor{no}\AgdaSpace{}%
\AgdaSymbol{λ}\AgdaSpace{}%
\AgdaSymbol{\{}\AgdaSpace{}%
\AgdaSymbol{(}\AgdaBound{k}\AgdaSpace{}%
\AgdaOperator{\AgdaInductiveConstructor{,}}\AgdaSpace{}%
\AgdaSymbol{())}\AgdaSpace{}%
\AgdaSymbol{\}}\<%
\\
%
\>[2]\AgdaInductiveConstructor{suc}\AgdaSpace{}%
\AgdaBound{i}\AgdaSpace{}%
\AgdaOperator{\AgdaFunction{⊝}}\AgdaSpace{}%
\AgdaInductiveConstructor{suc}\AgdaSpace{}%
\AgdaBound{j}\AgdaSpace{}%
\AgdaKeyword{with}\AgdaSpace{}%
\AgdaBound{i}\AgdaSpace{}%
\AgdaOperator{\AgdaFunction{⊝}}\AgdaSpace{}%
\AgdaBound{j}\<%
\\
%
\>[2]\AgdaSymbol{...}\AgdaSpace{}%
\AgdaSymbol{|}\AgdaSpace{}%
\AgdaInductiveConstructor{yes}\AgdaSpace{}%
\AgdaSymbol{(}\AgdaBound{k}\AgdaSpace{}%
\AgdaOperator{\AgdaInductiveConstructor{,}}\AgdaSpace{}%
\AgdaBound{p}\AgdaSymbol{)}\AgdaSpace{}%
\AgdaSymbol{=}\AgdaSpace{}%
\AgdaInductiveConstructor{yes}\AgdaSpace{}%
\AgdaSymbol{(}\AgdaBound{k}\AgdaSpace{}%
\AgdaOperator{\AgdaInductiveConstructor{,}}\AgdaSpace{}%
\AgdaFunction{cong}\AgdaSpace{}%
\AgdaInductiveConstructor{suc}\AgdaSpace{}%
\AgdaBound{p}\AgdaSymbol{)}\<%
\\
%
\>[2]\AgdaSymbol{...}\AgdaSpace{}%
\AgdaSymbol{|}\AgdaSpace{}%
\AgdaInductiveConstructor{no}\AgdaSpace{}%
\AgdaBound{¬p}\AgdaSpace{}%
\AgdaSymbol{=}\AgdaSpace{}%
\AgdaInductiveConstructor{no}\AgdaSpace{}%
\AgdaSymbol{λ}\AgdaSpace{}%
\AgdaSymbol{\{}\AgdaSpace{}%
\AgdaSymbol{(}\AgdaBound{k}\AgdaSpace{}%
\AgdaOperator{\AgdaInductiveConstructor{,}}\AgdaSpace{}%
\AgdaBound{p}\AgdaSymbol{)}\AgdaSpace{}%
\AgdaSymbol{→}\AgdaSpace{}%
\AgdaBound{¬p}\AgdaSpace{}%
\AgdaSymbol{(}\AgdaBound{k}\AgdaSpace{}%
\AgdaOperator{\AgdaInductiveConstructor{,}}\AgdaSpace{}%
\AgdaFunction{suc-injective}\AgdaSpace{}%
\AgdaBound{p}\AgdaSymbol{)}\AgdaSpace{}%
\AgdaSymbol{\}}\<%
\\
%
\\[\AgdaEmptyExtraSkip]%
%
\>[2]\AgdaFunction{inject-left-zero}\AgdaSpace{}%
\AgdaSymbol{:}\AgdaSpace{}%
\AgdaFunction{inject-left}\AgdaSpace{}%
\AgdaSymbol{\{}\AgdaGeneralizable{m}\AgdaSymbol{\}}\AgdaSpace{}%
\AgdaSymbol{\{}\AgdaGeneralizable{n}\AgdaSymbol{\}}\AgdaSpace{}%
\AgdaInductiveConstructor{zero}\AgdaSpace{}%
\AgdaOperator{\AgdaDatatype{≡}}\AgdaSpace{}%
\AgdaInductiveConstructor{zero}\<%
\\
%
\>[2]\AgdaFunction{inject-left-zero}\AgdaSpace{}%
\AgdaSymbol{\{}\AgdaBound{m}\AgdaSymbol{\}}\AgdaSpace{}%
\AgdaSymbol{\{}\AgdaBound{n}\AgdaSymbol{\}}\AgdaSpace{}%
\AgdaKeyword{rewrite}\AgdaSpace{}%
\AgdaFunction{+-comm}\AgdaSpace{}%
\AgdaBound{n}\AgdaSpace{}%
\AgdaBound{m}\AgdaSpace{}%
\AgdaSymbol{=}\AgdaSpace{}%
\AgdaInductiveConstructor{refl}\<%
\\
%
\\[\AgdaEmptyExtraSkip]%
%
\>[2]\AgdaFunction{suc-not-zero}\AgdaSpace{}%
\AgdaSymbol{:}\AgdaSpace{}%
\AgdaSymbol{\{}\AgdaBound{i}\AgdaSpace{}%
\AgdaSymbol{:}\AgdaSpace{}%
\AgdaDatatype{Fin}\AgdaSpace{}%
\AgdaGeneralizable{m}\AgdaSymbol{\}}\AgdaSpace{}%
\AgdaSymbol{→}\AgdaSpace{}%
\AgdaOperator{\AgdaDatatype{\AgdaUnderscore{}≡\AgdaUnderscore{}}}\AgdaSpace{}%
\AgdaSymbol{\{}\AgdaArgument{A}\AgdaSpace{}%
\AgdaSymbol{=}\AgdaSpace{}%
\AgdaDatatype{Fin}\AgdaSpace{}%
\AgdaSymbol{(}\AgdaInductiveConstructor{suc}\AgdaSpace{}%
\AgdaGeneralizable{m}\AgdaSymbol{)\}}\AgdaSpace{}%
\AgdaSymbol{(}\AgdaInductiveConstructor{suc}\AgdaSpace{}%
\AgdaBound{i}\AgdaSymbol{)}\AgdaSpace{}%
\AgdaInductiveConstructor{zero}\AgdaSpace{}%
\AgdaSymbol{→}\AgdaSpace{}%
\AgdaFunction{⊥}\<%
\\
%
\>[2]\AgdaFunction{suc-not-zero}\AgdaSpace{}%
\AgdaSymbol{()}\<%
\\
%
\\[\AgdaEmptyExtraSkip]%
%
\>[2]\AgdaFunction{inject-left-suc}\AgdaSpace{}%
\AgdaSymbol{:}\AgdaSpace{}%
\AgdaSymbol{∀}\AgdaSpace{}%
\AgdaSymbol{(}\AgdaBound{i}\AgdaSpace{}%
\AgdaSymbol{:}\AgdaSpace{}%
\AgdaDatatype{Fin}\AgdaSpace{}%
\AgdaGeneralizable{m}\AgdaSymbol{)}\AgdaSpace{}%
\AgdaSymbol{→}\AgdaSpace{}%
\AgdaFunction{inject-left}\AgdaSpace{}%
\AgdaSymbol{\{}\AgdaGeneralizable{m}\AgdaSymbol{\}}\AgdaSpace{}%
\AgdaSymbol{\{}\AgdaGeneralizable{n}\AgdaSymbol{\}}\AgdaSpace{}%
\AgdaSymbol{(}\AgdaInductiveConstructor{suc}\AgdaSpace{}%
\AgdaBound{i}\AgdaSymbol{)}\AgdaSpace{}%
\AgdaOperator{\AgdaDatatype{≡}}\AgdaSpace{}%
\AgdaInductiveConstructor{zero}\AgdaSpace{}%
\AgdaSymbol{→}\AgdaSpace{}%
\AgdaFunction{⊥}\<%
\\
%
\>[2]\AgdaFunction{inject-left-suc}\AgdaSpace{}%
\AgdaSymbol{\{}\AgdaBound{m}\AgdaSymbol{\}}\AgdaSpace{}%
\AgdaSymbol{\{}\AgdaBound{n}\AgdaSymbol{\}}\AgdaSpace{}%
\AgdaBound{i}\AgdaSpace{}%
\AgdaBound{p}\AgdaSpace{}%
\AgdaKeyword{rewrite}\AgdaSpace{}%
\AgdaFunction{+-comm}\AgdaSpace{}%
\AgdaBound{n}\AgdaSpace{}%
\AgdaBound{m}\AgdaSpace{}%
\AgdaSymbol{=}\AgdaSpace{}%
\AgdaFunction{suc-not-zero}\AgdaSpace{}%
\AgdaBound{p}\<%
\\
%
\\[\AgdaEmptyExtraSkip]%
%
\>[2]\AgdaFunction{zero-suc-⊥}\AgdaSpace{}%
\AgdaSymbol{:}\AgdaSpace{}%
\AgdaSymbol{∀}\AgdaSpace{}%
\AgdaSymbol{\{}\AgdaBound{i}\AgdaSpace{}%
\AgdaSymbol{:}\AgdaSpace{}%
\AgdaDatatype{Fin}\AgdaSpace{}%
\AgdaGeneralizable{n}\AgdaSymbol{\}}\AgdaSpace{}%
\AgdaSymbol{→}\AgdaSpace{}%
\AgdaOperator{\AgdaDatatype{\AgdaUnderscore{}≡\AgdaUnderscore{}}}\AgdaSpace{}%
\AgdaSymbol{\{}\AgdaArgument{A}\AgdaSpace{}%
\AgdaSymbol{=}\AgdaSpace{}%
\AgdaDatatype{Fin}\AgdaSpace{}%
\AgdaSymbol{(}\AgdaInductiveConstructor{suc}\AgdaSpace{}%
\AgdaGeneralizable{n}\AgdaSymbol{)\}}\AgdaSpace{}%
\AgdaInductiveConstructor{zero}\AgdaSpace{}%
\AgdaSymbol{(}\AgdaInductiveConstructor{suc}\AgdaSpace{}%
\AgdaBound{i}\AgdaSymbol{)}\AgdaSpace{}%
\AgdaSymbol{→}\AgdaSpace{}%
\AgdaFunction{⊥}\<%
\\
%
\>[2]\AgdaFunction{zero-suc-⊥}\AgdaSpace{}%
\AgdaSymbol{()}\<%
\\
%
\\[\AgdaEmptyExtraSkip]%
%
\>[2]\AgdaComment{--\ TODO:\ this\ is\ annoying\ to\ do\ inductively\ on\ Fin,\ it\ is\ easier\ to}\<%
\\
%
\>[2]\AgdaComment{--\ \ \ \ \ \ \ implement\ this\ via\ Fin\ n\ =\ Σ\ ℕ\ (\AgdaUnderscore{}<\ n)\ representation}\<%
\\
%
\>[2]\AgdaComment{--\ minusx\ :\ (i\ :\ Fin\ (m\ +\ n))\ →\ (j\ :\ Fin\ (suc\ n))\ →\ Dec\ (∃\ λ\ k\ →\ k\ ⊕\ j\ ≡\ i)}\<%
\\
%
\>[2]\AgdaComment{--\ minusx\ \{zero\}\ i\ zero\ =\ no\ λ\ \{\ (()\ ,\ \AgdaUnderscore{})\ \}}\<%
\\
%
\>[2]\AgdaComment{--\ minusx\ \{suc\ m\}\ \{n\}\ zero\ zero\ =\ yes\ (zero\ ,\ inject-left-zero\ \{n\}\ \{m\})}\<%
\\
%
\>[2]\AgdaComment{--\ minusx\ \{suc\ m\}\ \{n\}\ (suc\ i)\ zero\ with\ minusx\ \{m\}\ i\ zero}\<%
\\
%
\>[2]\AgdaComment{--\ ...\ |\ yes\ (j\ ,\ p)\ =\ yes\ (suc\ j\ ,\ cong\ suc\ p)}\<%
\\
%
\>[2]\AgdaComment{--\ ...\ |\ no\ ¬p\ =\ no\ λ\ \{\ (zero\ ,\ p)\ →\ let\ rr\ =\ trans\ (sym\ \$\ inject-left-zero\ \{n\}\ \{m\})\ p\ }\<%
\\
%
\>[2]\AgdaComment{--\ \ \ \ \ \ \ \ \ \ \ \ \ \ \ \ \ \ \ \ \ \ \ \ \ \ \ \ \ \ \ \ \ \ \ in\ zero-suc-⊥\ rr}\<%
\\
%
\>[2]\AgdaComment{--\ \ \ \ \ \ \ \ \ \ \ \ \ \ \ \ \ \ \ \ ;\ (suc\ j\ ,\ p)\ →\ ¬p\ (j\ ,\ suc-injective\ p)\ \}}\<%
\\
%
\\[\AgdaEmptyExtraSkip]%
%
\>[2]\AgdaComment{--\ minusx\ \{zero\}\ i\ (suc\ j)\ =\ no\ λ\ \{\ (()\ ,\ p)\ \}}\<%
\\
%
\>[2]\AgdaComment{--\ minusx\ \{suc\ m\}\ zero\ (suc\ j)\ =\ no\ λ\ \{\ (zero\ ,\ p)\ →\ inject-left-suc\ j\ p}\<%
\\
%
\>[2]\AgdaComment{--\ \ \ \ \ \ \ \ \ \ \ \ \ \ \ \ \ \ \ \ \ \ \ \ \ \ \ \ \ \ \ \ \ \ \ \ ;\ (suc\ k\ ,\ ())\ \}}\<%
\\
%
\>[2]\AgdaComment{--\ minusx\ \{suc\ m\}\ \{suc\ n\}\ (suc\ i)\ (suc\ j)\ =\ ?\ }\<%
\end{code}
Recall that the type \AF{Fin} $n$ is a type for natural numbers $i$ that
are bounded by $n$ (\ie{} $i < n$).  Plus adds two bounded indices $i$ and $j$
where $i < m$ and $j < 1 + n$ (both $i$ and $j$ are non-negative as any
element of \AF{Fin}).
The indices $i$ and $j$ are added as natural numbers, so there is
no easy way to apply type isomorphisms such as \AD{Fin} $(m + n)$ $\cong$
\AD{Fin} $m$ $⊎$ \AD{Fin} n.  Minus is a partial inverse of plus described below.

While both definitions look innocent, their types carry non-trivial
information about the bounds.  Consider the bounds in the \AF{\_⊕\_} operation:
\begin{mathpar}
  \inferrule*
    {i < m \and j < 1 + n}
    {i+j < m + n}
\end{mathpar}
This looks a little surprising, but this indeed holds for natural numbers.
Readers may convince themselves by considering the maximum value that $i$ and $j$
can possibly take.  The \AF{\_⊕\_} operation have partial inverses making it possible
to define left and right subtraction.  We consider left subtraction \AF{\_⊝\_}.
Its type says that there exists a decision procedure for finding $k$ of type
\AF{Fin} (1 + \AB{n}) (\eg{} $k < 1 + n$) together with the proof that $k$ is
an inverse of \AF{⊕}.
In some sense \AF{Dec} is similar to \AF{Maybe} type, except it forces one
to prove why the value does not exist as opposed to just returning \AC{nothing}.
For example, if we were to evaluate $i ⊝ j$ where $i = 1 < 3 + 5$ and $j = 2 < 3$,
we will get a proof that there is no natural number $k < 1 + 5$ such that $2 ⊕ k ≡ 1$.
Here dependent types come very useful, as we eliminate the possibility of
introducing off-by-one errors in the definition of \AF{⊝}.


Now we are ready to define a 1-dimensional convolution.
A side note for mathematically inclined readers: we use the term
\emph{convolution} in the way it is used in machine learning.  Technically,
we compute a cross-correlation, because the array of weights is not flipped.
However, in practice this is not a problem, as we assume that weights are
stored flipped in memory.

We define type synonyms \AF{Vec} and \AF{Ix} which are 1-dimensional versions
of \AF{Ar} and \AF{P}.
\begin{mathpar}
\codeblock{\begin{code}%
%
\>[2]\AgdaFunction{Vec}\AgdaSpace{}%
\AgdaSymbol{:}\AgdaSpace{}%
\AgdaDatatype{ℕ}\AgdaSpace{}%
\AgdaSymbol{→}\AgdaSpace{}%
\AgdaPrimitive{Set}\AgdaSpace{}%
\AgdaSymbol{→}\AgdaSpace{}%
\AgdaPrimitive{Set}\<%
\\
%
\>[2]\AgdaFunction{Vec}\AgdaSpace{}%
\AgdaBound{m}\AgdaSpace{}%
\AgdaBound{X}\AgdaSpace{}%
\AgdaSymbol{=}\AgdaSpace{}%
\AgdaFunction{Ar}\AgdaSpace{}%
\AgdaSymbol{(}\AgdaInductiveConstructor{ι}\AgdaSpace{}%
\AgdaBound{m}\AgdaSymbol{)}\AgdaSpace{}%
\AgdaBound{X}\<%
\end{code}}
\and
\codeblock{\begin{code}%
%
\>[2]\AgdaFunction{Ix}\AgdaSpace{}%
\AgdaSymbol{:}\AgdaSpace{}%
\AgdaDatatype{ℕ}\AgdaSpace{}%
\AgdaSymbol{→}\AgdaSpace{}%
\AgdaPrimitive{Set}\<%
\\
%
\>[2]\AgdaFunction{Ix}\AgdaSpace{}%
\AgdaBound{m}\AgdaSpace{}%
\AgdaSymbol{=}\AgdaSpace{}%
\AgdaDatatype{P}\AgdaSpace{}%
\AgdaSymbol{(}\AgdaInductiveConstructor{ι}\AgdaSpace{}%
\AgdaBound{m}\AgdaSymbol{)}\<%
\end{code}}
\end{mathpar}
We introduce the \AF{slide₁} primitive that selects a $(1+n)$-element vector
from the $(m+n)$-element vector starting at the offset $i$.  Then,
following~\cite{cnn-array}, we compute $m$-element array of slides
and then sum it up.
\begin{mathpar}
\codeblock{\begin{code}%
%
\>[2]\AgdaFunction{slide₁}\AgdaSpace{}%
\AgdaSymbol{:}\AgdaSpace{}%
\AgdaFunction{Ix}\AgdaSpace{}%
\AgdaGeneralizable{m}\AgdaSpace{}%
\AgdaSymbol{→}\AgdaSpace{}%
\AgdaFunction{Vec}\AgdaSpace{}%
\AgdaSymbol{(}\AgdaGeneralizable{m}\AgdaSpace{}%
\AgdaOperator{\AgdaPrimitive{+}}\AgdaSpace{}%
\AgdaGeneralizable{n}\AgdaSymbol{)}\AgdaSpace{}%
\AgdaGeneralizable{X}\AgdaSpace{}%
\AgdaSymbol{→}\AgdaSpace{}%
\AgdaFunction{Vec}\AgdaSpace{}%
\AgdaSymbol{(}\AgdaNumber{1}\AgdaSpace{}%
\AgdaOperator{\AgdaPrimitive{+}}\AgdaSpace{}%
\AgdaGeneralizable{n}\AgdaSymbol{)}\AgdaSpace{}%
\AgdaGeneralizable{X}\<%
\\
%
\>[2]\AgdaFunction{slide₁}\AgdaSpace{}%
\AgdaSymbol{(}\AgdaInductiveConstructor{ι}\AgdaSpace{}%
\AgdaBound{i}\AgdaSymbol{)}\AgdaSpace{}%
\AgdaBound{v}\AgdaSpace{}%
\AgdaSymbol{(}\AgdaInductiveConstructor{ι}\AgdaSpace{}%
\AgdaBound{j}\AgdaSymbol{)}\AgdaSpace{}%
\AgdaSymbol{=}\AgdaSpace{}%
\AgdaBound{v}\AgdaSpace{}%
\AgdaSymbol{(}\AgdaInductiveConstructor{ι}\AgdaSpace{}%
\AgdaSymbol{(}\AgdaBound{i}\AgdaSpace{}%
\AgdaOperator{\AgdaFunction{⊕}}\AgdaSpace{}%
\AgdaBound{j}\AgdaSymbol{))}\<%
\\
%
\\[\AgdaEmptyExtraSkip]%
%
\>[2]\AgdaFunction{conv₁}\AgdaSpace{}%
\AgdaSymbol{:}\AgdaSpace{}%
\AgdaFunction{Vec}\AgdaSpace{}%
\AgdaSymbol{(}\AgdaGeneralizable{m}\AgdaSpace{}%
\AgdaOperator{\AgdaPrimitive{+}}\AgdaSpace{}%
\AgdaGeneralizable{n}\AgdaSymbol{)}\AgdaSpace{}%
\AgdaDatatype{ℕ}\AgdaSpace{}%
\AgdaSymbol{→}\AgdaSpace{}%
\AgdaFunction{Vec}\AgdaSpace{}%
\AgdaGeneralizable{m}\AgdaSpace{}%
\AgdaDatatype{ℕ}\AgdaSpace{}%
\AgdaSymbol{→}\AgdaSpace{}%
\AgdaFunction{Vec}\AgdaSpace{}%
\AgdaSymbol{(}\AgdaNumber{1}\AgdaSpace{}%
\AgdaOperator{\AgdaPrimitive{+}}\AgdaSpace{}%
\AgdaGeneralizable{n}\AgdaSymbol{)}\AgdaSpace{}%
\AgdaDatatype{ℕ}\<%
\\
%
\>[2]\AgdaFunction{conv₁}\AgdaSpace{}%
\AgdaBound{a}\AgdaSpace{}%
\AgdaBound{w}\AgdaSpace{}%
\AgdaSymbol{=}\AgdaSpace{}%
\AgdaFunction{sum}\AgdaSpace{}%
\AgdaSymbol{(}\AgdaFunction{zipWith}\AgdaSpace{}%
\AgdaOperator{\AgdaPrimitive{\AgdaUnderscore{}+\AgdaUnderscore{}}}\AgdaSymbol{)}\AgdaSpace{}%
\AgdaSymbol{(}\AgdaFunction{K}\AgdaSpace{}%
\AgdaNumber{0}\AgdaSymbol{)}\AgdaSpace{}%
\AgdaSymbol{(λ}\AgdaSpace{}%
\AgdaBound{i}\AgdaSpace{}%
\AgdaSymbol{→}\AgdaSpace{}%
\AgdaFunction{map}\AgdaSpace{}%
\AgdaSymbol{(}\AgdaBound{w}\AgdaSpace{}%
\AgdaBound{i}\AgdaSpace{}%
\AgdaOperator{\AgdaPrimitive{*\AgdaUnderscore{}}}\AgdaSymbol{)}\AgdaSpace{}%
\AgdaSymbol{(}\AgdaFunction{slide₁}\AgdaSpace{}%
\AgdaBound{i}\AgdaSpace{}%
\AgdaBound{a}\AgdaSymbol{))}\<%
\end{code}}
\end{mathpar}
Note that in the definition of \AF{conv₁} we use a standard array language
trick --- we pull summation to the outside.  For example, for $m = 3$, $n = 2$,
a straight-forward way to compute (\AF{conv₁} $[a_1, a_2, a_3, a_4, a_5]$
$[w_1, w_2, w_3]$) would be $[a_1w_1 + a_2w_2 + a_3w_3, a_2w_1 + a_3w_2 +
a_4w_3,\dots]$.  However, the above definition proceeds as $w_1[a_1,a_2,a_3] +
w_2[a_2,a_3,a_4] + w_3[a_3,a_4,a_5]$ which computes the same result.  Such
definition makes it easy to replace the implementation of slide, obtaining
other versions of convolution such as the one with constant or cyclic
boundaries.  As we demonstrate in the next section, this pattern generalises
nicely to higher ranks.



\subsection{Generalisation\label{sec:general-ix-ops}}
Now we generalise 1-dimensional slide for arrays of higher ranks.
This requires generalising vector shapes $m + n$ and $1 + n$ for the cases
when $m$ and $n$ for arbitrary shapes.  In case of addition, we need a witness
that both shapes
have the same length.  If they do, their components are added point-wise.
We define a three-way relation \AF{\_+\_≈\_} that combines the witness and
the action.  That is, the type \AB{p} \AF{+} \AB{q} \AF{≈} \AB{r} says that
$p$ and $q$ have the same length and that $r$ is a point-wise addition
of $p$ and $q$.  A similar relation \AF{suc\_≈\_} is introduced for $1 + n$
case, and \AF{\_*\_≈\_} witnesses point-wise
multiplication that will be needed for blocking.  We define these relations
in two steps.  Firstly, we give a generalised pointwise relations for binary
and ternary relations on natural numbers:
\begin{mathpar}
\codeblock{\begin{code}%
%
\>[2]\AgdaKeyword{data}%
\>[956I]\AgdaDatatype{Pw₂}\AgdaSpace{}%
\AgdaSymbol{(}\AgdaBound{R}\AgdaSpace{}%
\AgdaSymbol{:}\AgdaSpace{}%
\AgdaSymbol{(}\AgdaBound{a}\AgdaSpace{}%
\AgdaBound{b}\AgdaSpace{}%
\AgdaSymbol{:}\AgdaSpace{}%
\AgdaDatatype{ℕ}\AgdaSymbol{)}\AgdaSpace{}%
\AgdaSymbol{→}\AgdaSpace{}%
\AgdaPrimitive{Set}\AgdaSymbol{)}\<%
\\
\>[.][@{}l@{}]\<[956I]%
\>[7]\AgdaSymbol{:}\AgdaSpace{}%
\AgdaSymbol{(}\AgdaBound{a}\AgdaSpace{}%
\AgdaBound{b}\AgdaSpace{}%
\AgdaSymbol{:}\AgdaSpace{}%
\AgdaDatatype{S}\AgdaSymbol{)}\AgdaSpace{}%
\AgdaSymbol{→}\AgdaSpace{}%
\AgdaPrimitive{Set}\AgdaSpace{}%
\AgdaKeyword{where}\AgdaSpace{}%
\AgdaKeyword{instance}\<%
\\
\>[2][@{}l@{\AgdaIndent{0}}]%
\>[6]\AgdaInductiveConstructor{[]}%
\>[12]\AgdaSymbol{:}\AgdaSpace{}%
\AgdaDatatype{Pw₂}\AgdaSpace{}%
\AgdaBound{R}\AgdaSpace{}%
\AgdaInductiveConstructor{[]}\AgdaSpace{}%
\AgdaInductiveConstructor{[]}\<%
\\
%
\>[6]\AgdaInductiveConstructor{cons}%
\>[12]\AgdaSymbol{:}\AgdaSpace{}%
\AgdaSymbol{⦃}\AgdaSpace{}%
\AgdaBound{R}\AgdaSpace{}%
\AgdaGeneralizable{m}\AgdaSpace{}%
\AgdaGeneralizable{n}\AgdaSpace{}%
\AgdaSymbol{⦄}\AgdaSpace{}%
\AgdaSymbol{→}\AgdaSpace{}%
\AgdaSymbol{⦃}\AgdaSpace{}%
\AgdaDatatype{Pw₂}\AgdaSpace{}%
\AgdaBound{R}\AgdaSpace{}%
\AgdaGeneralizable{s}\AgdaSpace{}%
\AgdaGeneralizable{p}\AgdaSpace{}%
\AgdaSymbol{⦄}\<%
\\
%
\>[12]\AgdaSymbol{→}\AgdaSpace{}%
\AgdaDatatype{Pw₂}\AgdaSpace{}%
\AgdaBound{R}\AgdaSpace{}%
\AgdaSymbol{(}\AgdaGeneralizable{m}\AgdaSpace{}%
\AgdaOperator{\AgdaInductiveConstructor{∷}}\AgdaSpace{}%
\AgdaGeneralizable{s}\AgdaSymbol{)}\AgdaSpace{}%
\AgdaSymbol{(}\AgdaGeneralizable{n}\AgdaSpace{}%
\AgdaOperator{\AgdaInductiveConstructor{∷}}\AgdaSpace{}%
\AgdaGeneralizable{p}\AgdaSymbol{)}\<%
\end{code}}
\and
\codeblock{\begin{code}%
%
\>[2]\AgdaKeyword{data}%
\>[997I]\AgdaDatatype{Pw₃}\AgdaSpace{}%
\AgdaSymbol{(}\AgdaBound{R}\AgdaSpace{}%
\AgdaSymbol{:}\AgdaSpace{}%
\AgdaSymbol{(}\AgdaBound{a}\AgdaSpace{}%
\AgdaBound{b}\AgdaSpace{}%
\AgdaBound{c}\AgdaSpace{}%
\AgdaSymbol{:}\AgdaSpace{}%
\AgdaDatatype{ℕ}\AgdaSymbol{)}\AgdaSpace{}%
\AgdaSymbol{→}\AgdaSpace{}%
\AgdaPrimitive{Set}\AgdaSymbol{)}\<%
\\
\>[.][@{}l@{}]\<[997I]%
\>[7]\AgdaSymbol{:}\AgdaSpace{}%
\AgdaSymbol{(}\AgdaBound{a}\AgdaSpace{}%
\AgdaBound{b}\AgdaSpace{}%
\AgdaBound{c}\AgdaSpace{}%
\AgdaSymbol{:}\AgdaSpace{}%
\AgdaDatatype{S}\AgdaSymbol{)}\AgdaSpace{}%
\AgdaSymbol{→}\AgdaSpace{}%
\AgdaPrimitive{Set}\AgdaSpace{}%
\AgdaKeyword{where}\AgdaSpace{}%
\AgdaKeyword{instance}\<%
\\
\>[2][@{}l@{\AgdaIndent{0}}]%
\>[6]\AgdaInductiveConstructor{[]}%
\>[12]\AgdaSymbol{:}\AgdaSpace{}%
\AgdaDatatype{Pw₃}\AgdaSpace{}%
\AgdaBound{R}\AgdaSpace{}%
\AgdaInductiveConstructor{[]}\AgdaSpace{}%
\AgdaInductiveConstructor{[]}\AgdaSpace{}%
\AgdaInductiveConstructor{[]}\<%
\\
%
\>[6]\AgdaInductiveConstructor{cons}%
\>[12]\AgdaSymbol{:}\AgdaSpace{}%
\AgdaSymbol{⦃}\AgdaSpace{}%
\AgdaBound{R}\AgdaSpace{}%
\AgdaGeneralizable{m}\AgdaSpace{}%
\AgdaGeneralizable{n}\AgdaSpace{}%
\AgdaGeneralizable{k}\AgdaSpace{}%
\AgdaSymbol{⦄}\AgdaSpace{}%
\AgdaSymbol{→}\AgdaSpace{}%
\AgdaSymbol{⦃}\AgdaSpace{}%
\AgdaDatatype{Pw₃}\AgdaSpace{}%
\AgdaBound{R}\AgdaSpace{}%
\AgdaGeneralizable{s}\AgdaSpace{}%
\AgdaGeneralizable{p}\AgdaSpace{}%
\AgdaGeneralizable{q}\AgdaSpace{}%
\AgdaSymbol{⦄}\<%
\\
%
\>[12]\AgdaSymbol{→}\AgdaSpace{}%
\AgdaDatatype{Pw₃}\AgdaSpace{}%
\AgdaBound{R}\AgdaSpace{}%
\AgdaSymbol{(}\AgdaGeneralizable{m}\AgdaSpace{}%
\AgdaOperator{\AgdaInductiveConstructor{∷}}\AgdaSpace{}%
\AgdaGeneralizable{s}\AgdaSymbol{)}\AgdaSpace{}%
\AgdaSymbol{(}\AgdaGeneralizable{n}\AgdaSpace{}%
\AgdaOperator{\AgdaInductiveConstructor{∷}}\AgdaSpace{}%
\AgdaGeneralizable{p}\AgdaSymbol{)}\AgdaSpace{}%
\AgdaSymbol{(}\AgdaGeneralizable{k}\AgdaSpace{}%
\AgdaOperator{\AgdaInductiveConstructor{∷}}\AgdaSpace{}%
\AgdaGeneralizable{q}\AgdaSymbol{)}\<%
\end{code}}
\end{mathpar}
While the definition is straight-forward, note that we mark constructors
with the keyword \AK{instance} and we turn the arguments of \AC{cons}
into instance arguments\footnote{See \url{https://agda.readthedocs.io/en/v2.7.0.1/language/instance-arguments.html} for more details.}.  These arguments
behave like the hidden arguments, except Agda will apply an instance
search when solving them.  This allows us to omit these proofs in
a larger number of cases than if we were to use hidden arguments.

\begin{code}[hide]%
%
\>[2]\AgdaKeyword{infix}\AgdaSpace{}%
\AgdaNumber{5}\AgdaSpace{}%
\AgdaOperator{\AgdaFunction{\AgdaUnderscore{}+\AgdaUnderscore{}≈\AgdaUnderscore{}}}\<%
\\
%
\>[2]\AgdaKeyword{infix}\AgdaSpace{}%
\AgdaNumber{5}\AgdaSpace{}%
\AgdaOperator{\AgdaFunction{suc\AgdaUnderscore{}≈\AgdaUnderscore{}}}\<%
\\
%
\>[2]\AgdaKeyword{infix}\AgdaSpace{}%
\AgdaNumber{5}\AgdaSpace{}%
\AgdaOperator{\AgdaFunction{\AgdaUnderscore{}*\AgdaUnderscore{}≈\AgdaUnderscore{}}}\<%
\\
%
\>[2]\AgdaKeyword{infixl}\AgdaSpace{}%
\AgdaNumber{8}\AgdaSpace{}%
\AgdaOperator{\AgdaFunction{\AgdaUnderscore{}⊝ₚ\AgdaUnderscore{}}}\<%
\end{code}

The second step is to define the actual relations.  With the help of composition
combinators ($f$ \AF{∘} $g$ = λ x → $f$ ($g$ x)) and ($f$ \AF{∘₂} $g$ = λ x y → $f$ ($g$ x y))
the definitions are as follows.
\begin{mathpar}
\codeblock{\begin{code}%
%
\>[2]\AgdaOperator{\AgdaFunction{\AgdaUnderscore{}+\AgdaUnderscore{}≈\AgdaUnderscore{}}}\AgdaSpace{}%
\AgdaSymbol{:}\AgdaSpace{}%
\AgdaSymbol{(}\AgdaBound{s}\AgdaSpace{}%
\AgdaBound{p}\AgdaSpace{}%
\AgdaBound{q}\AgdaSpace{}%
\AgdaSymbol{:}\AgdaSpace{}%
\AgdaDatatype{S}\AgdaSymbol{)}\AgdaSpace{}%
\AgdaSymbol{→}\AgdaSpace{}%
\AgdaPrimitive{Set}\<%
\\
%
\>[2]\AgdaOperator{\AgdaFunction{\AgdaUnderscore{}+\AgdaUnderscore{}≈\AgdaUnderscore{}}}\AgdaSpace{}%
\AgdaSymbol{=}\AgdaSpace{}%
\AgdaDatatype{Pw₃}\AgdaSpace{}%
\AgdaSymbol{(}\AgdaOperator{\AgdaDatatype{\AgdaUnderscore{}≡\AgdaUnderscore{}}}\AgdaSpace{}%
\AgdaOperator{\AgdaFunction{∘₂}}\AgdaSpace{}%
\AgdaOperator{\AgdaPrimitive{\AgdaUnderscore{}+\AgdaUnderscore{}}}\AgdaSymbol{)}\<%
\end{code}}
\and
\codeblock{\begin{code}%
%
\>[2]\AgdaOperator{\AgdaFunction{\AgdaUnderscore{}*\AgdaUnderscore{}≈\AgdaUnderscore{}}}\AgdaSpace{}%
\AgdaSymbol{:}\AgdaSpace{}%
\AgdaSymbol{(}\AgdaBound{s}\AgdaSpace{}%
\AgdaBound{p}\AgdaSpace{}%
\AgdaBound{q}\AgdaSpace{}%
\AgdaSymbol{:}\AgdaSpace{}%
\AgdaDatatype{S}\AgdaSymbol{)}\AgdaSpace{}%
\AgdaSymbol{→}\AgdaSpace{}%
\AgdaPrimitive{Set}\<%
\\
%
\>[2]\AgdaOperator{\AgdaFunction{\AgdaUnderscore{}*\AgdaUnderscore{}≈\AgdaUnderscore{}}}\AgdaSpace{}%
\AgdaSymbol{=}\AgdaSpace{}%
\AgdaDatatype{Pw₃}\AgdaSpace{}%
\AgdaSymbol{(}\AgdaOperator{\AgdaDatatype{\AgdaUnderscore{}≡\AgdaUnderscore{}}}\AgdaSpace{}%
\AgdaOperator{\AgdaFunction{∘₂}}\AgdaSpace{}%
\AgdaOperator{\AgdaPrimitive{\AgdaUnderscore{}*\AgdaUnderscore{}}}\AgdaSymbol{)}\<%
\end{code}}
\and
\codeblock{\begin{code}%
%
\>[2]\AgdaOperator{\AgdaFunction{suc\AgdaUnderscore{}≈\AgdaUnderscore{}}}\AgdaSpace{}%
\AgdaSymbol{:}\AgdaSpace{}%
\AgdaSymbol{(}\AgdaBound{s}\AgdaSpace{}%
\AgdaBound{p}\AgdaSpace{}%
\AgdaSymbol{:}\AgdaSpace{}%
\AgdaDatatype{S}\AgdaSymbol{)}\AgdaSpace{}%
\AgdaSymbol{→}\AgdaSpace{}%
\AgdaPrimitive{Set}\<%
\\
%
\>[2]\AgdaOperator{\AgdaFunction{suc\AgdaUnderscore{}≈\AgdaUnderscore{}}}\AgdaSpace{}%
\AgdaSymbol{=}\AgdaSpace{}%
\AgdaDatatype{Pw₂}\AgdaSpace{}%
\AgdaSymbol{(}\AgdaOperator{\AgdaDatatype{\AgdaUnderscore{}≡\AgdaUnderscore{}}}\AgdaSpace{}%
\AgdaOperator{\AgdaFunction{∘}}\AgdaSpace{}%
\AgdaInductiveConstructor{suc}\AgdaSymbol{)}\<%
\end{code}}
\end{mathpar}

With these relations in place, we could define generalised convolution
similarly to \AF{sum} where we recurse over the shape, performing one
operation at a time.  However, there is a good point made
in~\cite{cnn-array} about shifting the shape recursion into index computation.
% Talk about mental model of runtime where arrays are flat and indices are offsets
Therefore we define \AF{\_⊕ₚ\_} and \AF{\_⊝ₚ\_} which generalise \AF{\_⊕\_} and
\AF{\_⊝\_} for higher ranks.  Once again, \AD{Dec} type forces \AF{⊝ₚ} to justify
the cases when the inverse does not exist.
\begin{mathpar}
\codeblock{\begin{code}%
%
\>[2]\AgdaOperator{\AgdaFunction{\AgdaUnderscore{}⊕ₚ\AgdaUnderscore{}}}\AgdaSpace{}%
\AgdaSymbol{:}\AgdaSpace{}%
\AgdaDatatype{P}\AgdaSpace{}%
\AgdaGeneralizable{s}\AgdaSpace{}%
\AgdaSymbol{→}\AgdaSpace{}%
\AgdaDatatype{P}\AgdaSpace{}%
\AgdaGeneralizable{u}\AgdaSpace{}%
\AgdaSymbol{→}\AgdaSpace{}%
\AgdaOperator{\AgdaFunction{suc}}\AgdaSpace{}%
\AgdaGeneralizable{p}\AgdaSpace{}%
\AgdaOperator{\AgdaFunction{≈}}\AgdaSpace{}%
\AgdaGeneralizable{u}\AgdaSpace{}%
\AgdaSymbol{→}\AgdaSpace{}%
\AgdaGeneralizable{s}\AgdaSpace{}%
\AgdaOperator{\AgdaFunction{+}}\AgdaSpace{}%
\AgdaGeneralizable{p}\AgdaSpace{}%
\AgdaOperator{\AgdaFunction{≈}}\AgdaSpace{}%
\AgdaGeneralizable{r}\AgdaSpace{}%
\AgdaSymbol{→}\AgdaSpace{}%
\AgdaDatatype{P}\AgdaSpace{}%
\AgdaGeneralizable{r}\<%
\\
%
\>[2]\AgdaOperator{\AgdaFunction{\AgdaUnderscore{}⊝ₚ\AgdaUnderscore{}}}\AgdaSpace{}%
\AgdaSymbol{:}\AgdaSpace{}%
\AgdaSymbol{(}\AgdaBound{i}\AgdaSpace{}%
\AgdaSymbol{:}\AgdaSpace{}%
\AgdaDatatype{P}\AgdaSpace{}%
\AgdaGeneralizable{r}\AgdaSymbol{)}\AgdaSpace{}%
\AgdaSymbol{(}\AgdaBound{j}\AgdaSpace{}%
\AgdaSymbol{:}\AgdaSpace{}%
\AgdaDatatype{P}\AgdaSpace{}%
\AgdaGeneralizable{s}\AgdaSymbol{)}\AgdaSpace{}%
\AgdaSymbol{(}\AgdaBound{su}\AgdaSpace{}%
\AgdaSymbol{:}\AgdaSpace{}%
\AgdaOperator{\AgdaFunction{suc}}\AgdaSpace{}%
\AgdaGeneralizable{p}\AgdaSpace{}%
\AgdaOperator{\AgdaFunction{≈}}\AgdaSpace{}%
\AgdaGeneralizable{u}\AgdaSymbol{)}\AgdaSpace{}%
\AgdaSymbol{(}\AgdaBound{sp}\AgdaSpace{}%
\AgdaSymbol{:}\AgdaSpace{}%
\AgdaGeneralizable{s}\AgdaSpace{}%
\AgdaOperator{\AgdaFunction{+}}\AgdaSpace{}%
\AgdaGeneralizable{p}\AgdaSpace{}%
\AgdaOperator{\AgdaFunction{≈}}\AgdaSpace{}%
\AgdaGeneralizable{r}\AgdaSymbol{)}\AgdaSpace{}%
\AgdaSymbol{→}\AgdaSpace{}%
\AgdaRecord{Dec}\AgdaSpace{}%
\AgdaSymbol{(}\AgdaFunction{∃}\AgdaSpace{}%
\AgdaSymbol{λ}\AgdaSpace{}%
\AgdaBound{k}\AgdaSpace{}%
\AgdaSymbol{→}\AgdaSpace{}%
\AgdaSymbol{(}\AgdaBound{j}\AgdaSpace{}%
\AgdaOperator{\AgdaFunction{⊕ₚ}}\AgdaSpace{}%
\AgdaBound{k}\AgdaSymbol{)}\AgdaSpace{}%
\AgdaBound{su}\AgdaSpace{}%
\AgdaBound{sp}\AgdaSpace{}%
\AgdaOperator{\AgdaDatatype{≡}}\AgdaSpace{}%
\AgdaBound{i}\AgdaSymbol{)}\<%
\end{code}}
\end{mathpar}
The implementations of \AF{⊕ₚ} and \AF{⊝ₚ} simply apply \AF{⊕} and \AF{⊝}.
In the \AF{⊝} case a little plumbing is required when constructing the
proof of (non-)existence of the inverse.
\begin{code}[hide]%
%
\>[2]\AgdaSymbol{(}\AgdaBound{i}\AgdaSpace{}%
\AgdaOperator{\AgdaFunction{⊕ₚ}}\AgdaSpace{}%
\AgdaBound{j}\AgdaSymbol{)}\AgdaSpace{}%
\AgdaInductiveConstructor{[]}\AgdaSpace{}%
\AgdaInductiveConstructor{[]}\AgdaSpace{}%
\AgdaSymbol{=}\AgdaSpace{}%
\AgdaBound{j}\<%
\\
%
\>[2]\AgdaSymbol{((}\AgdaBound{i}\AgdaSpace{}%
\AgdaOperator{\AgdaInductiveConstructor{∷}}\AgdaSpace{}%
\AgdaBound{is}\AgdaSymbol{)}\AgdaSpace{}%
\AgdaOperator{\AgdaFunction{⊕ₚ}}\AgdaSpace{}%
\AgdaSymbol{(}\AgdaBound{j}\AgdaSpace{}%
\AgdaOperator{\AgdaInductiveConstructor{∷}}\AgdaSpace{}%
\AgdaBound{js}\AgdaSymbol{))}\AgdaSpace{}%
\AgdaSymbol{(}\AgdaInductiveConstructor{cons}\AgdaSpace{}%
\AgdaSymbol{⦃}\AgdaSpace{}%
\AgdaInductiveConstructor{refl}\AgdaSpace{}%
\AgdaSymbol{⦄}\AgdaSpace{}%
\AgdaSymbol{⦃}\AgdaSpace{}%
\AgdaBound{sp}\AgdaSpace{}%
\AgdaSymbol{⦄)}\AgdaSpace{}%
\AgdaSymbol{(}\AgdaInductiveConstructor{cons}\AgdaSpace{}%
\AgdaSymbol{⦃}\AgdaSpace{}%
\AgdaInductiveConstructor{refl}\AgdaSpace{}%
\AgdaSymbol{⦄}\AgdaSpace{}%
\AgdaSymbol{⦃}\AgdaSpace{}%
\AgdaBound{s+p}\AgdaSpace{}%
\AgdaSymbol{⦄)}\<%
\\
\>[2][@{}l@{\AgdaIndent{0}}]%
\>[4]\AgdaSymbol{=}\AgdaSpace{}%
\AgdaSymbol{(}\AgdaBound{i}\AgdaSpace{}%
\AgdaOperator{\AgdaFunction{⊕}}\AgdaSpace{}%
\AgdaBound{j}\AgdaSymbol{)}\AgdaSpace{}%
\AgdaOperator{\AgdaInductiveConstructor{∷}}\AgdaSpace{}%
\AgdaSymbol{(}\AgdaBound{is}\AgdaSpace{}%
\AgdaOperator{\AgdaFunction{⊕ₚ}}\AgdaSpace{}%
\AgdaBound{js}\AgdaSymbol{)}\AgdaSpace{}%
\AgdaBound{sp}\AgdaSpace{}%
\AgdaBound{s+p}\<%
\\
%
\\[\AgdaEmptyExtraSkip]%
%
\>[2]\AgdaSymbol{(}\AgdaInductiveConstructor{[]}\AgdaSpace{}%
\AgdaOperator{\AgdaFunction{⊝ₚ}}\AgdaSpace{}%
\AgdaBound{j}\AgdaSymbol{)}\AgdaSpace{}%
\AgdaInductiveConstructor{[]}\AgdaSpace{}%
\AgdaInductiveConstructor{[]}\AgdaSpace{}%
\AgdaSymbol{=}\AgdaSpace{}%
\AgdaInductiveConstructor{yes}\AgdaSpace{}%
\AgdaSymbol{(}\AgdaInductiveConstructor{[]}\AgdaSpace{}%
\AgdaOperator{\AgdaInductiveConstructor{,}}\AgdaSpace{}%
\AgdaInductiveConstructor{refl}\AgdaSymbol{)}\<%
\\
%
\>[2]\AgdaSymbol{((}\AgdaBound{i}\AgdaSpace{}%
\AgdaOperator{\AgdaInductiveConstructor{∷}}%
\>[1192I]\AgdaBound{is}\AgdaSymbol{)}\AgdaSpace{}%
\AgdaOperator{\AgdaFunction{⊝ₚ}}\AgdaSpace{}%
\AgdaSymbol{(}\AgdaBound{j}\AgdaSpace{}%
\AgdaOperator{\AgdaInductiveConstructor{∷}}\AgdaSpace{}%
\AgdaBound{js}\AgdaSymbol{))}\AgdaSpace{}%
\AgdaSymbol{(}\AgdaInductiveConstructor{cons}\AgdaSpace{}%
\AgdaSymbol{⦃}\AgdaSpace{}%
\AgdaInductiveConstructor{refl}\AgdaSpace{}%
\AgdaSymbol{⦄}\AgdaSpace{}%
\AgdaSymbol{⦃}\AgdaSpace{}%
\AgdaBound{sp}\AgdaSpace{}%
\AgdaSymbol{⦄)}\AgdaSpace{}%
\AgdaSymbol{(}\AgdaInductiveConstructor{cons}\AgdaSpace{}%
\AgdaSymbol{⦃}\AgdaSpace{}%
\AgdaInductiveConstructor{refl}\AgdaSpace{}%
\AgdaSymbol{⦄}\AgdaSpace{}%
\AgdaSymbol{⦃}\AgdaSpace{}%
\AgdaBound{s+p}\AgdaSpace{}%
\AgdaSymbol{⦄)}\<%
\\
\>[.][@{}l@{}]\<[1192I]%
\>[8]\AgdaKeyword{with}\AgdaSpace{}%
\AgdaBound{i}\AgdaSpace{}%
\AgdaOperator{\AgdaFunction{⊝}}\AgdaSpace{}%
\AgdaBound{j}\<%
\\
%
\>[2]\AgdaSymbol{...}\AgdaSpace{}%
\AgdaSymbol{|}\AgdaSpace{}%
\AgdaInductiveConstructor{no}\AgdaSpace{}%
\AgdaBound{¬p}\AgdaSpace{}%
\AgdaSymbol{=}\AgdaSpace{}%
\AgdaInductiveConstructor{no}\AgdaSpace{}%
\AgdaSymbol{λ}\AgdaSpace{}%
\AgdaSymbol{\{}\AgdaSpace{}%
\AgdaSymbol{((}\AgdaBound{k}\AgdaSpace{}%
\AgdaOperator{\AgdaInductiveConstructor{∷}}\AgdaSpace{}%
\AgdaSymbol{\AgdaUnderscore{})}\AgdaSpace{}%
\AgdaOperator{\AgdaInductiveConstructor{,}}\AgdaSpace{}%
\AgdaInductiveConstructor{refl}\AgdaSymbol{)}\AgdaSpace{}%
\AgdaSymbol{→}\AgdaSpace{}%
\AgdaBound{¬p}\AgdaSpace{}%
\AgdaSymbol{(}\AgdaBound{k}\AgdaSpace{}%
\AgdaOperator{\AgdaInductiveConstructor{,}}\AgdaSpace{}%
\AgdaInductiveConstructor{refl}\AgdaSymbol{)}\AgdaSpace{}%
\AgdaSymbol{\}}\<%
\\
%
\>[2]\AgdaSymbol{...}\AgdaSpace{}%
\AgdaSymbol{|}\AgdaSpace{}%
\AgdaInductiveConstructor{yes}\AgdaSpace{}%
\AgdaSymbol{(}\AgdaBound{k}\AgdaSpace{}%
\AgdaOperator{\AgdaInductiveConstructor{,}}\AgdaSpace{}%
\AgdaBound{p}\AgdaSymbol{)}\AgdaSpace{}%
\AgdaKeyword{with}\AgdaSpace{}%
\AgdaSymbol{(}\AgdaBound{is}\AgdaSpace{}%
\AgdaOperator{\AgdaFunction{⊝ₚ}}\AgdaSpace{}%
\AgdaBound{js}\AgdaSymbol{)}\AgdaSpace{}%
\AgdaBound{sp}\AgdaSpace{}%
\AgdaBound{s+p}\<%
\\
%
\>[2]\AgdaSymbol{...}\AgdaSpace{}%
\AgdaSymbol{|}\AgdaSpace{}%
\AgdaInductiveConstructor{no}\AgdaSpace{}%
\AgdaBound{¬q}\AgdaSpace{}%
\AgdaSymbol{=}\AgdaSpace{}%
\AgdaInductiveConstructor{no}\AgdaSpace{}%
\AgdaSymbol{λ}\AgdaSpace{}%
\AgdaSymbol{\{}\AgdaSpace{}%
\AgdaSymbol{((\AgdaUnderscore{}}\AgdaSpace{}%
\AgdaOperator{\AgdaInductiveConstructor{∷}}\AgdaSpace{}%
\AgdaBound{xs}\AgdaSymbol{)}\AgdaSpace{}%
\AgdaOperator{\AgdaInductiveConstructor{,}}\AgdaSpace{}%
\AgdaInductiveConstructor{refl}\AgdaSymbol{)}\AgdaSpace{}%
\AgdaSymbol{→}\AgdaSpace{}%
\AgdaBound{¬q}\AgdaSpace{}%
\AgdaSymbol{(}\AgdaBound{xs}\AgdaSpace{}%
\AgdaOperator{\AgdaInductiveConstructor{,}}\AgdaSpace{}%
\AgdaInductiveConstructor{refl}\AgdaSymbol{)}\AgdaSpace{}%
\AgdaSymbol{\}}\<%
\\
%
\>[2]\AgdaSymbol{...}\AgdaSpace{}%
\AgdaSymbol{|}\AgdaSpace{}%
\AgdaInductiveConstructor{yes}\AgdaSpace{}%
\AgdaSymbol{(}\AgdaBound{ks}\AgdaSpace{}%
\AgdaOperator{\AgdaInductiveConstructor{,}}\AgdaSpace{}%
\AgdaBound{q}\AgdaSymbol{)}\AgdaSpace{}%
\AgdaSymbol{=}\AgdaSpace{}%
\AgdaInductiveConstructor{yes}\AgdaSpace{}%
\AgdaSymbol{(}\AgdaBound{k}\AgdaSpace{}%
\AgdaOperator{\AgdaInductiveConstructor{∷}}\AgdaSpace{}%
\AgdaBound{ks}\AgdaSpace{}%
\AgdaOperator{\AgdaInductiveConstructor{,}}\AgdaSpace{}%
\AgdaFunction{cong₂}\AgdaSpace{}%
\AgdaOperator{\AgdaInductiveConstructor{\AgdaUnderscore{}∷\AgdaUnderscore{}}}\AgdaSpace{}%
\AgdaBound{p}\AgdaSpace{}%
\AgdaBound{q}\AgdaSymbol{)}\<%
\end{code}

Generalised \AF{slide} looks very similar to its 1-dimensional
counterpart, except that \AF{⊕} is replaced with \AF{⊕ₚ}
We also introduce a section of \AF{slide} that we call \AF{backslide}.
It embeds a $(1+p)$-dimensional array into a $(s+p)$-dimensional
one at the offset $i$ using \AB{def} to fill the outer region.
\begin{mathpar}
\codeblock{\begin{code}%
%
\>[2]\AgdaFunction{slide}\AgdaSpace{}%
\AgdaSymbol{:}\AgdaSpace{}%
\AgdaDatatype{P}\AgdaSpace{}%
\AgdaGeneralizable{s}\AgdaSpace{}%
\AgdaSymbol{→}\AgdaSpace{}%
\AgdaGeneralizable{s}\AgdaSpace{}%
\AgdaOperator{\AgdaFunction{+}}\AgdaSpace{}%
\AgdaGeneralizable{p}\AgdaSpace{}%
\AgdaOperator{\AgdaFunction{≈}}\AgdaSpace{}%
\AgdaGeneralizable{r}\AgdaSpace{}%
\AgdaSymbol{→}\AgdaSpace{}%
\AgdaFunction{Ar}\AgdaSpace{}%
\AgdaGeneralizable{r}\AgdaSpace{}%
\AgdaGeneralizable{X}\AgdaSpace{}%
\AgdaSymbol{→}\AgdaSpace{}%
\AgdaOperator{\AgdaFunction{suc}}\AgdaSpace{}%
\AgdaGeneralizable{p}\AgdaSpace{}%
\AgdaOperator{\AgdaFunction{≈}}\AgdaSpace{}%
\AgdaGeneralizable{u}\AgdaSpace{}%
\AgdaSymbol{→}\AgdaSpace{}%
\AgdaFunction{Ar}\AgdaSpace{}%
\AgdaGeneralizable{u}\AgdaSpace{}%
\AgdaGeneralizable{X}\<%
\\
%
\>[2]\AgdaFunction{slide}\AgdaSpace{}%
\AgdaBound{i}\AgdaSpace{}%
\AgdaBound{pl}\AgdaSpace{}%
\AgdaBound{a}\AgdaSpace{}%
\AgdaBound{su}\AgdaSpace{}%
\AgdaBound{j}\AgdaSpace{}%
\AgdaSymbol{=}\AgdaSpace{}%
\AgdaBound{a}\AgdaSpace{}%
\AgdaSymbol{((}\AgdaBound{i}\AgdaSpace{}%
\AgdaOperator{\AgdaFunction{⊕ₚ}}\AgdaSpace{}%
\AgdaBound{j}\AgdaSymbol{)}\AgdaSpace{}%
\AgdaBound{su}\AgdaSpace{}%
\AgdaBound{pl}\AgdaSymbol{)}\<%
\\
%
\\[\AgdaEmptyExtraSkip]%
%
\>[2]\AgdaFunction{backslide}\AgdaSpace{}%
\AgdaSymbol{:}\AgdaSpace{}%
\AgdaDatatype{P}\AgdaSpace{}%
\AgdaGeneralizable{s}\AgdaSpace{}%
\AgdaSymbol{→}\AgdaSpace{}%
\AgdaFunction{Ar}\AgdaSpace{}%
\AgdaGeneralizable{u}\AgdaSpace{}%
\AgdaGeneralizable{X}\AgdaSpace{}%
\AgdaSymbol{→}\AgdaSpace{}%
\AgdaOperator{\AgdaFunction{suc}}\AgdaSpace{}%
\AgdaGeneralizable{p}\AgdaSpace{}%
\AgdaOperator{\AgdaFunction{≈}}\AgdaSpace{}%
\AgdaGeneralizable{u}\AgdaSpace{}%
\AgdaSymbol{→}\AgdaSpace{}%
\AgdaSymbol{(}\AgdaBound{def}\AgdaSpace{}%
\AgdaSymbol{:}\AgdaSpace{}%
\AgdaGeneralizable{X}\AgdaSymbol{)}\AgdaSpace{}%
\AgdaSymbol{→}\AgdaSpace{}%
\AgdaGeneralizable{s}\AgdaSpace{}%
\AgdaOperator{\AgdaFunction{+}}\AgdaSpace{}%
\AgdaGeneralizable{p}\AgdaSpace{}%
\AgdaOperator{\AgdaFunction{≈}}\AgdaSpace{}%
\AgdaGeneralizable{r}\AgdaSpace{}%
\AgdaSymbol{→}\AgdaSpace{}%
\AgdaFunction{Ar}\AgdaSpace{}%
\AgdaGeneralizable{r}\AgdaSpace{}%
\AgdaGeneralizable{X}\<%
\\
%
\>[2]\AgdaFunction{backslide}\AgdaSpace{}%
\AgdaBound{i}\AgdaSpace{}%
\AgdaBound{a}\AgdaSpace{}%
\AgdaBound{su}\AgdaSpace{}%
\AgdaBound{def}\AgdaSpace{}%
\AgdaBound{pl}\AgdaSpace{}%
\AgdaBound{j}\AgdaSpace{}%
\AgdaKeyword{with}\AgdaSpace{}%
\AgdaSymbol{((}\AgdaBound{j}\AgdaSpace{}%
\AgdaOperator{\AgdaFunction{⊝ₚ}}\AgdaSpace{}%
\AgdaBound{i}\AgdaSymbol{)}\AgdaSpace{}%
\AgdaBound{su}\AgdaSpace{}%
\AgdaBound{pl}\AgdaSymbol{)}\<%
\\
%
\>[2]\AgdaSymbol{...}\AgdaSpace{}%
\AgdaSymbol{|}\AgdaSpace{}%
\AgdaInductiveConstructor{yes}\AgdaSpace{}%
\AgdaSymbol{(}\AgdaBound{k}\AgdaSpace{}%
\AgdaOperator{\AgdaInductiveConstructor{,}}\AgdaSpace{}%
\AgdaSymbol{\AgdaUnderscore{})}%
\>[21]\AgdaSymbol{=}\AgdaSpace{}%
\AgdaBound{a}\AgdaSpace{}%
\AgdaBound{k}\<%
\\
%
\>[2]\AgdaCatchallClause{\AgdaSymbol{...}}\AgdaSpace{}%
\AgdaCatchallClause{\AgdaSymbol{|}}\AgdaSpace{}%
\AgdaCatchallClause{\AgdaSymbol{\AgdaUnderscore{}}}%
\>[21]\AgdaSymbol{=}\AgdaSpace{}%
\AgdaBound{def}\<%
\end{code}}
\end{mathpar}

\paragraph{Remark on indexing} We would like to address a general remark that
is often made by functional programmers that index-oriented definitions such as
\AF{slide} and \AF{backslide} should be replaced by some construction that use
algebraic data types.  While this is of course a matter of taste, here are
important points that justify our choice. Firstly, array computations that use
explicit indices are easier to compile into efficient code. At runtime, arrays
will be represented as flat regions of memory without cons cells or other
pointer-connected structures. Index computations will be turned into offset
computations that are efficient on most architectures.  Secondly, many
rank-polymorphic operations on arrays are easier to express via index
manipulation (our indices have non-trivial structure) rather than via
traversals of algebraic data structures.  For example, consider a data
structure for a rank-polymorphic array similar to \AD{Ar}.  One needs something
like a free monad over a \AD{Vec} type, which can be easily defined.  Now,
consider defining a generalised transpose on such representation.  Transpose of
an \AD{Ar} array is simply a selection on a reversed index: λ ix → a
(\AF{reverse} ix). In case of free monads, this is a significantly more
complicated recursive expression.  Finally, when arrays are
functions, fusion equalities (\eg{} map f ∘ map g $\cong$ map (f ∘ g))
come for free through normalisation, which makes formal reasoning easier.





\subsection{CNN primitives\label{sec:ar-cnn-prim}}
Here we implement CNN-specific primitives that are needed for our running example.
All these primitives operate on arrays of reals.  We use builtin Agda floats in
the rest of the section that we refer to as \AD{ℝ}.  The only reason for this
is the ability to evaluate our specification with concrete values.
Later we are going to abstract over concrete implementation of \AD{ℝ}.

Generalised convolution is given by \AF{conv}, and it is almost identical to its
1-dimensional counterpart (except it uses \AF{slide} instead of \AF{slide₁}).
The \AF{mconv} runs $u$ \AF{conv}s (conceptually in parallel) and then it adds a
corresponding bias from the array $b$ (of shape $u$) to each convolution.
\begin{code}[hide]%
\>[0]\AgdaKeyword{module}\AgdaSpace{}%
\AgdaModule{CNN}\AgdaSpace{}%
\AgdaKeyword{where}\<%
\\
\>[0][@{}l@{\AgdaIndent{0}}]%
\>[2]\AgdaKeyword{open}\AgdaSpace{}%
\AgdaKeyword{import}\AgdaSpace{}%
\AgdaModule{Data.Nat}\AgdaSpace{}%
\AgdaSymbol{as}\AgdaSpace{}%
\AgdaModule{ℕ}\AgdaSpace{}%
\AgdaKeyword{using}\AgdaSpace{}%
\AgdaSymbol{(}\AgdaDatatype{ℕ}\AgdaSymbol{)}\<%
\\
%
\>[2]\AgdaKeyword{open}\AgdaSpace{}%
\AgdaKeyword{import}\AgdaSpace{}%
\AgdaModule{Data.Float}\AgdaSpace{}%
\AgdaSymbol{as}\AgdaSpace{}%
\AgdaModule{F}\AgdaSpace{}%
\AgdaKeyword{using}\AgdaSpace{}%
\AgdaSymbol{(}\AgdaPrimitive{\AgdaUnderscore{}+\AgdaUnderscore{}}\AgdaSymbol{;}\AgdaSpace{}%
\AgdaPrimitive{\AgdaUnderscore{}*\AgdaUnderscore{}}\AgdaSymbol{;}\AgdaSpace{}%
\AgdaPrimitive{\AgdaUnderscore{}÷\AgdaUnderscore{}}\AgdaSymbol{;}\AgdaSpace{}%
\AgdaPrimitive{e\textasciicircum{}\AgdaUnderscore{}}\AgdaSymbol{;}\AgdaSpace{}%
\AgdaPrimitive{-\AgdaUnderscore{}}\AgdaSymbol{;}\AgdaSpace{}%
\AgdaPrimitive{fromℕ}\AgdaSymbol{)}\AgdaSpace{}%
\AgdaKeyword{renaming}\AgdaSpace{}%
\AgdaSymbol{(}\AgdaPostulate{Float}\AgdaSpace{}%
\AgdaSymbol{to}\AgdaSpace{}%
\AgdaPostulate{ℝ}\AgdaSymbol{)}\<%
\\
%
\>[2]\AgdaKeyword{open}\AgdaSpace{}%
\AgdaKeyword{import}\AgdaSpace{}%
\AgdaModule{Data.Product}\AgdaSpace{}%
\AgdaSymbol{as}\AgdaSpace{}%
\AgdaModule{Prod}\AgdaSpace{}%
\AgdaKeyword{using}\AgdaSpace{}%
\AgdaSymbol{()}\<%
\\
%
\>[2]\AgdaKeyword{open}\AgdaSpace{}%
\AgdaKeyword{import}\AgdaSpace{}%
\AgdaModule{Data.Fin}\AgdaSpace{}%
\AgdaSymbol{as}\AgdaSpace{}%
\AgdaModule{F}\AgdaSpace{}%
\AgdaKeyword{using}\AgdaSpace{}%
\AgdaSymbol{(}\AgdaInductiveConstructor{zero}\AgdaSymbol{;}\AgdaSpace{}%
\AgdaInductiveConstructor{suc}\AgdaSymbol{;}\AgdaSpace{}%
\AgdaDatatype{Fin}\AgdaSymbol{;}\AgdaSpace{}%
\AgdaFunction{combine}\AgdaSymbol{;}\AgdaSpace{}%
\AgdaFunction{remQuot}\AgdaSymbol{;}\AgdaSpace{}%
\AgdaFunction{fromℕ<}\AgdaSymbol{;}\AgdaSpace{}%
\AgdaFunction{inject+}\AgdaSymbol{;}\AgdaSpace{}%
\AgdaFunction{splitAt}\AgdaSymbol{)}\<%
\\
%
\>[2]\AgdaKeyword{open}\AgdaSpace{}%
\AgdaModule{Array}\<%
\end{code}

\begin{code}%
%
\>[2]\AgdaFunction{conv}\AgdaSpace{}%
\AgdaSymbol{:}\AgdaSpace{}%
\AgdaGeneralizable{s}\AgdaSpace{}%
\AgdaOperator{\AgdaFunction{+}}\AgdaSpace{}%
\AgdaGeneralizable{p}\AgdaSpace{}%
\AgdaOperator{\AgdaFunction{≈}}\AgdaSpace{}%
\AgdaGeneralizable{r}\AgdaSpace{}%
\AgdaSymbol{→}\AgdaSpace{}%
\AgdaFunction{Ar}\AgdaSpace{}%
\AgdaGeneralizable{r}\AgdaSpace{}%
\AgdaPostulate{ℝ}\AgdaSpace{}%
\AgdaSymbol{→}\AgdaSpace{}%
\AgdaFunction{Ar}\AgdaSpace{}%
\AgdaGeneralizable{s}\AgdaSpace{}%
\AgdaPostulate{ℝ}\AgdaSpace{}%
\AgdaSymbol{→}\AgdaSpace{}%
\AgdaOperator{\AgdaFunction{suc}}\AgdaSpace{}%
\AgdaGeneralizable{p}\AgdaSpace{}%
\AgdaOperator{\AgdaFunction{≈}}\AgdaSpace{}%
\AgdaGeneralizable{u}\AgdaSpace{}%
\AgdaSymbol{→}\AgdaSpace{}%
\AgdaFunction{Ar}\AgdaSpace{}%
\AgdaGeneralizable{u}\AgdaSpace{}%
\AgdaPostulate{ℝ}\<%
\\
%
\>[2]\AgdaFunction{conv}\AgdaSpace{}%
\AgdaBound{sp}\AgdaSpace{}%
\AgdaBound{a}\AgdaSpace{}%
\AgdaBound{w}\AgdaSpace{}%
\AgdaBound{su}\AgdaSpace{}%
\AgdaSymbol{=}\AgdaSpace{}%
\AgdaFunction{sum}\AgdaSpace{}%
\AgdaSymbol{(}\AgdaFunction{zipWith}\AgdaSpace{}%
\AgdaOperator{\AgdaPrimitive{\AgdaUnderscore{}+\AgdaUnderscore{}}}\AgdaSymbol{)}\AgdaSpace{}%
\AgdaSymbol{(}\AgdaFunction{K}\AgdaSpace{}%
\AgdaNumber{0.0}\AgdaSymbol{)}\AgdaSpace{}%
\AgdaSymbol{λ}\AgdaSpace{}%
\AgdaBound{i}\AgdaSpace{}%
\AgdaSymbol{→}\AgdaSpace{}%
\AgdaFunction{map}\AgdaSpace{}%
\AgdaSymbol{(}\AgdaBound{w}\AgdaSpace{}%
\AgdaBound{i}\AgdaSpace{}%
\AgdaOperator{\AgdaPrimitive{*\AgdaUnderscore{}}}\AgdaSymbol{)}\AgdaSpace{}%
\AgdaSymbol{(}\AgdaFunction{slide}\AgdaSpace{}%
\AgdaBound{i}\AgdaSpace{}%
\AgdaBound{sp}\AgdaSpace{}%
\AgdaBound{a}\AgdaSpace{}%
\AgdaBound{su}\AgdaSymbol{)}\<%
\\
%
\\[\AgdaEmptyExtraSkip]%
%
\>[2]\AgdaFunction{mconv}\AgdaSpace{}%
\AgdaSymbol{:}\AgdaSpace{}%
\AgdaSymbol{⦃}\AgdaSpace{}%
\AgdaGeneralizable{s}\AgdaSpace{}%
\AgdaOperator{\AgdaFunction{+}}\AgdaSpace{}%
\AgdaGeneralizable{p}\AgdaSpace{}%
\AgdaOperator{\AgdaFunction{≈}}\AgdaSpace{}%
\AgdaGeneralizable{r}\AgdaSpace{}%
\AgdaSymbol{⦄}\AgdaSpace{}%
\AgdaSymbol{→}\AgdaSpace{}%
\AgdaFunction{Ar}\AgdaSpace{}%
\AgdaGeneralizable{r}\AgdaSpace{}%
\AgdaPostulate{ℝ}\AgdaSpace{}%
\AgdaSymbol{→}\AgdaSpace{}%
\AgdaFunction{Ar}\AgdaSpace{}%
\AgdaSymbol{(}\AgdaGeneralizable{u}\AgdaSpace{}%
\AgdaOperator{\AgdaFunction{⊗}}\AgdaSpace{}%
\AgdaGeneralizable{s}\AgdaSymbol{)}\AgdaSpace{}%
\AgdaPostulate{ℝ}\AgdaSpace{}%
\AgdaSymbol{→}\AgdaSpace{}%
\AgdaFunction{Ar}\AgdaSpace{}%
\AgdaGeneralizable{u}\AgdaSpace{}%
\AgdaPostulate{ℝ}\AgdaSpace{}%
\AgdaSymbol{→}\AgdaSpace{}%
\AgdaSymbol{⦃}\AgdaSpace{}%
\AgdaOperator{\AgdaFunction{suc}}\AgdaSpace{}%
\AgdaGeneralizable{p}\AgdaSpace{}%
\AgdaOperator{\AgdaFunction{≈}}\AgdaSpace{}%
\AgdaGeneralizable{q}\AgdaSpace{}%
\AgdaSymbol{⦄}\AgdaSpace{}%
\AgdaSymbol{→}\AgdaSpace{}%
\AgdaFunction{Ar}\AgdaSpace{}%
\AgdaSymbol{(}\AgdaGeneralizable{u}\AgdaSpace{}%
\AgdaOperator{\AgdaFunction{⊗}}\AgdaSpace{}%
\AgdaGeneralizable{q}\AgdaSymbol{)}\AgdaSpace{}%
\AgdaPostulate{ℝ}\<%
\\
%
\>[2]\AgdaFunction{mconv}\AgdaSpace{}%
\AgdaSymbol{⦃}\AgdaSpace{}%
\AgdaBound{sp}\AgdaSpace{}%
\AgdaSymbol{⦄}\AgdaSpace{}%
\AgdaBound{inp}\AgdaSpace{}%
\AgdaBound{w}\AgdaSpace{}%
\AgdaBound{b}\AgdaSpace{}%
\AgdaSymbol{⦃}\AgdaSpace{}%
\AgdaBound{su}\AgdaSpace{}%
\AgdaSymbol{⦄}\AgdaSpace{}%
\AgdaSymbol{=}\AgdaSpace{}%
\AgdaFunction{unnest}\AgdaSpace{}%
\AgdaSymbol{λ}\AgdaSpace{}%
\AgdaBound{i}\AgdaSpace{}%
\AgdaSymbol{→}\AgdaSpace{}%
\AgdaFunction{map}\AgdaSpace{}%
\AgdaSymbol{(}\AgdaBound{b}\AgdaSpace{}%
\AgdaBound{i}\AgdaSpace{}%
\AgdaOperator{\AgdaPrimitive{+\AgdaUnderscore{}}}\AgdaSymbol{)}\AgdaSpace{}%
\AgdaSymbol{(}\AgdaFunction{conv}\AgdaSpace{}%
\AgdaBound{sp}\AgdaSpace{}%
\AgdaBound{inp}\AgdaSpace{}%
\AgdaSymbol{(}\AgdaFunction{nest}\AgdaSpace{}%
\AgdaBound{w}\AgdaSpace{}%
\AgdaBound{i}\AgdaSymbol{)}\AgdaSpace{}%
\AgdaBound{su}\AgdaSymbol{)}\<%
\end{code}
The logistic function computes ${1}/(1 + e^{-x})$ for every element in the array.
\begin{mathpar}
\codeblock{\begin{code}%
%
\>[2]\AgdaFunction{logistic}\AgdaSpace{}%
\AgdaSymbol{:}\AgdaSpace{}%
\AgdaFunction{Ar}\AgdaSpace{}%
\AgdaGeneralizable{s}\AgdaSpace{}%
\AgdaPostulate{ℝ}\AgdaSpace{}%
\AgdaSymbol{→}\AgdaSpace{}%
\AgdaFunction{Ar}\AgdaSpace{}%
\AgdaGeneralizable{s}\AgdaSpace{}%
\AgdaPostulate{ℝ}\<%
\\
%
\>[2]\AgdaFunction{logistic}\AgdaSpace{}%
\AgdaSymbol{=}\AgdaSpace{}%
\AgdaFunction{map}\AgdaSpace{}%
\AgdaSymbol{λ}\AgdaSpace{}%
\AgdaBound{x}\AgdaSpace{}%
\AgdaSymbol{→}\AgdaSpace{}%
\AgdaNumber{1.0}\AgdaSpace{}%
\AgdaOperator{\AgdaPrimitive{÷}}\AgdaSpace{}%
\AgdaSymbol{(}\AgdaNumber{1.0}\AgdaSpace{}%
\AgdaOperator{\AgdaPrimitive{+}}\AgdaSpace{}%
\AgdaOperator{\AgdaPrimitive{e\textasciicircum{}}}\AgdaSpace{}%
\AgdaSymbol{(}\AgdaOperator{\AgdaPrimitive{-}}\AgdaSpace{}%
\AgdaBound{x}\AgdaSymbol{))}\<%
\end{code}}
\end{mathpar}

\paragraph{Average Pooling}
One of the steps of the machine learning algorithm is average pooling which
splits an array into sub-blocks and computes the average for every such
block.  Implementing this pattern generally is tricky as we have to
preserve the local neighbourhood within the blocks.  Working with a
blocked array would be inconvenient as the blocked shape
does not go well with \AF{slides}.  We solve this by introducing
blocked selections \AF{selb} into arrays of shape $(s * p)$ as well
as blocked array constructor \AF{imapb} that builds an array of
shape $(s * p)$ out of $s$ blocks of shape $p$.  Defining these
operations we require pairing and projections of the blocked indices
which is achieved by applying division and modulo operation on the
components.  The types of these operations are as follows:
\begin{mathpar}
\codeblock{\begin{code}%
%
\>[2]\AgdaFunction{ix-div}\AgdaSpace{}%
\AgdaSymbol{:}\AgdaSpace{}%
\AgdaDatatype{P}\AgdaSpace{}%
\AgdaGeneralizable{q}\AgdaSpace{}%
\AgdaSymbol{→}\AgdaSpace{}%
\AgdaGeneralizable{s}\AgdaSpace{}%
\AgdaOperator{\AgdaFunction{*}}\AgdaSpace{}%
\AgdaGeneralizable{p}\AgdaSpace{}%
\AgdaOperator{\AgdaFunction{≈}}\AgdaSpace{}%
\AgdaGeneralizable{q}\AgdaSpace{}%
\AgdaSymbol{→}\AgdaSpace{}%
\AgdaDatatype{P}\AgdaSpace{}%
\AgdaGeneralizable{s}\<%
\end{code}}
\and
\codeblock{\begin{code}%
%
\>[2]\AgdaFunction{ix-mod}\AgdaSpace{}%
\AgdaSymbol{:}\AgdaSpace{}%
\AgdaDatatype{P}\AgdaSpace{}%
\AgdaGeneralizable{q}\AgdaSpace{}%
\AgdaSymbol{→}\AgdaSpace{}%
\AgdaGeneralizable{s}\AgdaSpace{}%
\AgdaOperator{\AgdaFunction{*}}\AgdaSpace{}%
\AgdaGeneralizable{p}\AgdaSpace{}%
\AgdaOperator{\AgdaFunction{≈}}\AgdaSpace{}%
\AgdaGeneralizable{q}\AgdaSpace{}%
\AgdaSymbol{→}\AgdaSpace{}%
\AgdaDatatype{P}\AgdaSpace{}%
\AgdaGeneralizable{p}\<%
\end{code}}
\and
\codeblock{\begin{code}%
%
\>[2]\AgdaFunction{ix-combine}\AgdaSpace{}%
\AgdaSymbol{:}\AgdaSpace{}%
\AgdaDatatype{P}\AgdaSpace{}%
\AgdaGeneralizable{s}\AgdaSpace{}%
\AgdaSymbol{→}\AgdaSpace{}%
\AgdaDatatype{P}\AgdaSpace{}%
\AgdaGeneralizable{p}\AgdaSpace{}%
\AgdaSymbol{→}\AgdaSpace{}%
\AgdaGeneralizable{s}\AgdaSpace{}%
\AgdaOperator{\AgdaFunction{*}}\AgdaSpace{}%
\AgdaGeneralizable{p}\AgdaSpace{}%
\AgdaOperator{\AgdaFunction{≈}}\AgdaSpace{}%
\AgdaGeneralizable{q}\AgdaSpace{}%
\AgdaSymbol{→}\AgdaSpace{}%
\AgdaDatatype{P}\AgdaSpace{}%
\AgdaGeneralizable{q}\<%
\end{code}}
\end{mathpar}
\begin{code}[hide]%
%
\>[2]\AgdaFunction{ix-div}\AgdaSpace{}%
\AgdaBound{is}\AgdaSpace{}%
\AgdaInductiveConstructor{[]}\AgdaSpace{}%
\AgdaSymbol{=}\AgdaSpace{}%
\AgdaBound{is}\<%
\\
%
\>[2]\AgdaFunction{ix-div}\AgdaSpace{}%
\AgdaSymbol{(}\AgdaBound{i}\AgdaSpace{}%
\AgdaOperator{\AgdaInductiveConstructor{∷}}\AgdaSpace{}%
\AgdaBound{is}\AgdaSymbol{)}\AgdaSpace{}%
\AgdaSymbol{(}\AgdaInductiveConstructor{cons}\AgdaSpace{}%
\AgdaSymbol{⦃}\AgdaSpace{}%
\AgdaInductiveConstructor{refl}\AgdaSpace{}%
\AgdaSymbol{⦄}\AgdaSpace{}%
\AgdaSymbol{⦃}\AgdaSpace{}%
\AgdaBound{pf}\AgdaSpace{}%
\AgdaSymbol{⦄)}\<%
\\
\>[2][@{}l@{\AgdaIndent{0}}]%
\>[4]\AgdaSymbol{=}\AgdaSpace{}%
\AgdaField{Prod.proj₁}\AgdaSpace{}%
\AgdaSymbol{(}\AgdaFunction{F.remQuot}\AgdaSpace{}%
\AgdaSymbol{\AgdaUnderscore{}}\AgdaSpace{}%
\AgdaBound{i}\AgdaSymbol{)}\AgdaSpace{}%
\AgdaOperator{\AgdaInductiveConstructor{∷}}\AgdaSpace{}%
\AgdaFunction{ix-div}\AgdaSpace{}%
\AgdaBound{is}\AgdaSpace{}%
\AgdaBound{pf}\<%
\\
%
\\[\AgdaEmptyExtraSkip]%
%
\>[2]\AgdaFunction{ix-mod}\AgdaSpace{}%
\AgdaBound{is}\AgdaSpace{}%
\AgdaInductiveConstructor{[]}\AgdaSpace{}%
\AgdaSymbol{=}\AgdaSpace{}%
\AgdaBound{is}\<%
\\
%
\>[2]\AgdaFunction{ix-mod}\AgdaSpace{}%
\AgdaSymbol{(}\AgdaBound{i}\AgdaSpace{}%
\AgdaOperator{\AgdaInductiveConstructor{∷}}\AgdaSpace{}%
\AgdaBound{is}\AgdaSymbol{)}\AgdaSpace{}%
\AgdaSymbol{(}\AgdaInductiveConstructor{cons}\AgdaSpace{}%
\AgdaSymbol{\{}\AgdaArgument{m}\AgdaSpace{}%
\AgdaSymbol{=}\AgdaSpace{}%
\AgdaBound{m}\AgdaSymbol{\}}\AgdaSpace{}%
\AgdaSymbol{⦃}\AgdaSpace{}%
\AgdaInductiveConstructor{refl}\AgdaSpace{}%
\AgdaSymbol{⦄}\AgdaSpace{}%
\AgdaSymbol{⦃}\AgdaSpace{}%
\AgdaBound{pf}\AgdaSpace{}%
\AgdaSymbol{⦄)}\<%
\\
\>[2][@{}l@{\AgdaIndent{0}}]%
\>[4]\AgdaSymbol{=}\AgdaSpace{}%
\AgdaField{Prod.proj₂}\AgdaSpace{}%
\AgdaSymbol{(}\AgdaFunction{F.remQuot}\AgdaSpace{}%
\AgdaSymbol{\{}\AgdaBound{m}\AgdaSymbol{\}}\AgdaSpace{}%
\AgdaSymbol{\AgdaUnderscore{}}\AgdaSpace{}%
\AgdaBound{i}\AgdaSymbol{)}\AgdaSpace{}%
\AgdaOperator{\AgdaInductiveConstructor{∷}}\AgdaSpace{}%
\AgdaFunction{ix-mod}\AgdaSpace{}%
\AgdaBound{is}\AgdaSpace{}%
\AgdaBound{pf}\<%
\\
%
\\[\AgdaEmptyExtraSkip]%
%
\>[2]\AgdaFunction{ix-combine}\AgdaSpace{}%
\AgdaBound{i}\AgdaSpace{}%
\AgdaBound{j}\AgdaSpace{}%
\AgdaInductiveConstructor{[]}\AgdaSpace{}%
\AgdaSymbol{=}\AgdaSpace{}%
\AgdaBound{j}\<%
\\
%
\>[2]\AgdaFunction{ix-combine}\AgdaSpace{}%
\AgdaSymbol{(}\AgdaBound{i}\AgdaSpace{}%
\AgdaOperator{\AgdaInductiveConstructor{∷}}\AgdaSpace{}%
\AgdaBound{is}\AgdaSymbol{)}\AgdaSpace{}%
\AgdaSymbol{(}\AgdaBound{j}\AgdaSpace{}%
\AgdaOperator{\AgdaInductiveConstructor{∷}}\AgdaSpace{}%
\AgdaBound{js}\AgdaSymbol{)}\AgdaSpace{}%
\AgdaSymbol{(}\AgdaInductiveConstructor{cons}\AgdaSpace{}%
\AgdaSymbol{⦃}\AgdaSpace{}%
\AgdaInductiveConstructor{refl}\AgdaSpace{}%
\AgdaSymbol{⦄}\AgdaSpace{}%
\AgdaSymbol{⦃}\AgdaSpace{}%
\AgdaBound{ps}\AgdaSpace{}%
\AgdaSymbol{⦄)}\<%
\\
\>[2][@{}l@{\AgdaIndent{0}}]%
\>[4]\AgdaSymbol{=}\AgdaSpace{}%
\AgdaFunction{F.combine}\AgdaSpace{}%
\AgdaBound{i}\AgdaSpace{}%
\AgdaBound{j}\AgdaSpace{}%
\AgdaOperator{\AgdaInductiveConstructor{∷}}\AgdaSpace{}%
\AgdaFunction{ix-combine}\AgdaSpace{}%
\AgdaBound{is}\AgdaSpace{}%
\AgdaBound{js}\AgdaSpace{}%
\AgdaBound{ps}\<%
\end{code}
With these operations, definitions of \AF{selb} and \AF{imapb}
are:

\begin{mathpar}
\codeblock{\begin{code}%
%
\>[2]\AgdaFunction{selb}\AgdaSpace{}%
\AgdaSymbol{:}\AgdaSpace{}%
\AgdaFunction{Ar}\AgdaSpace{}%
\AgdaGeneralizable{q}\AgdaSpace{}%
\AgdaGeneralizable{X}\AgdaSpace{}%
\AgdaSymbol{→}\AgdaSpace{}%
\AgdaGeneralizable{p}\AgdaSpace{}%
\AgdaOperator{\AgdaFunction{*}}\AgdaSpace{}%
\AgdaGeneralizable{s}\AgdaSpace{}%
\AgdaOperator{\AgdaFunction{≈}}\AgdaSpace{}%
\AgdaGeneralizable{q}\AgdaSpace{}%
\AgdaSymbol{→}\AgdaSpace{}%
\AgdaDatatype{P}\AgdaSpace{}%
\AgdaGeneralizable{p}\AgdaSpace{}%
\AgdaSymbol{→}\AgdaSpace{}%
\AgdaFunction{Ar}\AgdaSpace{}%
\AgdaGeneralizable{s}\AgdaSpace{}%
\AgdaGeneralizable{X}\<%
\\
%
\>[2]\AgdaFunction{selb}\AgdaSpace{}%
\AgdaBound{a}\AgdaSpace{}%
\AgdaBound{p}\AgdaSpace{}%
\AgdaBound{i}\AgdaSpace{}%
\AgdaBound{j}\AgdaSpace{}%
\AgdaSymbol{=}\AgdaSpace{}%
\AgdaBound{a}\AgdaSpace{}%
\AgdaSymbol{(}\AgdaFunction{ix-combine}\AgdaSpace{}%
\AgdaBound{i}\AgdaSpace{}%
\AgdaBound{j}\AgdaSpace{}%
\AgdaBound{p}\AgdaSymbol{)}\<%
\end{code}}
\and
\codeblock{\begin{code}%
%
\>[2]\AgdaFunction{imapb}\AgdaSpace{}%
\AgdaSymbol{:}\AgdaSpace{}%
\AgdaFunction{Ar}\AgdaSpace{}%
\AgdaGeneralizable{s}\AgdaSpace{}%
\AgdaSymbol{(}\AgdaFunction{Ar}\AgdaSpace{}%
\AgdaGeneralizable{p}\AgdaSpace{}%
\AgdaGeneralizable{X}\AgdaSymbol{)}\AgdaSpace{}%
\AgdaSymbol{→}\AgdaSpace{}%
\AgdaGeneralizable{s}\AgdaSpace{}%
\AgdaOperator{\AgdaFunction{*}}\AgdaSpace{}%
\AgdaGeneralizable{p}\AgdaSpace{}%
\AgdaOperator{\AgdaFunction{≈}}\AgdaSpace{}%
\AgdaGeneralizable{q}\AgdaSpace{}%
\AgdaSymbol{→}\AgdaSpace{}%
\AgdaFunction{Ar}\AgdaSpace{}%
\AgdaGeneralizable{q}\AgdaSpace{}%
\AgdaGeneralizable{X}\<%
\\
%
\>[2]\AgdaFunction{imapb}\AgdaSpace{}%
\AgdaBound{a}\AgdaSpace{}%
\AgdaBound{p}\AgdaSpace{}%
\AgdaBound{i}\AgdaSpace{}%
\AgdaSymbol{=}\AgdaSpace{}%
\AgdaBound{a}\AgdaSpace{}%
\AgdaSymbol{(}\AgdaFunction{ix-div}\AgdaSpace{}%
\AgdaBound{i}\AgdaSpace{}%
\AgdaBound{p}\AgdaSymbol{)}\AgdaSpace{}%
\AgdaSymbol{(}\AgdaFunction{ix-mod}\AgdaSpace{}%
\AgdaBound{i}\AgdaSpace{}%
\AgdaBound{p}\AgdaSymbol{)}\<%
\end{code}}
\end{mathpar}
We define an average pooling that is specialised to
2-dimensional cases as needed per our running example.
\begin{mathpar}
\codeblock{\begin{code}%
%
\>[2]\AgdaFunction{avgp₂}\AgdaSpace{}%
\AgdaSymbol{:}\AgdaSpace{}%
\AgdaSymbol{(}\AgdaBound{m}\AgdaSpace{}%
\AgdaBound{n}\AgdaSpace{}%
\AgdaSymbol{:}\AgdaSpace{}%
\AgdaDatatype{ℕ}\AgdaSymbol{)}\AgdaSpace{}%
\AgdaSymbol{→}\AgdaSpace{}%
\AgdaFunction{Ar}\AgdaSpace{}%
\AgdaSymbol{(}\AgdaBound{m}\AgdaSpace{}%
\AgdaOperator{\AgdaPrimitive{ℕ.*}}\AgdaSpace{}%
\AgdaNumber{2}\AgdaSpace{}%
\AgdaOperator{\AgdaInductiveConstructor{∷}}\AgdaSpace{}%
\AgdaBound{n}\AgdaSpace{}%
\AgdaOperator{\AgdaPrimitive{ℕ.*}}\AgdaSpace{}%
\AgdaNumber{2}\AgdaSpace{}%
\AgdaOperator{\AgdaInductiveConstructor{∷}}\AgdaSpace{}%
\AgdaInductiveConstructor{[]}\AgdaSymbol{)}\AgdaSpace{}%
\AgdaPostulate{ℝ}\AgdaSpace{}%
\AgdaSymbol{→}\AgdaSpace{}%
\AgdaFunction{Ar}\AgdaSpace{}%
\AgdaSymbol{(}\AgdaBound{m}\AgdaSpace{}%
\AgdaOperator{\AgdaInductiveConstructor{∷}}\AgdaSpace{}%
\AgdaBound{n}\AgdaSpace{}%
\AgdaOperator{\AgdaInductiveConstructor{∷}}\AgdaSpace{}%
\AgdaInductiveConstructor{[]}\AgdaSymbol{)}\AgdaSpace{}%
\AgdaPostulate{ℝ}\<%
\\
%
\>[2]\AgdaFunction{avgp₂}\AgdaSpace{}%
\AgdaBound{m}\AgdaSpace{}%
\AgdaBound{n}\AgdaSpace{}%
\AgdaBound{a}\AgdaSpace{}%
\AgdaSymbol{=}\AgdaSpace{}%
\AgdaFunction{map}\AgdaSpace{}%
\AgdaSymbol{((}\AgdaOperator{\AgdaPrimitive{\AgdaUnderscore{}÷}}\AgdaSpace{}%
\AgdaPrimitive{fromℕ}\AgdaSpace{}%
\AgdaNumber{4}\AgdaSymbol{)}\AgdaSpace{}%
\AgdaOperator{\AgdaFunction{∘}}\AgdaSpace{}%
\AgdaFunction{sum}\AgdaSpace{}%
\AgdaOperator{\AgdaPrimitive{\AgdaUnderscore{}+\AgdaUnderscore{}}}\AgdaSpace{}%
\AgdaNumber{0.0}\AgdaSymbol{)}\AgdaSpace{}%
\AgdaSymbol{(}\AgdaFunction{selb}\AgdaSpace{}%
\AgdaBound{a}\AgdaSpace{}%
\AgdaFunction{it}\AgdaSymbol{)}\<%
\end{code}}
\end{mathpar}
Note that \AF{avgp₂} forces a programmer to provide explicit sizes
of the blocked array, and it will not admit arrays of shape such as
$2 * m \times 2 * n$, because $m * 2$ is not definitionally equal to $2 * m$.

With these primitives we implement a forward part of the CNN
as follows.  The \AB{inp} argument is the image of a hand-written digit, all
the other arguments are weights, and the function returns the 10-element vector
with probabilities which digit that is.  Note that type annotations in let are
purely for documentation --- Agda infers them automatically and these lines
can be removed.  Note also that all the \AF{mconv} applications do not require
explicit proofs as Agda can compute them from the shape information provided
in types.
%\begin{mathpar}
%\codeblock{
\begin{code}%
%
\>[2]\AgdaFunction{forward}%
\>[1733I]\AgdaSymbol{:}\AgdaSpace{}%
\AgdaSymbol{(}\AgdaBound{inp}%
\>[18]\AgdaSymbol{:}%
\>[21]\AgdaFunction{Ar}\AgdaSpace{}%
\AgdaSymbol{(}\AgdaNumber{28}\AgdaSpace{}%
\AgdaOperator{\AgdaInductiveConstructor{∷}}\AgdaSpace{}%
\AgdaNumber{28}\AgdaSpace{}%
\AgdaOperator{\AgdaInductiveConstructor{∷}}\AgdaSpace{}%
\AgdaInductiveConstructor{[]}\AgdaSymbol{)}\AgdaSpace{}%
\AgdaPostulate{ℝ}\AgdaSymbol{)}\AgdaSpace{}%
\AgdaSymbol{→}\AgdaSpace{}%
\AgdaSymbol{(}\AgdaBound{k₁}\AgdaSpace{}%
\AgdaSymbol{:}\AgdaSpace{}%
\AgdaFunction{Ar}\AgdaSpace{}%
\AgdaSymbol{(}\AgdaNumber{6}\AgdaSpace{}%
\AgdaOperator{\AgdaInductiveConstructor{∷}}\AgdaSpace{}%
\AgdaNumber{5}\AgdaSpace{}%
\AgdaOperator{\AgdaInductiveConstructor{∷}}\AgdaSpace{}%
\AgdaNumber{5}\AgdaSpace{}%
\AgdaOperator{\AgdaInductiveConstructor{∷}}\AgdaSpace{}%
\AgdaInductiveConstructor{[]}\AgdaSymbol{)}\AgdaSpace{}%
\AgdaPostulate{ℝ}\AgdaSymbol{)}\<%
\\
\>[.][@{}l@{}]\<[1733I]%
\>[10]\AgdaSymbol{→}\AgdaSpace{}%
\AgdaSymbol{(}\AgdaBound{b₁}%
\>[18]\AgdaSymbol{:}%
\>[21]\AgdaFunction{Ar}\AgdaSpace{}%
\AgdaSymbol{(}\AgdaNumber{6}%
\>[28]\AgdaOperator{\AgdaInductiveConstructor{∷}}\AgdaSpace{}%
\AgdaInductiveConstructor{[]}\AgdaSymbol{)}\AgdaSpace{}%
\AgdaPostulate{ℝ}\AgdaSymbol{)}%
\>[42]\AgdaSymbol{→}\AgdaSpace{}%
\AgdaSymbol{(}\AgdaBound{k₂}\AgdaSpace{}%
\AgdaSymbol{:}\AgdaSpace{}%
\AgdaFunction{Ar}\AgdaSpace{}%
\AgdaSymbol{(}\AgdaNumber{12}\AgdaSpace{}%
\AgdaOperator{\AgdaInductiveConstructor{∷}}\AgdaSpace{}%
\AgdaNumber{6}\AgdaSpace{}%
\AgdaOperator{\AgdaInductiveConstructor{∷}}\AgdaSpace{}%
\AgdaNumber{5}\AgdaSpace{}%
\AgdaOperator{\AgdaInductiveConstructor{∷}}\AgdaSpace{}%
\AgdaNumber{5}\AgdaSpace{}%
\AgdaOperator{\AgdaInductiveConstructor{∷}}\AgdaSpace{}%
\AgdaInductiveConstructor{[]}\AgdaSymbol{)}\AgdaSpace{}%
\AgdaPostulate{ℝ}\AgdaSymbol{)}\<%
\\
%
\>[10]\AgdaSymbol{→}\AgdaSpace{}%
\AgdaSymbol{(}\AgdaBound{b₂}%
\>[18]\AgdaSymbol{:}%
\>[21]\AgdaFunction{Ar}\AgdaSpace{}%
\AgdaSymbol{(}\AgdaNumber{12}\AgdaSpace{}%
\AgdaOperator{\AgdaInductiveConstructor{∷}}\AgdaSpace{}%
\AgdaInductiveConstructor{[]}\AgdaSymbol{)}\AgdaSpace{}%
\AgdaPostulate{ℝ}\AgdaSymbol{)}%
\>[42]\AgdaSymbol{→}\AgdaSpace{}%
\AgdaSymbol{(}\AgdaBound{fc}\AgdaSpace{}%
\AgdaSymbol{:}\AgdaSpace{}%
\AgdaFunction{Ar}\AgdaSpace{}%
\AgdaSymbol{(}\AgdaNumber{10}\AgdaSpace{}%
\AgdaOperator{\AgdaInductiveConstructor{∷}}\AgdaSpace{}%
\AgdaNumber{12}\AgdaSpace{}%
\AgdaOperator{\AgdaInductiveConstructor{∷}}\AgdaSpace{}%
\AgdaNumber{1}\AgdaSpace{}%
\AgdaOperator{\AgdaInductiveConstructor{∷}}\AgdaSpace{}%
\AgdaNumber{4}\AgdaSpace{}%
\AgdaOperator{\AgdaInductiveConstructor{∷}}\AgdaSpace{}%
\AgdaNumber{4}\AgdaSpace{}%
\AgdaOperator{\AgdaInductiveConstructor{∷}}\AgdaSpace{}%
\AgdaInductiveConstructor{[]}\AgdaSymbol{)}\AgdaSpace{}%
\AgdaPostulate{ℝ}\AgdaSymbol{)}\<%
\\
%
\>[10]\AgdaSymbol{→}\AgdaSpace{}%
\AgdaSymbol{(}\AgdaBound{b}%
\>[18]\AgdaSymbol{:}%
\>[21]\AgdaFunction{Ar}\AgdaSpace{}%
\AgdaSymbol{(}\AgdaNumber{10}\AgdaSpace{}%
\AgdaOperator{\AgdaInductiveConstructor{∷}}\AgdaSpace{}%
\AgdaInductiveConstructor{[]}\AgdaSymbol{)}\AgdaSpace{}%
\AgdaPostulate{ℝ}\AgdaSymbol{)}%
\>[42]\AgdaSymbol{→}\AgdaSpace{}%
\AgdaFunction{Ar}\AgdaSpace{}%
\AgdaSymbol{(}\AgdaNumber{10}\AgdaSpace{}%
\AgdaOperator{\AgdaInductiveConstructor{∷}}\AgdaSpace{}%
\AgdaNumber{1}\AgdaSpace{}%
\AgdaOperator{\AgdaInductiveConstructor{∷}}\AgdaSpace{}%
\AgdaNumber{1}\AgdaSpace{}%
\AgdaOperator{\AgdaInductiveConstructor{∷}}\AgdaSpace{}%
\AgdaNumber{1}\AgdaSpace{}%
\AgdaOperator{\AgdaInductiveConstructor{∷}}\AgdaSpace{}%
\AgdaNumber{1}\AgdaSpace{}%
\AgdaOperator{\AgdaInductiveConstructor{∷}}\AgdaSpace{}%
\AgdaInductiveConstructor{[]}\AgdaSymbol{)}\AgdaSpace{}%
\AgdaPostulate{ℝ}\<%
\\
%
\>[2]\AgdaFunction{forward}\AgdaSpace{}%
\AgdaBound{inp}\AgdaSpace{}%
\AgdaBound{k₁}\AgdaSpace{}%
\AgdaBound{b₁}\AgdaSpace{}%
\AgdaBound{k₂}\AgdaSpace{}%
\AgdaBound{b₂}\AgdaSpace{}%
\AgdaBound{fc}\AgdaSpace{}%
\AgdaBound{b}\AgdaSpace{}%
\AgdaSymbol{=}\AgdaSpace{}%
\AgdaKeyword{let}\<%
\\
\>[2][@{}l@{\AgdaIndent{0}}]%
\>[6]\AgdaBound{c₁}\AgdaSpace{}%
\AgdaSymbol{:}\AgdaSpace{}%
\AgdaFunction{Ar}\AgdaSpace{}%
\AgdaSymbol{(}\AgdaNumber{6}\AgdaSpace{}%
\AgdaOperator{\AgdaInductiveConstructor{∷}}\AgdaSpace{}%
\AgdaNumber{24}\AgdaSpace{}%
\AgdaOperator{\AgdaInductiveConstructor{∷}}\AgdaSpace{}%
\AgdaNumber{24}\AgdaSpace{}%
\AgdaOperator{\AgdaInductiveConstructor{∷}}\AgdaSpace{}%
\AgdaInductiveConstructor{[]}\AgdaSymbol{)}\AgdaSpace{}%
\AgdaPostulate{ℝ}\<%
\\
%
\>[6]\AgdaBound{c₁}\AgdaSpace{}%
\AgdaSymbol{=}\AgdaSpace{}%
\AgdaFunction{logistic}\AgdaSpace{}%
\AgdaOperator{\AgdaFunction{\$}}\AgdaSpace{}%
\AgdaFunction{mconv}\AgdaSpace{}%
\AgdaBound{inp}\AgdaSpace{}%
\AgdaBound{k₁}\AgdaSpace{}%
\AgdaBound{b₁}\<%
\\
%
\\[\AgdaEmptyExtraSkip]%
%
\>[6]\AgdaBound{s₁}\AgdaSpace{}%
\AgdaSymbol{:}\AgdaSpace{}%
\AgdaFunction{Ar}\AgdaSpace{}%
\AgdaSymbol{(}\AgdaNumber{6}\AgdaSpace{}%
\AgdaOperator{\AgdaInductiveConstructor{∷}}\AgdaSpace{}%
\AgdaNumber{12}\AgdaSpace{}%
\AgdaOperator{\AgdaInductiveConstructor{∷}}\AgdaSpace{}%
\AgdaNumber{12}\AgdaSpace{}%
\AgdaOperator{\AgdaInductiveConstructor{∷}}\AgdaSpace{}%
\AgdaInductiveConstructor{[]}\AgdaSymbol{)}\AgdaSpace{}%
\AgdaPostulate{ℝ}\<%
\\
%
\>[6]\AgdaBound{s₁}\AgdaSpace{}%
\AgdaSymbol{=}\AgdaSpace{}%
\AgdaFunction{unnest}\AgdaSpace{}%
\AgdaSymbol{\{}\AgdaArgument{s}\AgdaSpace{}%
\AgdaSymbol{=}\AgdaSpace{}%
\AgdaNumber{6}\AgdaSpace{}%
\AgdaOperator{\AgdaInductiveConstructor{∷}}\AgdaSpace{}%
\AgdaInductiveConstructor{[]}\AgdaSymbol{\}}\AgdaSpace{}%
\AgdaOperator{\AgdaFunction{\$}}\AgdaSpace{}%
\AgdaFunction{map}\AgdaSpace{}%
\AgdaSymbol{(}\AgdaFunction{avgp₂}\AgdaSpace{}%
\AgdaNumber{12}\AgdaSpace{}%
\AgdaNumber{12}\AgdaSymbol{)}\AgdaSpace{}%
\AgdaSymbol{(}\AgdaFunction{nest}\AgdaSpace{}%
\AgdaBound{c₁}\AgdaSymbol{)}\<%
\\
%
\\[\AgdaEmptyExtraSkip]%
%
\>[6]\AgdaBound{c₂}\AgdaSpace{}%
\AgdaSymbol{:}\AgdaSpace{}%
\AgdaFunction{Ar}\AgdaSpace{}%
\AgdaSymbol{(}\AgdaNumber{12}\AgdaSpace{}%
\AgdaOperator{\AgdaInductiveConstructor{∷}}\AgdaSpace{}%
\AgdaNumber{1}\AgdaSpace{}%
\AgdaOperator{\AgdaInductiveConstructor{∷}}\AgdaSpace{}%
\AgdaNumber{8}\AgdaSpace{}%
\AgdaOperator{\AgdaInductiveConstructor{∷}}\AgdaSpace{}%
\AgdaNumber{8}\AgdaSpace{}%
\AgdaOperator{\AgdaInductiveConstructor{∷}}\AgdaSpace{}%
\AgdaInductiveConstructor{[]}\AgdaSymbol{)}\AgdaSpace{}%
\AgdaPostulate{ℝ}\<%
\\
%
\>[6]\AgdaBound{c₂}\AgdaSpace{}%
\AgdaSymbol{=}\AgdaSpace{}%
\AgdaFunction{logistic}\AgdaSpace{}%
\AgdaOperator{\AgdaFunction{\$}}\AgdaSpace{}%
\AgdaFunction{mconv}%
\>[29]\AgdaBound{s₁}\AgdaSpace{}%
\AgdaBound{k₂}\AgdaSpace{}%
\AgdaBound{b₂}\<%
\\
%
\\[\AgdaEmptyExtraSkip]%
%
\>[6]\AgdaBound{s₂}\AgdaSpace{}%
\AgdaSymbol{:}\AgdaSpace{}%
\AgdaFunction{Ar}\AgdaSpace{}%
\AgdaSymbol{(}\AgdaNumber{12}\AgdaSpace{}%
\AgdaOperator{\AgdaInductiveConstructor{∷}}\AgdaSpace{}%
\AgdaNumber{1}\AgdaSpace{}%
\AgdaOperator{\AgdaInductiveConstructor{∷}}\AgdaSpace{}%
\AgdaNumber{4}\AgdaSpace{}%
\AgdaOperator{\AgdaInductiveConstructor{∷}}\AgdaSpace{}%
\AgdaNumber{4}\AgdaSpace{}%
\AgdaOperator{\AgdaInductiveConstructor{∷}}\AgdaSpace{}%
\AgdaInductiveConstructor{[]}\AgdaSymbol{)}\AgdaSpace{}%
\AgdaPostulate{ℝ}\<%
\\
%
\>[6]\AgdaBound{s₂}\AgdaSpace{}%
\AgdaSymbol{=}\AgdaSpace{}%
\AgdaFunction{unnest}\AgdaSpace{}%
\AgdaSymbol{\{}\AgdaArgument{s}\AgdaSpace{}%
\AgdaSymbol{=}\AgdaSpace{}%
\AgdaNumber{12}\AgdaSpace{}%
\AgdaOperator{\AgdaInductiveConstructor{∷}}\AgdaSpace{}%
\AgdaNumber{1}\AgdaSpace{}%
\AgdaOperator{\AgdaInductiveConstructor{∷}}\AgdaSpace{}%
\AgdaInductiveConstructor{[]}\AgdaSymbol{\}}\AgdaSpace{}%
\AgdaOperator{\AgdaFunction{\$}}\AgdaSpace{}%
\AgdaFunction{map}\AgdaSpace{}%
\AgdaSymbol{(}\AgdaFunction{avgp₂}\AgdaSpace{}%
\AgdaNumber{4}\AgdaSpace{}%
\AgdaNumber{4}\AgdaSymbol{)}\AgdaSpace{}%
\AgdaSymbol{(}\AgdaFunction{nest}\AgdaSpace{}%
\AgdaBound{c₂}\AgdaSymbol{)}\<%
\\
%
\\[\AgdaEmptyExtraSkip]%
%
\>[6]\AgdaBound{r}\AgdaSpace{}%
\AgdaSymbol{=}\AgdaSpace{}%
\AgdaFunction{logistic}\AgdaSpace{}%
\AgdaOperator{\AgdaFunction{\$}}\AgdaSpace{}%
\AgdaFunction{mconv}\AgdaSpace{}%
\AgdaBound{s₂}\AgdaSpace{}%
\AgdaBound{fc}\AgdaSpace{}%
\AgdaBound{b}\<%
\\
\>[2][@{}l@{\AgdaIndent{0}}]%
\>[4]\AgdaKeyword{in}\AgdaSpace{}%
\AgdaBound{r}\<%
\end{code}
%}
%\end{mathpar}


\begin{code}[hide]%
\>[0]\AgdaSymbol{\{-\#}\AgdaSpace{}%
\AgdaKeyword{OPTIONS}%
\>[13]\AgdaPragma{--backtracking-instance-search}\AgdaSpace{}%
\AgdaSymbol{\#-\}}\<%
\\
\>[0]\AgdaKeyword{open}\AgdaSpace{}%
\AgdaKeyword{import}\AgdaSpace{}%
\AgdaModule{Relation.Binary.PropositionalEquality}\<%
\\
\>[0]\AgdaKeyword{open}\AgdaSpace{}%
\AgdaKeyword{import}\AgdaSpace{}%
\AgdaModule{Relation.Nullary}\<%
\\
\>[0]\AgdaKeyword{open}\AgdaSpace{}%
\AgdaKeyword{import}\AgdaSpace{}%
\AgdaModule{Data.List}\AgdaSpace{}%
\AgdaKeyword{using}\AgdaSpace{}%
\AgdaSymbol{(}\AgdaDatatype{List}\AgdaSymbol{;}\AgdaSpace{}%
\AgdaInductiveConstructor{[]}\AgdaSymbol{;}\AgdaSpace{}%
\AgdaOperator{\AgdaInductiveConstructor{\AgdaUnderscore{}∷\AgdaUnderscore{}}}\AgdaSymbol{)}\<%
\\
\>[0]\AgdaKeyword{open}\AgdaSpace{}%
\AgdaKeyword{import}\AgdaSpace{}%
\AgdaModule{Data.Nat}\AgdaSpace{}%
\AgdaKeyword{using}\AgdaSpace{}%
\AgdaSymbol{(}\AgdaDatatype{ℕ}\AgdaSymbol{;}\AgdaSpace{}%
\AgdaInductiveConstructor{zero}\AgdaSymbol{;}\AgdaSpace{}%
\AgdaInductiveConstructor{suc}\AgdaSymbol{)}\<%
\\
\>[0]\AgdaKeyword{open}\AgdaSpace{}%
\AgdaKeyword{import}\AgdaSpace{}%
\AgdaModule{Data.Empty}\<%
\\
\>[0]\AgdaComment{--open\ import\ Function\ hiding\ (⟨\AgdaUnderscore{}⟩)}\<%
\\
%
\\[\AgdaEmptyExtraSkip]%
\>[0]\AgdaComment{--\ Our\ local\ files.}\<%
\\
\>[0]\AgdaKeyword{open}\AgdaSpace{}%
\AgdaKeyword{import}\AgdaSpace{}%
\AgdaModule{arrays}\<%
\\
\>[0]\AgdaKeyword{module}\AgdaSpace{}%
\AgdaModule{\AgdaUnderscore{}}\AgdaSpace{}%
\AgdaKeyword{where}\<%
\end{code}
\section{Embedded DSL}

Any implementation of automatic differentiation has to decide which operations
are supported.  Surely, it does not make sense to compute derivatives
of a function that opens a file.  This choice, no matter how it is implemented,
can be seen as a definition of an embedded language.
Once we accept to identify an embedded language, the idea of embedding it in a way
that facilitates extraction actually appears rather naturally and thus advances
the approach that we propose in this paper.

Coming back to our example, we have to choose the primitives that the embedded language
should support. They need to be sufficient to express AD as well as to define CNNs.
The main trade-off here is the choice of the level of abstraction of these primitives:
low-level primitives are easier to differentiate, but they make the overall expressions
more complex which also adds to the challenge of optimising code.
Making this choice is difficult and, most likely, requires quite some adjustment 
when striving for performance.
Here we see a key benefit of the approach we propose in this paper:
the use of a single framework for the embedding, the optimisation, and the extraction
makes the implementation comparatively small, allowing for quick adjustments in the
level of abstraction, code optimisation and its extraction.
We start with a pragmatic approach; we include the primitives that are either shared
by the model and the back-end
or that can be easily implemented in the back-end language.

It turns out that it is possible to choose the primitives in a way that the derivatives 
can be expressed in the very same embedded language. 
While this at first glance may just seem to be just a nice coincidence, it turns out
that this has several tangible benefits: high-order derivatives can be computed 
by the same transformation and we can share all optimisations between the code itself
and its derivatives.
\begin{code}[hide]%
\>[0]\AgdaKeyword{module}\AgdaSpace{}%
\AgdaModule{Lang}\AgdaSpace{}%
\AgdaKeyword{where}\<%
\\
\>[0][@{}l@{\AgdaIndent{0}}]%
\>[2]\AgdaComment{--open\ Array\ hiding\ (sum;\ slide;\ backslide)}\<%
\\
%
\>[2]\AgdaKeyword{open}\AgdaSpace{}%
\AgdaKeyword{import}\AgdaSpace{}%
\AgdaModule{Data.Nat}\AgdaSpace{}%
\AgdaKeyword{using}\AgdaSpace{}%
\AgdaSymbol{(}\AgdaDatatype{ℕ}\AgdaSymbol{;}\AgdaSpace{}%
\AgdaInductiveConstructor{zero}\AgdaSymbol{;}\AgdaSpace{}%
\AgdaInductiveConstructor{suc}\AgdaSymbol{)}\<%
\\
%
\>[2]\AgdaKeyword{infixl}\AgdaSpace{}%
\AgdaNumber{15}\AgdaSpace{}%
\AgdaOperator{\AgdaInductiveConstructor{\AgdaUnderscore{}▹\AgdaUnderscore{}}}\<%
\\
%
\>[2]\AgdaKeyword{module}\AgdaSpace{}%
\AgdaModule{Ar}\AgdaSpace{}%
\AgdaKeyword{where}\<%
\\
\>[2][@{}l@{\AgdaIndent{0}}]%
\>[4]\AgdaKeyword{open}\AgdaSpace{}%
\AgdaModule{Array}\AgdaSpace{}%
\AgdaKeyword{public}\<%
\\
%
\>[4]\AgdaKeyword{open}\AgdaSpace{}%
\AgdaModule{CNN}\AgdaSpace{}%
\AgdaKeyword{public}\<%
\\
%
\\[\AgdaEmptyExtraSkip]%
%
\>[2]\AgdaKeyword{open}\AgdaSpace{}%
\AgdaModule{Ar}\AgdaSpace{}%
\AgdaKeyword{hiding}\AgdaSpace{}%
\AgdaSymbol{(}\AgdaFunction{sum}\AgdaSymbol{;}\AgdaSpace{}%
\AgdaFunction{slide}\AgdaSymbol{;}\AgdaSpace{}%
\AgdaFunction{backslide}\AgdaSymbol{;}\AgdaSpace{}%
\AgdaFunction{imapb}\AgdaSymbol{;}\AgdaSpace{}%
\AgdaFunction{selb}\AgdaSymbol{;}\AgdaSpace{}%
\AgdaFunction{logistic}\AgdaSymbol{)}\<%
\\
\>[0]\<%
\end{code}

As we operate within a dependently-typed proof-assistant, we can easily make our
embedded language well-scoped and intrinsically typed (shaped in our case).  Our
context \AF{Ctx} is a snoc-list of shapes where each shape has a tag indicating whether
it is an index or an array.  We use de Bruijn variables given by the relation
\AF{\_∈\_} in the usual way.  We also define variables \AB{v₁}, \AB{v₂}, \etc{}
by iteratively applying \AC{vₛ} to \AC{v₀} (definition not shown).
\begin{mathpar}
\codeblock{\begin{code}%
\>[0][@{}l@{\AgdaIndent{1}}]%
\>[2]\AgdaKeyword{data}\AgdaSpace{}%
\AgdaDatatype{IS}\AgdaSpace{}%
\AgdaSymbol{:}\AgdaSpace{}%
\AgdaPrimitive{Set}\AgdaSpace{}%
\AgdaKeyword{where}\<%
\\
\>[2][@{}l@{\AgdaIndent{0}}]%
\>[4]\AgdaInductiveConstructor{ix}%
\>[8]\AgdaSymbol{:}\AgdaSpace{}%
\AgdaDatatype{S}\AgdaSpace{}%
\AgdaSymbol{→}\AgdaSpace{}%
\AgdaDatatype{IS}\<%
\\
%
\>[4]\AgdaInductiveConstructor{ar}%
\>[8]\AgdaSymbol{:}\AgdaSpace{}%
\AgdaDatatype{S}\AgdaSpace{}%
\AgdaSymbol{→}\AgdaSpace{}%
\AgdaDatatype{IS}\<%
\end{code}}
\and
\codeblock{\begin{code}%
%
\>[2]\AgdaKeyword{data}\AgdaSpace{}%
\AgdaDatatype{Ctx}\AgdaSpace{}%
\AgdaSymbol{:}\AgdaSpace{}%
\AgdaPrimitive{Set}\AgdaSpace{}%
\AgdaKeyword{where}\<%
\\
\>[2][@{}l@{\AgdaIndent{0}}]%
\>[4]\AgdaInductiveConstructor{ε}%
\>[9]\AgdaSymbol{:}\AgdaSpace{}%
\AgdaDatatype{Ctx}\<%
\\
%
\>[4]\AgdaOperator{\AgdaInductiveConstructor{\AgdaUnderscore{}▹\AgdaUnderscore{}}}%
\>[9]\AgdaSymbol{:}\AgdaSpace{}%
\AgdaDatatype{Ctx}\AgdaSpace{}%
\AgdaSymbol{→}\AgdaSpace{}%
\AgdaDatatype{IS}\AgdaSpace{}%
\AgdaSymbol{→}\AgdaSpace{}%
\AgdaDatatype{Ctx}\<%
\end{code}}
\and
\codeblock{\begin{code}[hide]%
%
\>[2]\AgdaKeyword{variable}\<%
\\
\>[2][@{}l@{\AgdaIndent{0}}]%
\>[4]\AgdaGeneralizable{Γ}\AgdaSpace{}%
\AgdaGeneralizable{Δ}\AgdaSpace{}%
\AgdaGeneralizable{Ξ}\AgdaSpace{}%
\AgdaGeneralizable{Ψ}\AgdaSpace{}%
\AgdaSymbol{:}\AgdaSpace{}%
\AgdaDatatype{Ctx}\<%
\\
%
\>[4]\AgdaGeneralizable{is}\AgdaSpace{}%
\AgdaGeneralizable{ip}\AgdaSpace{}%
\AgdaGeneralizable{iq}\AgdaSpace{}%
\AgdaGeneralizable{ir}\AgdaSpace{}%
\AgdaSymbol{:}\AgdaSpace{}%
\AgdaDatatype{IS}\<%
\end{code}
\begin{code}%
%
\>[2]\AgdaKeyword{data}\AgdaSpace{}%
\AgdaOperator{\AgdaDatatype{\AgdaUnderscore{}∈\AgdaUnderscore{}}}\AgdaSpace{}%
\AgdaSymbol{:}\AgdaSpace{}%
\AgdaDatatype{IS}\AgdaSpace{}%
\AgdaSymbol{→}\AgdaSpace{}%
\AgdaDatatype{Ctx}\AgdaSpace{}%
\AgdaSymbol{→}\AgdaSpace{}%
\AgdaPrimitive{Set}\AgdaSpace{}%
\AgdaKeyword{where}\<%
\\
\>[2][@{}l@{\AgdaIndent{0}}]%
\>[4]\AgdaInductiveConstructor{v₀}%
\>[8]\AgdaSymbol{:}\AgdaSpace{}%
\AgdaGeneralizable{is}\AgdaSpace{}%
\AgdaOperator{\AgdaDatatype{∈}}\AgdaSpace{}%
\AgdaSymbol{(}\AgdaGeneralizable{Γ}\AgdaSpace{}%
\AgdaOperator{\AgdaInductiveConstructor{▹}}\AgdaSpace{}%
\AgdaGeneralizable{is}\AgdaSymbol{)}\<%
\\
%
\>[4]\AgdaInductiveConstructor{vₛ}%
\>[8]\AgdaSymbol{:}\AgdaSpace{}%
\AgdaGeneralizable{is}\AgdaSpace{}%
\AgdaOperator{\AgdaDatatype{∈}}\AgdaSpace{}%
\AgdaGeneralizable{Γ}\AgdaSpace{}%
\AgdaSymbol{→}\AgdaSpace{}%
\AgdaGeneralizable{is}\AgdaSpace{}%
\AgdaOperator{\AgdaDatatype{∈}}\AgdaSpace{}%
\AgdaSymbol{(}\AgdaGeneralizable{Γ}\AgdaSpace{}%
\AgdaOperator{\AgdaInductiveConstructor{▹}}\AgdaSpace{}%
\AgdaGeneralizable{ip}\AgdaSymbol{)}\<%
\end{code}}
\end{mathpar}
Note that while our contexts are non-dependent (\ie{} the shapes do not depend on the
terms), we use non-trivial shape dependencies within the constructors.  The embedded
language does not have a notion of shape as a value, therefore all the shape dependencies
are handled by Agda, keeping our language simply typed (shaped).  This separation is
very helpful when it comes to writing embedded programs.% XXX: more explanation? 
\begin{code}[hide]%
%
\>[2]\AgdaComment{--pattern\ v₀\ =\ v₀}\<%
\\
%
\>[2]\AgdaKeyword{pattern}\AgdaSpace{}%
\AgdaInductiveConstructor{v₁}\AgdaSpace{}%
\AgdaSymbol{=}\AgdaSpace{}%
\AgdaInductiveConstructor{vₛ}\AgdaSpace{}%
\AgdaInductiveConstructor{v₀}\<%
\\
%
\>[2]\AgdaKeyword{pattern}\AgdaSpace{}%
\AgdaInductiveConstructor{v₂}\AgdaSpace{}%
\AgdaSymbol{=}\AgdaSpace{}%
\AgdaInductiveConstructor{vₛ}\AgdaSpace{}%
\AgdaInductiveConstructor{v₁}\<%
\\
%
\>[2]\AgdaKeyword{pattern}\AgdaSpace{}%
\AgdaInductiveConstructor{v₃}\AgdaSpace{}%
\AgdaSymbol{=}\AgdaSpace{}%
\AgdaInductiveConstructor{vₛ}\AgdaSpace{}%
\AgdaInductiveConstructor{v₂}\<%
\\
%
\>[2]\AgdaKeyword{pattern}\AgdaSpace{}%
\AgdaInductiveConstructor{v₄}\AgdaSpace{}%
\AgdaSymbol{=}\AgdaSpace{}%
\AgdaInductiveConstructor{vₛ}\AgdaSpace{}%
\AgdaInductiveConstructor{v₃}\<%
\\
%
\>[2]\AgdaKeyword{pattern}\AgdaSpace{}%
\AgdaInductiveConstructor{v₅}\AgdaSpace{}%
\AgdaSymbol{=}\AgdaSpace{}%
\AgdaInductiveConstructor{vₛ}\AgdaSpace{}%
\AgdaInductiveConstructor{v₄}\<%
\\
%
\>[2]\AgdaKeyword{pattern}\AgdaSpace{}%
\AgdaInductiveConstructor{v₆}\AgdaSpace{}%
\AgdaSymbol{=}\AgdaSpace{}%
\AgdaInductiveConstructor{vₛ}\AgdaSpace{}%
\AgdaInductiveConstructor{v₅}\<%
\\
%
\>[2]\AgdaKeyword{pattern}\AgdaSpace{}%
\AgdaInductiveConstructor{v₇}\AgdaSpace{}%
\AgdaSymbol{=}\AgdaSpace{}%
\AgdaInductiveConstructor{vₛ}\AgdaSpace{}%
\AgdaInductiveConstructor{v₆}\<%
\\
%
\>[2]\AgdaKeyword{pattern}\AgdaSpace{}%
\AgdaInductiveConstructor{v₈}\AgdaSpace{}%
\AgdaSymbol{=}\AgdaSpace{}%
\AgdaInductiveConstructor{vₛ}\AgdaSpace{}%
\AgdaInductiveConstructor{v₇}\<%
\\
%
\>[2]\AgdaKeyword{pattern}\AgdaSpace{}%
\AgdaInductiveConstructor{v₉}\AgdaSpace{}%
\AgdaSymbol{=}\AgdaSpace{}%
\AgdaInductiveConstructor{vₛ}\AgdaSpace{}%
\AgdaInductiveConstructor{v₈}\<%
\\
%
\\[\AgdaEmptyExtraSkip]%
%
\>[2]\AgdaKeyword{infixl}\AgdaSpace{}%
\AgdaNumber{10}\AgdaSpace{}%
\AgdaOperator{\AgdaInductiveConstructor{\AgdaUnderscore{}⊞\AgdaUnderscore{}}}\<%
\\
%
\>[2]\AgdaKeyword{infixl}\AgdaSpace{}%
\AgdaNumber{15}\AgdaSpace{}%
\AgdaOperator{\AgdaInductiveConstructor{\AgdaUnderscore{}⊠\AgdaUnderscore{}}}\<%
\end{code}
We start with two helper definitions: a singleton shape that we call \AF{unit}
and the type for binary operations that we support (for now only addition and
multiplication).
\begin{mathpar}
\codeblock{\begin{code}%
%
\>[2]\AgdaFunction{unit}\AgdaSpace{}%
\AgdaSymbol{:}\AgdaSpace{}%
\AgdaDatatype{S}\<%
\\
%
\>[2]\AgdaFunction{unit}\AgdaSpace{}%
\AgdaSymbol{=}\AgdaSpace{}%
\AgdaInductiveConstructor{[]}\<%
\end{code}}
\and
\codeblock{\begin{code}%
%
\>[2]\AgdaKeyword{data}\AgdaSpace{}%
\AgdaDatatype{Bop}\AgdaSpace{}%
\AgdaSymbol{:}\AgdaSpace{}%
\AgdaPrimitive{Set}\AgdaSpace{}%
\AgdaKeyword{where}\<%
\\
\>[2][@{}l@{\AgdaIndent{0}}]%
\>[4]\AgdaInductiveConstructor{plus}\AgdaSpace{}%
\AgdaInductiveConstructor{mul}\AgdaSpace{}%
\AgdaSymbol{:}\AgdaSpace{}%
\AgdaDatatype{Bop}\<%
\end{code}}
\end{mathpar}

The embedded language \AF{E} includes: variables \AC{var}; constants 0 and 1 given
by \AC{zero} and \AC{one} correspondingly; three flavours of array constructor/eliminator
pairs given by \AC{imapₛ}/\AC{selₛ}, \AC{imap}/\AC{sel} and \AC{imapb}/\AC{selb};
summation \AC{sum}; conditional \AC{zero-but} where the predicate is fixed to equality
of two indices and the else branch is zero; \AC{slide} and \AC{backslide} exactly
as described before; and numerical operations.  The latter includes \AC{logistic},
plus and multiplication, division by a constant \AC{scaledown}, and unary \AC{minus}.
The definition of the embedded language \AF{E} follows.  We also introduce the
syntax for infix plus and multiplication denoted \AC{⊞} and \AC{⊠} correspondingly.
%\begin{mathpar}
%\codeblock{
\begin{code}%
%
\>[2]\AgdaKeyword{data}\AgdaSpace{}%
\AgdaDatatype{E}\AgdaSpace{}%
\AgdaSymbol{:}\AgdaSpace{}%
\AgdaDatatype{Ctx}\AgdaSpace{}%
\AgdaSymbol{→}\AgdaSpace{}%
\AgdaDatatype{IS}\AgdaSpace{}%
\AgdaSymbol{→}\AgdaSpace{}%
\AgdaPrimitive{Set}\AgdaSpace{}%
\AgdaKeyword{where}\<%
\\
\>[2][@{}l@{\AgdaIndent{0}}]%
\>[4]\AgdaInductiveConstructor{var}%
\>[15]\AgdaSymbol{:}\AgdaSpace{}%
\AgdaGeneralizable{is}\AgdaSpace{}%
\AgdaOperator{\AgdaDatatype{∈}}\AgdaSpace{}%
\AgdaGeneralizable{Γ}\AgdaSpace{}%
\AgdaSymbol{→}\AgdaSpace{}%
\AgdaDatatype{E}\AgdaSpace{}%
\AgdaGeneralizable{Γ}\AgdaSpace{}%
\AgdaGeneralizable{is}\<%
\\
%
\>[4]\AgdaInductiveConstructor{zero}%
\>[15]\AgdaSymbol{:}\AgdaSpace{}%
\AgdaDatatype{E}\AgdaSpace{}%
\AgdaGeneralizable{Γ}\AgdaSpace{}%
\AgdaSymbol{(}\AgdaInductiveConstructor{ar}\AgdaSpace{}%
\AgdaGeneralizable{s}\AgdaSymbol{)}\<%
\\
%
\>[4]\AgdaInductiveConstructor{one}%
\>[15]\AgdaSymbol{:}\AgdaSpace{}%
\AgdaDatatype{E}\AgdaSpace{}%
\AgdaGeneralizable{Γ}\AgdaSpace{}%
\AgdaSymbol{(}\AgdaInductiveConstructor{ar}\AgdaSpace{}%
\AgdaGeneralizable{s}\AgdaSymbol{)}\<%
\\
%
\\[\AgdaEmptyExtraSkip]%
%
\>[4]\AgdaInductiveConstructor{imaps}%
\>[15]\AgdaSymbol{:}\AgdaSpace{}%
\AgdaDatatype{E}\AgdaSpace{}%
\AgdaSymbol{(}\AgdaGeneralizable{Γ}\AgdaSpace{}%
\AgdaOperator{\AgdaInductiveConstructor{▹}}\AgdaSpace{}%
\AgdaInductiveConstructor{ix}\AgdaSpace{}%
\AgdaGeneralizable{s}\AgdaSymbol{)}\AgdaSpace{}%
\AgdaSymbol{(}\AgdaInductiveConstructor{ar}\AgdaSpace{}%
\AgdaFunction{unit}\AgdaSymbol{)}\AgdaSpace{}%
\AgdaSymbol{→}\AgdaSpace{}%
\AgdaDatatype{E}\AgdaSpace{}%
\AgdaGeneralizable{Γ}\AgdaSpace{}%
\AgdaSymbol{(}\AgdaInductiveConstructor{ar}\AgdaSpace{}%
\AgdaGeneralizable{s}\AgdaSymbol{)}\<%
\\
%
\>[4]\AgdaInductiveConstructor{sels}%
\>[15]\AgdaSymbol{:}\AgdaSpace{}%
\AgdaDatatype{E}\AgdaSpace{}%
\AgdaGeneralizable{Γ}\AgdaSpace{}%
\AgdaSymbol{(}\AgdaInductiveConstructor{ar}\AgdaSpace{}%
\AgdaGeneralizable{s}\AgdaSymbol{)}\AgdaSpace{}%
\AgdaSymbol{→}\AgdaSpace{}%
\AgdaDatatype{E}\AgdaSpace{}%
\AgdaGeneralizable{Γ}\AgdaSpace{}%
\AgdaSymbol{(}\AgdaInductiveConstructor{ix}\AgdaSpace{}%
\AgdaGeneralizable{s}\AgdaSymbol{)}\AgdaSpace{}%
\AgdaSymbol{→}\AgdaSpace{}%
\AgdaDatatype{E}\AgdaSpace{}%
\AgdaGeneralizable{Γ}\AgdaSpace{}%
\AgdaSymbol{(}\AgdaInductiveConstructor{ar}\AgdaSpace{}%
\AgdaFunction{unit}\AgdaSymbol{)}\<%
\\
%
\\[\AgdaEmptyExtraSkip]%
%
\>[4]\AgdaInductiveConstructor{imap}%
\>[15]\AgdaSymbol{:}\AgdaSpace{}%
\AgdaDatatype{E}\AgdaSpace{}%
\AgdaSymbol{(}\AgdaGeneralizable{Γ}\AgdaSpace{}%
\AgdaOperator{\AgdaInductiveConstructor{▹}}\AgdaSpace{}%
\AgdaInductiveConstructor{ix}\AgdaSpace{}%
\AgdaGeneralizable{s}\AgdaSymbol{)}\AgdaSpace{}%
\AgdaSymbol{(}\AgdaInductiveConstructor{ar}\AgdaSpace{}%
\AgdaGeneralizable{p}\AgdaSymbol{)}\AgdaSpace{}%
\AgdaSymbol{→}\AgdaSpace{}%
\AgdaDatatype{E}\AgdaSpace{}%
\AgdaGeneralizable{Γ}\AgdaSpace{}%
\AgdaSymbol{(}\AgdaInductiveConstructor{ar}\AgdaSpace{}%
\AgdaSymbol{(}\AgdaGeneralizable{s}\AgdaSpace{}%
\AgdaOperator{\AgdaFunction{⊗}}\AgdaSpace{}%
\AgdaGeneralizable{p}\AgdaSymbol{))}\<%
\\
%
\>[4]\AgdaInductiveConstructor{sel}%
\>[15]\AgdaSymbol{:}\AgdaSpace{}%
\AgdaDatatype{E}\AgdaSpace{}%
\AgdaGeneralizable{Γ}\AgdaSpace{}%
\AgdaSymbol{(}\AgdaInductiveConstructor{ar}\AgdaSpace{}%
\AgdaSymbol{(}\AgdaGeneralizable{s}\AgdaSpace{}%
\AgdaOperator{\AgdaFunction{⊗}}\AgdaSpace{}%
\AgdaGeneralizable{p}\AgdaSymbol{))}\AgdaSpace{}%
\AgdaSymbol{→}\AgdaSpace{}%
\AgdaDatatype{E}\AgdaSpace{}%
\AgdaGeneralizable{Γ}\AgdaSpace{}%
\AgdaSymbol{(}\AgdaInductiveConstructor{ix}\AgdaSpace{}%
\AgdaGeneralizable{s}\AgdaSymbol{)}\AgdaSpace{}%
\AgdaSymbol{→}\AgdaSpace{}%
\AgdaDatatype{E}\AgdaSpace{}%
\AgdaGeneralizable{Γ}\AgdaSpace{}%
\AgdaSymbol{(}\AgdaInductiveConstructor{ar}\AgdaSpace{}%
\AgdaGeneralizable{p}\AgdaSymbol{)}\<%
\\
%
\\[\AgdaEmptyExtraSkip]%
%
\>[4]\AgdaInductiveConstructor{imapb}%
\>[15]\AgdaSymbol{:}\AgdaSpace{}%
\AgdaGeneralizable{s}\AgdaSpace{}%
\AgdaOperator{\AgdaFunction{*}}\AgdaSpace{}%
\AgdaGeneralizable{p}\AgdaSpace{}%
\AgdaOperator{\AgdaFunction{≈}}\AgdaSpace{}%
\AgdaGeneralizable{q}\AgdaSpace{}%
\AgdaSymbol{→}\AgdaSpace{}%
\AgdaDatatype{E}\AgdaSpace{}%
\AgdaSymbol{(}\AgdaGeneralizable{Γ}\AgdaSpace{}%
\AgdaOperator{\AgdaInductiveConstructor{▹}}\AgdaSpace{}%
\AgdaInductiveConstructor{ix}\AgdaSpace{}%
\AgdaGeneralizable{s}\AgdaSymbol{)}\AgdaSpace{}%
\AgdaSymbol{(}\AgdaInductiveConstructor{ar}\AgdaSpace{}%
\AgdaGeneralizable{p}\AgdaSymbol{)}\AgdaSpace{}%
\AgdaSymbol{→}\AgdaSpace{}%
\AgdaDatatype{E}\AgdaSpace{}%
\AgdaGeneralizable{Γ}\AgdaSpace{}%
\AgdaSymbol{(}\AgdaInductiveConstructor{ar}\AgdaSpace{}%
\AgdaGeneralizable{q}\AgdaSymbol{)}\<%
\\
%
\>[4]\AgdaInductiveConstructor{selb}%
\>[15]\AgdaSymbol{:}\AgdaSpace{}%
\AgdaGeneralizable{s}\AgdaSpace{}%
\AgdaOperator{\AgdaFunction{*}}\AgdaSpace{}%
\AgdaGeneralizable{p}\AgdaSpace{}%
\AgdaOperator{\AgdaFunction{≈}}\AgdaSpace{}%
\AgdaGeneralizable{q}\AgdaSpace{}%
\AgdaSymbol{→}\AgdaSpace{}%
\AgdaDatatype{E}\AgdaSpace{}%
\AgdaGeneralizable{Γ}\AgdaSpace{}%
\AgdaSymbol{(}\AgdaInductiveConstructor{ar}\AgdaSpace{}%
\AgdaGeneralizable{q}\AgdaSymbol{)}\AgdaSpace{}%
\AgdaSymbol{→}\AgdaSpace{}%
\AgdaDatatype{E}\AgdaSpace{}%
\AgdaGeneralizable{Γ}\AgdaSpace{}%
\AgdaSymbol{(}\AgdaInductiveConstructor{ix}\AgdaSpace{}%
\AgdaGeneralizable{s}\AgdaSymbol{)}\AgdaSpace{}%
\AgdaSymbol{→}\AgdaSpace{}%
\AgdaDatatype{E}\AgdaSpace{}%
\AgdaGeneralizable{Γ}\AgdaSpace{}%
\AgdaSymbol{(}\AgdaInductiveConstructor{ar}\AgdaSpace{}%
\AgdaGeneralizable{p}\AgdaSymbol{)}\<%
\\
%
\\[\AgdaEmptyExtraSkip]%
%
\>[4]\AgdaInductiveConstructor{sum}%
\>[15]\AgdaSymbol{:}\AgdaSpace{}%
\AgdaDatatype{E}\AgdaSpace{}%
\AgdaSymbol{(}\AgdaGeneralizable{Γ}\AgdaSpace{}%
\AgdaOperator{\AgdaInductiveConstructor{▹}}\AgdaSpace{}%
\AgdaInductiveConstructor{ix}\AgdaSpace{}%
\AgdaGeneralizable{s}\AgdaSymbol{)}\AgdaSpace{}%
\AgdaSymbol{(}\AgdaInductiveConstructor{ar}\AgdaSpace{}%
\AgdaGeneralizable{p}\AgdaSymbol{)}\AgdaSpace{}%
\AgdaSymbol{→}\AgdaSpace{}%
\AgdaDatatype{E}\AgdaSpace{}%
\AgdaGeneralizable{Γ}\AgdaSpace{}%
\AgdaSymbol{(}\AgdaInductiveConstructor{ar}\AgdaSpace{}%
\AgdaGeneralizable{p}\AgdaSymbol{)}\<%
\\
%
\>[4]\AgdaInductiveConstructor{zero-but}%
\>[15]\AgdaSymbol{:}\AgdaSpace{}%
\AgdaDatatype{E}\AgdaSpace{}%
\AgdaGeneralizable{Γ}\AgdaSpace{}%
\AgdaSymbol{(}\AgdaInductiveConstructor{ix}\AgdaSpace{}%
\AgdaGeneralizable{s}\AgdaSymbol{)}\AgdaSpace{}%
\AgdaSymbol{→}\AgdaSpace{}%
\AgdaDatatype{E}\AgdaSpace{}%
\AgdaGeneralizable{Γ}\AgdaSpace{}%
\AgdaSymbol{(}\AgdaInductiveConstructor{ix}\AgdaSpace{}%
\AgdaGeneralizable{s}\AgdaSymbol{)}\AgdaSpace{}%
\AgdaSymbol{→}\AgdaSpace{}%
\AgdaDatatype{E}\AgdaSpace{}%
\AgdaGeneralizable{Γ}\AgdaSpace{}%
\AgdaSymbol{(}\AgdaInductiveConstructor{ar}\AgdaSpace{}%
\AgdaGeneralizable{p}\AgdaSymbol{)}\AgdaSpace{}%
\AgdaSymbol{→}\AgdaSpace{}%
\AgdaDatatype{E}\AgdaSpace{}%
\AgdaGeneralizable{Γ}\AgdaSpace{}%
\AgdaSymbol{(}\AgdaInductiveConstructor{ar}\AgdaSpace{}%
\AgdaGeneralizable{p}\AgdaSymbol{)}\<%
\\
%
\\[\AgdaEmptyExtraSkip]%
%
\>[4]\AgdaInductiveConstructor{slide}%
\>[15]\AgdaSymbol{:}\AgdaSpace{}%
\AgdaDatatype{E}\AgdaSpace{}%
\AgdaGeneralizable{Γ}\AgdaSpace{}%
\AgdaSymbol{(}\AgdaInductiveConstructor{ix}\AgdaSpace{}%
\AgdaGeneralizable{s}\AgdaSymbol{)}\AgdaSpace{}%
\AgdaSymbol{→}\AgdaSpace{}%
\AgdaGeneralizable{s}\AgdaSpace{}%
\AgdaOperator{\AgdaFunction{+}}\AgdaSpace{}%
\AgdaGeneralizable{p}\AgdaSpace{}%
\AgdaOperator{\AgdaFunction{≈}}\AgdaSpace{}%
\AgdaGeneralizable{r}\AgdaSpace{}%
\AgdaSymbol{→}\AgdaSpace{}%
\AgdaDatatype{E}\AgdaSpace{}%
\AgdaGeneralizable{Γ}\AgdaSpace{}%
\AgdaSymbol{(}\AgdaInductiveConstructor{ar}\AgdaSpace{}%
\AgdaGeneralizable{r}\AgdaSymbol{)}\AgdaSpace{}%
\AgdaSymbol{→}\AgdaSpace{}%
\AgdaOperator{\AgdaFunction{suc}}\AgdaSpace{}%
\AgdaGeneralizable{p}\AgdaSpace{}%
\AgdaOperator{\AgdaFunction{≈}}\AgdaSpace{}%
\AgdaGeneralizable{u}\AgdaSpace{}%
\AgdaSymbol{→}\AgdaSpace{}%
\AgdaDatatype{E}\AgdaSpace{}%
\AgdaGeneralizable{Γ}\AgdaSpace{}%
\AgdaSymbol{(}\AgdaInductiveConstructor{ar}\AgdaSpace{}%
\AgdaGeneralizable{u}\AgdaSymbol{)}\<%
\\
%
\>[4]\AgdaInductiveConstructor{backslide}%
\>[15]\AgdaSymbol{:}\AgdaSpace{}%
\AgdaDatatype{E}\AgdaSpace{}%
\AgdaGeneralizable{Γ}\AgdaSpace{}%
\AgdaSymbol{(}\AgdaInductiveConstructor{ix}\AgdaSpace{}%
\AgdaGeneralizable{s}\AgdaSymbol{)}\AgdaSpace{}%
\AgdaSymbol{→}\AgdaSpace{}%
\AgdaDatatype{E}\AgdaSpace{}%
\AgdaGeneralizable{Γ}\AgdaSpace{}%
\AgdaSymbol{(}\AgdaInductiveConstructor{ar}\AgdaSpace{}%
\AgdaGeneralizable{u}\AgdaSymbol{)}\AgdaSpace{}%
\AgdaSymbol{→}\AgdaSpace{}%
\AgdaOperator{\AgdaFunction{suc}}\AgdaSpace{}%
\AgdaGeneralizable{p}\AgdaSpace{}%
\AgdaOperator{\AgdaFunction{≈}}\AgdaSpace{}%
\AgdaGeneralizable{u}\AgdaSpace{}%
\AgdaSymbol{→}\AgdaSpace{}%
\AgdaGeneralizable{s}\AgdaSpace{}%
\AgdaOperator{\AgdaFunction{+}}\AgdaSpace{}%
\AgdaGeneralizable{p}\AgdaSpace{}%
\AgdaOperator{\AgdaFunction{≈}}\AgdaSpace{}%
\AgdaGeneralizable{r}\AgdaSpace{}%
\AgdaSymbol{→}\AgdaSpace{}%
\AgdaDatatype{E}\AgdaSpace{}%
\AgdaGeneralizable{Γ}\AgdaSpace{}%
\AgdaSymbol{(}\AgdaInductiveConstructor{ar}\AgdaSpace{}%
\AgdaGeneralizable{r}\AgdaSymbol{)}\<%
\\
%
\\[\AgdaEmptyExtraSkip]%
%
\>[4]\AgdaInductiveConstructor{logistic}%
\>[15]\AgdaSymbol{:}\AgdaSpace{}%
\AgdaDatatype{E}\AgdaSpace{}%
\AgdaGeneralizable{Γ}\AgdaSpace{}%
\AgdaSymbol{(}\AgdaInductiveConstructor{ar}\AgdaSpace{}%
\AgdaGeneralizable{s}\AgdaSymbol{)}\AgdaSpace{}%
\AgdaSymbol{→}\AgdaSpace{}%
\AgdaDatatype{E}\AgdaSpace{}%
\AgdaGeneralizable{Γ}\AgdaSpace{}%
\AgdaSymbol{(}\AgdaInductiveConstructor{ar}\AgdaSpace{}%
\AgdaGeneralizable{s}\AgdaSymbol{)}\<%
\\
%
\>[4]\AgdaInductiveConstructor{bin}%
\>[15]\AgdaSymbol{:}\AgdaSpace{}%
\AgdaDatatype{Bop}\AgdaSpace{}%
\AgdaSymbol{→}\AgdaSpace{}%
\AgdaDatatype{E}\AgdaSpace{}%
\AgdaGeneralizable{Γ}\AgdaSpace{}%
\AgdaSymbol{(}\AgdaInductiveConstructor{ar}\AgdaSpace{}%
\AgdaGeneralizable{s}\AgdaSymbol{)}\AgdaSpace{}%
\AgdaSymbol{→}\AgdaSpace{}%
\AgdaDatatype{E}\AgdaSpace{}%
\AgdaGeneralizable{Γ}\AgdaSpace{}%
\AgdaSymbol{(}\AgdaInductiveConstructor{ar}\AgdaSpace{}%
\AgdaGeneralizable{s}\AgdaSymbol{)}\AgdaSpace{}%
\AgdaSymbol{→}\AgdaSpace{}%
\AgdaDatatype{E}\AgdaSpace{}%
\AgdaGeneralizable{Γ}\AgdaSpace{}%
\AgdaSymbol{(}\AgdaInductiveConstructor{ar}\AgdaSpace{}%
\AgdaGeneralizable{s}\AgdaSymbol{)}\<%
\\
%
\>[4]\AgdaInductiveConstructor{scaledown}%
\>[15]\AgdaSymbol{:}\AgdaSpace{}%
\AgdaDatatype{ℕ}\AgdaSpace{}%
\AgdaSymbol{→}\AgdaSpace{}%
\AgdaDatatype{E}\AgdaSpace{}%
\AgdaGeneralizable{Γ}\AgdaSpace{}%
\AgdaSymbol{(}\AgdaInductiveConstructor{ar}\AgdaSpace{}%
\AgdaGeneralizable{s}\AgdaSymbol{)}\AgdaSpace{}%
\AgdaSymbol{→}\AgdaSpace{}%
\AgdaDatatype{E}\AgdaSpace{}%
\AgdaGeneralizable{Γ}\AgdaSpace{}%
\AgdaSymbol{(}\AgdaInductiveConstructor{ar}\AgdaSpace{}%
\AgdaGeneralizable{s}\AgdaSymbol{)}\<%
\\
%
\>[4]\AgdaInductiveConstructor{minus}%
\>[15]\AgdaSymbol{:}\AgdaSpace{}%
\AgdaDatatype{E}\AgdaSpace{}%
\AgdaGeneralizable{Γ}\AgdaSpace{}%
\AgdaSymbol{(}\AgdaInductiveConstructor{ar}\AgdaSpace{}%
\AgdaGeneralizable{s}\AgdaSymbol{)}\AgdaSpace{}%
\AgdaSymbol{→}\AgdaSpace{}%
\AgdaDatatype{E}\AgdaSpace{}%
\AgdaGeneralizable{Γ}\AgdaSpace{}%
\AgdaSymbol{(}\AgdaInductiveConstructor{ar}\AgdaSpace{}%
\AgdaGeneralizable{s}\AgdaSymbol{)}\<%
\\
%
\>[4]\AgdaInductiveConstructor{let′}%
\>[15]\AgdaSymbol{:}\AgdaSpace{}%
\AgdaDatatype{E}\AgdaSpace{}%
\AgdaGeneralizable{Γ}\AgdaSpace{}%
\AgdaSymbol{(}\AgdaInductiveConstructor{ar}\AgdaSpace{}%
\AgdaGeneralizable{s}\AgdaSymbol{)}\AgdaSpace{}%
\AgdaSymbol{→}\AgdaSpace{}%
\AgdaDatatype{E}\AgdaSpace{}%
\AgdaSymbol{(}\AgdaGeneralizable{Γ}\AgdaSpace{}%
\AgdaOperator{\AgdaInductiveConstructor{▹}}\AgdaSpace{}%
\AgdaInductiveConstructor{ar}\AgdaSpace{}%
\AgdaGeneralizable{s}\AgdaSymbol{)}\AgdaSpace{}%
\AgdaSymbol{(}\AgdaInductiveConstructor{ar}\AgdaSpace{}%
\AgdaGeneralizable{p}\AgdaSymbol{)}\AgdaSpace{}%
\AgdaSymbol{→}\AgdaSpace{}%
\AgdaDatatype{E}\AgdaSpace{}%
\AgdaGeneralizable{Γ}\AgdaSpace{}%
\AgdaSymbol{(}\AgdaInductiveConstructor{ar}\AgdaSpace{}%
\AgdaGeneralizable{p}\AgdaSymbol{)}\<%
\\
%
\\[\AgdaEmptyExtraSkip]%
%
\>[2]\AgdaKeyword{pattern}\AgdaSpace{}%
\AgdaOperator{\AgdaInductiveConstructor{\AgdaUnderscore{}⊠\AgdaUnderscore{}}}\AgdaSpace{}%
\AgdaBound{a}\AgdaSpace{}%
\AgdaBound{b}\AgdaSpace{}%
\AgdaSymbol{=}\AgdaSpace{}%
\AgdaInductiveConstructor{bin}\AgdaSpace{}%
\AgdaInductiveConstructor{mul}\AgdaSpace{}%
\AgdaBound{a}\AgdaSpace{}%
\AgdaBound{b}\<%
\\
%
\>[2]\AgdaKeyword{pattern}\AgdaSpace{}%
\AgdaOperator{\AgdaInductiveConstructor{\AgdaUnderscore{}⊞\AgdaUnderscore{}}}\AgdaSpace{}%
\AgdaBound{a}\AgdaSpace{}%
\AgdaBound{b}\AgdaSpace{}%
\AgdaSymbol{=}\AgdaSpace{}%
\AgdaInductiveConstructor{bin}\AgdaSpace{}%
\AgdaInductiveConstructor{plus}\AgdaSpace{}%
\AgdaBound{a}\AgdaSpace{}%
\AgdaBound{b}\<%
\end{code}
%}
%\end{mathpar}

\subsection{Reals}
\todo[inline]{Explain that this is our module for abstract reals with
  operations and their properties and that we parametrise our
  evaluator witht this module.}

\begin{code}%
\>[0]\AgdaKeyword{record}\AgdaSpace{}%
\AgdaRecord{Real}\AgdaSpace{}%
\AgdaSymbol{:}\AgdaSpace{}%
\AgdaPrimitive{Set₁}\AgdaSpace{}%
\AgdaKeyword{where}\<%
\\
\>[0][@{}l@{\AgdaIndent{0}}]%
\>[2]\AgdaKeyword{field}\<%
\\
\>[2][@{}l@{\AgdaIndent{0}}]%
\>[4]\AgdaField{R}\AgdaSpace{}%
\AgdaSymbol{:}\AgdaSpace{}%
\AgdaPrimitive{Set}\<%
\\
%
\>[4]\AgdaField{fromℕ}\AgdaSpace{}%
\AgdaSymbol{:}\AgdaSpace{}%
\AgdaDatatype{ℕ}\AgdaSpace{}%
\AgdaSymbol{→}\AgdaSpace{}%
\AgdaField{R}\<%
\\
%
\>[4]\AgdaOperator{\AgdaField{\AgdaUnderscore{}+\AgdaUnderscore{}}}\AgdaSpace{}%
\AgdaOperator{\AgdaField{\AgdaUnderscore{}*\AgdaUnderscore{}}}\AgdaSpace{}%
\AgdaOperator{\AgdaField{\AgdaUnderscore{}÷\AgdaUnderscore{}}}\AgdaSpace{}%
\AgdaSymbol{:}\AgdaSpace{}%
\AgdaField{R}\AgdaSpace{}%
\AgdaSymbol{→}\AgdaSpace{}%
\AgdaField{R}\AgdaSpace{}%
\AgdaSymbol{→}\AgdaSpace{}%
\AgdaField{R}\<%
\\
%
\>[4]\AgdaOperator{\AgdaField{-\AgdaUnderscore{}}}\AgdaSpace{}%
\AgdaOperator{\AgdaField{e\textasciicircum{}\AgdaUnderscore{}}}\AgdaSpace{}%
\AgdaSymbol{:}\AgdaSpace{}%
\AgdaField{R}\AgdaSpace{}%
\AgdaSymbol{→}\AgdaSpace{}%
\AgdaField{R}\<%
\end{code}
\begin{code}[hide]%
%
\>[2]\AgdaKeyword{infixl}\AgdaSpace{}%
\AgdaNumber{10}\AgdaSpace{}%
\AgdaOperator{\AgdaField{\AgdaUnderscore{}+\AgdaUnderscore{}}}\<%
\\
%
\>[2]\AgdaKeyword{infixl}\AgdaSpace{}%
\AgdaNumber{15}\AgdaSpace{}%
\AgdaOperator{\AgdaField{\AgdaUnderscore{}*\AgdaUnderscore{}}}\<%
\\
%
\>[2]\AgdaKeyword{infixl}\AgdaSpace{}%
\AgdaNumber{15}\AgdaSpace{}%
\AgdaOperator{\AgdaField{\AgdaUnderscore{}÷\AgdaUnderscore{}}}\<%
\end{code}
\begin{code}%
%
\>[2]\AgdaFunction{logisticʳ}\AgdaSpace{}%
\AgdaSymbol{:}\AgdaSpace{}%
\AgdaField{R}\AgdaSpace{}%
\AgdaSymbol{→}\AgdaSpace{}%
\AgdaField{R}\<%
\\
%
\>[2]\AgdaFunction{logisticʳ}\AgdaSpace{}%
\AgdaBound{x}\AgdaSpace{}%
\AgdaSymbol{=}\AgdaSpace{}%
\AgdaField{fromℕ}\AgdaSpace{}%
\AgdaNumber{1}\AgdaSpace{}%
\AgdaOperator{\AgdaField{÷}}\AgdaSpace{}%
\AgdaSymbol{(}\AgdaField{fromℕ}\AgdaSpace{}%
\AgdaNumber{1}\AgdaSpace{}%
\AgdaOperator{\AgdaField{+}}\AgdaSpace{}%
\AgdaOperator{\AgdaField{e\textasciicircum{}}}\AgdaSpace{}%
\AgdaSymbol{(}\AgdaOperator{\AgdaField{-}}\AgdaSpace{}%
\AgdaBound{x}\AgdaSymbol{))}\<%
\end{code}


\subsection{Evaluation}
\todo[inline]{The text in this section needs adjustment}
\begin{code}[hide]%
\>[0]\AgdaKeyword{module}\AgdaSpace{}%
\AgdaModule{Eval}\AgdaSpace{}%
\AgdaSymbol{(}\AgdaBound{real}\AgdaSpace{}%
\AgdaSymbol{:}\AgdaSpace{}%
\AgdaRecord{Real}\AgdaSymbol{)}\AgdaSpace{}%
\AgdaKeyword{where}\<%
\\
\>[0][@{}l@{\AgdaIndent{0}}]%
\>[2]\AgdaComment{--open\ import\ Data.Float\ as\ F\ renaming\ (Float\ to\ ℝ)\ hiding\ (⌊\AgdaUnderscore{}⌋)}\<%
\\
%
\>[2]\AgdaKeyword{open}\AgdaSpace{}%
\AgdaKeyword{import}\AgdaSpace{}%
\AgdaModule{Data.Unit}\<%
\\
%
\>[2]\AgdaKeyword{open}\AgdaSpace{}%
\AgdaKeyword{import}\AgdaSpace{}%
\AgdaModule{Data.Product}\AgdaSpace{}%
\AgdaKeyword{using}\AgdaSpace{}%
\AgdaSymbol{(}\AgdaOperator{\AgdaFunction{\AgdaUnderscore{}×\AgdaUnderscore{}}}\AgdaSymbol{;}\AgdaSpace{}%
\AgdaField{proj₁}\AgdaSymbol{;}\AgdaSpace{}%
\AgdaField{proj₂}\AgdaSymbol{;}\AgdaSpace{}%
\AgdaOperator{\AgdaInductiveConstructor{\AgdaUnderscore{},\AgdaUnderscore{}}}\AgdaSymbol{)}\<%
\\
%
\>[2]\AgdaKeyword{open}\AgdaSpace{}%
\AgdaKeyword{import}\AgdaSpace{}%
\AgdaModule{Data.Fin}\AgdaSpace{}%
\AgdaKeyword{using}\AgdaSpace{}%
\AgdaSymbol{(}\AgdaDatatype{Fin}\AgdaSymbol{;}\AgdaSpace{}%
\AgdaInductiveConstructor{zero}\AgdaSymbol{;}\AgdaSpace{}%
\AgdaInductiveConstructor{suc}\AgdaSymbol{;}\AgdaSpace{}%
\AgdaOperator{\AgdaFunction{\#\AgdaUnderscore{}}}\AgdaSymbol{)}\<%
\\
%
\>[2]\AgdaKeyword{open}\AgdaSpace{}%
\AgdaKeyword{import}\AgdaSpace{}%
\AgdaModule{Relation.Nullary.Decidable}\<%
\\
%
\>[2]\AgdaKeyword{open}\AgdaSpace{}%
\AgdaKeyword{import}\AgdaSpace{}%
\AgdaModule{Data.Bool}\<%
\\
%
\\[\AgdaEmptyExtraSkip]%
%
\>[2]\AgdaKeyword{open}\AgdaSpace{}%
\AgdaModule{Lang}\<%
\\
%
\>[2]\AgdaKeyword{open}\AgdaSpace{}%
\AgdaModule{Array}\<%
\\
%
\>[2]\AgdaKeyword{open}\AgdaSpace{}%
\AgdaModule{Real}\AgdaSpace{}%
\AgdaBound{real}\<%
\end{code}

We define the interpretation \AF{⟦\_⟧} for (\AF{E} \AB{Γ} \AB{is}) into the value
(\AF{Val} \AB{is}) in the environment (\AF{Env} \AB{Γ}).  The values are either
arrays or positions of the corresponding shape.  Environments for the given context
\AB{Γ} are tuples of values of the corresponding shapes.  The \AF{lookup} function
translates variables within the context into variables within the environment.
\begin{mathpar}
\codeblock{\begin{code}%
%
\>[2]\AgdaFunction{Val}\AgdaSpace{}%
\AgdaSymbol{:}\AgdaSpace{}%
\AgdaDatatype{IS}\AgdaSpace{}%
\AgdaSymbol{→}\AgdaSpace{}%
\AgdaPrimitive{Set}\<%
\\
%
\>[2]\AgdaFunction{Val}\AgdaSpace{}%
\AgdaSymbol{(}\AgdaInductiveConstructor{ar}\AgdaSpace{}%
\AgdaBound{s}\AgdaSymbol{)}%
\>[14]\AgdaSymbol{=}\AgdaSpace{}%
\AgdaFunction{Ar}\AgdaSpace{}%
\AgdaBound{s}\AgdaSpace{}%
\AgdaField{R}\<%
\\
%
\>[2]\AgdaFunction{Val}\AgdaSpace{}%
\AgdaSymbol{(}\AgdaInductiveConstructor{ix}\AgdaSpace{}%
\AgdaBound{s}\AgdaSymbol{)}%
\>[14]\AgdaSymbol{=}\AgdaSpace{}%
\AgdaDatatype{P}\AgdaSpace{}%
\AgdaBound{s}\<%
\end{code}}
\and
\codeblock{\begin{code}%
%
\>[2]\AgdaFunction{Env}\AgdaSpace{}%
\AgdaSymbol{:}\AgdaSpace{}%
\AgdaDatatype{Ctx}\AgdaSpace{}%
\AgdaSymbol{→}\AgdaSpace{}%
\AgdaPrimitive{Set}\<%
\\
%
\>[2]\AgdaFunction{Env}\AgdaSpace{}%
\AgdaInductiveConstructor{ε}%
\>[16]\AgdaSymbol{=}\AgdaSpace{}%
\AgdaRecord{⊤}\<%
\\
%
\>[2]\AgdaFunction{Env}\AgdaSpace{}%
\AgdaSymbol{(}\AgdaBound{Γ}\AgdaSpace{}%
\AgdaOperator{\AgdaInductiveConstructor{▹}}\AgdaSpace{}%
\AgdaBound{is}\AgdaSymbol{)}%
\>[16]\AgdaSymbol{=}\AgdaSpace{}%
\AgdaFunction{Env}\AgdaSpace{}%
\AgdaBound{Γ}\AgdaSpace{}%
\AgdaOperator{\AgdaFunction{×}}\AgdaSpace{}%
\AgdaFunction{Val}\AgdaSpace{}%
\AgdaBound{is}\<%
\end{code}}
\and
\codeblock{\begin{code}%
%
\>[2]\AgdaFunction{lookup}\AgdaSpace{}%
\AgdaSymbol{:}\AgdaSpace{}%
\AgdaGeneralizable{is}\AgdaSpace{}%
\AgdaOperator{\AgdaDatatype{∈}}\AgdaSpace{}%
\AgdaGeneralizable{Γ}\AgdaSpace{}%
\AgdaSymbol{→}\AgdaSpace{}%
\AgdaFunction{Env}\AgdaSpace{}%
\AgdaGeneralizable{Γ}\AgdaSpace{}%
\AgdaSymbol{→}\AgdaSpace{}%
\AgdaFunction{Val}\AgdaSpace{}%
\AgdaGeneralizable{is}\<%
\\
%
\>[2]\AgdaFunction{lookup}\AgdaSpace{}%
\AgdaInductiveConstructor{v₀}%
\>[17]\AgdaSymbol{(}\AgdaBound{ρ}\AgdaSpace{}%
\AgdaOperator{\AgdaInductiveConstructor{,}}\AgdaSpace{}%
\AgdaBound{x}\AgdaSymbol{)}%
\>[26]\AgdaSymbol{=}\AgdaSpace{}%
\AgdaBound{x}\<%
\\
%
\>[2]\AgdaFunction{lookup}\AgdaSpace{}%
\AgdaSymbol{(}\AgdaInductiveConstructor{vₛ}\AgdaSpace{}%
\AgdaBound{v}\AgdaSymbol{)}%
\>[17]\AgdaSymbol{(}\AgdaBound{ρ}\AgdaSpace{}%
\AgdaOperator{\AgdaInductiveConstructor{,}}\AgdaSpace{}%
\AgdaSymbol{\AgdaUnderscore{})}%
\>[26]\AgdaSymbol{=}\AgdaSpace{}%
\AgdaFunction{lookup}\AgdaSpace{}%
\AgdaBound{v}\AgdaSpace{}%
\AgdaBound{ρ}\<%
\end{code}}
\end{mathpar}

In the definition of \AF{⟦\_⟧} we wrap the environment argument into double braces.
This is an Agda-specific syntax for instance arguments\footnote{%
  For more details on instance arguments see:
  \url{https://agda.readthedocs.io/en/v2.6.3/language/instance-arguments.html}}
which behave similarly to hidden arguments, but they have a more powerful resolution
algorithm.  As a result we can omit mentioning the environment in recursive calls
when it is passed unchanged.
\begin{code}%
%
\>[2]\AgdaOperator{\AgdaFunction{⟦\AgdaUnderscore{}⟧}}\AgdaSpace{}%
\AgdaSymbol{:}\AgdaSpace{}%
\AgdaDatatype{E}\AgdaSpace{}%
\AgdaGeneralizable{Γ}\AgdaSpace{}%
\AgdaGeneralizable{is}\AgdaSpace{}%
\AgdaSymbol{→}\AgdaSpace{}%
\AgdaFunction{Env}\AgdaSpace{}%
\AgdaGeneralizable{Γ}\AgdaSpace{}%
\AgdaSymbol{→}\AgdaSpace{}%
\AgdaFunction{Val}\AgdaSpace{}%
\AgdaGeneralizable{is}\<%
\\
%
\>[2]\AgdaOperator{\AgdaFunction{⟦}}\AgdaSpace{}%
\AgdaInductiveConstructor{var}\AgdaSpace{}%
\AgdaBound{x}%
\>[24]\AgdaOperator{\AgdaFunction{⟧}}\AgdaSpace{}%
\AgdaBound{ρ}%
\>[29]\AgdaSymbol{=}\AgdaSpace{}%
\AgdaFunction{lookup}\AgdaSpace{}%
\AgdaBound{x}\AgdaSpace{}%
\AgdaBound{ρ}\<%
\\
%
\>[2]\AgdaOperator{\AgdaFunction{⟦}}\AgdaSpace{}%
\AgdaInductiveConstructor{zero}%
\>[24]\AgdaOperator{\AgdaFunction{⟧}}\AgdaSpace{}%
\AgdaBound{ρ}%
\>[29]\AgdaSymbol{=}\AgdaSpace{}%
\AgdaFunction{K}\AgdaSpace{}%
\AgdaSymbol{(}\AgdaField{fromℕ}\AgdaSpace{}%
\AgdaNumber{0}\AgdaSymbol{)}\<%
\\
%
\>[2]\AgdaOperator{\AgdaFunction{⟦}}\AgdaSpace{}%
\AgdaInductiveConstructor{one}%
\>[24]\AgdaOperator{\AgdaFunction{⟧}}\AgdaSpace{}%
\AgdaBound{ρ}%
\>[29]\AgdaSymbol{=}\AgdaSpace{}%
\AgdaFunction{K}\AgdaSpace{}%
\AgdaSymbol{(}\AgdaField{fromℕ}\AgdaSpace{}%
\AgdaNumber{1}\AgdaSymbol{)}\<%
\\
%
\>[2]\AgdaOperator{\AgdaFunction{⟦}}\AgdaSpace{}%
\AgdaInductiveConstructor{imaps}\AgdaSpace{}%
\AgdaBound{e}%
\>[24]\AgdaOperator{\AgdaFunction{⟧}}\AgdaSpace{}%
\AgdaBound{ρ}%
\>[29]\AgdaSymbol{=}\AgdaSpace{}%
\AgdaSymbol{λ}\AgdaSpace{}%
\AgdaBound{i}\AgdaSpace{}%
\AgdaSymbol{→}\AgdaSpace{}%
\AgdaOperator{\AgdaFunction{⟦}}\AgdaSpace{}%
\AgdaBound{e}\AgdaSpace{}%
\AgdaOperator{\AgdaFunction{⟧}}\AgdaSpace{}%
\AgdaSymbol{(}\AgdaBound{ρ}\AgdaSpace{}%
\AgdaOperator{\AgdaInductiveConstructor{,}}\AgdaSpace{}%
\AgdaBound{i}\AgdaSymbol{)}\AgdaSpace{}%
\AgdaInductiveConstructor{[]}\<%
\\
%
\>[2]\AgdaOperator{\AgdaFunction{⟦}}\AgdaSpace{}%
\AgdaInductiveConstructor{sels}\AgdaSpace{}%
\AgdaBound{e}\AgdaSpace{}%
\AgdaBound{e₁}%
\>[24]\AgdaOperator{\AgdaFunction{⟧}}\AgdaSpace{}%
\AgdaBound{ρ}%
\>[29]\AgdaSymbol{=}\AgdaSpace{}%
\AgdaFunction{K}\AgdaSpace{}%
\AgdaSymbol{(}\AgdaOperator{\AgdaFunction{⟦}}\AgdaSpace{}%
\AgdaBound{e}\AgdaSpace{}%
\AgdaOperator{\AgdaFunction{⟧}}\AgdaSpace{}%
\AgdaBound{ρ}\AgdaSpace{}%
\AgdaSymbol{(}\AgdaOperator{\AgdaFunction{⟦}}\AgdaSpace{}%
\AgdaBound{e₁}\AgdaSpace{}%
\AgdaOperator{\AgdaFunction{⟧}}\AgdaSpace{}%
\AgdaBound{ρ}\AgdaSymbol{))}\<%
\\
%
\>[2]\AgdaOperator{\AgdaFunction{⟦}}\AgdaSpace{}%
\AgdaInductiveConstructor{imap}\AgdaSpace{}%
\AgdaBound{e}%
\>[24]\AgdaOperator{\AgdaFunction{⟧}}\AgdaSpace{}%
\AgdaBound{ρ}%
\>[29]\AgdaSymbol{=}\AgdaSpace{}%
\AgdaFunction{unnest}\AgdaSpace{}%
\AgdaSymbol{λ}\AgdaSpace{}%
\AgdaBound{i}\AgdaSpace{}%
\AgdaSymbol{→}\AgdaSpace{}%
\AgdaOperator{\AgdaFunction{⟦}}\AgdaSpace{}%
\AgdaBound{e}\AgdaSpace{}%
\AgdaOperator{\AgdaFunction{⟧}}\AgdaSpace{}%
\AgdaSymbol{(}\AgdaBound{ρ}\AgdaSpace{}%
\AgdaOperator{\AgdaInductiveConstructor{,}}\AgdaSpace{}%
\AgdaBound{i}\AgdaSymbol{)}\<%
\\
%
\>[2]\AgdaOperator{\AgdaFunction{⟦}}\AgdaSpace{}%
\AgdaInductiveConstructor{sel}\AgdaSpace{}%
\AgdaBound{e}\AgdaSpace{}%
\AgdaBound{e₁}%
\>[24]\AgdaOperator{\AgdaFunction{⟧}}\AgdaSpace{}%
\AgdaBound{ρ}%
\>[29]\AgdaSymbol{=}\AgdaSpace{}%
\AgdaFunction{nest}\AgdaSpace{}%
\AgdaSymbol{(}\AgdaOperator{\AgdaFunction{⟦}}\AgdaSpace{}%
\AgdaBound{e}\AgdaSpace{}%
\AgdaOperator{\AgdaFunction{⟧}}\AgdaSpace{}%
\AgdaBound{ρ}\AgdaSymbol{)}\AgdaSpace{}%
\AgdaSymbol{(}\AgdaOperator{\AgdaFunction{⟦}}\AgdaSpace{}%
\AgdaBound{e₁}\AgdaSpace{}%
\AgdaOperator{\AgdaFunction{⟧}}\AgdaSpace{}%
\AgdaBound{ρ}\AgdaSymbol{)}\<%
\\
%
\>[2]\AgdaOperator{\AgdaFunction{⟦}}\AgdaSpace{}%
\AgdaInductiveConstructor{imapb}\AgdaSpace{}%
\AgdaBound{m}\AgdaSpace{}%
\AgdaBound{e}%
\>[24]\AgdaOperator{\AgdaFunction{⟧}}\AgdaSpace{}%
\AgdaBound{ρ}%
\>[29]\AgdaSymbol{=}\AgdaSpace{}%
\AgdaFunction{Ar.imapb}\AgdaSpace{}%
\AgdaSymbol{(λ}\AgdaSpace{}%
\AgdaBound{i}\AgdaSpace{}%
\AgdaSymbol{→}\AgdaSpace{}%
\AgdaOperator{\AgdaFunction{⟦}}\AgdaSpace{}%
\AgdaBound{e}\AgdaSpace{}%
\AgdaOperator{\AgdaFunction{⟧}}\AgdaSpace{}%
\AgdaSymbol{(}\AgdaBound{ρ}\AgdaSpace{}%
\AgdaOperator{\AgdaInductiveConstructor{,}}\AgdaSpace{}%
\AgdaBound{i}\AgdaSymbol{))}\AgdaSpace{}%
\AgdaBound{m}\<%
\\
%
\>[2]\AgdaOperator{\AgdaFunction{⟦}}\AgdaSpace{}%
\AgdaInductiveConstructor{selb}\AgdaSpace{}%
\AgdaBound{m}\AgdaSpace{}%
\AgdaBound{e}\AgdaSpace{}%
\AgdaBound{e₁}%
\>[24]\AgdaOperator{\AgdaFunction{⟧}}\AgdaSpace{}%
\AgdaBound{ρ}%
\>[29]\AgdaSymbol{=}\AgdaSpace{}%
\AgdaFunction{Ar.selb}\AgdaSpace{}%
\AgdaSymbol{(}\AgdaOperator{\AgdaFunction{⟦}}\AgdaSpace{}%
\AgdaBound{e}\AgdaSpace{}%
\AgdaOperator{\AgdaFunction{⟧}}\AgdaSpace{}%
\AgdaBound{ρ}\AgdaSymbol{)}\AgdaSpace{}%
\AgdaBound{m}\AgdaSpace{}%
\AgdaSymbol{(}\AgdaOperator{\AgdaFunction{⟦}}\AgdaSpace{}%
\AgdaBound{e₁}\AgdaSpace{}%
\AgdaOperator{\AgdaFunction{⟧}}\AgdaSpace{}%
\AgdaBound{ρ}\AgdaSymbol{)}\<%
\\
%
\>[2]\AgdaOperator{\AgdaFunction{⟦}}\AgdaSpace{}%
\AgdaInductiveConstructor{sum}\AgdaSpace{}%
\AgdaBound{e}%
\>[24]\AgdaOperator{\AgdaFunction{⟧}}\AgdaSpace{}%
\AgdaBound{ρ}%
\>[29]\AgdaSymbol{=}\AgdaSpace{}%
\AgdaFunction{Ar.sum}\AgdaSpace{}%
\AgdaSymbol{(}\AgdaFunction{Ar.zipWith}\AgdaSpace{}%
\AgdaOperator{\AgdaField{\AgdaUnderscore{}+\AgdaUnderscore{}}}\AgdaSymbol{)}\AgdaSpace{}%
\AgdaSymbol{(}\AgdaFunction{K}\AgdaSpace{}%
\AgdaSymbol{(}\AgdaField{fromℕ}\AgdaSpace{}%
\AgdaNumber{0}\AgdaSymbol{))}\AgdaSpace{}%
\AgdaSymbol{(λ}\AgdaSpace{}%
\AgdaBound{i}\AgdaSpace{}%
\AgdaSymbol{→}\AgdaSpace{}%
\AgdaOperator{\AgdaFunction{⟦}}\AgdaSpace{}%
\AgdaBound{e}\AgdaSpace{}%
\AgdaOperator{\AgdaFunction{⟧}}\AgdaSpace{}%
\AgdaSymbol{(}\AgdaBound{ρ}\AgdaSpace{}%
\AgdaOperator{\AgdaInductiveConstructor{,}}\AgdaSpace{}%
\AgdaBound{i}\AgdaSymbol{))}\<%
\\
%
\>[2]\AgdaOperator{\AgdaFunction{⟦}}\AgdaSpace{}%
\AgdaInductiveConstructor{zero-but}\AgdaSpace{}%
\AgdaBound{i}\AgdaSpace{}%
\AgdaBound{j}\AgdaSpace{}%
\AgdaBound{e}%
\>[24]\AgdaOperator{\AgdaFunction{⟧}}\AgdaSpace{}%
\AgdaBound{ρ}%
\>[29]\AgdaSymbol{=}\AgdaSpace{}%
\AgdaOperator{\AgdaFunction{if}}\AgdaSpace{}%
\AgdaOperator{\AgdaFunction{⌊}}\AgdaSpace{}%
\AgdaOperator{\AgdaFunction{⟦}}\AgdaSpace{}%
\AgdaBound{i}\AgdaSpace{}%
\AgdaOperator{\AgdaFunction{⟧}}\AgdaSpace{}%
\AgdaBound{ρ}\AgdaSpace{}%
\AgdaOperator{\AgdaFunction{≟ₚ}}\AgdaSpace{}%
\AgdaOperator{\AgdaFunction{⟦}}\AgdaSpace{}%
\AgdaBound{j}\AgdaSpace{}%
\AgdaOperator{\AgdaFunction{⟧}}\AgdaSpace{}%
\AgdaBound{ρ}\AgdaSpace{}%
\AgdaOperator{\AgdaFunction{⌋}}\AgdaSpace{}%
\AgdaOperator{\AgdaFunction{then}}\AgdaSpace{}%
\AgdaOperator{\AgdaFunction{⟦}}\AgdaSpace{}%
\AgdaBound{e}\AgdaSpace{}%
\AgdaOperator{\AgdaFunction{⟧}}\AgdaSpace{}%
\AgdaBound{ρ}\AgdaSpace{}%
\AgdaOperator{\AgdaFunction{else}}\AgdaSpace{}%
\AgdaFunction{K}\AgdaSpace{}%
\AgdaSymbol{(}\AgdaField{fromℕ}\AgdaSpace{}%
\AgdaNumber{0}\AgdaSymbol{)}\<%
\\
%
\>[2]\AgdaOperator{\AgdaFunction{⟦}}\AgdaSpace{}%
\AgdaInductiveConstructor{slide}\AgdaSpace{}%
\AgdaBound{e}\AgdaSpace{}%
\AgdaBound{p}\AgdaSpace{}%
\AgdaBound{e₁}\AgdaSpace{}%
\AgdaBound{s}%
\>[24]\AgdaOperator{\AgdaFunction{⟧}}\AgdaSpace{}%
\AgdaBound{ρ}%
\>[29]\AgdaSymbol{=}\AgdaSpace{}%
\AgdaFunction{Ar.slide}\AgdaSpace{}%
\AgdaSymbol{(}\AgdaOperator{\AgdaFunction{⟦}}\AgdaSpace{}%
\AgdaBound{e}\AgdaSpace{}%
\AgdaOperator{\AgdaFunction{⟧}}\AgdaSpace{}%
\AgdaBound{ρ}\AgdaSymbol{)}\AgdaSpace{}%
\AgdaBound{p}\AgdaSpace{}%
\AgdaSymbol{(}\AgdaOperator{\AgdaFunction{⟦}}\AgdaSpace{}%
\AgdaBound{e₁}\AgdaSpace{}%
\AgdaOperator{\AgdaFunction{⟧}}\AgdaSpace{}%
\AgdaBound{ρ}\AgdaSymbol{)}\AgdaSpace{}%
\AgdaBound{s}\<%
\\
%
\>[2]\AgdaOperator{\AgdaFunction{⟦}}\AgdaSpace{}%
\AgdaInductiveConstructor{backslide}\AgdaSpace{}%
\AgdaBound{e}\AgdaSpace{}%
\AgdaBound{e₁}\AgdaSpace{}%
\AgdaBound{s}\AgdaSpace{}%
\AgdaBound{p}%
\>[24]\AgdaOperator{\AgdaFunction{⟧}}\AgdaSpace{}%
\AgdaBound{ρ}%
\>[29]\AgdaSymbol{=}\AgdaSpace{}%
\AgdaFunction{Ar.backslide}\AgdaSpace{}%
\AgdaSymbol{(}\AgdaOperator{\AgdaFunction{⟦}}\AgdaSpace{}%
\AgdaBound{e}\AgdaSpace{}%
\AgdaOperator{\AgdaFunction{⟧}}\AgdaSpace{}%
\AgdaBound{ρ}\AgdaSymbol{)}\AgdaSpace{}%
\AgdaSymbol{(}\AgdaOperator{\AgdaFunction{⟦}}\AgdaSpace{}%
\AgdaBound{e₁}\AgdaSpace{}%
\AgdaOperator{\AgdaFunction{⟧}}\AgdaSpace{}%
\AgdaBound{ρ}\AgdaSymbol{)}\AgdaSpace{}%
\AgdaBound{s}\AgdaSpace{}%
\AgdaSymbol{(}\AgdaField{fromℕ}\AgdaSpace{}%
\AgdaNumber{0}\AgdaSymbol{)}\AgdaSpace{}%
\AgdaBound{p}\<%
\\
%
\>[2]\AgdaOperator{\AgdaFunction{⟦}}\AgdaSpace{}%
\AgdaInductiveConstructor{logistic}\AgdaSpace{}%
\AgdaBound{e}%
\>[24]\AgdaOperator{\AgdaFunction{⟧}}\AgdaSpace{}%
\AgdaBound{ρ}%
\>[29]\AgdaSymbol{=}\AgdaSpace{}%
\AgdaFunction{Ar.map}\AgdaSpace{}%
\AgdaFunction{logisticʳ}\AgdaSpace{}%
\AgdaSymbol{(}\AgdaOperator{\AgdaFunction{⟦}}\AgdaSpace{}%
\AgdaBound{e}\AgdaSpace{}%
\AgdaOperator{\AgdaFunction{⟧}}\AgdaSpace{}%
\AgdaBound{ρ}\AgdaSymbol{)}\<%
\\
%
\>[2]\AgdaOperator{\AgdaFunction{⟦}}\AgdaSpace{}%
\AgdaBound{e}\AgdaSpace{}%
\AgdaOperator{\AgdaInductiveConstructor{⊞}}\AgdaSpace{}%
\AgdaBound{e₁}%
\>[24]\AgdaOperator{\AgdaFunction{⟧}}\AgdaSpace{}%
\AgdaBound{ρ}%
\>[29]\AgdaSymbol{=}\AgdaSpace{}%
\AgdaFunction{Ar.zipWith}\AgdaSpace{}%
\AgdaOperator{\AgdaField{\AgdaUnderscore{}+\AgdaUnderscore{}}}\AgdaSpace{}%
\AgdaSymbol{(}\AgdaOperator{\AgdaFunction{⟦}}\AgdaSpace{}%
\AgdaBound{e}\AgdaSpace{}%
\AgdaOperator{\AgdaFunction{⟧}}\AgdaSpace{}%
\AgdaBound{ρ}\AgdaSymbol{)}\AgdaSpace{}%
\AgdaSymbol{(}\AgdaOperator{\AgdaFunction{⟦}}\AgdaSpace{}%
\AgdaBound{e₁}\AgdaSpace{}%
\AgdaOperator{\AgdaFunction{⟧}}\AgdaSpace{}%
\AgdaBound{ρ}\AgdaSymbol{)}\<%
\\
%
\>[2]\AgdaOperator{\AgdaFunction{⟦}}\AgdaSpace{}%
\AgdaBound{e}\AgdaSpace{}%
\AgdaOperator{\AgdaInductiveConstructor{⊠}}\AgdaSpace{}%
\AgdaBound{e₁}%
\>[24]\AgdaOperator{\AgdaFunction{⟧}}\AgdaSpace{}%
\AgdaBound{ρ}%
\>[29]\AgdaSymbol{=}\AgdaSpace{}%
\AgdaFunction{Ar.zipWith}\AgdaSpace{}%
\AgdaOperator{\AgdaField{\AgdaUnderscore{}*\AgdaUnderscore{}}}\AgdaSpace{}%
\AgdaSymbol{(}\AgdaOperator{\AgdaFunction{⟦}}\AgdaSpace{}%
\AgdaBound{e}\AgdaSpace{}%
\AgdaOperator{\AgdaFunction{⟧}}\AgdaSpace{}%
\AgdaBound{ρ}\AgdaSymbol{)}\AgdaSpace{}%
\AgdaSymbol{(}\AgdaOperator{\AgdaFunction{⟦}}\AgdaSpace{}%
\AgdaBound{e₁}\AgdaSpace{}%
\AgdaOperator{\AgdaFunction{⟧}}\AgdaSpace{}%
\AgdaBound{ρ}\AgdaSymbol{)}\<%
\\
%
\>[2]\AgdaOperator{\AgdaFunction{⟦}}\AgdaSpace{}%
\AgdaInductiveConstructor{scaledown}\AgdaSpace{}%
\AgdaBound{n}\AgdaSpace{}%
\AgdaBound{e}%
\>[24]\AgdaOperator{\AgdaFunction{⟧}}\AgdaSpace{}%
\AgdaBound{ρ}%
\>[29]\AgdaSymbol{=}\AgdaSpace{}%
\AgdaFunction{Ar.map}\AgdaSpace{}%
\AgdaSymbol{(}\AgdaOperator{\AgdaField{\AgdaUnderscore{}÷}}\AgdaSpace{}%
\AgdaField{fromℕ}\AgdaSpace{}%
\AgdaBound{n}\AgdaSymbol{)}\AgdaSpace{}%
\AgdaSymbol{(}\AgdaOperator{\AgdaFunction{⟦}}\AgdaSpace{}%
\AgdaBound{e}\AgdaSpace{}%
\AgdaOperator{\AgdaFunction{⟧}}\AgdaSpace{}%
\AgdaBound{ρ}\AgdaSymbol{)}\<%
\\
%
\>[2]\AgdaOperator{\AgdaFunction{⟦}}\AgdaSpace{}%
\AgdaInductiveConstructor{minus}\AgdaSpace{}%
\AgdaBound{e}%
\>[24]\AgdaOperator{\AgdaFunction{⟧}}\AgdaSpace{}%
\AgdaBound{ρ}%
\>[29]\AgdaSymbol{=}\AgdaSpace{}%
\AgdaFunction{Ar.map}\AgdaSpace{}%
\AgdaOperator{\AgdaField{-\AgdaUnderscore{}}}\AgdaSpace{}%
\AgdaSymbol{(}\AgdaOperator{\AgdaFunction{⟦}}\AgdaSpace{}%
\AgdaBound{e}\AgdaSpace{}%
\AgdaOperator{\AgdaFunction{⟧}}\AgdaSpace{}%
\AgdaBound{ρ}\AgdaSymbol{)}\<%
\\
%
\>[2]\AgdaOperator{\AgdaFunction{⟦}}\AgdaSpace{}%
\AgdaInductiveConstructor{let′}\AgdaSpace{}%
\AgdaBound{e}\AgdaSpace{}%
\AgdaBound{e₁}%
\>[24]\AgdaOperator{\AgdaFunction{⟧}}\AgdaSpace{}%
\AgdaBound{ρ}%
\>[29]\AgdaSymbol{=}\AgdaSpace{}%
\AgdaOperator{\AgdaFunction{⟦}}\AgdaSpace{}%
\AgdaBound{e₁}\AgdaSpace{}%
\AgdaOperator{\AgdaFunction{⟧}}\AgdaSpace{}%
\AgdaSymbol{(}\AgdaBound{ρ}\AgdaSpace{}%
\AgdaOperator{\AgdaInductiveConstructor{,}}\AgdaSpace{}%
\AgdaOperator{\AgdaFunction{⟦}}\AgdaSpace{}%
\AgdaBound{e}\AgdaSpace{}%
\AgdaOperator{\AgdaFunction{⟧}}\AgdaSpace{}%
\AgdaBound{ρ}\AgdaSymbol{)}\<%
\end{code}
With the above definition we can better explain the choices of language constructors.
The most important question to clarify is why do we have three array
constructors/eliminators.  As the only conceptual datatype of our language is
an array (of some shape), we do not have any direct way to talk about array elements.
Therefore, we model the type of array elements (scalars) as arrays of a singleton shape.
As can be seen, scalar selection \AC{selₛ} returns a singleton array
(application of \AF{K}) where all the element(s) are equal to the element we are
selecting.  The corresponding array constructor \AC{imapₛ} makes sure that if we
compute \AB{s} elements of the shape \AF{unit}, we produce an array of shape \AB{s}
(and not \AB{s} \AC{⊗} \AF{unit}).  This soves the problem of constructing arays
from scalars, but how do we construct an array of a product shape?  Given that we
have an expression in the context (\AB{Γ} \AC{▹} \AC{ix} \AB{s} \AC{▹} \AC{ix} \AB{p}),
we need to produce an array of \AB{s} \AC{⊗} \AB{p}.  There are several ways how to
solve this (\eg{} introducing nest/unnest or projections and pairing on indices),
but it is clear that we need something more than just an \AC{imapₛ}.
This is the reason to introduce \AC{imap}/\AC{sel} pair which operates on arrays
of product shapes.  As average pooling operates on blocked arrays, we need
a construction to express this in \AF{E}.  One could introduce explicit 
\AF{block}/\AF{unblock}, but we merge blocking/unlocking action with
imap/sel obtaining \AC{imapb}/\AC{selb}.  Our \AC{sum} constructor 
gets an argument in the extended context which is summation index, so 
conceptually we generate the values at every summation index before
summing these values together.  As a result, we only need
one instance of \AC{sum} which makes our expressions a little tidier.




\subsection{Weakening and Substitution}
\todo[inline]{Adjust the text}
As our language has explicit de Bruin variables (as opposed to HOAS~\cite{hoas} approaches),
we need the means to do weakening and substitution when we optimise expressions in \AF{E}.
Our language is intrinsically typed(shaped) which
makes the definition of both operations challenging.  However, this problem has
been well-understood, and we adopt the solution from~\cite{subst}.  We only show the
basic mechanisms of the definition, for full details refer to~\cite{subst}.

The key structure that gives rise to weakening and substitution is a function that
computes the context \AB{Γ} \emph{without} the variable \AB{v} 
(denoted \AB{Γ} \AF{/} \AB{v}).  Then we define the weakening for
variables (\AF{wkv}) and expressions (\AF{wk}) that take a variable or expression
in the context without the variable \AF{v} and return this variable or expression
in the context where \AB{v} is present.
\begin{code}[hide]%
\>[0]\AgdaKeyword{module}\AgdaSpace{}%
\AgdaModule{WkSub}\AgdaSpace{}%
\AgdaKeyword{where}\<%
\\
\>[0][@{}l@{\AgdaIndent{0}}]%
\>[2]\AgdaKeyword{open}\AgdaSpace{}%
\AgdaModule{Lang}\<%
\end{code}
\begin{mathpar}
\codeblock{\begin{code}%
%
\>[2]\AgdaKeyword{data}\AgdaSpace{}%
\AgdaOperator{\AgdaDatatype{\AgdaUnderscore{}⊆\AgdaUnderscore{}}}\AgdaSpace{}%
\AgdaSymbol{:}\AgdaSpace{}%
\AgdaDatatype{Ctx}\AgdaSpace{}%
\AgdaSymbol{→}\AgdaSpace{}%
\AgdaDatatype{Ctx}\AgdaSpace{}%
\AgdaSymbol{→}\AgdaSpace{}%
\AgdaPrimitive{Set}\AgdaSpace{}%
\AgdaKeyword{where}\<%
\\
\>[2][@{}l@{\AgdaIndent{0}}]%
\>[4]\AgdaInductiveConstructor{ε}%
\>[9]\AgdaSymbol{:}\AgdaSpace{}%
\AgdaInductiveConstructor{ε}\AgdaSpace{}%
\AgdaOperator{\AgdaDatatype{⊆}}\AgdaSpace{}%
\AgdaInductiveConstructor{ε}\<%
\\
%
\>[4]\AgdaInductiveConstructor{skip}\AgdaSpace{}%
\AgdaSymbol{:}\AgdaSpace{}%
\AgdaGeneralizable{Γ}\AgdaSpace{}%
\AgdaOperator{\AgdaDatatype{⊆}}\AgdaSpace{}%
\AgdaGeneralizable{Δ}\AgdaSpace{}%
\AgdaSymbol{→}\AgdaSpace{}%
\AgdaGeneralizable{Γ}\AgdaSpace{}%
\AgdaOperator{\AgdaDatatype{⊆}}\AgdaSpace{}%
\AgdaSymbol{(}\AgdaGeneralizable{Δ}\AgdaSpace{}%
\AgdaOperator{\AgdaInductiveConstructor{▹}}\AgdaSpace{}%
\AgdaGeneralizable{is}\AgdaSymbol{)}\<%
\\
%
\>[4]\AgdaInductiveConstructor{keep}\AgdaSpace{}%
\AgdaSymbol{:}\AgdaSpace{}%
\AgdaGeneralizable{Γ}\AgdaSpace{}%
\AgdaOperator{\AgdaDatatype{⊆}}\AgdaSpace{}%
\AgdaGeneralizable{Δ}\AgdaSpace{}%
\AgdaSymbol{→}\AgdaSpace{}%
\AgdaSymbol{(}\AgdaGeneralizable{Γ}\AgdaSpace{}%
\AgdaOperator{\AgdaInductiveConstructor{▹}}\AgdaSpace{}%
\AgdaGeneralizable{is}\AgdaSymbol{)}\AgdaSpace{}%
\AgdaOperator{\AgdaDatatype{⊆}}\AgdaSpace{}%
\AgdaSymbol{(}\AgdaGeneralizable{Δ}\AgdaSpace{}%
\AgdaOperator{\AgdaInductiveConstructor{▹}}\AgdaSpace{}%
\AgdaGeneralizable{is}\AgdaSymbol{)}\<%
\end{code}}
\and
\codeblock{\begin{code}  %
%
\>[2]\AgdaFunction{wkv}\AgdaSpace{}%
\AgdaSymbol{:}\AgdaSpace{}%
\AgdaGeneralizable{Γ}\AgdaSpace{}%
\AgdaOperator{\AgdaDatatype{⊆}}\AgdaSpace{}%
\AgdaGeneralizable{Δ}\AgdaSpace{}%
\AgdaSymbol{→}\AgdaSpace{}%
\AgdaGeneralizable{is}\AgdaSpace{}%
\AgdaOperator{\AgdaDatatype{∈}}\AgdaSpace{}%
\AgdaGeneralizable{Γ}\AgdaSpace{}%
\AgdaSymbol{→}\AgdaSpace{}%
\AgdaGeneralizable{is}\AgdaSpace{}%
\AgdaOperator{\AgdaDatatype{∈}}\AgdaSpace{}%
\AgdaGeneralizable{Δ}\<%
\\
%
\>[2]\AgdaFunction{wk}\AgdaSpace{}%
\AgdaSymbol{:}\AgdaSpace{}%
\AgdaGeneralizable{Γ}\AgdaSpace{}%
\AgdaOperator{\AgdaDatatype{⊆}}\AgdaSpace{}%
\AgdaGeneralizable{Δ}\AgdaSpace{}%
\AgdaSymbol{→}\AgdaSpace{}%
\AgdaDatatype{E}\AgdaSpace{}%
\AgdaGeneralizable{Γ}\AgdaSpace{}%
\AgdaGeneralizable{is}\AgdaSpace{}%
\AgdaSymbol{→}\AgdaSpace{}%
\AgdaDatatype{E}\AgdaSpace{}%
\AgdaGeneralizable{Δ}\AgdaSpace{}%
\AgdaGeneralizable{is}\<%
\end{code}}
\end{mathpar}

\todo[inline]{Expand this (or hide?)}
\begin{code}%
%
\>[2]\AgdaFunction{⊆-eq}\AgdaSpace{}%
\AgdaSymbol{:}\AgdaSpace{}%
\AgdaGeneralizable{Γ}\AgdaSpace{}%
\AgdaOperator{\AgdaDatatype{⊆}}\AgdaSpace{}%
\AgdaGeneralizable{Γ}\<%
\\
%
\>[2]\AgdaFunction{⊆-eq}\AgdaSpace{}%
\AgdaSymbol{\{}\AgdaInductiveConstructor{ε}\AgdaSymbol{\}}\AgdaSpace{}%
\AgdaSymbol{=}\AgdaSpace{}%
\AgdaInductiveConstructor{ε}\<%
\\
%
\>[2]\AgdaFunction{⊆-eq}\AgdaSpace{}%
\AgdaSymbol{\{}\AgdaBound{Γ}\AgdaSpace{}%
\AgdaOperator{\AgdaInductiveConstructor{▹}}\AgdaSpace{}%
\AgdaBound{x}\AgdaSymbol{\}}\AgdaSpace{}%
\AgdaSymbol{=}\AgdaSpace{}%
\AgdaInductiveConstructor{keep}\AgdaSpace{}%
\AgdaFunction{⊆-eq}\<%
\\
%
\\[\AgdaEmptyExtraSkip]%
%
\>[2]\AgdaOperator{\AgdaFunction{\AgdaUnderscore{}↑}}\AgdaSpace{}%
\AgdaSymbol{:}\AgdaSpace{}%
\AgdaDatatype{E}\AgdaSpace{}%
\AgdaGeneralizable{Γ}\AgdaSpace{}%
\AgdaGeneralizable{is}\AgdaSpace{}%
\AgdaSymbol{→}\AgdaSpace{}%
\AgdaDatatype{E}\AgdaSpace{}%
\AgdaSymbol{(}\AgdaGeneralizable{Γ}\AgdaSpace{}%
\AgdaOperator{\AgdaInductiveConstructor{▹}}\AgdaSpace{}%
\AgdaGeneralizable{ip}\AgdaSymbol{)}\AgdaSpace{}%
\AgdaGeneralizable{is}\<%
\\
%
\>[2]\AgdaOperator{\AgdaFunction{\AgdaUnderscore{}↑}}\AgdaSpace{}%
\AgdaSymbol{=}\AgdaSpace{}%
\AgdaFunction{wk}\AgdaSpace{}%
\AgdaSymbol{(}\AgdaInductiveConstructor{skip}\AgdaSpace{}%
\AgdaFunction{⊆-eq}\AgdaSymbol{)}\<%
\end{code}


% We give ourselves a nicer syntax for common cases when expressions
% are lifted into the context with extra one or two variables:
% \begin{code}[hide]
%   infixr 18 ↑_
%   infixr 18 ↑↑_
% \end{code}
% \begin{mathpar}
% \codeblock{\begin{code}
%   ↑_ : E Γ is → E (Γ ▹ ip) is
%   ↑_ = wk v₀
% \end{code}}
% \and
% \codeblock{\begin{code}
%   ↑↑_ : E Γ is → E (Γ ▹ ip ▹ iq) is
%   ↑↑_ = ↑_ ∘ ↑_
% \end{code}}
% \end{mathpar}
\begin{code}[hide]%
%
\>[2]\AgdaFunction{wkv}\AgdaSpace{}%
\AgdaSymbol{(}\AgdaInductiveConstructor{skip}\AgdaSpace{}%
\AgdaBound{s}\AgdaSymbol{)}\AgdaSpace{}%
\AgdaBound{v}\AgdaSpace{}%
\AgdaSymbol{=}\AgdaSpace{}%
\AgdaInductiveConstructor{vₛ}\AgdaSpace{}%
\AgdaSymbol{(}\AgdaFunction{wkv}\AgdaSpace{}%
\AgdaBound{s}\AgdaSpace{}%
\AgdaBound{v}\AgdaSymbol{)}\<%
\\
%
\>[2]\AgdaFunction{wkv}\AgdaSpace{}%
\AgdaSymbol{(}\AgdaInductiveConstructor{keep}\AgdaSpace{}%
\AgdaBound{s}\AgdaSymbol{)}\AgdaSpace{}%
\AgdaInductiveConstructor{v₀}\AgdaSpace{}%
\AgdaSymbol{=}\AgdaSpace{}%
\AgdaInductiveConstructor{v₀}\<%
\\
%
\>[2]\AgdaFunction{wkv}\AgdaSpace{}%
\AgdaSymbol{(}\AgdaInductiveConstructor{keep}\AgdaSpace{}%
\AgdaBound{s}\AgdaSymbol{)}\AgdaSpace{}%
\AgdaSymbol{(}\AgdaInductiveConstructor{vₛ}\AgdaSpace{}%
\AgdaBound{v}\AgdaSymbol{)}\AgdaSpace{}%
\AgdaSymbol{=}\AgdaSpace{}%
\AgdaInductiveConstructor{vₛ}\AgdaSpace{}%
\AgdaSymbol{(}\AgdaFunction{wkv}\AgdaSpace{}%
\AgdaBound{s}\AgdaSpace{}%
\AgdaBound{v}\AgdaSymbol{)}\<%
\\
%
\\[\AgdaEmptyExtraSkip]%
%
\>[2]\AgdaFunction{wk}\AgdaSpace{}%
\AgdaBound{s}\AgdaSpace{}%
\AgdaSymbol{(}\AgdaInductiveConstructor{var}\AgdaSpace{}%
\AgdaBound{x}\AgdaSymbol{)}\AgdaSpace{}%
\AgdaSymbol{=}\AgdaSpace{}%
\AgdaInductiveConstructor{var}\AgdaSpace{}%
\AgdaSymbol{(}\AgdaFunction{wkv}\AgdaSpace{}%
\AgdaBound{s}\AgdaSpace{}%
\AgdaBound{x}\AgdaSymbol{)}\<%
\\
%
\>[2]\AgdaFunction{wk}\AgdaSpace{}%
\AgdaBound{s}\AgdaSpace{}%
\AgdaInductiveConstructor{zero}\AgdaSpace{}%
\AgdaSymbol{=}\AgdaSpace{}%
\AgdaInductiveConstructor{zero}\<%
\\
%
\>[2]\AgdaFunction{wk}\AgdaSpace{}%
\AgdaBound{s}\AgdaSpace{}%
\AgdaInductiveConstructor{one}\AgdaSpace{}%
\AgdaSymbol{=}\AgdaSpace{}%
\AgdaInductiveConstructor{one}\<%
\\
%
\>[2]\AgdaFunction{wk}\AgdaSpace{}%
\AgdaBound{s}\AgdaSpace{}%
\AgdaSymbol{(}\AgdaInductiveConstructor{imaps}\AgdaSpace{}%
\AgdaBound{e}\AgdaSymbol{)}\AgdaSpace{}%
\AgdaSymbol{=}\AgdaSpace{}%
\AgdaInductiveConstructor{imaps}\AgdaSpace{}%
\AgdaSymbol{(}\AgdaFunction{wk}\AgdaSpace{}%
\AgdaSymbol{(}\AgdaInductiveConstructor{keep}\AgdaSpace{}%
\AgdaBound{s}\AgdaSymbol{)}\AgdaSpace{}%
\AgdaBound{e}\AgdaSymbol{)}\<%
\\
%
\>[2]\AgdaFunction{wk}\AgdaSpace{}%
\AgdaBound{s}\AgdaSpace{}%
\AgdaSymbol{(}\AgdaInductiveConstructor{sels}\AgdaSpace{}%
\AgdaBound{e}\AgdaSpace{}%
\AgdaBound{e₁}\AgdaSymbol{)}\AgdaSpace{}%
\AgdaSymbol{=}\AgdaSpace{}%
\AgdaInductiveConstructor{sels}\AgdaSpace{}%
\AgdaSymbol{(}\AgdaFunction{wk}\AgdaSpace{}%
\AgdaBound{s}\AgdaSpace{}%
\AgdaBound{e}\AgdaSymbol{)}\AgdaSpace{}%
\AgdaSymbol{(}\AgdaFunction{wk}\AgdaSpace{}%
\AgdaBound{s}\AgdaSpace{}%
\AgdaBound{e₁}\AgdaSymbol{)}\<%
\\
%
\>[2]\AgdaFunction{wk}\AgdaSpace{}%
\AgdaBound{s}\AgdaSpace{}%
\AgdaSymbol{(}\AgdaInductiveConstructor{imap}\AgdaSpace{}%
\AgdaBound{e}\AgdaSymbol{)}\AgdaSpace{}%
\AgdaSymbol{=}\AgdaSpace{}%
\AgdaInductiveConstructor{imap}\AgdaSpace{}%
\AgdaSymbol{(}\AgdaFunction{wk}\AgdaSpace{}%
\AgdaSymbol{(}\AgdaInductiveConstructor{keep}\AgdaSpace{}%
\AgdaBound{s}\AgdaSymbol{)}\AgdaSpace{}%
\AgdaBound{e}\AgdaSymbol{)}\<%
\\
%
\>[2]\AgdaFunction{wk}\AgdaSpace{}%
\AgdaBound{s}\AgdaSpace{}%
\AgdaSymbol{(}\AgdaInductiveConstructor{sel}\AgdaSpace{}%
\AgdaBound{e}\AgdaSpace{}%
\AgdaBound{e₁}\AgdaSymbol{)}\AgdaSpace{}%
\AgdaSymbol{=}\AgdaSpace{}%
\AgdaInductiveConstructor{sel}\AgdaSpace{}%
\AgdaSymbol{(}\AgdaFunction{wk}\AgdaSpace{}%
\AgdaBound{s}\AgdaSpace{}%
\AgdaBound{e}\AgdaSymbol{)}\AgdaSpace{}%
\AgdaSymbol{(}\AgdaFunction{wk}\AgdaSpace{}%
\AgdaBound{s}\AgdaSpace{}%
\AgdaBound{e₁}\AgdaSymbol{)}\<%
\\
%
\>[2]\AgdaFunction{wk}\AgdaSpace{}%
\AgdaBound{s}\AgdaSpace{}%
\AgdaSymbol{(}\AgdaInductiveConstructor{imapb}\AgdaSpace{}%
\AgdaBound{x}\AgdaSpace{}%
\AgdaBound{e}\AgdaSymbol{)}\AgdaSpace{}%
\AgdaSymbol{=}\AgdaSpace{}%
\AgdaInductiveConstructor{imapb}\AgdaSpace{}%
\AgdaBound{x}\AgdaSpace{}%
\AgdaSymbol{(}\AgdaFunction{wk}\AgdaSpace{}%
\AgdaSymbol{(}\AgdaInductiveConstructor{keep}\AgdaSpace{}%
\AgdaBound{s}\AgdaSymbol{)}\AgdaSpace{}%
\AgdaBound{e}\AgdaSymbol{)}\<%
\\
%
\>[2]\AgdaFunction{wk}\AgdaSpace{}%
\AgdaBound{s}\AgdaSpace{}%
\AgdaSymbol{(}\AgdaInductiveConstructor{selb}\AgdaSpace{}%
\AgdaBound{x}\AgdaSpace{}%
\AgdaBound{e}\AgdaSpace{}%
\AgdaBound{e₁}\AgdaSymbol{)}\AgdaSpace{}%
\AgdaSymbol{=}\AgdaSpace{}%
\AgdaInductiveConstructor{selb}\AgdaSpace{}%
\AgdaBound{x}\AgdaSpace{}%
\AgdaSymbol{(}\AgdaFunction{wk}\AgdaSpace{}%
\AgdaBound{s}\AgdaSpace{}%
\AgdaBound{e}\AgdaSymbol{)}\AgdaSpace{}%
\AgdaSymbol{(}\AgdaFunction{wk}\AgdaSpace{}%
\AgdaBound{s}\AgdaSpace{}%
\AgdaBound{e₁}\AgdaSymbol{)}\<%
\\
%
\>[2]\AgdaFunction{wk}\AgdaSpace{}%
\AgdaBound{s}\AgdaSpace{}%
\AgdaSymbol{(}\AgdaInductiveConstructor{sum}\AgdaSpace{}%
\AgdaBound{e}\AgdaSymbol{)}\AgdaSpace{}%
\AgdaSymbol{=}\AgdaSpace{}%
\AgdaInductiveConstructor{sum}\AgdaSpace{}%
\AgdaSymbol{(}\AgdaFunction{wk}\AgdaSpace{}%
\AgdaSymbol{(}\AgdaInductiveConstructor{keep}\AgdaSpace{}%
\AgdaBound{s}\AgdaSymbol{)}\AgdaSpace{}%
\AgdaBound{e}\AgdaSymbol{)}\<%
\\
%
\>[2]\AgdaFunction{wk}\AgdaSpace{}%
\AgdaBound{s}\AgdaSpace{}%
\AgdaSymbol{(}\AgdaInductiveConstructor{zero-but}\AgdaSpace{}%
\AgdaBound{e}\AgdaSpace{}%
\AgdaBound{e₁}\AgdaSpace{}%
\AgdaBound{e₂}\AgdaSymbol{)}\AgdaSpace{}%
\AgdaSymbol{=}\AgdaSpace{}%
\AgdaInductiveConstructor{zero-but}\AgdaSpace{}%
\AgdaSymbol{(}\AgdaFunction{wk}\AgdaSpace{}%
\AgdaBound{s}\AgdaSpace{}%
\AgdaBound{e}\AgdaSymbol{)}\AgdaSpace{}%
\AgdaSymbol{(}\AgdaFunction{wk}\AgdaSpace{}%
\AgdaBound{s}\AgdaSpace{}%
\AgdaBound{e₁}\AgdaSymbol{)}\AgdaSpace{}%
\AgdaSymbol{(}\AgdaFunction{wk}\AgdaSpace{}%
\AgdaBound{s}\AgdaSpace{}%
\AgdaBound{e₂}\AgdaSymbol{)}\<%
\\
%
\>[2]\AgdaFunction{wk}\AgdaSpace{}%
\AgdaBound{s}\AgdaSpace{}%
\AgdaSymbol{(}\AgdaInductiveConstructor{slide}\AgdaSpace{}%
\AgdaBound{e}\AgdaSpace{}%
\AgdaBound{x}\AgdaSpace{}%
\AgdaBound{e₁}\AgdaSpace{}%
\AgdaBound{x₁}\AgdaSymbol{)}\AgdaSpace{}%
\AgdaSymbol{=}\AgdaSpace{}%
\AgdaInductiveConstructor{slide}\AgdaSpace{}%
\AgdaSymbol{(}\AgdaFunction{wk}\AgdaSpace{}%
\AgdaBound{s}\AgdaSpace{}%
\AgdaBound{e}\AgdaSymbol{)}\AgdaSpace{}%
\AgdaBound{x}\AgdaSpace{}%
\AgdaSymbol{(}\AgdaFunction{wk}\AgdaSpace{}%
\AgdaBound{s}\AgdaSpace{}%
\AgdaBound{e₁}\AgdaSymbol{)}\AgdaSpace{}%
\AgdaBound{x₁}\<%
\\
%
\>[2]\AgdaFunction{wk}\AgdaSpace{}%
\AgdaBound{s}\AgdaSpace{}%
\AgdaSymbol{(}\AgdaInductiveConstructor{backslide}\AgdaSpace{}%
\AgdaBound{e}\AgdaSpace{}%
\AgdaBound{e₁}\AgdaSpace{}%
\AgdaBound{x}\AgdaSpace{}%
\AgdaBound{x₁}\AgdaSymbol{)}\AgdaSpace{}%
\AgdaSymbol{=}\AgdaSpace{}%
\AgdaInductiveConstructor{backslide}\AgdaSpace{}%
\AgdaSymbol{(}\AgdaFunction{wk}\AgdaSpace{}%
\AgdaBound{s}\AgdaSpace{}%
\AgdaBound{e}\AgdaSymbol{)}\AgdaSpace{}%
\AgdaSymbol{(}\AgdaFunction{wk}\AgdaSpace{}%
\AgdaBound{s}\AgdaSpace{}%
\AgdaBound{e₁}\AgdaSymbol{)}\AgdaSpace{}%
\AgdaBound{x}\AgdaSpace{}%
\AgdaBound{x₁}\<%
\\
%
\>[2]\AgdaFunction{wk}\AgdaSpace{}%
\AgdaBound{s}\AgdaSpace{}%
\AgdaSymbol{(}\AgdaInductiveConstructor{logistic}\AgdaSpace{}%
\AgdaBound{e}\AgdaSymbol{)}\AgdaSpace{}%
\AgdaSymbol{=}\AgdaSpace{}%
\AgdaInductiveConstructor{logistic}\AgdaSpace{}%
\AgdaSymbol{(}\AgdaFunction{wk}\AgdaSpace{}%
\AgdaBound{s}\AgdaSpace{}%
\AgdaBound{e}\AgdaSymbol{)}\<%
\\
%
\>[2]\AgdaFunction{wk}\AgdaSpace{}%
\AgdaBound{s}\AgdaSpace{}%
\AgdaSymbol{(}\AgdaInductiveConstructor{bin}\AgdaSpace{}%
\AgdaBound{x}\AgdaSpace{}%
\AgdaBound{e}\AgdaSpace{}%
\AgdaBound{e₁}\AgdaSymbol{)}\AgdaSpace{}%
\AgdaSymbol{=}\AgdaSpace{}%
\AgdaInductiveConstructor{bin}\AgdaSpace{}%
\AgdaBound{x}\AgdaSpace{}%
\AgdaSymbol{(}\AgdaFunction{wk}\AgdaSpace{}%
\AgdaBound{s}\AgdaSpace{}%
\AgdaBound{e}\AgdaSymbol{)}\AgdaSpace{}%
\AgdaSymbol{(}\AgdaFunction{wk}\AgdaSpace{}%
\AgdaBound{s}\AgdaSpace{}%
\AgdaBound{e₁}\AgdaSymbol{)}\<%
\\
%
\>[2]\AgdaFunction{wk}\AgdaSpace{}%
\AgdaBound{s}\AgdaSpace{}%
\AgdaSymbol{(}\AgdaInductiveConstructor{scaledown}\AgdaSpace{}%
\AgdaBound{x}\AgdaSpace{}%
\AgdaBound{e}\AgdaSymbol{)}\AgdaSpace{}%
\AgdaSymbol{=}\AgdaSpace{}%
\AgdaInductiveConstructor{scaledown}\AgdaSpace{}%
\AgdaBound{x}\AgdaSpace{}%
\AgdaSymbol{(}\AgdaFunction{wk}\AgdaSpace{}%
\AgdaBound{s}\AgdaSpace{}%
\AgdaBound{e}\AgdaSymbol{)}\<%
\\
%
\>[2]\AgdaFunction{wk}\AgdaSpace{}%
\AgdaBound{s}\AgdaSpace{}%
\AgdaSymbol{(}\AgdaInductiveConstructor{minus}\AgdaSpace{}%
\AgdaBound{e}\AgdaSymbol{)}\AgdaSpace{}%
\AgdaSymbol{=}\AgdaSpace{}%
\AgdaInductiveConstructor{minus}\AgdaSpace{}%
\AgdaSymbol{(}\AgdaFunction{wk}\AgdaSpace{}%
\AgdaBound{s}\AgdaSpace{}%
\AgdaBound{e}\AgdaSymbol{)}\<%
\\
%
\>[2]\AgdaFunction{wk}\AgdaSpace{}%
\AgdaBound{s}\AgdaSpace{}%
\AgdaSymbol{(}\AgdaInductiveConstructor{let′}\AgdaSpace{}%
\AgdaBound{e}\AgdaSpace{}%
\AgdaBound{e₁}\AgdaSymbol{)}\AgdaSpace{}%
\AgdaSymbol{=}\AgdaSpace{}%
\AgdaInductiveConstructor{let′}\AgdaSpace{}%
\AgdaSymbol{(}\AgdaFunction{wk}\AgdaSpace{}%
\AgdaBound{s}\AgdaSpace{}%
\AgdaBound{e}\AgdaSymbol{)}\AgdaSpace{}%
\AgdaSymbol{(}\AgdaFunction{wk}\AgdaSpace{}%
\AgdaSymbol{(}\AgdaInductiveConstructor{keep}\AgdaSpace{}%
\AgdaBound{s}\AgdaSymbol{)}\AgdaSpace{}%
\AgdaBound{e₁}\AgdaSymbol{)}\<%
\end{code} 

\todo[inline]{Say something about substitution}
\begin{code}%
%
\>[2]\AgdaKeyword{data}\AgdaSpace{}%
\AgdaDatatype{Sub}\AgdaSpace{}%
\AgdaSymbol{(}\AgdaBound{Γ}\AgdaSpace{}%
\AgdaSymbol{:}\AgdaSpace{}%
\AgdaDatatype{Ctx}\AgdaSymbol{)}\AgdaSpace{}%
\AgdaSymbol{:}\AgdaSpace{}%
\AgdaDatatype{Ctx}\AgdaSpace{}%
\AgdaSymbol{→}\AgdaSpace{}%
\AgdaPrimitive{Set}\AgdaSpace{}%
\AgdaKeyword{where}\<%
\\
\>[2][@{}l@{\AgdaIndent{0}}]%
\>[4]\AgdaInductiveConstructor{ε}%
\>[8]\AgdaSymbol{:}\AgdaSpace{}%
\AgdaDatatype{Sub}\AgdaSpace{}%
\AgdaBound{Γ}\AgdaSpace{}%
\AgdaInductiveConstructor{ε}\<%
\\
%
\>[4]\AgdaOperator{\AgdaInductiveConstructor{\AgdaUnderscore{}▹\AgdaUnderscore{}}}\AgdaSpace{}%
\AgdaSymbol{:}\AgdaSpace{}%
\AgdaDatatype{Sub}\AgdaSpace{}%
\AgdaBound{Γ}\AgdaSpace{}%
\AgdaGeneralizable{Δ}\AgdaSpace{}%
\AgdaSymbol{→}\AgdaSpace{}%
\AgdaDatatype{E}\AgdaSpace{}%
\AgdaBound{Γ}\AgdaSpace{}%
\AgdaGeneralizable{is}\AgdaSpace{}%
\AgdaSymbol{→}\AgdaSpace{}%
\AgdaDatatype{Sub}\AgdaSpace{}%
\AgdaBound{Γ}\AgdaSpace{}%
\AgdaSymbol{(}\AgdaGeneralizable{Δ}\AgdaSpace{}%
\AgdaOperator{\AgdaInductiveConstructor{▹}}\AgdaSpace{}%
\AgdaGeneralizable{is}\AgdaSymbol{)}\<%
\\
%
\\[\AgdaEmptyExtraSkip]%
%
\>[2]\AgdaFunction{sub}\AgdaSpace{}%
\AgdaSymbol{:}\AgdaSpace{}%
\AgdaDatatype{E}\AgdaSpace{}%
\AgdaGeneralizable{Δ}\AgdaSpace{}%
\AgdaGeneralizable{is}\AgdaSpace{}%
\AgdaSymbol{→}\AgdaSpace{}%
\AgdaDatatype{Sub}\AgdaSpace{}%
\AgdaGeneralizable{Γ}\AgdaSpace{}%
\AgdaGeneralizable{Δ}\AgdaSpace{}%
\AgdaSymbol{→}\AgdaSpace{}%
\AgdaDatatype{E}\AgdaSpace{}%
\AgdaGeneralizable{Γ}\AgdaSpace{}%
\AgdaGeneralizable{is}\<%
\end{code}
\begin{code}[hide]%
%
\>[2]\AgdaFunction{wks}\AgdaSpace{}%
\AgdaSymbol{:}\AgdaSpace{}%
\AgdaDatatype{Sub}\AgdaSpace{}%
\AgdaGeneralizable{Γ}\AgdaSpace{}%
\AgdaGeneralizable{Δ}\AgdaSpace{}%
\AgdaSymbol{→}\AgdaSpace{}%
\AgdaGeneralizable{Γ}\AgdaSpace{}%
\AgdaOperator{\AgdaDatatype{⊆}}\AgdaSpace{}%
\AgdaGeneralizable{Ψ}\AgdaSpace{}%
\AgdaSymbol{→}\AgdaSpace{}%
\AgdaDatatype{Sub}\AgdaSpace{}%
\AgdaGeneralizable{Ψ}\AgdaSpace{}%
\AgdaGeneralizable{Δ}\<%
\\
%
\>[2]\AgdaFunction{wks}\AgdaSpace{}%
\AgdaInductiveConstructor{ε}\AgdaSpace{}%
\AgdaBound{p}\AgdaSpace{}%
\AgdaSymbol{=}\AgdaSpace{}%
\AgdaInductiveConstructor{ε}\<%
\\
%
\>[2]\AgdaFunction{wks}\AgdaSpace{}%
\AgdaSymbol{(}\AgdaBound{s}\AgdaSpace{}%
\AgdaOperator{\AgdaInductiveConstructor{▹}}\AgdaSpace{}%
\AgdaBound{x}\AgdaSymbol{)}\AgdaSpace{}%
\AgdaBound{p}\AgdaSpace{}%
\AgdaSymbol{=}\AgdaSpace{}%
\AgdaSymbol{(}\AgdaFunction{wks}\AgdaSpace{}%
\AgdaBound{s}\AgdaSpace{}%
\AgdaBound{p}\AgdaSymbol{)}\AgdaSpace{}%
\AgdaOperator{\AgdaInductiveConstructor{▹}}\AgdaSpace{}%
\AgdaFunction{wk}\AgdaSpace{}%
\AgdaBound{p}\AgdaSpace{}%
\AgdaBound{x}\<%
\\
\>[0]\<%
\\
%
\>[2]\AgdaFunction{sdrop}\AgdaSpace{}%
\AgdaSymbol{:}\AgdaSpace{}%
\AgdaDatatype{Sub}\AgdaSpace{}%
\AgdaGeneralizable{Γ}\AgdaSpace{}%
\AgdaGeneralizable{Δ}\AgdaSpace{}%
\AgdaSymbol{→}\AgdaSpace{}%
\AgdaDatatype{Sub}\AgdaSpace{}%
\AgdaSymbol{(}\AgdaGeneralizable{Γ}\AgdaSpace{}%
\AgdaOperator{\AgdaInductiveConstructor{▹}}\AgdaSpace{}%
\AgdaGeneralizable{is}\AgdaSymbol{)}\AgdaSpace{}%
\AgdaGeneralizable{Δ}\<%
\\
%
\>[2]\AgdaFunction{sdrop}\AgdaSpace{}%
\AgdaBound{s}\AgdaSpace{}%
\AgdaSymbol{=}\AgdaSpace{}%
\AgdaFunction{wks}\AgdaSpace{}%
\AgdaBound{s}\AgdaSpace{}%
\AgdaSymbol{(}\AgdaInductiveConstructor{skip}\AgdaSpace{}%
\AgdaFunction{⊆-eq}\AgdaSymbol{)}\<%
\\
%
\\[\AgdaEmptyExtraSkip]%
%
\>[2]\AgdaFunction{skeep}\AgdaSpace{}%
\AgdaSymbol{:}\AgdaSpace{}%
\AgdaDatatype{Sub}\AgdaSpace{}%
\AgdaGeneralizable{Γ}\AgdaSpace{}%
\AgdaGeneralizable{Δ}\AgdaSpace{}%
\AgdaSymbol{→}\AgdaSpace{}%
\AgdaDatatype{Sub}\AgdaSpace{}%
\AgdaSymbol{(}\AgdaGeneralizable{Γ}\AgdaSpace{}%
\AgdaOperator{\AgdaInductiveConstructor{▹}}\AgdaSpace{}%
\AgdaGeneralizable{is}\AgdaSymbol{)}\AgdaSpace{}%
\AgdaSymbol{(}\AgdaGeneralizable{Δ}\AgdaSpace{}%
\AgdaOperator{\AgdaInductiveConstructor{▹}}\AgdaSpace{}%
\AgdaGeneralizable{is}\AgdaSymbol{)}\<%
\\
%
\>[2]\AgdaFunction{skeep}\AgdaSpace{}%
\AgdaBound{s}\AgdaSpace{}%
\AgdaSymbol{=}\AgdaSpace{}%
\AgdaFunction{sdrop}\AgdaSpace{}%
\AgdaBound{s}\AgdaSpace{}%
\AgdaOperator{\AgdaInductiveConstructor{▹}}\AgdaSpace{}%
\AgdaInductiveConstructor{var}\AgdaSpace{}%
\AgdaInductiveConstructor{v₀}\<%
\\
%
\\[\AgdaEmptyExtraSkip]%
%
\>[2]\AgdaFunction{subv}\AgdaSpace{}%
\AgdaSymbol{:}\AgdaSpace{}%
\AgdaDatatype{Sub}\AgdaSpace{}%
\AgdaGeneralizable{Γ}\AgdaSpace{}%
\AgdaGeneralizable{Δ}\AgdaSpace{}%
\AgdaSymbol{→}\AgdaSpace{}%
\AgdaGeneralizable{is}\AgdaSpace{}%
\AgdaOperator{\AgdaDatatype{∈}}\AgdaSpace{}%
\AgdaGeneralizable{Δ}\AgdaSpace{}%
\AgdaSymbol{→}\AgdaSpace{}%
\AgdaDatatype{E}\AgdaSpace{}%
\AgdaGeneralizable{Γ}\AgdaSpace{}%
\AgdaGeneralizable{is}\<%
\\
%
\>[2]\AgdaFunction{subv}\AgdaSpace{}%
\AgdaSymbol{(}\AgdaBound{s}\AgdaSpace{}%
\AgdaOperator{\AgdaInductiveConstructor{▹}}\AgdaSpace{}%
\AgdaBound{x}\AgdaSymbol{)}\AgdaSpace{}%
\AgdaInductiveConstructor{v₀}\AgdaSpace{}%
\AgdaSymbol{=}\AgdaSpace{}%
\AgdaBound{x}\<%
\\
%
\>[2]\AgdaFunction{subv}\AgdaSpace{}%
\AgdaSymbol{(}\AgdaBound{s}\AgdaSpace{}%
\AgdaOperator{\AgdaInductiveConstructor{▹}}\AgdaSpace{}%
\AgdaBound{x}\AgdaSymbol{)}\AgdaSpace{}%
\AgdaSymbol{(}\AgdaInductiveConstructor{vₛ}\AgdaSpace{}%
\AgdaBound{v}\AgdaSymbol{)}\AgdaSpace{}%
\AgdaSymbol{=}\AgdaSpace{}%
\AgdaFunction{subv}\AgdaSpace{}%
\AgdaBound{s}\AgdaSpace{}%
\AgdaBound{v}\<%
\\
\>[0]\<%
\\
%
\>[2]\AgdaFunction{sub}\AgdaSpace{}%
\AgdaSymbol{(}\AgdaInductiveConstructor{var}\AgdaSpace{}%
\AgdaBound{x}\AgdaSymbol{)}\AgdaSpace{}%
\AgdaBound{s}\AgdaSpace{}%
\AgdaSymbol{=}\AgdaSpace{}%
\AgdaFunction{subv}\AgdaSpace{}%
\AgdaBound{s}\AgdaSpace{}%
\AgdaBound{x}\<%
\\
%
\>[2]\AgdaFunction{sub}\AgdaSpace{}%
\AgdaInductiveConstructor{zero}\AgdaSpace{}%
\AgdaBound{s}\AgdaSpace{}%
\AgdaSymbol{=}\AgdaSpace{}%
\AgdaInductiveConstructor{zero}\<%
\\
%
\>[2]\AgdaFunction{sub}\AgdaSpace{}%
\AgdaInductiveConstructor{one}\AgdaSpace{}%
\AgdaBound{s}\AgdaSpace{}%
\AgdaSymbol{=}\AgdaSpace{}%
\AgdaInductiveConstructor{one}\<%
\\
%
\>[2]\AgdaFunction{sub}\AgdaSpace{}%
\AgdaSymbol{(}\AgdaInductiveConstructor{imaps}\AgdaSpace{}%
\AgdaBound{e}\AgdaSymbol{)}\AgdaSpace{}%
\AgdaBound{s}\AgdaSpace{}%
\AgdaSymbol{=}\AgdaSpace{}%
\AgdaInductiveConstructor{imaps}\AgdaSpace{}%
\AgdaSymbol{(}\AgdaFunction{sub}\AgdaSpace{}%
\AgdaBound{e}\AgdaSpace{}%
\AgdaSymbol{(}\AgdaFunction{skeep}\AgdaSpace{}%
\AgdaBound{s}\AgdaSymbol{))}\<%
\\
%
\>[2]\AgdaFunction{sub}\AgdaSpace{}%
\AgdaSymbol{(}\AgdaInductiveConstructor{sels}\AgdaSpace{}%
\AgdaBound{e}\AgdaSpace{}%
\AgdaBound{e₁}\AgdaSymbol{)}\AgdaSpace{}%
\AgdaBound{s}\AgdaSpace{}%
\AgdaSymbol{=}\AgdaSpace{}%
\AgdaInductiveConstructor{sels}\AgdaSpace{}%
\AgdaSymbol{(}\AgdaFunction{sub}\AgdaSpace{}%
\AgdaBound{e}\AgdaSpace{}%
\AgdaBound{s}\AgdaSymbol{)}\AgdaSpace{}%
\AgdaSymbol{(}\AgdaFunction{sub}\AgdaSpace{}%
\AgdaBound{e₁}\AgdaSpace{}%
\AgdaBound{s}\AgdaSymbol{)}\<%
\\
%
\>[2]\AgdaFunction{sub}\AgdaSpace{}%
\AgdaSymbol{(}\AgdaInductiveConstructor{imap}\AgdaSpace{}%
\AgdaBound{e}\AgdaSymbol{)}\AgdaSpace{}%
\AgdaBound{s}\AgdaSpace{}%
\AgdaSymbol{=}\AgdaSpace{}%
\AgdaInductiveConstructor{imap}\AgdaSpace{}%
\AgdaSymbol{(}\AgdaFunction{sub}\AgdaSpace{}%
\AgdaBound{e}\AgdaSpace{}%
\AgdaSymbol{(}\AgdaFunction{skeep}\AgdaSpace{}%
\AgdaBound{s}\AgdaSymbol{))}\<%
\\
%
\>[2]\AgdaFunction{sub}\AgdaSpace{}%
\AgdaSymbol{(}\AgdaInductiveConstructor{sel}\AgdaSpace{}%
\AgdaBound{e}\AgdaSpace{}%
\AgdaBound{e₁}\AgdaSymbol{)}\AgdaSpace{}%
\AgdaBound{s}\AgdaSpace{}%
\AgdaSymbol{=}\AgdaSpace{}%
\AgdaInductiveConstructor{sel}\AgdaSpace{}%
\AgdaSymbol{(}\AgdaFunction{sub}\AgdaSpace{}%
\AgdaBound{e}\AgdaSpace{}%
\AgdaBound{s}\AgdaSymbol{)}\AgdaSpace{}%
\AgdaSymbol{(}\AgdaFunction{sub}\AgdaSpace{}%
\AgdaBound{e₁}\AgdaSpace{}%
\AgdaBound{s}\AgdaSymbol{)}\<%
\\
%
\>[2]\AgdaFunction{sub}\AgdaSpace{}%
\AgdaSymbol{(}\AgdaInductiveConstructor{imapb}\AgdaSpace{}%
\AgdaBound{x}\AgdaSpace{}%
\AgdaBound{e}\AgdaSymbol{)}\AgdaSpace{}%
\AgdaBound{s}\AgdaSpace{}%
\AgdaSymbol{=}\AgdaSpace{}%
\AgdaInductiveConstructor{imapb}\AgdaSpace{}%
\AgdaBound{x}\AgdaSpace{}%
\AgdaSymbol{(}\AgdaFunction{sub}\AgdaSpace{}%
\AgdaBound{e}\AgdaSpace{}%
\AgdaSymbol{(}\AgdaFunction{skeep}\AgdaSpace{}%
\AgdaBound{s}\AgdaSymbol{))}\<%
\\
%
\>[2]\AgdaFunction{sub}\AgdaSpace{}%
\AgdaSymbol{(}\AgdaInductiveConstructor{selb}\AgdaSpace{}%
\AgdaBound{x}\AgdaSpace{}%
\AgdaBound{e}\AgdaSpace{}%
\AgdaBound{e₁}\AgdaSymbol{)}\AgdaSpace{}%
\AgdaBound{s}\AgdaSpace{}%
\AgdaSymbol{=}\AgdaSpace{}%
\AgdaInductiveConstructor{selb}\AgdaSpace{}%
\AgdaBound{x}\AgdaSpace{}%
\AgdaSymbol{(}\AgdaFunction{sub}\AgdaSpace{}%
\AgdaBound{e}\AgdaSpace{}%
\AgdaBound{s}\AgdaSymbol{)}\AgdaSpace{}%
\AgdaSymbol{(}\AgdaFunction{sub}\AgdaSpace{}%
\AgdaBound{e₁}\AgdaSpace{}%
\AgdaBound{s}\AgdaSymbol{)}\<%
\\
%
\>[2]\AgdaFunction{sub}\AgdaSpace{}%
\AgdaSymbol{(}\AgdaInductiveConstructor{sum}\AgdaSpace{}%
\AgdaBound{e}\AgdaSymbol{)}\AgdaSpace{}%
\AgdaBound{s}\AgdaSpace{}%
\AgdaSymbol{=}\AgdaSpace{}%
\AgdaInductiveConstructor{sum}\AgdaSpace{}%
\AgdaSymbol{(}\AgdaFunction{sub}\AgdaSpace{}%
\AgdaBound{e}\AgdaSpace{}%
\AgdaSymbol{(}\AgdaFunction{skeep}\AgdaSpace{}%
\AgdaBound{s}\AgdaSymbol{))}\<%
\\
%
\>[2]\AgdaFunction{sub}\AgdaSpace{}%
\AgdaSymbol{(}\AgdaInductiveConstructor{zero-but}\AgdaSpace{}%
\AgdaBound{e}\AgdaSpace{}%
\AgdaBound{e₁}\AgdaSpace{}%
\AgdaBound{e₂}\AgdaSymbol{)}\AgdaSpace{}%
\AgdaBound{s}\AgdaSpace{}%
\AgdaSymbol{=}\AgdaSpace{}%
\AgdaInductiveConstructor{zero-but}\AgdaSpace{}%
\AgdaSymbol{(}\AgdaFunction{sub}\AgdaSpace{}%
\AgdaBound{e}\AgdaSpace{}%
\AgdaBound{s}\AgdaSymbol{)}\AgdaSpace{}%
\AgdaSymbol{(}\AgdaFunction{sub}\AgdaSpace{}%
\AgdaBound{e₁}\AgdaSpace{}%
\AgdaBound{s}\AgdaSymbol{)}\AgdaSpace{}%
\AgdaSymbol{(}\AgdaFunction{sub}\AgdaSpace{}%
\AgdaBound{e₂}\AgdaSpace{}%
\AgdaBound{s}\AgdaSymbol{)}\<%
\\
%
\>[2]\AgdaFunction{sub}\AgdaSpace{}%
\AgdaSymbol{(}\AgdaInductiveConstructor{slide}\AgdaSpace{}%
\AgdaBound{e}\AgdaSpace{}%
\AgdaBound{x}\AgdaSpace{}%
\AgdaBound{e₁}\AgdaSpace{}%
\AgdaBound{x₁}\AgdaSymbol{)}\AgdaSpace{}%
\AgdaBound{s}\AgdaSpace{}%
\AgdaSymbol{=}\AgdaSpace{}%
\AgdaInductiveConstructor{slide}\AgdaSpace{}%
\AgdaSymbol{(}\AgdaFunction{sub}\AgdaSpace{}%
\AgdaBound{e}\AgdaSpace{}%
\AgdaBound{s}\AgdaSymbol{)}\AgdaSpace{}%
\AgdaBound{x}\AgdaSpace{}%
\AgdaSymbol{(}\AgdaFunction{sub}\AgdaSpace{}%
\AgdaBound{e₁}\AgdaSpace{}%
\AgdaBound{s}\AgdaSymbol{)}\AgdaSpace{}%
\AgdaBound{x₁}\<%
\\
%
\>[2]\AgdaFunction{sub}\AgdaSpace{}%
\AgdaSymbol{(}\AgdaInductiveConstructor{backslide}\AgdaSpace{}%
\AgdaBound{e}\AgdaSpace{}%
\AgdaBound{e₁}\AgdaSpace{}%
\AgdaBound{x}\AgdaSpace{}%
\AgdaBound{x₁}\AgdaSymbol{)}\AgdaSpace{}%
\AgdaBound{s}\AgdaSpace{}%
\AgdaSymbol{=}\AgdaSpace{}%
\AgdaInductiveConstructor{backslide}\AgdaSpace{}%
\AgdaSymbol{(}\AgdaFunction{sub}\AgdaSpace{}%
\AgdaBound{e}\AgdaSpace{}%
\AgdaBound{s}\AgdaSymbol{)}\AgdaSpace{}%
\AgdaSymbol{(}\AgdaFunction{sub}\AgdaSpace{}%
\AgdaBound{e₁}\AgdaSpace{}%
\AgdaBound{s}\AgdaSymbol{)}\AgdaSpace{}%
\AgdaBound{x}\AgdaSpace{}%
\AgdaBound{x₁}\<%
\\
%
\>[2]\AgdaFunction{sub}\AgdaSpace{}%
\AgdaSymbol{(}\AgdaInductiveConstructor{logistic}\AgdaSpace{}%
\AgdaBound{e}\AgdaSymbol{)}\AgdaSpace{}%
\AgdaBound{s}\AgdaSpace{}%
\AgdaSymbol{=}\AgdaSpace{}%
\AgdaInductiveConstructor{logistic}\AgdaSpace{}%
\AgdaSymbol{(}\AgdaFunction{sub}\AgdaSpace{}%
\AgdaBound{e}\AgdaSpace{}%
\AgdaBound{s}\AgdaSymbol{)}\<%
\\
%
\>[2]\AgdaFunction{sub}\AgdaSpace{}%
\AgdaSymbol{(}\AgdaInductiveConstructor{bin}\AgdaSpace{}%
\AgdaBound{x}\AgdaSpace{}%
\AgdaBound{e}\AgdaSpace{}%
\AgdaBound{e₁}\AgdaSymbol{)}\AgdaSpace{}%
\AgdaBound{s}\AgdaSpace{}%
\AgdaSymbol{=}\AgdaSpace{}%
\AgdaInductiveConstructor{bin}\AgdaSpace{}%
\AgdaBound{x}\AgdaSpace{}%
\AgdaSymbol{(}\AgdaFunction{sub}\AgdaSpace{}%
\AgdaBound{e}\AgdaSpace{}%
\AgdaBound{s}\AgdaSymbol{)}\AgdaSpace{}%
\AgdaSymbol{(}\AgdaFunction{sub}\AgdaSpace{}%
\AgdaBound{e₁}\AgdaSpace{}%
\AgdaBound{s}\AgdaSymbol{)}\<%
\\
%
\>[2]\AgdaFunction{sub}\AgdaSpace{}%
\AgdaSymbol{(}\AgdaInductiveConstructor{scaledown}\AgdaSpace{}%
\AgdaBound{x}\AgdaSpace{}%
\AgdaBound{e}\AgdaSymbol{)}\AgdaSpace{}%
\AgdaBound{s}\AgdaSpace{}%
\AgdaSymbol{=}\AgdaSpace{}%
\AgdaInductiveConstructor{scaledown}\AgdaSpace{}%
\AgdaBound{x}\AgdaSpace{}%
\AgdaSymbol{(}\AgdaFunction{sub}\AgdaSpace{}%
\AgdaBound{e}\AgdaSpace{}%
\AgdaBound{s}\AgdaSymbol{)}\<%
\\
%
\>[2]\AgdaFunction{sub}\AgdaSpace{}%
\AgdaSymbol{(}\AgdaInductiveConstructor{minus}\AgdaSpace{}%
\AgdaBound{e}\AgdaSymbol{)}\AgdaSpace{}%
\AgdaBound{s}\AgdaSpace{}%
\AgdaSymbol{=}\AgdaSpace{}%
\AgdaInductiveConstructor{minus}\AgdaSpace{}%
\AgdaSymbol{(}\AgdaFunction{sub}\AgdaSpace{}%
\AgdaBound{e}\AgdaSpace{}%
\AgdaBound{s}\AgdaSymbol{)}\<%
\\
%
\>[2]\AgdaFunction{sub}\AgdaSpace{}%
\AgdaSymbol{(}\AgdaInductiveConstructor{let′}\AgdaSpace{}%
\AgdaBound{e}\AgdaSpace{}%
\AgdaBound{e₁}\AgdaSymbol{)}\AgdaSpace{}%
\AgdaBound{s}\AgdaSpace{}%
\AgdaSymbol{=}\AgdaSpace{}%
\AgdaInductiveConstructor{let′}\AgdaSpace{}%
\AgdaSymbol{(}\AgdaFunction{sub}\AgdaSpace{}%
\AgdaBound{e}\AgdaSpace{}%
\AgdaBound{s}\AgdaSymbol{)}\AgdaSpace{}%
\AgdaSymbol{(}\AgdaFunction{sub}\AgdaSpace{}%
\AgdaBound{e₁}\AgdaSpace{}%
\AgdaSymbol{(}\AgdaFunction{skeep}\AgdaSpace{}%
\AgdaBound{s}\AgdaSymbol{))}\<%
\\
%
\\[\AgdaEmptyExtraSkip]%
%
\>[2]\AgdaOperator{\AgdaFunction{\AgdaUnderscore{}∙ˢ\AgdaUnderscore{}}}\AgdaSpace{}%
\AgdaSymbol{:}\AgdaSpace{}%
\AgdaDatatype{Sub}\AgdaSpace{}%
\AgdaGeneralizable{Δ}\AgdaSpace{}%
\AgdaGeneralizable{Ψ}\AgdaSpace{}%
\AgdaSymbol{→}\AgdaSpace{}%
\AgdaDatatype{Sub}\AgdaSpace{}%
\AgdaGeneralizable{Γ}\AgdaSpace{}%
\AgdaGeneralizable{Δ}\AgdaSpace{}%
\AgdaSymbol{→}\AgdaSpace{}%
\AgdaDatatype{Sub}\AgdaSpace{}%
\AgdaGeneralizable{Γ}\AgdaSpace{}%
\AgdaGeneralizable{Ψ}\<%
\\
%
\>[2]\AgdaInductiveConstructor{ε}\AgdaSpace{}%
\AgdaOperator{\AgdaFunction{∙ˢ}}\AgdaSpace{}%
\AgdaBound{t}\AgdaSpace{}%
\AgdaSymbol{=}\AgdaSpace{}%
\AgdaInductiveConstructor{ε}\<%
\\
%
\>[2]\AgdaSymbol{(}\AgdaBound{s}\AgdaSpace{}%
\AgdaOperator{\AgdaInductiveConstructor{▹}}\AgdaSpace{}%
\AgdaBound{x}\AgdaSymbol{)}\AgdaSpace{}%
\AgdaOperator{\AgdaFunction{∙ˢ}}\AgdaSpace{}%
\AgdaBound{t}\AgdaSpace{}%
\AgdaSymbol{=}\AgdaSpace{}%
\AgdaSymbol{(}\AgdaBound{s}\AgdaSpace{}%
\AgdaOperator{\AgdaFunction{∙ˢ}}\AgdaSpace{}%
\AgdaBound{t}\AgdaSymbol{)}\AgdaSpace{}%
\AgdaOperator{\AgdaInductiveConstructor{▹}}\AgdaSpace{}%
\AgdaFunction{sub}\AgdaSpace{}%
\AgdaBound{x}\AgdaSpace{}%
\AgdaBound{t}\<%
\end{code}
We can define identity substitution as folows:
\begin{code}%
%
\>[2]\AgdaFunction{sub-id}\AgdaSpace{}%
\AgdaSymbol{:}\AgdaSpace{}%
\AgdaDatatype{Sub}\AgdaSpace{}%
\AgdaGeneralizable{Γ}\AgdaSpace{}%
\AgdaGeneralizable{Γ}\<%
\\
%
\>[2]\AgdaFunction{sub-id}\AgdaSpace{}%
\AgdaSymbol{\{}\AgdaInductiveConstructor{ε}\AgdaSymbol{\}}\AgdaSpace{}%
\AgdaSymbol{=}\AgdaSpace{}%
\AgdaInductiveConstructor{ε}\<%
\\
%
\>[2]\AgdaFunction{sub-id}\AgdaSpace{}%
\AgdaSymbol{\{}\AgdaBound{Γ}\AgdaSpace{}%
\AgdaOperator{\AgdaInductiveConstructor{▹}}\AgdaSpace{}%
\AgdaBound{x}\AgdaSymbol{\}}\AgdaSpace{}%
\AgdaSymbol{=}\AgdaSpace{}%
\AgdaFunction{skeep}\AgdaSpace{}%
\AgdaFunction{sub-id}\<%
\end{code}
As our context do not encode explicit dependencies between the variables,
we can easily define a substitution that swaps two top variables in the
context.  This will be used later for optimising programs in our DSL.
\begin{code}%
%
\>[2]\AgdaFunction{sub-swap}\AgdaSpace{}%
\AgdaSymbol{:}\AgdaSpace{}%
\AgdaDatatype{Sub}\AgdaSpace{}%
\AgdaSymbol{(}\AgdaGeneralizable{Γ}\AgdaSpace{}%
\AgdaOperator{\AgdaInductiveConstructor{▹}}\AgdaSpace{}%
\AgdaGeneralizable{is}\AgdaSpace{}%
\AgdaOperator{\AgdaInductiveConstructor{▹}}\AgdaSpace{}%
\AgdaGeneralizable{ip}\AgdaSymbol{)}\AgdaSpace{}%
\AgdaSymbol{(}\AgdaGeneralizable{Γ}\AgdaSpace{}%
\AgdaOperator{\AgdaInductiveConstructor{▹}}\AgdaSpace{}%
\AgdaGeneralizable{ip}\AgdaSpace{}%
\AgdaOperator{\AgdaInductiveConstructor{▹}}\AgdaSpace{}%
\AgdaGeneralizable{is}\AgdaSymbol{)}\<%
\\
%
\>[2]\AgdaFunction{sub-swap}\AgdaSpace{}%
\AgdaSymbol{=}\AgdaSpace{}%
\AgdaSymbol{(}\AgdaFunction{sdrop}\AgdaSpace{}%
\AgdaSymbol{(}\AgdaFunction{sdrop}\AgdaSpace{}%
\AgdaFunction{sub-id}\AgdaSymbol{)}\AgdaSpace{}%
\AgdaOperator{\AgdaInductiveConstructor{▹}}\AgdaSpace{}%
\AgdaInductiveConstructor{var}\AgdaSpace{}%
\AgdaInductiveConstructor{v₀}\AgdaSymbol{)}\AgdaSpace{}%
\AgdaOperator{\AgdaInductiveConstructor{▹}}\AgdaSpace{}%
\AgdaInductiveConstructor{var}\AgdaSpace{}%
\AgdaSymbol{(}\AgdaInductiveConstructor{vₛ}\AgdaSpace{}%
\AgdaInductiveConstructor{v₀}\AgdaSymbol{)}\<%
\end{code}

\paragraph{Syntax}
\todo[inline]{Explain that we want to simplify the life of programers by
introducing HOAS-like syntax, which is difficult, as our DSL is intrinsically-typed.}

\begin{code}[hide]%
\>[0]\AgdaKeyword{module}\AgdaSpace{}%
\AgdaModule{Syntax}\AgdaSpace{}%
\AgdaKeyword{where}\<%
\\
\>[0][@{}l@{\AgdaIndent{0}}]%
\>[2]\AgdaKeyword{open}\AgdaSpace{}%
\AgdaModule{Lang}\<%
\\
%
\>[2]\AgdaKeyword{open}\AgdaSpace{}%
\AgdaKeyword{import}\AgdaSpace{}%
\AgdaModule{Data.List}\AgdaSpace{}%
\AgdaSymbol{as}\AgdaSpace{}%
\AgdaModule{L}\AgdaSpace{}%
\AgdaKeyword{using}\AgdaSpace{}%
\AgdaSymbol{(}\AgdaDatatype{List}\AgdaSymbol{;}\AgdaSpace{}%
\AgdaInductiveConstructor{[]}\AgdaSymbol{;}\AgdaSpace{}%
\AgdaOperator{\AgdaInductiveConstructor{\AgdaUnderscore{}∷\AgdaUnderscore{}}}\AgdaSymbol{)}\<%
\\
%
\>[2]\AgdaKeyword{open}\AgdaSpace{}%
\AgdaModule{Array}\AgdaSpace{}%
\AgdaKeyword{hiding}\AgdaSpace{}%
\AgdaSymbol{(}\AgdaFunction{sum}\AgdaSymbol{)}\<%
\end{code}
\begin{code}%
%
\>[2]\AgdaKeyword{data}\AgdaSpace{}%
\AgdaDatatype{Prefix}\AgdaSpace{}%
\AgdaSymbol{:}\AgdaSpace{}%
\AgdaSymbol{(}\AgdaBound{Γ}\AgdaSpace{}%
\AgdaBound{Δ}\AgdaSpace{}%
\AgdaSymbol{:}\AgdaSpace{}%
\AgdaDatatype{Ctx}\AgdaSymbol{)}\AgdaSpace{}%
\AgdaSymbol{→}\AgdaSpace{}%
\AgdaPrimitive{Set}\AgdaSpace{}%
\AgdaKeyword{where}\<%
\\
\>[2][@{}l@{\AgdaIndent{0}}]%
\>[4]\AgdaKeyword{instance}\<%
\\
\>[4][@{}l@{\AgdaIndent{0}}]%
\>[6]\AgdaInductiveConstructor{zero}\AgdaSpace{}%
\AgdaSymbol{:}\AgdaSpace{}%
\AgdaDatatype{Prefix}\AgdaSpace{}%
\AgdaGeneralizable{Γ}\AgdaSpace{}%
\AgdaGeneralizable{Γ}\<%
\\
%
\>[6]\AgdaInductiveConstructor{suc}%
\>[11]\AgdaSymbol{:}\AgdaSpace{}%
\AgdaSymbol{⦃}\AgdaSpace{}%
\AgdaDatatype{Prefix}\AgdaSpace{}%
\AgdaGeneralizable{Γ}\AgdaSpace{}%
\AgdaGeneralizable{Δ}\AgdaSpace{}%
\AgdaSymbol{⦄}\AgdaSpace{}%
\AgdaSymbol{→}\AgdaSpace{}%
\AgdaDatatype{Prefix}\AgdaSpace{}%
\AgdaGeneralizable{Γ}\AgdaSpace{}%
\AgdaSymbol{(}\AgdaGeneralizable{Δ}\AgdaSpace{}%
\AgdaOperator{\AgdaInductiveConstructor{▹}}\AgdaSpace{}%
\AgdaGeneralizable{is}\AgdaSymbol{)}\<%
\\
%
\\[\AgdaEmptyExtraSkip]%
%
\>[2]\AgdaComment{--\ A\ term\ that\ can\ be\ lifted\ into\ larger\ contexts}\<%
\\
%
\>[2]\AgdaFunction{GE}\AgdaSpace{}%
\AgdaSymbol{:}\AgdaSpace{}%
\AgdaDatatype{Ctx}\AgdaSpace{}%
\AgdaSymbol{→}\AgdaSpace{}%
\AgdaDatatype{IS}\AgdaSpace{}%
\AgdaSymbol{→}\AgdaSpace{}%
\AgdaPrimitive{Set}\<%
\\
%
\>[2]\AgdaFunction{GE}\AgdaSpace{}%
\AgdaBound{Γ}\AgdaSpace{}%
\AgdaBound{is}\AgdaSpace{}%
\AgdaSymbol{=}\AgdaSpace{}%
\AgdaSymbol{∀}\AgdaSpace{}%
\AgdaSymbol{\{}\AgdaBound{Δ}\AgdaSymbol{\}}\AgdaSpace{}%
\AgdaSymbol{→}\AgdaSpace{}%
\AgdaSymbol{⦃}\AgdaSpace{}%
\AgdaDatatype{Prefix}\AgdaSpace{}%
\AgdaBound{Γ}\AgdaSpace{}%
\AgdaBound{Δ}\AgdaSpace{}%
\AgdaSymbol{⦄}\AgdaSpace{}%
\AgdaSymbol{→}\AgdaSpace{}%
\AgdaDatatype{E}\AgdaSpace{}%
\AgdaBound{Δ}\AgdaSpace{}%
\AgdaBound{is}\<%
\\
%
\\[\AgdaEmptyExtraSkip]%
%
\>[2]\AgdaComment{--\ A\ variable\ that\ can\ be\ lifted\ into\ larger\ contexts}\<%
\\
%
\>[2]\AgdaFunction{GVar}\AgdaSpace{}%
\AgdaSymbol{:}\AgdaSpace{}%
\AgdaDatatype{Ctx}\AgdaSpace{}%
\AgdaSymbol{→}\AgdaSpace{}%
\AgdaDatatype{IS}\AgdaSpace{}%
\AgdaSymbol{→}\AgdaSpace{}%
\AgdaPrimitive{Set}\<%
\\
%
\>[2]\AgdaFunction{GVar}\AgdaSpace{}%
\AgdaBound{Γ}\AgdaSpace{}%
\AgdaBound{is}\AgdaSpace{}%
\AgdaSymbol{=}\AgdaSpace{}%
\AgdaSymbol{∀}\AgdaSpace{}%
\AgdaSymbol{\{}\AgdaBound{Δ}\AgdaSymbol{\}}\AgdaSpace{}%
\AgdaSymbol{→}\AgdaSpace{}%
\AgdaSymbol{⦃}\AgdaSpace{}%
\AgdaBound{p}\AgdaSpace{}%
\AgdaSymbol{:}\AgdaSpace{}%
\AgdaDatatype{Prefix}\AgdaSpace{}%
\AgdaBound{Γ}\AgdaSpace{}%
\AgdaBound{Δ}\AgdaSpace{}%
\AgdaSymbol{⦄}\AgdaSpace{}%
\AgdaSymbol{→}\AgdaSpace{}%
\AgdaBound{is}\AgdaSpace{}%
\AgdaOperator{\AgdaDatatype{∈}}\AgdaSpace{}%
\AgdaBound{Δ}\<%
\\
%
\\[\AgdaEmptyExtraSkip]%
%
\>[2]\AgdaComment{--\ Lift\ var}\<%
\\
%
\>[2]\AgdaFunction{V}\AgdaSpace{}%
\AgdaSymbol{:}\AgdaSpace{}%
\AgdaGeneralizable{is}\AgdaSpace{}%
\AgdaOperator{\AgdaDatatype{∈}}\AgdaSpace{}%
\AgdaGeneralizable{Γ}\AgdaSpace{}%
\AgdaSymbol{→}\AgdaSpace{}%
\AgdaFunction{GVar}\AgdaSpace{}%
\AgdaGeneralizable{Γ}\AgdaSpace{}%
\AgdaGeneralizable{is}\<%
\\
%
\>[2]\AgdaFunction{V}\AgdaSpace{}%
\AgdaBound{v}\AgdaSpace{}%
\AgdaSymbol{⦃}\AgdaSpace{}%
\AgdaArgument{p}\AgdaSpace{}%
\AgdaSymbol{=}\AgdaSpace{}%
\AgdaInductiveConstructor{zero}\AgdaSpace{}%
\AgdaSymbol{⦄}\AgdaSpace{}%
\AgdaSymbol{=}\AgdaSpace{}%
\AgdaBound{v}\<%
\\
%
\>[2]\AgdaFunction{V}\AgdaSpace{}%
\AgdaBound{v}\AgdaSpace{}%
\AgdaSymbol{⦃}\AgdaSpace{}%
\AgdaArgument{p}\AgdaSpace{}%
\AgdaSymbol{=}\AgdaSpace{}%
\AgdaInductiveConstructor{suc}%
\>[17]\AgdaSymbol{⦄}\AgdaSpace{}%
\AgdaSymbol{=}\AgdaSpace{}%
\AgdaInductiveConstructor{vₛ}\AgdaSpace{}%
\AgdaSymbol{(}\AgdaFunction{V}\AgdaSpace{}%
\AgdaBound{v}\AgdaSymbol{)}\<%
\end{code}
We can implement HOAS operators that will allow to use bound variables
under further binders.
\begin{code}%
%
\>[2]\AgdaComment{--\ Use\ GE\ GVar\ and\ V\ to\ define\ HOAS-style\ imap,\ imaps,\ and\ impab}\<%
\\
%
\>[2]\AgdaFunction{Imap}%
\>[1600I]\AgdaSymbol{:}\AgdaSpace{}%
\AgdaSymbol{∀}\AgdaSpace{}%
\AgdaSymbol{\{}\AgdaBound{Γ}\AgdaSymbol{\}}\<%
\\
\>[.][@{}l@{}]\<[1600I]%
\>[7]\AgdaSymbol{→}\AgdaSpace{}%
\AgdaSymbol{(}\AgdaFunction{GE}\AgdaSpace{}%
\AgdaSymbol{(}\AgdaBound{Γ}\AgdaSpace{}%
\AgdaOperator{\AgdaInductiveConstructor{▹}}\AgdaSpace{}%
\AgdaInductiveConstructor{ix}\AgdaSpace{}%
\AgdaGeneralizable{s}\AgdaSymbol{)}\AgdaSpace{}%
\AgdaSymbol{(}\AgdaInductiveConstructor{ix}\AgdaSpace{}%
\AgdaGeneralizable{s}\AgdaSymbol{)}\AgdaSpace{}%
\AgdaSymbol{→}\AgdaSpace{}%
\AgdaDatatype{E}\AgdaSpace{}%
\AgdaSymbol{(}\AgdaBound{Γ}\AgdaSpace{}%
\AgdaOperator{\AgdaInductiveConstructor{▹}}\AgdaSpace{}%
\AgdaInductiveConstructor{ix}\AgdaSpace{}%
\AgdaGeneralizable{s}\AgdaSymbol{)}\AgdaSpace{}%
\AgdaSymbol{(}\AgdaInductiveConstructor{ar}\AgdaSpace{}%
\AgdaGeneralizable{p}\AgdaSymbol{))}\<%
\\
%
\>[7]\AgdaSymbol{→}\AgdaSpace{}%
\AgdaDatatype{E}\AgdaSpace{}%
\AgdaBound{Γ}\AgdaSpace{}%
\AgdaSymbol{(}\AgdaInductiveConstructor{ar}\AgdaSpace{}%
\AgdaSymbol{(}\AgdaGeneralizable{s}\AgdaSpace{}%
\AgdaOperator{\AgdaFunction{⊗}}\AgdaSpace{}%
\AgdaGeneralizable{p}\AgdaSymbol{))}\<%
\\
%
\>[2]\AgdaFunction{Imap}\AgdaSpace{}%
\AgdaBound{f}\AgdaSpace{}%
\AgdaSymbol{=}\AgdaSpace{}%
\AgdaInductiveConstructor{imap}\AgdaSpace{}%
\AgdaSymbol{(}\AgdaBound{f}\AgdaSpace{}%
\AgdaSymbol{λ}\AgdaSpace{}%
\AgdaSymbol{\{}\AgdaBound{Δ}\AgdaSymbol{\}}\AgdaSpace{}%
\AgdaSymbol{⦃}\AgdaSpace{}%
\AgdaBound{p}\AgdaSpace{}%
\AgdaSymbol{⦄}\AgdaSpace{}%
\AgdaSymbol{→}\AgdaSpace{}%
\AgdaInductiveConstructor{var}\AgdaSpace{}%
\AgdaSymbol{(}\AgdaFunction{V}\AgdaSpace{}%
\AgdaInductiveConstructor{v₀}\AgdaSymbol{))}\<%
\end{code}
We do the same wrappers for \AC{sum}, \AC{imaps}, \AC{imapb}, and the one
for \AC{let′}, providing a familiar let-like syntax.
\begin{code}[hide]%
%
\>[2]\AgdaFunction{Sum}%
\>[1637I]\AgdaSymbol{:}\AgdaSpace{}%
\AgdaSymbol{∀}\AgdaSpace{}%
\AgdaSymbol{\{}\AgdaBound{Γ}\AgdaSymbol{\}}\<%
\\
\>[1637I][@{}l@{\AgdaIndent{0}}]%
\>[7]\AgdaSymbol{→}\AgdaSpace{}%
\AgdaSymbol{(}\AgdaFunction{GE}\AgdaSpace{}%
\AgdaSymbol{(}\AgdaBound{Γ}\AgdaSpace{}%
\AgdaOperator{\AgdaInductiveConstructor{▹}}\AgdaSpace{}%
\AgdaInductiveConstructor{ix}\AgdaSpace{}%
\AgdaGeneralizable{s}\AgdaSymbol{)}\AgdaSpace{}%
\AgdaSymbol{(}\AgdaInductiveConstructor{ix}\AgdaSpace{}%
\AgdaGeneralizable{s}\AgdaSymbol{)}\AgdaSpace{}%
\AgdaSymbol{→}\AgdaSpace{}%
\AgdaDatatype{E}\AgdaSpace{}%
\AgdaSymbol{(}\AgdaBound{Γ}\AgdaSpace{}%
\AgdaOperator{\AgdaInductiveConstructor{▹}}\AgdaSpace{}%
\AgdaInductiveConstructor{ix}\AgdaSpace{}%
\AgdaGeneralizable{s}\AgdaSymbol{)}\AgdaSpace{}%
\AgdaSymbol{(}\AgdaInductiveConstructor{ar}\AgdaSpace{}%
\AgdaGeneralizable{p}\AgdaSymbol{))}\<%
\\
%
\>[7]\AgdaSymbol{→}\AgdaSpace{}%
\AgdaDatatype{E}\AgdaSpace{}%
\AgdaBound{Γ}\AgdaSpace{}%
\AgdaSymbol{(}\AgdaInductiveConstructor{ar}\AgdaSpace{}%
\AgdaGeneralizable{p}\AgdaSymbol{)}\<%
\\
%
\>[2]\AgdaFunction{Sum}\AgdaSpace{}%
\AgdaBound{f}\AgdaSpace{}%
\AgdaSymbol{=}\AgdaSpace{}%
\AgdaInductiveConstructor{sum}\AgdaSpace{}%
\AgdaSymbol{(}\AgdaBound{f}\AgdaSpace{}%
\AgdaSymbol{λ}\AgdaSpace{}%
\AgdaSymbol{\{}\AgdaBound{Δ}\AgdaSymbol{\}}\AgdaSpace{}%
\AgdaSymbol{⦃}\AgdaSpace{}%
\AgdaBound{p}\AgdaSpace{}%
\AgdaSymbol{⦄}\AgdaSpace{}%
\AgdaSymbol{→}\AgdaSpace{}%
\AgdaInductiveConstructor{var}\AgdaSpace{}%
\AgdaSymbol{(}\AgdaFunction{V}\AgdaSpace{}%
\AgdaInductiveConstructor{v₀}\AgdaSymbol{))}\<%
\\
%
\\[\AgdaEmptyExtraSkip]%
%
\>[2]\AgdaFunction{Imaps}%
\>[1672I]\AgdaSymbol{:}\AgdaSpace{}%
\AgdaSymbol{∀}\AgdaSpace{}%
\AgdaSymbol{\{}\AgdaBound{Γ}\AgdaSymbol{\}}\<%
\\
\>[.][@{}l@{}]\<[1672I]%
\>[8]\AgdaSymbol{→}\AgdaSpace{}%
\AgdaSymbol{(}\AgdaFunction{GE}\AgdaSpace{}%
\AgdaSymbol{(}\AgdaBound{Γ}\AgdaSpace{}%
\AgdaOperator{\AgdaInductiveConstructor{▹}}\AgdaSpace{}%
\AgdaInductiveConstructor{ix}\AgdaSpace{}%
\AgdaGeneralizable{s}\AgdaSymbol{)}\AgdaSpace{}%
\AgdaSymbol{(}\AgdaInductiveConstructor{ix}\AgdaSpace{}%
\AgdaGeneralizable{s}\AgdaSymbol{)}\AgdaSpace{}%
\AgdaSymbol{→}\AgdaSpace{}%
\AgdaDatatype{E}\AgdaSpace{}%
\AgdaSymbol{(}\AgdaBound{Γ}\AgdaSpace{}%
\AgdaOperator{\AgdaInductiveConstructor{▹}}\AgdaSpace{}%
\AgdaInductiveConstructor{ix}\AgdaSpace{}%
\AgdaGeneralizable{s}\AgdaSymbol{)}\AgdaSpace{}%
\AgdaSymbol{(}\AgdaInductiveConstructor{ar}\AgdaSpace{}%
\AgdaFunction{unit}\AgdaSymbol{))}\<%
\\
%
\>[8]\AgdaSymbol{→}\AgdaSpace{}%
\AgdaDatatype{E}\AgdaSpace{}%
\AgdaBound{Γ}\AgdaSpace{}%
\AgdaSymbol{(}\AgdaInductiveConstructor{ar}\AgdaSpace{}%
\AgdaGeneralizable{s}\AgdaSymbol{)}\<%
\\
%
\>[2]\AgdaFunction{Imaps}\AgdaSpace{}%
\AgdaBound{f}\AgdaSpace{}%
\AgdaSymbol{=}\AgdaSpace{}%
\AgdaInductiveConstructor{imaps}\AgdaSpace{}%
\AgdaSymbol{(}\AgdaBound{f}\AgdaSpace{}%
\AgdaSymbol{λ}\AgdaSpace{}%
\AgdaSymbol{\{}\AgdaBound{Δ}\AgdaSymbol{\}}\AgdaSpace{}%
\AgdaSymbol{⦃}\AgdaSpace{}%
\AgdaBound{p}\AgdaSpace{}%
\AgdaSymbol{⦄}\AgdaSpace{}%
\AgdaSymbol{→}\AgdaSpace{}%
\AgdaInductiveConstructor{var}\AgdaSpace{}%
\AgdaSymbol{(}\AgdaFunction{V}\AgdaSpace{}%
\AgdaInductiveConstructor{v₀}\AgdaSymbol{))}\<%
\\
%
\\[\AgdaEmptyExtraSkip]%
%
\>[2]\AgdaFunction{Imapb}%
\>[1707I]\AgdaSymbol{:}\AgdaSpace{}%
\AgdaSymbol{∀}\AgdaSpace{}%
\AgdaSymbol{\{}\AgdaBound{Γ}\AgdaSymbol{\}}\<%
\\
\>[.][@{}l@{}]\<[1707I]%
\>[8]\AgdaSymbol{→}\AgdaSpace{}%
\AgdaGeneralizable{s}\AgdaSpace{}%
\AgdaOperator{\AgdaFunction{*}}\AgdaSpace{}%
\AgdaGeneralizable{p}\AgdaSpace{}%
\AgdaOperator{\AgdaFunction{≈}}\AgdaSpace{}%
\AgdaGeneralizable{q}\<%
\\
%
\>[8]\AgdaSymbol{→}\AgdaSpace{}%
\AgdaSymbol{(}\AgdaFunction{GE}\AgdaSpace{}%
\AgdaSymbol{(}\AgdaBound{Γ}\AgdaSpace{}%
\AgdaOperator{\AgdaInductiveConstructor{▹}}\AgdaSpace{}%
\AgdaInductiveConstructor{ix}\AgdaSpace{}%
\AgdaGeneralizable{s}\AgdaSymbol{)}\AgdaSpace{}%
\AgdaSymbol{(}\AgdaInductiveConstructor{ix}\AgdaSpace{}%
\AgdaGeneralizable{s}\AgdaSymbol{)}\AgdaSpace{}%
\AgdaSymbol{→}\AgdaSpace{}%
\AgdaDatatype{E}\AgdaSpace{}%
\AgdaSymbol{(}\AgdaBound{Γ}\AgdaSpace{}%
\AgdaOperator{\AgdaInductiveConstructor{▹}}\AgdaSpace{}%
\AgdaInductiveConstructor{ix}\AgdaSpace{}%
\AgdaGeneralizable{s}\AgdaSymbol{)}\AgdaSpace{}%
\AgdaSymbol{(}\AgdaInductiveConstructor{ar}\AgdaSpace{}%
\AgdaGeneralizable{p}\AgdaSymbol{))}\<%
\\
%
\>[8]\AgdaSymbol{→}\AgdaSpace{}%
\AgdaDatatype{E}\AgdaSpace{}%
\AgdaBound{Γ}\AgdaSpace{}%
\AgdaSymbol{(}\AgdaInductiveConstructor{ar}\AgdaSpace{}%
\AgdaGeneralizable{q}\AgdaSymbol{)}\<%
\\
%
\>[2]\AgdaFunction{Imapb}\AgdaSpace{}%
\AgdaBound{p}\AgdaSpace{}%
\AgdaBound{f}\AgdaSpace{}%
\AgdaSymbol{=}\AgdaSpace{}%
\AgdaInductiveConstructor{imapb}\AgdaSpace{}%
\AgdaBound{p}\AgdaSpace{}%
\AgdaSymbol{(}\AgdaBound{f}\AgdaSpace{}%
\AgdaSymbol{λ}\AgdaSpace{}%
\AgdaSymbol{\{}\AgdaBound{Δ}\AgdaSymbol{\}}\AgdaSpace{}%
\AgdaSymbol{⦃}\AgdaSpace{}%
\AgdaBound{p}\AgdaSpace{}%
\AgdaSymbol{⦄}\AgdaSpace{}%
\AgdaSymbol{→}\AgdaSpace{}%
\AgdaInductiveConstructor{var}\AgdaSpace{}%
\AgdaSymbol{(}\AgdaFunction{V}\AgdaSpace{}%
\AgdaInductiveConstructor{v₀}\AgdaSymbol{))}\<%
\end{code}
\begin{code}%
%
\>[2]\AgdaFunction{Let-syntax}\AgdaSpace{}%
\AgdaSymbol{:}\AgdaSpace{}%
\AgdaSymbol{∀}\AgdaSpace{}%
\AgdaSymbol{\{}\AgdaBound{Γ}\AgdaSymbol{\}}\<%
\\
\>[2][@{}l@{\AgdaIndent{0}}]%
\>[6]\AgdaSymbol{→}\AgdaSpace{}%
\AgdaSymbol{(}\AgdaDatatype{E}\AgdaSpace{}%
\AgdaBound{Γ}\AgdaSpace{}%
\AgdaSymbol{(}\AgdaInductiveConstructor{ar}\AgdaSpace{}%
\AgdaGeneralizable{s}\AgdaSymbol{))}\<%
\\
%
\>[6]\AgdaSymbol{→}\AgdaSpace{}%
\AgdaSymbol{(}\AgdaFunction{GE}\AgdaSpace{}%
\AgdaSymbol{(}\AgdaBound{Γ}\AgdaSpace{}%
\AgdaOperator{\AgdaInductiveConstructor{▹}}\AgdaSpace{}%
\AgdaSymbol{(}\AgdaInductiveConstructor{ar}\AgdaSpace{}%
\AgdaGeneralizable{s}\AgdaSymbol{))}\AgdaSpace{}%
\AgdaSymbol{(}\AgdaInductiveConstructor{ar}\AgdaSpace{}%
\AgdaGeneralizable{s}\AgdaSymbol{)}\AgdaSpace{}%
\AgdaSymbol{→}\AgdaSpace{}%
\AgdaDatatype{E}\AgdaSpace{}%
\AgdaSymbol{(}\AgdaBound{Γ}\AgdaSpace{}%
\AgdaOperator{\AgdaInductiveConstructor{▹}}\AgdaSpace{}%
\AgdaSymbol{(}\AgdaInductiveConstructor{ar}\AgdaSpace{}%
\AgdaGeneralizable{s}\AgdaSymbol{))}\AgdaSpace{}%
\AgdaSymbol{(}\AgdaInductiveConstructor{ar}\AgdaSpace{}%
\AgdaGeneralizable{p}\AgdaSymbol{))}\<%
\\
%
\>[6]\AgdaSymbol{→}\AgdaSpace{}%
\AgdaDatatype{E}\AgdaSpace{}%
\AgdaBound{Γ}\AgdaSpace{}%
\AgdaSymbol{(}\AgdaInductiveConstructor{ar}\AgdaSpace{}%
\AgdaGeneralizable{p}\AgdaSymbol{)}\<%
\\
%
\>[2]\AgdaFunction{Let-syntax}\AgdaSpace{}%
\AgdaBound{x}\AgdaSpace{}%
\AgdaBound{f}\AgdaSpace{}%
\AgdaSymbol{=}\AgdaSpace{}%
\AgdaInductiveConstructor{let′}\AgdaSpace{}%
\AgdaBound{x}\AgdaSpace{}%
\AgdaSymbol{(}\AgdaBound{f}\AgdaSpace{}%
\AgdaSymbol{λ}\AgdaSpace{}%
\AgdaSymbol{\{}\AgdaBound{Δ}\AgdaSymbol{\}}\AgdaSpace{}%
\AgdaSymbol{⦃}\AgdaSpace{}%
\AgdaBound{p}\AgdaSpace{}%
\AgdaSymbol{⦄}\AgdaSpace{}%
\AgdaSymbol{→}\AgdaSpace{}%
\AgdaInductiveConstructor{var}\AgdaSpace{}%
\AgdaSymbol{(}\AgdaFunction{V}\AgdaSpace{}%
\AgdaInductiveConstructor{v₀}\AgdaSymbol{))}\<%
\\
\>[0]\<%
\\
%
\>[2]\AgdaKeyword{syntax}\AgdaSpace{}%
\AgdaFunction{Let-syntax}\AgdaSpace{}%
\AgdaBound{e}\AgdaSpace{}%
\AgdaSymbol{(λ}\AgdaSpace{}%
\AgdaBound{x}\AgdaSpace{}%
\AgdaSymbol{→}\AgdaSpace{}%
\AgdaBound{e'}\AgdaSymbol{)}\AgdaSpace{}%
\AgdaSymbol{=}\AgdaSpace{}%
\AgdaFunction{Let}\AgdaSpace{}%
\AgdaBound{x}\AgdaSpace{}%
\AgdaFunction{:=}\AgdaSpace{}%
\AgdaBound{e}\AgdaSpace{}%
\AgdaFunction{In}\AgdaSpace{}%
\AgdaBound{e'}\<%
\end{code}

The final convenience operator that we are missing is the ability
to represent contexts in the HOAS style.
\begin{code}[hide]%
%
\>[2]\AgdaKeyword{infixl}\AgdaSpace{}%
\AgdaNumber{3}\AgdaSpace{}%
\AgdaFunction{Let-syntax}\<%
\\
%
\\[\AgdaEmptyExtraSkip]%
%
\>[2]\AgdaComment{--\ Extend\ context\ with\ a\ list\ of\ types}\<%
\\
%
\>[2]\AgdaComment{--\ (List\ is\ a\ context\ that\ grows\ to\ the\ left)}\<%
\end{code}
First of all, we define a helper function that appends a list of \AF{IS}
at the end of some context \AF{Γ}:
\begin{code}%
%
\>[2]\AgdaFunction{ext}\AgdaSpace{}%
\AgdaSymbol{:}\AgdaSpace{}%
\AgdaDatatype{Ctx}\AgdaSpace{}%
\AgdaSymbol{→}\AgdaSpace{}%
\AgdaDatatype{List}\AgdaSpace{}%
\AgdaDatatype{IS}\AgdaSpace{}%
\AgdaSymbol{→}\AgdaSpace{}%
\AgdaDatatype{Ctx}\<%
\\
%
\>[2]\AgdaFunction{ext}\AgdaSpace{}%
\AgdaBound{Γ}\AgdaSpace{}%
\AgdaInductiveConstructor{[]}\AgdaSpace{}%
\AgdaSymbol{=}\AgdaSpace{}%
\AgdaBound{Γ}\<%
\\
%
\>[2]\AgdaFunction{ext}\AgdaSpace{}%
\AgdaBound{Γ}\AgdaSpace{}%
\AgdaSymbol{(}\AgdaBound{x}\AgdaSpace{}%
\AgdaOperator{\AgdaInductiveConstructor{∷}}\AgdaSpace{}%
\AgdaBound{l}\AgdaSymbol{)}\AgdaSpace{}%
\AgdaSymbol{=}\AgdaSpace{}%
\AgdaFunction{ext}\AgdaSpace{}%
\AgdaSymbol{(}\AgdaBound{Γ}\AgdaSpace{}%
\AgdaOperator{\AgdaInductiveConstructor{▹}}\AgdaSpace{}%
\AgdaBound{x}\AgdaSymbol{)}\AgdaSpace{}%
\AgdaBound{l}\<%
\end{code}
\begin{code}[hide]%
%
\>[2]\AgdaComment{--\ Turn\ the\ list\ of\ IS\ into\ the\ following\ function:}\<%
\\
%
\>[2]\AgdaComment{--\ \ \ l\ =\ [a,\ b,\ c]}\<%
\\
%
\>[2]\AgdaComment{--\ \ \ X\ =\ X}\<%
\\
%
\>[2]\AgdaComment{--\ \ \ Γ\ =\ Γ}\<%
\\
%
\>[2]\AgdaComment{--\ \ \ ----------------------------}\<%
\\
%
\>[2]\AgdaComment{--\ \ \ GE\ Γ\ a\ →\ GE\ Γ\ b\ →\ GE\ Γ\ c\ →\ X}\<%
\\
%
\>[2]\AgdaFunction{lfunh}\AgdaSpace{}%
\AgdaSymbol{:}\AgdaSpace{}%
\AgdaSymbol{(}\AgdaBound{l}\AgdaSpace{}%
\AgdaSymbol{:}\AgdaSpace{}%
\AgdaDatatype{List}\AgdaSpace{}%
\AgdaDatatype{IS}\AgdaSymbol{)}\AgdaSpace{}%
\AgdaSymbol{(}\AgdaBound{X}\AgdaSpace{}%
\AgdaSymbol{:}\AgdaSpace{}%
\AgdaPrimitive{Set}\AgdaSymbol{)}\AgdaSpace{}%
\AgdaSymbol{(}\AgdaBound{Γ}\AgdaSpace{}%
\AgdaSymbol{:}\AgdaSpace{}%
\AgdaDatatype{Ctx}\AgdaSymbol{)}\AgdaSpace{}%
\AgdaSymbol{→}\AgdaSpace{}%
\AgdaPrimitive{Set}\<%
\\
%
\>[2]\AgdaFunction{lfunh}\AgdaSpace{}%
\AgdaInductiveConstructor{[]}\AgdaSpace{}%
\AgdaBound{X}\AgdaSpace{}%
\AgdaBound{Γ}\AgdaSpace{}%
\AgdaSymbol{=}\AgdaSpace{}%
\AgdaBound{X}\<%
\\
%
\>[2]\AgdaFunction{lfunh}\AgdaSpace{}%
\AgdaSymbol{(}\AgdaBound{a}\AgdaSpace{}%
\AgdaOperator{\AgdaInductiveConstructor{∷}}\AgdaSpace{}%
\AgdaBound{l}\AgdaSymbol{)}\AgdaSpace{}%
\AgdaBound{X}\AgdaSpace{}%
\AgdaBound{Γ}\AgdaSpace{}%
\AgdaSymbol{=}\AgdaSpace{}%
\AgdaFunction{GE}\AgdaSpace{}%
\AgdaBound{Γ}\AgdaSpace{}%
\AgdaBound{a}\AgdaSpace{}%
\AgdaSymbol{→}\AgdaSpace{}%
\AgdaFunction{lfunh}\AgdaSpace{}%
\AgdaBound{l}\AgdaSpace{}%
\AgdaBound{X}\AgdaSpace{}%
\AgdaBound{Γ}\<%
\\
%
\\[\AgdaEmptyExtraSkip]%
%
\>[2]\AgdaComment{--\ Diagonalise\ lfunh:}\<%
\\
%
\>[2]\AgdaComment{--\ \ \ l\ =\ [a,\ b]}\<%
\\
%
\>[2]\AgdaComment{--\ \ \ Γ\ =\ Γ}\<%
\\
%
\>[2]\AgdaComment{--\ \ \ is\ =\ is}\<%
\\
%
\>[2]\AgdaComment{--\ \ \ ---------------------------------------------}\<%
\\
%
\>[2]\AgdaComment{--\ \ \ GE\ (ext\ Γ\ l)\ a\ →\ GE\ (ext\ Γ\ l)\ →\ E\ (ext\ Γ\ l)\ is}\<%
\end{code}
Then, we define \AF{lfun} that for the given list of \AF{IS}-es
(l = [is₁, \dots, isₙ]), some context \AF{Γ} and some \AF{IS}
ip computes an Agda function of type (\AF{GE} (\AF{ext} \AF{Γ} l) is₁ →
\dots → \AF{GE} (\AF{ext} \AF{Γ} l) isₙ → \AD{E} (\AF{ext} Γ l) ip).
The function \AF{lvar} lifts a variable in some context Γ into
an extended context.
\begin{code}%
%
\>[2]\AgdaFunction{lfun}\AgdaSpace{}%
\AgdaSymbol{:}\AgdaSpace{}%
\AgdaSymbol{(}\AgdaBound{l}\AgdaSpace{}%
\AgdaSymbol{:}\AgdaSpace{}%
\AgdaDatatype{List}\AgdaSpace{}%
\AgdaDatatype{IS}\AgdaSymbol{)}%
\>[24]\AgdaSymbol{(}\AgdaBound{Γ}\AgdaSpace{}%
\AgdaSymbol{:}\AgdaSpace{}%
\AgdaDatatype{Ctx}\AgdaSymbol{)}\AgdaSpace{}%
\AgdaSymbol{(}\AgdaBound{is}\AgdaSpace{}%
\AgdaSymbol{:}\AgdaSpace{}%
\AgdaDatatype{IS}\AgdaSymbol{)}\AgdaSpace{}%
\AgdaSymbol{→}\AgdaSpace{}%
\AgdaPrimitive{Set}\<%
\\
%
\>[2]\AgdaFunction{lvar}\AgdaSpace{}%
\AgdaSymbol{:}\AgdaSpace{}%
\AgdaSymbol{∀}\AgdaSpace{}%
\AgdaBound{l}\AgdaSpace{}%
\AgdaSymbol{→}\AgdaSpace{}%
\AgdaGeneralizable{is}\AgdaSpace{}%
\AgdaOperator{\AgdaDatatype{∈}}\AgdaSpace{}%
\AgdaGeneralizable{Γ}\AgdaSpace{}%
\AgdaSymbol{→}\AgdaSpace{}%
\AgdaFunction{GE}\AgdaSpace{}%
\AgdaSymbol{(}\AgdaFunction{ext}\AgdaSpace{}%
\AgdaGeneralizable{Γ}\AgdaSpace{}%
\AgdaBound{l}\AgdaSymbol{)}\AgdaSpace{}%
\AgdaGeneralizable{is}\<%
\end{code}
\begin{code}[hide]%
%
\>[2]\AgdaFunction{lfun}\AgdaSpace{}%
\AgdaBound{l}\AgdaSpace{}%
\AgdaBound{Γ}\AgdaSpace{}%
\AgdaBound{τ}\AgdaSpace{}%
\AgdaSymbol{=}\AgdaSpace{}%
\AgdaFunction{lfunh}\AgdaSpace{}%
\AgdaBound{l}\AgdaSpace{}%
\AgdaSymbol{(}\AgdaDatatype{E}\AgdaSpace{}%
\AgdaSymbol{(}\AgdaFunction{ext}\AgdaSpace{}%
\AgdaBound{Γ}\AgdaSpace{}%
\AgdaBound{l}\AgdaSymbol{)}\AgdaSpace{}%
\AgdaBound{τ}\AgdaSymbol{)}\AgdaSpace{}%
\AgdaSymbol{(}\AgdaFunction{ext}\AgdaSpace{}%
\AgdaBound{Γ}\AgdaSpace{}%
\AgdaBound{l}\AgdaSymbol{)}\<%
\\
%
\>[2]\AgdaFunction{lvar}\AgdaSpace{}%
\AgdaInductiveConstructor{[]}\AgdaSpace{}%
\AgdaBound{v}\AgdaSpace{}%
\AgdaSymbol{=}\AgdaSpace{}%
\AgdaInductiveConstructor{var}\AgdaSpace{}%
\AgdaSymbol{(}\AgdaFunction{V}\AgdaSpace{}%
\AgdaBound{v}\AgdaSymbol{)}\<%
\\
%
\>[2]\AgdaFunction{lvar}\AgdaSpace{}%
\AgdaSymbol{(}\AgdaBound{x}\AgdaSpace{}%
\AgdaOperator{\AgdaInductiveConstructor{∷}}\AgdaSpace{}%
\AgdaBound{l}\AgdaSymbol{)}\AgdaSpace{}%
\AgdaBound{v}\AgdaSpace{}%
\AgdaSymbol{=}\AgdaSpace{}%
\AgdaFunction{lvar}\AgdaSpace{}%
\AgdaBound{l}\AgdaSpace{}%
\AgdaSymbol{(}\AgdaInductiveConstructor{vₛ}\AgdaSpace{}%
\AgdaBound{v}\AgdaSymbol{)}\<%
\\
%
\\[\AgdaEmptyExtraSkip]%
%
\>[2]\AgdaComment{--\ Apply\ function\ to\ the\ corresponding\ variables\ of\ the\ context}\<%
\end{code}
With these helper functions we define the \AF{Lcon} helper that
for the given list of types $l$, resulting type \AB{is}, the initial
context Γ and the function of type \AF{lfun} l Γ \AB{is} computes
the expression in the context \AF{ext} Γ l.
\begin{code}%
%
\>[2]\AgdaFunction{Lcon}\AgdaSpace{}%
\AgdaSymbol{:}\AgdaSpace{}%
\AgdaSymbol{∀}\AgdaSpace{}%
\AgdaBound{l}\AgdaSpace{}%
\AgdaBound{is}\AgdaSpace{}%
\AgdaBound{Γ}\AgdaSpace{}%
\AgdaSymbol{→}\AgdaSpace{}%
\AgdaSymbol{(}\AgdaBound{f}\AgdaSpace{}%
\AgdaSymbol{:}\AgdaSpace{}%
\AgdaFunction{lfun}\AgdaSpace{}%
\AgdaBound{l}\AgdaSpace{}%
\AgdaBound{Γ}\AgdaSpace{}%
\AgdaBound{is}\AgdaSymbol{)}\AgdaSpace{}%
\AgdaSymbol{→}\AgdaSpace{}%
\AgdaDatatype{E}\AgdaSpace{}%
\AgdaSymbol{(}\AgdaFunction{ext}\AgdaSpace{}%
\AgdaBound{Γ}\AgdaSpace{}%
\AgdaBound{l}\AgdaSymbol{)}\AgdaSpace{}%
\AgdaBound{is}\<%
\\
%
\>[2]\AgdaFunction{Lcon}\AgdaSpace{}%
\AgdaInductiveConstructor{[]}%
\>[15]\AgdaBound{is}\AgdaSpace{}%
\AgdaBound{Γ}\AgdaSpace{}%
\AgdaBound{f}\AgdaSpace{}%
\AgdaSymbol{=}\AgdaSpace{}%
\AgdaBound{f}\<%
\\
%
\>[2]\AgdaFunction{Lcon}\AgdaSpace{}%
\AgdaSymbol{(}\AgdaBound{x}\AgdaSpace{}%
\AgdaOperator{\AgdaInductiveConstructor{∷}}\AgdaSpace{}%
\AgdaBound{l}\AgdaSymbol{)}\AgdaSpace{}%
\AgdaBound{is}\AgdaSpace{}%
\AgdaBound{Γ}\AgdaSpace{}%
\AgdaBound{f}\AgdaSpace{}%
\AgdaSymbol{=}\AgdaSpace{}%
\AgdaFunction{Lcon}\AgdaSpace{}%
\AgdaBound{l}\AgdaSpace{}%
\AgdaBound{is}\AgdaSpace{}%
\AgdaSymbol{(}\AgdaBound{Γ}\AgdaSpace{}%
\AgdaOperator{\AgdaInductiveConstructor{▹}}\AgdaSpace{}%
\AgdaBound{x}\AgdaSymbol{)}\AgdaSpace{}%
\AgdaSymbol{(}\AgdaBound{f}\AgdaSpace{}%
\AgdaSymbol{(}\AgdaFunction{lvar}\AgdaSpace{}%
\AgdaBound{l}\AgdaSpace{}%
\AgdaInductiveConstructor{v₀}\AgdaSymbol{))}\<%
\end{code}
This means that we can bind the last $n$ elements of the
context to Agda variables and use them safely under binders.
For example, consider this expression:
\begin{code}%
%
\>[2]\AgdaFunction{\AgdaUnderscore{}}\AgdaSpace{}%
\AgdaSymbol{:}\AgdaSpace{}%
\AgdaDatatype{E}\AgdaSpace{}%
\AgdaSymbol{\AgdaUnderscore{}}\AgdaSpace{}%
\AgdaSymbol{\AgdaUnderscore{}}\<%
\\
%
\>[2]\AgdaSymbol{\AgdaUnderscore{}}\AgdaSpace{}%
\AgdaSymbol{=}%
\>[1957I]\AgdaFunction{Lcon}\AgdaSpace{}%
\AgdaSymbol{(}\AgdaInductiveConstructor{ar}\AgdaSpace{}%
\AgdaSymbol{(}\AgdaInductiveConstructor{ι}\AgdaSpace{}%
\AgdaNumber{5}\AgdaSymbol{)}\AgdaSpace{}%
\AgdaOperator{\AgdaInductiveConstructor{∷}}\AgdaSpace{}%
\AgdaInductiveConstructor{ar}\AgdaSpace{}%
\AgdaSymbol{(}\AgdaNumber{5}\AgdaSpace{}%
\AgdaOperator{\AgdaInductiveConstructor{∷}}\AgdaSpace{}%
\AgdaNumber{5}\AgdaSpace{}%
\AgdaOperator{\AgdaInductiveConstructor{∷}}\AgdaSpace{}%
\AgdaInductiveConstructor{[]}\AgdaSymbol{)}\AgdaSpace{}%
\AgdaOperator{\AgdaInductiveConstructor{∷}}\AgdaSpace{}%
\AgdaInductiveConstructor{[]}\AgdaSymbol{)}\AgdaSpace{}%
\AgdaSymbol{(}\AgdaInductiveConstructor{ar}\AgdaSpace{}%
\AgdaInductiveConstructor{[]}\AgdaSymbol{)}\AgdaSpace{}%
\AgdaInductiveConstructor{ε}\<%
\\
\>[.][@{}l@{}]\<[1957I]%
\>[6]\AgdaSymbol{λ}\AgdaSpace{}%
\AgdaBound{a}\AgdaSpace{}%
\AgdaBound{b}\AgdaSpace{}%
\AgdaSymbol{→}\AgdaSpace{}%
\AgdaFunction{Sum}\AgdaSpace{}%
\AgdaSymbol{λ}\AgdaSpace{}%
\AgdaBound{i}\AgdaSpace{}%
\AgdaSymbol{→}\AgdaSpace{}%
\AgdaInductiveConstructor{sels}\AgdaSpace{}%
\AgdaBound{a}\AgdaSpace{}%
\AgdaBound{i}\AgdaSpace{}%
\AgdaOperator{\AgdaInductiveConstructor{⊞}}\AgdaSpace{}%
\AgdaInductiveConstructor{sels}\AgdaSpace{}%
\AgdaSymbol{(}\AgdaInductiveConstructor{sel}\AgdaSpace{}%
\AgdaBound{b}\AgdaSpace{}%
\AgdaBound{i}\AgdaSymbol{)}\AgdaSpace{}%
\AgdaBound{i}\<%
\end{code}
where we \AB{a} and \AB{b} are bound to the arguments of
the Agda's lambda term and which are used when computing
expression in the context (ε ▹ 5 ∷ [] ▹ 5 ∷ 5 ∷ []). 



\paragraph{Primitives}
We are defining primitives that are needed for expresison our
running example in $E$.  We consider an example for the \AF{conv}-olution:
\begin{code}[hide]%
\>[0]\AgdaKeyword{module}\AgdaSpace{}%
\AgdaModule{Primitives}\AgdaSpace{}%
\AgdaKeyword{where}\<%
\\
%
\\[\AgdaEmptyExtraSkip]%
\>[0][@{}l@{\AgdaIndent{0}}]%
\>[2]\AgdaKeyword{open}\AgdaSpace{}%
\AgdaKeyword{import}\AgdaSpace{}%
\AgdaModule{Data.List}\AgdaSpace{}%
\AgdaSymbol{as}\AgdaSpace{}%
\AgdaModule{L}\AgdaSpace{}%
\AgdaKeyword{using}\AgdaSpace{}%
\AgdaSymbol{(}\AgdaDatatype{List}\AgdaSymbol{;}\AgdaSpace{}%
\AgdaInductiveConstructor{[]}\AgdaSymbol{;}\AgdaSpace{}%
\AgdaOperator{\AgdaInductiveConstructor{\AgdaUnderscore{}∷\AgdaUnderscore{}}}\AgdaSymbol{)}\<%
\\
%
\>[2]\AgdaKeyword{open}\AgdaSpace{}%
\AgdaKeyword{import}\AgdaSpace{}%
\AgdaModule{Data.Nat}\AgdaSpace{}%
\AgdaSymbol{as}\AgdaSpace{}%
\AgdaModule{ℕ}\AgdaSpace{}%
\AgdaKeyword{using}\AgdaSpace{}%
\AgdaSymbol{(}\AgdaDatatype{ℕ}\AgdaSymbol{;}\AgdaSpace{}%
\AgdaInductiveConstructor{zero}\AgdaSymbol{;}\AgdaSpace{}%
\AgdaInductiveConstructor{suc}\AgdaSymbol{)}\<%
\\
%
\>[2]\AgdaKeyword{open}\AgdaSpace{}%
\AgdaKeyword{import}\AgdaSpace{}%
\AgdaModule{Function}\AgdaSpace{}%
\AgdaKeyword{using}\AgdaSpace{}%
\AgdaSymbol{(}\AgdaOperator{\AgdaFunction{\AgdaUnderscore{}\$\AgdaUnderscore{}}}\AgdaSymbol{;}\AgdaSpace{}%
\AgdaFunction{it}\AgdaSymbol{;}\AgdaSpace{}%
\AgdaOperator{\AgdaFunction{\AgdaUnderscore{}∋\AgdaUnderscore{}}}\AgdaSymbol{)}\<%
\\
%
\>[2]\AgdaKeyword{open}\AgdaSpace{}%
\AgdaKeyword{import}\AgdaSpace{}%
\AgdaModule{Relation.Binary.PropositionalEquality}\<%
\\
%
\>[2]\AgdaKeyword{open}\AgdaSpace{}%
\AgdaModule{Array}\AgdaSpace{}%
\AgdaKeyword{hiding}\AgdaSpace{}%
\AgdaSymbol{(}\AgdaFunction{slide}\AgdaSymbol{)}\<%
\\
%
\>[2]\AgdaKeyword{open}\AgdaSpace{}%
\AgdaModule{Syntax}\<%
\\
%
\>[2]\AgdaKeyword{open}\AgdaSpace{}%
\AgdaModule{WkSub}\<%
\\
%
\>[2]\AgdaKeyword{open}\AgdaSpace{}%
\AgdaModule{Lang}\<%
\\
%
\\[\AgdaEmptyExtraSkip]%
%
\>[2]\AgdaFunction{fromPrefix}\AgdaSpace{}%
\AgdaSymbol{:}\AgdaSpace{}%
\AgdaDatatype{Prefix}\AgdaSpace{}%
\AgdaGeneralizable{Γ}\AgdaSpace{}%
\AgdaGeneralizable{Δ}\AgdaSpace{}%
\AgdaSymbol{→}\AgdaSpace{}%
\AgdaGeneralizable{Γ}\AgdaSpace{}%
\AgdaOperator{\AgdaDatatype{⊆}}\AgdaSpace{}%
\AgdaGeneralizable{Δ}\<%
\\
%
\>[2]\AgdaFunction{fromPrefix}\AgdaSpace{}%
\AgdaInductiveConstructor{zero}\AgdaSpace{}%
\AgdaSymbol{=}\AgdaSpace{}%
\AgdaFunction{⊆-eq}\<%
\\
%
\>[2]\AgdaFunction{fromPrefix}\AgdaSpace{}%
\AgdaSymbol{(}\AgdaInductiveConstructor{suc}\AgdaSpace{}%
\AgdaSymbol{⦃}\AgdaSpace{}%
\AgdaBound{p}\AgdaSpace{}%
\AgdaSymbol{⦄)}\AgdaSpace{}%
\AgdaSymbol{=}\AgdaSpace{}%
\AgdaInductiveConstructor{skip}\AgdaSpace{}%
\AgdaSymbol{(}\AgdaFunction{fromPrefix}\AgdaSpace{}%
\AgdaBound{p}\AgdaSymbol{)}\<%
\\
\>[0]\<%
\\
%
\>[2]\AgdaFunction{wkp}\AgdaSpace{}%
\AgdaSymbol{:}\AgdaSpace{}%
\AgdaDatatype{Prefix}\AgdaSpace{}%
\AgdaGeneralizable{Γ}\AgdaSpace{}%
\AgdaGeneralizable{Δ}\AgdaSpace{}%
\AgdaSymbol{→}\AgdaSpace{}%
\AgdaDatatype{E}\AgdaSpace{}%
\AgdaGeneralizable{Γ}\AgdaSpace{}%
\AgdaGeneralizable{is}\AgdaSpace{}%
\AgdaSymbol{→}\AgdaSpace{}%
\AgdaDatatype{E}\AgdaSpace{}%
\AgdaGeneralizable{Δ}\AgdaSpace{}%
\AgdaGeneralizable{is}\<%
\\
%
\>[2]\AgdaFunction{wkp}\AgdaSpace{}%
\AgdaBound{p}\AgdaSpace{}%
\AgdaSymbol{=}\AgdaSpace{}%
\AgdaFunction{wk}\AgdaSpace{}%
\AgdaSymbol{(}\AgdaFunction{fromPrefix}\AgdaSpace{}%
\AgdaBound{p}\AgdaSymbol{)}\<%
\\
%
\\[\AgdaEmptyExtraSkip]%
%
\>[2]\AgdaOperator{\AgdaFunction{⟨\AgdaUnderscore{}⟩}}\AgdaSpace{}%
\AgdaSymbol{:}\AgdaSpace{}%
\AgdaDatatype{E}\AgdaSpace{}%
\AgdaGeneralizable{Γ}\AgdaSpace{}%
\AgdaGeneralizable{is}\AgdaSpace{}%
\AgdaSymbol{→}\AgdaSpace{}%
\AgdaFunction{GE}\AgdaSpace{}%
\AgdaGeneralizable{Γ}\AgdaSpace{}%
\AgdaGeneralizable{is}\<%
\\
%
\>[2]\AgdaOperator{\AgdaFunction{⟨\AgdaUnderscore{}⟩}}\AgdaSpace{}%
\AgdaBound{t}\AgdaSpace{}%
\AgdaSymbol{\{}\AgdaBound{Δ}\AgdaSymbol{\}}\AgdaSpace{}%
\AgdaSymbol{⦃}\AgdaSpace{}%
\AgdaBound{p}\AgdaSpace{}%
\AgdaSymbol{⦄}\AgdaSpace{}%
\AgdaSymbol{=}\AgdaSpace{}%
\AgdaFunction{wkp}\AgdaSpace{}%
\AgdaBound{p}\AgdaSpace{}%
\AgdaBound{t}\<%
\end{code}
\begin{code}%
%
\>[2]\AgdaFunction{conv}%
\>[2074I]\AgdaSymbol{:}\AgdaSpace{}%
\AgdaSymbol{∀}\AgdaSpace{}%
\AgdaSymbol{\{}\AgdaBound{Γ}\AgdaSymbol{\}}\AgdaSpace{}%
\AgdaSymbol{→}\AgdaSpace{}%
\AgdaDatatype{E}\AgdaSpace{}%
\AgdaBound{Γ}\AgdaSpace{}%
\AgdaSymbol{(}\AgdaInductiveConstructor{ar}\AgdaSpace{}%
\AgdaGeneralizable{r}\AgdaSymbol{)}\AgdaSpace{}%
\AgdaSymbol{→}\AgdaSpace{}%
\AgdaSymbol{⦃}\AgdaSpace{}%
\AgdaGeneralizable{s}\AgdaSpace{}%
\AgdaOperator{\AgdaFunction{+}}\AgdaSpace{}%
\AgdaGeneralizable{p}\AgdaSpace{}%
\AgdaOperator{\AgdaFunction{≈}}\AgdaSpace{}%
\AgdaGeneralizable{r}\AgdaSpace{}%
\AgdaSymbol{⦄}\AgdaSpace{}%
\AgdaSymbol{→}\AgdaSpace{}%
\AgdaDatatype{E}\AgdaSpace{}%
\AgdaBound{Γ}\AgdaSpace{}%
\AgdaSymbol{(}\AgdaInductiveConstructor{ar}\AgdaSpace{}%
\AgdaGeneralizable{s}\AgdaSymbol{)}\AgdaSpace{}%
\AgdaSymbol{→}\AgdaSpace{}%
\AgdaSymbol{⦃}\AgdaSpace{}%
\AgdaOperator{\AgdaFunction{suc}}\AgdaSpace{}%
\AgdaGeneralizable{p}\AgdaSpace{}%
\AgdaOperator{\AgdaFunction{≈}}\AgdaSpace{}%
\AgdaGeneralizable{u}\AgdaSpace{}%
\AgdaSymbol{⦄}\<%
\\
\>[.][@{}l@{}]\<[2074I]%
\>[7]\AgdaSymbol{→}\AgdaSpace{}%
\AgdaDatatype{E}\AgdaSpace{}%
\AgdaBound{Γ}\AgdaSpace{}%
\AgdaSymbol{(}\AgdaInductiveConstructor{ar}\AgdaSpace{}%
\AgdaGeneralizable{u}\AgdaSymbol{)}\<%
\\
%
\>[2]\AgdaFunction{conv}\AgdaSpace{}%
\AgdaBound{f}\AgdaSpace{}%
\AgdaSymbol{⦃}\AgdaSpace{}%
\AgdaBound{s+p}\AgdaSpace{}%
\AgdaSymbol{⦄}\AgdaSpace{}%
\AgdaBound{g}\AgdaSpace{}%
\AgdaSymbol{⦃}\AgdaSpace{}%
\AgdaBound{ss}\AgdaSpace{}%
\AgdaSymbol{⦄}\<%
\\
\>[2][@{}l@{\AgdaIndent{0}}]%
\>[4]\AgdaSymbol{=}\AgdaSpace{}%
\AgdaFunction{Sum}\AgdaSpace{}%
\AgdaSymbol{λ}\AgdaSpace{}%
\AgdaBound{i}\AgdaSpace{}%
\AgdaSymbol{→}\AgdaSpace{}%
\AgdaSymbol{(}\AgdaInductiveConstructor{slide}\AgdaSpace{}%
\AgdaBound{i}\AgdaSpace{}%
\AgdaBound{s+p}\AgdaSpace{}%
\AgdaOperator{\AgdaFunction{⟨}}\AgdaSpace{}%
\AgdaBound{f}\AgdaSpace{}%
\AgdaOperator{\AgdaFunction{⟩}}\AgdaSpace{}%
\AgdaBound{ss}\AgdaSymbol{)}\AgdaSpace{}%
\AgdaOperator{\AgdaInductiveConstructor{⊠}}\AgdaSpace{}%
\AgdaFunction{Imaps}\AgdaSpace{}%
\AgdaSymbol{λ}\AgdaSpace{}%
\AgdaBound{j}\AgdaSpace{}%
\AgdaSymbol{→}\AgdaSpace{}%
\AgdaInductiveConstructor{sels}\AgdaSpace{}%
\AgdaOperator{\AgdaFunction{⟨}}\AgdaSpace{}%
\AgdaBound{g}\AgdaSpace{}%
\AgdaOperator{\AgdaFunction{⟩}}\AgdaSpace{}%
\AgdaBound{i}\<%
\\
%
\\[\AgdaEmptyExtraSkip]%
%
\>[2]\AgdaFunction{mconv}\AgdaSpace{}%
\AgdaSymbol{:}%
\>[2136I]\AgdaSymbol{⦃}\AgdaSpace{}%
\AgdaGeneralizable{s}\AgdaSpace{}%
\AgdaOperator{\AgdaFunction{+}}\AgdaSpace{}%
\AgdaGeneralizable{p}\AgdaSpace{}%
\AgdaOperator{\AgdaFunction{≈}}\AgdaSpace{}%
\AgdaGeneralizable{r}\AgdaSpace{}%
\AgdaSymbol{⦄}\AgdaSpace{}%
\AgdaSymbol{→}\AgdaSpace{}%
\AgdaSymbol{(}\AgdaBound{inp}\AgdaSpace{}%
\AgdaSymbol{:}\AgdaSpace{}%
\AgdaDatatype{E}\AgdaSpace{}%
\AgdaGeneralizable{Γ}\AgdaSpace{}%
\AgdaSymbol{(}\AgdaInductiveConstructor{ar}\AgdaSpace{}%
\AgdaGeneralizable{r}\AgdaSymbol{))}\AgdaSpace{}%
\AgdaSymbol{(}\AgdaBound{ws}\AgdaSpace{}%
\AgdaSymbol{:}\AgdaSpace{}%
\AgdaDatatype{E}\AgdaSpace{}%
\AgdaGeneralizable{Γ}\AgdaSpace{}%
\AgdaSymbol{(}\AgdaInductiveConstructor{ar}\AgdaSpace{}%
\AgdaSymbol{(}\AgdaGeneralizable{u}\AgdaSpace{}%
\AgdaOperator{\AgdaFunction{⊗}}\AgdaSpace{}%
\AgdaGeneralizable{s}\AgdaSymbol{)))}\<%
\\
\>[.][@{}l@{}]\<[2136I]%
\>[10]\AgdaSymbol{(}\AgdaBound{bᵥ}\AgdaSpace{}%
\AgdaSymbol{:}\AgdaSpace{}%
\AgdaDatatype{E}\AgdaSpace{}%
\AgdaGeneralizable{Γ}\AgdaSpace{}%
\AgdaSymbol{(}\AgdaInductiveConstructor{ar}\AgdaSpace{}%
\AgdaGeneralizable{u}\AgdaSymbol{))}\AgdaSpace{}%
\AgdaSymbol{→}\AgdaSpace{}%
\AgdaSymbol{⦃}\AgdaSpace{}%
\AgdaOperator{\AgdaFunction{suc}}\AgdaSpace{}%
\AgdaGeneralizable{p}\AgdaSpace{}%
\AgdaOperator{\AgdaFunction{≈}}\AgdaSpace{}%
\AgdaGeneralizable{w}\AgdaSpace{}%
\AgdaSymbol{⦄}\AgdaSpace{}%
\AgdaSymbol{→}\AgdaSpace{}%
\AgdaDatatype{E}\AgdaSpace{}%
\AgdaGeneralizable{Γ}\AgdaSpace{}%
\AgdaSymbol{(}\AgdaInductiveConstructor{ar}\AgdaSpace{}%
\AgdaSymbol{(}\AgdaGeneralizable{u}\AgdaSpace{}%
\AgdaOperator{\AgdaFunction{⊗}}\AgdaSpace{}%
\AgdaGeneralizable{w}\AgdaSymbol{))}\<%
\\
%
\>[2]\AgdaFunction{mconv}\AgdaSpace{}%
\AgdaSymbol{⦃}\AgdaSpace{}%
\AgdaBound{sp}\AgdaSpace{}%
\AgdaSymbol{⦄}\AgdaSpace{}%
\AgdaBound{inp}\AgdaSpace{}%
\AgdaBound{wᵥ}\AgdaSpace{}%
\AgdaBound{bᵥ}\AgdaSpace{}%
\AgdaSymbol{⦃}\AgdaSpace{}%
\AgdaBound{su}\AgdaSpace{}%
\AgdaSymbol{⦄}\AgdaSpace{}%
\AgdaSymbol{=}\<%
\\
\>[2][@{}l@{\AgdaIndent{0}}]%
\>[4]\AgdaFunction{Imap}\AgdaSpace{}%
\AgdaSymbol{λ}\AgdaSpace{}%
\AgdaBound{i}\AgdaSpace{}%
\AgdaSymbol{→}\AgdaSpace{}%
\AgdaFunction{conv}\AgdaSpace{}%
\AgdaOperator{\AgdaFunction{⟨}}\AgdaSpace{}%
\AgdaBound{inp}\AgdaSpace{}%
\AgdaOperator{\AgdaFunction{⟩}}\AgdaSpace{}%
\AgdaSymbol{(}\AgdaInductiveConstructor{sel}\AgdaSpace{}%
\AgdaOperator{\AgdaFunction{⟨}}\AgdaSpace{}%
\AgdaBound{wᵥ}\AgdaSpace{}%
\AgdaOperator{\AgdaFunction{⟩}}\AgdaSpace{}%
\AgdaBound{i}\AgdaSymbol{)}\AgdaSpace{}%
\AgdaOperator{\AgdaInductiveConstructor{⊞}}\AgdaSpace{}%
\AgdaFunction{Imaps}\AgdaSpace{}%
\AgdaSymbol{λ}\AgdaSpace{}%
\AgdaBound{\AgdaUnderscore{}}\AgdaSpace{}%
\AgdaSymbol{→}\AgdaSpace{}%
\AgdaInductiveConstructor{sels}\AgdaSpace{}%
\AgdaOperator{\AgdaFunction{⟨}}\AgdaSpace{}%
\AgdaBound{bᵥ}\AgdaSpace{}%
\AgdaOperator{\AgdaFunction{⟩}}\AgdaSpace{}%
\AgdaBound{i}\<%
\\
%
\\[\AgdaEmptyExtraSkip]%
%
\>[2]\AgdaFunction{avgp₂}%
\>[2209I]\AgdaSymbol{:}\AgdaSpace{}%
\AgdaSymbol{∀}\AgdaSpace{}%
\AgdaBound{m}\AgdaSpace{}%
\AgdaBound{n}\AgdaSpace{}%
\AgdaSymbol{→}\AgdaSpace{}%
\AgdaSymbol{(}\AgdaBound{a}\AgdaSpace{}%
\AgdaSymbol{:}\AgdaSpace{}%
\AgdaDatatype{E}\AgdaSpace{}%
\AgdaGeneralizable{Γ}\AgdaSpace{}%
\AgdaSymbol{(}\AgdaInductiveConstructor{ar}\AgdaSpace{}%
\AgdaSymbol{(}\AgdaBound{m}\AgdaSpace{}%
\AgdaOperator{\AgdaPrimitive{ℕ.*}}\AgdaSpace{}%
\AgdaNumber{2}\AgdaSpace{}%
\AgdaOperator{\AgdaInductiveConstructor{∷}}\AgdaSpace{}%
\AgdaBound{n}\AgdaSpace{}%
\AgdaOperator{\AgdaPrimitive{ℕ.*}}\AgdaSpace{}%
\AgdaNumber{2}\AgdaSpace{}%
\AgdaOperator{\AgdaInductiveConstructor{∷}}\AgdaSpace{}%
\AgdaInductiveConstructor{[]}\AgdaSymbol{)))}\<%
\\
\>[.][@{}l@{}]\<[2209I]%
\>[8]\AgdaSymbol{→}\AgdaSpace{}%
\AgdaDatatype{E}\AgdaSpace{}%
\AgdaGeneralizable{Γ}\AgdaSpace{}%
\AgdaSymbol{(}\AgdaInductiveConstructor{ar}\AgdaSpace{}%
\AgdaSymbol{(}\AgdaBound{m}\AgdaSpace{}%
\AgdaOperator{\AgdaInductiveConstructor{∷}}\AgdaSpace{}%
\AgdaBound{n}\AgdaSpace{}%
\AgdaOperator{\AgdaInductiveConstructor{∷}}\AgdaSpace{}%
\AgdaInductiveConstructor{[]}\AgdaSymbol{))}\<%
\\
%
\>[2]\AgdaFunction{avgp₂}\AgdaSpace{}%
\AgdaBound{m}\AgdaSpace{}%
\AgdaBound{n}\AgdaSpace{}%
\AgdaBound{a}\AgdaSpace{}%
\AgdaSymbol{=}\<%
\\
\>[2][@{}l@{\AgdaIndent{0}}]%
\>[4]\AgdaFunction{Imaps}\AgdaSpace{}%
\AgdaSymbol{λ}\AgdaSpace{}%
\AgdaBound{i}\AgdaSpace{}%
\AgdaSymbol{→}\AgdaSpace{}%
\AgdaInductiveConstructor{scaledown}\AgdaSpace{}%
\AgdaNumber{4}\AgdaSpace{}%
\AgdaOperator{\AgdaFunction{\$}}\AgdaSpace{}%
\AgdaFunction{Sum}\AgdaSpace{}%
\AgdaSymbol{λ}\AgdaSpace{}%
\AgdaBound{j}\AgdaSpace{}%
\AgdaSymbol{→}\AgdaSpace{}%
\AgdaInductiveConstructor{sels}\AgdaSpace{}%
\AgdaSymbol{(}\AgdaInductiveConstructor{selb}\AgdaSpace{}%
\AgdaFunction{it}\AgdaSpace{}%
\AgdaOperator{\AgdaFunction{⟨}}\AgdaSpace{}%
\AgdaBound{a}\AgdaSpace{}%
\AgdaOperator{\AgdaFunction{⟩}}\AgdaSpace{}%
\AgdaBound{i}\AgdaSymbol{)}\AgdaSpace{}%
\AgdaBound{j}\<%
\\
%
\\[\AgdaEmptyExtraSkip]%
%
\>[2]\AgdaFunction{sqerr}\AgdaSpace{}%
\AgdaSymbol{:}\AgdaSpace{}%
\AgdaSymbol{(}\AgdaBound{r}\AgdaSpace{}%
\AgdaBound{o}\AgdaSpace{}%
\AgdaSymbol{:}\AgdaSpace{}%
\AgdaDatatype{E}\AgdaSpace{}%
\AgdaGeneralizable{Γ}\AgdaSpace{}%
\AgdaSymbol{(}\AgdaInductiveConstructor{ar}\AgdaSpace{}%
\AgdaInductiveConstructor{[]}\AgdaSymbol{))}\AgdaSpace{}%
\AgdaSymbol{→}\AgdaSpace{}%
\AgdaDatatype{E}\AgdaSpace{}%
\AgdaGeneralizable{Γ}\AgdaSpace{}%
\AgdaSymbol{(}\AgdaInductiveConstructor{ar}\AgdaSpace{}%
\AgdaInductiveConstructor{[]}\AgdaSymbol{)}\<%
\\
%
\>[2]\AgdaFunction{sqerr}\AgdaSpace{}%
\AgdaBound{r}\AgdaSpace{}%
\AgdaBound{o}\AgdaSpace{}%
\AgdaSymbol{=}\AgdaSpace{}%
\AgdaInductiveConstructor{scaledown}\AgdaSpace{}%
\AgdaNumber{2}\AgdaSpace{}%
\AgdaSymbol{((}\AgdaBound{r}\AgdaSpace{}%
\AgdaOperator{\AgdaInductiveConstructor{⊞}}\AgdaSpace{}%
\AgdaSymbol{(}\AgdaInductiveConstructor{minus}\AgdaSpace{}%
\AgdaBound{o}\AgdaSymbol{))}\AgdaSpace{}%
\AgdaOperator{\AgdaInductiveConstructor{⊠}}\AgdaSpace{}%
\AgdaSymbol{(}\AgdaBound{r}\AgdaSpace{}%
\AgdaOperator{\AgdaInductiveConstructor{⊞}}\AgdaSpace{}%
\AgdaSymbol{(}\AgdaInductiveConstructor{minus}\AgdaSpace{}%
\AgdaBound{o}\AgdaSymbol{)))}\<%
\\
%
\\[\AgdaEmptyExtraSkip]%
%
\>[2]\AgdaFunction{meansqerr}\AgdaSpace{}%
\AgdaSymbol{:}\AgdaSpace{}%
\AgdaSymbol{(}\AgdaBound{r}\AgdaSpace{}%
\AgdaBound{o}\AgdaSpace{}%
\AgdaSymbol{:}\AgdaSpace{}%
\AgdaDatatype{E}\AgdaSpace{}%
\AgdaGeneralizable{Γ}\AgdaSpace{}%
\AgdaSymbol{(}\AgdaInductiveConstructor{ar}\AgdaSpace{}%
\AgdaGeneralizable{s}\AgdaSymbol{))}\AgdaSpace{}%
\AgdaSymbol{→}\AgdaSpace{}%
\AgdaDatatype{E}\AgdaSpace{}%
\AgdaGeneralizable{Γ}\AgdaSpace{}%
\AgdaSymbol{(}\AgdaInductiveConstructor{ar}\AgdaSpace{}%
\AgdaInductiveConstructor{[]}\AgdaSymbol{)}\<%
\\
%
\>[2]\AgdaFunction{meansqerr}\AgdaSpace{}%
\AgdaBound{r}\AgdaSpace{}%
\AgdaBound{o}\AgdaSpace{}%
\AgdaSymbol{=}\AgdaSpace{}%
\AgdaFunction{Sum}\AgdaSpace{}%
\AgdaSymbol{λ}\AgdaSpace{}%
\AgdaBound{i}\AgdaSpace{}%
\AgdaSymbol{→}\AgdaSpace{}%
\AgdaFunction{sqerr}\AgdaSpace{}%
\AgdaSymbol{(}\AgdaInductiveConstructor{sels}\AgdaSpace{}%
\AgdaOperator{\AgdaFunction{⟨}}\AgdaSpace{}%
\AgdaBound{r}\AgdaSpace{}%
\AgdaOperator{\AgdaFunction{⟩}}\AgdaSpace{}%
\AgdaBound{i}\AgdaSymbol{)}\AgdaSpace{}%
\AgdaSymbol{(}\AgdaInductiveConstructor{sels}\AgdaSpace{}%
\AgdaOperator{\AgdaFunction{⟨}}\AgdaSpace{}%
\AgdaBound{o}\AgdaSpace{}%
\AgdaOperator{\AgdaFunction{⟩}}\AgdaSpace{}%
\AgdaBound{i}\AgdaSymbol{)}\<%
\end{code}
where \AF{⟨\_⟩} lifts an expression into a generalised expression
simply by applying weakening according to the prefix.

Finally, the CNN embedded in $E$ is given as follows:
\begin{code}%
%
\>[2]\AgdaFunction{cnn}\AgdaSpace{}%
\AgdaSymbol{:}\AgdaSpace{}%
\AgdaDatatype{E}\AgdaSpace{}%
\AgdaSymbol{\AgdaUnderscore{}}\AgdaSpace{}%
\AgdaSymbol{\AgdaUnderscore{}}\<%
\\
%
\>[2]\AgdaFunction{cnn}\AgdaSpace{}%
\AgdaSymbol{=}%
\>[2321I]\AgdaFunction{Lcon}%
\>[2322I]\AgdaSymbol{(}%
\>[16]\AgdaInductiveConstructor{ar}\AgdaSpace{}%
\AgdaSymbol{(}\AgdaNumber{28}\AgdaSpace{}%
\AgdaOperator{\AgdaInductiveConstructor{∷}}\AgdaSpace{}%
\AgdaNumber{28}\AgdaSpace{}%
\AgdaOperator{\AgdaInductiveConstructor{∷}}\AgdaSpace{}%
\AgdaInductiveConstructor{[]}\AgdaSymbol{)}\AgdaSpace{}%
\AgdaOperator{\AgdaInductiveConstructor{∷}}\AgdaSpace{}%
\AgdaInductiveConstructor{ar}\AgdaSpace{}%
\AgdaSymbol{(}\AgdaNumber{6}\AgdaSpace{}%
\AgdaOperator{\AgdaInductiveConstructor{∷}}\AgdaSpace{}%
\AgdaNumber{5}\AgdaSpace{}%
\AgdaOperator{\AgdaInductiveConstructor{∷}}\AgdaSpace{}%
\AgdaNumber{5}\AgdaSpace{}%
\AgdaOperator{\AgdaInductiveConstructor{∷}}\AgdaSpace{}%
\AgdaInductiveConstructor{[]}\AgdaSymbol{)}\<%
\\
\>[2322I][@{}l@{\AgdaIndent{0}}]%
\>[14]\AgdaOperator{\AgdaInductiveConstructor{∷}}\AgdaSpace{}%
\AgdaInductiveConstructor{ar}\AgdaSpace{}%
\AgdaSymbol{(}\AgdaNumber{6}\AgdaSpace{}%
\AgdaOperator{\AgdaInductiveConstructor{∷}}\AgdaSpace{}%
\AgdaInductiveConstructor{[]}\AgdaSymbol{)}%
\>[34]\AgdaOperator{\AgdaInductiveConstructor{∷}}\AgdaSpace{}%
\AgdaInductiveConstructor{ar}\AgdaSpace{}%
\AgdaSymbol{(}\AgdaNumber{12}\AgdaSpace{}%
\AgdaOperator{\AgdaInductiveConstructor{∷}}\AgdaSpace{}%
\AgdaNumber{6}\AgdaSpace{}%
\AgdaOperator{\AgdaInductiveConstructor{∷}}\AgdaSpace{}%
\AgdaNumber{5}\AgdaSpace{}%
\AgdaOperator{\AgdaInductiveConstructor{∷}}\AgdaSpace{}%
\AgdaNumber{5}\AgdaSpace{}%
\AgdaOperator{\AgdaInductiveConstructor{∷}}\AgdaSpace{}%
\AgdaInductiveConstructor{[]}\AgdaSymbol{)}\<%
\\
%
\>[14]\AgdaOperator{\AgdaInductiveConstructor{∷}}\AgdaSpace{}%
\AgdaInductiveConstructor{ar}\AgdaSpace{}%
\AgdaSymbol{(}\AgdaNumber{12}\AgdaSpace{}%
\AgdaOperator{\AgdaInductiveConstructor{∷}}\AgdaSpace{}%
\AgdaInductiveConstructor{[]}\AgdaSymbol{)}%
\>[34]\AgdaOperator{\AgdaInductiveConstructor{∷}}\AgdaSpace{}%
\AgdaInductiveConstructor{ar}\AgdaSpace{}%
\AgdaSymbol{(}\AgdaNumber{10}\AgdaSpace{}%
\AgdaOperator{\AgdaInductiveConstructor{∷}}\AgdaSpace{}%
\AgdaNumber{12}\AgdaSpace{}%
\AgdaOperator{\AgdaInductiveConstructor{∷}}\AgdaSpace{}%
\AgdaNumber{1}\AgdaSpace{}%
\AgdaOperator{\AgdaInductiveConstructor{∷}}\AgdaSpace{}%
\AgdaNumber{4}\AgdaSpace{}%
\AgdaOperator{\AgdaInductiveConstructor{∷}}\AgdaSpace{}%
\AgdaNumber{4}\AgdaSpace{}%
\AgdaOperator{\AgdaInductiveConstructor{∷}}\AgdaSpace{}%
\AgdaInductiveConstructor{[]}\AgdaSymbol{)}\<%
\\
%
\>[14]\AgdaOperator{\AgdaInductiveConstructor{∷}}\AgdaSpace{}%
\AgdaInductiveConstructor{ar}\AgdaSpace{}%
\AgdaSymbol{(}\AgdaNumber{10}\AgdaSpace{}%
\AgdaOperator{\AgdaInductiveConstructor{∷}}\AgdaSpace{}%
\AgdaInductiveConstructor{[]}\AgdaSymbol{)}%
\>[34]\AgdaOperator{\AgdaInductiveConstructor{∷}}\AgdaSpace{}%
\AgdaInductiveConstructor{ar}\AgdaSpace{}%
\AgdaSymbol{(}\AgdaNumber{10}\AgdaSpace{}%
\AgdaOperator{\AgdaInductiveConstructor{∷}}\AgdaSpace{}%
\AgdaNumber{1}\AgdaSpace{}%
\AgdaOperator{\AgdaInductiveConstructor{∷}}\AgdaSpace{}%
\AgdaNumber{1}\AgdaSpace{}%
\AgdaOperator{\AgdaInductiveConstructor{∷}}\AgdaSpace{}%
\AgdaNumber{1}\AgdaSpace{}%
\AgdaOperator{\AgdaInductiveConstructor{∷}}\AgdaSpace{}%
\AgdaNumber{1}\AgdaSpace{}%
\AgdaOperator{\AgdaInductiveConstructor{∷}}\AgdaSpace{}%
\AgdaInductiveConstructor{[]}\AgdaSymbol{)}\<%
\\
%
\>[14]\AgdaOperator{\AgdaInductiveConstructor{∷}}\AgdaSpace{}%
\AgdaInductiveConstructor{[]}\AgdaSymbol{)}\<%
\\
\>[.][@{}l@{}]\<[2322I]%
\>[13]\AgdaSymbol{(}\AgdaInductiveConstructor{ar}\AgdaSpace{}%
\AgdaInductiveConstructor{[]}\AgdaSymbol{)}\AgdaSpace{}%
\AgdaInductiveConstructor{ε}\<%
\\
\>[.][@{}l@{}]\<[2321I]%
\>[8]\AgdaSymbol{λ}\AgdaSpace{}%
\AgdaBound{inp}\AgdaSpace{}%
\AgdaBound{k₁}\AgdaSpace{}%
\AgdaBound{b₁}\AgdaSpace{}%
\AgdaBound{k₂}\AgdaSpace{}%
\AgdaBound{b₂}\AgdaSpace{}%
\AgdaBound{fc}\AgdaSpace{}%
\AgdaBound{b}\AgdaSpace{}%
\AgdaBound{target}\AgdaSpace{}%
\AgdaSymbol{→}\<%
\\
%
\>[8]\AgdaFunction{Let}\AgdaSpace{}%
\AgdaBound{c₁₁}\AgdaSpace{}%
\AgdaFunction{:=}\AgdaSpace{}%
\AgdaFunction{mconv}\AgdaSpace{}%
\AgdaBound{inp}\AgdaSpace{}%
\AgdaBound{k₁}\AgdaSpace{}%
\AgdaBound{b₁}%
\>[36]\AgdaFunction{In}\<%
\\
%
\>[8]\AgdaFunction{Let}\AgdaSpace{}%
\AgdaBound{c₁}%
\>[16]\AgdaFunction{:=}\AgdaSpace{}%
\AgdaInductiveConstructor{logistic}\AgdaSpace{}%
\AgdaBound{c₁₁}\AgdaSpace{}%
\AgdaFunction{In}\<%
\\
%
\>[8]\AgdaFunction{Let}\AgdaSpace{}%
\AgdaBound{s₁}%
\>[16]\AgdaFunction{:=}\AgdaSpace{}%
\AgdaSymbol{(}\AgdaFunction{Imap}\AgdaSpace{}%
\AgdaSymbol{\{}\AgdaArgument{s}\AgdaSpace{}%
\AgdaSymbol{=}\AgdaSpace{}%
\AgdaNumber{6}\AgdaSpace{}%
\AgdaOperator{\AgdaInductiveConstructor{∷}}\AgdaSpace{}%
\AgdaInductiveConstructor{[]}\AgdaSymbol{\}}\AgdaSpace{}%
\AgdaSymbol{λ}\AgdaSpace{}%
\AgdaBound{i}\AgdaSpace{}%
\AgdaSymbol{→}\AgdaSpace{}%
\AgdaFunction{avgp₂}\AgdaSpace{}%
\AgdaNumber{12}\AgdaSpace{}%
\AgdaNumber{12}\AgdaSpace{}%
\AgdaSymbol{(}\AgdaInductiveConstructor{sel}\AgdaSpace{}%
\AgdaBound{c₁}\AgdaSpace{}%
\AgdaBound{i}\AgdaSymbol{))}\AgdaSpace{}%
\AgdaFunction{In}\<%
\\
%
\>[8]\AgdaFunction{Let}\AgdaSpace{}%
\AgdaBound{c₂₁}\AgdaSpace{}%
\AgdaFunction{:=}\AgdaSpace{}%
\AgdaFunction{mconv}\AgdaSpace{}%
\AgdaBound{s₁}\AgdaSpace{}%
\AgdaBound{k₂}\AgdaSpace{}%
\AgdaBound{b₂}\AgdaSpace{}%
\AgdaFunction{In}\<%
\\
%
\>[8]\AgdaFunction{Let}\AgdaSpace{}%
\AgdaBound{c₂}%
\>[16]\AgdaFunction{:=}\AgdaSpace{}%
\AgdaInductiveConstructor{logistic}\AgdaSpace{}%
\AgdaBound{c₂₁}\AgdaSpace{}%
\AgdaFunction{In}\<%
\\
%
\>[8]\AgdaFunction{Let}\AgdaSpace{}%
\AgdaBound{s₂}%
\>[16]\AgdaFunction{:=}\AgdaSpace{}%
\AgdaSymbol{(}\AgdaFunction{Imap}\AgdaSpace{}%
\AgdaSymbol{\{}\AgdaArgument{s}\AgdaSpace{}%
\AgdaSymbol{=}\AgdaSpace{}%
\AgdaNumber{12}\AgdaSpace{}%
\AgdaOperator{\AgdaInductiveConstructor{∷}}\AgdaSpace{}%
\AgdaNumber{1}\AgdaSpace{}%
\AgdaOperator{\AgdaInductiveConstructor{∷}}\AgdaSpace{}%
\AgdaInductiveConstructor{[]}\AgdaSymbol{\}}\AgdaSpace{}%
\AgdaSymbol{λ}\AgdaSpace{}%
\AgdaBound{i}\AgdaSpace{}%
\AgdaSymbol{→}\AgdaSpace{}%
\AgdaFunction{avgp₂}\AgdaSpace{}%
\AgdaNumber{4}\AgdaSpace{}%
\AgdaNumber{4}\AgdaSpace{}%
\AgdaSymbol{(}\AgdaInductiveConstructor{sel}\AgdaSpace{}%
\AgdaBound{c₂}\AgdaSpace{}%
\AgdaBound{i}\AgdaSymbol{))}\AgdaSpace{}%
\AgdaFunction{In}\<%
\\
%
\>[8]\AgdaFunction{Let}\AgdaSpace{}%
\AgdaBound{o₁}%
\>[16]\AgdaFunction{:=}\AgdaSpace{}%
\AgdaFunction{mconv}\AgdaSpace{}%
\AgdaBound{s₂}\AgdaSpace{}%
\AgdaBound{fc}\AgdaSpace{}%
\AgdaBound{b}\AgdaSpace{}%
\AgdaFunction{In}\<%
\\
%
\>[8]\AgdaFunction{Let}\AgdaSpace{}%
\AgdaBound{o}%
\>[16]\AgdaFunction{:=}\AgdaSpace{}%
\AgdaInductiveConstructor{logistic}\AgdaSpace{}%
\AgdaBound{o₁}\AgdaSpace{}%
\AgdaFunction{In}\<%
\\
%
\>[8]\AgdaFunction{Let}\AgdaSpace{}%
\AgdaBound{e}%
\>[16]\AgdaFunction{:=}\AgdaSpace{}%
\AgdaFunction{meansqerr}\AgdaSpace{}%
\AgdaBound{target}\AgdaSpace{}%
\AgdaBound{o}\AgdaSpace{}%
\AgdaFunction{In}\<%
\\
%
\>[8]\AgdaBound{e}\<%
\\
\>[0]\<%
\end{code}

% \paragraph{Building Blocks}
% Now we implement the remaining building blocks in \AD{E} that are needed
% to define our CNN.
% \begin{code}[hide]
% module BB where
%   open import Data.Nat as ℕ using (ℕ; zero; suc)
%   open Array hiding (sum; slide; backslide)
%   open Lang
%   open SubWk using (wk; ↑_; ↑↑_)
% 
%   --_⊞_ _⊠_ : (a b : E Γ (ar s)) → E Γ (ar s)
%   Imapₛ : (E (Γ ▹ ix s) (ix s) → E (Γ ▹ ix s) (ar unit)) → E Γ (ar s)
%   Imap : (E (Γ ▹ ix s) (ix s) → E (Γ ▹ ix s) (ar p)) → E Γ (ar (s ⊗ p))
%   Sum : (E (Γ ▹ ix s) (ix s) → E (Γ ▹ ix s) (ar p)) → E Γ (ar p)
% \end{code}
% We start with a several convenience functions that wrap \AC{imap}s and \AC{sum}
% such that when we write (\AF{Imap} \AB{λ} \AB{i} \AB{→} \AB{⋯}), Agda's variable
% $i$ is mapped to the \AF{E}'s variable \AC{v₀}.
% \begin{mathpar}
% \codeblock{\begin{code}
%   Imapₛ f = imapₛ (f (var v₀))
% \end{code}}
% \and
% \codeblock{\begin{code}
%   Imap f = imap (f (var v₀))
% \end{code}}
% \and
% \codeblock{\begin{code}
%   Sum f = sum (f (var v₀))
% \end{code}}
% \end{mathpar}
% 
% The remaining operations are \AF{conv}, \AF{mconv} and \AF{avgp₂} which
% can be defined as functions on \AF{E} as follows.
% \begin{code}
%   conv : E Γ (ar r) → s + p ≈ r → E Γ (ar s) → suc p ≈ u → E Γ (ar u)
%   conv f sp g su = Sum λ i → slide i sp (↑ f) su ⊠ Imapₛ λ _ → selₛ (↑↑ g) (↑ i)
% 
%   mconv : s + p ≈ r → (inp : E Γ (ar r)) (we : E Γ (ar (u ⊗ s))) (b : E Γ (ar u))
%         → suc p ≈ w → E Γ (ar (u ⊗ w))
%   mconv sp inp we b su = Imap λ i → conv (↑ inp) sp (sel (↑ we) i) su ⊞ Imapₛ λ _ → selₛ (↑↑ b) (↑ i)
% 
%   avgp₂ : ∀ m n → (a : E Γ (ar (ι (m ℕ.* 2) ⊗ ι (n ℕ.* 2)))) → E Γ (ar (ι m ⊗ ι n))
%   avgp₂ m n a = Imapₛ λ i → scaledown 4 $ Sum λ j → selₛ (selb (ι ⊗ ι) (↑↑ a) (↑ i)) j
% 
% \end{code}
% Note that these definitions are not very different from those found in
% Section~\ref{sec:array-theory}.  Some operations such as \AF{nest} and \AF{unnest}
% got inlined into \AF{E}'s operators, and all we really have to take care of is 
% weakening of the expressions whenever we go under binders.


\begin{code}[hide]%
\>[0]\AgdaKeyword{open}\AgdaSpace{}%
\AgdaKeyword{import}\AgdaSpace{}%
\AgdaModule{Relation.Binary.PropositionalEquality}\<%
\\
\>[0]\AgdaKeyword{open}\AgdaSpace{}%
\AgdaKeyword{import}\AgdaSpace{}%
\AgdaModule{Relation.Nullary}\<%
\\
\>[0]\AgdaKeyword{open}\AgdaSpace{}%
\AgdaKeyword{import}\AgdaSpace{}%
\AgdaModule{Data.List}\AgdaSpace{}%
\AgdaKeyword{using}\AgdaSpace{}%
\AgdaSymbol{(}\AgdaDatatype{List}\AgdaSymbol{;}\AgdaSpace{}%
\AgdaInductiveConstructor{[]}\AgdaSymbol{;}\AgdaSpace{}%
\AgdaOperator{\AgdaInductiveConstructor{\AgdaUnderscore{}∷\AgdaUnderscore{}}}\AgdaSymbol{)}\<%
\\
\>[0]\AgdaKeyword{open}\AgdaSpace{}%
\AgdaKeyword{import}\AgdaSpace{}%
\AgdaModule{Data.Empty}\<%
\\
\>[0]\AgdaKeyword{open}\AgdaSpace{}%
\AgdaKeyword{import}\AgdaSpace{}%
\AgdaModule{Function}\<%
\\
%
\\[\AgdaEmptyExtraSkip]%
\>[0]\AgdaComment{--\ Our\ local\ files.}\<%
\\
\>[0]\AgdaKeyword{open}\AgdaSpace{}%
\AgdaKeyword{import}\AgdaSpace{}%
\AgdaModule{arrays}\<%
\\
\>[0]\AgdaKeyword{open}\AgdaSpace{}%
\AgdaKeyword{import}\AgdaSpace{}%
\AgdaModule{lang}\<%
\\
\>[0]\AgdaKeyword{module}\AgdaSpace{}%
\AgdaModule{\AgdaUnderscore{}}\AgdaSpace{}%
\AgdaKeyword{where}\<%
\end{code}

\section{Automatic Differentiation\label{sec:ad}}

We implement automatic differentiation in reverse mode
for expressions in \AF{E}.  We focus on reverse mode because it is
of most interest in machine learning, and it is more challenging to implement.
We start with a brief introduction of the AD, for much more in-depth
explanations refer to~\cite{autodiff-survey, backprop-stlc}.   Consider differentiating
a function composition consisting of three functions:
\[ 
   y = (f \circ g \circ h)\ x
\]
rewrite it using temporary variables:
\begin{eqnarray*}
  w_0 &=& x \\
  w_1 &=& h\ w_0 \\
  w_2 &=& g\ w_1 \\
  w_3 &=& f\ w_2 = y
\end{eqnarray*}
The chain rule gives us 
$\frac{\partial y}{\partial x} 
  = \frac{\partial y}{\partial w_2}
    \frac{\partial w_2}{\partial w_1}
    \frac{\partial w_1}{\partial x}$.  The difference between the forward and reverse
    mode lies in the direction that we traverse the chain rule.  In forward mode we
    traverse the chain inside-out, and the revers mode traverses the chain outside-in
    thus computing recursive relation:
$\frac{\partial y}{\partial w_i}
  = \frac{\partial y}{\partial w_{i+1}}
    \frac{\partial w_{i+1}}{\partial w_i}$.  For our example, we compute
$\frac{\partial y}{\partial w_2}$, then $\frac{\partial w_2}{\partial w_1}$ and
finally $\frac{\partial w_1}{\partial x}$.  While there seem to be no difference for
functions of one variable, there is a big difference for functions of $n$ variables
as we can compute derivatives of all the non-dependent variables simultaneously.
Consider an example of the $z = f\ x\ y = sin(xy + x)$:
\begin{eqnarray*}
  w_0 &=& x \\
  w_1 &=& y \\
  w_2 &=& w_1w_2\\
  w_3 &=& w_2 + w_0 \\
  w_4 &=& \sin w_3 = z
\end{eqnarray*}
We compute the adjoints $\bar{w}_i = \frac{\partial y}{\partial w_i}$ using the following
rule.  If $w_i$ has successors in the computational graph, we can apply the chain rule
as follows:
\[ 
    \bar{w}_i = \sum_{j \in succ\ i} \bar{w}_j\frac{\partial w_j}{\partial w_i}
\]
For our example:
\begin{eqnarray*}
  \bar{w}_4 &=& 1 = \frac{\partial z}{\partial z} \\
  \bar{w}_3 &=& \bar{w}_4 \cos w_3\\
  \bar{w}_2 &=& \bar{w}_3 \cdot 1 \\
  \bar{w}_1 &=& \bar{w}_2 w_0 \\
  \bar{w}_0 &=& \bar{w}_3 + \bar{w}_2 w_1
\end{eqnarray*}
If we inline all the $\bar{w}_i$ definitions and inspect the values of partial derivatives
with respect to $x$ and $y$ we obtain expected results:
$\frac{\partial z}{\partial x} = \cos (xy + x)(y + 1)$ and
$\frac{\partial z}{\partial y} = \cos (xy + x)x$.


In the implementation of the AD for \AF{E} in some context \AB{Γ}, we would like to obtain
all the partial derivatives with respect to the variables in context \AB{Γ}.  Each partial
derivative is itself an expression \AF{E} in context \AF{Γ}.  Therefore, we need to define
a data type for an environment of \AB{Γ}-many expressions in context \AB{Γ}.  We call this
environment \AF{Env} defined as follows:
\begin{code}[hide]%
\>[0]\AgdaKeyword{module}\AgdaSpace{}%
\AgdaModule{AD}\AgdaSpace{}%
\AgdaKeyword{where}\<%
\\
\>[0][@{}l@{\AgdaIndent{0}}]%
\>[2]\AgdaKeyword{open}\AgdaSpace{}%
\AgdaKeyword{import}\AgdaSpace{}%
\AgdaModule{Data.Unit}\<%
\\
%
\>[2]\AgdaKeyword{open}\AgdaSpace{}%
\AgdaKeyword{import}\AgdaSpace{}%
\AgdaModule{Data.Product}\AgdaSpace{}%
\AgdaSymbol{as}\AgdaSpace{}%
\AgdaModule{Prod}\<%
\\
%
\>[2]\AgdaKeyword{open}\AgdaSpace{}%
\AgdaModule{Array}\AgdaSpace{}%
\AgdaKeyword{hiding}\AgdaSpace{}%
\AgdaSymbol{(}\AgdaFunction{sum}\AgdaSymbol{;}\AgdaSpace{}%
\AgdaFunction{backslide}\AgdaSymbol{;}\AgdaSpace{}%
\AgdaFunction{slide}\AgdaSymbol{)}\<%
\\
%
\>[2]\AgdaKeyword{open}\AgdaSpace{}%
\AgdaModule{SubWk}\AgdaSpace{}%
\AgdaKeyword{using}\AgdaSpace{}%
\AgdaSymbol{(}\AgdaFunction{wk}\AgdaSymbol{;}\AgdaSpace{}%
\AgdaOperator{\AgdaFunction{↑\AgdaUnderscore{}}}\AgdaSymbol{;}\AgdaSpace{}%
\AgdaOperator{\AgdaFunction{↑↑\AgdaUnderscore{}}}\AgdaSymbol{)}\<%
\\
%
\>[2]\AgdaKeyword{open}\AgdaSpace{}%
\AgdaModule{Lang}\<%
\\
%
\\[\AgdaEmptyExtraSkip]%
%
\>[2]\AgdaComment{--\ Left-associated\ pairing}\<%
\\
%
\>[2]\AgdaKeyword{infixl}\AgdaSpace{}%
\AgdaNumber{4}\AgdaSpace{}%
\AgdaOperator{\AgdaFunction{\AgdaUnderscore{},,\AgdaUnderscore{}}}\<%
\\
%
\>[2]\AgdaOperator{\AgdaFunction{\AgdaUnderscore{},,\AgdaUnderscore{}}}\AgdaSpace{}%
\AgdaSymbol{:}\AgdaSpace{}%
\AgdaGeneralizable{X}\AgdaSpace{}%
\AgdaSymbol{→}\AgdaSpace{}%
\AgdaGeneralizable{Y}\AgdaSpace{}%
\AgdaSymbol{→}\AgdaSpace{}%
\AgdaGeneralizable{X}\AgdaSpace{}%
\AgdaOperator{\AgdaFunction{×}}\AgdaSpace{}%
\AgdaGeneralizable{Y}\<%
\\
%
\>[2]\AgdaOperator{\AgdaFunction{\AgdaUnderscore{},,\AgdaUnderscore{}}}\AgdaSpace{}%
\AgdaSymbol{=}\AgdaSpace{}%
\AgdaOperator{\AgdaFunction{Prod.\AgdaUnderscore{},′\AgdaUnderscore{}}}\<%
\end{code}
\begin{code}%
%
\>[2]\AgdaFunction{Env}\AgdaSpace{}%
\AgdaSymbol{:}\AgdaSpace{}%
\AgdaDatatype{Ctx}\AgdaSpace{}%
\AgdaSymbol{→}\AgdaSpace{}%
\AgdaDatatype{Ctx}\AgdaSpace{}%
\AgdaSymbol{→}\AgdaSpace{}%
\AgdaPrimitive{Set}\<%
\\
%
\>[2]\AgdaFunction{Env}\AgdaSpace{}%
\AgdaInductiveConstructor{ε}%
\>[18]\AgdaBound{Δ}%
\>[21]\AgdaSymbol{=}\AgdaSpace{}%
\AgdaRecord{⊤}\<%
\\
%
\>[2]\AgdaFunction{Env}\AgdaSpace{}%
\AgdaSymbol{(}\AgdaBound{Γ}\AgdaSpace{}%
\AgdaOperator{\AgdaInductiveConstructor{▹}}\AgdaSpace{}%
\AgdaInductiveConstructor{ar}\AgdaSpace{}%
\AgdaBound{s}\AgdaSymbol{)}%
\>[18]\AgdaBound{Δ}%
\>[21]\AgdaSymbol{=}\AgdaSpace{}%
\AgdaFunction{Env}\AgdaSpace{}%
\AgdaBound{Γ}\AgdaSpace{}%
\AgdaBound{Δ}\AgdaSpace{}%
\AgdaOperator{\AgdaFunction{×}}\AgdaSpace{}%
\AgdaDatatype{E}\AgdaSpace{}%
\AgdaBound{Δ}\AgdaSpace{}%
\AgdaSymbol{(}\AgdaInductiveConstructor{ar}\AgdaSpace{}%
\AgdaBound{s}\AgdaSymbol{)}\<%
\\
%
\>[2]\AgdaFunction{Env}\AgdaSpace{}%
\AgdaSymbol{(}\AgdaBound{Γ}\AgdaSpace{}%
\AgdaOperator{\AgdaInductiveConstructor{▹}}\AgdaSpace{}%
\AgdaInductiveConstructor{ix}\AgdaSpace{}%
\AgdaBound{s}\AgdaSymbol{)}%
\>[18]\AgdaBound{Δ}%
\>[21]\AgdaSymbol{=}\AgdaSpace{}%
\AgdaFunction{Env}\AgdaSpace{}%
\AgdaBound{Γ}\AgdaSpace{}%
\AgdaBound{Δ}\<%
\end{code}
Note that \AF{Env} only keeps array expressions, as (i) derivatives for indices do
not exist; and (ii) we can always make an initial environment by populating all the
elements with \AC{zero}s.  

We define several helper operations to manipulate environments: \AF{env-zero} is 
an environment where all the values are \AC{zero}s; \AF{update} modifies the 
expression at the $v$-th position by applying $f$ to it; \AF{env-map} applies the function
$f$ from \AF{E} to \AF{E} to all the elements of the environment; and \AF{env-zipWith}
applies the binary function $f$ on two environments point-wise.  The types of these
helper functions follow.  As environments are very similar to lists, the implementation
of the above functions are straight-forward.
\begin{code}%
%
\>[2]\AgdaFunction{env-zero}\AgdaSpace{}%
\AgdaSymbol{:}\AgdaSpace{}%
\AgdaFunction{Env}\AgdaSpace{}%
\AgdaGeneralizable{Γ}\AgdaSpace{}%
\AgdaGeneralizable{Δ}\<%
\\
%
\>[2]\AgdaFunction{update}\AgdaSpace{}%
\AgdaSymbol{:}\AgdaSpace{}%
\AgdaFunction{Env}\AgdaSpace{}%
\AgdaGeneralizable{Γ}\AgdaSpace{}%
\AgdaGeneralizable{Δ}\AgdaSpace{}%
\AgdaSymbol{→}\AgdaSpace{}%
\AgdaSymbol{(}\AgdaBound{v}\AgdaSpace{}%
\AgdaSymbol{:}\AgdaSpace{}%
\AgdaInductiveConstructor{ar}\AgdaSpace{}%
\AgdaGeneralizable{s}\AgdaSpace{}%
\AgdaOperator{\AgdaDatatype{∈}}\AgdaSpace{}%
\AgdaGeneralizable{Γ}\AgdaSymbol{)}\AgdaSpace{}%
\AgdaSymbol{→}\AgdaSpace{}%
\AgdaSymbol{(}\AgdaBound{f}\AgdaSpace{}%
\AgdaSymbol{:}\AgdaSpace{}%
\AgdaDatatype{E}\AgdaSpace{}%
\AgdaGeneralizable{Δ}\AgdaSpace{}%
\AgdaSymbol{(}\AgdaInductiveConstructor{ar}\AgdaSpace{}%
\AgdaGeneralizable{s}\AgdaSymbol{)}\AgdaSpace{}%
\AgdaSymbol{→}\AgdaSpace{}%
\AgdaDatatype{E}\AgdaSpace{}%
\AgdaGeneralizable{Δ}\AgdaSpace{}%
\AgdaSymbol{(}\AgdaInductiveConstructor{ar}\AgdaSpace{}%
\AgdaGeneralizable{s}\AgdaSymbol{))}\AgdaSpace{}%
\AgdaSymbol{→}\AgdaSpace{}%
\AgdaFunction{Env}\AgdaSpace{}%
\AgdaGeneralizable{Γ}\AgdaSpace{}%
\AgdaGeneralizable{Δ}\<%
\\
%
\>[2]\AgdaFunction{env-map}\AgdaSpace{}%
\AgdaSymbol{:}\AgdaSpace{}%
\AgdaSymbol{∀}\AgdaSpace{}%
\AgdaSymbol{\{}\AgdaBound{Γ}\AgdaSpace{}%
\AgdaBound{Δ}\AgdaSpace{}%
\AgdaBound{Ψ}\AgdaSymbol{\}}\AgdaSpace{}%
\AgdaSymbol{→}\AgdaSpace{}%
\AgdaSymbol{(}\AgdaBound{f}\AgdaSpace{}%
\AgdaSymbol{:}\AgdaSpace{}%
\AgdaSymbol{∀}\AgdaSpace{}%
\AgdaSymbol{\{}\AgdaBound{s}\AgdaSymbol{\}}\AgdaSpace{}%
\AgdaSymbol{→}\AgdaSpace{}%
\AgdaDatatype{E}\AgdaSpace{}%
\AgdaBound{Δ}\AgdaSpace{}%
\AgdaSymbol{(}\AgdaInductiveConstructor{ar}\AgdaSpace{}%
\AgdaBound{s}\AgdaSymbol{)}\AgdaSpace{}%
\AgdaSymbol{→}\AgdaSpace{}%
\AgdaDatatype{E}\AgdaSpace{}%
\AgdaBound{Ψ}\AgdaSpace{}%
\AgdaSymbol{(}\AgdaInductiveConstructor{ar}\AgdaSpace{}%
\AgdaBound{s}\AgdaSymbol{))}\AgdaSpace{}%
\AgdaSymbol{→}\AgdaSpace{}%
\AgdaFunction{Env}\AgdaSpace{}%
\AgdaBound{Γ}\AgdaSpace{}%
\AgdaBound{Δ}\AgdaSpace{}%
\AgdaSymbol{→}\AgdaSpace{}%
\AgdaFunction{Env}\AgdaSpace{}%
\AgdaBound{Γ}\AgdaSpace{}%
\AgdaBound{Ψ}\<%
\\
%
\>[2]\AgdaFunction{env-zipWith}%
\>[15]\AgdaSymbol{:}\AgdaSpace{}%
\AgdaSymbol{∀}\AgdaSpace{}%
\AgdaSymbol{\{}\AgdaBound{Γ}\AgdaSpace{}%
\AgdaBound{Δ}\AgdaSpace{}%
\AgdaBound{Ψ}\AgdaSpace{}%
\AgdaBound{Ξ}\AgdaSymbol{\}}\AgdaSpace{}%
\AgdaSymbol{→}\AgdaSpace{}%
\AgdaSymbol{(}\AgdaBound{f}\AgdaSpace{}%
\AgdaSymbol{:}\AgdaSpace{}%
\AgdaSymbol{∀}\AgdaSpace{}%
\AgdaSymbol{\{}\AgdaBound{s}\AgdaSymbol{\}}\AgdaSpace{}%
\AgdaSymbol{→}\AgdaSpace{}%
\AgdaDatatype{E}\AgdaSpace{}%
\AgdaBound{Δ}\AgdaSpace{}%
\AgdaSymbol{(}\AgdaInductiveConstructor{ar}\AgdaSpace{}%
\AgdaBound{s}\AgdaSymbol{)}\AgdaSpace{}%
\AgdaSymbol{→}\AgdaSpace{}%
\AgdaDatatype{E}\AgdaSpace{}%
\AgdaBound{Ψ}\AgdaSpace{}%
\AgdaSymbol{(}\AgdaInductiveConstructor{ar}\AgdaSpace{}%
\AgdaBound{s}\AgdaSymbol{)}\AgdaSpace{}%
\AgdaSymbol{→}\AgdaSpace{}%
\AgdaDatatype{E}\AgdaSpace{}%
\AgdaBound{Ξ}\AgdaSpace{}%
\AgdaSymbol{(}\AgdaInductiveConstructor{ar}\AgdaSpace{}%
\AgdaBound{s}\AgdaSymbol{))}\<%
\\
%
\>[15]\AgdaSymbol{→}\AgdaSpace{}%
\AgdaFunction{Env}\AgdaSpace{}%
\AgdaBound{Γ}\AgdaSpace{}%
\AgdaBound{Δ}\AgdaSpace{}%
\AgdaSymbol{→}\AgdaSpace{}%
\AgdaFunction{Env}\AgdaSpace{}%
\AgdaBound{Γ}\AgdaSpace{}%
\AgdaBound{Ψ}\AgdaSpace{}%
\AgdaSymbol{→}\AgdaSpace{}%
\AgdaFunction{Env}\AgdaSpace{}%
\AgdaBound{Γ}\AgdaSpace{}%
\AgdaBound{Ξ}\<%
\end{code}
\begin{code}[hide]%
%
\>[2]\AgdaFunction{update}\AgdaSpace{}%
\AgdaSymbol{\{}\AgdaBound{Γ}\AgdaSpace{}%
\AgdaOperator{\AgdaInductiveConstructor{▹}}\AgdaSpace{}%
\AgdaInductiveConstructor{ar}\AgdaSpace{}%
\AgdaBound{s}\AgdaSymbol{\}}\AgdaSpace{}%
\AgdaSymbol{(}\AgdaBound{ρ}\AgdaSpace{}%
\AgdaOperator{\AgdaInductiveConstructor{,}}\AgdaSpace{}%
\AgdaBound{e}\AgdaSymbol{)}\AgdaSpace{}%
\AgdaInductiveConstructor{v₀}\AgdaSpace{}%
\AgdaBound{f}\AgdaSpace{}%
\AgdaSymbol{=}\AgdaSpace{}%
\AgdaBound{ρ}\AgdaSpace{}%
\AgdaOperator{\AgdaInductiveConstructor{,}}\AgdaSpace{}%
\AgdaBound{f}\AgdaSpace{}%
\AgdaBound{e}\<%
\\
%
\>[2]\AgdaFunction{update}\AgdaSpace{}%
\AgdaSymbol{\{}\AgdaBound{Γ}\AgdaSpace{}%
\AgdaOperator{\AgdaInductiveConstructor{▹}}\AgdaSpace{}%
\AgdaInductiveConstructor{ix}\AgdaSpace{}%
\AgdaBound{s}\AgdaSymbol{\}}\AgdaSpace{}%
\AgdaBound{ρ}\AgdaSpace{}%
\AgdaSymbol{(}\AgdaInductiveConstructor{vₛ}\AgdaSpace{}%
\AgdaBound{x}\AgdaSymbol{)}\AgdaSpace{}%
\AgdaBound{f}\AgdaSpace{}%
\AgdaSymbol{=}\AgdaSpace{}%
\AgdaFunction{update}\AgdaSpace{}%
\AgdaBound{ρ}\AgdaSpace{}%
\AgdaBound{x}\AgdaSpace{}%
\AgdaBound{f}\<%
\\
%
\>[2]\AgdaFunction{update}\AgdaSpace{}%
\AgdaSymbol{\{}\AgdaBound{Γ}\AgdaSpace{}%
\AgdaOperator{\AgdaInductiveConstructor{▹}}\AgdaSpace{}%
\AgdaInductiveConstructor{ar}\AgdaSpace{}%
\AgdaBound{s}\AgdaSymbol{\}}\AgdaSpace{}%
\AgdaSymbol{(}\AgdaBound{ρ}\AgdaSpace{}%
\AgdaOperator{\AgdaInductiveConstructor{,}}\AgdaSpace{}%
\AgdaBound{e}\AgdaSymbol{)}\AgdaSpace{}%
\AgdaSymbol{(}\AgdaInductiveConstructor{vₛ}\AgdaSpace{}%
\AgdaBound{x}\AgdaSymbol{)}\AgdaSpace{}%
\AgdaBound{f}\AgdaSpace{}%
\AgdaSymbol{=}\AgdaSpace{}%
\AgdaFunction{update}\AgdaSpace{}%
\AgdaBound{ρ}\AgdaSpace{}%
\AgdaBound{x}\AgdaSpace{}%
\AgdaBound{f}\AgdaSpace{}%
\AgdaOperator{\AgdaInductiveConstructor{,}}\AgdaSpace{}%
\AgdaBound{e}\<%
\\
%
\\[\AgdaEmptyExtraSkip]%
%
\>[2]\AgdaFunction{env-ix}\AgdaSpace{}%
\AgdaSymbol{:}\AgdaSpace{}%
\AgdaFunction{Env}\AgdaSpace{}%
\AgdaGeneralizable{Γ}\AgdaSpace{}%
\AgdaGeneralizable{Δ}\AgdaSpace{}%
\AgdaSymbol{→}\AgdaSpace{}%
\AgdaSymbol{(}\AgdaBound{ix}\AgdaSpace{}%
\AgdaSymbol{:}\AgdaSpace{}%
\AgdaSymbol{(}\AgdaInductiveConstructor{ar}\AgdaSpace{}%
\AgdaGeneralizable{s}\AgdaSymbol{)}\AgdaSpace{}%
\AgdaOperator{\AgdaDatatype{∈}}\AgdaSpace{}%
\AgdaGeneralizable{Γ}\AgdaSymbol{)}\AgdaSpace{}%
\AgdaSymbol{→}\AgdaSpace{}%
\AgdaDatatype{E}\AgdaSpace{}%
\AgdaGeneralizable{Δ}\AgdaSpace{}%
\AgdaSymbol{(}\AgdaInductiveConstructor{ar}\AgdaSpace{}%
\AgdaGeneralizable{s}\AgdaSymbol{)}\<%
\\
%
\>[2]\AgdaFunction{env-ix}\AgdaSpace{}%
\AgdaSymbol{\{}\AgdaBound{Γ}\AgdaSpace{}%
\AgdaOperator{\AgdaInductiveConstructor{▹}}\AgdaSpace{}%
\AgdaInductiveConstructor{ix}\AgdaSpace{}%
\AgdaBound{s}\AgdaSymbol{\}}\AgdaSpace{}%
\AgdaBound{ρ}\AgdaSpace{}%
\AgdaSymbol{(}\AgdaInductiveConstructor{vₛ}\AgdaSpace{}%
\AgdaBound{x}\AgdaSymbol{)}\AgdaSpace{}%
\AgdaSymbol{=}\AgdaSpace{}%
\AgdaFunction{env-ix}\AgdaSpace{}%
\AgdaBound{ρ}\AgdaSpace{}%
\AgdaBound{x}\<%
\\
%
\>[2]\AgdaFunction{env-ix}\AgdaSpace{}%
\AgdaSymbol{\{}\AgdaBound{Γ}\AgdaSpace{}%
\AgdaOperator{\AgdaInductiveConstructor{▹}}\AgdaSpace{}%
\AgdaInductiveConstructor{ar}\AgdaSpace{}%
\AgdaBound{s}\AgdaSymbol{\}}\AgdaSpace{}%
\AgdaSymbol{(}\AgdaBound{ρ}\AgdaSpace{}%
\AgdaOperator{\AgdaInductiveConstructor{,}}\AgdaSpace{}%
\AgdaBound{e}\AgdaSymbol{)}\AgdaSpace{}%
\AgdaInductiveConstructor{v₀}\AgdaSpace{}%
\AgdaSymbol{=}\AgdaSpace{}%
\AgdaBound{e}\<%
\\
%
\>[2]\AgdaFunction{env-ix}\AgdaSpace{}%
\AgdaSymbol{\{}\AgdaBound{Γ}\AgdaSpace{}%
\AgdaOperator{\AgdaInductiveConstructor{▹}}\AgdaSpace{}%
\AgdaInductiveConstructor{ar}\AgdaSpace{}%
\AgdaBound{s}\AgdaSymbol{\}}\AgdaSpace{}%
\AgdaSymbol{(}\AgdaBound{ρ}\AgdaSpace{}%
\AgdaOperator{\AgdaInductiveConstructor{,}}\AgdaSpace{}%
\AgdaBound{e}\AgdaSymbol{)}\AgdaSpace{}%
\AgdaSymbol{(}\AgdaInductiveConstructor{vₛ}\AgdaSpace{}%
\AgdaBound{x}\AgdaSymbol{)}\AgdaSpace{}%
\AgdaSymbol{=}\AgdaSpace{}%
\AgdaFunction{env-ix}\AgdaSpace{}%
\AgdaBound{ρ}\AgdaSpace{}%
\AgdaBound{x}\<%
\\
%
\\[\AgdaEmptyExtraSkip]%
%
\>[2]\AgdaComment{--\ Update\ array\ values\ in\ the\ environment}\<%
\\
%
\>[2]\AgdaFunction{env-imap}\AgdaSpace{}%
\AgdaSymbol{:}\AgdaSpace{}%
\AgdaSymbol{(∀}\AgdaSpace{}%
\AgdaSymbol{\{}\AgdaBound{s}\AgdaSymbol{\}}\AgdaSpace{}%
\AgdaSymbol{→}\AgdaSpace{}%
\AgdaSymbol{(}\AgdaInductiveConstructor{ar}\AgdaSpace{}%
\AgdaBound{s}\AgdaSymbol{)}\AgdaSpace{}%
\AgdaOperator{\AgdaDatatype{∈}}\AgdaSpace{}%
\AgdaGeneralizable{Γ}\AgdaSpace{}%
\AgdaSymbol{→}\AgdaSpace{}%
\AgdaDatatype{E}\AgdaSpace{}%
\AgdaGeneralizable{Δ}\AgdaSpace{}%
\AgdaSymbol{(}\AgdaInductiveConstructor{ar}\AgdaSpace{}%
\AgdaBound{s}\AgdaSymbol{))}\AgdaSpace{}%
\AgdaSymbol{→}\AgdaSpace{}%
\AgdaFunction{Env}\AgdaSpace{}%
\AgdaGeneralizable{Γ}\AgdaSpace{}%
\AgdaGeneralizable{Δ}\AgdaSpace{}%
\AgdaComment{--→\ Env\ Γ\ Δ}\<%
\\
%
\>[2]\AgdaFunction{env-imap}\AgdaSpace{}%
\AgdaSymbol{\{}\AgdaArgument{Γ}\AgdaSpace{}%
\AgdaSymbol{=}\AgdaSpace{}%
\AgdaInductiveConstructor{ε}\AgdaSymbol{\}}%
\>[23]\AgdaBound{f}\AgdaSpace{}%
\AgdaSymbol{=}\AgdaSpace{}%
\AgdaInductiveConstructor{tt}\<%
\\
%
\>[2]\AgdaFunction{env-imap}\AgdaSpace{}%
\AgdaSymbol{\{}\AgdaArgument{Γ}\AgdaSpace{}%
\AgdaSymbol{=}\AgdaSpace{}%
\AgdaBound{Γ}\AgdaSpace{}%
\AgdaOperator{\AgdaInductiveConstructor{▹}}\AgdaSpace{}%
\AgdaInductiveConstructor{ar}\AgdaSpace{}%
\AgdaBound{s}\AgdaSymbol{\}}\AgdaSpace{}%
\AgdaBound{f}\AgdaSpace{}%
\AgdaSymbol{=}\AgdaSpace{}%
\AgdaFunction{env-imap}\AgdaSpace{}%
\AgdaSymbol{(}\AgdaBound{f}\AgdaSpace{}%
\AgdaOperator{\AgdaFunction{∘}}\AgdaSpace{}%
\AgdaInductiveConstructor{vₛ}\AgdaSymbol{)}\AgdaSpace{}%
\AgdaOperator{\AgdaInductiveConstructor{,}}\AgdaSpace{}%
\AgdaBound{f}\AgdaSpace{}%
\AgdaInductiveConstructor{v₀}\<%
\\
%
\>[2]\AgdaFunction{env-imap}\AgdaSpace{}%
\AgdaSymbol{\{}\AgdaArgument{Γ}\AgdaSpace{}%
\AgdaSymbol{=}\AgdaSpace{}%
\AgdaBound{Γ}\AgdaSpace{}%
\AgdaOperator{\AgdaInductiveConstructor{▹}}\AgdaSpace{}%
\AgdaInductiveConstructor{ix}\AgdaSpace{}%
\AgdaBound{s}\AgdaSymbol{\}}\AgdaSpace{}%
\AgdaBound{f}\AgdaSpace{}%
\AgdaSymbol{=}\AgdaSpace{}%
\AgdaFunction{env-imap}\AgdaSpace{}%
\AgdaSymbol{(}\AgdaBound{f}\AgdaSpace{}%
\AgdaOperator{\AgdaFunction{∘}}\AgdaSpace{}%
\AgdaInductiveConstructor{vₛ}\AgdaSymbol{)}\<%
\\
%
\\[\AgdaEmptyExtraSkip]%
%
\>[2]\AgdaFunction{env-map}\AgdaSpace{}%
\AgdaSymbol{\{}\AgdaArgument{Γ}\AgdaSpace{}%
\AgdaSymbol{=}\AgdaSpace{}%
\AgdaInductiveConstructor{ε}\AgdaSymbol{\}}\AgdaSpace{}%
\AgdaBound{f}\AgdaSpace{}%
\AgdaBound{ρ}\AgdaSpace{}%
\AgdaSymbol{=}\AgdaSpace{}%
\AgdaInductiveConstructor{tt}\<%
\\
%
\>[2]\AgdaFunction{env-map}\AgdaSpace{}%
\AgdaSymbol{\{}\AgdaArgument{Γ}\AgdaSpace{}%
\AgdaSymbol{=}\AgdaSpace{}%
\AgdaBound{Γ}\AgdaSpace{}%
\AgdaOperator{\AgdaInductiveConstructor{▹}}\AgdaSpace{}%
\AgdaInductiveConstructor{ix}\AgdaSpace{}%
\AgdaBound{s}\AgdaSymbol{\}}\AgdaSpace{}%
\AgdaBound{f}\AgdaSpace{}%
\AgdaBound{ρ}\AgdaSpace{}%
\AgdaSymbol{=}\AgdaSpace{}%
\AgdaFunction{env-map}\AgdaSpace{}%
\AgdaSymbol{\{}\AgdaArgument{Γ}\AgdaSpace{}%
\AgdaSymbol{=}\AgdaSpace{}%
\AgdaBound{Γ}\AgdaSymbol{\}}\AgdaSpace{}%
\AgdaBound{f}\AgdaSpace{}%
\AgdaBound{ρ}\<%
\\
%
\>[2]\AgdaFunction{env-map}\AgdaSpace{}%
\AgdaSymbol{\{}\AgdaArgument{Γ}\AgdaSpace{}%
\AgdaSymbol{=}\AgdaSpace{}%
\AgdaBound{Γ}\AgdaSpace{}%
\AgdaOperator{\AgdaInductiveConstructor{▹}}\AgdaSpace{}%
\AgdaInductiveConstructor{ar}\AgdaSpace{}%
\AgdaBound{s}\AgdaSymbol{\}}\AgdaSpace{}%
\AgdaBound{f}\AgdaSpace{}%
\AgdaSymbol{(}\AgdaBound{ρ}\AgdaSpace{}%
\AgdaOperator{\AgdaInductiveConstructor{,}}\AgdaSpace{}%
\AgdaBound{e}\AgdaSymbol{)}\AgdaSpace{}%
\AgdaSymbol{=}\AgdaSpace{}%
\AgdaFunction{env-map}\AgdaSpace{}%
\AgdaSymbol{\{}\AgdaArgument{Γ}\AgdaSpace{}%
\AgdaSymbol{=}\AgdaSpace{}%
\AgdaBound{Γ}\AgdaSymbol{\}}\AgdaSpace{}%
\AgdaBound{f}\AgdaSpace{}%
\AgdaBound{ρ}\AgdaSpace{}%
\AgdaOperator{\AgdaInductiveConstructor{,}}\AgdaSpace{}%
\AgdaBound{f}\AgdaSpace{}%
\AgdaBound{e}\<%
\\
%
\\[\AgdaEmptyExtraSkip]%
%
\>[2]\AgdaFunction{env-zero}\AgdaSpace{}%
\AgdaSymbol{\{}\AgdaInductiveConstructor{ε}\AgdaSymbol{\}}\AgdaSpace{}%
\AgdaSymbol{=}\AgdaSpace{}%
\AgdaSymbol{\AgdaUnderscore{}}\<%
\\
%
\>[2]\AgdaFunction{env-zero}\AgdaSpace{}%
\AgdaSymbol{\{}\AgdaBound{Γ}\AgdaSpace{}%
\AgdaOperator{\AgdaInductiveConstructor{▹}}\AgdaSpace{}%
\AgdaInductiveConstructor{ix}\AgdaSpace{}%
\AgdaBound{x}\AgdaSymbol{\}}\AgdaSpace{}%
\AgdaSymbol{=}\AgdaSpace{}%
\AgdaFunction{env-zero}\AgdaSpace{}%
\AgdaSymbol{\{}\AgdaBound{Γ}\AgdaSymbol{\}}\<%
\\
%
\>[2]\AgdaFunction{env-zero}\AgdaSpace{}%
\AgdaSymbol{\{}\AgdaBound{Γ}\AgdaSpace{}%
\AgdaOperator{\AgdaInductiveConstructor{▹}}\AgdaSpace{}%
\AgdaInductiveConstructor{ar}\AgdaSpace{}%
\AgdaBound{x}\AgdaSymbol{\}}\AgdaSpace{}%
\AgdaSymbol{=}\AgdaSpace{}%
\AgdaFunction{env-zero}\AgdaSpace{}%
\AgdaSymbol{\{}\AgdaBound{Γ}\AgdaSymbol{\}}\AgdaSpace{}%
\AgdaOperator{\AgdaInductiveConstructor{,}}\AgdaSpace{}%
\AgdaInductiveConstructor{zero}\<%
\\
%
\\[\AgdaEmptyExtraSkip]%
%
\>[2]\AgdaFunction{env-zipWith}\AgdaSpace{}%
\AgdaSymbol{\{}\AgdaInductiveConstructor{ε}\AgdaSymbol{\}}\AgdaSpace{}%
\AgdaBound{f}\AgdaSpace{}%
\AgdaBound{l}\AgdaSpace{}%
\AgdaBound{r}\AgdaSpace{}%
\AgdaSymbol{=}\AgdaSpace{}%
\AgdaSymbol{\AgdaUnderscore{}}\<%
\\
%
\>[2]\AgdaFunction{env-zipWith}\AgdaSpace{}%
\AgdaSymbol{\{}\AgdaBound{Γ}\AgdaSpace{}%
\AgdaOperator{\AgdaInductiveConstructor{▹}}\AgdaSpace{}%
\AgdaInductiveConstructor{ix}\AgdaSpace{}%
\AgdaBound{x}\AgdaSymbol{\}}\AgdaSpace{}%
\AgdaBound{f}\AgdaSpace{}%
\AgdaBound{l}\AgdaSpace{}%
\AgdaBound{r}\AgdaSpace{}%
\AgdaSymbol{=}\AgdaSpace{}%
\AgdaFunction{env-zipWith}\AgdaSpace{}%
\AgdaSymbol{\{}\AgdaBound{Γ}\AgdaSymbol{\}}\AgdaSpace{}%
\AgdaBound{f}\AgdaSpace{}%
\AgdaBound{l}\AgdaSpace{}%
\AgdaBound{r}\<%
\\
%
\>[2]\AgdaFunction{env-zipWith}\AgdaSpace{}%
\AgdaSymbol{\{}\AgdaBound{Γ}\AgdaSpace{}%
\AgdaOperator{\AgdaInductiveConstructor{▹}}\AgdaSpace{}%
\AgdaInductiveConstructor{ar}\AgdaSpace{}%
\AgdaBound{x}\AgdaSymbol{\}}\AgdaSpace{}%
\AgdaBound{f}\AgdaSpace{}%
\AgdaSymbol{(}\AgdaBound{l}\AgdaSpace{}%
\AgdaOperator{\AgdaInductiveConstructor{,}}\AgdaSpace{}%
\AgdaBound{e₁}\AgdaSymbol{)}\AgdaSpace{}%
\AgdaSymbol{(}\AgdaBound{r}\AgdaSpace{}%
\AgdaOperator{\AgdaInductiveConstructor{,}}\AgdaSpace{}%
\AgdaBound{e₂}\AgdaSymbol{)}\AgdaSpace{}%
\AgdaSymbol{=}\AgdaSpace{}%
\AgdaFunction{env-zipWith}\AgdaSpace{}%
\AgdaSymbol{\{}\AgdaBound{Γ}\AgdaSymbol{\}}\AgdaSpace{}%
\AgdaBound{f}\AgdaSpace{}%
\AgdaBound{l}\AgdaSpace{}%
\AgdaBound{r}\AgdaSpace{}%
\AgdaOperator{\AgdaInductiveConstructor{,}}\AgdaSpace{}%
\AgdaBound{f}\AgdaSpace{}%
\AgdaBound{e₁}\AgdaSpace{}%
\AgdaBound{e₂}\<%
\end{code}

We define the function \AF{∇} that takes an expression \AF{E} and the seed
which is the multiplier on the left of the chain, and we compute a function
from that updates the environment.
\begin{code}%
%
\>[2]\AgdaFunction{∇}\AgdaSpace{}%
\AgdaSymbol{:}\AgdaSpace{}%
\AgdaDatatype{E}\AgdaSpace{}%
\AgdaGeneralizable{Δ}\AgdaSpace{}%
\AgdaGeneralizable{is}\AgdaSpace{}%
\AgdaSymbol{→}\AgdaSpace{}%
\AgdaSymbol{(}\AgdaBound{seed}\AgdaSpace{}%
\AgdaSymbol{:}\AgdaSpace{}%
\AgdaDatatype{E}\AgdaSpace{}%
\AgdaGeneralizable{Δ}\AgdaSpace{}%
\AgdaGeneralizable{is}\AgdaSymbol{)}\AgdaSpace{}%
\AgdaSymbol{→}\AgdaSpace{}%
\AgdaFunction{Env}\AgdaSpace{}%
\AgdaGeneralizable{Δ}\AgdaSpace{}%
\AgdaGeneralizable{Δ}\AgdaSpace{}%
\AgdaSymbol{→}\AgdaSpace{}%
\AgdaFunction{Env}\AgdaSpace{}%
\AgdaGeneralizable{Δ}\AgdaSpace{}%
\AgdaGeneralizable{Δ}\<%
\\
%
\\[\AgdaEmptyExtraSkip]%
%
\>[2]\AgdaFunction{map-sum}\AgdaSpace{}%
\AgdaSymbol{:}\AgdaSpace{}%
\AgdaSymbol{(}\AgdaBound{e}\AgdaSpace{}%
\AgdaBound{s}\AgdaSpace{}%
\AgdaSymbol{:}\AgdaSpace{}%
\AgdaDatatype{E}\AgdaSpace{}%
\AgdaSymbol{(}\AgdaGeneralizable{Δ}\AgdaSpace{}%
\AgdaOperator{\AgdaInductiveConstructor{▹}}\AgdaSpace{}%
\AgdaInductiveConstructor{ix}\AgdaSpace{}%
\AgdaGeneralizable{s}\AgdaSymbol{)}\AgdaSpace{}%
\AgdaGeneralizable{ip}\AgdaSymbol{)}\AgdaSpace{}%
\AgdaSymbol{→}\AgdaSpace{}%
\AgdaFunction{Env}\AgdaSpace{}%
\AgdaGeneralizable{Δ}\AgdaSpace{}%
\AgdaGeneralizable{Δ}\AgdaSpace{}%
\AgdaSymbol{→}\AgdaSpace{}%
\AgdaFunction{Env}\AgdaSpace{}%
\AgdaGeneralizable{Δ}\AgdaSpace{}%
\AgdaGeneralizable{Δ}\<%
\\
%
\>[2]\AgdaFunction{map-sum}\AgdaSpace{}%
\AgdaSymbol{\{}\AgdaBound{Δ}\AgdaSymbol{\}}\AgdaSpace{}%
\AgdaBound{e}\AgdaSpace{}%
\AgdaBound{s}\AgdaSpace{}%
\AgdaBound{δ}\AgdaSpace{}%
\AgdaSymbol{=}\AgdaSpace{}%
\AgdaFunction{env-zipWith}\AgdaSpace{}%
\AgdaSymbol{\{}\AgdaBound{Δ}\AgdaSymbol{\}}\AgdaSpace{}%
\AgdaOperator{\AgdaInductiveConstructor{\AgdaUnderscore{}⊞\AgdaUnderscore{}}}\AgdaSpace{}%
\AgdaSymbol{(}\AgdaFunction{env-map}\AgdaSpace{}%
\AgdaSymbol{\{}\AgdaBound{Δ}\AgdaSymbol{\}}\AgdaSpace{}%
\AgdaInductiveConstructor{sum}\AgdaSpace{}%
\AgdaSymbol{(}\AgdaFunction{∇}\AgdaSpace{}%
\AgdaBound{e}\AgdaSpace{}%
\AgdaBound{s}\AgdaSpace{}%
\AgdaSymbol{(}\AgdaFunction{env-zero}\AgdaSpace{}%
\AgdaSymbol{\{}\AgdaBound{Δ}\AgdaSymbol{\})))}\AgdaSpace{}%
\AgdaBound{δ}\<%
\\
%
\\[\AgdaEmptyExtraSkip]%
%
\>[2]\AgdaFunction{∇}\AgdaSpace{}%
\AgdaSymbol{(}\AgdaInductiveConstructor{zero}\AgdaSymbol{)}%
\>[27]\AgdaBound{s}\AgdaSpace{}%
\AgdaBound{δ}\AgdaSpace{}%
\AgdaSymbol{=}\AgdaSpace{}%
\AgdaBound{δ}\<%
\\
%
\>[2]\AgdaFunction{∇}\AgdaSpace{}%
\AgdaSymbol{(}\AgdaInductiveConstructor{one}\AgdaSymbol{)}%
\>[27]\AgdaBound{s}\AgdaSpace{}%
\AgdaBound{δ}\AgdaSpace{}%
\AgdaSymbol{=}\AgdaSpace{}%
\AgdaBound{δ}\<%
\\
%
\>[2]\AgdaFunction{∇}\AgdaSpace{}%
\AgdaSymbol{(}\AgdaInductiveConstructor{var}\AgdaSpace{}%
\AgdaSymbol{\{}\AgdaInductiveConstructor{ix}\AgdaSpace{}%
\AgdaSymbol{\AgdaUnderscore{}\}}\AgdaSpace{}%
\AgdaBound{x}\AgdaSymbol{)}%
\>[27]\AgdaBound{s}\AgdaSpace{}%
\AgdaBound{δ}\AgdaSpace{}%
\AgdaSymbol{=}\AgdaSpace{}%
\AgdaBound{δ}\<%
\\
%
\>[2]\AgdaFunction{∇}\AgdaSpace{}%
\AgdaSymbol{(}\AgdaInductiveConstructor{var}\AgdaSpace{}%
\AgdaSymbol{\{}\AgdaInductiveConstructor{ar}\AgdaSpace{}%
\AgdaSymbol{\AgdaUnderscore{}\}}\AgdaSpace{}%
\AgdaBound{x}\AgdaSymbol{)}%
\>[27]\AgdaBound{s}\AgdaSpace{}%
\AgdaBound{δ}\AgdaSpace{}%
\AgdaSymbol{=}\AgdaSpace{}%
\AgdaFunction{update}\AgdaSpace{}%
\AgdaBound{δ}\AgdaSpace{}%
\AgdaBound{x}\AgdaSpace{}%
\AgdaSymbol{(}\AgdaOperator{\AgdaInductiveConstructor{\AgdaUnderscore{}⊞}}\AgdaSpace{}%
\AgdaBound{s}\AgdaSymbol{)}\<%
\\
%
\\[\AgdaEmptyExtraSkip]%
%
\>[2]\AgdaFunction{∇}\AgdaSpace{}%
\AgdaSymbol{(}\AgdaInductiveConstructor{imapₛ}\AgdaSpace{}%
\AgdaBound{e}\AgdaSymbol{)}%
\>[27]\AgdaBound{s}%
\>[31]\AgdaSymbol{=}\AgdaSpace{}%
\AgdaFunction{map-sum}\AgdaSpace{}%
\AgdaBound{e}\AgdaSpace{}%
\AgdaSymbol{(}\AgdaInductiveConstructor{selₛ}%
\>[52]\AgdaSymbol{(}\AgdaOperator{\AgdaFunction{↑}}\AgdaSpace{}%
\AgdaBound{s}\AgdaSymbol{)}\AgdaSpace{}%
\AgdaSymbol{(}\AgdaInductiveConstructor{var}\AgdaSpace{}%
\AgdaInductiveConstructor{v₀}\AgdaSymbol{))}\<%
\\
%
\>[2]\AgdaFunction{∇}\AgdaSpace{}%
\AgdaSymbol{(}\AgdaInductiveConstructor{imap}\AgdaSpace{}%
\AgdaBound{e}\AgdaSymbol{)}%
\>[27]\AgdaBound{s}%
\>[31]\AgdaSymbol{=}\AgdaSpace{}%
\AgdaFunction{map-sum}\AgdaSpace{}%
\AgdaBound{e}\AgdaSpace{}%
\AgdaSymbol{(}\AgdaInductiveConstructor{sel}%
\>[52]\AgdaSymbol{(}\AgdaOperator{\AgdaFunction{↑}}\AgdaSpace{}%
\AgdaBound{s}\AgdaSymbol{)}\AgdaSpace{}%
\AgdaSymbol{(}\AgdaInductiveConstructor{var}\AgdaSpace{}%
\AgdaInductiveConstructor{v₀}\AgdaSymbol{))}\<%
\\
%
\>[2]\AgdaFunction{∇}\AgdaSpace{}%
\AgdaSymbol{(}\AgdaInductiveConstructor{imapb}\AgdaSpace{}%
\AgdaBound{m}\AgdaSpace{}%
\AgdaBound{e}\AgdaSymbol{)}%
\>[27]\AgdaBound{s}%
\>[31]\AgdaSymbol{=}\AgdaSpace{}%
\AgdaFunction{map-sum}\AgdaSpace{}%
\AgdaBound{e}\AgdaSpace{}%
\AgdaSymbol{(}\AgdaInductiveConstructor{selb}\AgdaSpace{}%
\AgdaBound{m}%
\>[52]\AgdaSymbol{(}\AgdaOperator{\AgdaFunction{↑}}\AgdaSpace{}%
\AgdaBound{s}\AgdaSymbol{)}\AgdaSpace{}%
\AgdaSymbol{(}\AgdaInductiveConstructor{var}\AgdaSpace{}%
\AgdaInductiveConstructor{v₀}\AgdaSymbol{))}\<%
\\
%
\\[\AgdaEmptyExtraSkip]%
%
\>[2]\AgdaFunction{∇}\AgdaSpace{}%
\AgdaSymbol{(}\AgdaInductiveConstructor{selₛ}\AgdaSpace{}%
\AgdaBound{e}\AgdaSpace{}%
\AgdaBound{i}\AgdaSymbol{)}%
\>[27]\AgdaBound{s}%
\>[31]\AgdaSymbol{=}\AgdaSpace{}%
\AgdaFunction{∇}\AgdaSpace{}%
\AgdaBound{e}\AgdaSpace{}%
\AgdaSymbol{(}\AgdaInductiveConstructor{imapₛ}%
\>[47]\AgdaSymbol{(}\AgdaInductiveConstructor{zero-but}\AgdaSpace{}%
\AgdaSymbol{(}\AgdaInductiveConstructor{var}\AgdaSpace{}%
\AgdaInductiveConstructor{v₀}\AgdaSymbol{)}\AgdaSpace{}%
\AgdaSymbol{(}\AgdaOperator{\AgdaFunction{↑}}\AgdaSpace{}%
\AgdaBound{i}\AgdaSymbol{)}\AgdaSpace{}%
\AgdaSymbol{(}\AgdaOperator{\AgdaFunction{↑}}\AgdaSpace{}%
\AgdaBound{s}\AgdaSymbol{)))}\<%
\\
%
\>[2]\AgdaFunction{∇}\AgdaSpace{}%
\AgdaSymbol{(}\AgdaInductiveConstructor{sel}\AgdaSpace{}%
\AgdaBound{e}\AgdaSpace{}%
\AgdaBound{i}\AgdaSymbol{)}%
\>[27]\AgdaBound{s}%
\>[31]\AgdaSymbol{=}\AgdaSpace{}%
\AgdaFunction{∇}\AgdaSpace{}%
\AgdaBound{e}\AgdaSpace{}%
\AgdaSymbol{(}\AgdaInductiveConstructor{imap}%
\>[47]\AgdaSymbol{(}\AgdaInductiveConstructor{zero-but}\AgdaSpace{}%
\AgdaSymbol{(}\AgdaInductiveConstructor{var}\AgdaSpace{}%
\AgdaInductiveConstructor{v₀}\AgdaSymbol{)}\AgdaSpace{}%
\AgdaSymbol{(}\AgdaOperator{\AgdaFunction{↑}}\AgdaSpace{}%
\AgdaBound{i}\AgdaSymbol{)}\AgdaSpace{}%
\AgdaSymbol{(}\AgdaOperator{\AgdaFunction{↑}}\AgdaSpace{}%
\AgdaBound{s}\AgdaSymbol{)))}\<%
\\
%
\>[2]\AgdaFunction{∇}\AgdaSpace{}%
\AgdaSymbol{(}\AgdaInductiveConstructor{selb}\AgdaSpace{}%
\AgdaBound{m}\AgdaSpace{}%
\AgdaBound{e}\AgdaSpace{}%
\AgdaBound{i}\AgdaSymbol{)}%
\>[27]\AgdaBound{s}%
\>[31]\AgdaSymbol{=}\AgdaSpace{}%
\AgdaFunction{∇}\AgdaSpace{}%
\AgdaBound{e}\AgdaSpace{}%
\AgdaSymbol{(}\AgdaInductiveConstructor{imapb}\AgdaSpace{}%
\AgdaBound{m}%
\>[47]\AgdaSymbol{(}\AgdaInductiveConstructor{zero-but}\AgdaSpace{}%
\AgdaSymbol{(}\AgdaInductiveConstructor{var}\AgdaSpace{}%
\AgdaInductiveConstructor{v₀}\AgdaSymbol{)}\AgdaSpace{}%
\AgdaSymbol{(}\AgdaOperator{\AgdaFunction{↑}}\AgdaSpace{}%
\AgdaBound{i}\AgdaSymbol{)}\AgdaSpace{}%
\AgdaSymbol{(}\AgdaOperator{\AgdaFunction{↑}}\AgdaSpace{}%
\AgdaBound{s}\AgdaSymbol{)))}\<%
\\
%
\\[\AgdaEmptyExtraSkip]%
%
\>[2]\AgdaFunction{∇}\AgdaSpace{}%
\AgdaSymbol{(}\AgdaInductiveConstructor{zero-but}\AgdaSpace{}%
\AgdaBound{i}\AgdaSpace{}%
\AgdaBound{j}\AgdaSpace{}%
\AgdaBound{e}\AgdaSymbol{)}%
\>[27]\AgdaBound{s}%
\>[31]\AgdaSymbol{=}\AgdaSpace{}%
\AgdaFunction{∇}\AgdaSpace{}%
\AgdaBound{e}\AgdaSpace{}%
\AgdaSymbol{(}\AgdaInductiveConstructor{zero-but}\AgdaSpace{}%
\AgdaBound{i}\AgdaSpace{}%
\AgdaBound{j}\AgdaSpace{}%
\AgdaBound{s}\AgdaSymbol{)}\<%
\\
%
\>[2]\AgdaFunction{∇}\AgdaSpace{}%
\AgdaSymbol{(}\AgdaInductiveConstructor{sum}\AgdaSpace{}%
\AgdaBound{e}\AgdaSymbol{)}%
\>[27]\AgdaBound{s}%
\>[31]\AgdaSymbol{=}\AgdaSpace{}%
\AgdaFunction{map-sum}\AgdaSpace{}%
\AgdaBound{e}\AgdaSpace{}%
\AgdaSymbol{(}\AgdaOperator{\AgdaFunction{↑}}\AgdaSpace{}%
\AgdaBound{s}\AgdaSymbol{)}\<%
\\
%
\\[\AgdaEmptyExtraSkip]%
%
\>[2]\AgdaFunction{∇}\AgdaSpace{}%
\AgdaSymbol{(}\AgdaBound{e}\AgdaSpace{}%
\AgdaOperator{\AgdaInductiveConstructor{⊞}}\AgdaSpace{}%
\AgdaBound{e₁}\AgdaSymbol{)}%
\>[27]\AgdaBound{s}%
\>[31]\AgdaSymbol{=}\AgdaSpace{}%
\AgdaFunction{∇}\AgdaSpace{}%
\AgdaBound{e}\AgdaSpace{}%
\AgdaBound{s}\AgdaSpace{}%
\AgdaOperator{\AgdaFunction{∘}}\AgdaSpace{}%
\AgdaFunction{∇}\AgdaSpace{}%
\AgdaBound{e₁}\AgdaSpace{}%
\AgdaBound{s}\<%
\\
%
\>[2]\AgdaFunction{∇}\AgdaSpace{}%
\AgdaSymbol{(}\AgdaBound{e}\AgdaSpace{}%
\AgdaOperator{\AgdaInductiveConstructor{⊠}}\AgdaSpace{}%
\AgdaBound{e₁}\AgdaSymbol{)}%
\>[27]\AgdaBound{s}%
\>[31]\AgdaSymbol{=}\AgdaSpace{}%
\AgdaFunction{∇}\AgdaSpace{}%
\AgdaBound{e}\AgdaSpace{}%
\AgdaSymbol{(}\AgdaBound{s}\AgdaSpace{}%
\AgdaOperator{\AgdaInductiveConstructor{⊠}}\AgdaSpace{}%
\AgdaBound{e₁}\AgdaSymbol{)}\AgdaSpace{}%
\AgdaOperator{\AgdaFunction{∘}}\AgdaSpace{}%
\AgdaFunction{∇}\AgdaSpace{}%
\AgdaBound{e₁}\AgdaSpace{}%
\AgdaSymbol{(}\AgdaBound{s}\AgdaSpace{}%
\AgdaOperator{\AgdaInductiveConstructor{⊠}}\AgdaSpace{}%
\AgdaBound{e}\AgdaSymbol{)}\<%
\\
%
\>[2]\AgdaFunction{∇}\AgdaSpace{}%
\AgdaSymbol{(}\AgdaInductiveConstructor{slide}\AgdaSpace{}%
\AgdaBound{i}\AgdaSpace{}%
\AgdaBound{pl}\AgdaSpace{}%
\AgdaBound{e}\AgdaSpace{}%
\AgdaBound{su}\AgdaSymbol{)}%
\>[27]\AgdaBound{s}%
\>[31]\AgdaSymbol{=}\AgdaSpace{}%
\AgdaFunction{∇}\AgdaSpace{}%
\AgdaBound{e}\AgdaSpace{}%
\AgdaSymbol{(}\AgdaInductiveConstructor{backslide}\AgdaSpace{}%
\AgdaBound{i}\AgdaSpace{}%
\AgdaBound{s}\AgdaSpace{}%
\AgdaBound{su}\AgdaSpace{}%
\AgdaBound{pl}\AgdaSymbol{)}\<%
\\
%
\>[2]\AgdaFunction{∇}\AgdaSpace{}%
\AgdaSymbol{(}\AgdaInductiveConstructor{backslide}\AgdaSpace{}%
\AgdaBound{i}\AgdaSpace{}%
\AgdaBound{e}\AgdaSpace{}%
\AgdaBound{su}\AgdaSpace{}%
\AgdaBound{pl}\AgdaSymbol{)}%
\>[27]\AgdaBound{s}%
\>[31]\AgdaSymbol{=}\AgdaSpace{}%
\AgdaFunction{∇}\AgdaSpace{}%
\AgdaBound{e}\AgdaSpace{}%
\AgdaSymbol{(}\AgdaInductiveConstructor{slide}\AgdaSpace{}%
\AgdaBound{i}\AgdaSpace{}%
\AgdaBound{pl}\AgdaSpace{}%
\AgdaBound{s}\AgdaSpace{}%
\AgdaBound{su}\AgdaSymbol{)}\<%
\\
%
\\[\AgdaEmptyExtraSkip]%
%
\>[2]\AgdaFunction{∇}\AgdaSpace{}%
\AgdaSymbol{(}\AgdaInductiveConstructor{scaledown}\AgdaSpace{}%
\AgdaBound{x}\AgdaSpace{}%
\AgdaBound{e}\AgdaSymbol{)}%
\>[27]\AgdaBound{s}%
\>[31]\AgdaSymbol{=}\AgdaSpace{}%
\AgdaFunction{∇}\AgdaSpace{}%
\AgdaBound{e}\AgdaSpace{}%
\AgdaSymbol{(}\AgdaInductiveConstructor{scaledown}\AgdaSpace{}%
\AgdaBound{x}\AgdaSpace{}%
\AgdaBound{s}\AgdaSymbol{)}\<%
\\
%
\>[2]\AgdaFunction{∇}\AgdaSpace{}%
\AgdaSymbol{(}\AgdaInductiveConstructor{minus}\AgdaSpace{}%
\AgdaBound{e}\AgdaSymbol{)}%
\>[27]\AgdaBound{s}%
\>[31]\AgdaSymbol{=}\AgdaSpace{}%
\AgdaFunction{∇}\AgdaSpace{}%
\AgdaBound{e}\AgdaSpace{}%
\AgdaSymbol{(}\AgdaInductiveConstructor{minus}\AgdaSpace{}%
\AgdaBound{s}\AgdaSymbol{)}\<%
\\
%
\>[2]\AgdaFunction{∇}\AgdaSpace{}%
\AgdaSymbol{(}\AgdaInductiveConstructor{logistic}\AgdaSpace{}%
\AgdaBound{e}\AgdaSymbol{)}%
\>[27]\AgdaBound{s}%
\>[31]\AgdaSymbol{=}\AgdaSpace{}%
\AgdaFunction{∇}\AgdaSpace{}%
\AgdaBound{e}\AgdaSpace{}%
\AgdaSymbol{(}\AgdaBound{s}\AgdaSpace{}%
\AgdaOperator{\AgdaInductiveConstructor{⊠}}\AgdaSpace{}%
\AgdaInductiveConstructor{logistic}\AgdaSpace{}%
\AgdaBound{e}\AgdaSpace{}%
\AgdaOperator{\AgdaInductiveConstructor{⊠}}\AgdaSpace{}%
\AgdaSymbol{(}\AgdaInductiveConstructor{one}\AgdaSpace{}%
\AgdaOperator{\AgdaInductiveConstructor{⊞}}\AgdaSpace{}%
\AgdaInductiveConstructor{minus}\AgdaSpace{}%
\AgdaSymbol{(}\AgdaInductiveConstructor{logistic}\AgdaSpace{}%
\AgdaBound{e}\AgdaSymbol{)))}\<%
\end{code}
Let us now walk through the cases.  Derivative of constants (\AC{zero} and \AC{one})
is zero, so nothing needs to be updated in the environment.  Index variables are
not stored in the environment, so no updates are needed either.  If we differentiate
the variable $x$ with some seed \AB{s}, we update the $x$-th position in the environment
by adding \AB{s} to it.  Differentiation of \AC{imap}s proceeds as follows: we
recursively apply \AF{∇} to $e$ (in the context \AB{Γ} \AC{▹} (\AC{ix} \AB{p}))
with the element of the original seed \AB{s} selected at the top variable.  This
gives us the environment in the extended context, then we map \AC{sum} to every
element of the environment to accumulate the derivatives at every index.
When differentiating selections we recurse on the array we are
selecting from with the seed that is zero everywhere except the index we were
selecting at.  Differentiating
conditional is straight-forward, as $i$ and $j$ must be in the context, we can
simply differentiate $e$ with the condition on seed.  If indices were equal, we will
compute the update, otherwise we will differentiate with seed \AC{zero} which
has no effect.  As we are operating in a total language, there is no need to worry
about pulling the expression out of conditional.  The argument of \AC{sum}
lives is in the extended context, so we apply the same rules as for the \AC{imap} family,
except we propagate the original seed to all the summands.  Addition and multiplication
rules are straight-forward application of differentiation rules.  The \AC{slide}/\AC{backslide}
pair forms a satisfying \AF{∇}-symmetry.  Finally, \AC{scaledown}, \AC{minus} and
\AC{logistic} follow the rules of differentiation.





\subsection{Optimisation\label{sec:opt}}

Our algorithm often delivers expressions that are not computationally efficient.
While we can hope for the backend to take care of this, it is relatively
easy to implement a number of rewriting rules prior to extraction.  
We constructed \AF{E} such that no computation is happening in the shape
or context positions.  As a result, dependent pattern-matching is always
applicable on \AF{E} expressions, and our optimisations can be formulated
very concisely.  We omit constant-folding like rewrites such as addition
with zero and multiplication by one and focus on less trivial cases that have
to do with selections a+nd sum.  Consider the snippet of the optimiser for
\AF{selₛ} and \AF{sum}.

\begin{code}[hide]%
\>[0]\AgdaKeyword{module}\AgdaSpace{}%
\AgdaModule{Opt}\AgdaSpace{}%
\AgdaKeyword{where}\<%
\\
\>[0][@{}l@{\AgdaIndent{0}}]%
\>[2]\AgdaKeyword{open}\AgdaSpace{}%
\AgdaKeyword{import}\AgdaSpace{}%
\AgdaModule{Data.Nat}\AgdaSpace{}%
\AgdaSymbol{as}\AgdaSpace{}%
\AgdaModule{ℕ}\AgdaSpace{}%
\AgdaKeyword{using}\AgdaSpace{}%
\AgdaSymbol{(}\AgdaDatatype{ℕ}\AgdaSymbol{;}\AgdaSpace{}%
\AgdaInductiveConstructor{zero}\AgdaSymbol{;}\AgdaSpace{}%
\AgdaInductiveConstructor{suc}\AgdaSymbol{)}\<%
\\
%
\>[2]\AgdaKeyword{open}\AgdaSpace{}%
\AgdaModule{Lang}\<%
\\
%
\>[2]\AgdaKeyword{open}\AgdaSpace{}%
\AgdaModule{SubWk}\<%
\\
%
\>[2]\AgdaComment{--open\ Eval\ using\ (sub;\ ctx-swap;\ ↑\AgdaUnderscore{};\ ↑↑\AgdaUnderscore{};\ eq?)}\<%
\\
%
\>[2]\AgdaKeyword{open}\AgdaSpace{}%
\AgdaModule{Array}\AgdaSpace{}%
\AgdaKeyword{hiding}\AgdaSpace{}%
\AgdaSymbol{(}\AgdaFunction{sum}\AgdaSymbol{;}\AgdaSpace{}%
\AgdaFunction{slide}\AgdaSymbol{;}\AgdaFunction{backslide}\AgdaSymbol{)}\<%
\\
%
\>[2]\AgdaKeyword{open}\AgdaSpace{}%
\AgdaModule{BB}\<%
\\
%
\>[2]\AgdaKeyword{open}\AgdaSpace{}%
\AgdaModule{AD}\<%
\end{code}
\begin{code}%
%
\>[2]\AgdaFunction{opt}\AgdaSpace{}%
\AgdaSymbol{:}\AgdaSpace{}%
\AgdaDatatype{E}\AgdaSpace{}%
\AgdaGeneralizable{Γ}\AgdaSpace{}%
\AgdaGeneralizable{is}\AgdaSpace{}%
\AgdaSymbol{→}\AgdaSpace{}%
\AgdaDatatype{E}\AgdaSpace{}%
\AgdaGeneralizable{Γ}\AgdaSpace{}%
\AgdaGeneralizable{is}\<%
\\
%
\>[2]\AgdaFunction{opt}\AgdaSpace{}%
\AgdaSymbol{(}\AgdaInductiveConstructor{selₛ}\AgdaSpace{}%
\AgdaBound{e}\AgdaSpace{}%
\AgdaBound{e₁}\AgdaSymbol{)}\AgdaSpace{}%
\AgdaKeyword{with}\AgdaSpace{}%
\AgdaFunction{opt}\AgdaSpace{}%
\AgdaBound{e}\AgdaSpace{}%
\AgdaSymbol{|}\AgdaSpace{}%
\AgdaFunction{opt}\AgdaSpace{}%
\AgdaBound{e₁}\<%
\\
%
\>[2]\AgdaSymbol{...}\AgdaSpace{}%
\AgdaSymbol{|}\AgdaSpace{}%
\AgdaInductiveConstructor{zero}%
\>[24]\AgdaSymbol{|}\AgdaSpace{}%
\AgdaBound{i}\AgdaSpace{}%
\AgdaSymbol{=}\AgdaSpace{}%
\AgdaInductiveConstructor{zero}\<%
\\
%
\>[2]\AgdaSymbol{...}\AgdaSpace{}%
\AgdaSymbol{|}\AgdaSpace{}%
\AgdaInductiveConstructor{one}%
\>[24]\AgdaSymbol{|}\AgdaSpace{}%
\AgdaBound{i}\AgdaSpace{}%
\AgdaSymbol{=}\AgdaSpace{}%
\AgdaInductiveConstructor{one}\<%
\\
%
\>[2]\AgdaSymbol{...}\AgdaSpace{}%
\AgdaSymbol{|}\AgdaSpace{}%
\AgdaInductiveConstructor{imapₛ}\AgdaSpace{}%
\AgdaBound{e}%
\>[24]\AgdaSymbol{|}\AgdaSpace{}%
\AgdaBound{i}\AgdaSpace{}%
\AgdaSymbol{=}\AgdaSpace{}%
\AgdaFunction{sub}\AgdaSpace{}%
\AgdaInductiveConstructor{v₀}\AgdaSpace{}%
\AgdaBound{e}\AgdaSpace{}%
\AgdaBound{i}\<%
\\
%
\>[2]\AgdaSymbol{...}\AgdaSpace{}%
\AgdaSymbol{|}\AgdaSpace{}%
\AgdaInductiveConstructor{bin}\AgdaSpace{}%
\AgdaBound{op}\AgdaSpace{}%
\AgdaBound{a}\AgdaSpace{}%
\AgdaBound{b}%
\>[24]\AgdaSymbol{|}\AgdaSpace{}%
\AgdaBound{i}\AgdaSpace{}%
\AgdaSymbol{=}\AgdaSpace{}%
\AgdaInductiveConstructor{bin}\AgdaSpace{}%
\AgdaBound{op}\AgdaSpace{}%
\AgdaSymbol{(}\AgdaInductiveConstructor{selₛ}\AgdaSpace{}%
\AgdaBound{a}\AgdaSpace{}%
\AgdaBound{i}\AgdaSymbol{)}\AgdaSpace{}%
\AgdaSymbol{(}\AgdaInductiveConstructor{selₛ}\AgdaSpace{}%
\AgdaBound{b}\AgdaSpace{}%
\AgdaBound{i}\AgdaSymbol{)}\<%
\\
%
\>[2]\AgdaSymbol{...}\AgdaSpace{}%
\AgdaSymbol{|}\AgdaSpace{}%
\AgdaInductiveConstructor{sum}\AgdaSpace{}%
\AgdaBound{e}%
\>[24]\AgdaSymbol{|}\AgdaSpace{}%
\AgdaBound{i}\AgdaSpace{}%
\AgdaSymbol{=}\AgdaSpace{}%
\AgdaInductiveConstructor{sum}\AgdaSpace{}%
\AgdaSymbol{(}\AgdaInductiveConstructor{selₛ}\AgdaSpace{}%
\AgdaBound{e}\AgdaSpace{}%
\AgdaSymbol{(}\AgdaOperator{\AgdaFunction{↑}}\AgdaSpace{}%
\AgdaBound{i}\AgdaSymbol{))}\<%
\\
%
\>[2]\AgdaSymbol{...}\AgdaSpace{}%
\AgdaSymbol{|}\AgdaSpace{}%
\AgdaInductiveConstructor{zero-but}\AgdaSpace{}%
\AgdaBound{i}\AgdaSpace{}%
\AgdaBound{j}\AgdaSpace{}%
\AgdaBound{a}%
\>[24]\AgdaSymbol{|}\AgdaSpace{}%
\AgdaBound{k}\AgdaSpace{}%
\AgdaSymbol{=}\AgdaSpace{}%
\AgdaInductiveConstructor{zero-but}\AgdaSpace{}%
\AgdaBound{i}\AgdaSpace{}%
\AgdaBound{j}\AgdaSpace{}%
\AgdaSymbol{(}\AgdaInductiveConstructor{selₛ}\AgdaSpace{}%
\AgdaBound{a}\AgdaSpace{}%
\AgdaBound{k}\AgdaSymbol{)}\<%
\\
%
\>[2]\AgdaCatchallClause{\AgdaSymbol{...}}\AgdaSpace{}%
\AgdaCatchallClause{\AgdaSymbol{|}}\AgdaSpace{}%
\AgdaCatchallClause{\AgdaBound{a}}%
\>[24]\AgdaCatchallClause{\AgdaSymbol{|}}\AgdaSpace{}%
\AgdaCatchallClause{\AgdaBound{i}}\AgdaSpace{}%
\AgdaSymbol{=}\AgdaSpace{}%
\AgdaInductiveConstructor{selₛ}\AgdaSpace{}%
\AgdaBound{a}\AgdaSpace{}%
\AgdaBound{i}\<%
\\
%
\\[\AgdaEmptyExtraSkip]%
%
\>[2]\AgdaFunction{opt}\AgdaSpace{}%
\AgdaSymbol{(}\AgdaInductiveConstructor{sum}\AgdaSpace{}%
\AgdaBound{e}\AgdaSymbol{)}\AgdaSpace{}%
\AgdaKeyword{with}\AgdaSpace{}%
\AgdaFunction{opt}\AgdaSpace{}%
\AgdaBound{e}\<%
\\
%
\>[2]\AgdaSymbol{...}\AgdaSpace{}%
\AgdaSymbol{|}\AgdaSpace{}%
\AgdaInductiveConstructor{zero}%
\>[24]\AgdaSymbol{=}\AgdaSpace{}%
\AgdaInductiveConstructor{zero}\<%
\\
%
\>[2]\AgdaSymbol{...}\AgdaSpace{}%
\AgdaSymbol{|}\AgdaSpace{}%
\AgdaInductiveConstructor{imap}\AgdaSpace{}%
\AgdaBound{a}%
\>[24]\AgdaSymbol{=}\AgdaSpace{}%
\AgdaInductiveConstructor{imap}%
\>[35]\AgdaSymbol{(}\AgdaInductiveConstructor{sum}\AgdaSpace{}%
\AgdaSymbol{(}\AgdaFunction{ctx-swap}\AgdaSpace{}%
\AgdaInductiveConstructor{v₁}\AgdaSpace{}%
\AgdaBound{a}\AgdaSymbol{))}\<%
\\
%
\>[2]\AgdaSymbol{...}\AgdaSpace{}%
\AgdaSymbol{|}\AgdaSpace{}%
\AgdaInductiveConstructor{imapₛ}\AgdaSpace{}%
\AgdaBound{a}%
\>[24]\AgdaSymbol{=}\AgdaSpace{}%
\AgdaInductiveConstructor{imapₛ}%
\>[35]\AgdaSymbol{(}\AgdaInductiveConstructor{sum}\AgdaSpace{}%
\AgdaSymbol{(}\AgdaFunction{ctx-swap}\AgdaSpace{}%
\AgdaInductiveConstructor{v₁}\AgdaSpace{}%
\AgdaBound{a}\AgdaSymbol{))}\<%
\\
%
\>[2]\AgdaSymbol{...}\AgdaSpace{}%
\AgdaSymbol{|}\AgdaSpace{}%
\AgdaInductiveConstructor{imapb}\AgdaSpace{}%
\AgdaBound{m}\AgdaSpace{}%
\AgdaBound{a}%
\>[24]\AgdaSymbol{=}\AgdaSpace{}%
\AgdaInductiveConstructor{imapb}\AgdaSpace{}%
\AgdaBound{m}%
\>[35]\AgdaSymbol{(}\AgdaInductiveConstructor{sum}\AgdaSpace{}%
\AgdaSymbol{(}\AgdaFunction{ctx-swap}\AgdaSpace{}%
\AgdaInductiveConstructor{v₁}\AgdaSpace{}%
\AgdaBound{a}\AgdaSymbol{))}\<%
\\
%
\>[2]\AgdaSymbol{...}\AgdaSpace{}%
\AgdaSymbol{|}\AgdaSpace{}%
\AgdaInductiveConstructor{zero-but}\AgdaSpace{}%
\AgdaSymbol{(}\AgdaInductiveConstructor{var}\AgdaSpace{}%
\AgdaBound{i}\AgdaSymbol{)}\AgdaSpace{}%
\AgdaSymbol{(}\AgdaInductiveConstructor{var}\AgdaSpace{}%
\AgdaBound{j}\AgdaSymbol{)}\AgdaSpace{}%
\AgdaBound{a}\AgdaSpace{}%
\AgdaKeyword{with}\AgdaSpace{}%
\AgdaFunction{eq?}\AgdaSpace{}%
\AgdaInductiveConstructor{v₀}\AgdaSpace{}%
\AgdaBound{i}\AgdaSpace{}%
\AgdaSymbol{|}\AgdaSpace{}%
\AgdaFunction{eq?}\AgdaSpace{}%
\AgdaInductiveConstructor{v₀}\AgdaSpace{}%
\AgdaBound{j}\<%
\\
%
\>[2]\AgdaSymbol{...}\AgdaSpace{}%
\AgdaSymbol{|}\AgdaSpace{}%
\AgdaInductiveConstructor{eq}%
\>[18]\AgdaSymbol{|}\AgdaSpace{}%
\AgdaInductiveConstructor{eq}%
\>[30]\AgdaSymbol{=}\AgdaSpace{}%
\AgdaInductiveConstructor{sum}\AgdaSpace{}%
\AgdaBound{a}\<%
\\
%
\>[2]\AgdaSymbol{...}\AgdaSpace{}%
\AgdaSymbol{|}\AgdaSpace{}%
\AgdaInductiveConstructor{neq}\AgdaSpace{}%
\AgdaSymbol{\AgdaUnderscore{}}\AgdaSpace{}%
\AgdaBound{i′}%
\>[18]\AgdaSymbol{|}\AgdaSpace{}%
\AgdaInductiveConstructor{eq}%
\>[30]\AgdaSymbol{=}\AgdaSpace{}%
\AgdaFunction{sub}\AgdaSpace{}%
\AgdaInductiveConstructor{v₀}\AgdaSpace{}%
\AgdaBound{a}\AgdaSpace{}%
\AgdaSymbol{(}\AgdaInductiveConstructor{var}\AgdaSpace{}%
\AgdaBound{i′}\AgdaSymbol{)}\<%
\\
%
\>[2]\AgdaSymbol{...}\AgdaSpace{}%
\AgdaSymbol{|}\AgdaSpace{}%
\AgdaInductiveConstructor{eq}%
\>[18]\AgdaSymbol{|}\AgdaSpace{}%
\AgdaInductiveConstructor{neq}\AgdaSpace{}%
\AgdaSymbol{\AgdaUnderscore{}}\AgdaSpace{}%
\AgdaBound{j′}%
\>[30]\AgdaSymbol{=}\AgdaSpace{}%
\AgdaFunction{sub}\AgdaSpace{}%
\AgdaInductiveConstructor{v₀}\AgdaSpace{}%
\AgdaBound{a}\AgdaSpace{}%
\AgdaSymbol{(}\AgdaInductiveConstructor{var}\AgdaSpace{}%
\AgdaBound{j′}\AgdaSymbol{)}\<%
\\
%
\>[2]\AgdaSymbol{...}\AgdaSpace{}%
\AgdaSymbol{|}\AgdaSpace{}%
\AgdaInductiveConstructor{neq}\AgdaSpace{}%
\AgdaSymbol{\AgdaUnderscore{}}\AgdaSpace{}%
\AgdaBound{i′}%
\>[18]\AgdaSymbol{|}\AgdaSpace{}%
\AgdaInductiveConstructor{neq}\AgdaSpace{}%
\AgdaSymbol{\AgdaUnderscore{}}\AgdaSpace{}%
\AgdaBound{j′}%
\>[30]\AgdaSymbol{=}\AgdaSpace{}%
\AgdaInductiveConstructor{zero-but}\AgdaSpace{}%
\AgdaSymbol{(}\AgdaInductiveConstructor{var}\AgdaSpace{}%
\AgdaBound{i′}\AgdaSymbol{)}\AgdaSpace{}%
\AgdaSymbol{(}\AgdaInductiveConstructor{var}\AgdaSpace{}%
\AgdaBound{j′}\AgdaSymbol{)}\AgdaSpace{}%
\AgdaSymbol{(}\AgdaInductiveConstructor{sum}\AgdaSpace{}%
\AgdaBound{a}\AgdaSymbol{)}\<%
\\
%
\>[2]\AgdaCatchallClause{\AgdaFunction{opt}}\AgdaSpace{}%
\AgdaCatchallClause{\AgdaSymbol{(}}\AgdaCatchallClause{\AgdaInductiveConstructor{sum}}\AgdaSpace{}%
\AgdaCatchallClause{\AgdaBound{e}}\AgdaCatchallClause{\AgdaSymbol{)}}\AgdaSpace{}%
\AgdaCatchallClause{\AgdaSymbol{|}}\AgdaSpace{}%
\AgdaCatchallClause{\AgdaBound{a}}\AgdaSpace{}%
\AgdaSymbol{=}\AgdaSpace{}%
\AgdaInductiveConstructor{sum}\AgdaSpace{}%
\AgdaBound{a}\<%
\\
%
\>[2]\AgdaComment{--\ ⋯}\<%
\end{code}
Selection into \AC{zero} and \AC{one} is \AF{zero} and \AC{one}, as our constants
are shape-polymorphic.  Selection into an \AF{imapₛ} is evaluation of the \AC{imapₛ}
body at the given index (this is an array version of the $\beta$-rule).  Selection
from the binary operation is a binary operation of selections.  Selection into \AC{sum}
is the \AC{sum} of selections.  Selection into conditional is the same as conditional
over selection.  Summing \AC{zero} is \AC{zero}.  Summing $s$-many $p$-shaped arrays
is the same as computing the sum of $i$-th index of every array for all $p$ indices.
If we have a sum of the conditional with the predicate is the equality of indices
$i$ and $j$ and we know that $i$ and $j$ are variables, we can compare the index
variable of the \AC{sum} with $i$ and $j$.  If they match, then conditional will
be triggered at every iteration so it can be removed.  If only one of them match,
and we are comparing variables of the same shape, there will be exactly one case
(for non-empty shapes) where this conditional will be triggered.  Therefore, all
the iterations except the one at the non-matching variable will turn to zero, and
we can simply return the expressions substituted at this variable.  If the shape
of the index variables is empty, we are in the absurd case, as we cannot possibly
create an element of an empty type.  Finally, if none of the variables match,
the iteration within the \AC{sum} do not affect the result of the predicate ---
it will be either true or false for all the iterations.  Therefore, we can lift
the conditional outside of the sum.
\begin{code}[hide]%
%
\>[2]\AgdaFunction{opt}\AgdaSpace{}%
\AgdaInductiveConstructor{zero}\AgdaSpace{}%
\AgdaSymbol{=}\AgdaSpace{}%
\AgdaInductiveConstructor{zero}\<%
\\
%
\>[2]\AgdaFunction{opt}\AgdaSpace{}%
\AgdaInductiveConstructor{one}\AgdaSpace{}%
\AgdaSymbol{=}\AgdaSpace{}%
\AgdaInductiveConstructor{one}\<%
\\
\>[0]\<%
\\
%
\>[2]\AgdaFunction{opt}\AgdaSpace{}%
\AgdaSymbol{(}\AgdaInductiveConstructor{var}\AgdaSpace{}%
\AgdaBound{x}\AgdaSymbol{)}\AgdaSpace{}%
\AgdaSymbol{=}\AgdaSpace{}%
\AgdaInductiveConstructor{var}\AgdaSpace{}%
\AgdaBound{x}\<%
\\
\>[0]\<%
\\
%
\>[2]\AgdaFunction{opt}\AgdaSpace{}%
\AgdaSymbol{(}\AgdaInductiveConstructor{imapₛ}\AgdaSpace{}%
\AgdaBound{e}\AgdaSymbol{)}\AgdaSpace{}%
\AgdaSymbol{=}\AgdaSpace{}%
\AgdaInductiveConstructor{imapₛ}\AgdaSpace{}%
\AgdaSymbol{(}\AgdaFunction{opt}\AgdaSpace{}%
\AgdaBound{e}\AgdaSymbol{)}\<%
\\
\>[0]\<%
\\
%
\>[2]\AgdaComment{--\ Literal\ copy\ of\ the\ above,\ replaing\ scalar\ versions}\<%
\\
%
\>[2]\AgdaComment{--\ with\ normal\ one}\<%
\\
%
\>[2]\AgdaFunction{opt}\AgdaSpace{}%
\AgdaSymbol{(}\AgdaInductiveConstructor{imap}\AgdaSpace{}%
\AgdaBound{e}\AgdaSymbol{)}\AgdaSpace{}%
\AgdaSymbol{=}\AgdaSpace{}%
\AgdaInductiveConstructor{imap}\AgdaSpace{}%
\AgdaSymbol{(}\AgdaFunction{opt}\AgdaSpace{}%
\AgdaBound{e}\AgdaSymbol{)}\<%
\\
%
\>[2]\AgdaFunction{opt}\AgdaSpace{}%
\AgdaSymbol{(}\AgdaInductiveConstructor{sel}\AgdaSpace{}%
\AgdaBound{e}\AgdaSpace{}%
\AgdaBound{e₁}\AgdaSymbol{)}\AgdaSpace{}%
\AgdaKeyword{with}\AgdaSpace{}%
\AgdaFunction{opt}\AgdaSpace{}%
\AgdaBound{e}\AgdaSpace{}%
\AgdaSymbol{|}\AgdaSpace{}%
\AgdaFunction{opt}\AgdaSpace{}%
\AgdaBound{e₁}\<%
\\
%
\>[2]\AgdaSymbol{...}\AgdaSpace{}%
\AgdaSymbol{|}\AgdaSpace{}%
\AgdaInductiveConstructor{zero}\AgdaSpace{}%
\AgdaSymbol{|}\AgdaSpace{}%
\AgdaBound{i}\AgdaSpace{}%
\AgdaSymbol{=}\AgdaSpace{}%
\AgdaInductiveConstructor{zero}\<%
\\
%
\>[2]\AgdaSymbol{...}\AgdaSpace{}%
\AgdaSymbol{|}\AgdaSpace{}%
\AgdaInductiveConstructor{one}\AgdaSpace{}%
\AgdaSymbol{|}\AgdaSpace{}%
\AgdaBound{i}\AgdaSpace{}%
\AgdaSymbol{=}\AgdaSpace{}%
\AgdaInductiveConstructor{one}\<%
\\
%
\>[2]\AgdaSymbol{...}\AgdaSpace{}%
\AgdaSymbol{|}\AgdaSpace{}%
\AgdaInductiveConstructor{imap}\AgdaSpace{}%
\AgdaBound{e}\AgdaSpace{}%
\AgdaSymbol{|}\AgdaSpace{}%
\AgdaBound{i}\AgdaSpace{}%
\AgdaSymbol{=}\AgdaSpace{}%
\AgdaFunction{sub}\AgdaSpace{}%
\AgdaInductiveConstructor{v₀}\AgdaSpace{}%
\AgdaBound{e}\AgdaSpace{}%
\AgdaBound{i}\<%
\\
%
\>[2]\AgdaComment{--...\ |\ imapb\ m\ e\ |\ i\ =\ ?}\<%
\\
%
\>[2]\AgdaSymbol{...}\AgdaSpace{}%
\AgdaSymbol{|}\AgdaSpace{}%
\AgdaInductiveConstructor{bin}\AgdaSpace{}%
\AgdaBound{op}\AgdaSpace{}%
\AgdaBound{a}\AgdaSpace{}%
\AgdaBound{b}\AgdaSpace{}%
\AgdaSymbol{|}\AgdaSpace{}%
\AgdaBound{i}\AgdaSpace{}%
\AgdaSymbol{=}\AgdaSpace{}%
\AgdaInductiveConstructor{bin}\AgdaSpace{}%
\AgdaBound{op}\AgdaSpace{}%
\AgdaSymbol{(}\AgdaInductiveConstructor{sel}\AgdaSpace{}%
\AgdaBound{a}\AgdaSpace{}%
\AgdaBound{i}\AgdaSymbol{)}\AgdaSpace{}%
\AgdaSymbol{(}\AgdaInductiveConstructor{sel}\AgdaSpace{}%
\AgdaBound{b}\AgdaSpace{}%
\AgdaBound{i}\AgdaSymbol{)}\<%
\\
%
\>[2]\AgdaSymbol{...}\AgdaSpace{}%
\AgdaSymbol{|}\AgdaSpace{}%
\AgdaInductiveConstructor{sum}\AgdaSpace{}%
\AgdaBound{e}\AgdaSpace{}%
\AgdaSymbol{|}\AgdaSpace{}%
\AgdaBound{i}\AgdaSpace{}%
\AgdaSymbol{=}\AgdaSpace{}%
\AgdaInductiveConstructor{sum}\AgdaSpace{}%
\AgdaSymbol{(}\AgdaInductiveConstructor{sel}\AgdaSpace{}%
\AgdaBound{e}\AgdaSpace{}%
\AgdaSymbol{(}\AgdaFunction{wk}\AgdaSpace{}%
\AgdaInductiveConstructor{v₀}\AgdaSpace{}%
\AgdaBound{i}\AgdaSymbol{))}\<%
\\
%
\>[2]\AgdaSymbol{...}\AgdaSpace{}%
\AgdaSymbol{|}\AgdaSpace{}%
\AgdaInductiveConstructor{zero-but}\AgdaSpace{}%
\AgdaBound{i}\AgdaSpace{}%
\AgdaBound{j}\AgdaSpace{}%
\AgdaBound{a}\AgdaSpace{}%
\AgdaSymbol{|}\AgdaSpace{}%
\AgdaBound{k}\AgdaSpace{}%
\AgdaSymbol{=}\AgdaSpace{}%
\AgdaInductiveConstructor{zero-but}\AgdaSpace{}%
\AgdaBound{i}\AgdaSpace{}%
\AgdaBound{j}\AgdaSpace{}%
\AgdaSymbol{(}\AgdaInductiveConstructor{sel}\AgdaSpace{}%
\AgdaBound{a}\AgdaSpace{}%
\AgdaBound{k}\AgdaSymbol{)}\<%
\\
%
\>[2]\AgdaCatchallClause{\AgdaSymbol{...}}\AgdaSpace{}%
\AgdaCatchallClause{\AgdaSymbol{|}}\AgdaSpace{}%
\AgdaCatchallClause{\AgdaBound{a}}\AgdaSpace{}%
\AgdaCatchallClause{\AgdaSymbol{|}}\AgdaSpace{}%
\AgdaCatchallClause{\AgdaBound{i}}\AgdaSpace{}%
\AgdaSymbol{=}\AgdaSpace{}%
\AgdaInductiveConstructor{sel}\AgdaSpace{}%
\AgdaBound{a}\AgdaSpace{}%
\AgdaBound{i}\<%
\\
\>[0]\<%
\\
%
\>[2]\AgdaComment{--\ Literal\ copy\ of\ the\ above\ for\ the\ blocked\ version}\<%
\\
%
\>[2]\AgdaFunction{opt}\AgdaSpace{}%
\AgdaSymbol{(}\AgdaInductiveConstructor{imapb}\AgdaSpace{}%
\AgdaBound{m}\AgdaSpace{}%
\AgdaBound{e}\AgdaSymbol{)}\AgdaSpace{}%
\AgdaSymbol{=}\AgdaSpace{}%
\AgdaInductiveConstructor{imapb}\AgdaSpace{}%
\AgdaBound{m}\AgdaSpace{}%
\AgdaSymbol{(}\AgdaFunction{opt}\AgdaSpace{}%
\AgdaBound{e}\AgdaSymbol{)}\<%
\\
%
\>[2]\AgdaFunction{opt}\AgdaSpace{}%
\AgdaSymbol{(}\AgdaInductiveConstructor{selb}\AgdaSpace{}%
\AgdaBound{m}\AgdaSpace{}%
\AgdaBound{e}\AgdaSpace{}%
\AgdaBound{k}\AgdaSymbol{)}\AgdaSpace{}%
\AgdaKeyword{with}\AgdaSpace{}%
\AgdaFunction{opt}\AgdaSpace{}%
\AgdaBound{e}\<%
\\
%
\>[2]\AgdaSymbol{...}\AgdaSpace{}%
\AgdaSymbol{|}\AgdaSpace{}%
\AgdaInductiveConstructor{zero}\AgdaSpace{}%
\AgdaSymbol{=}\AgdaSpace{}%
\AgdaInductiveConstructor{zero}\<%
\\
%
\>[2]\AgdaSymbol{...}\AgdaSpace{}%
\AgdaSymbol{|}\AgdaSpace{}%
\AgdaInductiveConstructor{one}\AgdaSpace{}%
\AgdaSymbol{=}\AgdaSpace{}%
\AgdaInductiveConstructor{one}\<%
\\
%
\>[2]\AgdaSymbol{...}\AgdaSpace{}%
\AgdaSymbol{|}\AgdaSpace{}%
\AgdaInductiveConstructor{sum}\AgdaSpace{}%
\AgdaBound{e}\AgdaSpace{}%
\AgdaSymbol{=}\AgdaSpace{}%
\AgdaInductiveConstructor{sum}\AgdaSpace{}%
\AgdaSymbol{(}\AgdaInductiveConstructor{selb}\AgdaSpace{}%
\AgdaBound{m}\AgdaSpace{}%
\AgdaBound{e}\AgdaSpace{}%
\AgdaSymbol{(}\AgdaOperator{\AgdaFunction{↑}}\AgdaSpace{}%
\AgdaBound{k}\AgdaSpace{}%
\AgdaComment{\{-\ var\ \$\ vₛ\ k-\}}\AgdaSymbol{))}\<%
\\
%
\>[2]\AgdaSymbol{...}\AgdaSpace{}%
\AgdaSymbol{|}\AgdaSpace{}%
\AgdaInductiveConstructor{zero-but}\AgdaSpace{}%
\AgdaBound{i}\AgdaSpace{}%
\AgdaBound{j}\AgdaSpace{}%
\AgdaBound{a}\AgdaSpace{}%
\AgdaSymbol{=}\AgdaSpace{}%
\AgdaInductiveConstructor{zero-but}\AgdaSpace{}%
\AgdaBound{i}\AgdaSpace{}%
\AgdaBound{j}\AgdaSpace{}%
\AgdaSymbol{(}\AgdaInductiveConstructor{selb}\AgdaSpace{}%
\AgdaBound{m}\AgdaSpace{}%
\AgdaBound{a}\AgdaSpace{}%
\AgdaBound{k}\AgdaSymbol{)}\<%
\\
%
\>[2]\AgdaSymbol{...}\AgdaSpace{}%
\AgdaSymbol{|}\AgdaSpace{}%
\AgdaInductiveConstructor{bin}\AgdaSpace{}%
\AgdaBound{op}\AgdaSpace{}%
\AgdaBound{a}\AgdaSpace{}%
\AgdaBound{b}\AgdaSpace{}%
\AgdaSymbol{=}\AgdaSpace{}%
\AgdaInductiveConstructor{bin}\AgdaSpace{}%
\AgdaBound{op}\AgdaSpace{}%
\AgdaSymbol{(}\AgdaInductiveConstructor{selb}\AgdaSpace{}%
\AgdaBound{m}\AgdaSpace{}%
\AgdaBound{a}\AgdaSpace{}%
\AgdaBound{k}\AgdaSymbol{)}\AgdaSpace{}%
\AgdaSymbol{(}\AgdaInductiveConstructor{selb}\AgdaSpace{}%
\AgdaBound{m}\AgdaSpace{}%
\AgdaBound{b}\AgdaSpace{}%
\AgdaBound{k}\AgdaSymbol{)}\<%
\\
%
\>[2]\AgdaCatchallClause{\AgdaFunction{opt}}\AgdaSpace{}%
\AgdaCatchallClause{\AgdaSymbol{(}}\AgdaCatchallClause{\AgdaInductiveConstructor{selb}}\AgdaSpace{}%
\AgdaCatchallClause{\AgdaBound{m}}\AgdaSpace{}%
\AgdaCatchallClause{\AgdaBound{e}}\AgdaSpace{}%
\AgdaCatchallClause{\AgdaBound{j}}\AgdaCatchallClause{\AgdaSymbol{)}}\AgdaSpace{}%
\AgdaCatchallClause{\AgdaSymbol{|}}\AgdaSpace{}%
\AgdaCatchallClause{\AgdaBound{a}}\AgdaSpace{}%
\AgdaSymbol{=}\AgdaSpace{}%
\AgdaInductiveConstructor{selb}\AgdaSpace{}%
\AgdaBound{m}\AgdaSpace{}%
\AgdaBound{a}\AgdaSpace{}%
\AgdaBound{j}\<%
\\
\>[0]\<%
\\
\>[0]\<%
\\
%
\>[2]\AgdaFunction{opt}\AgdaSpace{}%
\AgdaSymbol{(}\AgdaInductiveConstructor{zero-but}\AgdaSpace{}%
\AgdaSymbol{(}\AgdaInductiveConstructor{var}\AgdaSpace{}%
\AgdaBound{i}\AgdaSymbol{)}\AgdaSpace{}%
\AgdaSymbol{(}\AgdaInductiveConstructor{var}\AgdaSpace{}%
\AgdaBound{j}\AgdaSymbol{)}\AgdaSpace{}%
\AgdaBound{e}\AgdaSymbol{)}\AgdaSpace{}%
\AgdaKeyword{with}\AgdaSpace{}%
\AgdaFunction{opt}\AgdaSpace{}%
\AgdaBound{e}\<%
\\
%
\>[2]\AgdaSymbol{...}\AgdaSpace{}%
\AgdaSymbol{|}\AgdaSpace{}%
\AgdaBound{a}\AgdaSpace{}%
\AgdaKeyword{with}\AgdaSpace{}%
\AgdaFunction{eq?}\AgdaSpace{}%
\AgdaBound{i}\AgdaSpace{}%
\AgdaBound{j}\<%
\\
%
\>[2]\AgdaSymbol{...}\AgdaSpace{}%
\AgdaSymbol{|}\AgdaSpace{}%
\AgdaInductiveConstructor{eq}\AgdaSpace{}%
\AgdaSymbol{=}\AgdaSpace{}%
\AgdaBound{a}\<%
\\
%
\>[2]\AgdaSymbol{...}\AgdaSpace{}%
\AgdaSymbol{|}\AgdaSpace{}%
\AgdaInductiveConstructor{neq}\AgdaSpace{}%
\AgdaSymbol{\AgdaUnderscore{}}\AgdaSpace{}%
\AgdaSymbol{\AgdaUnderscore{}}\AgdaSpace{}%
\AgdaSymbol{=}\AgdaSpace{}%
\AgdaInductiveConstructor{zero-but}\AgdaSpace{}%
\AgdaSymbol{(}\AgdaInductiveConstructor{var}\AgdaSpace{}%
\AgdaBound{i}\AgdaSymbol{)}\AgdaSpace{}%
\AgdaSymbol{(}\AgdaInductiveConstructor{var}\AgdaSpace{}%
\AgdaBound{j}\AgdaSymbol{)}\AgdaSpace{}%
\AgdaBound{a}\<%
\\
%
\>[2]\AgdaComment{--opt\ (zero-but\ i\ j\ e)\ =\ zero-but\ i\ j\ (opt\ e)}\<%
\\
\>[0]\<%
\\
%
\>[2]\AgdaFunction{opt}\AgdaSpace{}%
\AgdaSymbol{(}\AgdaInductiveConstructor{bin}\AgdaSpace{}%
\AgdaInductiveConstructor{plus}\AgdaSpace{}%
\AgdaBound{e}\AgdaSpace{}%
\AgdaBound{e₁}\AgdaSymbol{)}\AgdaSpace{}%
\AgdaKeyword{with}\AgdaSpace{}%
\AgdaFunction{opt}\AgdaSpace{}%
\AgdaBound{e}\AgdaSpace{}%
\AgdaSymbol{|}\AgdaSpace{}%
\AgdaFunction{opt}\AgdaSpace{}%
\AgdaBound{e₁}\<%
\\
%
\>[2]\AgdaSymbol{...}\AgdaSpace{}%
\AgdaSymbol{|}\AgdaSpace{}%
\AgdaInductiveConstructor{zero}\AgdaSpace{}%
\AgdaSymbol{|}\AgdaSpace{}%
\AgdaBound{b}\AgdaSpace{}%
\AgdaSymbol{=}\AgdaSpace{}%
\AgdaBound{b}\<%
\\
%
\>[2]\AgdaCatchallClause{\AgdaSymbol{...}}\AgdaSpace{}%
\AgdaCatchallClause{\AgdaSymbol{|}}\AgdaSpace{}%
\AgdaCatchallClause{\AgdaBound{a}}\AgdaSpace{}%
\AgdaCatchallClause{\AgdaSymbol{|}}\AgdaSpace{}%
\AgdaCatchallClause{\AgdaInductiveConstructor{zero}}\AgdaSpace{}%
\AgdaSymbol{=}\AgdaSpace{}%
\AgdaBound{a}\<%
\\
%
\>[2]\AgdaCatchallClause{\AgdaSymbol{...}}\AgdaSpace{}%
\AgdaCatchallClause{\AgdaSymbol{|}}\AgdaSpace{}%
\AgdaCatchallClause{\AgdaSymbol{(}}\AgdaCatchallClause{\AgdaInductiveConstructor{zero-but}}\AgdaSpace{}%
\AgdaCatchallClause{\AgdaBound{i}}\AgdaSpace{}%
\AgdaCatchallClause{\AgdaBound{j}}\AgdaSpace{}%
\AgdaCatchallClause{\AgdaBound{e}}\AgdaCatchallClause{\AgdaSymbol{)}}\AgdaSpace{}%
\AgdaCatchallClause{\AgdaSymbol{|}}\AgdaSpace{}%
\AgdaCatchallClause{\AgdaBound{b}}\AgdaSpace{}%
\AgdaSymbol{=}\AgdaSpace{}%
\AgdaInductiveConstructor{zero-but}\AgdaSpace{}%
\AgdaBound{i}\AgdaSpace{}%
\AgdaBound{j}\AgdaSpace{}%
\AgdaSymbol{(}\AgdaInductiveConstructor{bin}\AgdaSpace{}%
\AgdaInductiveConstructor{plus}\AgdaSpace{}%
\AgdaBound{e}\AgdaSpace{}%
\AgdaBound{b}\AgdaSymbol{)}\<%
\\
%
\>[2]\AgdaCatchallClause{\AgdaSymbol{...}}\AgdaSpace{}%
\AgdaCatchallClause{\AgdaSymbol{|}}\AgdaSpace{}%
\AgdaCatchallClause{\AgdaBound{a}}\AgdaSpace{}%
\AgdaCatchallClause{\AgdaSymbol{|}}\AgdaSpace{}%
\AgdaCatchallClause{\AgdaSymbol{(}}\AgdaCatchallClause{\AgdaInductiveConstructor{zero-but}}\AgdaSpace{}%
\AgdaCatchallClause{\AgdaBound{i}}\AgdaSpace{}%
\AgdaCatchallClause{\AgdaBound{j}}\AgdaSpace{}%
\AgdaCatchallClause{\AgdaBound{e}}\AgdaCatchallClause{\AgdaSymbol{)}}\AgdaSpace{}%
\AgdaSymbol{=}\AgdaSpace{}%
\AgdaInductiveConstructor{zero-but}\AgdaSpace{}%
\AgdaBound{i}\AgdaSpace{}%
\AgdaBound{j}\AgdaSpace{}%
\AgdaSymbol{(}\AgdaInductiveConstructor{bin}\AgdaSpace{}%
\AgdaInductiveConstructor{plus}\AgdaSpace{}%
\AgdaBound{a}\AgdaSpace{}%
\AgdaBound{e}\AgdaSymbol{)}\<%
\\
%
\\[\AgdaEmptyExtraSkip]%
%
\>[2]\AgdaCatchallClause{\AgdaSymbol{...}}\AgdaSpace{}%
\AgdaCatchallClause{\AgdaSymbol{|}}\AgdaSpace{}%
\AgdaCatchallClause{\AgdaInductiveConstructor{imapₛ}}\AgdaSpace{}%
\AgdaCatchallClause{\AgdaBound{a}}\AgdaSpace{}%
\AgdaCatchallClause{\AgdaSymbol{|}}\AgdaSpace{}%
\AgdaCatchallClause{\AgdaBound{b}}\AgdaSpace{}%
\AgdaSymbol{=}\AgdaSpace{}%
\AgdaInductiveConstructor{imapₛ}\AgdaSpace{}%
\AgdaSymbol{(}\AgdaInductiveConstructor{bin}\AgdaSpace{}%
\AgdaInductiveConstructor{plus}\AgdaSpace{}%
\AgdaBound{a}\AgdaSpace{}%
\AgdaSymbol{(}\AgdaInductiveConstructor{selₛ}\AgdaSpace{}%
\AgdaSymbol{(}\AgdaOperator{\AgdaFunction{↑}}\AgdaSpace{}%
\AgdaBound{b}\AgdaSymbol{)}\AgdaSpace{}%
\AgdaSymbol{(}\AgdaInductiveConstructor{var}\AgdaSpace{}%
\AgdaInductiveConstructor{v₀}\AgdaSymbol{)))}\<%
\\
%
\>[2]\AgdaCatchallClause{\AgdaSymbol{...}}\AgdaSpace{}%
\AgdaCatchallClause{\AgdaSymbol{|}}\AgdaSpace{}%
\AgdaCatchallClause{\AgdaBound{a}}\AgdaSpace{}%
\AgdaCatchallClause{\AgdaSymbol{|}}\AgdaSpace{}%
\AgdaCatchallClause{\AgdaInductiveConstructor{imapₛ}}\AgdaSpace{}%
\AgdaCatchallClause{\AgdaBound{b}}\AgdaSpace{}%
\AgdaSymbol{=}\AgdaSpace{}%
\AgdaInductiveConstructor{imapₛ}\AgdaSpace{}%
\AgdaSymbol{(}\AgdaInductiveConstructor{bin}\AgdaSpace{}%
\AgdaInductiveConstructor{plus}\AgdaSpace{}%
\AgdaSymbol{(}\AgdaInductiveConstructor{selₛ}\AgdaSpace{}%
\AgdaSymbol{(}\AgdaOperator{\AgdaFunction{↑}}\AgdaSpace{}%
\AgdaBound{a}\AgdaSymbol{)}\AgdaSpace{}%
\AgdaSymbol{(}\AgdaInductiveConstructor{var}\AgdaSpace{}%
\AgdaInductiveConstructor{v₀}\AgdaSymbol{))}\AgdaSpace{}%
\AgdaBound{b}\AgdaSymbol{)}\<%
\\
%
\>[2]\AgdaCatchallClause{\AgdaSymbol{...}}\AgdaSpace{}%
\AgdaCatchallClause{\AgdaSymbol{|}}\AgdaSpace{}%
\AgdaCatchallClause{\AgdaInductiveConstructor{imap}}\AgdaSpace{}%
\AgdaCatchallClause{\AgdaBound{a}}\AgdaSpace{}%
\AgdaCatchallClause{\AgdaSymbol{|}}\AgdaSpace{}%
\AgdaCatchallClause{\AgdaBound{b}}\AgdaSpace{}%
\AgdaSymbol{=}\AgdaSpace{}%
\AgdaInductiveConstructor{imap}\AgdaSpace{}%
\AgdaSymbol{(}\AgdaInductiveConstructor{bin}\AgdaSpace{}%
\AgdaInductiveConstructor{plus}\AgdaSpace{}%
\AgdaBound{a}\AgdaSpace{}%
\AgdaSymbol{(}\AgdaInductiveConstructor{sel}\AgdaSpace{}%
\AgdaSymbol{(}\AgdaOperator{\AgdaFunction{↑}}\AgdaSpace{}%
\AgdaBound{b}\AgdaSymbol{)}\AgdaSpace{}%
\AgdaSymbol{(}\AgdaInductiveConstructor{var}\AgdaSpace{}%
\AgdaInductiveConstructor{v₀}\AgdaSymbol{)))}\<%
\\
%
\>[2]\AgdaCatchallClause{\AgdaSymbol{...}}\AgdaSpace{}%
\AgdaCatchallClause{\AgdaSymbol{|}}\AgdaSpace{}%
\AgdaCatchallClause{\AgdaBound{a}}\AgdaSpace{}%
\AgdaCatchallClause{\AgdaSymbol{|}}\AgdaSpace{}%
\AgdaCatchallClause{\AgdaInductiveConstructor{imap}}\AgdaSpace{}%
\AgdaCatchallClause{\AgdaBound{b}}\AgdaSpace{}%
\AgdaSymbol{=}\AgdaSpace{}%
\AgdaInductiveConstructor{imap}\AgdaSpace{}%
\AgdaSymbol{(}\AgdaInductiveConstructor{bin}\AgdaSpace{}%
\AgdaInductiveConstructor{plus}\AgdaSpace{}%
\AgdaSymbol{(}\AgdaInductiveConstructor{sel}\AgdaSpace{}%
\AgdaSymbol{(}\AgdaOperator{\AgdaFunction{↑}}\AgdaSpace{}%
\AgdaBound{a}\AgdaSymbol{)}\AgdaSpace{}%
\AgdaSymbol{(}\AgdaInductiveConstructor{var}\AgdaSpace{}%
\AgdaInductiveConstructor{v₀}\AgdaSymbol{))}\AgdaSpace{}%
\AgdaBound{b}\AgdaSymbol{)}\<%
\\
%
\>[2]\AgdaCatchallClause{\AgdaSymbol{...}}\AgdaSpace{}%
\AgdaCatchallClause{\AgdaSymbol{|}}\AgdaSpace{}%
\AgdaCatchallClause{\AgdaInductiveConstructor{imapb}}\AgdaSpace{}%
\AgdaCatchallClause{\AgdaBound{m}}\AgdaSpace{}%
\AgdaCatchallClause{\AgdaBound{a}}\AgdaSpace{}%
\AgdaCatchallClause{\AgdaSymbol{|}}\AgdaSpace{}%
\AgdaCatchallClause{\AgdaBound{b}}\AgdaSpace{}%
\AgdaSymbol{=}\AgdaSpace{}%
\AgdaInductiveConstructor{imapb}\AgdaSpace{}%
\AgdaBound{m}\AgdaSpace{}%
\AgdaSymbol{(}\AgdaInductiveConstructor{bin}\AgdaSpace{}%
\AgdaInductiveConstructor{plus}\AgdaSpace{}%
\AgdaBound{a}\AgdaSpace{}%
\AgdaSymbol{(}\AgdaInductiveConstructor{selb}\AgdaSpace{}%
\AgdaBound{m}\AgdaSpace{}%
\AgdaSymbol{(}\AgdaOperator{\AgdaFunction{↑}}\AgdaSpace{}%
\AgdaBound{b}\AgdaSymbol{)}\AgdaSpace{}%
\AgdaSymbol{(}\AgdaInductiveConstructor{var}\AgdaSpace{}%
\AgdaInductiveConstructor{v₀}\AgdaSymbol{)))}\<%
\\
%
\>[2]\AgdaCatchallClause{\AgdaSymbol{...}}\AgdaSpace{}%
\AgdaCatchallClause{\AgdaSymbol{|}}\AgdaSpace{}%
\AgdaCatchallClause{\AgdaBound{a}}\AgdaSpace{}%
\AgdaCatchallClause{\AgdaSymbol{|}}\AgdaSpace{}%
\AgdaCatchallClause{\AgdaInductiveConstructor{imapb}}\AgdaSpace{}%
\AgdaCatchallClause{\AgdaBound{m}}\AgdaSpace{}%
\AgdaCatchallClause{\AgdaBound{b}}\AgdaSpace{}%
\AgdaSymbol{=}\AgdaSpace{}%
\AgdaInductiveConstructor{imapb}\AgdaSpace{}%
\AgdaBound{m}\AgdaSpace{}%
\AgdaSymbol{(}\AgdaInductiveConstructor{bin}\AgdaSpace{}%
\AgdaInductiveConstructor{plus}\AgdaSpace{}%
\AgdaSymbol{(}\AgdaInductiveConstructor{selb}\AgdaSpace{}%
\AgdaBound{m}\AgdaSpace{}%
\AgdaSymbol{(}\AgdaOperator{\AgdaFunction{↑}}\AgdaSpace{}%
\AgdaBound{a}\AgdaSymbol{)}\AgdaSpace{}%
\AgdaSymbol{(}\AgdaInductiveConstructor{var}\AgdaSpace{}%
\AgdaInductiveConstructor{v₀}\AgdaSymbol{))}\AgdaSpace{}%
\AgdaBound{b}\AgdaSymbol{)}\<%
\\
%
\\[\AgdaEmptyExtraSkip]%
%
\>[2]\AgdaCatchallClause{\AgdaSymbol{...}}\AgdaSpace{}%
\AgdaCatchallClause{\AgdaSymbol{|}}\AgdaSpace{}%
\AgdaCatchallClause{\AgdaBound{a}}\AgdaSpace{}%
\AgdaCatchallClause{\AgdaSymbol{|}}\AgdaSpace{}%
\AgdaCatchallClause{\AgdaBound{b}}\AgdaSpace{}%
\AgdaSymbol{=}\AgdaSpace{}%
\AgdaInductiveConstructor{bin}\AgdaSpace{}%
\AgdaInductiveConstructor{plus}\AgdaSpace{}%
\AgdaBound{a}\AgdaSpace{}%
\AgdaBound{b}\<%
\\
%
\>[2]\AgdaFunction{opt}\AgdaSpace{}%
\AgdaSymbol{(}\AgdaInductiveConstructor{bin}\AgdaSpace{}%
\AgdaInductiveConstructor{mul}\AgdaSpace{}%
\AgdaBound{e}\AgdaSpace{}%
\AgdaBound{e₁}\AgdaSymbol{)}\AgdaSpace{}%
\AgdaKeyword{with}\AgdaSpace{}%
\AgdaFunction{opt}\AgdaSpace{}%
\AgdaBound{e}\AgdaSpace{}%
\AgdaSymbol{|}\AgdaSpace{}%
\AgdaFunction{opt}\AgdaSpace{}%
\AgdaBound{e₁}\<%
\\
%
\>[2]\AgdaSymbol{...}\AgdaSpace{}%
\AgdaSymbol{|}\AgdaSpace{}%
\AgdaInductiveConstructor{zero}\AgdaSpace{}%
\AgdaSymbol{|}\AgdaSpace{}%
\AgdaBound{b}\AgdaSpace{}%
\AgdaSymbol{=}\AgdaSpace{}%
\AgdaInductiveConstructor{zero}\<%
\\
%
\>[2]\AgdaCatchallClause{\AgdaSymbol{...}}\AgdaSpace{}%
\AgdaCatchallClause{\AgdaSymbol{|}}\AgdaSpace{}%
\AgdaCatchallClause{\AgdaBound{a}}\AgdaSpace{}%
\AgdaCatchallClause{\AgdaSymbol{|}}\AgdaSpace{}%
\AgdaCatchallClause{\AgdaInductiveConstructor{zero}}\AgdaSpace{}%
\AgdaSymbol{=}\AgdaSpace{}%
\AgdaInductiveConstructor{zero}\<%
\\
%
\>[2]\AgdaCatchallClause{\AgdaSymbol{...}}\AgdaSpace{}%
\AgdaCatchallClause{\AgdaSymbol{|}}\AgdaSpace{}%
\AgdaCatchallClause{\AgdaInductiveConstructor{one}}\AgdaSpace{}%
\AgdaCatchallClause{\AgdaSymbol{|}}\AgdaSpace{}%
\AgdaCatchallClause{\AgdaBound{b}}\AgdaSpace{}%
\AgdaSymbol{=}\AgdaSpace{}%
\AgdaBound{b}\<%
\\
%
\>[2]\AgdaCatchallClause{\AgdaSymbol{...}}\AgdaSpace{}%
\AgdaCatchallClause{\AgdaSymbol{|}}\AgdaSpace{}%
\AgdaCatchallClause{\AgdaBound{a}}\AgdaSpace{}%
\AgdaCatchallClause{\AgdaSymbol{|}}\AgdaSpace{}%
\AgdaCatchallClause{\AgdaInductiveConstructor{one}}\AgdaSpace{}%
\AgdaSymbol{=}\AgdaSpace{}%
\AgdaBound{a}\<%
\\
%
\>[2]\AgdaCatchallClause{\AgdaSymbol{...}}\AgdaSpace{}%
\AgdaCatchallClause{\AgdaSymbol{|}}\AgdaSpace{}%
\AgdaCatchallClause{\AgdaSymbol{(}}\AgdaCatchallClause{\AgdaInductiveConstructor{zero-but}}\AgdaSpace{}%
\AgdaCatchallClause{\AgdaBound{i}}\AgdaSpace{}%
\AgdaCatchallClause{\AgdaBound{j}}\AgdaSpace{}%
\AgdaCatchallClause{\AgdaBound{e}}\AgdaCatchallClause{\AgdaSymbol{)}}\AgdaSpace{}%
\AgdaCatchallClause{\AgdaSymbol{|}}\AgdaSpace{}%
\AgdaCatchallClause{\AgdaBound{b}}\AgdaSpace{}%
\AgdaSymbol{=}\AgdaSpace{}%
\AgdaInductiveConstructor{zero-but}\AgdaSpace{}%
\AgdaBound{i}\AgdaSpace{}%
\AgdaBound{j}\AgdaSpace{}%
\AgdaSymbol{(}\AgdaInductiveConstructor{bin}\AgdaSpace{}%
\AgdaInductiveConstructor{mul}\AgdaSpace{}%
\AgdaBound{e}\AgdaSpace{}%
\AgdaBound{b}\AgdaSymbol{)}\<%
\\
%
\>[2]\AgdaCatchallClause{\AgdaSymbol{...}}\AgdaSpace{}%
\AgdaCatchallClause{\AgdaSymbol{|}}\AgdaSpace{}%
\AgdaCatchallClause{\AgdaBound{a}}\AgdaSpace{}%
\AgdaCatchallClause{\AgdaSymbol{|}}\AgdaSpace{}%
\AgdaCatchallClause{\AgdaSymbol{(}}\AgdaCatchallClause{\AgdaInductiveConstructor{zero-but}}\AgdaSpace{}%
\AgdaCatchallClause{\AgdaBound{i}}\AgdaSpace{}%
\AgdaCatchallClause{\AgdaBound{j}}\AgdaSpace{}%
\AgdaCatchallClause{\AgdaBound{e}}\AgdaCatchallClause{\AgdaSymbol{)}}\AgdaSpace{}%
\AgdaSymbol{=}\AgdaSpace{}%
\AgdaInductiveConstructor{zero-but}\AgdaSpace{}%
\AgdaBound{i}\AgdaSpace{}%
\AgdaBound{j}\AgdaSpace{}%
\AgdaSymbol{(}\AgdaInductiveConstructor{bin}\AgdaSpace{}%
\AgdaInductiveConstructor{mul}\AgdaSpace{}%
\AgdaBound{a}\AgdaSpace{}%
\AgdaBound{e}\AgdaSymbol{)}\<%
\\
\>[0]\<%
\\
%
\>[2]\AgdaCatchallClause{\AgdaSymbol{...}}\AgdaSpace{}%
\AgdaCatchallClause{\AgdaSymbol{|}}\AgdaSpace{}%
\AgdaCatchallClause{\AgdaInductiveConstructor{imapₛ}}\AgdaSpace{}%
\AgdaCatchallClause{\AgdaBound{a}}\AgdaSpace{}%
\AgdaCatchallClause{\AgdaSymbol{|}}\AgdaSpace{}%
\AgdaCatchallClause{\AgdaBound{b}}\AgdaSpace{}%
\AgdaSymbol{=}\AgdaSpace{}%
\AgdaInductiveConstructor{imapₛ}\AgdaSpace{}%
\AgdaSymbol{(}\AgdaInductiveConstructor{bin}\AgdaSpace{}%
\AgdaInductiveConstructor{mul}\AgdaSpace{}%
\AgdaBound{a}\AgdaSpace{}%
\AgdaSymbol{(}\AgdaInductiveConstructor{selₛ}\AgdaSpace{}%
\AgdaSymbol{(}\AgdaOperator{\AgdaFunction{↑}}\AgdaSpace{}%
\AgdaBound{b}\AgdaSymbol{)}\AgdaSpace{}%
\AgdaSymbol{(}\AgdaInductiveConstructor{var}\AgdaSpace{}%
\AgdaInductiveConstructor{v₀}\AgdaSymbol{)))}\<%
\\
%
\>[2]\AgdaCatchallClause{\AgdaSymbol{...}}\AgdaSpace{}%
\AgdaCatchallClause{\AgdaSymbol{|}}\AgdaSpace{}%
\AgdaCatchallClause{\AgdaBound{a}}\AgdaSpace{}%
\AgdaCatchallClause{\AgdaSymbol{|}}\AgdaSpace{}%
\AgdaCatchallClause{\AgdaInductiveConstructor{imapₛ}}\AgdaSpace{}%
\AgdaCatchallClause{\AgdaBound{b}}\AgdaSpace{}%
\AgdaSymbol{=}\AgdaSpace{}%
\AgdaInductiveConstructor{imapₛ}\AgdaSpace{}%
\AgdaSymbol{(}\AgdaInductiveConstructor{bin}\AgdaSpace{}%
\AgdaInductiveConstructor{mul}\AgdaSpace{}%
\AgdaSymbol{(}\AgdaInductiveConstructor{selₛ}\AgdaSpace{}%
\AgdaSymbol{(}\AgdaOperator{\AgdaFunction{↑}}\AgdaSpace{}%
\AgdaBound{a}\AgdaSymbol{)}\AgdaSpace{}%
\AgdaSymbol{(}\AgdaInductiveConstructor{var}\AgdaSpace{}%
\AgdaInductiveConstructor{v₀}\AgdaSymbol{))}\AgdaSpace{}%
\AgdaBound{b}\AgdaSymbol{)}\<%
\\
%
\>[2]\AgdaCatchallClause{\AgdaSymbol{...}}\AgdaSpace{}%
\AgdaCatchallClause{\AgdaSymbol{|}}\AgdaSpace{}%
\AgdaCatchallClause{\AgdaInductiveConstructor{imap}}\AgdaSpace{}%
\AgdaCatchallClause{\AgdaBound{a}}\AgdaSpace{}%
\AgdaCatchallClause{\AgdaSymbol{|}}\AgdaSpace{}%
\AgdaCatchallClause{\AgdaBound{b}}\AgdaSpace{}%
\AgdaSymbol{=}\AgdaSpace{}%
\AgdaInductiveConstructor{imap}\AgdaSpace{}%
\AgdaSymbol{(}\AgdaInductiveConstructor{bin}\AgdaSpace{}%
\AgdaInductiveConstructor{mul}\AgdaSpace{}%
\AgdaBound{a}\AgdaSpace{}%
\AgdaSymbol{(}\AgdaInductiveConstructor{sel}\AgdaSpace{}%
\AgdaSymbol{(}\AgdaOperator{\AgdaFunction{↑}}\AgdaSpace{}%
\AgdaBound{b}\AgdaSymbol{)}\AgdaSpace{}%
\AgdaSymbol{(}\AgdaInductiveConstructor{var}\AgdaSpace{}%
\AgdaInductiveConstructor{v₀}\AgdaSymbol{)))}\<%
\\
%
\>[2]\AgdaCatchallClause{\AgdaSymbol{...}}\AgdaSpace{}%
\AgdaCatchallClause{\AgdaSymbol{|}}\AgdaSpace{}%
\AgdaCatchallClause{\AgdaBound{a}}\AgdaSpace{}%
\AgdaCatchallClause{\AgdaSymbol{|}}\AgdaSpace{}%
\AgdaCatchallClause{\AgdaInductiveConstructor{imap}}\AgdaSpace{}%
\AgdaCatchallClause{\AgdaBound{b}}\AgdaSpace{}%
\AgdaSymbol{=}\AgdaSpace{}%
\AgdaInductiveConstructor{imap}\AgdaSpace{}%
\AgdaSymbol{(}\AgdaInductiveConstructor{bin}\AgdaSpace{}%
\AgdaInductiveConstructor{mul}\AgdaSpace{}%
\AgdaSymbol{(}\AgdaInductiveConstructor{sel}\AgdaSpace{}%
\AgdaSymbol{(}\AgdaOperator{\AgdaFunction{↑}}\AgdaSpace{}%
\AgdaBound{a}\AgdaSymbol{)}\AgdaSpace{}%
\AgdaSymbol{(}\AgdaInductiveConstructor{var}\AgdaSpace{}%
\AgdaInductiveConstructor{v₀}\AgdaSymbol{))}\AgdaSpace{}%
\AgdaBound{b}\AgdaSymbol{)}\<%
\\
%
\>[2]\AgdaCatchallClause{\AgdaSymbol{...}}\AgdaSpace{}%
\AgdaCatchallClause{\AgdaSymbol{|}}\AgdaSpace{}%
\AgdaCatchallClause{\AgdaInductiveConstructor{imapb}}\AgdaSpace{}%
\AgdaCatchallClause{\AgdaBound{m}}\AgdaSpace{}%
\AgdaCatchallClause{\AgdaBound{a}}\AgdaSpace{}%
\AgdaCatchallClause{\AgdaSymbol{|}}\AgdaSpace{}%
\AgdaCatchallClause{\AgdaBound{b}}\AgdaSpace{}%
\AgdaSymbol{=}\AgdaSpace{}%
\AgdaInductiveConstructor{imapb}\AgdaSpace{}%
\AgdaBound{m}\AgdaSpace{}%
\AgdaSymbol{(}\AgdaInductiveConstructor{bin}\AgdaSpace{}%
\AgdaInductiveConstructor{mul}\AgdaSpace{}%
\AgdaBound{a}\AgdaSpace{}%
\AgdaSymbol{(}\AgdaInductiveConstructor{selb}\AgdaSpace{}%
\AgdaBound{m}\AgdaSpace{}%
\AgdaSymbol{(}\AgdaOperator{\AgdaFunction{↑}}\AgdaSpace{}%
\AgdaBound{b}\AgdaSymbol{)}\AgdaSpace{}%
\AgdaSymbol{(}\AgdaInductiveConstructor{var}\AgdaSpace{}%
\AgdaInductiveConstructor{v₀}\AgdaSymbol{)))}\<%
\\
%
\>[2]\AgdaCatchallClause{\AgdaSymbol{...}}\AgdaSpace{}%
\AgdaCatchallClause{\AgdaSymbol{|}}\AgdaSpace{}%
\AgdaCatchallClause{\AgdaBound{a}}\AgdaSpace{}%
\AgdaCatchallClause{\AgdaSymbol{|}}\AgdaSpace{}%
\AgdaCatchallClause{\AgdaInductiveConstructor{imapb}}\AgdaSpace{}%
\AgdaCatchallClause{\AgdaBound{m}}\AgdaSpace{}%
\AgdaCatchallClause{\AgdaBound{b}}\AgdaSpace{}%
\AgdaSymbol{=}\AgdaSpace{}%
\AgdaInductiveConstructor{imapb}\AgdaSpace{}%
\AgdaBound{m}\AgdaSpace{}%
\AgdaSymbol{(}\AgdaInductiveConstructor{bin}\AgdaSpace{}%
\AgdaInductiveConstructor{mul}\AgdaSpace{}%
\AgdaSymbol{(}\AgdaInductiveConstructor{selb}\AgdaSpace{}%
\AgdaBound{m}\AgdaSpace{}%
\AgdaSymbol{(}\AgdaOperator{\AgdaFunction{↑}}\AgdaSpace{}%
\AgdaBound{a}\AgdaSymbol{)}\AgdaSpace{}%
\AgdaSymbol{(}\AgdaInductiveConstructor{var}\AgdaSpace{}%
\AgdaInductiveConstructor{v₀}\AgdaSymbol{))}\AgdaSpace{}%
\AgdaBound{b}\AgdaSymbol{)}\<%
\\
\>[0]\<%
\\
%
\>[2]\AgdaCatchallClause{\AgdaSymbol{...}}\AgdaSpace{}%
\AgdaCatchallClause{\AgdaSymbol{|}}\AgdaSpace{}%
\AgdaCatchallClause{\AgdaBound{a}}\AgdaSpace{}%
\AgdaCatchallClause{\AgdaSymbol{|}}\AgdaSpace{}%
\AgdaCatchallClause{\AgdaBound{b}}\AgdaSpace{}%
\AgdaSymbol{=}\AgdaSpace{}%
\AgdaInductiveConstructor{bin}\AgdaSpace{}%
\AgdaInductiveConstructor{mul}\AgdaSpace{}%
\AgdaBound{a}\AgdaSpace{}%
\AgdaBound{b}\<%
\\
\>[0]\<%
\\
%
\>[2]\AgdaComment{--\ XXX:\ not\ calling\ opt\ on\ e,\ as\ this\ is\ index}\<%
\\
%
\>[2]\AgdaFunction{opt}\AgdaSpace{}%
\AgdaSymbol{(}\AgdaInductiveConstructor{slide}\AgdaSpace{}%
\AgdaBound{i}\AgdaSpace{}%
\AgdaBound{pl}\AgdaSpace{}%
\AgdaBound{e}\AgdaSpace{}%
\AgdaBound{su}\AgdaSymbol{)}\AgdaSpace{}%
\AgdaKeyword{with}\AgdaSpace{}%
\AgdaFunction{opt}\AgdaSpace{}%
\AgdaBound{e}\<%
\\
%
\>[2]\AgdaSymbol{...}\AgdaSpace{}%
\AgdaSymbol{|}\AgdaSpace{}%
\AgdaInductiveConstructor{zero}\AgdaSpace{}%
\AgdaSymbol{=}\AgdaSpace{}%
\AgdaInductiveConstructor{zero}\<%
\\
%
\>[2]\AgdaCatchallClause{\AgdaSymbol{...}}\AgdaSpace{}%
\AgdaCatchallClause{\AgdaSymbol{|}}\AgdaSpace{}%
\AgdaCatchallClause{\AgdaBound{a}}\AgdaSpace{}%
\AgdaSymbol{=}\AgdaSpace{}%
\AgdaInductiveConstructor{slide}\AgdaSpace{}%
\AgdaBound{i}\AgdaSpace{}%
\AgdaBound{pl}\AgdaSpace{}%
\AgdaBound{a}\AgdaSpace{}%
\AgdaBound{su}\<%
\\
%
\>[2]\AgdaFunction{opt}\AgdaSpace{}%
\AgdaSymbol{(}\AgdaInductiveConstructor{backslide}\AgdaSpace{}%
\AgdaBound{i}\AgdaSpace{}%
\AgdaBound{e}\AgdaSpace{}%
\AgdaBound{su}\AgdaSpace{}%
\AgdaBound{pl}\AgdaSymbol{)}\AgdaSpace{}%
\AgdaKeyword{with}\AgdaSpace{}%
\AgdaFunction{opt}\AgdaSpace{}%
\AgdaBound{e}\<%
\\
%
\>[2]\AgdaSymbol{...}\AgdaSpace{}%
\AgdaSymbol{|}\AgdaSpace{}%
\AgdaInductiveConstructor{zero}\AgdaSpace{}%
\AgdaSymbol{=}\AgdaSpace{}%
\AgdaInductiveConstructor{zero}\<%
\\
%
\>[2]\AgdaCatchallClause{\AgdaSymbol{...}}\AgdaSpace{}%
\AgdaCatchallClause{\AgdaSymbol{|}}\AgdaSpace{}%
\AgdaCatchallClause{\AgdaBound{a}}\AgdaSpace{}%
\AgdaSymbol{=}\AgdaSpace{}%
\AgdaInductiveConstructor{backslide}\AgdaSpace{}%
\AgdaBound{i}\AgdaSpace{}%
\AgdaBound{a}\AgdaSpace{}%
\AgdaBound{su}\AgdaSpace{}%
\AgdaBound{pl}\<%
\\
%
\>[2]\AgdaFunction{opt}\AgdaSpace{}%
\AgdaSymbol{(}\AgdaInductiveConstructor{scaledown}\AgdaSpace{}%
\AgdaBound{x}\AgdaSpace{}%
\AgdaBound{e}\AgdaSymbol{)}\AgdaSpace{}%
\AgdaKeyword{with}\AgdaSpace{}%
\AgdaFunction{opt}\AgdaSpace{}%
\AgdaBound{e}\<%
\\
%
\>[2]\AgdaSymbol{...}\AgdaSpace{}%
\AgdaSymbol{|}\AgdaSpace{}%
\AgdaInductiveConstructor{scaledown}\AgdaSpace{}%
\AgdaBound{y}\AgdaSpace{}%
\AgdaBound{a}\AgdaSpace{}%
\AgdaSymbol{=}\AgdaSpace{}%
\AgdaInductiveConstructor{scaledown}\AgdaSpace{}%
\AgdaSymbol{(}\AgdaBound{x}\AgdaSpace{}%
\AgdaOperator{\AgdaPrimitive{ℕ.*}}\AgdaSpace{}%
\AgdaBound{y}\AgdaSymbol{)}\AgdaSpace{}%
\AgdaBound{a}\<%
\\
%
\>[2]\AgdaCatchallClause{\AgdaSymbol{...}}\AgdaSpace{}%
\AgdaCatchallClause{\AgdaSymbol{|}}\AgdaSpace{}%
\AgdaCatchallClause{\AgdaBound{a}}\AgdaSpace{}%
\AgdaSymbol{=}\AgdaSpace{}%
\AgdaInductiveConstructor{scaledown}\AgdaSpace{}%
\AgdaBound{x}\AgdaSpace{}%
\AgdaBound{a}\<%
\\
%
\>[2]\AgdaComment{--\ TODO:\ propogate\ minues\ inside\ of\ +,\ *,\ imap,\ etc.}\<%
\\
%
\>[2]\AgdaFunction{opt}\AgdaSpace{}%
\AgdaSymbol{(}\AgdaInductiveConstructor{minus}\AgdaSpace{}%
\AgdaBound{e}\AgdaSymbol{)}\AgdaSpace{}%
\AgdaKeyword{with}\AgdaSpace{}%
\AgdaFunction{opt}\AgdaSpace{}%
\AgdaBound{e}\<%
\\
%
\>[2]\AgdaSymbol{...}\AgdaSpace{}%
\AgdaSymbol{|}\AgdaSpace{}%
\AgdaInductiveConstructor{minus}\AgdaSpace{}%
\AgdaBound{a}\AgdaSpace{}%
\AgdaSymbol{=}\AgdaSpace{}%
\AgdaBound{a}\<%
\\
%
\>[2]\AgdaSymbol{...}\AgdaSpace{}%
\AgdaSymbol{|}\AgdaSpace{}%
\AgdaInductiveConstructor{imapₛ}\AgdaSpace{}%
\AgdaBound{a}\AgdaSpace{}%
\AgdaSymbol{=}\AgdaSpace{}%
\AgdaInductiveConstructor{imapₛ}\AgdaSpace{}%
\AgdaSymbol{(}\AgdaInductiveConstructor{minus}\AgdaSpace{}%
\AgdaBound{a}\AgdaSymbol{)}\<%
\\
%
\>[2]\AgdaSymbol{...}\AgdaSpace{}%
\AgdaSymbol{|}\AgdaSpace{}%
\AgdaInductiveConstructor{imap}\AgdaSpace{}%
\AgdaBound{a}\AgdaSpace{}%
\AgdaSymbol{=}\AgdaSpace{}%
\AgdaInductiveConstructor{imap}\AgdaSpace{}%
\AgdaSymbol{(}\AgdaInductiveConstructor{minus}\AgdaSpace{}%
\AgdaBound{a}\AgdaSymbol{)}\<%
\\
%
\>[2]\AgdaSymbol{...}\AgdaSpace{}%
\AgdaSymbol{|}\AgdaSpace{}%
\AgdaInductiveConstructor{imapb}\AgdaSpace{}%
\AgdaBound{m}\AgdaSpace{}%
\AgdaBound{a}\AgdaSpace{}%
\AgdaSymbol{=}\AgdaSpace{}%
\AgdaInductiveConstructor{imapb}\AgdaSpace{}%
\AgdaBound{m}\AgdaSpace{}%
\AgdaSymbol{(}\AgdaInductiveConstructor{minus}\AgdaSpace{}%
\AgdaBound{a}\AgdaSymbol{)}\<%
\\
%
\>[2]\AgdaSymbol{...}\AgdaSpace{}%
\AgdaSymbol{|}\AgdaSpace{}%
\AgdaInductiveConstructor{sum}\AgdaSpace{}%
\AgdaBound{e}\AgdaSpace{}%
\AgdaSymbol{=}\AgdaSpace{}%
\AgdaInductiveConstructor{sum}\AgdaSpace{}%
\AgdaSymbol{(}\AgdaInductiveConstructor{minus}\AgdaSpace{}%
\AgdaBound{e}\AgdaSymbol{)}\<%
\\
%
\>[2]\AgdaSymbol{...}\AgdaSpace{}%
\AgdaSymbol{|}\AgdaSpace{}%
\AgdaInductiveConstructor{bin}\AgdaSpace{}%
\AgdaInductiveConstructor{plus}\AgdaSpace{}%
\AgdaBound{a}\AgdaSpace{}%
\AgdaBound{b}\AgdaSpace{}%
\AgdaSymbol{=}\AgdaSpace{}%
\AgdaInductiveConstructor{bin}\AgdaSpace{}%
\AgdaInductiveConstructor{plus}\AgdaSpace{}%
\AgdaSymbol{(}\AgdaInductiveConstructor{minus}\AgdaSpace{}%
\AgdaBound{a}\AgdaSymbol{)}\AgdaSpace{}%
\AgdaSymbol{(}\AgdaInductiveConstructor{minus}\AgdaSpace{}%
\AgdaBound{b}\AgdaSymbol{)}\<%
\\
%
\>[2]\AgdaSymbol{...}\AgdaSpace{}%
\AgdaSymbol{|}\AgdaSpace{}%
\AgdaInductiveConstructor{bin}\AgdaSpace{}%
\AgdaInductiveConstructor{mul}\AgdaSpace{}%
\AgdaBound{a}\AgdaSpace{}%
\AgdaBound{b}\AgdaSpace{}%
\AgdaSymbol{=}\AgdaSpace{}%
\AgdaInductiveConstructor{bin}\AgdaSpace{}%
\AgdaInductiveConstructor{plus}\AgdaSpace{}%
\AgdaSymbol{(}\AgdaInductiveConstructor{minus}\AgdaSpace{}%
\AgdaBound{a}\AgdaSymbol{)}\AgdaSpace{}%
\AgdaBound{b}\<%
\\
%
\>[2]\AgdaCatchallClause{\AgdaSymbol{...}}\AgdaSpace{}%
\AgdaCatchallClause{\AgdaSymbol{|}}\AgdaSpace{}%
\AgdaCatchallClause{\AgdaBound{a}}\AgdaSpace{}%
\AgdaSymbol{=}\AgdaSpace{}%
\AgdaInductiveConstructor{minus}\AgdaSpace{}%
\AgdaBound{a}\<%
\\
%
\>[2]\AgdaFunction{opt}\AgdaSpace{}%
\AgdaSymbol{(}\AgdaInductiveConstructor{logistic}\AgdaSpace{}%
\AgdaBound{e}\AgdaSymbol{)}\AgdaSpace{}%
\AgdaKeyword{with}\AgdaSpace{}%
\AgdaFunction{opt}\AgdaSpace{}%
\AgdaBound{e}\<%
\\
%
\>[2]\AgdaSymbol{...}\AgdaSpace{}%
\AgdaSymbol{|}\AgdaSpace{}%
\AgdaInductiveConstructor{imapₛ}\AgdaSpace{}%
\AgdaBound{a}\AgdaSpace{}%
\AgdaSymbol{=}\AgdaSpace{}%
\AgdaInductiveConstructor{imapₛ}\AgdaSpace{}%
\AgdaSymbol{(}\AgdaInductiveConstructor{logistic}\AgdaSpace{}%
\AgdaBound{a}\AgdaSymbol{)}\<%
\\
%
\>[2]\AgdaSymbol{...}\AgdaSpace{}%
\AgdaSymbol{|}\AgdaSpace{}%
\AgdaInductiveConstructor{imap}\AgdaSpace{}%
\AgdaBound{a}\AgdaSpace{}%
\AgdaSymbol{=}\AgdaSpace{}%
\AgdaInductiveConstructor{imap}\AgdaSpace{}%
\AgdaSymbol{(}\AgdaInductiveConstructor{logistic}\AgdaSpace{}%
\AgdaBound{a}\AgdaSymbol{)}\<%
\\
%
\>[2]\AgdaCatchallClause{\AgdaSymbol{...}}\AgdaSpace{}%
\AgdaCatchallClause{\AgdaSymbol{|}}\AgdaSpace{}%
\AgdaCatchallClause{\AgdaBound{a}}\AgdaSpace{}%
\AgdaSymbol{=}\AgdaSpace{}%
\AgdaInductiveConstructor{logistic}\AgdaSpace{}%
\AgdaBound{a}\<%
\\
%
\\[\AgdaEmptyExtraSkip]%
%
\\[\AgdaEmptyExtraSkip]%
%
\>[2]\AgdaFunction{multiopt}\AgdaSpace{}%
\AgdaSymbol{:}\AgdaSpace{}%
\AgdaDatatype{ℕ}\AgdaSpace{}%
\AgdaSymbol{→}\AgdaSpace{}%
\AgdaDatatype{E}\AgdaSpace{}%
\AgdaGeneralizable{Γ}\AgdaSpace{}%
\AgdaGeneralizable{is}\AgdaSpace{}%
\AgdaSymbol{→}\AgdaSpace{}%
\AgdaDatatype{E}\AgdaSpace{}%
\AgdaGeneralizable{Γ}\AgdaSpace{}%
\AgdaGeneralizable{is}\<%
\\
%
\>[2]\AgdaFunction{multiopt}\AgdaSpace{}%
\AgdaInductiveConstructor{zero}\AgdaSpace{}%
\AgdaBound{e}\AgdaSpace{}%
\AgdaSymbol{=}\AgdaSpace{}%
\AgdaBound{e}\<%
\\
%
\>[2]\AgdaFunction{multiopt}\AgdaSpace{}%
\AgdaSymbol{(}\AgdaInductiveConstructor{suc}\AgdaSpace{}%
\AgdaBound{n}\AgdaSymbol{)}\AgdaSpace{}%
\AgdaBound{e}\AgdaSpace{}%
\AgdaSymbol{=}\AgdaSpace{}%
\AgdaFunction{opt}\AgdaSpace{}%
\AgdaSymbol{(}\AgdaFunction{multiopt}\AgdaSpace{}%
\AgdaBound{n}\AgdaSpace{}%
\AgdaBound{e}\AgdaSymbol{)}\<%
\\
%
\\[\AgdaEmptyExtraSkip]%
%
\>[2]\AgdaKeyword{module}\AgdaSpace{}%
\AgdaModule{TryOpt}\AgdaSpace{}%
\AgdaKeyword{where}\<%
\end{code}

Let us observe optimisation effects when computing derivatives of
the scalar dot-product defined as follows.
\begin{code}%
\>[2][@{}l@{\AgdaIndent{1}}]%
\>[4]\AgdaFunction{dotp}\AgdaSpace{}%
\AgdaSymbol{:}\AgdaSpace{}%
\AgdaDatatype{E}\AgdaSpace{}%
\AgdaGeneralizable{Γ}\AgdaSpace{}%
\AgdaSymbol{(}\AgdaInductiveConstructor{ar}\AgdaSpace{}%
\AgdaGeneralizable{s}\AgdaSymbol{)}\AgdaSpace{}%
\AgdaSymbol{→}\AgdaSpace{}%
\AgdaDatatype{E}\AgdaSpace{}%
\AgdaGeneralizable{Γ}\AgdaSpace{}%
\AgdaSymbol{(}\AgdaInductiveConstructor{ar}\AgdaSpace{}%
\AgdaGeneralizable{s}\AgdaSymbol{)}\AgdaSpace{}%
\AgdaSymbol{→}\AgdaSpace{}%
\AgdaDatatype{E}\AgdaSpace{}%
\AgdaGeneralizable{Γ}\AgdaSpace{}%
\AgdaSymbol{(}\AgdaInductiveConstructor{ar}\AgdaSpace{}%
\AgdaFunction{unit}\AgdaSymbol{)}\<%
\\
%
\>[4]\AgdaFunction{dotp}\AgdaSpace{}%
\AgdaBound{a}\AgdaSpace{}%
\AgdaBound{b}\AgdaSpace{}%
\AgdaSymbol{=}\AgdaSpace{}%
\AgdaFunction{Sum}\AgdaSpace{}%
\AgdaSymbol{λ}\AgdaSpace{}%
\AgdaBound{i}\AgdaSpace{}%
\AgdaSymbol{→}\AgdaSpace{}%
\AgdaInductiveConstructor{selₛ}\AgdaSpace{}%
\AgdaSymbol{(}\AgdaOperator{\AgdaFunction{↑}}\AgdaSpace{}%
\AgdaBound{a}\AgdaSymbol{)}\AgdaSpace{}%
\AgdaBound{i}\AgdaSpace{}%
\AgdaOperator{\AgdaInductiveConstructor{⊠}}\AgdaSpace{}%
\AgdaInductiveConstructor{selₛ}\AgdaSpace{}%
\AgdaSymbol{(}\AgdaOperator{\AgdaFunction{↑}}\AgdaSpace{}%
\AgdaBound{b}\AgdaSymbol{)}\AgdaSpace{}%
\AgdaBound{i}\<%
\end{code}
\begin{code}[hide]%
%
\>[4]\AgdaFunction{C}\AgdaSpace{}%
\AgdaSymbol{:}\AgdaSpace{}%
\AgdaDatatype{Ctx}\<%
\\
%
\>[4]\AgdaFunction{a}\AgdaSpace{}%
\AgdaSymbol{:}\AgdaSpace{}%
\AgdaDatatype{E}\AgdaSpace{}%
\AgdaFunction{C}\AgdaSpace{}%
\AgdaSymbol{\AgdaUnderscore{}}\<%
\\
%
\>[4]\AgdaFunction{b}\AgdaSpace{}%
\AgdaSymbol{:}\AgdaSpace{}%
\AgdaDatatype{E}\AgdaSpace{}%
\AgdaFunction{C}\AgdaSpace{}%
\AgdaSymbol{\AgdaUnderscore{}}\<%
\\
%
\>[4]\AgdaFunction{seed}\AgdaSpace{}%
\AgdaSymbol{:}\AgdaSpace{}%
\AgdaDatatype{E}\AgdaSpace{}%
\AgdaFunction{C}\AgdaSpace{}%
\AgdaSymbol{\AgdaUnderscore{}}\<%
\end{code}
We define the context \AF{C} where two top variables are of 5-element vector shape
and the last variable (\AC{v₂}) is of scalar shape.  We bind these variables to Agda
variables for convenience.
\begin{code}%
%
\>[4]\AgdaFunction{C}\AgdaSpace{}%
\AgdaSymbol{=}\AgdaSpace{}%
\AgdaInductiveConstructor{ε}\AgdaSpace{}%
\AgdaOperator{\AgdaInductiveConstructor{▹}}%
\>[13]\AgdaInductiveConstructor{ar}\AgdaSpace{}%
\AgdaSymbol{(}\AgdaInductiveConstructor{ι}\AgdaSpace{}%
\AgdaNumber{1}\AgdaSymbol{)}%
\>[28]\AgdaOperator{\AgdaInductiveConstructor{▹}}%
\>[31]\AgdaInductiveConstructor{ar}\AgdaSpace{}%
\AgdaSymbol{(}\AgdaInductiveConstructor{ι}\AgdaSpace{}%
\AgdaNumber{5}\AgdaSymbol{)}%
\>[43]\AgdaOperator{\AgdaInductiveConstructor{▹}}%
\>[46]\AgdaInductiveConstructor{ar}\AgdaSpace{}%
\AgdaSymbol{(}\AgdaInductiveConstructor{ι}\AgdaSpace{}%
\AgdaNumber{5}\AgdaSymbol{);}\<%
\\
%
\>[13]\AgdaFunction{seed}\AgdaSpace{}%
\AgdaSymbol{=}\AgdaSpace{}%
\AgdaInductiveConstructor{var}\AgdaSpace{}%
\AgdaInductiveConstructor{v₂}%
\>[28]\AgdaSymbol{;}%
\>[31]\AgdaFunction{a}\AgdaSpace{}%
\AgdaSymbol{=}\AgdaSpace{}%
\AgdaInductiveConstructor{var}\AgdaSpace{}%
\AgdaInductiveConstructor{v₁}%
\>[43]\AgdaSymbol{;}%
\>[46]\AgdaFunction{b}%
\>[49]\AgdaSymbol{=}\AgdaSpace{}%
\AgdaInductiveConstructor{var}\AgdaSpace{}%
\AgdaInductiveConstructor{v₀}\<%
\end{code}
\begin{code}[hide]%
%
\>[4]\AgdaFunction{∂a}%
\>[11]\AgdaSymbol{=}\AgdaSpace{}%
\AgdaFunction{env-ix}\AgdaSpace{}%
\AgdaSymbol{\{}\AgdaFunction{C}\AgdaSymbol{\}}\AgdaSpace{}%
\AgdaSymbol{(}\AgdaFunction{∇}\AgdaSpace{}%
\AgdaSymbol{\{}\AgdaFunction{C}\AgdaSymbol{\}}\AgdaSpace{}%
\AgdaSymbol{(}\AgdaFunction{dotp}\AgdaSpace{}%
\AgdaFunction{a}\AgdaSpace{}%
\AgdaFunction{b}\AgdaSymbol{)}\AgdaSpace{}%
\AgdaFunction{seed}\AgdaSpace{}%
\AgdaSymbol{(}\AgdaFunction{env-zero}\AgdaSpace{}%
\AgdaSymbol{\{}\AgdaFunction{C}\AgdaSymbol{\}))}\AgdaSpace{}%
\AgdaInductiveConstructor{v₁}\<%
\\
%
\>[4]\AgdaFunction{∂a′}%
\>[11]\AgdaSymbol{=}\AgdaSpace{}%
\AgdaFunction{multiopt}\AgdaSpace{}%
\AgdaNumber{3}\AgdaSpace{}%
\AgdaFunction{∂a}\<%
\end{code}
We compute the derivatives of \AF{dotp a b} with seed \AF{seed} and we inspect
the $a$-th position in the returned environment that we call \AF{∂a}.  Then we repeatedly
apply \AF{opt} (three times) to \AF{∂a} and save it in \AF{∂a′}.  We force Agda to
verify that the content of the variables is as follows:
\begin{code}%
%
\>[4]\AgdaFunction{non-opt}%
\>[14]\AgdaSymbol{:}\AgdaSpace{}%
\AgdaFunction{∂a}%
\>[21]\AgdaOperator{\AgdaDatatype{≡}}\AgdaSpace{}%
\AgdaSymbol{(}\AgdaFunction{Sum}\AgdaSpace{}%
\AgdaSymbol{λ}\AgdaSpace{}%
\AgdaBound{i}\AgdaSpace{}%
\AgdaSymbol{→}\AgdaSpace{}%
\AgdaInductiveConstructor{zero}\AgdaSpace{}%
\AgdaOperator{\AgdaInductiveConstructor{⊞}}\AgdaSpace{}%
\AgdaFunction{Imapₛ}\AgdaSpace{}%
\AgdaSymbol{λ}\AgdaSpace{}%
\AgdaBound{j}\AgdaSpace{}%
\AgdaSymbol{→}\AgdaSpace{}%
\AgdaInductiveConstructor{zero-but}\AgdaSpace{}%
\AgdaBound{j}\AgdaSpace{}%
\AgdaSymbol{(}\AgdaOperator{\AgdaFunction{↑}}\AgdaSpace{}%
\AgdaBound{i}\AgdaSymbol{)}\AgdaSpace{}%
\AgdaSymbol{(}\AgdaOperator{\AgdaFunction{↑↑}}\AgdaSpace{}%
\AgdaFunction{seed}\AgdaSpace{}%
\AgdaOperator{\AgdaInductiveConstructor{⊠}}\AgdaSpace{}%
\AgdaInductiveConstructor{selₛ}\AgdaSpace{}%
\AgdaSymbol{(}\AgdaOperator{\AgdaFunction{↑↑}}\AgdaSpace{}%
\AgdaFunction{b}\AgdaSymbol{)}\AgdaSpace{}%
\AgdaSymbol{(}\AgdaOperator{\AgdaFunction{↑}}\AgdaSpace{}%
\AgdaBound{i}\AgdaSymbol{)))}\AgdaSpace{}%
\AgdaOperator{\AgdaInductiveConstructor{⊞}}\AgdaSpace{}%
\AgdaInductiveConstructor{zero}\<%
\\
%
\>[4]\AgdaFunction{with-opt}%
\>[14]\AgdaSymbol{:}\AgdaSpace{}%
\AgdaFunction{∂a′}%
\>[21]\AgdaOperator{\AgdaDatatype{≡}}\AgdaSpace{}%
\AgdaFunction{Imapₛ}\AgdaSpace{}%
\AgdaSymbol{λ}\AgdaSpace{}%
\AgdaBound{i}\AgdaSpace{}%
\AgdaSymbol{→}\AgdaSpace{}%
\AgdaSymbol{(}\AgdaOperator{\AgdaFunction{↑}}\AgdaSpace{}%
\AgdaFunction{seed}\AgdaSpace{}%
\AgdaOperator{\AgdaInductiveConstructor{⊠}}\AgdaSpace{}%
\AgdaInductiveConstructor{selₛ}\AgdaSpace{}%
\AgdaSymbol{(}\AgdaOperator{\AgdaFunction{↑}}\AgdaSpace{}%
\AgdaFunction{b}\AgdaSymbol{)}\AgdaSpace{}%
\AgdaBound{i}\AgdaSymbol{)}\<%
\end{code}
\begin{code}[hide]%
%
\>[4]\AgdaFunction{non-opt}\AgdaSpace{}%
\AgdaSymbol{=}\AgdaSpace{}%
\AgdaInductiveConstructor{refl}\<%
\\
%
\>[4]\AgdaFunction{with-opt}\AgdaSpace{}%
\AgdaSymbol{=}\AgdaSpace{}%
\AgdaInductiveConstructor{refl}\<%
\\
\>[0]\AgdaComment{--\ open\ Lang}\<%
\\
\>[0]\AgdaComment{--\ open\ SubWk}\<%
\end{code}
As it can be seen, \AF{∂a} sums-up the arrays, where only one element is non-zero at
every iteration.  Such a computation is highly inefficient when executed directly,
as it needs to compute all the inner arrays before summing them up.  However, the
optimised version correctly rewrites \AF{∂a} into \AC{imap} that multiplies
the \AB{seed} by $b$, which is the expected answer.  This reduces complexity
of the expression form squared to linear.

\subsection{Extraction}

Extraction from \AF{E} into SaC translates constructors of \AF{E} into
corresponding SaC expressions or function calls.  The translation starts with
a definition of an environment (\AF{SEnv} \AB{Γ}) that assigns SaC variable names
to all positions in \AB{Γ}.  The assumption here is that when we compile
expressions in context \AF{Γ}, variable names of the corresponding shapes are
available in SaC.

Next, we have to take care of shapes.  Array shapes in \AF{E} are binary trees,
but array shapes in SaC are 1-dimensional arrays (flattened binary trees).
When some expression in \AF{E} is of product shape, we usually have to
supply left or right subshapes of the product to SaC. These are always available
through implicit arguments of \AF{E} constructors. Assuming that by the
time we come to extraction, all the \AF{E} shapes are constants, we can
always generate shape expressions in SaC.  This is implemented in \AF{show-shape}.
Relaxing the assumption about constant shapes is possible but requires
extension of \AF{E} so that we can always bind the shapes used in \AF{E}
to some expressions in SaC.

We also need a source of fresh variables so that we can generate indices
for \AC{imap} expressions.  We define a stateful function \AF{iv} that
generates a fresh index variable.  

Extraction is given by \AF{to-sac} that translates the expression $e$ in
the environment $\rho$.  The function is stateful so that we can generate
fresh variables when needed.

The definitions of \AF{SEnv}, \AF{iv}, {\AF{show-shape}, and \AF{to-sac} follow.
\begin{code}[hide]%
\>[0]\AgdaKeyword{module}\AgdaSpace{}%
\AgdaModule{Sac}\AgdaSpace{}%
\AgdaKeyword{where}\<%
\\
\>[0][@{}l@{\AgdaIndent{0}}]%
\>[2]\AgdaKeyword{open}\AgdaSpace{}%
\AgdaKeyword{import}\AgdaSpace{}%
\AgdaModule{Data.Unit}\<%
\\
%
\>[2]\AgdaKeyword{open}\AgdaSpace{}%
\AgdaKeyword{import}\AgdaSpace{}%
\AgdaModule{Data.Product}\<%
\\
%
\>[2]\AgdaKeyword{open}\AgdaSpace{}%
\AgdaKeyword{import}\AgdaSpace{}%
\AgdaModule{Data.List}\AgdaSpace{}%
\AgdaSymbol{as}\AgdaSpace{}%
\AgdaModule{L}\AgdaSpace{}%
\AgdaKeyword{using}\AgdaSpace{}%
\AgdaSymbol{(}\AgdaDatatype{List}\AgdaSymbol{;}\AgdaSpace{}%
\AgdaInductiveConstructor{[]}\AgdaSymbol{;}\AgdaSpace{}%
\AgdaOperator{\AgdaInductiveConstructor{\AgdaUnderscore{}∷\AgdaUnderscore{}}}\AgdaSymbol{;}\AgdaSpace{}%
\AgdaOperator{\AgdaFunction{\AgdaUnderscore{}++\AgdaUnderscore{}}}\AgdaSymbol{)}\<%
\\
%
\>[2]\AgdaKeyword{open}\AgdaSpace{}%
\AgdaKeyword{import}\AgdaSpace{}%
\AgdaModule{Data.Nat}\AgdaSpace{}%
\AgdaSymbol{as}\AgdaSpace{}%
\AgdaModule{ℕ}\AgdaSpace{}%
\AgdaKeyword{using}\AgdaSpace{}%
\AgdaSymbol{(}\AgdaDatatype{ℕ}\AgdaSymbol{;}\AgdaSpace{}%
\AgdaInductiveConstructor{zero}\AgdaSymbol{;}\AgdaSpace{}%
\AgdaInductiveConstructor{suc}\AgdaSymbol{)}\<%
\\
%
\>[2]\AgdaKeyword{open}\AgdaSpace{}%
\AgdaKeyword{import}\AgdaSpace{}%
\AgdaModule{Data.Nat.Show}\AgdaSpace{}%
\AgdaKeyword{using}\AgdaSpace{}%
\AgdaSymbol{()}\AgdaSpace{}%
\AgdaKeyword{renaming}\AgdaSpace{}%
\AgdaSymbol{(}\AgdaFunction{show}\AgdaSpace{}%
\AgdaSymbol{to}\AgdaSpace{}%
\AgdaFunction{show-nat}\AgdaSymbol{)}\<%
\\
%
\>[2]\AgdaKeyword{open}\AgdaSpace{}%
\AgdaKeyword{import}\AgdaSpace{}%
\AgdaModule{Data.String}\AgdaSpace{}%
\AgdaKeyword{hiding}\AgdaSpace{}%
\AgdaSymbol{(}\AgdaOperator{\AgdaFunction{\AgdaUnderscore{}++\AgdaUnderscore{}}}\AgdaSymbol{)}\<%
\\
%
\>[2]\AgdaKeyword{open}\AgdaSpace{}%
\AgdaKeyword{import}\AgdaSpace{}%
\AgdaModule{Text.Printf}\<%
\\
%
\>[2]\AgdaKeyword{open}\AgdaSpace{}%
\AgdaKeyword{import}\AgdaSpace{}%
\AgdaModule{Category.Monad.State}\AgdaSpace{}%
\AgdaComment{--using\ (State;\ StateMonad;\ RawMonadState)}\<%
\\
%
\>[2]\AgdaKeyword{open}\AgdaSpace{}%
\AgdaKeyword{import}\AgdaSpace{}%
\AgdaModule{Category.Monad}\AgdaSpace{}%
\AgdaKeyword{using}\AgdaSpace{}%
\AgdaSymbol{(}\AgdaFunction{RawMonad}\AgdaSymbol{)}\<%
\\
%
\>[2]\AgdaComment{--open\ RawMonad\ \{\{...\}\}\ public}\<%
\\
%
\>[2]\AgdaKeyword{open}\AgdaSpace{}%
\AgdaModule{RawMonadState}\AgdaSpace{}%
\AgdaSymbol{\{\{...\}\}}\AgdaSpace{}%
\AgdaKeyword{public}\<%
\\
%
\>[2]\AgdaKeyword{open}\AgdaSpace{}%
\AgdaModule{Lang}\<%
\\
%
\>[2]\AgdaKeyword{open}\AgdaSpace{}%
\AgdaModule{Array}\AgdaSpace{}%
\AgdaKeyword{hiding}\AgdaSpace{}%
\AgdaSymbol{(}\AgdaFunction{sum}\AgdaSymbol{;}\AgdaSpace{}%
\AgdaFunction{slide}\AgdaSymbol{;}\AgdaSpace{}%
\AgdaFunction{backslide}\AgdaSymbol{)}\<%
\\
%
\>[2]\AgdaKeyword{open}\AgdaSpace{}%
\AgdaModule{SubWk}\<%
\\
%
\\[\AgdaEmptyExtraSkip]%
%
\>[2]\AgdaKeyword{instance}\<%
\\
\>[2][@{}l@{\AgdaIndent{0}}]%
\>[4]\AgdaComment{--\ stateMon\ :\ ∀\ \{S\ :\ Set\}\ →\ RawMonad\ (State\ S)}\<%
\\
%
\>[4]\AgdaComment{--\ stateMon\ \{S\}\ =\ StateMonad\ S}\<%
\\
%
\\[\AgdaEmptyExtraSkip]%
%
\>[4]\AgdaFunction{stateMonState}\AgdaSpace{}%
\AgdaSymbol{:}\AgdaSpace{}%
\AgdaSymbol{∀}\AgdaSpace{}%
\AgdaSymbol{\{}\AgdaBound{S}\AgdaSpace{}%
\AgdaSymbol{:}\AgdaSpace{}%
\AgdaPrimitive{Set}\AgdaSymbol{\}}\AgdaSpace{}%
\AgdaSymbol{→}\AgdaSpace{}%
\AgdaFunction{RawMonadState}\AgdaSpace{}%
\AgdaBound{S}\AgdaSpace{}%
\AgdaSymbol{(}\AgdaFunction{State}\AgdaSpace{}%
\AgdaBound{S}\AgdaSymbol{)}\<%
\\
%
\>[4]\AgdaFunction{stateMonState}\AgdaSpace{}%
\AgdaSymbol{\{}\AgdaBound{S}\AgdaSymbol{\}}\AgdaSpace{}%
\AgdaSymbol{=}\AgdaSpace{}%
\AgdaFunction{StateMonadState}\AgdaSpace{}%
\AgdaBound{S}\<%
\end{code}
\begin{mathpar}
\codeblock{\begin{code}%
%
\>[2]\AgdaFunction{SEnv}\AgdaSpace{}%
\AgdaSymbol{:}\AgdaSpace{}%
\AgdaDatatype{Ctx}\AgdaSpace{}%
\AgdaSymbol{→}\AgdaSpace{}%
\AgdaPrimitive{Set}\<%
\\
%
\>[2]\AgdaFunction{SEnv}\AgdaSpace{}%
\AgdaInductiveConstructor{ε}%
\>[17]\AgdaSymbol{=}\AgdaSpace{}%
\AgdaRecord{⊤}\<%
\\
%
\>[2]\AgdaFunction{SEnv}\AgdaSpace{}%
\AgdaSymbol{(}\AgdaBound{Γ}\AgdaSpace{}%
\AgdaOperator{\AgdaInductiveConstructor{▹}}\AgdaSpace{}%
\AgdaBound{is}\AgdaSymbol{)}%
\>[17]\AgdaSymbol{=}\AgdaSpace{}%
\AgdaFunction{SEnv}\AgdaSpace{}%
\AgdaBound{Γ}\AgdaSpace{}%
\AgdaOperator{\AgdaFunction{×}}\AgdaSpace{}%
\AgdaPostulate{String}\<%
\end{code}}
\and
\codeblock{\begin{code}%
%
\>[2]\AgdaFunction{iv}\AgdaSpace{}%
\AgdaSymbol{:}\AgdaSpace{}%
\AgdaDatatype{S}\AgdaSpace{}%
\AgdaSymbol{→}\AgdaSpace{}%
\AgdaFunction{State}\AgdaSpace{}%
\AgdaDatatype{ℕ}\AgdaSpace{}%
\AgdaPostulate{String}\<%
\\
%
\>[2]\AgdaFunction{iv}\AgdaSpace{}%
\AgdaBound{s}\AgdaSpace{}%
\AgdaSymbol{=}\AgdaSpace{}%
\AgdaKeyword{do}%
\>[13]\AgdaBound{v}\AgdaSpace{}%
\AgdaOperator{\AgdaFunction{←}}\AgdaSpace{}%
\AgdaField{get}\<%
\\
%
\>[13]\AgdaFunction{modify}\AgdaSpace{}%
\AgdaInductiveConstructor{suc}\<%
\\
%
\>[13]\AgdaFunction{return}\AgdaSpace{}%
\AgdaOperator{\AgdaFunction{\$}}\AgdaSpace{}%
\AgdaFunction{printf}\AgdaSpace{}%
\AgdaString{"x\%u"}\AgdaSpace{}%
\AgdaBound{v}\<%
\end{code}
\begin{code}[hide]%
\>[0]\<%
\\
%
\>[2]\AgdaFunction{lookup}\AgdaSpace{}%
\AgdaSymbol{:}\AgdaSpace{}%
\AgdaGeneralizable{is}\AgdaSpace{}%
\AgdaOperator{\AgdaDatatype{∈}}\AgdaSpace{}%
\AgdaGeneralizable{Γ}\AgdaSpace{}%
\AgdaSymbol{→}\AgdaSpace{}%
\AgdaFunction{SEnv}\AgdaSpace{}%
\AgdaGeneralizable{Γ}\AgdaSpace{}%
\AgdaSymbol{→}\AgdaSpace{}%
\AgdaPostulate{String}\<%
\\
%
\>[2]\AgdaFunction{lookup}\AgdaSpace{}%
\AgdaInductiveConstructor{v₀}%
\>[17]\AgdaSymbol{(}\AgdaBound{ρ}\AgdaSpace{}%
\AgdaOperator{\AgdaInductiveConstructor{,}}\AgdaSpace{}%
\AgdaBound{e}\AgdaSymbol{)}\AgdaSpace{}%
\AgdaSymbol{=}\AgdaSpace{}%
\AgdaBound{e}\<%
\\
%
\>[2]\AgdaFunction{lookup}\AgdaSpace{}%
\AgdaSymbol{(}\AgdaInductiveConstructor{vₛ}\AgdaSpace{}%
\AgdaBound{x}\AgdaSymbol{)}%
\>[17]\AgdaSymbol{(}\AgdaBound{ρ}\AgdaSpace{}%
\AgdaOperator{\AgdaInductiveConstructor{,}}\AgdaSpace{}%
\AgdaBound{e}\AgdaSymbol{)}\AgdaSpace{}%
\AgdaSymbol{=}\AgdaSpace{}%
\AgdaFunction{lookup}\AgdaSpace{}%
\AgdaBound{x}\AgdaSpace{}%
\AgdaBound{ρ}\<%
\\
%
\\[\AgdaEmptyExtraSkip]%
%
\\[\AgdaEmptyExtraSkip]%
%
\>[2]\AgdaComment{--\ show-shape\ :\ S\ →\ String}\<%
\\
%
\>[2]\AgdaComment{--\ show-shape\ (ι\ x)\ =\ show-nat\ x}\<%
\\
%
\>[2]\AgdaComment{--\ show-shape\ (s\ S.⊗\ p)\ =\ printf\ "⟨\%s,\ \%s⟩"\ (show-shape\ s)\ (show-shape\ p)}\<%
\\
%
\\[\AgdaEmptyExtraSkip]%
%
\>[2]\AgdaFunction{fresh-var}\AgdaSpace{}%
\AgdaSymbol{:}\AgdaSpace{}%
\AgdaDatatype{ℕ}\AgdaSpace{}%
\AgdaSymbol{→}\AgdaSpace{}%
\AgdaPostulate{String}\<%
\\
%
\>[2]\AgdaFunction{fresh-var}\AgdaSpace{}%
\AgdaBound{n}\AgdaSpace{}%
\AgdaSymbol{=}\AgdaSpace{}%
\AgdaFunction{printf}\AgdaSpace{}%
\AgdaString{"x\%u"}\AgdaSpace{}%
\AgdaBound{n}\<%
\\
%
\\[\AgdaEmptyExtraSkip]%
%
\>[2]\AgdaFunction{bop}\AgdaSpace{}%
\AgdaSymbol{:}\AgdaSpace{}%
\AgdaDatatype{Bop}\AgdaSpace{}%
\AgdaSymbol{->}\AgdaSpace{}%
\AgdaPostulate{String}\<%
\\
%
\>[2]\AgdaFunction{bop}\AgdaSpace{}%
\AgdaInductiveConstructor{plus}\AgdaSpace{}%
\AgdaSymbol{=}\AgdaSpace{}%
\AgdaString{"+"}\<%
\\
%
\>[2]\AgdaFunction{bop}\AgdaSpace{}%
\AgdaInductiveConstructor{mul}\AgdaSpace{}%
\AgdaSymbol{=}\AgdaSpace{}%
\AgdaString{"*"}\<%
\\
%
\\[\AgdaEmptyExtraSkip]%
%
\>[2]\AgdaFunction{dim}\AgdaSpace{}%
\AgdaSymbol{:}\AgdaSpace{}%
\AgdaDatatype{S}\AgdaSpace{}%
\AgdaSymbol{→}\AgdaSpace{}%
\AgdaDatatype{ℕ}\<%
\\
%
\>[2]\AgdaFunction{dim}\AgdaSpace{}%
\AgdaSymbol{(}\AgdaInductiveConstructor{ι}\AgdaSpace{}%
\AgdaSymbol{\AgdaUnderscore{})}\AgdaSpace{}%
\AgdaSymbol{=}\AgdaSpace{}%
\AgdaNumber{1}\<%
\\
%
\>[2]\AgdaFunction{dim}\AgdaSpace{}%
\AgdaSymbol{(}\AgdaBound{s}\AgdaSpace{}%
\AgdaOperator{\AgdaInductiveConstructor{Array.⊗}}\AgdaSpace{}%
\AgdaBound{p}\AgdaSymbol{)}\AgdaSpace{}%
\AgdaSymbol{=}\AgdaSpace{}%
\AgdaFunction{dim}\AgdaSpace{}%
\AgdaBound{s}\AgdaSpace{}%
\AgdaOperator{\AgdaPrimitive{ℕ.+}}\AgdaSpace{}%
\AgdaFunction{dim}\AgdaSpace{}%
\AgdaBound{p}\<%
\\
%
\\[\AgdaEmptyExtraSkip]%
%
\>[2]\AgdaFunction{ivl}\AgdaSpace{}%
\AgdaSymbol{:}\AgdaSpace{}%
\AgdaDatatype{S}\AgdaSpace{}%
\AgdaSymbol{→}\AgdaSpace{}%
\AgdaFunction{State}\AgdaSpace{}%
\AgdaDatatype{ℕ}\AgdaSpace{}%
\AgdaSymbol{(}\AgdaDatatype{List}\AgdaSpace{}%
\AgdaPostulate{String}\AgdaSymbol{)}\<%
\\
%
\>[2]\AgdaFunction{ivl}\AgdaSpace{}%
\AgdaSymbol{(}\AgdaInductiveConstructor{ι}\AgdaSpace{}%
\AgdaSymbol{\AgdaUnderscore{})}\AgdaSpace{}%
\AgdaSymbol{=}\AgdaSpace{}%
\AgdaKeyword{do}\<%
\\
\>[2][@{}l@{\AgdaIndent{0}}]%
\>[4]\AgdaBound{v}\AgdaSpace{}%
\AgdaOperator{\AgdaFunction{←}}\AgdaSpace{}%
\AgdaField{get}\<%
\\
%
\>[4]\AgdaFunction{modify}\AgdaSpace{}%
\AgdaInductiveConstructor{suc}\<%
\\
%
\>[4]\AgdaFunction{return}\AgdaSpace{}%
\AgdaOperator{\AgdaFunction{\$}}\AgdaSpace{}%
\AgdaSymbol{(}\AgdaFunction{fresh-var}\AgdaSpace{}%
\AgdaBound{v}\AgdaSpace{}%
\AgdaOperator{\AgdaInductiveConstructor{∷}}\AgdaSpace{}%
\AgdaInductiveConstructor{[]}\AgdaSymbol{)}\<%
\\
%
\>[2]\AgdaFunction{ivl}\AgdaSpace{}%
\AgdaSymbol{(}\AgdaBound{s}\AgdaSpace{}%
\AgdaOperator{\AgdaInductiveConstructor{S.⊗}}\AgdaSpace{}%
\AgdaBound{p}\AgdaSymbol{)}\AgdaSpace{}%
\AgdaSymbol{=}\AgdaSpace{}%
\AgdaKeyword{do}\<%
\\
\>[2][@{}l@{\AgdaIndent{0}}]%
\>[4]\AgdaBound{l}\AgdaSpace{}%
\AgdaOperator{\AgdaFunction{←}}\AgdaSpace{}%
\AgdaFunction{ivl}\AgdaSpace{}%
\AgdaBound{s}\<%
\\
%
\>[4]\AgdaBound{r}\AgdaSpace{}%
\AgdaOperator{\AgdaFunction{←}}\AgdaSpace{}%
\AgdaFunction{ivl}\AgdaSpace{}%
\AgdaBound{p}\<%
\\
%
\>[4]\AgdaFunction{return}\AgdaSpace{}%
\AgdaOperator{\AgdaFunction{\$}}\AgdaSpace{}%
\AgdaBound{l}\AgdaSpace{}%
\AgdaOperator{\AgdaFunction{L.++}}\AgdaSpace{}%
\AgdaBound{r}\<%
\\
\>[0]\<%
\\
%
\>[2]\AgdaComment{--iv\ s\ =\ printf\ "[\%s]"\ ∘\ intersperse\ ",\ "\ <\$>\ ivl\ s}\<%
\end{code}}
\and
\codeblock{\begin{code}%
%
\>[2]\AgdaFunction{show-shape}\AgdaSpace{}%
\AgdaSymbol{:}\AgdaSpace{}%
\AgdaDatatype{S}\AgdaSpace{}%
\AgdaSymbol{→}\AgdaSpace{}%
\AgdaPostulate{String}\<%
\\
%
\>[2]\AgdaFunction{show-shape}\AgdaSpace{}%
\AgdaBound{s}%
\>[1867I]\AgdaSymbol{=}\AgdaSpace{}%
\AgdaFunction{printf}\AgdaSpace{}%
\AgdaString{"[\%s]"}\<%
\\
\>[.][@{}l@{}]\<[1867I]%
\>[15]\AgdaOperator{\AgdaFunction{\$}}\AgdaSpace{}%
\AgdaFunction{intersperse}\AgdaSpace{}%
\AgdaString{",\ "}\<%
\\
%
\>[15]\AgdaOperator{\AgdaFunction{\$}}\AgdaSpace{}%
\AgdaFunction{go}\AgdaSpace{}%
\AgdaBound{s}\<%
\\
\>[2][@{}l@{\AgdaIndent{0}}]%
\>[4]\AgdaKeyword{where}\<%
\\
\>[4][@{}l@{\AgdaIndent{0}}]%
\>[6]\AgdaFunction{go}\AgdaSpace{}%
\AgdaSymbol{:}\AgdaSpace{}%
\AgdaDatatype{S}\AgdaSpace{}%
\AgdaSymbol{→}\AgdaSpace{}%
\AgdaDatatype{List}\AgdaSpace{}%
\AgdaPostulate{String}\<%
\\
%
\>[6]\AgdaFunction{go}\AgdaSpace{}%
\AgdaSymbol{(}\AgdaInductiveConstructor{ι}\AgdaSpace{}%
\AgdaBound{x}\AgdaSymbol{)}%
\>[18]\AgdaSymbol{=}\AgdaSpace{}%
\AgdaFunction{show-nat}\AgdaSpace{}%
\AgdaBound{x}\AgdaSpace{}%
\AgdaOperator{\AgdaInductiveConstructor{∷}}\AgdaSpace{}%
\AgdaInductiveConstructor{[]}\<%
\\
%
\>[6]\AgdaFunction{go}\AgdaSpace{}%
\AgdaSymbol{(}\AgdaBound{s}\AgdaSpace{}%
\AgdaOperator{\AgdaInductiveConstructor{⊗}}\AgdaSpace{}%
\AgdaBound{p}\AgdaSymbol{)}%
\>[18]\AgdaSymbol{=}\AgdaSpace{}%
\AgdaFunction{go}\AgdaSpace{}%
\AgdaBound{s}\AgdaSpace{}%
\AgdaOperator{\AgdaFunction{++}}\AgdaSpace{}%
\AgdaFunction{go}\AgdaSpace{}%
\AgdaBound{p}\<%
\end{code}}
\and
\codeblock{\begin{code}%
%
\>[2]\AgdaFunction{to-sac}\AgdaSpace{}%
\AgdaSymbol{:}\AgdaSpace{}%
\AgdaSymbol{(}\AgdaBound{e}\AgdaSpace{}%
\AgdaSymbol{:}\AgdaSpace{}%
\AgdaDatatype{E}\AgdaSpace{}%
\AgdaGeneralizable{Γ}\AgdaSpace{}%
\AgdaGeneralizable{is}\AgdaSymbol{)}\AgdaSpace{}%
\AgdaSymbol{→}\AgdaSpace{}%
\AgdaSymbol{(}\AgdaBound{ρ}\AgdaSpace{}%
\AgdaSymbol{:}\AgdaSpace{}%
\AgdaFunction{SEnv}\AgdaSpace{}%
\AgdaGeneralizable{Γ}\AgdaSymbol{)}\AgdaSpace{}%
\AgdaSymbol{→}\AgdaSpace{}%
\AgdaFunction{State}\AgdaSpace{}%
\AgdaDatatype{ℕ}\AgdaSpace{}%
\AgdaPostulate{String}\<%
\\
%
\>[2]\AgdaFunction{to-sac}\AgdaSpace{}%
\AgdaSymbol{(}\AgdaInductiveConstructor{imap}\AgdaSpace{}%
\AgdaSymbol{\{}\AgdaArgument{s}\AgdaSpace{}%
\AgdaSymbol{=}\AgdaSpace{}%
\AgdaBound{s}\AgdaSymbol{\}}\AgdaSpace{}%
\AgdaBound{e}\AgdaSymbol{)}\AgdaSpace{}%
\AgdaBound{ρ}\AgdaSpace{}%
\AgdaSymbol{=}\AgdaSpace{}%
\AgdaKeyword{do}\<%
\\
\>[2][@{}l@{\AgdaIndent{0}}]%
\>[5]\AgdaBound{i}\AgdaSpace{}%
\AgdaOperator{\AgdaFunction{←}}\AgdaSpace{}%
\AgdaFunction{iv}\AgdaSpace{}%
\AgdaBound{s}\<%
\\
%
\>[5]\AgdaBound{b}\AgdaSpace{}%
\AgdaOperator{\AgdaFunction{←}}\AgdaSpace{}%
\AgdaFunction{to-sac}\AgdaSpace{}%
\AgdaBound{e}\AgdaSpace{}%
\AgdaSymbol{(}\AgdaBound{ρ}\AgdaSpace{}%
\AgdaOperator{\AgdaInductiveConstructor{,}}\AgdaSpace{}%
\AgdaBound{i}\AgdaSymbol{)}\<%
\\
%
\>[5]\AgdaFunction{return}\AgdaSpace{}%
\AgdaOperator{\AgdaFunction{\$}}\AgdaSpace{}%
\AgdaFunction{printf}%
\>[1927I]\AgdaString{"\{\ \%s\ ->\ \%s\ |\ \%s\ <\ \%s\ \}"}\<%
\\
\>[.][@{}l@{}]\<[1927I]%
\>[21]\AgdaBound{i}\AgdaSpace{}%
\AgdaBound{b}\AgdaSpace{}%
\AgdaBound{i}\AgdaSpace{}%
\AgdaSymbol{(}\AgdaFunction{show-shape}\AgdaSpace{}%
\AgdaBound{s}\AgdaSymbol{)}\<%
\\
%
\>[2]\AgdaFunction{to-sac}\AgdaSpace{}%
\AgdaSymbol{(}\AgdaInductiveConstructor{sel}\AgdaSpace{}%
\AgdaBound{e}\AgdaSpace{}%
\AgdaBound{e₁}\AgdaSymbol{)}\AgdaSpace{}%
\AgdaBound{ρ}\AgdaSpace{}%
\AgdaSymbol{=}\<%
\\
\>[2][@{}l@{\AgdaIndent{0}}]%
\>[5]\AgdaFunction{printf}\AgdaSpace{}%
\AgdaString{"(\%s)[\%s]"}\AgdaSpace{}%
\AgdaOperator{\AgdaFunction{<\$>}}\AgdaSpace{}%
\AgdaFunction{to-sac}\AgdaSpace{}%
\AgdaBound{e}\AgdaSpace{}%
\AgdaBound{ρ}\AgdaSpace{}%
\AgdaOperator{\AgdaFunction{⊛}}\AgdaSpace{}%
\AgdaFunction{to-sac}\AgdaSpace{}%
\AgdaBound{e₁}\AgdaSpace{}%
\AgdaBound{ρ}\<%
\\
%
\>[2]\AgdaComment{--\ ⋯}\<%
\end{code}}
\end{mathpar}
\begin{code}[hide]%
%
\>[2]\AgdaFunction{to-sac}\AgdaSpace{}%
\AgdaInductiveConstructor{zero}\AgdaSpace{}%
\AgdaBound{ρ}\AgdaSpace{}%
\AgdaSymbol{=}\AgdaSpace{}%
\AgdaFunction{return}\AgdaSpace{}%
\AgdaString{"zero"}\<%
\\
%
\>[2]\AgdaFunction{to-sac}\AgdaSpace{}%
\AgdaInductiveConstructor{one}\AgdaSpace{}%
\AgdaBound{ρ}\AgdaSpace{}%
\AgdaSymbol{=}\AgdaSpace{}%
\AgdaFunction{return}\AgdaSpace{}%
\AgdaString{"one"}\<%
\\
%
\>[2]\AgdaFunction{to-sac}\AgdaSpace{}%
\AgdaSymbol{(}\AgdaInductiveConstructor{var}\AgdaSpace{}%
\AgdaBound{x}\AgdaSymbol{)}\AgdaSpace{}%
\AgdaBound{ρ}\AgdaSpace{}%
\AgdaSymbol{=}\AgdaSpace{}%
\AgdaFunction{return}\AgdaSpace{}%
\AgdaOperator{\AgdaFunction{\$}}\AgdaSpace{}%
\AgdaFunction{lookup}\AgdaSpace{}%
\AgdaBound{x}\AgdaSpace{}%
\AgdaBound{ρ}\<%
\\
%
\>[2]\AgdaFunction{to-sac}\AgdaSpace{}%
\AgdaSymbol{(}\AgdaInductiveConstructor{imapₛ}\AgdaSpace{}%
\AgdaSymbol{\{}\AgdaArgument{s}\AgdaSpace{}%
\AgdaSymbol{=}\AgdaSpace{}%
\AgdaBound{s}\AgdaSymbol{\}}\AgdaSpace{}%
\AgdaBound{e}\AgdaSymbol{)}\AgdaSpace{}%
\AgdaBound{ρ}\AgdaSpace{}%
\AgdaSymbol{=}\AgdaSpace{}%
\AgdaKeyword{do}\<%
\\
\>[2][@{}l@{\AgdaIndent{0}}]%
\>[5]\AgdaBound{i}\AgdaSpace{}%
\AgdaOperator{\AgdaFunction{←}}\AgdaSpace{}%
\AgdaFunction{iv}\AgdaSpace{}%
\AgdaBound{s}\<%
\\
%
\>[5]\AgdaBound{b}\AgdaSpace{}%
\AgdaOperator{\AgdaFunction{←}}\AgdaSpace{}%
\AgdaFunction{to-sac}\AgdaSpace{}%
\AgdaBound{e}\AgdaSpace{}%
\AgdaSymbol{(}\AgdaBound{ρ}\AgdaSpace{}%
\AgdaOperator{\AgdaInductiveConstructor{,}}\AgdaSpace{}%
\AgdaBound{i}\AgdaSymbol{)}\<%
\\
%
\>[5]\AgdaKeyword{let}\AgdaSpace{}%
\AgdaBound{sh}\AgdaSpace{}%
\AgdaSymbol{=}\AgdaSpace{}%
\AgdaFunction{show-shape}\AgdaSpace{}%
\AgdaBound{s}\<%
\\
%
\>[5]\AgdaComment{--return\ \$\ printf\ "\{\ \%s\ ->\ \%s\ |\ \%s\ <\ \%s\ \}"\ i\ b\ i\ sh}\<%
\\
%
\>[5]\AgdaFunction{return}\AgdaSpace{}%
\AgdaOperator{\AgdaFunction{\$}}\AgdaSpace{}%
\AgdaFunction{printf}\AgdaSpace{}%
\AgdaString{"IMAPS(\%s,\ (\%s),\ (\%s))"}\AgdaSpace{}%
\AgdaBound{i}\AgdaSpace{}%
\AgdaBound{b}\AgdaSpace{}%
\AgdaBound{sh}\<%
\\
%
\>[2]\AgdaFunction{to-sac}\AgdaSpace{}%
\AgdaSymbol{(}\AgdaInductiveConstructor{selₛ}\AgdaSpace{}%
\AgdaBound{e}\AgdaSpace{}%
\AgdaBound{e₁}\AgdaSymbol{)}\AgdaSpace{}%
\AgdaBound{ρ}\AgdaSpace{}%
\AgdaSymbol{=}\AgdaSpace{}%
\AgdaKeyword{do}\<%
\\
\>[2][@{}l@{\AgdaIndent{0}}]%
\>[5]\AgdaBound{a}\AgdaSpace{}%
\AgdaOperator{\AgdaFunction{←}}\AgdaSpace{}%
\AgdaFunction{to-sac}\AgdaSpace{}%
\AgdaBound{e}\AgdaSpace{}%
\AgdaBound{ρ}\<%
\\
%
\>[5]\AgdaBound{i}\AgdaSpace{}%
\AgdaOperator{\AgdaFunction{←}}\AgdaSpace{}%
\AgdaFunction{to-sac}\AgdaSpace{}%
\AgdaBound{e₁}\AgdaSpace{}%
\AgdaBound{ρ}\<%
\\
%
\>[5]\AgdaComment{--return\ \$\ printf\ "(\%s)[\%s]"\ a\ i}\<%
\\
%
\>[5]\AgdaFunction{return}\AgdaSpace{}%
\AgdaOperator{\AgdaFunction{\$}}\AgdaSpace{}%
\AgdaFunction{printf}\AgdaSpace{}%
\AgdaString{"sels(\%s,\ \%s)"}\AgdaSpace{}%
\AgdaBound{a}\AgdaSpace{}%
\AgdaBound{i}\<%
\\
%
\\[\AgdaEmptyExtraSkip]%
%
\>[2]\AgdaComment{--\ Copy-paste\ from\ scalar\ versions}\<%
\\
%
\\[\AgdaEmptyExtraSkip]%
%
\>[2]\AgdaComment{--\ Copy-paste\ from\ scalar\ versions}\<%
\\
%
\>[2]\AgdaFunction{to-sac}\AgdaSpace{}%
\AgdaSymbol{(}\AgdaInductiveConstructor{imapb}\AgdaSpace{}%
\AgdaSymbol{\{}\AgdaArgument{s}\AgdaSpace{}%
\AgdaSymbol{=}\AgdaSpace{}%
\AgdaBound{s}\AgdaSymbol{\}\{}\AgdaBound{p}\AgdaSymbol{\}}\AgdaSpace{}%
\AgdaBound{m}\AgdaSpace{}%
\AgdaBound{e}\AgdaSymbol{)}\AgdaSpace{}%
\AgdaBound{ρ}\AgdaSpace{}%
\AgdaSymbol{=}\AgdaSpace{}%
\AgdaKeyword{do}\<%
\\
\>[2][@{}l@{\AgdaIndent{0}}]%
\>[5]\AgdaBound{i}\AgdaSpace{}%
\AgdaOperator{\AgdaFunction{←}}\AgdaSpace{}%
\AgdaFunction{iv}\AgdaSpace{}%
\AgdaBound{s}\<%
\\
%
\>[5]\AgdaBound{b}\AgdaSpace{}%
\AgdaOperator{\AgdaFunction{←}}\AgdaSpace{}%
\AgdaFunction{to-sac}\AgdaSpace{}%
\AgdaBound{e}\AgdaSpace{}%
\AgdaSymbol{(}\AgdaBound{ρ}\AgdaSpace{}%
\AgdaOperator{\AgdaInductiveConstructor{,}}\AgdaSpace{}%
\AgdaBound{i}\AgdaSymbol{)}\<%
\\
%
\>[5]\AgdaKeyword{let}\AgdaSpace{}%
\AgdaBound{sh-s}\AgdaSpace{}%
\AgdaSymbol{=}\AgdaSpace{}%
\AgdaFunction{show-shape}\AgdaSpace{}%
\AgdaBound{s}\<%
\\
%
\>[5]\AgdaKeyword{let}\AgdaSpace{}%
\AgdaBound{sh-p}\AgdaSpace{}%
\AgdaSymbol{=}\AgdaSpace{}%
\AgdaFunction{show-shape}\AgdaSpace{}%
\AgdaBound{p}\<%
\\
%
\>[5]\AgdaFunction{return}\AgdaSpace{}%
\AgdaOperator{\AgdaFunction{\$}}\AgdaSpace{}%
\AgdaFunction{printf}\AgdaSpace{}%
\AgdaString{"unblock(\{\ \%s\ ->\ \%s\ |\ \%s\ <\ \%s\ \},\ \%s)"}\AgdaSpace{}%
\AgdaBound{i}\AgdaSpace{}%
\AgdaBound{b}\AgdaSpace{}%
\AgdaBound{i}\AgdaSpace{}%
\AgdaBound{sh-s}\AgdaSpace{}%
\AgdaBound{sh-p}\<%
\\
%
\>[2]\AgdaFunction{to-sac}\AgdaSpace{}%
\AgdaSymbol{(}\AgdaInductiveConstructor{selb}\AgdaSpace{}%
\AgdaSymbol{\{}\AgdaArgument{p}\AgdaSpace{}%
\AgdaSymbol{=}\AgdaSpace{}%
\AgdaBound{p}\AgdaSymbol{\}}\AgdaSpace{}%
\AgdaBound{m}\AgdaSpace{}%
\AgdaBound{e}\AgdaSpace{}%
\AgdaBound{e₁}\AgdaSymbol{)}\AgdaSpace{}%
\AgdaBound{ρ}\AgdaSpace{}%
\AgdaSymbol{=}\AgdaSpace{}%
\AgdaKeyword{do}\<%
\\
\>[2][@{}l@{\AgdaIndent{0}}]%
\>[5]\AgdaBound{a}\AgdaSpace{}%
\AgdaOperator{\AgdaFunction{←}}\AgdaSpace{}%
\AgdaFunction{to-sac}\AgdaSpace{}%
\AgdaBound{e}\AgdaSpace{}%
\AgdaBound{ρ}\<%
\\
%
\>[5]\AgdaBound{i}\AgdaSpace{}%
\AgdaOperator{\AgdaFunction{←}}\AgdaSpace{}%
\AgdaFunction{to-sac}\AgdaSpace{}%
\AgdaBound{e₁}\AgdaSpace{}%
\AgdaBound{ρ}\<%
\\
%
\>[5]\AgdaKeyword{let}\AgdaSpace{}%
\AgdaBound{sh-p}\AgdaSpace{}%
\AgdaSymbol{=}\AgdaSpace{}%
\AgdaFunction{show-shape}\AgdaSpace{}%
\AgdaBound{p}\<%
\\
%
\>[5]\AgdaFunction{return}\AgdaSpace{}%
\AgdaOperator{\AgdaFunction{\$}}\AgdaSpace{}%
\AgdaFunction{printf}\AgdaSpace{}%
\AgdaString{"selb(\%s,\ \%s,\ \%s)"}\AgdaSpace{}%
\AgdaBound{a}\AgdaSpace{}%
\AgdaBound{i}\AgdaSpace{}%
\AgdaBound{sh-p}\<%
\\
%
\\[\AgdaEmptyExtraSkip]%
%
\>[2]\AgdaFunction{to-sac}\AgdaSpace{}%
\AgdaSymbol{(}\AgdaInductiveConstructor{zero-but}\AgdaSpace{}%
\AgdaBound{i}\AgdaSpace{}%
\AgdaBound{j}\AgdaSpace{}%
\AgdaBound{e}\AgdaSymbol{)}\AgdaSpace{}%
\AgdaBound{ρ}\<%
\\
\>[2][@{}l@{\AgdaIndent{0}}]%
\>[5]\AgdaSymbol{=}\AgdaSpace{}%
\AgdaFunction{printf}\AgdaSpace{}%
\AgdaString{"\%s\ ==\ \%s\ ?\ \%s\ :\ zero"}\AgdaSpace{}%
\AgdaOperator{\AgdaFunction{<\$>}}\AgdaSpace{}%
\AgdaSymbol{(}\AgdaFunction{to-sac}\AgdaSpace{}%
\AgdaBound{i}\AgdaSpace{}%
\AgdaBound{ρ}\AgdaSymbol{)}\AgdaSpace{}%
\AgdaOperator{\AgdaFunction{⊛}}\AgdaSpace{}%
\AgdaSymbol{(}\AgdaFunction{to-sac}\AgdaSpace{}%
\AgdaBound{j}\AgdaSpace{}%
\AgdaBound{ρ}\AgdaSymbol{)}\AgdaSpace{}%
\AgdaOperator{\AgdaFunction{⊛}}\AgdaSpace{}%
\AgdaSymbol{(}\AgdaFunction{to-sac}\AgdaSpace{}%
\AgdaBound{e}\AgdaSpace{}%
\AgdaBound{ρ}\AgdaSymbol{)}\<%
\\
%
\>[2]\AgdaFunction{to-sac}\AgdaSpace{}%
\AgdaSymbol{(}\AgdaInductiveConstructor{sum}\AgdaSpace{}%
\AgdaSymbol{\{}\AgdaArgument{s}\AgdaSpace{}%
\AgdaSymbol{=}\AgdaSpace{}%
\AgdaBound{s}\AgdaSymbol{\}}\AgdaSpace{}%
\AgdaSymbol{\{}\AgdaArgument{p}\AgdaSpace{}%
\AgdaSymbol{=}\AgdaSpace{}%
\AgdaBound{p}\AgdaSymbol{\}}\AgdaSpace{}%
\AgdaBound{e}\AgdaSymbol{)}\AgdaSpace{}%
\AgdaBound{ρ}\AgdaSpace{}%
\AgdaSymbol{=}\AgdaSpace{}%
\AgdaKeyword{do}\<%
\\
\>[2][@{}l@{\AgdaIndent{0}}]%
\>[5]\AgdaComment{--\ outer\ index\ }\<%
\\
%
\>[5]\AgdaBound{i}\AgdaSpace{}%
\AgdaOperator{\AgdaFunction{←}}\AgdaSpace{}%
\AgdaFunction{iv}\AgdaSpace{}%
\AgdaBound{s}\<%
\\
%
\>[5]\AgdaComment{--\ inner\ index\ which\ is\ juts\ a\ fresh\ name}\<%
\\
%
\>[5]\AgdaBound{j}\AgdaSpace{}%
\AgdaOperator{\AgdaFunction{←}}\AgdaSpace{}%
\AgdaFunction{iv}\AgdaSpace{}%
\AgdaBound{p}\<%
\\
%
\>[5]\AgdaBound{b}\AgdaSpace{}%
\AgdaOperator{\AgdaFunction{←}}\AgdaSpace{}%
\AgdaFunction{to-sac}\AgdaSpace{}%
\AgdaBound{e}\AgdaSpace{}%
\AgdaSymbol{(}\AgdaBound{ρ}\AgdaSpace{}%
\AgdaOperator{\AgdaInductiveConstructor{,}}\AgdaSpace{}%
\AgdaBound{i}\AgdaSymbol{)}\<%
\\
%
\>[5]\AgdaComment{--\ `s`\ is\ outer\ shape,\ and\ `p`\ is\ the\ inner\ one}\<%
\\
%
\>[5]\AgdaKeyword{let}\AgdaSpace{}%
\AgdaBound{sh-s}\AgdaSpace{}%
\AgdaSymbol{=}\AgdaSpace{}%
\AgdaFunction{show-shape}\AgdaSpace{}%
\AgdaBound{s}\<%
\\
%
\>[5]\AgdaKeyword{let}\AgdaSpace{}%
\AgdaBound{sh-p}\AgdaSpace{}%
\AgdaSymbol{=}\AgdaSpace{}%
\AgdaFunction{show-shape}\AgdaSpace{}%
\AgdaBound{p}\<%
\\
%
\>[5]\AgdaComment{--return\ \$\ printf\ "sumOuter(\%u,\ \{\ \%s\ ->\ \%s\ |\ \%s\ <\ \%s\})"\ (dim\ s)\ i\ b\ i\ sh-s}\<%
\\
%
\>[5]\AgdaComment{--\ sumOuter(ivOuter,\ ivInner,\ e,\ shOuter,\ shInner)}\<%
\\
%
\>[5]\AgdaFunction{return}\AgdaSpace{}%
\AgdaOperator{\AgdaFunction{\$}}\AgdaSpace{}%
\AgdaFunction{printf}\AgdaSpace{}%
\AgdaString{"sumOuter(\%s,\ \%s,\ \%s,\ (\%s),\ (\%s))"}\AgdaSpace{}%
\AgdaBound{i}\AgdaSpace{}%
\AgdaBound{j}\AgdaSpace{}%
\AgdaBound{b}\AgdaSpace{}%
\AgdaBound{sh-s}\AgdaSpace{}%
\AgdaBound{sh-p}\<%
\\
%
\>[2]\AgdaFunction{to-sac}\AgdaSpace{}%
\AgdaSymbol{(}\AgdaInductiveConstructor{bin}\AgdaSpace{}%
\AgdaBound{x}\AgdaSpace{}%
\AgdaBound{e}\AgdaSpace{}%
\AgdaBound{e₁}\AgdaSymbol{)}\AgdaSpace{}%
\AgdaBound{ρ}\AgdaSpace{}%
\AgdaSymbol{=}\AgdaSpace{}%
\AgdaKeyword{do}\<%
\\
\>[2][@{}l@{\AgdaIndent{0}}]%
\>[5]\AgdaBound{a}\AgdaSpace{}%
\AgdaOperator{\AgdaFunction{←}}\AgdaSpace{}%
\AgdaFunction{to-sac}\AgdaSpace{}%
\AgdaBound{e}\AgdaSpace{}%
\AgdaBound{ρ}\<%
\\
%
\>[5]\AgdaBound{b}\AgdaSpace{}%
\AgdaOperator{\AgdaFunction{←}}\AgdaSpace{}%
\AgdaFunction{to-sac}\AgdaSpace{}%
\AgdaBound{e₁}\AgdaSpace{}%
\AgdaBound{ρ}\<%
\\
%
\>[5]\AgdaFunction{return}\AgdaSpace{}%
\AgdaOperator{\AgdaFunction{\$}}\AgdaSpace{}%
\AgdaFunction{printf}\AgdaSpace{}%
\AgdaString{"(\%s)\ \%s\ (\%s)"}\AgdaSpace{}%
\AgdaBound{a}\AgdaSpace{}%
\AgdaSymbol{(}\AgdaFunction{bop}\AgdaSpace{}%
\AgdaBound{x}\AgdaSymbol{)}\AgdaSpace{}%
\AgdaBound{b}\<%
\\
%
\>[2]\AgdaFunction{to-sac}\AgdaSpace{}%
\AgdaSymbol{(}\AgdaInductiveConstructor{slide}\AgdaSpace{}%
\AgdaSymbol{\{}\AgdaArgument{p}\AgdaSpace{}%
\AgdaSymbol{=}\AgdaSpace{}%
\AgdaBound{p}\AgdaSymbol{\}}\AgdaSpace{}%
\AgdaBound{e}\AgdaSpace{}%
\AgdaBound{pl}\AgdaSpace{}%
\AgdaBound{e₁}\AgdaSpace{}%
\AgdaBound{su}\AgdaSymbol{)}\AgdaSpace{}%
\AgdaBound{ρ}\AgdaSpace{}%
\AgdaSymbol{=}\AgdaSpace{}%
\AgdaKeyword{do}\<%
\\
\>[2][@{}l@{\AgdaIndent{0}}]%
\>[5]\AgdaBound{i}\AgdaSpace{}%
\AgdaOperator{\AgdaFunction{←}}\AgdaSpace{}%
\AgdaFunction{to-sac}\AgdaSpace{}%
\AgdaBound{e}\AgdaSpace{}%
\AgdaBound{ρ}\<%
\\
%
\>[5]\AgdaBound{a}\AgdaSpace{}%
\AgdaOperator{\AgdaFunction{←}}\AgdaSpace{}%
\AgdaFunction{to-sac}\AgdaSpace{}%
\AgdaBound{e₁}\AgdaSpace{}%
\AgdaBound{ρ}\<%
\\
%
\>[5]\AgdaKeyword{let}\AgdaSpace{}%
\AgdaBound{sh-p}\AgdaSpace{}%
\AgdaSymbol{=}\AgdaSpace{}%
\AgdaFunction{show-shape}\AgdaSpace{}%
\AgdaBound{p}\<%
\\
%
\>[5]\AgdaFunction{return}\AgdaSpace{}%
\AgdaOperator{\AgdaFunction{\$}}\AgdaSpace{}%
\AgdaFunction{printf}\AgdaSpace{}%
\AgdaString{"slide(\%s,\ \%s,\ \%s)"}\AgdaSpace{}%
\AgdaBound{i}\AgdaSpace{}%
\AgdaBound{a}\AgdaSpace{}%
\AgdaBound{sh-p}\<%
\\
%
\>[2]\AgdaFunction{to-sac}\AgdaSpace{}%
\AgdaSymbol{(}\AgdaInductiveConstructor{backslide}\AgdaSpace{}%
\AgdaSymbol{\{}\AgdaArgument{r}\AgdaSpace{}%
\AgdaSymbol{=}\AgdaSpace{}%
\AgdaBound{r}\AgdaSymbol{\}}\AgdaSpace{}%
\AgdaBound{e}\AgdaSpace{}%
\AgdaBound{e₁}\AgdaSpace{}%
\AgdaBound{su}\AgdaSpace{}%
\AgdaBound{pl}\AgdaSymbol{)}\AgdaSpace{}%
\AgdaBound{ρ}\AgdaSpace{}%
\AgdaSymbol{=}\AgdaSpace{}%
\AgdaKeyword{do}\<%
\\
\>[2][@{}l@{\AgdaIndent{0}}]%
\>[5]\AgdaBound{i}\AgdaSpace{}%
\AgdaOperator{\AgdaFunction{←}}\AgdaSpace{}%
\AgdaFunction{to-sac}\AgdaSpace{}%
\AgdaBound{e}\AgdaSpace{}%
\AgdaBound{ρ}\<%
\\
%
\>[5]\AgdaBound{a}\AgdaSpace{}%
\AgdaOperator{\AgdaFunction{←}}\AgdaSpace{}%
\AgdaFunction{to-sac}\AgdaSpace{}%
\AgdaBound{e₁}\AgdaSpace{}%
\AgdaBound{ρ}\<%
\\
%
\>[5]\AgdaKeyword{let}\AgdaSpace{}%
\AgdaBound{sh-sp}\AgdaSpace{}%
\AgdaSymbol{=}\AgdaSpace{}%
\AgdaFunction{show-shape}\AgdaSpace{}%
\AgdaBound{r}\<%
\\
%
\>[5]\AgdaFunction{return}\AgdaSpace{}%
\AgdaOperator{\AgdaFunction{\$}}\AgdaSpace{}%
\AgdaFunction{printf}\AgdaSpace{}%
\AgdaString{"backlide(\%s,\ \%s,\ \%s)"}\AgdaSpace{}%
\AgdaBound{i}\AgdaSpace{}%
\AgdaBound{a}\AgdaSpace{}%
\AgdaBound{sh-sp}\<%
\\
%
\\[\AgdaEmptyExtraSkip]%
%
\>[2]\AgdaFunction{to-sac}\AgdaSpace{}%
\AgdaSymbol{(}\AgdaInductiveConstructor{scaledown}\AgdaSpace{}%
\AgdaBound{x}\AgdaSpace{}%
\AgdaBound{e}\AgdaSymbol{)}\AgdaSpace{}%
\AgdaBound{ρ}\AgdaSpace{}%
\AgdaSymbol{=}\AgdaSpace{}%
\AgdaKeyword{do}\<%
\\
\>[2][@{}l@{\AgdaIndent{0}}]%
\>[5]\AgdaBound{a}\AgdaSpace{}%
\AgdaOperator{\AgdaFunction{←}}\AgdaSpace{}%
\AgdaFunction{to-sac}\AgdaSpace{}%
\AgdaBound{e}\AgdaSpace{}%
\AgdaBound{ρ}\<%
\\
%
\>[5]\AgdaFunction{return}\AgdaSpace{}%
\AgdaOperator{\AgdaFunction{\$}}\AgdaSpace{}%
\AgdaFunction{printf}\AgdaSpace{}%
\AgdaString{"(\%s)\ /\ \%s"}\AgdaSpace{}%
\AgdaBound{a}\AgdaSpace{}%
\AgdaSymbol{(}\AgdaFunction{show-nat}\AgdaSpace{}%
\AgdaBound{x}\AgdaSymbol{)}\<%
\\
%
\\[\AgdaEmptyExtraSkip]%
%
\>[2]\AgdaFunction{to-sac}\AgdaSpace{}%
\AgdaSymbol{(}\AgdaInductiveConstructor{minus}\AgdaSpace{}%
\AgdaBound{e}\AgdaSymbol{)}\AgdaSpace{}%
\AgdaBound{ρ}\AgdaSpace{}%
\AgdaSymbol{=}\AgdaSpace{}%
\AgdaFunction{printf}\AgdaSpace{}%
\AgdaString{"-(\%s)"}\AgdaSpace{}%
\AgdaOperator{\AgdaFunction{<\$>}}\AgdaSpace{}%
\AgdaFunction{to-sac}\AgdaSpace{}%
\AgdaBound{e}\AgdaSpace{}%
\AgdaBound{ρ}\<%
\\
%
\>[2]\AgdaFunction{to-sac}\AgdaSpace{}%
\AgdaSymbol{(}\AgdaInductiveConstructor{logistic}\AgdaSpace{}%
\AgdaBound{e}\AgdaSymbol{)}\AgdaSpace{}%
\AgdaBound{ρ}\AgdaSpace{}%
\AgdaSymbol{=}\AgdaSpace{}%
\AgdaFunction{printf}\AgdaSpace{}%
\AgdaString{"logistics(\%s)"}\AgdaSpace{}%
\AgdaOperator{\AgdaFunction{<\$>}}\AgdaSpace{}%
\AgdaFunction{to-sac}\AgdaSpace{}%
\AgdaBound{e}\AgdaSpace{}%
\AgdaBound{ρ}\<%
\\
%
\\[\AgdaEmptyExtraSkip]%
%
\\[\AgdaEmptyExtraSkip]%
%
\>[2]\AgdaComment{--\ This\ can\ be\ made\ stateful,\ but\ we\ are\ assuming\ that}\<%
\\
%
\>[2]\AgdaComment{--\ vₛ\ is\ no\ need\ to\ make\ imap/sum\ index\ variables\ unique.}\<%
\\
%
\>[2]\AgdaFunction{env-sac}\AgdaSpace{}%
\AgdaSymbol{:}\AgdaSpace{}%
\AgdaFunction{AD.Env}\AgdaSpace{}%
\AgdaGeneralizable{Γ}\AgdaSpace{}%
\AgdaGeneralizable{Δ}\AgdaSpace{}%
\AgdaSymbol{→}\AgdaSpace{}%
\AgdaSymbol{(}\AgdaBound{vars}\AgdaSpace{}%
\AgdaSymbol{:}\AgdaSpace{}%
\AgdaFunction{SEnv}\AgdaSpace{}%
\AgdaGeneralizable{Δ}\AgdaSymbol{)}\AgdaSpace{}%
\AgdaSymbol{→}\AgdaSpace{}%
\AgdaFunction{SEnv}\AgdaSpace{}%
\AgdaGeneralizable{Γ}\<%
\\
%
\>[2]\AgdaFunction{env-sac}\AgdaSpace{}%
\AgdaSymbol{\{}\AgdaInductiveConstructor{ε}\AgdaSymbol{\}}\AgdaSpace{}%
\AgdaBound{ρ}\AgdaSpace{}%
\AgdaBound{σ}\AgdaSpace{}%
\AgdaSymbol{=}\AgdaSpace{}%
\AgdaSymbol{\AgdaUnderscore{}}\<%
\\
%
\>[2]\AgdaFunction{env-sac}\AgdaSpace{}%
\AgdaSymbol{\{}\AgdaBound{Γ}\AgdaSpace{}%
\AgdaOperator{\AgdaInductiveConstructor{▹}}\AgdaSpace{}%
\AgdaInductiveConstructor{ix}\AgdaSpace{}%
\AgdaBound{s}\AgdaSymbol{\}}\AgdaSpace{}%
\AgdaBound{ρ}\AgdaSpace{}%
\AgdaBound{σ}\AgdaSpace{}%
\AgdaSymbol{=}\AgdaSpace{}%
\AgdaFunction{env-sac}\AgdaSpace{}%
\AgdaBound{ρ}\AgdaSpace{}%
\AgdaBound{σ}\AgdaSpace{}%
\AgdaOperator{\AgdaInductiveConstructor{,}}\AgdaSpace{}%
\AgdaString{"--"}\<%
\\
%
\>[2]\AgdaFunction{env-sac}\AgdaSpace{}%
\AgdaSymbol{\{}\AgdaBound{Γ}\AgdaSpace{}%
\AgdaOperator{\AgdaInductiveConstructor{▹}}\AgdaSpace{}%
\AgdaInductiveConstructor{ar}\AgdaSpace{}%
\AgdaBound{s}\AgdaSymbol{\}}\AgdaSpace{}%
\AgdaSymbol{(}\AgdaBound{ρ}\AgdaSpace{}%
\AgdaOperator{\AgdaInductiveConstructor{,}}\AgdaSpace{}%
\AgdaBound{e}\AgdaSymbol{)}\AgdaSpace{}%
\AgdaBound{σ}\AgdaSpace{}%
\AgdaSymbol{=}\AgdaSpace{}%
\AgdaFunction{env-sac}\AgdaSpace{}%
\AgdaBound{ρ}\AgdaSpace{}%
\AgdaBound{σ}\AgdaSpace{}%
\AgdaOperator{\AgdaInductiveConstructor{,}}\AgdaSpace{}%
\AgdaField{proj₁}\AgdaSpace{}%
\AgdaSymbol{(}\AgdaFunction{to-sac}\AgdaSpace{}%
\AgdaBound{e}\AgdaSpace{}%
\AgdaBound{σ}\AgdaSpace{}%
\AgdaNumber{1}\AgdaSymbol{)}\<%
\\
%
\\[\AgdaEmptyExtraSkip]%
%
\>[2]\AgdaComment{--\ Reversed\ environment\ to\ list}\<%
\\
%
\>[2]\AgdaFunction{env-rev-list}\AgdaSpace{}%
\AgdaSymbol{:}\AgdaSpace{}%
\AgdaFunction{SEnv}\AgdaSpace{}%
\AgdaGeneralizable{Γ}\AgdaSpace{}%
\AgdaSymbol{→}\AgdaSpace{}%
\AgdaDatatype{List}\AgdaSpace{}%
\AgdaPostulate{String}\<%
\\
%
\>[2]\AgdaFunction{env-rev-list}\AgdaSpace{}%
\AgdaSymbol{\{}\AgdaInductiveConstructor{ε}\AgdaSymbol{\}}%
\>[23]\AgdaBound{ρ}\AgdaSpace{}%
\AgdaSymbol{=}\AgdaSpace{}%
\AgdaInductiveConstructor{[]}\<%
\\
%
\>[2]\AgdaFunction{env-rev-list}\AgdaSpace{}%
\AgdaSymbol{\{}\AgdaBound{Γ}\AgdaSpace{}%
\AgdaOperator{\AgdaInductiveConstructor{▹}}\AgdaSpace{}%
\AgdaSymbol{\AgdaUnderscore{}\}}\AgdaSpace{}%
\AgdaSymbol{(}\AgdaBound{ρ}\AgdaSpace{}%
\AgdaOperator{\AgdaInductiveConstructor{,}}\AgdaSpace{}%
\AgdaBound{x}\AgdaSymbol{)}\AgdaSpace{}%
\AgdaSymbol{=}\AgdaSpace{}%
\AgdaBound{x}\AgdaSpace{}%
\AgdaOperator{\AgdaInductiveConstructor{∷}}\AgdaSpace{}%
\AgdaFunction{env-rev-list}\AgdaSpace{}%
\AgdaBound{ρ}\<%
\\
\>[0]\<%
\\
%
\>[2]\AgdaComment{--\ zipWith\ for\ Environments}\<%
\\
%
\>[2]\AgdaFunction{zip-env}\AgdaSpace{}%
\AgdaSymbol{:}\AgdaSpace{}%
\AgdaSymbol{(}\AgdaPostulate{String}\AgdaSpace{}%
\AgdaSymbol{→}\AgdaSpace{}%
\AgdaPostulate{String}\AgdaSpace{}%
\AgdaSymbol{→}\AgdaSpace{}%
\AgdaPostulate{String}\AgdaSymbol{)}\AgdaSpace{}%
\AgdaSymbol{→}\AgdaSpace{}%
\AgdaFunction{SEnv}\AgdaSpace{}%
\AgdaGeneralizable{Γ}\AgdaSpace{}%
\AgdaSymbol{→}\AgdaSpace{}%
\AgdaFunction{SEnv}\AgdaSpace{}%
\AgdaGeneralizable{Γ}\AgdaSpace{}%
\AgdaSymbol{→}\AgdaSpace{}%
\AgdaFunction{SEnv}\AgdaSpace{}%
\AgdaGeneralizable{Γ}\<%
\\
%
\>[2]\AgdaFunction{zip-env}\AgdaSpace{}%
\AgdaSymbol{\{}\AgdaInductiveConstructor{ε}\AgdaSymbol{\}}%
\>[18]\AgdaBound{f}\AgdaSpace{}%
\AgdaInductiveConstructor{tt}%
\>[28]\AgdaInductiveConstructor{tt}%
\>[36]\AgdaSymbol{=}\AgdaSpace{}%
\AgdaInductiveConstructor{tt}\<%
\\
%
\>[2]\AgdaFunction{zip-env}\AgdaSpace{}%
\AgdaSymbol{\{}\AgdaBound{Γ}\AgdaSpace{}%
\AgdaOperator{\AgdaInductiveConstructor{▹}}\AgdaSpace{}%
\AgdaBound{x}\AgdaSymbol{\}}\AgdaSpace{}%
\AgdaBound{f}\AgdaSpace{}%
\AgdaSymbol{(}\AgdaBound{ν}\AgdaSpace{}%
\AgdaOperator{\AgdaInductiveConstructor{,}}\AgdaSpace{}%
\AgdaBound{n}\AgdaSymbol{)}\AgdaSpace{}%
\AgdaSymbol{(}\AgdaBound{ρ}\AgdaSpace{}%
\AgdaOperator{\AgdaInductiveConstructor{,}}\AgdaSpace{}%
\AgdaBound{e}\AgdaSymbol{)}\AgdaSpace{}%
\AgdaSymbol{=}\AgdaSpace{}%
\AgdaFunction{zip-env}\AgdaSpace{}%
\AgdaBound{f}\AgdaSpace{}%
\AgdaBound{ν}\AgdaSpace{}%
\AgdaBound{ρ}\AgdaSpace{}%
\AgdaOperator{\AgdaInductiveConstructor{,}}\AgdaSpace{}%
\AgdaBound{f}\AgdaSpace{}%
\AgdaBound{n}\AgdaSpace{}%
\AgdaBound{e}\<%
\end{code}

\subsubsection{SaC Primitives\label{sec:sac-primitives}}
As can be seen from the two cases of \AF{to-sac}, the extraction process is
not complicated. In essence, we define a small snippet of SaC code for 
each \AF{E} constructor.  Consider the \AC{imap}/\AC{sel}
family from the code snippet.  The \AC{imap} constructor maps directly to SaC's
tensor comprehensions~\cite{tensor-comp} expressed as: \texttt{\{ iv -> e | iv < s \}}.
This expression constructs arrays by evaluating \texttt{e} for every array non-negative index
vector
\texttt{iv} whose components are element-wise smaller than the shape \texttt{s}.  The shape of the resulting
array is concatenation of \texttt{s} and whatever the shape of \texttt{e} is.
Selections \AC{sel} correspond to the built-in array selection using
C-like syntax \texttt{e[iv]} where \texttt{e} is the array we are selecting
from and \texttt{iv} is the index vector.   Shape constraints are exactly as in
\AF{E}: if \texttt{e} is of shape \texttt{s ++ p}, and \texttt{iv} is bounded
by \texttt{s} then \texttt{e[iv]} is of shape \texttt{p}.

Scalar versions of imap/sel require a little wrapping.  For \AC{imapₛ} we
generate a tensor comprehension that selects inner expressions (they are
1-element vectors) at zero-th position.  For \AC{selₛ} we make selection into
an array and we wrap the result in a 1-d vector:
\begin{mathpar}
{\begin{varwidth}{0.9\textwidth}
\begin{lstlisting}[linewidth=.4\textwidth]
#define IMAPS(iv, e, shp) \
  {iv -> (e)[[0]] | iv < shp}
\end{lstlisting}
\end{varwidth}}
\and
{\begin{varwidth}{0.9\textwidth}
\begin{lstlisting}[linewidth=.55\textwidth]
inline float[1]
sels(float[d:shp] x, int[d] iv)
{
  return [x[iv]];
}
\end{lstlisting}
\end{varwidth}}
\end{mathpar}
When translating (\AC{imapₛ} \{ \AB{s} \} \AB{e}) we pick a fresh index variable
\texttt{iv}, then we translate \AB{e} (in the environment extended with \texttt{iv})
into \texttt{e'} and we generate \texttt{IMAPS(iv, e', shp)}, where \texttt{shp} is
a translation of \texttt{s}.  On the side of SaC we expand this macro as shown
above.  We could have expanded this macro on the Agda side, but this abstraction
makes it possible to make adjustments in the generated code without running Agda.
We map \AC{selₛ} into the \texttt{sels} function.  Consider the type of \texttt{sels}
which uses the recently added feature of SaC that makes it possible to encode
shape constraints in types~\cite{type-pattern}.  While these constraints are potentially checked at runtime,
they are very useful for readability and they provide some confidence about the
generated code.  The meaning of the type \texttt{float[d:shp]} is that it is
an array of base type \texttt{float} of rank \texttt{d} and shape \texttt{shp}.
When a variable of the same name is used within different arguments, it automatically
triggers the equality constraint between the corresponding ranks/shapes.

\paragraph{Blocking} Implementation of \AC{selb}/\AC{imapb} pair relies on
the notion of blocking, so we introduce the analogue to \AF{block}/\AF{unblock}
functionality in SaC as follows:
\begin{mathpar}
{\begin{varwidth}{0.9\textwidth}
\begin{lstlisting}[linewidth=.44\textwidth]
inline float[n:s,n:p]
block(float[n:sp] x, int[n] p)
     | all(s*p == sp)
     , all(p   >= 0)
{
  return { iv -> tile(p, iv * p, x) 
         | iv < sp / p};
}
\end{lstlisting}
\end{varwidth}}
\and
{\begin{varwidth}{0.9\textwidth}
\begin{lstlisting}[linewidth=.55\textwidth]
inline float[n:sp] 
unblock(float[n:s,n:p] a, int[n] p)
       | all(s*p == sp)
       , all(p   >= 0)
{
  return { iv -> a[(iv / p) ++ mod (iv, p)]
         | iv < s*p};
}
\end{lstlisting}
\end{varwidth}}
\end{mathpar}
The type \texttt{float[n:s,n:p]} denotes an array of the shape \texttt{s ++ p}
where \texttt{s} and \texttt{p} are of length \texttt{n}.  This is a product
shape in terms of our array theory.  As \texttt{sp} is just a variable that
is not related to \texttt{s} or \texttt{p}, we add two constraints (expressions
behind the bar after the function definition) saying that: (i) \texttt{sp} is
a point-wise product of \texttt{s} and \texttt{p}; (ii) all the elements of
the \texttt{p}-shape are greater than zero.  Keep in mind that these are potential
runtime constraints, they may be proved or flagged as disproved during compilation
but they do not provide a static guarantee. The implementation of block uses the \texttt{tile}
operation from the standard library of SaC. It selects a sub-array of the given shape at the given position.
In \texttt{unblock} we use a division and a modulo operation to remap the indices.
When translating \AC{selb}, we simply select into \texttt{block}-ed array.
When translating \AC{imapb}, we use the tensor comprehension as in case of
\AC{imap} to compute blocked array and then we call \texttt{unblock} on it.

\paragraph{Sliding} Slides and backslides are translated into calls to
the following SaC functions:
\begin{mathpar}
{\begin{varwidth}{0.9\textwidth}
\begin{lstlisting}
inline float[d:n1] 
slide(int[d] i, float[d:mn] x, int[d] n)       | all(n1        == n + 1)
                                               , all(n + 1 + i <= mn)
{
  return { iv -> x[iv + i] | iv < n + 1 };
}

inline float[d:mn]
backslide(int[d] i, float[d:n1] y, int[d] mn)  | all(i < 1 + mn - n1)
{
  return { iv -> y[iv - i] | i <= iv < n1 + i;
           iv -> 0f        |      iv < mn };
}
\end{lstlisting}
\end{varwidth}}
\end{mathpar}
Shape constraints become a little bit involved here because we implicitly
reconstruct the proof objects such as \AB{m} \AF{+} \AB{n} \AF{≈} \AB{mn}
and \AF{suc} \AB{n} \AF{≈} \AB{n1}.  Otherwise, \texttt{slide} selects a
sub-array of the shape (\texttt{n+1}) starting at the index \texttt{i}.
The \texttt{backslide} populates the sub-array with the elements of
\texttt{y} and the second partition of the tensor comprehension specifies
that all the other indices evaluate to zero.  Translation of \AC{slide}
and \AC{backslide} maps the arguments one-to-one, additionally providing
the $n$-shape in case of slide and the $(m+n)$ shape in case of backslide.

\paragraph{Summation} When translating (\AC{sum} \{\AB{s}\} \AB{e}), where
\AB{e} is of shape \AB{p} (and the index variable within the \AC{sum} is
bounded by \AB{s}), we map these arguments into the following SaC function:
\begin{lstlisting}
inline float[n:p] sumOuter(float[m:s,n:p] a, int[m] s, int[n] p) {
  return { jv -> sum({iv -> a[iv++jv] | iv < s}) | jv < p };
}
\end{lstlisting}
We use SaC's builtin \texttt{sum} function that sums-up all the elements
of the given array.

The rest of the constructions are mapped into regular arithmetic operations
that are provided by SaC.


\subsection{Local Variables}

The framework that we built so far computes derivatives of the variables in
the context.  This means that for complex expressions in \AF{E} (such as \AF{forward}),
all the let bindings will be inlined.  This is often not desirable both for performance
and readability.  Here we present a mechanism that introduce local variables
and preserves them during AD.
\begin{code}[hide]%
\>[0]\AgdaKeyword{module}\AgdaSpace{}%
\AgdaModule{DoubleChain}\AgdaSpace{}%
\AgdaKeyword{where}\<%
\\
\>[0][@{}l@{\AgdaIndent{0}}]%
\>[2]\AgdaComment{--\ In\ this\ module\ I\ want\ to\ preserve\ derivatives}\<%
\\
%
\>[2]\AgdaComment{--\ of\ the\ local\ variables\ in\ the\ chain\ (instead\ of\ inlining\ them)}\<%
\\
%
\>[2]\AgdaKeyword{open}\AgdaSpace{}%
\AgdaKeyword{import}\AgdaSpace{}%
\AgdaModule{Data.String}\<%
\\
%
\>[2]\AgdaKeyword{open}\AgdaSpace{}%
\AgdaKeyword{import}\AgdaSpace{}%
\AgdaModule{Text.Printf}\<%
\\
%
\>[2]\AgdaKeyword{open}\AgdaSpace{}%
\AgdaKeyword{import}\AgdaSpace{}%
\AgdaModule{Data.Product}\AgdaSpace{}%
\AgdaComment{--using\ (Σ;\ \AgdaUnderscore{}×\AgdaUnderscore{};\ \AgdaUnderscore{},\AgdaUnderscore{})}\<%
\\
%
\>[2]\AgdaKeyword{open}\AgdaSpace{}%
\AgdaKeyword{import}\AgdaSpace{}%
\AgdaModule{Data.Unit}\<%
\\
%
\>[2]\AgdaKeyword{open}\AgdaSpace{}%
\AgdaKeyword{import}\AgdaSpace{}%
\AgdaModule{Data.Nat}\AgdaSpace{}%
\AgdaSymbol{as}\AgdaSpace{}%
\AgdaModule{ℕ}\AgdaSpace{}%
\AgdaKeyword{using}\AgdaSpace{}%
\AgdaSymbol{(}\AgdaDatatype{ℕ}\AgdaSymbol{;}\AgdaSpace{}%
\AgdaInductiveConstructor{zero}\AgdaSymbol{;}\AgdaSpace{}%
\AgdaInductiveConstructor{suc}\AgdaSymbol{)}\<%
\\
%
\>[2]\AgdaKeyword{open}\AgdaSpace{}%
\AgdaKeyword{import}\AgdaSpace{}%
\AgdaModule{Data.List}\AgdaSpace{}%
\AgdaSymbol{as}\AgdaSpace{}%
\AgdaModule{L}\AgdaSpace{}%
\AgdaKeyword{using}\AgdaSpace{}%
\AgdaSymbol{(}\AgdaDatatype{List}\AgdaSymbol{;}\AgdaSpace{}%
\AgdaInductiveConstructor{[]}\AgdaSymbol{;}\AgdaSpace{}%
\AgdaOperator{\AgdaInductiveConstructor{\AgdaUnderscore{}∷\AgdaUnderscore{}}}\AgdaSymbol{)}\<%
\\
%
\>[2]\AgdaKeyword{open}\AgdaSpace{}%
\AgdaModule{Array}\AgdaSpace{}%
\AgdaKeyword{hiding}\AgdaSpace{}%
\AgdaSymbol{(}\AgdaFunction{sum}\AgdaSymbol{;}\AgdaSpace{}%
\AgdaFunction{slide}\AgdaSymbol{;}\AgdaSpace{}%
\AgdaFunction{backslide}\AgdaSymbol{)}\<%
\\
%
\>[2]\AgdaKeyword{open}\AgdaSpace{}%
\AgdaModule{Lang}\<%
\\
%
\>[2]\AgdaKeyword{open}\AgdaSpace{}%
\AgdaModule{SubWk}\<%
\\
%
\>[2]\AgdaKeyword{open}\AgdaSpace{}%
\AgdaModule{AD}\<%
\\
%
\>[2]\AgdaKeyword{open}\AgdaSpace{}%
\AgdaModule{Opt}\<%
\\
%
\>[2]\AgdaKeyword{open}\AgdaSpace{}%
\AgdaModule{BB}\<%
\\
%
\\[\AgdaEmptyExtraSkip]%
%
\>[2]\AgdaFunction{Env′}\AgdaSpace{}%
\AgdaSymbol{:}\AgdaSpace{}%
\AgdaDatatype{Ctx}\AgdaSpace{}%
\AgdaSymbol{→}\AgdaSpace{}%
\AgdaPrimitive{Set}\<%
\\
%
\>[2]\AgdaFunction{Env′}\AgdaSpace{}%
\AgdaBound{Γ}\AgdaSpace{}%
\AgdaSymbol{=}\AgdaSpace{}%
\AgdaFunction{Env}\AgdaSpace{}%
\AgdaBound{Γ}\AgdaSpace{}%
\AgdaBound{Γ}\<%
\end{code}

The key data structure that makes it possible to introduce local variables
is called \AF{Chain} which has two constructors.  The empty chain consists
of the names for all the variables in the context \AB{Γ}.  This represents the
case where no local variables have been introduced.  The \AC{\_▹\_} constructor
takes a chain in context \AB{Δ} and the array expression of shape \AB{p} in
the same context together with the variable name.  This produces the chain
in the context extended by two variables.  One variable is a place-holder
for the expression and the other variable is a placeholder for the derivative
of that expression.
\begin{code}%
%
\>[2]\AgdaKeyword{data}\AgdaSpace{}%
\AgdaDatatype{Chain}\AgdaSpace{}%
\AgdaSymbol{:}\AgdaSpace{}%
\AgdaDatatype{Ctx}\AgdaSpace{}%
\AgdaSymbol{→}\AgdaSpace{}%
\AgdaPrimitive{Set}\AgdaSpace{}%
\AgdaKeyword{where}\<%
\\
\>[2][@{}l@{\AgdaIndent{0}}]%
\>[4]\AgdaInductiveConstructor{ε}%
\>[9]\AgdaSymbol{:}\AgdaSpace{}%
\AgdaFunction{Sac.SEnv}\AgdaSpace{}%
\AgdaGeneralizable{Γ}\AgdaSpace{}%
\AgdaSymbol{→}\AgdaSpace{}%
\AgdaDatatype{Chain}\AgdaSpace{}%
\AgdaGeneralizable{Γ}\<%
\\
%
\>[4]\AgdaOperator{\AgdaInductiveConstructor{\AgdaUnderscore{}▹\AgdaUnderscore{}}}%
\>[9]\AgdaSymbol{:}\AgdaSpace{}%
\AgdaDatatype{Chain}\AgdaSpace{}%
\AgdaGeneralizable{Δ}\AgdaSpace{}%
\AgdaSymbol{→}\AgdaSpace{}%
\AgdaSymbol{(}\AgdaPostulate{String}\AgdaSpace{}%
\AgdaOperator{\AgdaFunction{×}}\AgdaSpace{}%
\AgdaDatatype{E}\AgdaSpace{}%
\AgdaGeneralizable{Δ}\AgdaSpace{}%
\AgdaSymbol{(}\AgdaInductiveConstructor{ar}\AgdaSpace{}%
\AgdaGeneralizable{p}\AgdaSymbol{))}\AgdaSpace{}%
\AgdaSymbol{→}\AgdaSpace{}%
\AgdaDatatype{Chain}\AgdaSpace{}%
\AgdaSymbol{(}\AgdaGeneralizable{Δ}\AgdaSpace{}%
\AgdaOperator{\AgdaInductiveConstructor{▹}}\AgdaSpace{}%
\AgdaInductiveConstructor{ar}\AgdaSpace{}%
\AgdaGeneralizable{p}\AgdaSpace{}%
\AgdaOperator{\AgdaInductiveConstructor{▹}}\AgdaSpace{}%
\AgdaInductiveConstructor{ar}\AgdaSpace{}%
\AgdaGeneralizable{p}\AgdaSymbol{)}\<%
\end{code}

The computation of the derivative in \AF{Chain}s follows the following
simple idea.  Consider the chain with two variables $a$ and
$b$ in the initial context \AB{Γ}, and two local variables $x$ and $y$.
Here is what happens when we compute the derivative of some expression
$e$ (that may depend on $a$, $b$, $x$, $y$) with some seed $s$ in the
empty $\delta_0$ environment. 

%\begin{table}
\begin{center}
\begin{tabular}{cc|cccc|l}
   $a$         &$b$         &$\partial{x}$& $x$         &$\partial{y}$&$y$       & \text{compute $\nabla\ e\ s\ \delta_0$}\\
   \hline
   $\delta_a$  &$\delta_b$  &-            & $\delta_x$  &-            &$\delta_y$& \text{assign $\delta_y$ to $\partial{y}$}\\
   $\delta_a$  &$\delta_b$  &-            & $\delta_x$  &$\delta_y$   &$\delta_y$& \text{compute $\nabla\ y_e\ \partial{y}$}\\
   $\delta'_a$ &$\delta'_b$ &-            & $\delta'_x$ &$\delta_y$   &$\delta_y$& \text{assign $\delta'_x$ to $\partial{x}$}\\
   $\delta'_a$ &$\delta'_b$ &-            & $\delta'_x$ &$\delta_y$   &$\delta_y$& \text{compute $\nabla\ x_e\ \partial{x}$}\\
   $\delta''_a$ &$\delta''_b$ &$\delta'_x$  & $\delta'_x$ &$\delta_y$   &$\delta_y$& \text{done}
\end{tabular}
\end{center}
%\end{table}

First of all, the computation of $e$ returns the environment $\delta$ that can
be found in the first line of the table.  Then we repeat the following steps while
traversing the chain backwards: we copy the $y$-th position of the $\delta$-environment
to the $\partial{y}$-th position, and we compute the expression $y_e$ that is assigned to $y$
($xx$ in this case) with the seed $\partial{y}$-th variable.  Just to clarify, the seed
is the variable $\partial{y}$ and not its value.  Then we repeat the same process
for $x$ and potentially all the other remaining local variables (not in this case) until
we hit the beginning of the chain.

At the end of the process we obtain an environment where derivatives for $a$ and
$b$ are expressed in terms of $\partial{x}$ and $\partial{y}$.  The remaining step
is to collect the values of $\partial{x}$ and $\partial{y}$ which can be found
at the corresponding positions in the $\delta$-environment.
\begin{code}[hide]%
%
\>[2]\AgdaKeyword{data}\AgdaSpace{}%
\AgdaDatatype{LCtx}\AgdaSpace{}%
\AgdaSymbol{:}\AgdaSpace{}%
\AgdaPrimitive{Set}\AgdaSpace{}%
\AgdaKeyword{where}\<%
\\
\>[2][@{}l@{\AgdaIndent{0}}]%
\>[4]\AgdaInductiveConstructor{[]}%
\>[8]\AgdaSymbol{:}\AgdaSpace{}%
\AgdaDatatype{LCtx}\<%
\\
%
\>[4]\AgdaOperator{\AgdaInductiveConstructor{\AgdaUnderscore{}◃\AgdaUnderscore{}}}\AgdaSpace{}%
\AgdaSymbol{:}\AgdaSpace{}%
\AgdaDatatype{IS}\AgdaSpace{}%
\AgdaSymbol{→}\AgdaSpace{}%
\AgdaDatatype{LCtx}\AgdaSpace{}%
\AgdaSymbol{→}\AgdaSpace{}%
\AgdaDatatype{LCtx}\<%
\\
%
\\[\AgdaEmptyExtraSkip]%
%
\>[2]\AgdaOperator{\AgdaFunction{\AgdaUnderscore{}<><\AgdaUnderscore{}}}\AgdaSpace{}%
\AgdaSymbol{:}\AgdaSpace{}%
\AgdaDatatype{Ctx}\AgdaSpace{}%
\AgdaSymbol{→}\AgdaSpace{}%
\AgdaDatatype{LCtx}\AgdaSpace{}%
\AgdaSymbol{→}\AgdaSpace{}%
\AgdaDatatype{Ctx}\<%
\\
%
\>[2]\AgdaBound{Γ}\AgdaSpace{}%
\AgdaOperator{\AgdaFunction{<><}}\AgdaSpace{}%
\AgdaInductiveConstructor{[]}\AgdaSpace{}%
\AgdaSymbol{=}\AgdaSpace{}%
\AgdaBound{Γ}\<%
\\
%
\>[2]\AgdaBound{Γ}\AgdaSpace{}%
\AgdaOperator{\AgdaFunction{<><}}\AgdaSpace{}%
\AgdaSymbol{(}\AgdaBound{x}\AgdaSpace{}%
\AgdaOperator{\AgdaInductiveConstructor{◃}}\AgdaSpace{}%
\AgdaBound{Δ}\AgdaSymbol{)}\AgdaSpace{}%
\AgdaSymbol{=}\AgdaSpace{}%
\AgdaSymbol{(}\AgdaBound{Γ}\AgdaSpace{}%
\AgdaOperator{\AgdaInductiveConstructor{▹}}\AgdaSpace{}%
\AgdaBound{x}\AgdaSymbol{)}\AgdaSpace{}%
\AgdaOperator{\AgdaFunction{<><}}\AgdaSpace{}%
\AgdaBound{Δ}\<%
\\
%
\\[\AgdaEmptyExtraSkip]%
%
\>[2]\AgdaKeyword{data}\AgdaSpace{}%
\AgdaDatatype{LEnv}\AgdaSpace{}%
\AgdaSymbol{:}\AgdaSpace{}%
\AgdaDatatype{LCtx}\AgdaSpace{}%
\AgdaSymbol{→}\AgdaSpace{}%
\AgdaDatatype{Ctx}\AgdaSpace{}%
\AgdaSymbol{→}\AgdaSpace{}%
\AgdaPrimitive{Set}\AgdaSpace{}%
\AgdaKeyword{where}\<%
\\
\>[2][@{}l@{\AgdaIndent{0}}]%
\>[4]\AgdaInductiveConstructor{[]}%
\>[8]\AgdaSymbol{:}\AgdaSpace{}%
\AgdaDatatype{LEnv}\AgdaSpace{}%
\AgdaInductiveConstructor{[]}\AgdaSpace{}%
\AgdaGeneralizable{Γ}\<%
\\
%
\>[4]\AgdaOperator{\AgdaInductiveConstructor{\AgdaUnderscore{}◃\AgdaUnderscore{}}}\AgdaSpace{}%
\AgdaSymbol{:}\AgdaSpace{}%
\AgdaSymbol{∀}\AgdaSpace{}%
\AgdaSymbol{\{}\AgdaBound{Δ′}\AgdaSymbol{\}}\AgdaSpace{}%
\AgdaSymbol{→}\AgdaSpace{}%
\AgdaDatatype{E}\AgdaSpace{}%
\AgdaGeneralizable{Γ}\AgdaSpace{}%
\AgdaSymbol{(}\AgdaInductiveConstructor{ar}\AgdaSpace{}%
\AgdaGeneralizable{s}\AgdaSymbol{)}\AgdaSpace{}%
\AgdaSymbol{→}\AgdaSpace{}%
\AgdaDatatype{LEnv}\AgdaSpace{}%
\AgdaBound{Δ′}\AgdaSpace{}%
\AgdaGeneralizable{Γ}\AgdaSpace{}%
\AgdaSymbol{→}\AgdaSpace{}%
\AgdaDatatype{LEnv}\AgdaSpace{}%
\AgdaSymbol{(}\AgdaInductiveConstructor{ar}\AgdaSpace{}%
\AgdaGeneralizable{s}\AgdaSpace{}%
\AgdaOperator{\AgdaInductiveConstructor{◃}}\AgdaSpace{}%
\AgdaBound{Δ′}\AgdaSymbol{)}\AgdaSpace{}%
\AgdaGeneralizable{Γ}\<%
\\
%
\\[\AgdaEmptyExtraSkip]%
%
\>[2]\AgdaKeyword{data}\AgdaSpace{}%
\AgdaDatatype{Postfix}\AgdaSpace{}%
\AgdaSymbol{:}\AgdaSpace{}%
\AgdaDatatype{Ctx}\AgdaSpace{}%
\AgdaSymbol{→}\AgdaSpace{}%
\AgdaDatatype{Ctx}\AgdaSpace{}%
\AgdaSymbol{→}\AgdaSpace{}%
\AgdaPrimitive{Set}\AgdaSpace{}%
\AgdaKeyword{where}\<%
\\
\>[2][@{}l@{\AgdaIndent{0}}]%
\>[4]\AgdaInductiveConstructor{done}\AgdaSpace{}%
\AgdaSymbol{:}\AgdaSpace{}%
\AgdaDatatype{Postfix}\AgdaSpace{}%
\AgdaInductiveConstructor{ε}\AgdaSpace{}%
\AgdaGeneralizable{Γ}\<%
\\
%
\>[4]\AgdaInductiveConstructor{next}\AgdaSpace{}%
\AgdaSymbol{:}\AgdaSpace{}%
\AgdaDatatype{Postfix}\AgdaSpace{}%
\AgdaGeneralizable{Γ}\AgdaSpace{}%
\AgdaGeneralizable{Δ}\AgdaSpace{}%
\AgdaSymbol{→}\AgdaSpace{}%
\AgdaDatatype{Postfix}\AgdaSpace{}%
\AgdaSymbol{(}\AgdaGeneralizable{Γ}\AgdaSpace{}%
\AgdaOperator{\AgdaInductiveConstructor{▹}}\AgdaSpace{}%
\AgdaInductiveConstructor{ar}\AgdaSpace{}%
\AgdaGeneralizable{s}\AgdaSymbol{)}\AgdaSpace{}%
\AgdaSymbol{(}\AgdaGeneralizable{Δ}\AgdaSpace{}%
\AgdaOperator{\AgdaInductiveConstructor{▹}}\AgdaSpace{}%
\AgdaInductiveConstructor{ar}\AgdaSpace{}%
\AgdaGeneralizable{s}\AgdaSymbol{)}\<%
\\
%
\\[\AgdaEmptyExtraSkip]%
%
\>[2]\AgdaFunction{double-ctx}\AgdaSpace{}%
\AgdaSymbol{:}\AgdaSpace{}%
\AgdaDatatype{Ctx}\AgdaSpace{}%
\AgdaSymbol{→}\AgdaSpace{}%
\AgdaDatatype{Ctx}\<%
\\
%
\>[2]\AgdaFunction{double-ctx}\AgdaSpace{}%
\AgdaInductiveConstructor{ε}\AgdaSpace{}%
\AgdaSymbol{=}\AgdaSpace{}%
\AgdaInductiveConstructor{ε}\<%
\\
%
\>[2]\AgdaFunction{double-ctx}\AgdaSpace{}%
\AgdaSymbol{(}\AgdaBound{Γ}\AgdaSpace{}%
\AgdaOperator{\AgdaInductiveConstructor{▹}}\AgdaSpace{}%
\AgdaBound{x}\AgdaSymbol{)}\AgdaSpace{}%
\AgdaSymbol{=}\AgdaSpace{}%
\AgdaFunction{double-ctx}\AgdaSpace{}%
\AgdaBound{Γ}\AgdaSpace{}%
\AgdaOperator{\AgdaInductiveConstructor{▹}}\AgdaSpace{}%
\AgdaBound{x}\AgdaSpace{}%
\AgdaOperator{\AgdaInductiveConstructor{▹}}\AgdaSpace{}%
\AgdaBound{x}\<%
\\
%
\\[\AgdaEmptyExtraSkip]%
%
\>[2]\AgdaFunction{chain-to-env}\AgdaSpace{}%
\AgdaSymbol{:}\AgdaSpace{}%
\AgdaDatatype{Chain}\AgdaSpace{}%
\AgdaGeneralizable{Γ}\AgdaSpace{}%
\AgdaSymbol{→}\AgdaSpace{}%
\AgdaRecord{Σ}\AgdaSpace{}%
\AgdaDatatype{Ctx}\AgdaSpace{}%
\AgdaSymbol{λ}\AgdaSpace{}%
\AgdaBound{Δ}\AgdaSpace{}%
\AgdaSymbol{→}\AgdaSpace{}%
\AgdaFunction{Env}\AgdaSpace{}%
\AgdaSymbol{(}\AgdaFunction{double-ctx}\AgdaSpace{}%
\AgdaBound{Δ}\AgdaSymbol{)}\AgdaSpace{}%
\AgdaGeneralizable{Γ}\AgdaSpace{}%
\AgdaOperator{\AgdaFunction{×}}\AgdaSpace{}%
\AgdaDatatype{Postfix}\AgdaSpace{}%
\AgdaSymbol{(}\AgdaFunction{double-ctx}\AgdaSpace{}%
\AgdaBound{Δ}\AgdaSymbol{)}\AgdaSpace{}%
\AgdaGeneralizable{Γ}\<%
\\
%
\>[2]\AgdaFunction{chain-to-env}\AgdaSpace{}%
\AgdaSymbol{(}\AgdaInductiveConstructor{ε}\AgdaSpace{}%
\AgdaBound{x}\AgdaSymbol{)}%
\>[23]\AgdaSymbol{=}\AgdaSpace{}%
\AgdaInductiveConstructor{ε}\AgdaSpace{}%
\AgdaOperator{\AgdaInductiveConstructor{,}}\AgdaSpace{}%
\AgdaInductiveConstructor{tt}\AgdaSpace{}%
\AgdaOperator{\AgdaInductiveConstructor{,}}\AgdaSpace{}%
\AgdaInductiveConstructor{done}\<%
\\
%
\>[2]\AgdaFunction{chain-to-env}\AgdaSpace{}%
\AgdaSymbol{(}\AgdaOperator{\AgdaInductiveConstructor{\AgdaUnderscore{}▹\AgdaUnderscore{}}}\AgdaSpace{}%
\AgdaSymbol{\{}\AgdaArgument{p}\AgdaSpace{}%
\AgdaSymbol{=}\AgdaSpace{}%
\AgdaBound{p}\AgdaSymbol{\}}\AgdaSpace{}%
\AgdaBound{c}\AgdaSpace{}%
\AgdaSymbol{(\AgdaUnderscore{}}\AgdaSpace{}%
\AgdaOperator{\AgdaInductiveConstructor{,}}\AgdaSpace{}%
\AgdaBound{x}\AgdaSymbol{))}\AgdaSpace{}%
\AgdaSymbol{=}\AgdaSpace{}%
\AgdaKeyword{let}\<%
\\
\>[2][@{}l@{\AgdaIndent{0}}]%
\>[4]\AgdaBound{Δ}\AgdaSpace{}%
\AgdaOperator{\AgdaInductiveConstructor{,}}\AgdaSpace{}%
\AgdaBound{ρ}\AgdaSpace{}%
\AgdaOperator{\AgdaInductiveConstructor{,}}\AgdaSpace{}%
\AgdaBound{po}\AgdaSpace{}%
\AgdaSymbol{=}\AgdaSpace{}%
\AgdaFunction{chain-to-env}\AgdaSpace{}%
\AgdaBound{c}\<%
\\
%
\>[4]\AgdaKeyword{in}\AgdaSpace{}%
\AgdaSymbol{(}\AgdaBound{Δ}\AgdaSpace{}%
\AgdaOperator{\AgdaInductiveConstructor{▹}}\AgdaSpace{}%
\AgdaInductiveConstructor{ar}\AgdaSpace{}%
\AgdaBound{p}\AgdaSymbol{)}\AgdaSpace{}%
\AgdaOperator{\AgdaInductiveConstructor{,}}\AgdaSpace{}%
\AgdaSymbol{((}\AgdaFunction{env-map}\AgdaSpace{}%
\AgdaSymbol{\{}\AgdaArgument{Γ}\AgdaSpace{}%
\AgdaSymbol{=}\AgdaSpace{}%
\AgdaFunction{double-ctx}\AgdaSpace{}%
\AgdaBound{Δ}\AgdaSymbol{\}}\AgdaSpace{}%
\AgdaSymbol{(}\AgdaOperator{\AgdaFunction{↑↑\AgdaUnderscore{}}}\AgdaSymbol{)}\AgdaSpace{}%
\AgdaBound{ρ}\AgdaSpace{}%
\AgdaOperator{\AgdaInductiveConstructor{,}}\AgdaSpace{}%
\AgdaInductiveConstructor{zero}\AgdaSymbol{)}\AgdaSpace{}%
\AgdaOperator{\AgdaInductiveConstructor{,}}\AgdaSpace{}%
\AgdaSymbol{(}\AgdaOperator{\AgdaFunction{↑}}\AgdaSpace{}%
\AgdaOperator{\AgdaFunction{↑}}\AgdaSpace{}%
\AgdaBound{x}\AgdaSymbol{))}\AgdaSpace{}%
\AgdaOperator{\AgdaInductiveConstructor{,}}\AgdaSpace{}%
\AgdaSymbol{(}\AgdaInductiveConstructor{next}\AgdaSpace{}%
\AgdaSymbol{(}\AgdaInductiveConstructor{next}\AgdaSpace{}%
\AgdaBound{po}\AgdaSymbol{))}\<%
\\
%
\\[\AgdaEmptyExtraSkip]%
%
\>[2]\AgdaFunction{pstep}\AgdaSpace{}%
\AgdaSymbol{:}\AgdaSpace{}%
\AgdaSymbol{∀}\AgdaSpace{}%
\AgdaSymbol{\{}\AgdaBound{Δ′}\AgdaSymbol{\}}\AgdaSpace{}%
\AgdaSymbol{→}\AgdaSpace{}%
\AgdaDatatype{Postfix}\AgdaSpace{}%
\AgdaSymbol{((}\AgdaGeneralizable{Δ}\AgdaSpace{}%
\AgdaOperator{\AgdaInductiveConstructor{▹}}\AgdaSpace{}%
\AgdaInductiveConstructor{ar}\AgdaSpace{}%
\AgdaGeneralizable{s}\AgdaSymbol{)}\AgdaSpace{}%
\AgdaOperator{\AgdaFunction{<><}}\AgdaSpace{}%
\AgdaBound{Δ′}\AgdaSymbol{)}\AgdaSpace{}%
\AgdaGeneralizable{Γ}\AgdaSpace{}%
\AgdaSymbol{→}\AgdaSpace{}%
\AgdaDatatype{Postfix}\AgdaSpace{}%
\AgdaSymbol{(}\AgdaGeneralizable{Δ}\AgdaSpace{}%
\AgdaOperator{\AgdaFunction{<><}}\AgdaSpace{}%
\AgdaSymbol{(}\AgdaInductiveConstructor{ar}\AgdaSpace{}%
\AgdaGeneralizable{s}\AgdaSpace{}%
\AgdaOperator{\AgdaInductiveConstructor{◃}}\AgdaSpace{}%
\AgdaBound{Δ′}\AgdaSymbol{))}\AgdaSpace{}%
\AgdaGeneralizable{Γ}\<%
\\
%
\>[2]\AgdaFunction{pstep}\AgdaSpace{}%
\AgdaSymbol{\{}\AgdaArgument{Δ′}\AgdaSpace{}%
\AgdaSymbol{=}\AgdaSpace{}%
\AgdaInductiveConstructor{[]}\AgdaSymbol{\}}\AgdaSpace{}%
\AgdaSymbol{(}\AgdaInductiveConstructor{next}\AgdaSpace{}%
\AgdaBound{p}\AgdaSymbol{)}\AgdaSpace{}%
\AgdaSymbol{=}\AgdaSpace{}%
\AgdaInductiveConstructor{next}\AgdaSpace{}%
\AgdaBound{p}\<%
\\
%
\>[2]\AgdaFunction{pstep}\AgdaSpace{}%
\AgdaSymbol{\{}\AgdaArgument{Δ′}\AgdaSpace{}%
\AgdaSymbol{=}\AgdaSpace{}%
\AgdaBound{x}\AgdaSpace{}%
\AgdaOperator{\AgdaInductiveConstructor{◃}}\AgdaSpace{}%
\AgdaBound{Δ′}\AgdaSymbol{\}}\AgdaSpace{}%
\AgdaBound{p}\AgdaSpace{}%
\AgdaSymbol{=}\AgdaSpace{}%
\AgdaBound{p}\<%
\\
%
\\[\AgdaEmptyExtraSkip]%
%
\>[2]\AgdaFunction{post-var}\AgdaSpace{}%
\AgdaSymbol{:}\AgdaSpace{}%
\AgdaSymbol{∀}\AgdaSpace{}%
\AgdaSymbol{\{}\AgdaBound{Δ′}\AgdaSymbol{\}}\AgdaSpace{}%
\AgdaSymbol{→}\AgdaSpace{}%
\AgdaDatatype{Postfix}\AgdaSpace{}%
\AgdaSymbol{(}\AgdaGeneralizable{Δ}\AgdaSpace{}%
\AgdaOperator{\AgdaFunction{<><}}\AgdaSpace{}%
\AgdaBound{Δ′}\AgdaSymbol{)}\AgdaSpace{}%
\AgdaGeneralizable{Γ}\AgdaSpace{}%
\AgdaSymbol{→}\AgdaSpace{}%
\AgdaGeneralizable{is}\AgdaSpace{}%
\AgdaOperator{\AgdaDatatype{∈}}\AgdaSpace{}%
\AgdaGeneralizable{Δ}\AgdaSpace{}%
\AgdaSymbol{→}\AgdaSpace{}%
\AgdaGeneralizable{is}\AgdaSpace{}%
\AgdaOperator{\AgdaDatatype{∈}}\AgdaSpace{}%
\AgdaGeneralizable{Γ}\<%
\\
%
\>[2]\AgdaFunction{post-var}\AgdaSpace{}%
\AgdaSymbol{\{}\AgdaArgument{Δ′}\AgdaSpace{}%
\AgdaSymbol{=}\AgdaSpace{}%
\AgdaInductiveConstructor{[]}\AgdaSymbol{\}}\AgdaSpace{}%
\AgdaSymbol{(}\AgdaInductiveConstructor{next}\AgdaSpace{}%
\AgdaBound{p}\AgdaSymbol{)}\AgdaSpace{}%
\AgdaInductiveConstructor{v₀}\AgdaSpace{}%
\AgdaSymbol{=}\AgdaSpace{}%
\AgdaInductiveConstructor{v₀}\<%
\\
%
\>[2]\AgdaFunction{post-var}\AgdaSpace{}%
\AgdaSymbol{\{}\AgdaArgument{Δ′}\AgdaSpace{}%
\AgdaSymbol{=}\AgdaSpace{}%
\AgdaInductiveConstructor{[]}\AgdaSymbol{\}}\AgdaSpace{}%
\AgdaSymbol{(}\AgdaInductiveConstructor{next}\AgdaSpace{}%
\AgdaBound{p}\AgdaSymbol{)}\AgdaSpace{}%
\AgdaSymbol{(}\AgdaInductiveConstructor{vₛ}\AgdaSpace{}%
\AgdaBound{x}\AgdaSymbol{)}\AgdaSpace{}%
\AgdaSymbol{=}\AgdaSpace{}%
\AgdaInductiveConstructor{vₛ}\AgdaSpace{}%
\AgdaSymbol{(}\AgdaFunction{post-var}\AgdaSpace{}%
\AgdaSymbol{\{}\AgdaArgument{Δ′}\AgdaSpace{}%
\AgdaSymbol{=}\AgdaSpace{}%
\AgdaInductiveConstructor{[]}\AgdaSymbol{\}}\AgdaSpace{}%
\AgdaBound{p}\AgdaSpace{}%
\AgdaBound{x}\AgdaSymbol{)}\<%
\\
%
\>[2]\AgdaFunction{post-var}\AgdaSpace{}%
\AgdaSymbol{\{}\AgdaArgument{Δ′}\AgdaSpace{}%
\AgdaSymbol{=}\AgdaSpace{}%
\AgdaBound{is}\AgdaSpace{}%
\AgdaOperator{\AgdaInductiveConstructor{◃}}\AgdaSpace{}%
\AgdaBound{Δ′}\AgdaSymbol{\}}\AgdaSpace{}%
\AgdaBound{p}\AgdaSpace{}%
\AgdaBound{x}\AgdaSpace{}%
\AgdaSymbol{=}\AgdaSpace{}%
\AgdaFunction{post-var}\AgdaSpace{}%
\AgdaSymbol{\{}\AgdaArgument{Δ′}\AgdaSpace{}%
\AgdaSymbol{=}\AgdaSpace{}%
\AgdaBound{Δ′}\AgdaSymbol{\}}\AgdaSpace{}%
\AgdaBound{p}\AgdaSpace{}%
\AgdaSymbol{(}\AgdaInductiveConstructor{vₛ}\AgdaSpace{}%
\AgdaBound{x}\AgdaSymbol{)}\<%
\\
%
\\[\AgdaEmptyExtraSkip]%
%
\>[2]\AgdaFunction{no-ix}\AgdaSpace{}%
\AgdaSymbol{:}\AgdaSpace{}%
\AgdaInductiveConstructor{ix}\AgdaSpace{}%
\AgdaGeneralizable{s}\AgdaSpace{}%
\AgdaOperator{\AgdaDatatype{∈}}\AgdaSpace{}%
\AgdaGeneralizable{Δ}\AgdaSpace{}%
\AgdaSymbol{→}\AgdaSpace{}%
\AgdaOperator{\AgdaFunction{¬}}\AgdaSpace{}%
\AgdaDatatype{Postfix}\AgdaSpace{}%
\AgdaGeneralizable{Δ}\AgdaSpace{}%
\AgdaGeneralizable{Γ}\<%
\\
%
\>[2]\AgdaFunction{no-ix}\AgdaSpace{}%
\AgdaInductiveConstructor{v₀}\AgdaSpace{}%
\AgdaSymbol{=}\AgdaSpace{}%
\AgdaSymbol{λ}\AgdaSpace{}%
\AgdaSymbol{()}\<%
\\
%
\>[2]\AgdaFunction{no-ix}\AgdaSpace{}%
\AgdaSymbol{(}\AgdaInductiveConstructor{vₛ}\AgdaSpace{}%
\AgdaBound{v}\AgdaSymbol{)}\AgdaSpace{}%
\AgdaSymbol{(}\AgdaInductiveConstructor{next}\AgdaSpace{}%
\AgdaBound{p}\AgdaSymbol{)}\AgdaSpace{}%
\AgdaSymbol{=}\AgdaSpace{}%
\AgdaFunction{no-ix}\AgdaSpace{}%
\AgdaBound{v}\AgdaSpace{}%
\AgdaBound{p}\<%
\\
%
\\[\AgdaEmptyExtraSkip]%
%
\>[2]\AgdaFunction{post-fish}\AgdaSpace{}%
\AgdaSymbol{:}\AgdaSpace{}%
\AgdaSymbol{∀}\AgdaSpace{}%
\AgdaBound{Δ′}\AgdaSpace{}%
\AgdaSymbol{→}\AgdaSpace{}%
\AgdaGeneralizable{is}\AgdaSpace{}%
\AgdaOperator{\AgdaDatatype{∈}}\AgdaSpace{}%
\AgdaGeneralizable{Δ}\AgdaSpace{}%
\AgdaSymbol{→}\AgdaSpace{}%
\AgdaGeneralizable{is}\AgdaSpace{}%
\AgdaOperator{\AgdaDatatype{∈}}\AgdaSpace{}%
\AgdaSymbol{(}\AgdaGeneralizable{Δ}\AgdaSpace{}%
\AgdaOperator{\AgdaFunction{<><}}\AgdaSpace{}%
\AgdaBound{Δ′}\AgdaSymbol{)}\<%
\\
%
\>[2]\AgdaFunction{post-fish}\AgdaSpace{}%
\AgdaInductiveConstructor{[]}\AgdaSpace{}%
\AgdaBound{v}\AgdaSpace{}%
\AgdaSymbol{=}\AgdaSpace{}%
\AgdaBound{v}\<%
\\
%
\>[2]\AgdaFunction{post-fish}\AgdaSpace{}%
\AgdaSymbol{(}\AgdaBound{x}\AgdaSpace{}%
\AgdaOperator{\AgdaInductiveConstructor{◃}}\AgdaSpace{}%
\AgdaBound{Δ′}\AgdaSymbol{)}\AgdaSpace{}%
\AgdaBound{v}\AgdaSpace{}%
\AgdaSymbol{=}\AgdaSpace{}%
\AgdaFunction{post-fish}\AgdaSpace{}%
\AgdaBound{Δ′}\AgdaSpace{}%
\AgdaSymbol{(}\AgdaInductiveConstructor{vₛ}\AgdaSpace{}%
\AgdaBound{v}\AgdaSymbol{)}\<%
\\
%
\\[\AgdaEmptyExtraSkip]%
%
\>[2]\AgdaFunction{gradc}\AgdaSpace{}%
\AgdaSymbol{:}\AgdaSpace{}%
\AgdaSymbol{∀}%
\>[2736I]\AgdaSymbol{\{}\AgdaBound{Δ′}\AgdaSymbol{\}}\AgdaSpace{}%
\AgdaSymbol{→}\AgdaSpace{}%
\AgdaFunction{Env}\AgdaSpace{}%
\AgdaSymbol{(}\AgdaFunction{double-ctx}\AgdaSpace{}%
\AgdaGeneralizable{Δ}\AgdaSymbol{)}\AgdaSpace{}%
\AgdaGeneralizable{Γ}\AgdaSpace{}%
\AgdaSymbol{→}\AgdaSpace{}%
\AgdaDatatype{LEnv}\AgdaSpace{}%
\AgdaBound{Δ′}\AgdaSpace{}%
\AgdaGeneralizable{Γ}\<%
\\
\>[.][@{}l@{}]\<[2736I]%
\>[12]\AgdaSymbol{→}\AgdaSpace{}%
\AgdaDatatype{Postfix}\AgdaSpace{}%
\AgdaSymbol{((}\AgdaFunction{double-ctx}\AgdaSpace{}%
\AgdaGeneralizable{Δ}\AgdaSymbol{)}\AgdaSpace{}%
\AgdaOperator{\AgdaFunction{<><}}\AgdaSpace{}%
\AgdaBound{Δ′}\AgdaSymbol{)}\AgdaSpace{}%
\AgdaGeneralizable{Γ}\AgdaSpace{}%
\AgdaSymbol{→}%
\>[51]\AgdaFunction{Env′}\AgdaSpace{}%
\AgdaGeneralizable{Γ}\AgdaSpace{}%
\AgdaSymbol{→}\AgdaSpace{}%
\AgdaFunction{Env′}\AgdaSpace{}%
\AgdaGeneralizable{Γ}\<%
\\
%
\>[2]\AgdaFunction{gradc}\AgdaSpace{}%
\AgdaSymbol{\{}\AgdaInductiveConstructor{ε}\AgdaSymbol{\}}%
\>[19]\AgdaSymbol{\{}\AgdaBound{Γ}\AgdaSymbol{\}}\AgdaSpace{}%
\AgdaSymbol{\{}\AgdaBound{Δ′}\AgdaSymbol{\}}\AgdaSpace{}%
\AgdaBound{ρ}\AgdaSpace{}%
\AgdaBound{ρ′}\AgdaSpace{}%
\AgdaBound{p}\AgdaSpace{}%
\AgdaBound{δ}\AgdaSpace{}%
\AgdaSymbol{=}\AgdaSpace{}%
\AgdaBound{δ}\<%
\\
%
\>[2]\AgdaFunction{gradc}\AgdaSpace{}%
\AgdaSymbol{\{}\AgdaBound{Δ}\AgdaSpace{}%
\AgdaOperator{\AgdaInductiveConstructor{▹}}\AgdaSpace{}%
\AgdaInductiveConstructor{ix}\AgdaSpace{}%
\AgdaBound{x}\AgdaSymbol{\}}\AgdaSpace{}%
\AgdaSymbol{\{}\AgdaBound{Γ}\AgdaSymbol{\}}\AgdaSpace{}%
\AgdaSymbol{\{}\AgdaBound{Δ′}\AgdaSymbol{\}}\AgdaSpace{}%
\AgdaBound{ρ}\AgdaSpace{}%
\AgdaBound{ρ′}\AgdaSpace{}%
\AgdaBound{p}\AgdaSpace{}%
\AgdaBound{δ}\AgdaSpace{}%
\AgdaSymbol{=}\AgdaSpace{}%
\AgdaFunction{⊥-elim}\AgdaSpace{}%
\AgdaSymbol{(}\AgdaFunction{no-ix}\AgdaSpace{}%
\AgdaSymbol{(}\AgdaFunction{post-fish}\AgdaSpace{}%
\AgdaBound{Δ′}\AgdaSpace{}%
\AgdaInductiveConstructor{v₀}\AgdaSymbol{)}\AgdaSpace{}%
\AgdaBound{p}\AgdaSymbol{)}\<%
\\
%
\>[2]\AgdaFunction{gradc}\AgdaSpace{}%
\AgdaSymbol{\{}\AgdaBound{Δ}\AgdaSpace{}%
\AgdaOperator{\AgdaInductiveConstructor{▹}}\AgdaSpace{}%
\AgdaInductiveConstructor{ar}\AgdaSpace{}%
\AgdaBound{x}\AgdaSymbol{\}}\AgdaSpace{}%
\AgdaSymbol{\{}\AgdaBound{Γ}\AgdaSymbol{\}}\AgdaSpace{}%
\AgdaSymbol{\{}\AgdaBound{Δ′}\AgdaSymbol{\}}\AgdaSpace{}%
\AgdaSymbol{((}\AgdaBound{ρ}\AgdaSpace{}%
\AgdaOperator{\AgdaInductiveConstructor{,}}\AgdaSpace{}%
\AgdaBound{z}\AgdaSymbol{)}\AgdaSpace{}%
\AgdaOperator{\AgdaInductiveConstructor{,}}\AgdaSpace{}%
\AgdaBound{e}\AgdaSymbol{)}\AgdaSpace{}%
\AgdaBound{ρ′}\AgdaSpace{}%
\AgdaBound{p}\AgdaSpace{}%
\AgdaBound{δ}\AgdaSpace{}%
\AgdaSymbol{=}\<%
\\
\>[2][@{}l@{\AgdaIndent{0}}]%
\>[4]\AgdaKeyword{let}\<%
\\
%
\>[4]\AgdaBound{ve}\AgdaSpace{}%
\AgdaSymbol{=}\AgdaSpace{}%
\AgdaFunction{post-var}\AgdaSpace{}%
\AgdaSymbol{\{}\AgdaArgument{Δ′}\AgdaSpace{}%
\AgdaSymbol{=}\AgdaSpace{}%
\AgdaBound{Δ′}\AgdaSymbol{\}}\AgdaSpace{}%
\AgdaBound{p}\AgdaSpace{}%
\AgdaInductiveConstructor{v₀}%
\>[34]\AgdaComment{--\ variable\ for\ e\ in\ Γ}\<%
\\
%
\>[4]\AgdaBound{vz}\AgdaSpace{}%
\AgdaSymbol{=}\AgdaSpace{}%
\AgdaFunction{post-var}\AgdaSpace{}%
\AgdaSymbol{\{}\AgdaArgument{Δ′}\AgdaSpace{}%
\AgdaSymbol{=}\AgdaSpace{}%
\AgdaBound{Δ′}\AgdaSymbol{\}}\AgdaSpace{}%
\AgdaBound{p}\AgdaSpace{}%
\AgdaInductiveConstructor{v₁}%
\>[34]\AgdaComment{--\ variable\ for\ z\ in\ Γ}\<%
\\
%
\>[4]\AgdaBound{s}%
\>[7]\AgdaSymbol{=}\AgdaSpace{}%
\AgdaFunction{env-ix}\AgdaSpace{}%
\AgdaBound{δ}\AgdaSpace{}%
\AgdaBound{ve}\<%
\\
%
\>[4]\AgdaBound{δ₁}\AgdaSpace{}%
\AgdaSymbol{=}\AgdaSpace{}%
\AgdaFunction{update}\AgdaSpace{}%
\AgdaBound{δ}\AgdaSpace{}%
\AgdaBound{vz}\AgdaSpace{}%
\AgdaSymbol{(}\AgdaFunction{const}\AgdaSpace{}%
\AgdaBound{s}\AgdaSymbol{)}%
\>[34]\AgdaComment{--\ save\ s\ in\ the\ z's\ position}\<%
\\
%
\>[4]\AgdaBound{δ₂}\AgdaSpace{}%
\AgdaSymbol{=}\AgdaSpace{}%
\AgdaFunction{∇}\AgdaSpace{}%
\AgdaBound{e}\AgdaSpace{}%
\AgdaSymbol{(}\AgdaInductiveConstructor{var}\AgdaSpace{}%
\AgdaBound{vz}\AgdaSymbol{)}\AgdaSpace{}%
\AgdaBound{δ₁}%
\>[34]\AgdaComment{--\ use\ vz\ position\ as\ seed}\<%
\\
%
\>[4]\AgdaKeyword{in}\AgdaSpace{}%
\AgdaFunction{gradc}\AgdaSpace{}%
\AgdaSymbol{\{}\AgdaBound{Δ}\AgdaSymbol{\}}\AgdaSpace{}%
\AgdaBound{ρ}\AgdaSpace{}%
\AgdaSymbol{(}\AgdaBound{z}\AgdaSpace{}%
\AgdaOperator{\AgdaInductiveConstructor{◃}}\AgdaSpace{}%
\AgdaSymbol{(}\AgdaBound{e}\AgdaSpace{}%
\AgdaOperator{\AgdaInductiveConstructor{◃}}\AgdaSpace{}%
\AgdaBound{ρ′}\AgdaSymbol{))}\AgdaSpace{}%
\AgdaSymbol{(}\AgdaFunction{pstep}\AgdaSpace{}%
\AgdaSymbol{\{}\AgdaArgument{Δ′}\AgdaSpace{}%
\AgdaSymbol{=}\AgdaSpace{}%
\AgdaInductiveConstructor{ar}\AgdaSpace{}%
\AgdaBound{x}\AgdaSpace{}%
\AgdaOperator{\AgdaInductiveConstructor{◃}}\AgdaSpace{}%
\AgdaBound{Δ′}\AgdaSymbol{\}}\AgdaSpace{}%
\AgdaSymbol{(}\AgdaFunction{pstep}\AgdaSpace{}%
\AgdaSymbol{\{}\AgdaArgument{Δ′}\AgdaSpace{}%
\AgdaSymbol{=}\AgdaSpace{}%
\AgdaBound{Δ′}\AgdaSymbol{\}}\AgdaSpace{}%
\AgdaBound{p}\AgdaSymbol{))}\AgdaSpace{}%
\AgdaBound{δ₂}\<%
\\
%
\\[\AgdaEmptyExtraSkip]%
%
\>[2]\AgdaFunction{chain-grad}\AgdaSpace{}%
\AgdaSymbol{:}\AgdaSpace{}%
\AgdaDatatype{Chain}\AgdaSpace{}%
\AgdaSymbol{(}\AgdaGeneralizable{Γ}\AgdaSpace{}%
\AgdaOperator{\AgdaInductiveConstructor{▹}}\AgdaSpace{}%
\AgdaInductiveConstructor{ar}\AgdaSpace{}%
\AgdaGeneralizable{s}\AgdaSymbol{)}\AgdaSpace{}%
\AgdaSymbol{→}\AgdaSpace{}%
\AgdaDatatype{E}\AgdaSpace{}%
\AgdaSymbol{(}\AgdaGeneralizable{Γ}\AgdaSpace{}%
\AgdaOperator{\AgdaInductiveConstructor{▹}}\AgdaSpace{}%
\AgdaInductiveConstructor{ar}\AgdaSpace{}%
\AgdaGeneralizable{s}\AgdaSymbol{)}\AgdaSpace{}%
\AgdaSymbol{(}\AgdaInductiveConstructor{ar}\AgdaSpace{}%
\AgdaGeneralizable{s}\AgdaSymbol{)}\AgdaSpace{}%
\AgdaSymbol{→}\AgdaSpace{}%
\AgdaFunction{Env′}\AgdaSpace{}%
\AgdaSymbol{(}\AgdaGeneralizable{Γ}\AgdaSpace{}%
\AgdaOperator{\AgdaInductiveConstructor{▹}}\AgdaSpace{}%
\AgdaInductiveConstructor{ar}\AgdaSpace{}%
\AgdaGeneralizable{s}\AgdaSymbol{)}\<%
\\
%
\>[2]\AgdaFunction{chain-grad}\AgdaSpace{}%
\AgdaSymbol{\{}\AgdaBound{Γ}\AgdaSymbol{\}}\AgdaSpace{}%
\AgdaSymbol{\{}\AgdaBound{s}\AgdaSymbol{\}}\AgdaSpace{}%
\AgdaBound{c}\AgdaSpace{}%
\AgdaBound{seed}\AgdaSpace{}%
\AgdaSymbol{=}\AgdaSpace{}%
\AgdaKeyword{let}\<%
\\
\>[2][@{}l@{\AgdaIndent{0}}]%
\>[4]\AgdaComment{--\ Well,\ this\ is\ a\ choice\ I\ suppose}\<%
\\
%
\>[4]\AgdaComment{--δ\ =\ ∇\ seed\ one\ (env-imap\ \{Γ\ =\ Γ\ ▹\ ar\ s\}\ (const\ zero))}\<%
\\
%
\>[4]\AgdaBound{δ}\AgdaSpace{}%
\AgdaSymbol{=}\AgdaSpace{}%
\AgdaFunction{env-imap}\AgdaSpace{}%
\AgdaSymbol{\{}\AgdaArgument{Γ}\AgdaSpace{}%
\AgdaSymbol{=}\AgdaSpace{}%
\AgdaBound{Γ}\AgdaSymbol{\}}\AgdaSpace{}%
\AgdaSymbol{(}\AgdaFunction{const}\AgdaSpace{}%
\AgdaInductiveConstructor{zero}\AgdaSymbol{)}\AgdaSpace{}%
\AgdaOperator{\AgdaInductiveConstructor{,}}\AgdaSpace{}%
\AgdaBound{seed}\<%
\\
%
\>[4]\AgdaBound{Δ}\AgdaSpace{}%
\AgdaOperator{\AgdaInductiveConstructor{,}}\AgdaSpace{}%
\AgdaBound{ρ}\AgdaSpace{}%
\AgdaOperator{\AgdaInductiveConstructor{,}}\AgdaSpace{}%
\AgdaBound{po}\AgdaSpace{}%
\AgdaSymbol{=}\AgdaSpace{}%
\AgdaFunction{chain-to-env}\AgdaSpace{}%
\AgdaBound{c}\<%
\\
%
\>[4]\AgdaKeyword{in}\AgdaSpace{}%
\AgdaFunction{env-map}\AgdaSpace{}%
\AgdaSymbol{\{}\AgdaArgument{Γ}\AgdaSpace{}%
\AgdaSymbol{=}\AgdaSpace{}%
\AgdaBound{Γ}\AgdaSpace{}%
\AgdaOperator{\AgdaInductiveConstructor{▹}}\AgdaSpace{}%
\AgdaInductiveConstructor{ar}\AgdaSpace{}%
\AgdaBound{s}\AgdaSymbol{\}}\AgdaSpace{}%
\AgdaSymbol{(}\AgdaFunction{multiopt}\AgdaSpace{}%
\AgdaNumber{10}\AgdaSymbol{)}\AgdaSpace{}%
\AgdaOperator{\AgdaFunction{\$}}\AgdaSpace{}%
\AgdaFunction{gradc}\AgdaSpace{}%
\AgdaBound{ρ}\AgdaSpace{}%
\AgdaInductiveConstructor{[]}\AgdaSpace{}%
\AgdaBound{po}\AgdaSpace{}%
\AgdaBound{δ}\<%
\\
%
\\[\AgdaEmptyExtraSkip]%
%
\>[2]\AgdaFunction{chain-sac-ctx}\AgdaSpace{}%
\AgdaSymbol{:}\AgdaSpace{}%
\AgdaDatatype{Chain}\AgdaSpace{}%
\AgdaGeneralizable{Γ}\AgdaSpace{}%
\AgdaSymbol{→}\AgdaSpace{}%
\AgdaFunction{Sac.SEnv}\AgdaSpace{}%
\AgdaGeneralizable{Γ}\<%
\\
%
\>[2]\AgdaFunction{chain-sac-ctx}\AgdaSpace{}%
\AgdaSymbol{(}\AgdaInductiveConstructor{ε}\AgdaSpace{}%
\AgdaBound{x}\AgdaSymbol{)}\AgdaSpace{}%
\AgdaSymbol{=}\AgdaSpace{}%
\AgdaBound{x}\<%
\\
%
\>[2]\AgdaFunction{chain-sac-ctx}\AgdaSpace{}%
\AgdaSymbol{(}\AgdaBound{c}\AgdaSpace{}%
\AgdaOperator{\AgdaInductiveConstructor{▹}}\AgdaSpace{}%
\AgdaSymbol{(}\AgdaBound{v}\AgdaSpace{}%
\AgdaOperator{\AgdaInductiveConstructor{,}}\AgdaSpace{}%
\AgdaSymbol{\AgdaUnderscore{}))}\AgdaSpace{}%
\AgdaSymbol{=}\AgdaSpace{}%
\AgdaFunction{chain-sac-ctx}\AgdaSpace{}%
\AgdaBound{c}\AgdaSpace{}%
\AgdaOperator{\AgdaFunction{,,}}\AgdaSpace{}%
\AgdaSymbol{(}\AgdaString{"∂/∂"}\AgdaSpace{}%
\AgdaOperator{\AgdaFunction{++}}\AgdaSpace{}%
\AgdaBound{v}\AgdaSymbol{)}\AgdaSpace{}%
\AgdaOperator{\AgdaFunction{,,}}\AgdaSpace{}%
\AgdaBound{v}\<%
\\
\>[0]\<%
\\
%
\>[2]\AgdaFunction{filter-grad}\AgdaSpace{}%
\AgdaSymbol{:}\AgdaSpace{}%
\AgdaDatatype{Chain}\AgdaSpace{}%
\AgdaGeneralizable{Γ}\AgdaSpace{}%
\AgdaSymbol{→}\AgdaSpace{}%
\AgdaFunction{Sac.SEnv}\AgdaSpace{}%
\AgdaGeneralizable{Γ}\AgdaSpace{}%
\AgdaSymbol{→}\AgdaSpace{}%
\AgdaDatatype{List}\AgdaSpace{}%
\AgdaPostulate{String}\<%
\\
%
\>[2]\AgdaFunction{filter-grad}\AgdaSpace{}%
\AgdaSymbol{(}\AgdaInductiveConstructor{ε}\AgdaSpace{}%
\AgdaBound{x}\AgdaSymbol{)}%
\>[22]\AgdaBound{δ}\AgdaSpace{}%
\AgdaSymbol{=}\AgdaSpace{}%
\AgdaFunction{Sac.env-rev-list}\AgdaSpace{}%
\AgdaBound{δ}\<%
\\
%
\>[2]\AgdaFunction{filter-grad}\AgdaSpace{}%
\AgdaSymbol{(}\AgdaBound{c}\AgdaSpace{}%
\AgdaOperator{\AgdaInductiveConstructor{▹}}\AgdaSpace{}%
\AgdaSymbol{\AgdaUnderscore{})}\AgdaSpace{}%
\AgdaSymbol{((}\AgdaBound{δ}\AgdaSpace{}%
\AgdaOperator{\AgdaInductiveConstructor{,}}\AgdaSpace{}%
\AgdaSymbol{\AgdaUnderscore{})}\AgdaOperator{\AgdaInductiveConstructor{,}}\AgdaSpace{}%
\AgdaBound{x}\AgdaSymbol{)}\AgdaSpace{}%
\AgdaSymbol{=}\AgdaSpace{}%
\AgdaBound{x}\AgdaSpace{}%
\AgdaOperator{\AgdaInductiveConstructor{∷}}\AgdaSpace{}%
\AgdaFunction{filter-grad}\AgdaSpace{}%
\AgdaBound{c}\AgdaSpace{}%
\AgdaBound{δ}\<%
\\
%
\\[\AgdaEmptyExtraSkip]%
%
\>[2]\AgdaFunction{chain-grad-sac}\AgdaSpace{}%
\AgdaSymbol{:}\AgdaSpace{}%
\AgdaDatatype{Chain}\AgdaSpace{}%
\AgdaGeneralizable{Γ}\AgdaSpace{}%
\AgdaSymbol{→}\AgdaSpace{}%
\AgdaFunction{Env′}\AgdaSpace{}%
\AgdaGeneralizable{Γ}\AgdaSpace{}%
\AgdaSymbol{→}\AgdaSpace{}%
\AgdaPostulate{String}\<%
\\
%
\>[2]\AgdaFunction{chain-grad-sac}\AgdaSpace{}%
\AgdaSymbol{\{}\AgdaBound{Γ}\AgdaSymbol{\}}\AgdaSpace{}%
\AgdaBound{c}\AgdaSpace{}%
\AgdaBound{δ}\AgdaSpace{}%
\AgdaSymbol{=}\AgdaSpace{}%
\AgdaKeyword{let}\<%
\\
\>[2][@{}l@{\AgdaIndent{0}}]%
\>[4]\AgdaBound{vars}\AgdaSpace{}%
\AgdaSymbol{=}\AgdaSpace{}%
\AgdaFunction{chain-sac-ctx}\AgdaSpace{}%
\AgdaBound{c}\<%
\\
%
\>[4]\AgdaBound{vals}\AgdaSpace{}%
\AgdaSymbol{=}\AgdaSpace{}%
\AgdaFunction{Sac.env-sac}\AgdaSpace{}%
\AgdaSymbol{\{}\AgdaBound{Γ}\AgdaSymbol{\}}\AgdaSpace{}%
\AgdaBound{δ}\AgdaSpace{}%
\AgdaBound{vars}\<%
\\
%
\>[4]\AgdaBound{assignments}\AgdaSpace{}%
\AgdaSymbol{=}\AgdaSpace{}%
\AgdaFunction{filter-grad}\AgdaSpace{}%
\AgdaBound{c}\AgdaSpace{}%
\AgdaOperator{\AgdaFunction{\$}}\AgdaSpace{}%
\AgdaFunction{Sac.zip-env}\AgdaSpace{}%
\AgdaSymbol{(}\AgdaFunction{printf}\AgdaSpace{}%
\AgdaString{"∂/∂\%s\ =\ \%s;"}\AgdaSymbol{)}\AgdaSpace{}%
\AgdaBound{vars}\AgdaSpace{}%
\AgdaBound{vals}\<%
\\
%
\>[4]\AgdaKeyword{in}\AgdaSpace{}%
\AgdaFunction{intersperse}\AgdaSpace{}%
\AgdaString{"\textbackslash{}n"}\AgdaSpace{}%
\AgdaBound{assignments}\<%
\\
%
\\[\AgdaEmptyExtraSkip]%
%
\>[2]\AgdaFunction{chain-sac-l}\AgdaSpace{}%
\AgdaSymbol{:}\AgdaSpace{}%
\AgdaDatatype{Chain}\AgdaSpace{}%
\AgdaGeneralizable{Γ}\AgdaSpace{}%
\AgdaSymbol{→}\AgdaSpace{}%
\AgdaDatatype{ℕ}\AgdaSpace{}%
\AgdaSymbol{→}\AgdaSpace{}%
\AgdaDatatype{List}\AgdaSpace{}%
\AgdaPostulate{String}\<%
\\
%
\>[2]\AgdaFunction{chain-sac-l}\AgdaSpace{}%
\AgdaSymbol{(}\AgdaInductiveConstructor{ε}\AgdaSpace{}%
\AgdaBound{x}\AgdaSymbol{)}\AgdaSpace{}%
\AgdaSymbol{\AgdaUnderscore{}}\AgdaSpace{}%
\AgdaSymbol{=}\AgdaSpace{}%
\AgdaInductiveConstructor{[]}\<%
\\
%
\>[2]\AgdaFunction{chain-sac-l}\AgdaSpace{}%
\AgdaSymbol{(}\AgdaBound{c}\AgdaSpace{}%
\AgdaOperator{\AgdaInductiveConstructor{▹}}\AgdaSpace{}%
\AgdaSymbol{(}\AgdaBound{v}\AgdaSpace{}%
\AgdaOperator{\AgdaInductiveConstructor{,}}\AgdaSpace{}%
\AgdaBound{e}\AgdaSymbol{))}\AgdaSpace{}%
\AgdaBound{n}\AgdaSpace{}%
\AgdaSymbol{=}%
\>[3008I]\AgdaKeyword{let}\AgdaSpace{}%
\AgdaBound{r}\AgdaSpace{}%
\AgdaOperator{\AgdaInductiveConstructor{,}}\AgdaSpace{}%
\AgdaBound{n′}\AgdaSpace{}%
\AgdaSymbol{=}\AgdaSpace{}%
\AgdaFunction{Sac.to-sac}\AgdaSpace{}%
\AgdaSymbol{(}\AgdaFunction{multiopt}\AgdaSpace{}%
\AgdaNumber{10}\AgdaSpace{}%
\AgdaBound{e}\AgdaSymbol{)}\AgdaSpace{}%
\AgdaSymbol{(}\AgdaFunction{chain-sac-ctx}\AgdaSpace{}%
\AgdaBound{c}\AgdaSymbol{)}\AgdaSpace{}%
\AgdaBound{n}\<%
\\
\>[.][@{}l@{}]\<[3008I]%
\>[32]\AgdaKeyword{in}\AgdaSpace{}%
\AgdaFunction{printf}\AgdaSpace{}%
\AgdaString{"\%s\ =\ \%s;"}\AgdaSpace{}%
\AgdaBound{v}\AgdaSpace{}%
\AgdaBound{r}\AgdaSpace{}%
\AgdaOperator{\AgdaInductiveConstructor{∷}}\AgdaSpace{}%
\AgdaFunction{chain-sac-l}\AgdaSpace{}%
\AgdaBound{c}\AgdaSpace{}%
\AgdaBound{n′}\<%
\\
%
\\[\AgdaEmptyExtraSkip]%
%
\>[2]\AgdaFunction{chain-sac}\AgdaSpace{}%
\AgdaSymbol{:}\AgdaSpace{}%
\AgdaDatatype{Chain}\AgdaSpace{}%
\AgdaGeneralizable{Γ}\AgdaSpace{}%
\AgdaSymbol{→}\AgdaSpace{}%
\AgdaPostulate{String}\<%
\\
%
\>[2]\AgdaFunction{chain-sac}\AgdaSpace{}%
\AgdaBound{c}\AgdaSpace{}%
\AgdaSymbol{=}\AgdaSpace{}%
\AgdaFunction{intersperse}\AgdaSpace{}%
\AgdaString{"\textbackslash{}n"}\AgdaSpace{}%
\AgdaOperator{\AgdaFunction{\$}}\AgdaSpace{}%
\AgdaFunction{L.reverse}\AgdaSpace{}%
\AgdaOperator{\AgdaFunction{\$}}\AgdaSpace{}%
\AgdaFunction{chain-sac-l}\AgdaSpace{}%
\AgdaBound{c}\AgdaSpace{}%
\AgdaNumber{1}\<%
\\
%
\\[\AgdaEmptyExtraSkip]%
%
\\[\AgdaEmptyExtraSkip]%
%
\>[2]\AgdaComment{--\ test-chain\ :\ Chain\ \AgdaUnderscore{}\ --(ε\ ▹\ ar\ (ι\ 3))}\<%
\\
%
\>[2]\AgdaComment{--\ test-chain\ =\ ε\ \{Γ\ =\ ε\ ▹\ ar\ (ι\ 3)\}\ (\AgdaUnderscore{}\ ,,\ "a")\ }\<%
\\
%
\>[2]\AgdaComment{--\ \ \ \ \ \ \ \ \ \ \ \ ▹\ ("r"\ ,\ mul-test)}\<%
\\
%
\>[2]\AgdaComment{--\ \ \ \ \ \ \ \ \ \ \ \ ▹\ ("r₁"\ ,\ (var\ v₀)\ ⊠\ (var\ v₂))}\<%
\\
%
\\[\AgdaEmptyExtraSkip]%
%
\>[2]\AgdaComment{--\ test-grad\ :\ String}\<%
\\
%
\>[2]\AgdaComment{--\ test-grad\ =\ chain-sac\ test-chain\ }\<%
\\
%
\>[2]\AgdaComment{--\ \ \ \ \ \ \ \ \ \ \ \ \ ++\ "\textbackslash{}n"\ ++\ chain-grad-sac\ test-chain\ (chain-grad\ test-chain\ one)}\<%
\end{code}

Let us consider a small example to see this in action.  We start with a little
convenience data structure \AF{ChainCtx} that keeps the shapes and the variable names
together.  We also define the function \AF{ce-split} that splits 
\AF{ChainCtx} into the context and the environment with variable names in that context:
\begin{code}%
%
\>[2]\AgdaKeyword{data}\AgdaSpace{}%
\AgdaDatatype{ChainCtx}\AgdaSpace{}%
\AgdaSymbol{:}\AgdaSpace{}%
\AgdaPrimitive{Set}\AgdaSpace{}%
\AgdaKeyword{where}\<%
\\
\>[2][@{}l@{\AgdaIndent{0}}]%
\>[4]\AgdaInductiveConstructor{ε}\AgdaSpace{}%
\AgdaSymbol{:}\AgdaSpace{}%
\AgdaDatatype{ChainCtx}\<%
\\
%
\>[4]\AgdaOperator{\AgdaInductiveConstructor{\AgdaUnderscore{}▹\AgdaUnderscore{}}}\AgdaSpace{}%
\AgdaSymbol{:}\AgdaSpace{}%
\AgdaDatatype{ChainCtx}\AgdaSpace{}%
\AgdaSymbol{→}\AgdaSpace{}%
\AgdaPostulate{String}\AgdaSpace{}%
\AgdaOperator{\AgdaFunction{×}}\AgdaSpace{}%
\AgdaDatatype{S}\AgdaSpace{}%
\AgdaSymbol{→}\AgdaSpace{}%
\AgdaDatatype{ChainCtx}\<%
\\
%
\\[\AgdaEmptyExtraSkip]%
%
\>[2]\AgdaFunction{ce-split}\AgdaSpace{}%
\AgdaSymbol{:}\AgdaSpace{}%
\AgdaDatatype{ChainCtx}\AgdaSpace{}%
\AgdaSymbol{→}\AgdaSpace{}%
\AgdaRecord{Σ}\AgdaSpace{}%
\AgdaDatatype{Ctx}\AgdaSpace{}%
\AgdaFunction{Sac.SEnv}\<%
\end{code}
\begin{code}[hide]%
%
\>[2]\AgdaFunction{ce-split}\AgdaSpace{}%
\AgdaInductiveConstructor{ε}\AgdaSpace{}%
\AgdaSymbol{=}\AgdaSpace{}%
\AgdaInductiveConstructor{ε}\AgdaSpace{}%
\AgdaOperator{\AgdaInductiveConstructor{,}}\AgdaSpace{}%
\AgdaInductiveConstructor{tt}\<%
\\
%
\>[2]\AgdaFunction{ce-split}\AgdaSpace{}%
\AgdaSymbol{(}\AgdaBound{cx}\AgdaSpace{}%
\AgdaOperator{\AgdaInductiveConstructor{▹}}\AgdaSpace{}%
\AgdaSymbol{(}\AgdaBound{v}\AgdaSpace{}%
\AgdaOperator{\AgdaInductiveConstructor{,}}\AgdaSpace{}%
\AgdaBound{s}\AgdaSymbol{))}\AgdaSpace{}%
\AgdaSymbol{=}\AgdaSpace{}%
\AgdaKeyword{let}\AgdaSpace{}%
\AgdaBound{Δ}\AgdaSpace{}%
\AgdaOperator{\AgdaInductiveConstructor{,}}\AgdaSpace{}%
\AgdaBound{ρ}\AgdaSpace{}%
\AgdaSymbol{=}\AgdaSpace{}%
\AgdaFunction{ce-split}\AgdaSpace{}%
\AgdaBound{cx}\AgdaSpace{}%
\AgdaKeyword{in}\AgdaSpace{}%
\AgdaSymbol{(}\AgdaBound{Δ}\AgdaSpace{}%
\AgdaOperator{\AgdaInductiveConstructor{▹}}\AgdaSpace{}%
\AgdaInductiveConstructor{ar}\AgdaSpace{}%
\AgdaBound{s}\AgdaSymbol{)}\AgdaSpace{}%
\AgdaOperator{\AgdaInductiveConstructor{,}}\AgdaSpace{}%
\AgdaSymbol{(}\AgdaBound{ρ}\AgdaSpace{}%
\AgdaOperator{\AgdaInductiveConstructor{,}}\AgdaSpace{}%
\AgdaBound{v}\AgdaSymbol{)}\<%
\\
%
\\[\AgdaEmptyExtraSkip]%
%
\>[2]\AgdaFunction{Product}\AgdaSpace{}%
\AgdaSymbol{:}\AgdaSpace{}%
\AgdaDatatype{ℕ}\AgdaSpace{}%
\AgdaSymbol{→}\AgdaSpace{}%
\AgdaPrimitive{Set}\AgdaSpace{}%
\AgdaSymbol{→}\AgdaSpace{}%
\AgdaPrimitive{Set}\<%
\\
%
\>[2]\AgdaFunction{Product}\AgdaSpace{}%
\AgdaNumber{0}%
\>[18]\AgdaBound{A}\AgdaSpace{}%
\AgdaSymbol{=}\AgdaSpace{}%
\AgdaRecord{⊤}\<%
\\
%
\>[2]\AgdaFunction{Product}\AgdaSpace{}%
\AgdaNumber{1}%
\>[18]\AgdaBound{A}\AgdaSpace{}%
\AgdaSymbol{=}\AgdaSpace{}%
\AgdaBound{A}\<%
\\
%
\>[2]\AgdaCatchallClause{\AgdaFunction{Product}}\AgdaSpace{}%
\AgdaCatchallClause{\AgdaSymbol{(}}\AgdaCatchallClause{\AgdaInductiveConstructor{suc}}\AgdaSpace{}%
\AgdaCatchallClause{\AgdaBound{n}}\AgdaCatchallClause{\AgdaSymbol{)}}\AgdaSpace{}%
\AgdaCatchallClause{\AgdaBound{A}}\AgdaSpace{}%
\AgdaSymbol{=}\AgdaSpace{}%
\AgdaBound{A}\AgdaSpace{}%
\AgdaOperator{\AgdaFunction{×}}\AgdaSpace{}%
\AgdaFunction{Product}\AgdaSpace{}%
\AgdaBound{n}\AgdaSpace{}%
\AgdaBound{A}\<%
\\
%
\\[\AgdaEmptyExtraSkip]%
%
\>[2]\AgdaFunction{Es}\AgdaSpace{}%
\AgdaSymbol{:}\AgdaSpace{}%
\AgdaSymbol{∀}\AgdaSpace{}%
\AgdaSymbol{\{}\AgdaBound{Γ}\AgdaSpace{}%
\AgdaSymbol{:}\AgdaSpace{}%
\AgdaDatatype{Ctx}\AgdaSymbol{\}}\AgdaSpace{}%
\AgdaSymbol{→}\AgdaSpace{}%
\AgdaSymbol{(}\AgdaBound{n}\AgdaSpace{}%
\AgdaSymbol{:}\AgdaSpace{}%
\AgdaDatatype{ℕ}\AgdaSymbol{)}\AgdaSpace{}%
\AgdaSymbol{→}\AgdaSpace{}%
\AgdaSymbol{\{}\AgdaFunction{Product}\AgdaSpace{}%
\AgdaBound{n}\AgdaSpace{}%
\AgdaDatatype{IS}\AgdaSymbol{\}}\AgdaSpace{}%
\AgdaSymbol{→}\AgdaSpace{}%
\AgdaPrimitive{Set}\<%
\\
%
\>[2]\AgdaFunction{Es}\AgdaSpace{}%
\AgdaSymbol{\{}\AgdaBound{Γ}\AgdaSymbol{\}}\AgdaSpace{}%
\AgdaNumber{0}%
\>[23]\AgdaSymbol{\{}\AgdaBound{is}\AgdaSymbol{\}}\AgdaSpace{}%
\AgdaSymbol{=}\AgdaSpace{}%
\AgdaRecord{⊤}\<%
\\
%
\>[2]\AgdaFunction{Es}\AgdaSpace{}%
\AgdaSymbol{\{}\AgdaBound{Γ}\AgdaSymbol{\}}\AgdaSpace{}%
\AgdaNumber{1}%
\>[23]\AgdaSymbol{\{}\AgdaBound{is}\AgdaSymbol{\}}\AgdaSpace{}%
\AgdaSymbol{=}\AgdaSpace{}%
\AgdaDatatype{E}\AgdaSpace{}%
\AgdaBound{Γ}\AgdaSpace{}%
\AgdaBound{is}\<%
\\
%
\>[2]\AgdaFunction{Es}\AgdaSpace{}%
\AgdaSymbol{\{}\AgdaBound{Γ}\AgdaSymbol{\}}\AgdaSpace{}%
\AgdaSymbol{(}\AgdaInductiveConstructor{suc}\AgdaSpace{}%
\AgdaSymbol{(}\AgdaInductiveConstructor{suc}\AgdaSpace{}%
\AgdaBound{n}\AgdaSymbol{))}\AgdaSpace{}%
\AgdaSymbol{\{}\AgdaBound{is}\AgdaSpace{}%
\AgdaOperator{\AgdaInductiveConstructor{,}}\AgdaSpace{}%
\AgdaBound{p}\AgdaSymbol{\}}%
\>[33]\AgdaSymbol{=}\AgdaSpace{}%
\AgdaDatatype{E}\AgdaSpace{}%
\AgdaBound{Γ}\AgdaSpace{}%
\AgdaBound{is}\AgdaSpace{}%
\AgdaOperator{\AgdaFunction{×}}\AgdaSpace{}%
\AgdaFunction{Es}\AgdaSpace{}%
\AgdaSymbol{\{}\AgdaBound{Γ}\AgdaSymbol{\}}\AgdaSpace{}%
\AgdaSymbol{(}\AgdaInductiveConstructor{suc}\AgdaSpace{}%
\AgdaBound{n}\AgdaSymbol{)}\AgdaSpace{}%
\AgdaSymbol{\{}\AgdaBound{p}\AgdaSymbol{\}}\<%
\\
%
\\[\AgdaEmptyExtraSkip]%
%
\>[2]\AgdaFunction{↑↑ₙ}\AgdaSpace{}%
\AgdaSymbol{:}\AgdaSpace{}%
\AgdaSymbol{∀}\AgdaSpace{}%
\AgdaSymbol{\{}\AgdaBound{Γ}\AgdaSpace{}%
\AgdaSymbol{:}\AgdaSpace{}%
\AgdaDatatype{Ctx}\AgdaSymbol{\}}\AgdaSpace{}%
\AgdaSymbol{\{}\AgdaBound{is}\AgdaSymbol{\}}\AgdaSpace{}%
\AgdaBound{n}\AgdaSpace{}%
\AgdaSymbol{\{}\AgdaBound{p}\AgdaSpace{}%
\AgdaSymbol{:}\AgdaSpace{}%
\AgdaFunction{Product}\AgdaSpace{}%
\AgdaBound{n}\AgdaSpace{}%
\AgdaDatatype{IS}\AgdaSymbol{\}}\AgdaSpace{}%
\AgdaSymbol{→}\AgdaSpace{}%
\AgdaFunction{Es}\AgdaSpace{}%
\AgdaSymbol{\{}\AgdaBound{Γ}\AgdaSymbol{\}}\AgdaSpace{}%
\AgdaBound{n}\AgdaSpace{}%
\AgdaSymbol{\{}\AgdaBound{p}\AgdaSymbol{\}}\AgdaSpace{}%
\AgdaSymbol{→}\AgdaSpace{}%
\AgdaFunction{Es}\AgdaSpace{}%
\AgdaSymbol{\{}\AgdaBound{Γ}\AgdaSpace{}%
\AgdaOperator{\AgdaInductiveConstructor{▹}}\AgdaSpace{}%
\AgdaBound{is}\AgdaSpace{}%
\AgdaOperator{\AgdaInductiveConstructor{▹}}\AgdaSpace{}%
\AgdaBound{is}\AgdaSymbol{\}}\AgdaSpace{}%
\AgdaBound{n}\AgdaSpace{}%
\AgdaSymbol{\{}\AgdaBound{p}\AgdaSymbol{\}}\<%
\\
%
\>[2]\AgdaFunction{↑↑ₙ}\AgdaSpace{}%
\AgdaNumber{0}\AgdaSpace{}%
\AgdaBound{es}\AgdaSpace{}%
\AgdaSymbol{=}\AgdaSpace{}%
\AgdaSymbol{\AgdaUnderscore{}}\<%
\\
%
\>[2]\AgdaFunction{↑↑ₙ}\AgdaSpace{}%
\AgdaNumber{1}\AgdaSpace{}%
\AgdaBound{e}%
\>[11]\AgdaSymbol{=}\AgdaSpace{}%
\AgdaOperator{\AgdaFunction{↑↑}}\AgdaSpace{}%
\AgdaBound{e}\<%
\\
%
\>[2]\AgdaFunction{↑↑ₙ}\AgdaSpace{}%
\AgdaSymbol{(}\AgdaInductiveConstructor{suc}\AgdaSpace{}%
\AgdaSymbol{(}\AgdaInductiveConstructor{suc}\AgdaSpace{}%
\AgdaBound{n}\AgdaSymbol{))}\AgdaSpace{}%
\AgdaSymbol{(}\AgdaBound{e}\AgdaSpace{}%
\AgdaOperator{\AgdaInductiveConstructor{,}}\AgdaSpace{}%
\AgdaBound{es}\AgdaSymbol{)}\AgdaSpace{}%
\AgdaSymbol{=}\AgdaSpace{}%
\AgdaOperator{\AgdaFunction{↑↑}}\AgdaSpace{}%
\AgdaBound{e}\AgdaSpace{}%
\AgdaOperator{\AgdaInductiveConstructor{,}}\AgdaSpace{}%
\AgdaFunction{↑↑ₙ}\AgdaSpace{}%
\AgdaSymbol{(}\AgdaInductiveConstructor{suc}\AgdaSpace{}%
\AgdaBound{n}\AgdaSymbol{)}\AgdaSpace{}%
\AgdaBound{es}\<%
\end{code}
Consider an initial environment of two 5-element vectors $a$ and $b$; local
computations $x = ab$ and $y = xx$; and the generated code when computing derivative
of $y$ (\AC{var v₀}) on the right.
\begin{mathpar}
\codeblock{\begin{code}%
%
\>[2]\AgdaFunction{test-chain}\AgdaSpace{}%
\AgdaSymbol{:}\AgdaSpace{}%
\AgdaDatatype{Chain}\AgdaSpace{}%
\AgdaSymbol{\AgdaUnderscore{}}\<%
\\
%
\>[2]\AgdaFunction{test-chain}\AgdaSpace{}%
\AgdaSymbol{=}\AgdaSpace{}%
\AgdaKeyword{let}\<%
\\
\>[2][@{}l@{\AgdaIndent{0}}]%
\>[4]\AgdaBound{Γ}\AgdaSpace{}%
\AgdaOperator{\AgdaInductiveConstructor{,}}\AgdaSpace{}%
\AgdaBound{ρ}\AgdaSpace{}%
\AgdaSymbol{=}\AgdaSpace{}%
\AgdaFunction{ce-split}\AgdaSpace{}%
\AgdaSymbol{(}\AgdaInductiveConstructor{ε}\AgdaSpace{}%
\AgdaOperator{\AgdaInductiveConstructor{▹}}\AgdaSpace{}%
\AgdaSymbol{(}\AgdaString{"a"}\AgdaSpace{}%
\AgdaOperator{\AgdaInductiveConstructor{,}}\AgdaSpace{}%
\AgdaInductiveConstructor{ι}\AgdaSpace{}%
\AgdaNumber{5}\AgdaSymbol{)}\AgdaSpace{}%
\AgdaOperator{\AgdaInductiveConstructor{▹}}\AgdaSpace{}%
\AgdaSymbol{(}\AgdaString{"b"}\AgdaSpace{}%
\AgdaOperator{\AgdaInductiveConstructor{,}}\AgdaSpace{}%
\AgdaInductiveConstructor{ι}\AgdaSpace{}%
\AgdaNumber{5}\AgdaSymbol{))}\<%
\\
%
\>[4]\AgdaBound{a}\AgdaSpace{}%
\AgdaSymbol{=}\AgdaSpace{}%
\AgdaInductiveConstructor{var}\AgdaSpace{}%
\AgdaInductiveConstructor{v₁}\AgdaSymbol{;}\AgdaSpace{}%
\AgdaBound{b}\AgdaSpace{}%
\AgdaSymbol{=}\AgdaSpace{}%
\AgdaInductiveConstructor{var}\AgdaSpace{}%
\AgdaInductiveConstructor{v₀}\<%
\\
%
\>[4]\AgdaBound{C₁}\AgdaSpace{}%
\AgdaSymbol{=}\AgdaSpace{}%
\AgdaInductiveConstructor{ε}\AgdaSpace{}%
\AgdaSymbol{\{}\AgdaBound{Γ}\AgdaSymbol{\}}\AgdaSpace{}%
\AgdaBound{ρ}%
\>[18]\AgdaOperator{\AgdaInductiveConstructor{▹}}\AgdaSpace{}%
\AgdaSymbol{(}\AgdaString{"x"}\AgdaSpace{}%
\AgdaOperator{\AgdaInductiveConstructor{,}}\AgdaSpace{}%
\AgdaBound{a}\AgdaSpace{}%
\AgdaOperator{\AgdaInductiveConstructor{⊠}}\AgdaSpace{}%
\AgdaBound{b}\AgdaSymbol{)}\<%
\\
%
\>[4]\AgdaBound{x}\AgdaSpace{}%
\AgdaSymbol{=}\AgdaSpace{}%
\AgdaInductiveConstructor{var}\AgdaSpace{}%
\AgdaInductiveConstructor{v₀}\<%
\\
%
\>[4]\AgdaBound{C₂}\AgdaSpace{}%
\AgdaSymbol{=}\AgdaSpace{}%
\AgdaBound{C₁}%
\>[18]\AgdaOperator{\AgdaInductiveConstructor{▹}}\AgdaSpace{}%
\AgdaSymbol{(}\AgdaString{"y"}\AgdaSpace{}%
\AgdaOperator{\AgdaInductiveConstructor{,}}\AgdaSpace{}%
\AgdaBound{x}\AgdaSpace{}%
\AgdaOperator{\AgdaInductiveConstructor{⊠}}\AgdaSpace{}%
\AgdaBound{x}\AgdaSymbol{)}\<%
\\
%
\>[4]\AgdaKeyword{in}\AgdaSpace{}%
\AgdaBound{C₂}\<%
\end{code}}
\and
{\begin{varwidth}{0.9\textwidth}
\begin{lstlisting}[linewidth=.44\textwidth]
x = (a) * (b);
y = (x) * (x);
ddy = one;
ddx = ((ddy) * (x)) + ((ddy) * (x));
ddb = (ddx) * (a);
dda = (ddx) * (b);
\end{lstlisting}
\end{varwidth}}
\end{mathpar}
Let us convince ourselves that the result is correct.  Our expression is $abab = a^2b^2$,
and its partial derivatives $\frac{\partial}{\partial a} = 2ab^2$,
$\frac{\partial}{\partial b} = 2ba^2$.  If we fold the assignments, we get:
\begin{eqnarray*}
   \text{dda} &= (x + x)b = (ab + ab)b = 2ab^2\\
   \text{ddb} &= (x + x)a = (ab + ab)a = 2ba^2
\end{eqnarray*}
Note that computations in $x$ and \texttt{ddx} are shared in further computations
which was the main goal of introducing this mechanism.

There are two inconveniences in the above implementation that we would like to
mention:
\begin{enumerate}
\item There is no restriction on using the placeholders for derivatives in the 
chain expressions, so in principle, one could write expression in terms of
variables and their derivatives.  However, this is not being handled and likely
to generate bogus terms.  If this is a useful feature, it requires more thinking
on how exactly it should work.  Otherwise it is easy to introduce restrictions
that rule out such cases.
\item If we define variables in the chain that do not contribute to the final
expression, we may introduce extra computations.  We do not compromise correctness,
as all inaccessible terms will get zero value.  However, direct execution of the
resulting expressions may introduce redundant computations.
\end{enumerate}
Both of these are future work.  For now, we make an assumption that placeholders
are not used in the expressions and that we do not insert bindings that do not
contribute to the final result.

\begin{code}[hide]%
%
\>[2]\AgdaFunction{test-chain-sac}\AgdaSpace{}%
\AgdaSymbol{:}\AgdaSpace{}%
\AgdaPostulate{String}\<%
\\
%
\>[2]\AgdaFunction{test-chain-sac}\<%
\\
\>[2][@{}l@{\AgdaIndent{0}}]%
\>[4]\AgdaSymbol{=}%
\>[3249I]\AgdaFunction{chain-sac}\AgdaSpace{}%
\AgdaFunction{test-chain}\<%
\\
\>[3249I][@{}l@{\AgdaIndent{0}}]%
\>[13]\AgdaOperator{\AgdaFunction{++}}\AgdaSpace{}%
\AgdaString{"\textbackslash{}n"}\AgdaSpace{}%
\AgdaOperator{\AgdaFunction{++}}\AgdaSpace{}%
\AgdaFunction{chain-grad-sac}\AgdaSpace{}%
\AgdaFunction{test-chain}\AgdaSpace{}%
\AgdaSymbol{(}\AgdaFunction{chain-grad}\AgdaSpace{}%
\AgdaFunction{test-chain}\AgdaSpace{}%
\AgdaSymbol{(}\AgdaInductiveConstructor{one}\AgdaSymbol{))}\<%
\\
\>[0]\<%
\end{code}

We present the specification of our case study in \AF{E} using \AF{Chain}.  We start
with the context \AF{cnn-ctx} that contains the \texttt{target} digit that
is depicted on the image, the input image \texttt{inp} and the weights of the network.
The definition of the chain is a one-to-one copy of the definition found in
Section~\ref{sec:cnn}.  The only real difference is that we have to take care of
maintaining bindings between Agda variables and the variables in \AF{E}.  Fortunately,
let expressions in Agda make it possible to shadow the binding, which comes very
useful in this case.

{\small
\begin{code}%
\>[0][@{}l@{\AgdaIndent{1}}]%
\>[2]\AgdaFunction{cnn-ctx}\AgdaSpace{}%
\AgdaSymbol{:}\AgdaSpace{}%
\AgdaDatatype{ChainCtx}\<%
\\
%
\>[2]\AgdaFunction{cnn-ctx}%
\>[11]\AgdaSymbol{=}\AgdaSpace{}%
\AgdaInductiveConstructor{ε}\<%
\\
%
\>[11]\AgdaOperator{\AgdaInductiveConstructor{▹}}\AgdaSpace{}%
\AgdaSymbol{(}\AgdaString{"target"}%
\>[24]\AgdaOperator{\AgdaInductiveConstructor{,}}\AgdaSpace{}%
\AgdaInductiveConstructor{ι}\AgdaSpace{}%
\AgdaNumber{10}\AgdaSpace{}%
\AgdaOperator{\AgdaInductiveConstructor{⊗}}\AgdaSpace{}%
\AgdaSymbol{(}\AgdaInductiveConstructor{ι}\AgdaSpace{}%
\AgdaNumber{1}\AgdaSpace{}%
\AgdaOperator{\AgdaInductiveConstructor{⊗}}\AgdaSpace{}%
\AgdaSymbol{(}\AgdaInductiveConstructor{ι}\AgdaSpace{}%
\AgdaNumber{1}\AgdaSpace{}%
\AgdaOperator{\AgdaInductiveConstructor{⊗}}\AgdaSpace{}%
\AgdaSymbol{(}\AgdaInductiveConstructor{ι}\AgdaSpace{}%
\AgdaNumber{1}\AgdaSpace{}%
\AgdaOperator{\AgdaInductiveConstructor{⊗}}\AgdaSpace{}%
\AgdaInductiveConstructor{ι}\AgdaSpace{}%
\AgdaNumber{1}\AgdaSymbol{))))}%
\>[66]\AgdaComment{--\ 7}\<%
\\
%
\>[11]\AgdaOperator{\AgdaInductiveConstructor{▹}}\AgdaSpace{}%
\AgdaSymbol{(}\AgdaString{"inp"}%
\>[24]\AgdaOperator{\AgdaInductiveConstructor{,}}\AgdaSpace{}%
\AgdaInductiveConstructor{ι}\AgdaSpace{}%
\AgdaNumber{28}\AgdaSpace{}%
\AgdaOperator{\AgdaInductiveConstructor{⊗}}\AgdaSpace{}%
\AgdaInductiveConstructor{ι}\AgdaSpace{}%
\AgdaNumber{28}\AgdaSymbol{)}%
\>[66]\AgdaComment{--\ 6}\<%
\\
%
\>[11]\AgdaOperator{\AgdaInductiveConstructor{▹}}\AgdaSpace{}%
\AgdaSymbol{(}\AgdaString{"k₁"}%
\>[24]\AgdaOperator{\AgdaInductiveConstructor{,}}\AgdaSpace{}%
\AgdaInductiveConstructor{ι}\AgdaSpace{}%
\AgdaNumber{6}\AgdaSpace{}%
\AgdaOperator{\AgdaInductiveConstructor{⊗}}\AgdaSpace{}%
\AgdaSymbol{(}\AgdaInductiveConstructor{ι}\AgdaSpace{}%
\AgdaNumber{5}\AgdaSpace{}%
\AgdaOperator{\AgdaInductiveConstructor{⊗}}\AgdaSpace{}%
\AgdaInductiveConstructor{ι}\AgdaSpace{}%
\AgdaNumber{5}\AgdaSymbol{))}%
\>[66]\AgdaComment{--\ 5}\<%
\\
%
\>[11]\AgdaOperator{\AgdaInductiveConstructor{▹}}\AgdaSpace{}%
\AgdaSymbol{(}\AgdaString{"b₁"}%
\>[24]\AgdaOperator{\AgdaInductiveConstructor{,}}\AgdaSpace{}%
\AgdaInductiveConstructor{ι}\AgdaSpace{}%
\AgdaNumber{6}\AgdaSymbol{)}%
\>[66]\AgdaComment{--\ 4}\<%
\\
%
\>[11]\AgdaOperator{\AgdaInductiveConstructor{▹}}\AgdaSpace{}%
\AgdaSymbol{(}\AgdaString{"k₂"}%
\>[24]\AgdaOperator{\AgdaInductiveConstructor{,}}\AgdaSpace{}%
\AgdaInductiveConstructor{ι}\AgdaSpace{}%
\AgdaNumber{12}\AgdaSpace{}%
\AgdaOperator{\AgdaInductiveConstructor{⊗}}\AgdaSpace{}%
\AgdaSymbol{(}\AgdaInductiveConstructor{ι}\AgdaSpace{}%
\AgdaNumber{6}\AgdaSpace{}%
\AgdaOperator{\AgdaInductiveConstructor{⊗}}\AgdaSpace{}%
\AgdaSymbol{(}\AgdaInductiveConstructor{ι}\AgdaSpace{}%
\AgdaNumber{5}\AgdaSpace{}%
\AgdaOperator{\AgdaInductiveConstructor{⊗}}\AgdaSpace{}%
\AgdaInductiveConstructor{ι}\AgdaSpace{}%
\AgdaNumber{5}\AgdaSymbol{)))}%
\>[66]\AgdaComment{--\ 3}\<%
\\
%
\>[11]\AgdaOperator{\AgdaInductiveConstructor{▹}}\AgdaSpace{}%
\AgdaSymbol{(}\AgdaString{"b₂"}%
\>[24]\AgdaOperator{\AgdaInductiveConstructor{,}}\AgdaSpace{}%
\AgdaInductiveConstructor{ι}\AgdaSpace{}%
\AgdaNumber{12}\AgdaSymbol{)}%
\>[66]\AgdaComment{--\ 2}\<%
\\
%
\>[11]\AgdaOperator{\AgdaInductiveConstructor{▹}}\AgdaSpace{}%
\AgdaSymbol{(}\AgdaString{"fc"}%
\>[24]\AgdaOperator{\AgdaInductiveConstructor{,}}\AgdaSpace{}%
\AgdaInductiveConstructor{ι}\AgdaSpace{}%
\AgdaNumber{10}\AgdaSpace{}%
\AgdaOperator{\AgdaInductiveConstructor{⊗}}\AgdaSpace{}%
\AgdaSymbol{(}\AgdaInductiveConstructor{ι}\AgdaSpace{}%
\AgdaNumber{12}\AgdaSpace{}%
\AgdaOperator{\AgdaInductiveConstructor{⊗}}\AgdaSpace{}%
\AgdaSymbol{(}\AgdaInductiveConstructor{ι}\AgdaSpace{}%
\AgdaNumber{1}\AgdaSpace{}%
\AgdaOperator{\AgdaInductiveConstructor{⊗}}\AgdaSpace{}%
\AgdaSymbol{(}\AgdaInductiveConstructor{ι}\AgdaSpace{}%
\AgdaNumber{4}\AgdaSpace{}%
\AgdaOperator{\AgdaInductiveConstructor{⊗}}\AgdaSpace{}%
\AgdaInductiveConstructor{ι}\AgdaSpace{}%
\AgdaNumber{4}\AgdaSymbol{))))}%
\>[66]\AgdaComment{--\ 1}\<%
\\
%
\>[11]\AgdaOperator{\AgdaInductiveConstructor{▹}}\AgdaSpace{}%
\AgdaSymbol{(}\AgdaString{"b"}%
\>[24]\AgdaOperator{\AgdaInductiveConstructor{,}}\AgdaSpace{}%
\AgdaInductiveConstructor{ι}\AgdaSpace{}%
\AgdaNumber{10}\AgdaSymbol{)}%
\>[66]\AgdaComment{--\ 0}\<%
\\
%
\\[\AgdaEmptyExtraSkip]%
%
\>[2]\AgdaFunction{cnn-chain}\AgdaSpace{}%
\AgdaSymbol{:}\AgdaSpace{}%
\AgdaDatatype{Chain}\AgdaSpace{}%
\AgdaSymbol{\AgdaUnderscore{}}\<%
\\
%
\>[2]\AgdaFunction{cnn-chain}\AgdaSpace{}%
\AgdaSymbol{=}\AgdaSpace{}%
\AgdaKeyword{let}\<%
\\
\>[2][@{}l@{\AgdaIndent{0}}]%
\>[6]\AgdaBound{Γ}\AgdaSpace{}%
\AgdaOperator{\AgdaInductiveConstructor{,}}\AgdaSpace{}%
\AgdaBound{ρ}\AgdaSpace{}%
\AgdaSymbol{=}\AgdaSpace{}%
\AgdaFunction{ce-split}\AgdaSpace{}%
\AgdaFunction{cnn-ctx}\<%
\\
%
\>[6]\AgdaBound{inp}\AgdaSpace{}%
\AgdaSymbol{=}\AgdaSpace{}%
\AgdaInductiveConstructor{var}\AgdaSpace{}%
\AgdaInductiveConstructor{v₆}\AgdaSymbol{;}\AgdaSpace{}%
\AgdaBound{k₁}\AgdaSpace{}%
\AgdaSymbol{=}\AgdaSpace{}%
\AgdaInductiveConstructor{var}\AgdaSpace{}%
\AgdaInductiveConstructor{v₅}\AgdaSymbol{;}\AgdaSpace{}%
\AgdaBound{b₁}\AgdaSpace{}%
\AgdaSymbol{=}\AgdaSpace{}%
\AgdaInductiveConstructor{var}\AgdaSpace{}%
\AgdaInductiveConstructor{v₄}\AgdaSymbol{;}\AgdaSpace{}%
\AgdaBound{k₂}\AgdaSpace{}%
\AgdaSymbol{=}\AgdaSpace{}%
\AgdaInductiveConstructor{var}\AgdaSpace{}%
\AgdaInductiveConstructor{v₃}\AgdaSymbol{;}\AgdaSpace{}%
\AgdaBound{b₂}\AgdaSpace{}%
\AgdaSymbol{=}\AgdaSpace{}%
\AgdaInductiveConstructor{var}\AgdaSpace{}%
\AgdaInductiveConstructor{v₂}\AgdaSymbol{;}\AgdaSpace{}%
\AgdaBound{fc}\AgdaSpace{}%
\AgdaSymbol{=}\AgdaSpace{}%
\AgdaInductiveConstructor{var}\AgdaSpace{}%
\AgdaInductiveConstructor{v₁}\AgdaSymbol{;}\AgdaSpace{}%
\AgdaBound{b}\AgdaSpace{}%
\AgdaSymbol{=}\AgdaSpace{}%
\AgdaInductiveConstructor{var}\AgdaSpace{}%
\AgdaInductiveConstructor{v₀}\<%
\\
%
\>[6]\AgdaBound{C₁}\AgdaSpace{}%
\AgdaSymbol{=}\AgdaSpace{}%
\AgdaInductiveConstructor{ε}\AgdaSpace{}%
\AgdaSymbol{\{}\AgdaBound{Γ}\AgdaSymbol{\}}\AgdaSpace{}%
\AgdaBound{ρ}\AgdaSpace{}%
\AgdaOperator{\AgdaInductiveConstructor{▹}}\AgdaSpace{}%
\AgdaSymbol{(}\AgdaString{"c₁₁"}\AgdaSpace{}%
\AgdaOperator{\AgdaInductiveConstructor{,}}\AgdaSpace{}%
\AgdaFunction{mconv}\AgdaSpace{}%
\AgdaSymbol{(}\AgdaInductiveConstructor{ι}\AgdaSpace{}%
\AgdaOperator{\AgdaInductiveConstructor{⊗}}\AgdaSpace{}%
\AgdaInductiveConstructor{ι}\AgdaSymbol{)}\AgdaSpace{}%
\AgdaBound{inp}\AgdaSpace{}%
\AgdaBound{k₁}\AgdaSpace{}%
\AgdaBound{b₁}\AgdaSpace{}%
\AgdaSymbol{(}\AgdaInductiveConstructor{ι}\AgdaSpace{}%
\AgdaOperator{\AgdaInductiveConstructor{⊗}}\AgdaSpace{}%
\AgdaInductiveConstructor{ι}\AgdaSymbol{));}%
\>[71]\AgdaBound{k₂}\AgdaSpace{}%
\AgdaSymbol{=}\AgdaSpace{}%
\AgdaOperator{\AgdaFunction{↑↑}}\AgdaSpace{}%
\AgdaBound{k₂}\AgdaSymbol{;}\AgdaSpace{}%
\AgdaBound{b₂}\AgdaSpace{}%
\AgdaSymbol{=}\AgdaSpace{}%
\AgdaOperator{\AgdaFunction{↑↑}}\AgdaSpace{}%
\AgdaBound{b₂}\AgdaSymbol{;}%
\>[96]\AgdaBound{fc}\AgdaSpace{}%
\AgdaSymbol{=}\AgdaSpace{}%
\AgdaOperator{\AgdaFunction{↑↑}}\AgdaSpace{}%
\AgdaBound{fc}\AgdaSymbol{;}\AgdaSpace{}%
\AgdaBound{b}\AgdaSpace{}%
\AgdaSymbol{=}\AgdaSpace{}%
\AgdaOperator{\AgdaFunction{↑↑}}\AgdaSpace{}%
\AgdaBound{b}\AgdaSymbol{;}\AgdaSpace{}%
\AgdaBound{c₁₁}\AgdaSpace{}%
\AgdaSymbol{=}\AgdaSpace{}%
\AgdaInductiveConstructor{var}\AgdaSpace{}%
\AgdaInductiveConstructor{v₀}\<%
\\
%
\>[6]\AgdaBound{C₂}\AgdaSpace{}%
\AgdaSymbol{=}\AgdaSpace{}%
\AgdaBound{C₁}\AgdaSpace{}%
\AgdaOperator{\AgdaInductiveConstructor{▹}}\AgdaSpace{}%
\AgdaSymbol{(}\AgdaString{"c₁"}%
\>[23]\AgdaOperator{\AgdaInductiveConstructor{,}}\AgdaSpace{}%
\AgdaInductiveConstructor{logistic}\AgdaSpace{}%
\AgdaBound{c₁₁}\AgdaSymbol{);}%
\>[71]\AgdaBound{k₂}\AgdaSpace{}%
\AgdaSymbol{=}\AgdaSpace{}%
\AgdaOperator{\AgdaFunction{↑↑}}\AgdaSpace{}%
\AgdaBound{k₂}\AgdaSymbol{;}\AgdaSpace{}%
\AgdaBound{b₂}\AgdaSpace{}%
\AgdaSymbol{=}\AgdaSpace{}%
\AgdaOperator{\AgdaFunction{↑↑}}\AgdaSpace{}%
\AgdaBound{b₂}\AgdaSymbol{;}%
\>[96]\AgdaBound{fc}\AgdaSpace{}%
\AgdaSymbol{=}\AgdaSpace{}%
\AgdaOperator{\AgdaFunction{↑↑}}\AgdaSpace{}%
\AgdaBound{fc}\AgdaSymbol{;}\AgdaSpace{}%
\AgdaBound{b}\AgdaSpace{}%
\AgdaSymbol{=}\AgdaSpace{}%
\AgdaOperator{\AgdaFunction{↑↑}}\AgdaSpace{}%
\AgdaBound{b}\AgdaSymbol{;}\AgdaSpace{}%
\AgdaBound{c₁}\AgdaSpace{}%
\AgdaSymbol{=}\AgdaSpace{}%
\AgdaInductiveConstructor{var}\AgdaSpace{}%
\AgdaInductiveConstructor{v₀}\<%
\\
%
\>[6]\AgdaBound{C₃}\AgdaSpace{}%
\AgdaSymbol{=}\AgdaSpace{}%
\AgdaBound{C₂}\AgdaSpace{}%
\AgdaOperator{\AgdaInductiveConstructor{▹}}\AgdaSpace{}%
\AgdaSymbol{(}\AgdaString{"s₁"}%
\>[23]\AgdaOperator{\AgdaInductiveConstructor{,}}\AgdaSpace{}%
\AgdaFunction{Imap}\AgdaSpace{}%
\AgdaSymbol{λ}\AgdaSpace{}%
\AgdaBound{i}\AgdaSpace{}%
\AgdaSymbol{→}\AgdaSpace{}%
\AgdaFunction{avgp₂}\AgdaSpace{}%
\AgdaNumber{12}\AgdaSpace{}%
\AgdaNumber{12}\AgdaSpace{}%
\AgdaSymbol{(}\AgdaInductiveConstructor{sel}\AgdaSpace{}%
\AgdaSymbol{(}\AgdaOperator{\AgdaFunction{↑}}\AgdaSpace{}%
\AgdaBound{c₁}\AgdaSymbol{)}\AgdaSpace{}%
\AgdaBound{i}\AgdaSymbol{));}%
\>[71]\AgdaBound{k₂}\AgdaSpace{}%
\AgdaSymbol{=}\AgdaSpace{}%
\AgdaOperator{\AgdaFunction{↑↑}}\AgdaSpace{}%
\AgdaBound{k₂}\AgdaSymbol{;}\AgdaSpace{}%
\AgdaBound{b₂}\AgdaSpace{}%
\AgdaSymbol{=}\AgdaSpace{}%
\AgdaOperator{\AgdaFunction{↑↑}}\AgdaSpace{}%
\AgdaBound{b₂}\AgdaSymbol{;}%
\>[96]\AgdaBound{fc}\AgdaSpace{}%
\AgdaSymbol{=}\AgdaSpace{}%
\AgdaOperator{\AgdaFunction{↑↑}}\AgdaSpace{}%
\AgdaBound{fc}\AgdaSymbol{;}\AgdaSpace{}%
\AgdaBound{b}\AgdaSpace{}%
\AgdaSymbol{=}\AgdaSpace{}%
\AgdaOperator{\AgdaFunction{↑↑}}\AgdaSpace{}%
\AgdaBound{b}\AgdaSymbol{;}\AgdaSpace{}%
\AgdaBound{s₁}\AgdaSpace{}%
\AgdaSymbol{=}\AgdaSpace{}%
\AgdaInductiveConstructor{var}\AgdaSpace{}%
\AgdaInductiveConstructor{v₀}\<%
\\
%
\>[6]\AgdaBound{C₄}\AgdaSpace{}%
\AgdaSymbol{=}\AgdaSpace{}%
\AgdaBound{C₃}\AgdaSpace{}%
\AgdaOperator{\AgdaInductiveConstructor{▹}}\AgdaSpace{}%
\AgdaSymbol{(}\AgdaString{"c₂₁"}\AgdaSpace{}%
\AgdaOperator{\AgdaInductiveConstructor{,}}\AgdaSpace{}%
\AgdaFunction{mconv}\AgdaSpace{}%
\AgdaSymbol{(}\AgdaInductiveConstructor{ι}\AgdaSpace{}%
\AgdaOperator{\AgdaInductiveConstructor{⊗}}\AgdaSpace{}%
\AgdaSymbol{(}\AgdaInductiveConstructor{ι}\AgdaSpace{}%
\AgdaOperator{\AgdaInductiveConstructor{⊗}}\AgdaSpace{}%
\AgdaInductiveConstructor{ι}\AgdaSymbol{))}\AgdaSpace{}%
\AgdaBound{s₁}\AgdaSpace{}%
\AgdaBound{k₂}\AgdaSpace{}%
\AgdaBound{b₂}\AgdaSpace{}%
\AgdaSymbol{(}\AgdaInductiveConstructor{ι}\AgdaSpace{}%
\AgdaOperator{\AgdaInductiveConstructor{⊗}}\AgdaSpace{}%
\AgdaSymbol{(}\AgdaInductiveConstructor{ι}\AgdaSpace{}%
\AgdaOperator{\AgdaInductiveConstructor{⊗}}\AgdaSpace{}%
\AgdaInductiveConstructor{ι}\AgdaSymbol{)));}%
\>[96]\AgdaBound{fc}\AgdaSpace{}%
\AgdaSymbol{=}\AgdaSpace{}%
\AgdaOperator{\AgdaFunction{↑↑}}\AgdaSpace{}%
\AgdaBound{fc}\AgdaSymbol{;}\AgdaSpace{}%
\AgdaBound{b}\AgdaSpace{}%
\AgdaSymbol{=}\AgdaSpace{}%
\AgdaOperator{\AgdaFunction{↑↑}}\AgdaSpace{}%
\AgdaBound{b}\AgdaSymbol{;}\AgdaSpace{}%
\AgdaBound{c₂₁}\AgdaSpace{}%
\AgdaSymbol{=}\AgdaSpace{}%
\AgdaInductiveConstructor{var}\AgdaSpace{}%
\AgdaInductiveConstructor{v₀}\<%
\\
%
\>[6]\AgdaBound{C₅}\AgdaSpace{}%
\AgdaSymbol{=}\AgdaSpace{}%
\AgdaBound{C₄}\AgdaSpace{}%
\AgdaOperator{\AgdaInductiveConstructor{▹}}\AgdaSpace{}%
\AgdaSymbol{(}\AgdaString{"c₂"}%
\>[23]\AgdaOperator{\AgdaInductiveConstructor{,}}\AgdaSpace{}%
\AgdaInductiveConstructor{logistic}\AgdaSpace{}%
\AgdaBound{c₂₁}\AgdaSymbol{);}%
\>[96]\AgdaBound{fc}\AgdaSpace{}%
\AgdaSymbol{=}\AgdaSpace{}%
\AgdaOperator{\AgdaFunction{↑↑}}\AgdaSpace{}%
\AgdaBound{fc}\AgdaSymbol{;}\AgdaSpace{}%
\AgdaBound{b}\AgdaSpace{}%
\AgdaSymbol{=}\AgdaSpace{}%
\AgdaOperator{\AgdaFunction{↑↑}}\AgdaSpace{}%
\AgdaBound{b}\AgdaSymbol{;}\AgdaSpace{}%
\AgdaBound{c₂}\AgdaSpace{}%
\AgdaSymbol{=}\AgdaSpace{}%
\AgdaInductiveConstructor{var}\AgdaSpace{}%
\AgdaInductiveConstructor{v₀}\<%
\\
%
\>[6]\AgdaBound{C₆}\AgdaSpace{}%
\AgdaSymbol{=}\AgdaSpace{}%
\AgdaBound{C₅}\AgdaSpace{}%
\AgdaOperator{\AgdaInductiveConstructor{▹}}\AgdaSpace{}%
\AgdaSymbol{(}\AgdaString{"s₂"}%
\>[23]\AgdaOperator{\AgdaInductiveConstructor{,}}\AgdaSpace{}%
\AgdaFunction{Imap}\AgdaSpace{}%
\AgdaSymbol{λ}\AgdaSpace{}%
\AgdaBound{i}\AgdaSpace{}%
\AgdaSymbol{→}\AgdaSpace{}%
\AgdaFunction{Imap}\AgdaSpace{}%
\AgdaSymbol{λ}\AgdaSpace{}%
\AgdaBound{j}\AgdaSpace{}%
\AgdaSymbol{→}\AgdaSpace{}%
\AgdaFunction{avgp₂}\AgdaSpace{}%
\AgdaNumber{4}\AgdaSpace{}%
\AgdaNumber{4}\AgdaSpace{}%
\AgdaSymbol{(}\AgdaInductiveConstructor{sel}\AgdaSpace{}%
\AgdaSymbol{(}\AgdaInductiveConstructor{sel}\AgdaSpace{}%
\AgdaSymbol{(}\AgdaOperator{\AgdaFunction{↑↑}}\AgdaSpace{}%
\AgdaBound{c₂}\AgdaSymbol{)}\AgdaSpace{}%
\AgdaSymbol{(}\AgdaOperator{\AgdaFunction{↑}}\AgdaSpace{}%
\AgdaBound{i}\AgdaSymbol{))}\AgdaSpace{}%
\AgdaBound{j}\AgdaSymbol{));}%
\>[96]\AgdaBound{fc}\AgdaSpace{}%
\AgdaSymbol{=}\AgdaSpace{}%
\AgdaOperator{\AgdaFunction{↑↑}}\AgdaSpace{}%
\AgdaBound{fc}\AgdaSymbol{;}\AgdaSpace{}%
\AgdaBound{b}\AgdaSpace{}%
\AgdaSymbol{=}\AgdaSpace{}%
\AgdaOperator{\AgdaFunction{↑↑}}\AgdaSpace{}%
\AgdaBound{b}\AgdaSymbol{;}\AgdaSpace{}%
\AgdaBound{s₂}\AgdaSpace{}%
\AgdaSymbol{=}\AgdaSpace{}%
\AgdaInductiveConstructor{var}\AgdaSpace{}%
\AgdaInductiveConstructor{v₀}\<%
\\
%
\>[6]\AgdaBound{C₇}\AgdaSpace{}%
\AgdaSymbol{=}\AgdaSpace{}%
\AgdaBound{C₆}\AgdaSpace{}%
\AgdaOperator{\AgdaInductiveConstructor{▹}}\AgdaSpace{}%
\AgdaSymbol{(}\AgdaString{"r₁"}%
\>[23]\AgdaOperator{\AgdaInductiveConstructor{,}}\AgdaSpace{}%
\AgdaFunction{mconv}\AgdaSpace{}%
\AgdaSymbol{(}\AgdaInductiveConstructor{ι}\AgdaSpace{}%
\AgdaOperator{\AgdaInductiveConstructor{⊗}}\AgdaSpace{}%
\AgdaSymbol{(}\AgdaInductiveConstructor{ι}\AgdaSpace{}%
\AgdaOperator{\AgdaInductiveConstructor{⊗}}\AgdaSpace{}%
\AgdaSymbol{(}\AgdaInductiveConstructor{ι}\AgdaSpace{}%
\AgdaOperator{\AgdaInductiveConstructor{⊗}}\AgdaSpace{}%
\AgdaInductiveConstructor{ι}\AgdaSymbol{)))}\AgdaSpace{}%
\AgdaBound{s₂}\AgdaSpace{}%
\AgdaBound{fc}\AgdaSpace{}%
\AgdaBound{b}\AgdaSpace{}%
\AgdaSymbol{(}\AgdaInductiveConstructor{ι}\AgdaSpace{}%
\AgdaOperator{\AgdaInductiveConstructor{⊗}}\AgdaSpace{}%
\AgdaSymbol{(}\AgdaInductiveConstructor{ι}\AgdaSpace{}%
\AgdaOperator{\AgdaInductiveConstructor{⊗}}\AgdaSpace{}%
\AgdaSymbol{(}\AgdaInductiveConstructor{ι}\AgdaSpace{}%
\AgdaOperator{\AgdaInductiveConstructor{⊗}}\AgdaSpace{}%
\AgdaInductiveConstructor{ι}\AgdaSymbol{))));}%
\>[96]\AgdaBound{r₁}\AgdaSpace{}%
\AgdaSymbol{=}\AgdaSpace{}%
\AgdaInductiveConstructor{var}\AgdaSpace{}%
\AgdaInductiveConstructor{v₀}\<%
\\
%
\>[6]\AgdaBound{C₈}\AgdaSpace{}%
\AgdaSymbol{=}\AgdaSpace{}%
\AgdaBound{C₇}\AgdaSpace{}%
\AgdaOperator{\AgdaInductiveConstructor{▹}}\AgdaSpace{}%
\AgdaSymbol{(}\AgdaString{"r"}%
\>[23]\AgdaOperator{\AgdaInductiveConstructor{,}}\AgdaSpace{}%
\AgdaInductiveConstructor{logistic}\AgdaSpace{}%
\AgdaBound{r₁}\AgdaSymbol{)}\<%
\\
%
\>[6]\AgdaKeyword{in}\AgdaSpace{}%
\AgdaBound{C₈}\<%
\end{code}

\begin{code}[hide]%
%
\>[2]\AgdaFunction{test-cnn}\AgdaSpace{}%
\AgdaSymbol{:}\AgdaSpace{}%
\AgdaPostulate{String}\<%
\\
%
\>[2]\AgdaFunction{test-cnn}\<%
\\
\>[2][@{}l@{\AgdaIndent{0}}]%
\>[4]\AgdaSymbol{=}%
\>[3570I]\AgdaKeyword{let}\<%
\\
\>[3570I][@{}l@{\AgdaIndent{0}}]%
\>[8]\AgdaComment{--\ 2*8\ +\ 7\ =\ 23}\<%
\\
%
\>[8]\AgdaBound{target}\AgdaSpace{}%
\AgdaSymbol{=}\AgdaSpace{}%
\AgdaOperator{\AgdaFunction{↑↑}}\AgdaSpace{}%
\AgdaOperator{\AgdaFunction{↑↑}}\AgdaSpace{}%
\AgdaOperator{\AgdaFunction{↑↑}}\AgdaSpace{}%
\AgdaOperator{\AgdaFunction{↑↑}}\AgdaSpace{}%
\AgdaOperator{\AgdaFunction{↑↑}}%
\>[33]\AgdaOperator{\AgdaFunction{↑↑}}\AgdaSpace{}%
\AgdaOperator{\AgdaFunction{↑↑}}\AgdaSpace{}%
\AgdaOperator{\AgdaFunction{↑↑}}\AgdaSpace{}%
\AgdaOperator{\AgdaFunction{↑↑}}\AgdaSpace{}%
\AgdaOperator{\AgdaFunction{↑↑}}%
\>[49]\AgdaOperator{\AgdaFunction{↑↑}}\AgdaSpace{}%
\AgdaOperator{\AgdaFunction{↑}}\AgdaSpace{}%
\AgdaSymbol{(}\AgdaInductiveConstructor{var}\AgdaSpace{}%
\AgdaInductiveConstructor{v₀}\AgdaSymbol{)}\<%
\\
\>[.][@{}l@{}]\<[3570I]%
\>[6]\AgdaKeyword{in}%
\>[3584I]\AgdaFunction{chain-sac}\AgdaSpace{}%
\AgdaFunction{cnn-chain}\<%
\\
\>[3584I][@{}l@{\AgdaIndent{0}}]%
\>[13]\AgdaOperator{\AgdaFunction{++}}\AgdaSpace{}%
\AgdaString{"\textbackslash{}n"}\AgdaSpace{}%
\AgdaOperator{\AgdaFunction{++}}\AgdaSpace{}%
\AgdaFunction{chain-grad-sac}\AgdaSpace{}%
\AgdaFunction{cnn-chain}\AgdaSpace{}%
\AgdaSymbol{(}\AgdaFunction{chain-grad}\AgdaSpace{}%
\AgdaFunction{cnn-chain}\AgdaSpace{}%
\AgdaSymbol{(}\AgdaInductiveConstructor{var}\AgdaSpace{}%
\AgdaInductiveConstructor{v₀}\AgdaSpace{}%
\AgdaOperator{\AgdaInductiveConstructor{⊞}}\AgdaSpace{}%
\AgdaInductiveConstructor{minus}\AgdaSpace{}%
\AgdaBound{target}\AgdaSymbol{))}\<%
\end{code}
}


\section{Performance\label{sec:performance}}

One of the goals of this work is to demonstrate that it is possible to
formulate the problem in a proof assistant and then pass it on to the
other system that can run the algorithm efficiently. In order to
substantiate this claim, we compare the running times of the code that
we generate from the specification of the CNN at the end of the
Section~\ref{sec:edsl} with an equivalent CNN implemented with
TensorFlow~\cite{ad-tf} and PyTorch~\cite{ad-pytorch}. This is a
limited study, and it does not exploit all the expressiveness provided
by our language, but nevertheless it shows the potential of achieving
performance that approaches that of established programs in
non-verified languages. We use the commonly used MNIST database, which
consists of 60000 greyscale images of handwritten digits, each 28 by
28 pixels~\cite{deng2012mnist}.

Our verified CNN code corresponds to a single training (forward and
backward) pass. To build a full CNN, we hand-implement the
``batching loop'' that iterates across the training set, in which we
invoke the Futhark code extracted from the specification. Since the
specification is known to be free of indexing errors, we instruct the
Futhark compiler to elide bounds checking.


Comparing performance of Futhark with Tensorflow and PyTorch is not
straight-forward.  Futhark is a classical ahead of time compiler,
whereas Tensorflow and PyTorch are large frameworks that depend on
many external libraries, therefore their performance is very
sensitive to system configuration.  Moreover, by default, Tensorflow
and PyTorch launch just-in-time compilers as a part of their runtime
which for our CNN takes about 5 seconds.
It is not clear whether there is an option to cache the result of
JIT compilation between the runs.  Here we pretend that this is the
case, and we exclude any identifiable startup costs and JIT time
from the measurements.

We run our experiments on an NVIDIA A100 with
CUDA 11.8 and cuDNN 8.6.0, and we summarise results for Futhark and
Tensorflow in Table~\ref{tab:performance}. 
While Futhark is competitive for the small workload, TensorFlow is
faster for the large ones. This is not surprising, as machine learning
frameworks are optimised for executing CNNs.  Individual
layers in the TensorFlow are ultimately implemented
using hand-tuned primitives from cuDNN, so it is very challenging
to outperform them from the general purpose Futhark compiler.
Futhark's advantage on the small workload is likely due to lower
fixed overheads in the generated code.

Our PyTorch version results in extremely high runtimes of 19 and 61
seconds, and we observe low GPU utilization. While this is likely to
be a problem with a system configuration, we report these numbers here
as an indication that performance of machine learning systems is
fragile.

Profiling our generated code, we find that 83\% of all GPU work occurs
in two kernels corresponding to the convolutional layers. These are also the layers
where cuDNN's implementation is significantly better than the one
that is generated by Futhark from a high-level specification.
For the rest 17\% of runtime, we observe that Futhark has
generated a rather large number of GPU kernels that perform relatively
little work.  Identifying a better strategy on reducing the number
of small kernels requires further research.  It might be beneficial to
inline some array computations to get rid of intermediate arrays;
or there might be a better way to split the computation into kernels.

Closing the observed performance gap is an interesting task.
While it is unlikely that Futhark can match cuDNN's performance
from the existing specification, our framework makes it possible
to adjust the generate code to help the compiler.  As long as cuDNN
does not use custom hardware for the entire kernel, an approach
based on high-level array blocking~\cite{rp-mm} might be applicable
in the context of GPUs.  We could clearly introduce specialised
primitives in the DSL, but it would result in a more
restricted language, which is less interesting from a research
perspective.

\begin{table}
\begin{tabular}{crrr}
\textbf{Size of training set} & \multicolumn{2}{c}{\textbf{Runtime}} & \textbf{Ratio} \\
& Futhark & TensorFlow & \\
$10000$ & $0.91s$ & $1.07s$ &  $0.85\times{}$ \\
$60000$ & $4.93s$ & $2.92s$ &  $1.68\times{}$
\end{tabular}
\caption{Training time for various training set sizes, comparing
  Futhark, Tensorflow, and PyTorch, running on an A100 GPU. The final column
  shows the speedup of TensorFlow compared to Futhark. We use a batch
  size of $1000$, a training rate of $0.05$, and train for $10$
  epochs.}
\label{tab:performance}
\end{table}

% Notes for Troels:
%
% - Leave Agda-side optimisations for Artem
% - Only backend: Futhark
% - Compared with:
%   - TensorFlow (on GPU)
%   - Hand-written Futhark (lower priority)
% - Will try for multicore numbers as well, if does not complicate story

\section{Related Work\label{sec:relatedwork}}

% - "Verified Tensor-Program Optimization Via High-level Scheduling Rewrites"
% - "Efficient Differentiable Programming in a Functional Array-Processing Language"
% - "Verified tensor-program optimization via high-level scheduling rewrites"
% - "Indexed Streams: A Formal Intermediate Representation for Fused Contraction Programs"
% - "You only linearize once"
% - Dependent ML
% - ATS
% - Jax
% - Remora

\section{Conclusions\label{sec:conclusions}}

The paper demonstrates a technique of developing high performance
applications with strong correctness guarantees, including the absence
of out-of-bound indexing, certain functions being inverses,
well-scopedness and well-typedness of the embedded DSL and
semantics-preserving optimisations.

The key insight lies in using a proof assistant in cooperation with a
high-performance language of choice. This gives a clear separation of
concerns that is very difficult to achieve within a single language.
The proof assistant is used to design a specification, prove all the
correctness invariants of interest and performs an extraction into a
high-performance language, in our case Futhark, although only a
relatively small part of our work is specific to that language.

Having a trusted specification as well as entire code-generation
pipeline within a single dependently-typed framework is incredibly
powerful. As we have demonstrated at the example of the neural
network, we can introduce domain-specific optimisations and
transformations, such as automatic differentiation. For our example,
the entire framework that includes array theory, DSL, optimisations
and extraction is about 2000 lines of Agda code.

A lot of pieces that we have developed in this paper can be reused in
other numerical applications. However, there are many more
opportunities that we did not yet explore. We can formalise other
correctness criteria for our specification spanning from verified
extraction to functional correctness of the actual application such
as CNN robustness.
Choosing DSL primitives so that specificational clarity and
performance requirements are at balance is an interesting question.
Guaranteeing asymptotic efficiency of the AD implementation is another
interesting research direction.  Currently we rely on
optimisations which is known to be a fragile technique, despite its efficiency 
in many cases.  Identifying a normal form for the optimisations would
be a more fundamental approach.  Richer languages with more
control flow may require us choosing more powerful
encoding for the adjoint structures.


There are indeed plenty of opportunities, but the key point is this.
Correctness and performance are competing requirements when it comes
to application design. Therefore, such a cooperation between
correctness-oriented and performance-oriented tools is likely to
persist. With this work we demonstrate that such cooperation can be
done using fairly straightforward means, by Agda standards, and obtain
compelling practical performance.



\bibliographystyle{ACM-Reference-Format}
\bibliography{paper}

\end{document}
