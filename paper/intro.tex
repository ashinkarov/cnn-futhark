\section{Introduction\label{sec:intro}}

The year is 2025, and we still have to make a formidable choice between
correctness and performance for all the programming projects that we start.
Low level programming languages such as C or Fortran make it possible to leverage
intricate hardware features for the price of poor analysability and
correctness guarantees.  Dependently-typed systems such as Lean or Agda make it possible
to describe arbitrary invariants within the given program, yet they can
rarely generate high-performance code.

The idealist inside of us is exasperated, because there should be a
perfect solution that caters for both cases. However, the practitioner
within us proposes a different perspective. Instead of using a single
language, we may use two languages in cooperation. Specifically, we
envision an alliance between a proof assistant and some
high-performance language of our choice, where we extract code from
the latter to the former.

To explore this idea, we investigate the concrete problem of automatic differentiation (AD) which
is often found in machine learning applications.  This is a convenient case study
as it comes with the following challenges.  From the correctness perspective,
it is crucially important to track the shapes and ranks of the tensors,
guaranteeing the absence of out-of-bound indexing.
This is a very common source of errors that can be difficult
to find.   Secondly, we have to compute derivatives of the given tensor expressions,
preserve safe indexing guarantees while we do so, and we have to be able to translate
the computed expressions into some high-performance language.
As machine learning applications
are known to be numerically intensive problems, our performance challenge lies
in running the program as fast as we can on the chosen hardware architecture.

We follow~\cite{cnn-array} which demonstrates that it is possible to implement
one of the canonical convolutional neural network (CNN) in the array
language SaC~\cite{sac1, sac2}, obtaining good sequential and parallel performance
that is competitive with TensorFlow~\cite{ad-tf} and PyTorch~\cite{ad-pytorch}.
Focusing on correctness, we propose
a theory of rank-polymorphic arrays~\cite{rank-poly} in Agda~\cite{agda-2-6-3}.
Within this framework, we encode the CNN from~\cite{cnn-array} and lift it 
into an embedded DSL.  We implement AD
(reverse mode) and domain-specific optimisations for expressions in that DSL.
Finally, we implement an extraction 
into Futhark (a functional array language), apply the CNN to the MNIST digit recognition problem,
and compare performance with TensorFlow.

As a result, we demonstrate an approach where the entire specification,
optimiser, AD and code generation are available to us within a proof
assistant of our choice.  We can prove facts about all the
stages of the pipeline and easily adjust them to our liking.
We argue that such a liberating
approach is feasible in practice, at least for the times of dialectic
of correctness and performance.


The contributions of this paper are as follows:
\begin{enumerate}
  \item a rank-polymorphic array theory and implementation of
        the CNN from~\cite{cnn-array} in Agda;
  \item an embedded DSL in Agda which supports AD (reverse mode);
  \item an extraction mechanism for generating Futhark code from the DSL; and
  \item an experimental evaluation of the generated codes.
\end{enumerate}

This paper is written in literate Agda, which guarantees that all the code
snippets have been type-checked.


