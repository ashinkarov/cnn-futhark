\section{Performance\label{sec:performance}}

One of the goals of this work is to demonstrate that it is possible to
formulate the problem in a proof assistant and then pass it on to the
other system that can run the algorithm efficiently. In order to
substantiate this claim, we compare the running times of the code that
we generate from the specification of the CNN at the end of the
Section~\ref{sec:edsl} with an equivalent CNN implemented with
TensorFlow~\cite{ad-tf}. This is a limited study, and it does not
exploit all the expressivity provided by our language, but
nevertheless it shows the potential of achieving performance that
approaches that of established (non-verified) languages. We use the
commonly used MNIST database, which consists of 60000 greyscale images
of handwritten digits, each 28 by 28 pixels~\cite{deng2012mnist}.

Our verified CNN code corresponds only to a single forward and
backward pass. To build a full CNN, we implement by hand the
``batching loop'' that iterates across the training set, in which we
invoke the Futhark code extracted from the specification. Since the
specification is known to be free of indexing errors, we instruct the
Futhark compiler to elide bounds checking.

Our experimental results have been obtained on an NVIDIA A100 with
CUDA 11.8 and cuDNN 8.6.0, and are shown in
Table~\ref{tab:performance}. While Futhark is competitive for the
small workload, TensorFlow is faster for the large ones. This is not
surprising, as the individual layers in the TensorFlow implementation
are ultimately implemented using hand-tuned primitives from cuDNN,
which the Futhark compiler, despite its optimisations, cannot match.
Futhark's minor advantage on the small workload is likely due to lower
fixed overheads in the generated code.

Profiling our generated code, we find that the Futhark compiler has
generated a rather large number of GPU kernels for the training loop,
most of which perform very little work, and contribute little to the
runtime. This is due to the normalisation issue discussed in
Section~\ref{sec:opt}, and shows that perhaps too many small arrays
are still manifested, that ought to be inlined. This is essentially
the common dilemma of how to balance recomputation with reuse.
However, the impact of these small GPU kernels on training performance
is minor--approximately 83\% of all GPU work occurs in two kernels
corresponding to the convolutional layers. These are also the layers
where cuDNN's implementation is significantly better than what the
Futhark compiler can produce.

\begin{table}
\begin{tabular}{crrr}
\textbf{Size of training set} & \multicolumn{2}{c}{\textbf{Runtime}} & \textbf{Ratio} \\
& Futhark & TensorFlow & \\
$10000$ & $0.91s$ & $1.07s$ & $0.85\times{}$ \\
$60000$ & $4.93s$ & $2.92s$ & $1.68\times{}$
\end{tabular}
\caption{Training time for various training set sizes, comparing
  Futhark and Tensorflow, running on an A100 GPU. The third column
  shows the speedup of TensorFlow compared to Futhark. We use a batch
  size of $1000$, a training rate of $0.05$, and train for $10$
  epochs.}
\label{tab:performance}
\end{table}

% Notes for Troels:
%
% - Leave Agda-side optimisations for Artem
% - Only backend: Futhark
% - Compared with:
%   - TensorFlow (on GPU)
%   - Hand-written Futhark (lower priority)
% - Will try for multicore numbers as well, if does not complicate story


% One of the goals of this work is to demonstrate that it is possible to formulate
% the problem in a proof assistant and then pass it on to the other system that can
% run the algorithm efficiently.  In order to substantiate this claim, we compare
% the running times of the code that we generate from the specification at the end of the
% Section~\ref{sec:ad} and the hand-written SaC code from~\cite{cnn-array}.
% We are not interested in an exhaustive performance study similar to what is provided
% in \cite{cnn-array}. Instead, we take the version from that paper as reference point 
% and we aim to find out whether we are in the same ballpark.

% We take the code from~\cite{cnn-array}, make sure that it runs, and we replace
% the hand-written CNN training with the Agda-generated one. 
% Our first observation is that both versions 
% produce the same results, and none of the shape constraints
% that we defined in Section~\ref{sec:sac-primitives} fired.  This means that our
% code generation is working.  Unfortunately, the runtime comparison revealed that
% our version is about 10$\times$ slower than the hand-written SaC version.

% We got in touch with the SaC team who provided a lot of support in identifying
% the causes of the disappointing difference in performance.  It turns out that the main culprit has
% to do with the inability to optimise away selections into tensor comprehensions in a few situations.
% With-Loop-Folding~\cite{wlf}, SaC's mechanism for fusing tensor comprehensions fails to fold
% tensor comprehensions that are nested and cannot be flattened statically.
% In simple terms, the expression \texttt{\{iv -> e(iv)\}[jv]} in some situation does not reduce to
% \texttt{e(jv)} which, in our generated code, is essential to match the hand-written performance.
% As a result, several arrays were created
% simply to make a single selection into them.  The original code never ran into this
% problem as the hand-written code avoided such patterns.  Our \AF{E} optimiser
% from Section~\ref{sec:opt} could not help either, because the problem was occurring after
% the SaC primitives such as slide and block were inlined.

% After numerous attempts on altering \AF{E} to fit SaC requirements and the SaC
% team trying to implement some of the missing optimisations, on the February 23rd
% (6 days before the ICFP deadline) we realised that the best runtime we can
% obtain is still 6$\times$ slower than the hand-written code.  The compiler is too
% sensitive to the flavour of the code that we pass to it, and when certain patterns
% are not recognised, there is very little one can do other than trying to fit
% those patterns.  However, this is not always possible with the generated code.
% Performance \emph{is} frustrating!

% \subsection{Generating C}

% After overcoming the natural instinct to give up, we realised that the real
% power of the proposed approach lies in the ability to modify any part of the
% code generation pipeline.  This includes swapping the back-end language of choice
% to something else.  Therefore, we decided to try generating C code instead of
% SaC code.

% While C is a canonical low-level language, it has excellent support for
% multi-dimensional arrays, given that the ranks are statically known.
% At runtime these arrays are flattened vectors, they do not have to live
% on the stack, and the language takes care of multi-dimensional indexing.

% However, the key difference between the C and SaC is memory management.
% SaC is a functional language that uses reference counting to automate
% operations on allocating and freeing memory.  In C these decisions are
% manual, and as we have seen before, excessive memory allocation is detrimental
% for the runtime.  For our use case we avoid memory management problem
% entirely, by assuming that all the variables in the \AF{Chain} have
% to be preallocated, and if we need any temporary array variables when
% extracting array values, we fail extraction.  This way we guarantee
% that no memory allocation is ever needed.

% Meeting such a requirement means that we need to optimise away operations
% like \AC{slide}/\AC{backslide} as they require conceptual array allocation.
% The same goes for \AC{imap}s appearing in some of the argument positions.
% Putting these considerations together, we extended \AF{E} with the following
% explicit operations on indices:
% \begin{code}[hide]
% open import Data.Nat as ℕ using (ℕ; zero; suc)
% open import Data.Unit
% open import Data.Empty
% open import Data.Product as Prod using (Σ; ∃; _,_; _×_; proj₁; proj₂)
% open import Relation.Nullary
% open import Relation.Binary.PropositionalEquality hiding ([_])
% open import Data.List as L using (List; []; _∷_)
% open import Function
% open import arrays
% open Array hiding (sum; slide; backslide)

% data IS : Set where
%   ix : S → IS
%   ar : S → IS

% infixl 15 _▹_
% data Ctx : Set where
%   ε : Ctx
%   _▹_ : Ctx → IS → Ctx

% variable
%   Γ Δ Ξ Ψ : Ctx
%   is ip iq ir : IS

% data _∈_ : IS → Ctx → Set where
%   here : is ∈ (Γ ▹ is)
%   there : is ∈ Γ → is ∈ (Γ ▹ ip)

% pattern v₀ = here
% pattern v₁ = there v₀
% pattern v₂ = there v₁
% pattern v₃ = there v₂
% pattern v₄ = there v₃
% pattern v₅ = there v₄
% pattern v₆ = there v₅
% pattern v₇ = there v₆
% pattern v₈ = there v₇
% pattern v₉ = there v₈

% unit : S
% unit = ι 1

% data Bop : Set where
%   plus mul : Bop
% \end{code}
% \begin{code}
% data E : Ctx → IS → Set where
%   div        : s * p ≈ q → (i : E Γ (ix q)) → E Γ (ix s)
%   mod        : s * p ≈ q → (i : E Γ (ix q)) → E Γ (ix p)
%   ix-plus    : (i : E Γ (ix s)) → (j : E Γ (ix u)) → suc p ≈ u → s + p ≈ r → E Γ (ix r)
%   ix-minus   : (i : E Γ (ix r)) → (j : E Γ (ix s)) → s + p ≈ r → suc p ≈ u 
%              → (e : E (Γ ▹ ix u) (ar q)) → E Γ (ar q)
%   ix-minusᵣ  : (i : E Γ (ix r)) → (j : E Γ (ix u)) → s + p ≈ r → suc p ≈ u 
%              → (e : E (Γ ▹ ix s) (ar q)) → E Γ (ar q)
%   -- ...
% \end{code}
% \begin{code}[hide]
%   zero one : E Γ (ar s)
%   var : is ∈ Γ → E Γ is

%   imapₛ : E (Γ ▹ ix s) (ar unit) → E Γ (ar s)
%   selₛ : E Γ (ar s) → E Γ (ix s) → E Γ (ar unit)

%   imap : E (Γ ▹ ix s) (ar p) → E Γ (ar (s ⊗ p))
%   sel : E Γ (ar (s ⊗ p)) → E Γ (ix s) → E Γ (ar p)

%   -- Blocked operations for avgpool 
%   imapb : s * p ≈ q → E (Γ ▹ ix s) (ar p) → E Γ (ar q)
%   selb : s * p ≈ q → E Γ (ar q) → E Γ (ix s) → E Γ (ar p)

%   -- zero-but i j e = i == j ? e : 0
%   zero-but : E Γ (ix s) → E Γ (ix s) → E Γ (ar p) → E Γ (ar p)
%   sum : E (Γ ▹ ix s) (ar p) → E Γ (ar p)
%   bin : Bop → E Γ (ar s) → E Γ (ar s) → E Γ (ar s)

%   slide : E Γ (ix s) → s + p ≈ r → E Γ (ar r)
%         → suc p ≈ u → E Γ (ar u)
%   backslide : E Γ (ix s) → E Γ (ar u) → suc p ≈ u
%             → s + p ≈ r → E Γ (ar r)
  
%   scaledown : ℕ → E Γ (ar s) → E Γ (ar s)
%   minus : E Γ (ar s) → E Γ (ar s)

%   logistic : E Γ (ar s) → E Γ (ar s)
% \end{code}
% The \AC{div} and \AC{mod} constructors perform point-wise division or modulo
% operation on the index $i$ and the shape $p$.  This is needed to express selections
% into blocked arrays as we have seen in Section~\ref{sec:sac-primitives}.
% The \AC{ix-plus} is a point-wise addition of $i$ and $j$.  The \AC{ix-minus} and
% \AC{ix-minusᵣ} correspond to left and right subtraction from the Section~\ref{sec:cnn}.
% The meaning of these constructors is follows: if $j$ can be subtracted from $i$
% (in the sense of existence of inverse to \AF{⊕ₚ} exists) then we evaluate $e$ at that index,
% otherwise we return zero.

% \subsubsection{Optimisations}
% We add the following optimisations to facilitate removal of temporary arrays in
% the generated code.  We show the only ones that we added, all the optimisations
% we defined before are still valid.
% \begin{code}[hide]
% _/_ : (Γ : Ctx) → is ∈ Γ → Ctx
% (Γ ▹ x) / here = Γ
% (Γ ▹ x) / there v = (Γ / v) ▹ x

% -- See the actual definition in the ./code directory in the
% -- root of the repo, here we just make a stub to explain the
% -- code below.
% postulate
%   wkv : (v : is ∈ Γ) → ip ∈ (Γ / v) → ip ∈ Γ
%   wk : (v : is ∈ Γ) → E (Γ / v) ip → E Γ ip

% -- Nicer syntax for common case:
% infixr 18 ↑_
% ↑_ : E Γ is → E (Γ ▹ ip) is
% ↑_ = wk here

% infixr 18 ↑↑_
% ↑↑_ : E Γ is → E (Γ ▹ ip ▹ iq) is
% ↑↑_ = ↑_ ∘ ↑_

% data Eq : is ∈ Γ → ip ∈ Γ → Set where
%   eq : {x : is ∈ Γ} → Eq x x
%   neq : (x : is ∈ Γ) → (y : ip ∈ (Γ / x)) → Eq x (wkv x y)

% postulate
%   eq? : (x : is ∈ Γ) → (y : ip ∈ Γ) → Eq x y
%   sub : (v : is ∈ Γ) → E Γ ip → E (Γ / v) is → E (Γ / v) ip

% \end{code}
% \begin{code}
% opt : E Γ is → E Γ is
% opt (selₛ e e₁) with opt e | opt e₁
% ... | imapb m e         | i = selₛ (sub v₀ e (div m i)) (mod m i)
% ... | slide i pl a su   | k = selₛ a (ix-plus i k su pl)
% --- | ... as before ...
% \end{code}
% Here we optimise away scalar selections into blocked imaps.  Recall that $m$ tells us
% that we have an array of shape $s * p$, and $e$ computes blocks of shape $p$.  If we
% are selecting into such a blocked array at the index $i$, we know that we are selecting
% $(i / p)$-th block, and from that block we are selecting $(i \% p)$ element.  Existence
% of explicit \AC{div} and \AC{mod} operations on indices makes it possible to implement
% this rewrite rule that is again very similar to $\beta$-reduction.
% \begin{code}[hide]
% ... | a                 | i = selₛ a i
% \end{code}
% \begin{code}
% opt (sum e) with opt e
% ... | zero-but (var i) (ix-plus (var j) (var k) su pl) a  with eq? v₀ i | eq? v₀ j | eq? v₀ k
% ... | neq _ i′  | neq _ j′  | eq        = ix-minus   (var i′) (var j′) pl su a
% ... | neq _ i′  | eq        | neq _ k′  = ix-minusᵣ  (var i′) (var k′) pl su a
% ... | _         | _         | _         = sum (zero-but (var i) (ix-plus (var j) (var k) su pl) a)
% --- | ... as before ...
% \end{code}
% \begin{code}[hide]
% opt (sum e) | a = sum a
% \end{code}
% Here we are dealing with the sum over summation index $t$ where the inner expression is
% a conditional on indices of the form \texttt{i == j + k ? e : 0}.  Here we apply the
% same comparison of index variables as before.  If $k$ happens to be the variable $t$,
% then overall sum will only add one non-zero element at $(i-j)$-th index, given that this
% (left) subtraction is possible in the sense of existence of the inverse to \AF{\_⊕ₚ\_} operation
% defined in Section~\ref{sec:general-ix-ops}.  The same happens when the summation index
% $t$ is equal to $j$, we only need to consider $(i-k)$-th element given that this (right)
% subtraction is possible.  One could cover other cases where $t$ is equal to $i$, or
% when $i$ and $j+k$ are swapped, but these are not occurring in our running example.

% \begin{code}
% opt (scaledown x e) with opt e
% ... | sum a = sum (scaledown x a)
% --- | ... as before ...
% \end{code}
% Finally, here is a rule that looks very innocent in the high-level language, yet
% becomes of importance in the low-level one.  The rule says that if we are summing
% the array and then dividing it by a constant, we should move division inside the
% summation.  The reason for this rewrite rule being important is when the result
% of the sum is non-scalar, we need to create a temporary array, before scaling down
% all its elements.  A language with first class arrays can obviously take care of
% such minor details, but in C we have to be explicit about it.
% \begin{code}[hide]
% ... | a = scaledown x a
% opt e = e
% \end{code}

% \subsubsection{Code Generation}
% Due to space limitations, we only consider the basic mechanisms we used in the
% code generator, all the code is available in supplementary materials.  We use
% heap-allocated multi-dimensional arrays that can be defined as follows:
% \begin{lstlisting}[language=C]
%   float(*k1)[6][5][5] = malloc(sizeof(*k1));
% \end{lstlisting}
% This ensures that \texttt{k1} is represented as a continuous region of memory
% of size $6*5*5$ floats.  When such arrays are indexed (\eg{} \texttt{(*k1)[i][j][k]}),
% the indices are translated into a single offset into the continuous memory.
% Therefore, there is no pointer chasing which makes this approach efficient at
% runtime.  As C uses row-major order to compute the offsets, we do obtain
% partial array selections on the left, \eg \texttt{(*k1)[i]} is a $5\times 5$
% array that can be further indexed or passed to \texttt{sizeof} that correctly
% identifies the size of this subarray.  Surely, this is a pointer into the \texttt{k1}
% array, so all the modifications to \texttt{(*k1)[i]} will modify \texttt{k1}.
% As a great convenience feature, C compiler tracks the ranges of the indices
% and produces warnings in cases when it figures out that ranges of indices
% and the array we are indexing do not match.

% Whenever we translate some $e$ in \AF{E} into C, we have to provide a storage
% where $e$ has to be written to.  In case of compiling the \AF{Chain} every
% local variable becomes such a storage for the bound expression.  Therefore,
% our extractor always has a result variable as an argument.

% For example, let us consider an expression $a ⊞ a$ of shape
% (\AC{ι} 5 \AC{⊗} \AC{ι} 5), where $a$ is mapped to the C variable
% \texttt{float (*a)[5][5]} that is written to the result variable
% \texttt{float (*r)[5][5]}.  Here is the code that we generate:
% \begin{lstlisting}[language=C]
%   for (size_t x1_1 = 0; x1_1 < 5; x1_1++) { 
%     for (size_t x1_2 = 0; x1_2 < 5; x1_2++) { 
%       (*r)[x1_1][x1_2] = ((*a)[x1_1][x1_2] + (*a)[x1_1][x1_2]);
%     }}
% \end{lstlisting}
% We started with checking that $a ⊞ a$ is a \emph{selectable} expression.
% This means that we can always generate expression at the given index.
% As we know that the shape of $a ⊞ a$ is (\AC{ι} 5 \AC{⊗} \AC{ι} 5),
% we need to generate a loop nest of that shape that assigns where
% we assign the expression at the given index to the result at the given
% index.

% We need to distinguish whether we are writing into the result or adding
% into it as in cases when dealing with \AF{sum}.  Consider the code that
% is generated for (\AC{sum} (\AC{selₛ} (\AB{a} (\AC{var v₀}))) where
% we are adding all the elements of the array $a$ into result variable
% \texttt{float (*r)[1]}:
% \begin{lstlisting}[language=C]
%   for (size_t x2_1 = 0; x2_1 < 5; x2_1++) {
%     for (size_t x2_2 = 0; x2_2 < 5; x2_2++) {
%       for (size_t x3_1 = 0; x3_1 < 1; x3_1++) { 
%         (*r)[x3_1] += (&(*a)[x2_1][x2_2])[x3_1];
%       }}}
% \end{lstlisting}
% Two things are happening here, first we generate \texttt{+=} assignment
% and we make an implicit assumption that resulting variables are initialised
% to zero.  In the extractor, additionally to the resulting variable we
% track whether we need to do an assignment or assignment with addition.
% Secondly, while $a$ is two-dimensional, we have three-dimensional loop
% nest.  The latter comes from the representation of scalars as 1-element
% vectors.  When we resolved the two-dimensional summation index \texttt{x2},
% we know that we need to assign into the object of shape (\AC{ι} 1), but
% the left-hand-side is a scalar (float).  The trick here is that in C we
% can always turn scalars into 1-element vectors by simply taking the address
% of the scalar.  This is why we have this 1-iteration for-loop over
% \texttt{x3\_1} that will be immediately optimised away by the C compiler.

% Finally, when we it comes to the operation on indices, such as addition,
% subtraction, division or modulo, we generate the corresponding operation
% on the individual loop indices.

% Remaining details of the code generation take care of traversing through
% the structure of \AF{E} with some plumbing that has to do with generating
% loop-nests around expressions and checking that they are selectable.

% \subsubsection{Running the Generated C Code}
% In order to run the generated C code we translate the boilerplate code
% from SaC to C.  While doing so, we made sure that our code can be run
% in parallel.  While the  SaC compiler does this automatically, there is one
% obvious loop that requires parallelisation which is computation of
% the batch.  When we train the CNN, we take a batch of images and the
% weights and we compute gradients for those weights per every image.
% After that we average all the gradients in the batch, and we update
% the weights, after all the batch is processed.  Clearly, all the
% gradient computations in the batch can run in parallel.  We achieve
% this by organising the batch loop such that all the gradients are
% stored in a separate memory region, and we parallelise this loop
% using OpenMP annotations.

% We verify that the code that we generate compute the same results
% as the hand-written SaC code.  Then we replicate the experiment from
% the~\cite{cnn-array} using 40 epochs, 100 images in the batch, and
% feeding 10000 training images.  We run the experiment on the 18-core
% 13th Gen Intel(R) Core(TM) i5-13600K machine using sac2c version
% \texttt{1.3.3-MijasCosta-1161-gb543c} and the GCC compiler
% version \texttt{12.2.0}.  The first thing that we learn is that
% our generated C code is sensible (factor of 3 running time)
% to the compilation flags that we enable.  We identified the set
% of flags that when passed to both compilers\footnote{SaC compiler
% generates C code, so we can control what flags it uses when
% compiling it.}, the runtime at the
% largest number of cores are 11s for the hand-written SaC implementation 
% and 13.5s for the generated C code, with
% very little variance.  This 20\% difference is orthogonal
% to parallel execution, as it is also observed when running
% the code on a single core.  The set of flags has to do with
% floating point operations: \texttt{-fno-signed-zeros} ignores
% the distinction between negative and positive zeroes that is given
% by IEEE 754 standard, allowing to reduce (-0.0*x) to 0.0;
% \texttt{-fno-math-errno} does not set errno after calling math functions;
% \texttt{-fno-trapping-math} and \texttt{-fassociative-math} make
% sure that we can assume associativity of floating point operations
% which does not hold according to the IEEE 754.

% The main performance difference comes from the fact that
% compiled SaC code uses less intermediate arrays, significantly reducing the number
% of memory writes.  There are numerous ways how to improve the performance
% of the generated C code, but for the purposes of this paper we consider that getting within
% 20\% of the hand-written SaC code is sufficient evidence for our hypothesis
% that the two-languages approach seems viable for achieving proved
% correctness and performance.
% We have automatic differentiation in the safe environment
% that generates the C code that runs almost as fast as the hand-written
% SaC code.
